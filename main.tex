%%%%%%%%%%%%%%%%%%%%%%%%%%%%%%%%%%%%%%%%%%%%%%%%%%%%
%%%     Language Science Press Master File       %%%
%%%         follow the instructions below        %%%
%%%%%%%%%%%%%%%%%%%%%%%%%%%%%%%%%%%%%%%%%%%%%%%%%%%%

% Everything following a % is ignored
% Some lines start with %. Remove the % to include them

\documentclass[output=book,nonflat,
  colorlinks,citecolor=brown,
%  draft,draftmode,
%  showindex,
%  nobabel,
%  booklanguage=german,
%  multiauthors,
		  ]{langscibook/langscibook}

%%%%%%%%%%%%%%%%%%%%%%%%%%%%%%%%%%%%%%%%%%%%%%%%%%%%
%%%          additional packages                 %%%
%%%%%%%%%%%%%%%%%%%%%%%%%%%%%%%%%%%%%%%%%%%%%%%%%%%%

\author{Yuto Niinaga}
\title{A grammar of Yuwan}
\subtitle{}
\renewcommand{\lsSeries}{sidl}
\renewcommand{\lsSeriesNumber}{}
\renewcommand{\lsID}{272}

% add all extra packages you need to load to this file  
\usepackage{tabularx} 
\usepackage{url} 
\urlstyle{same}

\usepackage[normalem]{ulem}

\usepackage{multicol}
\usepackage{enumitem}
\usepackage[linguistics,edges]{forest}
\usepackage{tikzpeople}

\usepackage{siunitx}

%%%%%%%%%%%%%%%%%%%%%%%%%%%%%%%%%%%%%%%%%%%%%%%%%%%%
%%%                                              %%%
%%%           Examples                           %%%
%%%                                              %%%
%%%%%%%%%%%%%%%%%%%%%%%%%%%%%%%%%%%%%%%%%%%%%%%%%%%% 
%% to add additional information to the right of examples, uncomment the following line
% \usepackage{jambox}
%% if you want the source line of examples to be in italics, uncomment the following line
% \renewcommand{\exfont}{\itshape}
\usepackage{./langscibook/langsci-basic}
\usepackage{./langscibook/langsci-optional}
\usepackage{./langscibook/langsci-lgr}
\usepackage{./langscibook/langsci-gb4e}

%% hyphenation points for line breaks
%% Normally, automatic hyphenation in LaTeX is very good
%% If a word is mis-hyphenated, add it to this file
%%
%% add information to TeX file before \begin{document} with:
%% %% hyphenation points for line breaks
%% Normally, automatic hyphenation in LaTeX is very good
%% If a word is mis-hyphenated, add it to this file
%%
%% add information to TeX file before \begin{document} with:
%% %% hyphenation points for line breaks
%% Normally, automatic hyphenation in LaTeX is very good
%% If a word is mis-hyphenated, add it to this file
%%
%% add information to TeX file before \begin{document} with:
%% \include{localhyphenation}
\hyphenation{
Dub-lin
}
\hyphenation{
Dub-lin
}
\hyphenation{
Dub-lin
}
\newcommand{\appref}[1]{Appendix \ref{#1}}
\newcommand{\fnref}[1]{Footnote \ref{#1}} 

\newenvironment{langscibars}{\begin{axis}[ybar,xtick=data, xticklabels from table={\mydata}{pos}, 
        width  = \textwidth,
	height = .3\textheight,
    	nodes near coords, 
	xtick=data,
	x tick label style={},  
	ymin=0,
        ]}{\end{axis}}
        
\newcommand{\langscibar}[1]{\addplot+ table [x=i, y=#1] {\mydata};\addlegendentry{#1};}

\newcommand{\langscidata}[1]{\pgfplotstableread{#1}\mydata;}

\addbibresource{localbibliography.bib}


\WarningsOff[microtype]
\WarningFilter{microtype}{Unknown slot}
\WarningFilter{scrbook}{package incompatibility detected}
%%%%%%%%%%%%%%%%%%%%%%%%%%%%%%%%%%%%%%%%%%%%%%%%%%%%
%%%             Frontmatter                      %%%
%%%%%%%%%%%%%%%%%%%%%%%%%%%%%%%%%%%%%%%%%%%%%%%%%%%%
\begin{document}


\maketitle
\frontmatter

\currentpdfbookmark{Contents}{name} % adds a PDF bookmark
{\sloppy\tableofcontents}
 \addchap{\lsAcknowledgementTitle} 
This grammar is the revised version of my PhD dissertation, which is submitted to the University of Tokyo in 2014. During the sixteen years that I have been working on this grammar, many people have offered data, ideas, comments, and other help. It is a pleasure to record my thanks to them.

My profound thanks to the speakers in Yuwan: Miyoko Fuji, Hisako Ishihara, Shigezo Mae, Fumio Matsumoto, Tama Mizobe, Masako Motoda, Nobuari Motoda, Sawako Motoda, Etsue Murano, Humizu Naka, Umine Shinozaki, Satoko Taira, Mitsuko Toshioka, and Sumie (Mutsu) Yamaki. My deepest gratitude should be expressed to Sachi (Tsuneko) Motoda, who taught me the Yuwan dialect when I first visited the village and who has been teaching it to me since then. Her special sense of humor always encouraged me to continue my research. My sincere thanks should be offered to Mioya Sunao, who spent a lot of time for helping me to transcribe the natural discourse in Yuwan. Without his help I could not have had the reliable data which my grammar is based on. I am also very grateful to the people in Suko district and the members of the Board of Education in Uken village.

Valuable advice was given from many researchers: Hayato Aoi, Reiko Aso, Kevin Baetscher, Joel Bradshaw, Shinichiro Fukuda, Margaret Ransdell-Green, Martin Haspelmath, Yuka Hayashi, David J. Iannucci, Shigehisa Karimata, Shinjiro Kazama, Nobuko Kibe, Kento Nagatsugu, Chiharu Obata, Shinji Ogawa, Mark Rosa, Mika Sakai, Matt Shibatani, Hiromi Shigeno, Rihito Shirata, Pellard Thomas, Nana Toyama, Tasaku Tsunoda, Zendo Uwano, and John Whitman. Above all, I owe endless gratitude to Michinori Shimoji, who had a substantial influence on this grammar. When I first read his PhD dissertation (\textit{A grammar of Irabu, a Southern Ryukyuan language}) submitted to the Australian National University, I realized this one is what I wanted to accomplish in my course of research. Most of the ideas, i.e. the construction of chapters, the terminologies, and the analyses of linguistic phenomena, largely depend on his grammar, although there are some reconstruction and modification since the target languages are different from each other.

The Language Science Press offered me a precious opportunity to publish my work. I genuinely appreciate their policy to make their publication available at no charge. An anonymous reviewer gave me critical remarks which were very precious to me. I could not find any word appropriate to show my gratitude to XXXX, XXX, XXX, and XXX, who found a huge number of typos in my original draft and gave me beneficial comments. Felix Kopecky kindly converted my original draft written as document files into that of LaTeX files.

At different times the research reported here has been supported by Research Institute for Languages and Cultures of Asia and Africa during April 2008 -- May 2009, and by Japan Society for the Promotion of Science during April 2010 -- May 2011. These supports are gratefully acknowledged.

Finally, I am particulary thankful to my family.

 % \addchap{\lsAbbreviationsTitle}
\addchap{Abbreviations and symbols}
\section*{Abbreviations}\enlargethispage{\baselineskip}
\begin{multicols}{2}
\begin{tabbing}
	extsc{ecs} verbs\=inflectional adjectival affix\kill
A \>  agent-like argument of \\ \> transitive verb; adjective \\
	extsc{abl} \>  ablative \\
	extsc{acc} \>  accusative \\
	extsc{adj} \>  inflectional adjectival affix \\
	extsc{adn}Z \>  adnominalizer \\
	extsc{advrs} \>  adversative \\
	extsc{advz} \>  adverbializer \\
	extsc{all} \>  allative \\
	extsc{appr} \>  approximative \\
	extsc{ass} \>  assertive \\
Aux. V \>  auxiliary verb \\
	extsc{av}C \>  auxiliary verb construction \\
	extsc{ben} \>  benefactive \\
C \>  any consonant \\
	extsc{cap} \>  capability \\
	extsc{caus} \>  causative \\
	extsc{cfm} \>  confirmation \\
	extsc{cfp} \>  clause-final particle \\
	extsc{clf} \>  classifier \\
	extsc{cmp} \>  comparative \\
	extsc{cnd} \>  conditional \\
Co \>  data from the conversation \\
	extsc{com} \>  comitative \\
	extsc{csl} \>  causal \\
	extsc{dat} \>  dative \\
	extsc{dim} \>  diminutive \\
	extsc{dirc} \>  directional \\
	extsc{dist} \>  distal \\
	extsc{drg} \>  derogative \\
	extsc{du}B \>  dubitative \\
	extsc{du} \>  dual \\
	extsc{ecs} \>  the existential, copula, \\ \> and stative verb \\
El \>  elicitational data \\
	extsc{fn} \>  formal nouns \\
	extsc{foc} \>  focus \\
Fo \>  data from the folktale \\
	extsc{gen} \>  genitive \\
G \>  glide slot in a syllable \\
	extsc{imp} \>  imperative \\
	extsc{indfz} \>  indefinitizer \\
	extsc{ingr} \>  ingressive \\
	extsc{inst} \>  instrumental \\
	extsc{int} \>  intentional \\
k.o. \>  a kind of \\
Lex. V \>  lexical verb \\
LF \>  lengthened (infinitival) form \\
lit. \>  literally \\
	extsc{lmt} \>  limitative \\
	extsc{loc} \>  locative \\
	extsc{lst} \>  listing \\
	extsc{lvc} \>  light verb construction \\
	extsc{lv} \>  light verb \\
	extsc{mes} \>  mesial \\
	extsc{mm}C \>  Mermaid construction \\
N/A \>  not applicable \\
	extsc{neg} \>  negative \\
N	extsc{hon} \>  non-honorific \\
	extsc{nlz} \>  nominalizer \\
	extsc{nom} \>  nominative \\
NP \>  nominal phrase \\
	extsc{npst} \>  non-past \\
	extsc{obl} \>  obligative \\
	extsc{odn} \>  ordinary number \\
P	extsc{ass} \>  passive \\
	extsc{pfc} \>  predicate of focus \\ \> construction \\
	extsc{pf} \>  pear film \\
	extsc{pl} \>  plural\\
	extsc{plq} \>  polar question\\
	extsc{pol} \>  politeness\\
	extsc{pos} \>  possibility\\
P \>  patient-like argument of \\ \> transitive verb \\
	extsc{prog} \>  progressive\\
	extsc{prox} \>  proximal\\
	extsc{prpr} \>  preparative\\
	extsc{pst} \>  past\\
	extsc{ptcp} \>  participle\\
	extsc{purp} \>  purposive\\
	extsc{qt} \>  quotation\\
	extsc{red} \>  redupulicant\\
	extsc{rfl} \>  reflexive\\
	extsc{rsl} \>  resultative\\
S \>  an argument of \\ \> intransitive verb\\
	extsc{sf} \>  simple (infinitival) form\\
	extsc{sg} \>  singular\\
	extsc{sim} \>  simultaneous\\
	extsc{sol} \>  solidarity\\
	extsc{st}V \>  stative verb\\
	extsc{sugs} \>  suggessive\\
	extsc{supp} \>  suppositional\\
	extsc{top} \>  topic\\
	extsc{umrk} \>  unmarked verbal affix\\
V \>  any vowel; verb\\
VP \>  verbal phrase\\
V\textsubscript{back} \>  back vowels\\
V\textsubscript{non-back} \>  non-back vowels\\
V\textsubscript{non-\textit{i}} \>  vowels excluding //i//\\
X \>  an anonymous \\ \> personal name
\end{tabbing}
\end{multicols}

\section*{Symbols}\enlargethispage{\baselineskip}
\begin{tabularx}{\textwidth}{@{}lQ@{}}
\#    & syllable boundary\\
\textsuperscript{\#}    & context is unnatural\\
\$    & word boundary\\
*    & ungrammatical expression ancestoral form (see also ‘Pre-note (b)’ in appendix)\\
+    & boundary of a compound boundary of reduplication boundary of a contracted adjectival predicate, boudary of the fusion of \textit{ccjɨ} (	extsc{qt}) and \textit{jˀ}{}- ‘say’\\
{}-    & affix boundary\\
=    & clitic boundary\\
A/B    & A or B\\
//A//    & “A” is a morphophoneme (or underlying form)\\
/A/    & “A” is a phoneme (or surface form)
\end{tabularx}

 \addchap{Transcription methods}

These transcription methods are inspired by those of Stuart \citet[7--9, 43--52]{McGill2009}.

\section*{Interlinear examples}

Each example is composed of four tiers: the surface tier (the phonemic representation), the underlying tier (the morphophonemic representation), the tier for morpheme-by-morpheme gloss, which conforms to the convention of the Leipzig Glossing Rules\footnote{These are available at \url{https://www.eva.mpg.de/lingua/pdf/Glossing-Rules.pdf}.} and the tier for free translation provided by the present author. The surface tier does not have morpheme boundaries. This way, it is possible to handle fusions and morphophonological alternations with interlinear morphemic glosses.

\ea
\glll mukasinu janagɨjaaccjəə nən.jaa.\jambox{surface tier}\\
      mukasi=nu  janagɨ+jaa=ccjɨ=ja  nə-an=jaa\jambox{underlying tier}\\
      old.days=	extsc{gen} dirty+house=	extsc{qt}=	extsc{top} exist-	extsc{neg}=	extsc{sol}\jambox{gloss tier}\\
\glt  ‘There is not (a house) like a dirty [i.e. outdated] house of the old days.’\jambox{free translation tier}
\z 

The following markers are used in a surface (if it is deleted, in an underlying) tier.

\begin{description}[font=\normalfont]
\item[,]  after an interjection or an adverbial clause; before the hearer’s nod assent; enclosing an inserted expression
\item[.]  after a sentence (not within a word); between syllable boundaries (within a word)\footnote{As mentioned in §\ref{bkm:Ref347173344}, there is no sequence [ɴ.V] (V: vowel) within a phonological word in Yuwan, so any sequence of /VnV/ within a phonological word in the surface form would be /V.nV/ [V.nV], not /Vn.V/ [Vɴ.V].}
\item[?]  after an interrogative sentence
\item[!]  after an imperative sentence
\item[..]  short pause
\item[...]  long pause
\item[xxx]  unintelligible speech
\item[(  )] enclosing a defective utterance or a misstatement
\item[{\textbar}  {\textbar}]  enclosing standard Japanese
\end{description} 

Additionally, the underlying tier is provided in \textit{italics}, the free translation is enclosed within single quotation marks, and information inferable from the context may be added with round brackets in the free translation. Some morphemes can be translated into more than one meaning (or function) in English, i.e. polysemy. In that case, we gloss it in the following order \citep[cf.][11--12]{Lehmann2004}: (1) if we can abstract the polysemous meanings into one meaning, we use the abstract meaning as its gloss; (2) if we cannot do this, we gloss the relevant meaning in each example. In the second case, I sacrificed the consistency of the glossing and the form, because it is helpful for the reader to know the correspondence between the glossing and the free translation. Finally, in the free translation, ‘...’ means there is a remaining portion of the sentence that has been left out.

In many cases, context is supplied for an example, and it is enclosed in square brackets on the upper side of examples. Paraphrases in English (with speaker 	extsc{id}) in quotation marks may follow the description of the context. In addition, if other kinds of information, e.g., syntactic constructions, are needed, another line may be added below the glossing line \citep[cf.][4--5]{Lehmann2004}.

\ea\relax
[Context: 	extsc{tm} and 	extsc{ms} were looking at the beams of TM’s house; MS: ‘There are few houses (that have the beams) like these.’]\\
	\textsc{tm}: \gllll mukasinu janagɨjaaccjəə nən.jaa.\\
          mukasi=nu janagɨ+jaa=ccjɨ=ja nə-an=jaa\\
          \{[old.days=	extsc{gen}] [dirty+house]\}=	extsc{qt}=	extsc{top} exist-	extsc{neg}=	extsc{sol}\\
          \{[Modifier] [Head]\}\textsubscript{NP}\\
   \glt\hspaceThis{\textsc{tm}:} ‘There is not (a house) like a dirty [i.e. outdated] house of the old days.’ [Co: 111113\_01.txt]

\z

Further, each example will be shown with the data of its source, i.e. genre of data and the file name of source, in the square brackets on the lower right side of examples (for more details on the abbreviations used to indicate the source data, see §\ref{bkm:Ref347173399}).

\section*{In-text example}

An in-text example is placed in the following order: surface forms in slash marks, underlying forms in \textit{italics}, morpheme-by-morpheme glosses, and free translation in single quotation marks, as in /janagɨjaaccjəə/ \textit{janagɨ+jaa=ccjɨ=ja} (dirty+house=	extsc{qt}=	extsc{top}) ‘like a dirty house.’ If we do not need to show a morpheme boundary, we will use a period in glosses to imply there are a few morphemes, such as /janagɨjaaccjəə/ (dirty.house.QT.TOP). Contrary to interlinear examples, the surface forms of in-text examples may show their morpheme boundaries if the need arises, such as /janagɨ+jaa=ccjə=ə/ (dirty+house=QT=TOP). Sometimes, IPA symbols are used to access the concrete sounds in square brackets, e.g., [jɑ̟nɑ̟gɨjɑ̟ːtt͡ɕɜː]. The underlying forms (i.e. morphophonemic) may be expressed not only with italics but also double slash marks, such as //ja//. Forms in the middle stage of morphophonemic processes are also shown in double slash marks. If the relevant form is not a grammatical word, i.e. bound roots or affixes like \textit{kam-} ‘eat’ or \textit{{}-ɨ} (	extsc{imp}), a hyphen is attached to mark the place of morpheme boundaries.

\section*{Orthography}

Yuwan has mainly six vowels [i, u, o̞, ɑ̟, ɨ, ɜ] (see §\ref{bkm:Ref312504424}). In many of the previous studies of Amami dialects (including that of Yuwan), the first four vowels have been transcribed into ‘i, u, o, a (\textit{a} in italic)’ but the last two vowels have been transcribed as ‘ï’ [ɨ] and ‘ë’ [ɜ]. In this grammar, [ɨ] and [ɜ] are transcribed as ‘ɨ’ and ‘ə’ since (1) they do not need diacritics, and (2) [ə] is closer to [ɜ] than [ë] (but we do not use ‘ɜ’ because it is not as familiar as ‘ə’).

Furthermore, Yuwan has glottalized consonants such as [ʔj, ʔw, ʔm, ʔn, ʔ͡t, ʔ͡k, ʔ͡ʨ], which have been transcribed as ‘ʔC’ or ‘Cˀ’ (C is any consonant), depending on the researcher’s interpretation of those phones. The latest IPA diacritics\footnote{Available at \url{http://www.langsci.ucl.ac.uk/ipa/IPA\_chart\_(C)2005.pdf}.} do not have ‘ˀ’ even though this diacritic is very useful to describe these consonants. In this grammar, the glottalized consonants are regarded as single phonemes (see §\ref{bkm:Ref347173461}) and transcribed as ‘jˀ, wˀ, mˀ, nˀ, tˀ, kˀ, and cˀ.’

Finally, Yuwan has homorganic nasals, and if we cannot infer their underlying form from the paradigmatic information, we recognize them as archiphonemes \citep[46--49]{Lass1984}. Yuwan has /m/ and /n/, which are homorganic. For example, in /jum-an/ [ju.mɑ̟ɴ] (read-	extsc{neg}) ‘do not read’ and /jum-gadɨ/ (read -until) [juŋ.gɑ̟.dɨ] ‘until (someone) reads,’ /m/ can be [m] or [ŋ] depending on the following phonemes. Similarly, in /ɨn=un/ [ʔɨ.nu.ɴ] (dog=also) ‘also a dog’ and /ɨn=gadɨ/ [ʔɨŋ.gɑ̟.dɨ] (dog=	extsc{lmt}) ‘as well as dogs,’ /n/ can be [n] or [ŋ] depending on the following phonemes. [ʔɑ̟m.mɑ̟ː] ‘mother,’ however, is made up of a single root, so we cannot know whether its first [m] would be /m/ or /n/. In this case, we recognize the existence of archiphoneme /N/ and avoid choosing the unique underlying phoneme. In this grammar, the archiphoneme is transcribed as ‘n,’ since the use of /N/ implies the exsistence of a phoneme other than /m/ and /n/. Thus, [ʔɑ̟m.mɑ̟ː] is \textit{anmaa} (see §\ref{bkm:Ref347174390} for more details). The other symbols used in this grammar coincide with their phonetic representations (or commonly accepted phonemic representations) (see also §\ref{bkm:Ref347174415}).

\mainmatter


%%%%%%%%%%%%%%%%%%%%%%%%%%%%%%%%%%%%%%%%%%%%%%%%%%%%
%%%             Chapters                         %%%
%%%%%%%%%%%%%%%%%%%%%%%%%%%%%%%%%%%%%%%%%%%%%%%%%%%%

\chapter{Introduction}
\hypertarget{RefHeadingToc395696949}{}
\section{Typological overview}
\hypertarget{RefHeadingToc395696950}{}
Yuwan has six vowels /i, ɨ, u, ə, o, a/ and twenty-two consonants /p, t, k, tˀ, kˀ, b, d, g, c, cˀ, s, h, z, m, n, mˀ, nˀ, w, j, wˀ, jˀ, r/, and its syllable structure is CGVV or CGVC (G: glide slot). Additionally, it has an agglutinative morphology, and its basic word order is SV or AOV. S and O are marked by the nominative case \textit{ga} (or \textit{nu}), and O is marked by the accusative case \textit{ba}, although there are some examples where O does not take any case.

\section{Geography}
\hypertarget{RefHeadingToc395696951}{}
Yuwan is spoken in the Yuwan district, in the western district of Amami Ōshima, an island situated just south of mainland Japan. The size of Amami Ōshima is about 710 km\textsuperscript{2}, and it is the biggest island of the Amami Islands, which includes seven other major islands. Amami Ōshima is situated in the northern part of the Ryūkyū archipelago but belongs to the Kagoshima prefecture, while most of the other Ryūkyū islands belong to the Okinawa prefecture. Amami Ryukyuan is a Northern Ryukyuan language. (The map in \figref{fig:key:1} was made in the following web site: \url{http://www.craftmap.box-i.net/japan/line.php}).

  
%%please move the includegraphics inside the {figure} environment
%%\includegraphics[width=\textwidth]{figures/grammarofyuwanfinal9-img001.png}
 

\begin{figure}
\textmd{\figref{fig:key:1}. Japan in the Far East}
\end{figure}

%%[Warning: Draw object ignored]
%%[Warning: Draw object ignored]
%%[Warning: Draw object ignored]
%%[Warning: Draw object ignored]
%%[Warning: Draw object ignored]
%%[Warning: Draw object ignored]
%%[Warning: Draw object ignored]
  
%%please move the includegraphics inside the {figure} environment
%%\includegraphics[width=\textwidth]{figures/grammarofyuwanfinal9-img002.png}
              
%%please move the includegraphics inside the {figure} environment
%%\includegraphics[width=\textwidth]{figures/grammarofyuwanfinal9-img003.png}
 

\begin{figure}
\textmd{\figref{fig:key:2}. Japan}
\end{figure}

\begin{figure}
\textmd{\figref{fig:key:5}. Uken village}
\end{figure}

Yuwan

%%[Warning: Draw object ignored]
%%[Warning: Draw object ignored]
%%[Warning: Draw object ignored]
%%[Warning: Draw object ignored]
  
%%please move the includegraphics inside the {figure} environment
%%\includegraphics[width=\textwidth]{figures/grammarofyuwanfinal9-img004.png}
              
%%please move the includegraphics inside the {figure} environment
%%\includegraphics[width=\textwidth]{figures/grammarofyuwanfinal9-img005.png}
 

\begin{figure}
\textmd{\figref{fig:key:4}. Amami islands}
\end{figure}

\begin{figure}
\textmd{\figref{fig:key:3}. Ryukyu islands}
\end{figure}

Amami Ōshima

The above maps in Figures 2-5 were made by the following free softwares:

a. “白地図 MapMap” (http://www5b.biglobe.ne.jp/t-kamada/CBuilder/mapmap.htm);

b. “白地図 KenMap” (http://www5b.biglobe.ne.jp/t-kamada/CBuilder/kenmap.htm).

\section{Affiliation}
\hypertarget{RefHeadingToc395696952}{}
According to \citet[771--774, 779--783]{Uemura1992}, Ryukyuan is in a sister relationship to Japanese, and Ryukyuan can be divided into two primary subgroups, Nothern group and Southern group. The Nothern group can be divided into Amami and Okinawa. According to \citet[263]{Pellard2009}, the accurate order of branching off of the three language groups, i.e. Amami, Okinawa, and Southern goup (“Sud” under “Ryukyu” in the following figure), is not clear. However, the subgrouping of Amami can be shown as in \figref{fig:key:6}. Yuwan belongs to “Ōshima” in this figure.

  
%%please move the includegraphics inside the {figure} environment
%%\includegraphics[width=\textwidth]{figures/grammarofyuwanfinal9-img006.emf}
 

\begin{figure}
\begin{forest} for tree={forked edge', grow=south}
[Ryukyu
 [?
  [Sud
  [Macro-yaeyama [Yonaguni] [Yaeyama]]
  [Miyako [Tarama] [Miyako\\commun,align=center]]
  ]
  [Okinawa [Sud] [Nord]]
  [Amami [Yoron] [,tikz+={\draw(.parent anchor)--(.child anchor);} [,tikz+={\draw(.parent anchor)--(.child anchor);} [Okinoerabu] [Tokunoshima]] [Oshima]]]
 ]
]
\end{forest}
\caption{\label{fig:key:6}Affiliation of Ryukyuan \citep[following][263]{Pellard2009}}
\end{figure}

\section{Sociolinguistic overview}
\hypertarget{RefHeadingToc395696953}{}\subsection{The number of speakers}
\hypertarget{RefHeadingToc395696954}{}
The population of Yuwan is 521 (valid as of January 1, 2010); however, a fewer number of people can speak the traditional dialect. The inhabitants are typically monolingual Japanese speakers or speak Japanese as a second language. In fact, the varieties of Japanese spoken here have been influenced by the traditional dialects of each location, especially in terms of the intonation and lexicon.

\subsection{Dialects}
\label{bkm:Ref367370857}\hypertarget{RefHeadingToc395696955}{}
In Amami Ōshima, there are many dialcects including Yuwan. There are some researches of linguistic geography about the dialects in Amami Ōshima: especially, \citet{HirayamaEtAl1966} and \citet{Shibata1984} among others. The detailed comparison among the lexemes in the dialects in Amami Ōshima is beyond the scope of this grammar. I present only one characteristic regarded as a major difference between Yuwan and the other dialects in Amami Ōshima. The phonetic sequence [ɾi] in the other dialects (and some [ɾ] in Koniya dialect) correspond to [i] in Yuwan (but not vice versa) (\citealt{HirayamaEtAl1966}: 71). \tabref{tab:key:1} illustrates this point with data from Yuwan, Suko, spoken in a village located about 800 meters from Yuwan, Ura, a Northern Amami dialect spoken in a village located about 32 km from Yuwan (the Ura data are provided courtesy of Dr. \name{Hiromi}{Shigeno} (p.c., 2009)), and Koniya, a Southern Amami dialect spoken in a village located about 15km from Yuwan (the Koniya data is taken from \citealt{HirayamaEtAl1966}: 70, which uses a phonetic symbol [r], but this phone is explained as “tap” (ibid.: 33). Thus, I transcribed it as [ɾ] in this example).

\begin{table}
% \hypertarget{RefHeadingToc395697266} 
\caption{Dialectal variation in Amami\label{tab:key:1}}
\begin{tabular}{lllll} 
\lsptoprule
       & Yuwan & Suko & Ura & Koniya\\\midrule
‘bird’ & [tui] & [tuɾi] &  [tuɾi] &  [tuɾ]\\
\lspbottomrule
\end{tabular}
\end{table}

Yuwan is spoken in a small district, so there do not appear to be regional variations; however, there seems to be a generational variation concerning honorific (and polite) expressions. Yuwan has an auxiliary verb \textit{moor-} (HON), which expresses the speaker’s respect for the subject of the clause (see \chapref{chap:3}). For example, in the case of /a-i/ \textit{ar-i} (exist-NPST) ‘exist’ vs. /a-tɨ moo-ju-i/ \textit{ar-tɨ} \textit{moor-jur-i} (exist-SEQ HON-UMRK- NPST) ‘would exist,’ the former is formed with the lexical verbal root \textit{ar-} ‘exist’ and it does not show the speaker’s respect to the subject, but the latter is created with both of the lexical verbal root \textit{ar-} ‘exist’ and the auxiliary verbal root \textit{moor-}, which expresses the speaker’s respect to the subject (see aslo §\ref{bkm:Ref367132477}). This honorific strategy is frequently used by older people, but not by younger people. Instead, younger people use a verbal affix \textit{{}-jawur} or \textit{{}-joor} to express respect for the hearer (not for the subject of the clause), e.g., /a-jawu-i/ \textit{ar-jawur-i} (exist-POL-NPST) ‘exist.’ Older speakers of Yuwan, however, are not likely to use this politeness affix.

Furthermore, there is another generational variation concerning morphophonological alternation. Yuwan has a topic marker \textit{ja}, and older speakers use the alternative form /na/ if its preceding word ends with a nasal consonant such as \textit{san} ‘three.’ However, younger speakers use /ja/ as the topic marker in any morphophonological environment. This variation is illustrated in the following example. Example (1-1) shows that the older speaker uses /na/ (TOP) after \textit{san} ‘three’ but the younger speaker does not.

\ea\label{ex:key:1}\relax[Context: The following examples are taken from a conversation between MS and TM, who are talking about the old educational system in Japan.]\\
MS: \glll {\textbar}roku, roku, san{\textbar}.ja arannən.\\
 roku roku san=ja ar-an-nən\\
 six six three=TOP COP-NEG-SEQ\\
\glt\hspaceThis{MS:} ‘(It) is not (divided into) six, six, three (years like now).’\\
TM: \glll {\textbar}roku, roku, san{\textbar}na arannən.\\
 roku roku san=ja ar-an-nən\\
 six six three=TOP COP-NEG-SEQ\\
\glt\hspaceThis{TM:} ‘(It) is not (divided into) six, six, three (years like now).’\\\hspaceThis{TM:} [Co: \texttt{120415\_00.txt}]
\z

\subsection{Viability}
\hypertarget{RefHeadingToc395696956}{}
The number of speakers of traditional Yuwan is decreasing. Typically, people over seventy years old can speak traditional Yuwan, and people who are fifty to sixty years old can speak a more or less traditional Yuwan, but people under fifty years old are only passively bilingual. The younger generations cannot speak or understand the traditional dialect; however, some of them use a few traditional expressions such as \textit{wan} ‘I’ or \textit{ccjɨ} (QT).

\subsection{Previous work}
\hypertarget{RefHeadingToc395696957}{}
In addition to the present study, there are two previous works on Yuwan: \citet{HirayamaEtAl1966} and \citet{UchimaEtAl1976}. The former compared the accent patterns and the lexicons among a number of Ryukyuan dialects, and only a small amount of information was presented about Yuwan. In fact, this study contained only thirty or so nominal lexical entries with their prosodic information. The latter, \citet{UchimaEtAl1976}, included a list of several hundred lexical items and several verb paradigms. However, the phonology of Yuwan has not yet been fully investigated, and its morphology has been only partially researched. The syntax of Yuwan has not been investigated at all, with the exception of \citet{Niinaga2008}, which describes the case system of Yuwan, and \citet{Niinaga2010}, which sketches a grammar of Yuwan.

A broader review of the literature brings to light a number of articles about Amami, of which Yuwan is a dialect. Here, only books or special issues of journals are mentioned. A brief comparison of several dialects of Amami can be found in \citet{HirayamaEtAl1966}. Lexical and phonological differences between some dialects in Amami Ōshima are discussed in \citet{Shibata1984}. Naze, which is spoken in the largest city in Amami Ōshima, is examined by \citet{Terashi1985}, and \citet{UemuraSuyama1997} describe its phonology, verbal morphology, and case markers. \citet{Shigeno2010} provides a sketch grammar of Ura, spoken in the northern part of Amami Ōshima. Yamatohama (or Yamatoma in the local pronunciation), spoken in the western part of Amami Ōshima, is the subject of study in Nagata et al. (1977–1980), which includes a detailed study of the lexicon but also some information on its grammar. \citet{UchimaEtAl1976} also describe the verbal morphology of Koniya, spoken in the southern part of Amami Ōshima. \citet{NakamotoUchima1978} provides a description of the lexicon and verbal morphology of Shitooke, spoken in the northern part of Kikai. \citet{ShirataEtAl2011} is a sketch grammar and a text of Kamikatetsu, spoken in the southern region of Kikai. \citet{OkamuraEtAl2009} describe the verbal morphology and list two thousand sentences in Asama, spoken in the northern region of Tokunoshima. \citet{KikuTakahashi2005} describe the lexicon of Yoron, and \citet{Yamada1981} focuses on the use of nominals in Yoron.

\section{Database for this study}
\label{bkm:Ref347173399}\hypertarget{RefHeadingToc395696958}{}
This grammar is based on a corpus of twelve texts (total duration is 4 hours) in addition to other elicited information that complements these texts. The data set was collected during the author’s field work in the region, which began in October 2006. The total length of time for the field work was 595 days. The details of the texts are shown in \tabref{tab:key:2}, and brief information about the speakers is shown in \tabref{tab:key:3}.

\begin{table}
\caption{\label{tab:key:2}Data of texts}
\resizebox{\textwidth}{!}{\begin{tabular}{l>{\ttfamily}lS[table-format=2.1] c c}
\lsptoprule
{Genre} & {\normalfont File ID} & {Duration} & {Main} & {Sub-speaker\footnote{(or hearer)}}\\
        &                       & {(min.)}   & {speaker}\\\midrule
{P(ear) F(ilm)} & 090222\_00.txt & 3.5 & {TM} & {(MM)}\\
                & 090225\_00.txt & 2.5 & {TM} & {(MM)}\\
                & 090305\_01.txt & 3 & {TM} & {(SM)}\\
                & 090827\_02.txt & 4 & {TM} & {(MY)}\\
{Fo(lktale)}    & 090307\_00.txt & 4 & {TM} & {(MM)}\\
{Co(nversations)} & 101020\_01.txt & 1 & {TM} & {MY}\\
                & 101023\_01.txt & 15 & {TM} & {MY}\\
                & 110328\_00.txt & 28 & {TM} & {US, MY, (MM)}\\
                & 111113\_01.txt & 28 & {TM} & {MS}\\
                & 111113\_02.txt & 22 & {TM} & {MS}\\
                & 120415\_00.txt & 63 & {TM} & {MS}\\
                & 120415\_01.wav & 66 & {TM} & {MS}\\
{El(icited)}   & N/A & {N/A} & {TM, MT} & {(the present author)}\\
\lspbottomrule
\end{tabular}}
\end{table}

The Pear Film is a silent six-minute film made at the University of California at Berkeley in 1975. It is helpful to collect the monologue data from the speaker.\footnote{A brief explanation of the Pear Film can be seen at \url{http://www.linguistics.ucsb.edu/faculty/chafe/pearfilm.htm}.} About the data classified in Pear Film, the speaker told the story to the hearer remembering the film (as soon as the speaker had watched it). About the folktale, the speakr heared it from her acquaintance who had told the story in a speech contest of the Amami dialects.

\begin{table}
\caption{\label{tab:key:3}Information about the Yuwan speakers}
\resizebox{\textwidth}{!}{\begin{tabular}{lllcc}
\lsptoprule
 ID & {First (Second) name} & {Family name} & {age in 2012} & {period of absence}\\
          &                         &              &               &  {from Yuwan}\\\midrule
TM & Sachi (Tsuneko) & Motoda & {89} & {14--21 years old}\\
US & Mine (Umine) & Shinozaki & {95} & {15--52}\\
MY & Sumie (Mutsu) & Yamaki & {88} & {28--49}\\
MT & Mitsuko & Toshioka & {78} & {24--26}\\
MM & Masako & Motoda & {73} & {15--38}\\
NM & Nobuari & Motoda & {62} & {20--29}\\
SM & Sawako & Motoda & {61} & {15--26}\\
MS & Mioya & Sunao & {59} & {16--53}\\
\lspbottomrule
\end{tabular}}
\end{table}


The recordings were transcribed by the present author with the help of some Yuwan speakers. In particular, Masako Motoda (MM), Nobuari Motoda (NM), and Mioya Sunao (MS) generously donated their time in order to help the present author’s transcription. During the recordings, I tried, when possible, to not be present in order to avoid promoting the speaker’s use of Standard Japanese, which was a lingua franca I shared with the Yuwan speakers. As for the elicitation data, the expressions in Yuwan that were produced by the present author and not by the speaker are not regarded as grammatical even if the speaker’s judgment was “grammatical.” In other words, I regard the elicitation data as grammatical only when the speaker pronounced the expression by herself.

Many of the examples in this grammar do not end at sentence-final positions -- i.e., they end with commas, not with periods. The verbs in Yuwan are rich with affixes that can mark subordinate clauses (see “the converb” in §\ref{bkm:Ref306802119}). This language is a “broadly characterizable as ‘chaining’” \citep[399]{Longacre2007} as well as Japanese. Just as the languages regarded as ‘chaining’ type in \citet{Longacre2007}, the finite verb occurs after ‘a sizeable strech of discourse which can on occasion be as long as two or three pages’ (ibid.: 400). Therefore, I have to omit the irrelevant parts from the clausal sequences.

Most of the data on the grammar of Yuwan comes from Sachi Motoda (TM), and the subsidiary information is taken from the other participants. All of these participants and their parents were born in Yuwan.

All of the examples in this grammar represent actual utterances of Yuwan speakers, and the sources of these utterances are clarified as much as possible. For example, the code “Co: \texttt{120415\_00.txt}” means the example was transcribed in the text file \texttt{120415\_00.txt} (the first six numbers indicate the recoding date, i.e. April 15, 2012), and its genre is “Co(nversation).” In the case of the elicited data, only the date of research is indicated after the abbreviation, e.g., “El: 120415.” In addition, the speaker ID is shown at the beginning of each transcription to represent who produced the utterance. For example, “TM: cjaa.” means the speaker TM said /cjaa/ (see also “Transcription methods” in the beginning of this book).

\section{Organization of this grammar}
\hypertarget{RefHeadingToc395696959}{}
In \chapref{chap:2}, the phonology of Yuwan is explained in detail. A brief explanation about the grammatical relations in Yuwan is given in \chapref{chap:3}. The descriptive preliminaries are presented in \chapref{chap:4} through a discussion of the basic construction and constituents of sentences of Yuwan. In \chapref{chap:5}, categories that can cross over several word classes, e.g., demonstratives or personal pronominals, are discussed. \chapref{chap:6} deals with nominal phrases, and \chapref{chap:7} investigates the detail of nominals. Verbal morphology is explained in detail in \chapref{chap:8}. \chapref{chap:9} explains three types of predicate phrases, i.e. verbal predicate, adjectival predicate, and nominal predicate. The details of particles are examined in \chapref{chap:10}. Finally, the inter-clausal phenomena is presented in \chapref{chap:11}. The appendix shows the detailed lists of morphophonological alternations of verbs.
  %add a percentage sign in front of the line to exclude this chapter from book
\chapter{Phonology}\label{chap:2}
\hypertarget{RefHeadingToc395696960}{}
In this chapter, I will present the phonology in Yuwan. The composition of grammatical words and phonological words will be shown in §\ref{bkm:Ref381193780}. The inventories of vowels and consonants will be shown in §\ref{bkm:Ref381193872}. The syllable structures and phonotactics will be discussed in §\ref{bkm:Ref302599307}. The phonological rules will be presented in §\ref{bkm:Ref302723494}. Finally, the nominal prosody will be discussed in §\ref{bkm:Ref301560567}.

\section{Segmentation}
\label{bkm:Ref381193780}\hypertarget{RefHeadingToc395696961}{}\label{bkm:Ref347179371}
A grammatical word (\textsc{gw}, henceforth simply “word” unless an explicit distinction between a grammatical word and a phonological word is necessary) is a morphosyntactic unit minimally consisting of a root, or it can consist of a root (or roots) plus an affix (or affixes) \citep[cf.][]{DixonAikhenvald2002}. In other cases, a grammatical word may consist of a single clitic. The above description is briefly summarized as follows.

\ea Grammatical words: \glllll\relax
[Root]\textsubscript{\textsc{gw}} [Root-Affix]\textsubscript{GW} [Root-Affix]\textsubscript{GW}=[Clitic]\textsubscript{GW}\\
anmaa anmataa anmatankja\footnotemark{}\\
{\itshape anmaa} {\itshape anmaa-taa} {\itshape anmaa-taa=nkja}\\
mother mother-\textsc{pl} mother-PL=\textsc{appr}\\
‘mother’ ‘mother and her fellow(s)’ ‘mother and her fellow(s)’\\
\z
\footnotetext{A sequence with the same vowel becomes a single vowel before a consonant that does not have a nucleus (see §\ref{bkm:Ref301832441} in detail). \textit{anmaa} ‘mother’ frequnetly becomes /anma/ when it is follwed by \textit{{}-taa} (\textsc{pl}).}

\noindent Taking the above distinction into consideration, we can recognize another unit, i.e., a phonological word.

\ea Phonological word: [Root (-Affix(es))]\textsubscript{\textsc{gw}} ([=Clitic(s)]\textsubscript{GW}) \z

\noindent A phonological word consists of a grammatical word optionally followed by a clitc (or clitics). A phonological word is the domain in which the following three rules apply: (A) phonological rule (see §\ref{bkm:Ref302723494}); (B) morphophonological rule (see §\ref{bkm:Ref356245430} and other relevant sections); and (C) prosodic rule (see §\ref{bkm:Ref301560567}), although the third criterion is in need of further research (see §\ref{bkm:Ref381418627}).

\section{Phonemes}
\label{bkm:Ref381193872}\hypertarget{RefHeadingToc395696962}{}\label{bkm:Ref347174415}\subsection{Vowels}
\label{bkm:Ref312504424}\hypertarget{RefHeadingToc395696963}{}\subsubsection{Short vowels}
\hypertarget{RefHeadingToc395696964}{}
Vowels are phonologically distinguished as below. Long vowels are treated as vowel sequences (see §\ref{bkm:Ref303982713}).

\begin{table}
\caption{Inventory of vowels}
\begin{tabular}{lccc} 
\lsptoprule
     & Front & Central & Back\\
     \midrule
High &  i    & ɨ       & u\\
Mid  & (e)   & ə [ɜ]   & o [o̞]\\
Low  &       &         & a [ɑ̟]\\
\lspbottomrule
\end{tabular}
\end{table}

Notes:
\begin{enumerate}[label=\alph*.]
\item High vowels: only /i/, /ɨ/, and /u/ are used as epenthetic vowels (see §\ref{bkm:Ref301838720}, §\ref{bkm:Ref347175824}, and \sectref{sec:key:8.2.1.4}). These vowels become voiceless between voiceless consonants or after a voiceless consonant at word-final positions;
\item \label{bkm:Ref347176670}Mid vowels: /e/, /ə/, and /o/ rarely appear as a single short vowel except for the case of vowel deletion (see §\ref{bkm:Ref301832441}). Within the total number of 1014 lexemes, the single short vowel /a/ appears in 468 lexemes, /u/ in 400, /ɨ/ in 260, /i/ in 200, /o/ in 16, and /ə/ in 4 (see the note “e” about /e/);
\item Front and central vowels: /i/ and /ɨ/ are contracted with \textit{ja} (\textsc{top}) into /əə/ (see §\ref{bkm:Ref367134300}); verbal stems that end with front or central vowels form a single stem class (see §\ref{bkm:Ref356245430});
\item Back vowels: /u/, /o/, and /a/ are contracted with \textit{ja} (\textsc{top}) into /oo/ (see §\ref{bkm:Ref367134300}); verbal stems that end with /ur/, /or/, and /ar/ form a single stem class (see §\ref{bkm:Ref356245430});
\item /e/ is used for a small number of loanwords from Standard Japanese (e.g., /sinsjei/ ‘teacher’) or interjections (e.g., /ude/ ‘hey’).
\end{enumerate}

The minimal contrasts of vowels are shown below. (The majority of the examples in this chapter are from elicited data, so the source information (see §\ref{bkm:Ref347173399}) is omitted.)

\ea 
\ea /i/ vs /ɨ/ vs /ə/ vs /u/\\
 /mii/  vs  /mɨɨ/  vs  /məə/  vs  /muu/\\
 ‘fruit’ {} ‘eye’ {}  ‘front’ {}  ‘alga’\\
\ex /i/ vs /o/\\
/kii/  vs  /koo/\\
‘yellow’ {}  ‘skin’\\
\ex /i/ vs /ɨ/ vs /a/\\
/jii/  vs  /jɨɨ/  vs  /jaa/\\
‘rush’ {}  ‘grip’ {}  ‘house’\\
\ex /ɨ/ vs /o/ vs /ə/\\
/sɨɨ/  vs  /soo/  vs  /səə/\\
‘vinegar’ {}  ‘stem’ {}  ‘alcohol’\\
\ex /u/ vs /o/ vs /ə/ vs /a/\\
/nuu/  vs  /noo/  vs  /nəə/  vs  /naa/\\
‘what’ {}  ‘fishing line’ {}  ‘elder sister’ {}  ‘name’\\
\z
\z

\subsubsection{Long vowels and diphthongs}
\label{bkm:Ref367392475}\hypertarget{RefHeadingToc395696965}{}\label{bkm:Ref347178348}
Every vowel in Yuwan can be lengthened, and this is treated as a vowel sequence (see also §\ref{bkm:Ref303982713}). All diphthongs in Yuwan are combinations of a particular vowel plus /i/.

\begin{table}
\caption{Long vowels and diphthongs}
\begin{tabular}{ *{8}{c} }
\lsptoprule
 V\textit{\textsubscript{1}} & V\textit{\textsubscript{2}} & /a/ & /u/ &  /i/ & /ɨ/ & /ə/ & /o/\\
 \midrule
 /a/  &   & aa & & ai      \\
 /u/  &   & & uu & ui      \\
 /i/  &   & & & ii         \\
 /ɨ/  &   & & & ɨi & ɨɨ    \\
 /ə/  &   & & & əi & & əə  \\
 /o/  &   & & & oi & & & oo\\
\lspbottomrule
\end{tabular}
\end{table}


\begin{table}
\caption{Examples of long vowels and diphthongs}
\begin{tabular}{ *{5}{l} }
\lsptoprule
 & \multicolumn{2}{l}{Long vowels}  & \multicolumn{2}{l}{Diphthongs}\\
 \midrule
{/a/} & {jaa} & {‘house’}  & {mai}  & {‘hip’}\\
{/u/} & {juu} & {‘boiled water’} & {jui} & {‘lily’}\\
{/i/} & {jii} & {‘rush’} & \multicolumn{2}{l}{(= long vowel)}\\
{/ɨ/} & {jumarɨɨ}  & {(read.P\textsc{ass}.\textsc{inf})} & jumarɨi & {(read.\textsc{pass}.\textsc{npst})}\\
{/ə/} & {jəəci} & {‘Yakeuchi’} & {jəito} &   {‘well’}\\
{/o/} & {joosɨ} & {‘atmosphere’}  & {joikwa} &  {‘silently’}\\
\lspbottomrule
\end{tabular}
\end{table}


In diphthongs, /ɨi/ is very rare and it occurs only in the combination of \textit{-arɨr} (P\textsc{ass}) and \textit{-i} (\textsc{npst}), i.e. \textit{-arɨr-i} (\textsc{pass}-N\textsc{pst}) > /-arɨi/, and the lexeme \textit{jɨɨi} ‘brother.’

\begin{table}
\caption{(Quasi-)minimal pairs of long and short vowels}
\begin{tabular}{ *{5}{l} } 
\lsptoprule
 & \multicolumn{2}{l}{Long vowels} & \multicolumn{2}{l}{Short vowels}\\
 \midrule
{/a/} & {mjaa} & {‘cat’} & {mja} & {‘k.o. shellfish’}\\
{/u/} & {tuuta} & {(pass.\textsc{pst})} & {tuta} & {(take.P\textsc{st})}\\
{/i/} & {jˀiicjasa} & {(say.want.\textsc{adj})} & {jˀicja} & {(say.\textsc{pst})}\\
{/ɨ/} & {cɨmɨɨ} & {‘k.o. shellfish’} & cɨmɨ & {‘nail’}\\
{/ə/} & {məərabɨ} & {‘young lady’} & {məngaa} & {‘good boy/girl’}\\
{/o/} & {goroogoro} & {‘growling’} & {gooruu} & {‘circle’}\\
\lspbottomrule
\end{tabular}
\end{table}

There are few lexemes where the vowels /ə/ or /o/ is short (see the note “\ref{bkm:Ref347176670}.” of \tabref{tab:2:8}). There are reasons to believe that they are underlyingly /əə/ or /oo/ (see §\ref{bkm:Ref301832441}).

Yuwan has a few morphemes that contain sounds such as [ɑ̟u] ([tɑ̟u] ‘plain,’ [ɑ̟uː] ‘blue,’ [jɑ̟ut͡ɕikkʷɜː] ‘naughty child,’ and [jɑ̟ur] (\textsc{pol})); however, the vowel sequence [ɑ̟u] can be regarded as /awu/ (not /au/) because of the morphophonological rule in §\ref{bkm:Ref367134300}. It suffices to note that the topic marker \textit{ja} retains its form after a long vowel or diphthong, but loses its form after a short vowel (by combining with the preceding short vowel).

\ea Rule for \textit{ja} (\textsc{top})
\ea After a long vowel or diphthong\\
\begin{tabbing}
    wunagu  \= ‘boiled water’ \=  +   ja   (\textsc{top})   > \=  juuja\kill
    juu  \> ‘boiled water’ \>  +   ja   (\textsc{top})   > \>  juuja\\
    mai  \> ‘hip’          \>  +   ja   (\textsc{top})   > \>  maija
\end{tabbing}
\ex After a short vowel\\
\begin{tabbing}
    wunagu  \= ‘boiled water’ \=  +   ja   (\textsc{top})   > \=  juuja\kill
    wunagu \>  ‘woman’  \> +  ja (\textsc{top})  >  \> wunagoo
\end{tabbing}
\z
\ex The case of [tɑ̟u] ‘plain’\\
\begin{tabbing}
    Phonologically: \= wunagu  \= ‘boiled water’ \=  +   ja   (\textsc{top})   > \=  juuja\kill
    Phonetically:  \> [tɑ̟u] \>  +   ja (\textsc{top})   >  \> [tɑ̟.ʷo̞ː]   (*[tɑ̟u.jɑ̟])\\
    Phonologically: \>   tawu \>  +   ja (\textsc{top})   >  \> tawoo   (*tauja)
\end{tabbing}
\z

In terms of the other morphemes with [ɑ̟u], such as [ɑ̟uː] ‘blue,’ we could not fully determine whether it is /auu/ or /awuu/. However, we do not assume there is a combination of a vowel plus /u/ (besides a vowel plus /i/) for diphthongs since there is no positive indication (considering the case of \textit{tawu} ‘plain’). Thus, we regard [ɑ̟u] in other morphemes as /awu/; that is, /awuu/ ‘blue,’ /jawucikkwəə/ ‘naughty boy,’ and /jawur/ (\textsc{pol}).

\subsection{Consonants}
\subsubsection{The inventory of consonant phonemes}

Yuwan has 22 consonants, listed in \tabref{tab:2:8}.

\begin{table}
\caption{Inventory of consonants\label{tab:2:8}\todo[inline]{please check 1st column}}
\begin{tabular}{rccccc}
\lsptoprule
     & Bilabial &  Alveolar &  Palatal &  Velar & Glottal\\\midrule
voiceless non-glottalized Stops & p & t  & &  k  \\
glottalized Stops  &  & tˀ & &   kˀ  \\
voiced  Stops  &  b  & d  & &  g  \\
voiceless non-glottalized Affricates &   & c      \\
glottalized Affricates  &  & cˀ      \\
voiceless Fricatives   &   & s  & &     h\\
voiced Fricatives  &   & z      \\
non-glottalized Nasals & m & n      \\
{glottalized Nasals} & mˀ & nˀ      \\
{non-glottalized Approximants} & w  & &  j    \\
{glottalized Approximants} & wˀ & &   jˀ    \\
Tap   & & r      \\
\lspbottomrule
\end{tabular}
\end{table}

Notes:
\begin{enumerate}[label=\alph*.]
\item Stops and fricatives have voice opposition;
\item Stops (except for /p/), affricates, nasals, and approximants have glottalization opposition;
\item Alveolar affricates and fricatives behave similarly in terms of morphophonological rules (see \sectref{sec:key:6.3.1.1}, \sectref{sec:key:6.3.1.2}, §\ref{bkm:Ref347177096}, and \sectref{sec:key:10.1.1.1});
\item Approximants and the tap behave similarly in terms of (morpho)phonological rules (§\ref{bkm:Ref304225942} and §\ref{bkm:Ref347177096}).
\end{enumerate}

The phoneme /p/ often appears as a geminate in the combination of a stem and affixes (or clitics). Yuwan has a very restricted number of lexical items that have /p/ (12 lexemes so far), where non-geminated lexemes are \textit{pon+wata} ‘big belly,’ \textit{anpəə} ‘appearance,’ \textit{piri} ‘tail end,’ and \textit{mai=nu} \textit{pɨɨ} (hip=\textsc{gen} hole) ‘anus,’ excluding onomatopoeia and alleged modern loan words. Additionally, /z/ can be realized as [(d͡)z] (or [(d͡)ʑ]) in Yuwan. However, we regard it as a voiced counterpart of the fricative /s/ since /s/ can precede all the vowels that /z/ can precede, but the affricate /c/ cannot precede all of these vowels. For example, there are phoneme sequences such as /za/ or /sa/, but not /ca/ (see the table in §\ref{bkm:Ref347177989}).

The glottalized phonemes could be analyzed as /ʔC/, reducing the total number of phonemes. This analysis would assume double onset slots for the word-initial syllable. However, it is difficult to propose that there is a slot for /ʔ/, since /ʔ/ cannot precede all the consonants. For example, it cannot precede fricatives or /r/. In addition, this analysis destroys the commonality of syllable structures within a word (see §\ref{bkm:Ref301830963}). Thus, I propose the analysis of /Cˀ/. Furtheremore, I do not assume [ʔ] that precedes word-initial vowel as a phoneme, i.e., [ʔɑ̟mɨ] ‘rain’ is /amɨ/ (not /ʔamɨ/), since the occurence of [ʔ] can be predicted by the phonological environments, i.e. a word-initial position preceding a vowel.

The minimal or quasi-minimal contrasts of consonants are shown below.

\ea Stops\\
\ea /t/ vs /tˀ/ vs /d/\\
\gll /tɨɨ/ vs /tˀɨɨ/ vs /dɨɨ/\\
‘hand’ {} ‘one (thing)’ {} /bamboo/\\
\ex /k/ vs /kˀ/ vs /g/\\
\gll /kuran/ vs /kˀura/ vs /gurusa/\\
‘Kuran’ {} ‘storehouse’ {} ‘fast’\\
\ex /kj/ vs /kˀj/\\
\gll /kjaaganaa/ vs /kˀjaa/\\
‘in coming’ {} ‘Kikai island’\\
\ex /p/ vs /t/ vs /k/\\
\gll /pɨɨ/ vs /tɨɨ/ vs /kɨɨ/\\
‘(ass)hole’ {} ‘hand’ {} ‘tree’\\
\ex /b/ vs /d/ vs /g/\\
\gll /baa/ vs /daa/ vs /gan/\\
‘No, thanks.’ {} ‘where’ {} ‘crab’\\
\z
\ex  Affricates and fricatives\\
\ea /c/ vs /z/ vs /s/\\
\gll /sɨcɨ/[sɨt͡sɨ] vs /sɨzɨ/[sɨ(d͡)zɨ] vs /sɨsɨ/[sɨsɨ]\\
‘coffin’ {} ‘tendon’ {} ‘soot’\\
\ex /cj/ vs /cˀj/\\
\gll /cjan/ [t͡ɕɑ̟ɴ] vs /cˀjan/[t͡ɕˀɑ̟ɴ]\\
‘coal tar’ {} ‘father’\\
\ex /s/ vs /h/\\
\gll /sɨɨsa/ vs /hɨɨsa/\\
‘sour’ {} ‘large’\\
\z
\ex  Nasals\\
\ea /m/ vs /mˀ/\\
\gll /mɨɨ/ vs /mˀɨɨ/\\
‘eye’ {}  ‘k.o.fruit’\\
\ex /n/ vs /nˀ/\\
\gll /njɨɨ/ vs /nˀjɨ/\\
‘load’ {}  ‘rice plant’\\
\ex /m/ vs /n/\\
\gll /mai/ vs /nai/\\
‘hip’ {}  ‘seed of cyad’\\
\z
\ex Approximants\\
\ea /w/ vs /wˀ/\\
\gll /waa/ vs /wˀaa/\\
‘my’ {}  ‘pig’\\
\ex /j/ vs /jˀ/\\
\gll /juu/ vs /jˀu/\\
‘boiled water’ {}  ‘fish’\\
\ex /w/ vs /j/\\
\gll /wɨɨ/ vs /jɨɨ/\\
‘tub’ {}  ‘handle’\\
\ex /r/ vs /d/\\
\gll /nuru/[nuɾu] vs /nudu/[nudu]\\
‘moss’ {}  ‘throat’\\
\z
\z

The minimal or quasi-minimal contrasts of geminates and single consonants are shown in \tabref{tab:2:9}.

\begin{table}
\caption{(Quasi-)minimal contrasts of geminates and single consonants\label{tab:2:9}}
\begin{tabular}{ *{5}{l} } 
\lsptoprule
& \multicolumn{2}{l}{Single} &  \multicolumn{2}{l}{Geminate} \\\midrule
/p/ & pocjoopocjo & ‘dripping’ & sippoo & ‘dull (sword)’\\
/b/ & cɨba  & ‘saliva’ & cɨbban & (copulate.\textsc{neg})\\
/t/ & utu   & ‘sound’ & uttui & ‘the day before yesterday’\\
/k/ & sikjan & (spread.\textsc{neg}) &  sikkjan & (sink.NEG)\\
/g/ & hɨgu & ‘k.o. tree’ &  hɨggɨ &  ‘(place name)’\\
/c/ & ucja  & (put.\textsc{pst})  & uccja  & (hit.P\textsc{st})\\
/s/ & kusan & ‘k.o. bamboo’ & kussan & (kill.\textsc{neg})\\
/z/ & azjəə & (taste.\textsc{top}) &  azzjəə & ‘grandfather’\\
/m/ & hɨma & ‘spare time’ & hɨnma & ‘daytime’\\
/n/ & sɨna   & ‘sand’ &  sɨnna &  (do.\textsc{proh})\\
\lspbottomrule
\end{tabular}
\end{table}

Geminate in the right-side column includes the case of archiphoneme /N/ plus /n/ (or /m/) (see §\ref{bkm:Ref347178311}).

\subsubsection{Homorganic nasals}

/n/ and /m/ are homorganic nasals; that is, they assimilate with the place of the following consonants.

\begin{table}\footnotesize
\caption{Homorganic nasals\todo[inline]{transpose table}}
\begin{tabular}{ *{6}{l} } 
\lsptoprule
  & Isolation  &  Before bilabials &  Before alveolars&   Before velars &  Before vowels\\\midrule
 /n/  & un [ʔuɴ]  & un=ba [ʔum.bɑ̟] &  un=doo [ʔun.do̞ː]  & un=gadɨ [ʔuŋ.gɑ̟.dɨ]  & un=un [ʔu.nuɴ]\\
      &sea        & sea=\textsc{acc}        &   sea=\textsc{ass}          &  sea=\textsc{lmt}             & sea=also\\
 /m/  & N/A      & jum-boo [jum.bo̞ː]  & jum-cja [jun.t͡ɕɑ̟] &  jum-gadɨ [juŋ.gɑ̟.dɨ] &  jum-an [ju.mɑ̟ɴ]\\
      & read-\textsc{cnd} &  read-want  & read-until &  read-\textsc{neg}\\
\lspbottomrule
\end{tabular}
\end{table}

In these cases, the underlying forms of the root-final homorganic nasals, i.e., \textit{un} ‘sea’ or \textit{jum-} ‘read,’ can be hypothesized by making use of the phones preceding vowels, such as /un=un/ [ʔu.nuɴ] ‘sea=also’ and /jum-an/ [ju.mɑ̟ɴ] ‘read-\textsc{neg}.’ However, we could not determine the underlying form of nasals that do not occur in morpheme boundaries, such as [ʔɑ̟m.mɑ̟ː] ‘mother,’ [tɨn.no̞ː.gi] ‘rainbow,’ and [iŋ.gɑ̟] ‘man.’ In these cases, we think these ostensible homorganic nasals are “archiphonemes” (\citealt{Lass1984}: 46-49, \citealt{Dixon2010}: 272). In this grammar, we use the letter \textit{n} for the orthographic representation of the archiphonemes, i.e., \textit{anmaa} ‘mother,’ \textit{tɨnnoogi} ‘rainbow,’ and \textit{jinga} ‘man’ (see also “Orthography” in the “Transcrption” in the beginning of this grammar).

\section{Syllable structure and phonotactics}
\label{bkm:Ref302599307}\hypertarget{RefHeadingToc395696969}{}\subsection{The syllable structure and morae}
\label{bkm:Ref301830963}\hypertarget{RefHeadingToc395696970}{}\label{bkm:Ref381399409}
Yuwan has the following syllable structures, and the corresponding morae are also shown. Parentheses indicate the slots are optional. In the syllables in Yuwan, the slot obligatorily filled by a phoneme is only V\textit{\textsubscript{1}}.

\begin{figure}
\caption{\todo[inline]{Please provide a caption}}
\begin{tabular}{cccc}
 (C\textit{\textsubscript{1}} & (G)) & V\textit{\textsubscript{1}} & (V\textit{\textsubscript{2}})  or  (C\textit{\textsubscript{2}})\\
 {}- & {}- & μ  & μ\\
\end{tabular}
\end{figure}

Notes: 
\begin{description}[font=\normalfont]
\item[C\textit{\textsubscript{1}}:] All consonants can fill this slot;
\item[G:] Only /w/ and /j/ can fill this slot;
\item[V\textit{\textsubscript{1}}:] All vowels can fill this slot;
\item[V\textit{\textsubscript{2}}:] The same vowel as V\textit{\textsubscript{1}} can fill this slot; /i/ can also fill this slot (see §\ref{bkm:Ref347178348});
\item[C\textit{\textsubscript{2}}:] Only /n/ can fill this slot at the final position of a phonological word; consonants, except for /h, r/, can fill this slot elsewhere.
\end{description}


Prosody tells us that V\textit{\textsubscript{1}} and V\textit{\textsubscript{2}} cannot be analyzed as /V\textit{\textsubscript{1}}.V\textit{\textsubscript{2}}/ (see §\ref{bkm:Ref301560567}). In addition, morphophonological behavior may also support this analysis (see §\ref{bkm:Ref367134300}). Both the syllable and mora are indispensable units in Yuwan.

There is a strong tendency for a phonological word to have two (or more) morae. The following words do not follow this tendency.

\ea 
\ea Nouns:\\
    /sja/ ‘below,’ /mja/\footnotemark{} ‘snail,’ /cˀju/ ‘person,’ /mˀa/ ‘horse,’ /jˀu/ ‘fish,’ /nˀjɨ/ ‘rice plant’
\ex Verbs:
\ea imperative forms: /mjɨ/ (see.\textsc{imp}), /jˀɨ/ (say.IMP), /jˀɨ/ (sit.IMP), /njɨ/ (boil.IMP)
\ex past forms: /sja/ (do.\textsc{pst}), /cˀja/ (come.P\textsc{st})
\ex sequential converbs:  /sjɨ/ (do.\textsc{seq}), /cˀjɨ/ (come.SEQ)
\z
\z
\z
\footnotetext{This word is pronounced as /mjaa/ with two morae by the speaker \textsc{mt}.}

It is probable that all of the examples had two syllables in the past considering their plausible counterparts in modern Japanese. Take, for example, the following nouns: /sita/ ‘below,’ /mina/ ‘snail’ (in old Japanese), /hito/ ‘person,’ /uma/ ‘horse,’ /iwo/ ‘fish’ (in old Japanese), and /ine/ ‘rice plant.’ Concerning verbs, it is difficult to do such a comparison. Nevertheless, all the plausible counterparts in Japanese have /i/ in the place of /j/ (or /jˀ/); for example, /sita/ (do.\textsc{pst}) and /kita/ (come.P\textsc{st}). Furthermore, there is a phenomenon which shows the bimoraic tendency applying to some verbal stems as if they were phonological words by themselves, i.e., the verbal stems preceding type D affixes (see the footnote Error: Reference source not found in §\ref{bkm:Ref347177096}).

\subsection{Phonotactics}
\label{bkm:Ref302599510}\hypertarget{RefHeadingToc395696971}{}
The following constraints (or tendencies) are determined from the behavior of monomorphemic and polymorphemic phonological words.

\ea Phonotactic constraints (or tendencies):\label{ex:2.8}
\ea Non-nasal resonants cannot be followed by approximants, i.e., /*rj/, /*jj/, and /*wj/ (see \sectref{sec:key:8.2.1.3});\label{ex:2.8a}
\ex Glottalized consonants can appear only at stem-initial positions (see below);\label{ex:2.8b}
\ex A sequence of consonants is geminate or its first consonant is nasal;\label{ex:2.8c}
\ex A monomorphemic word does not have voiced geminates (with the exception of the three lexemes /cɨbb/ ‘copulate,’ /azzjəə/ ‘grandfather,’ and /hɨggɨ/ ‘(place name)’). In addition, a phonological word made of polymorphemes tends to avoid voiced geminates (see §\ref{bkm:Ref347178914});\label{ex:2.8d}
\ex A monomorphemic word has a sequence with at most two vowels (with the exception of the three lexemes /jɨɨi/ ‘brother,’ /dooi/ ‘reason’ (sometimes pronounced as /doi/), and /tuuii/ ‘(place name)’); a phonological word made of polymorphemes tends to restrict a sequence made of three vowels (see §\ref{bkm:Ref301832441});\label{ex:2.8e}
\ex A monomorphemic word does not have the VVC\textsubscript{coda} sequence (with the exception of /koonmja/ ‘k.o. shellfish living in the river’\footnote{It creates a minimal pair with /konmja/ ‘a kind of shellfish living in the beach.’} and /sjoogoin/ ‘k.o. white radish,’ the latter thought to be a loan word from Modern Japanese); a phonological word made of polymorphemes tends to restrict the V\textit{\textsubscript{i}}V\textit{\textsubscript{i} }C\textsubscript{coda} sequence (see §\ref{bkm:Ref301832441});\label{ex:2.8f}
\ex A sequence of C\textsubscript{coda}.V never appears (see §\ref{bkm:Ref347173344});\label{ex:2.8g}
\ex A monomorphemic word does not have a sequence of a nasal coda followed by an onset /j/, i.e., */n.j/ and */m.j/; however, a phonological word consisting of more than one morpheme may have this sequence (see §\ref{bkm:Ref347179283});\label{ex:2.8h}
\ex The consonants that can precede /w/ filled in G slot are only /kˀ/, /k/ and /h/ (\tabref{tab:key:18} in \sectref{sec:key:2.3.2.5});\label{ex:2.8i}
\z
\z

Phonotactics determine the possible combinations of phonemes in a phonological word (see §\ref{bkm:Ref347179371}), and we have to pay attention to the following two types of sounds: (A) glottalized consonants, i.e., /Cˀ/ and (B) non-glottalized palatal approximant, i.e., /j/.

First, glottalized consonants can appear in a word-initial position such as \textit{jˀu} ‘fish,’ but cannot appear in a non-word-initial position in a simple word. For example, there is no word made of /VCˀV/; however, in the case of compounds, glottalized consonants can appear in a non-word-initial position, e.g., \textit{aa+jˀu} (red+fish) ‘red fish.’ In other words, glottalized consonants can appear in a stem-initial position. If we adopt the possibility of the occurence of glottalized consonants as a criterion of the phological word, there would be a mismatch among the criterion about glottalized consonants and that mentioned in \sectref{sec:key:2.1.} This type of mismatch between the criteria of phonological words, however, is not uncommon. In fact, \citet[18]{DixonAikhenvald2002} wrote that “(d)ifferent types of criteria are relevant to defining the phonological word in different languages. And the relative importance and weighting of criteria differ from language to language.” In this grammar, the possibility of the occurence of glottalized consonants is not adopted as the criterion of the phonological word, and I only mention its mismatch with other criteria.

Second, there are two types of morphemes beginning with /j/: one type palatalizes the preceding phoneme, as in (\ref{ex:2.9}a--b), while another type does not, as in (\ref{ex:2.9}c--e).

\ea Palatalization\\\label{ex:2.9}
\begin{tabular}{lllllllll}
   &   \multicolumn{2}{l}{Former}  & &  \multicolumn{2}{l}{Latter}   & &   {Latter}\\
a. & {\itshape jum-}  {‘read’}  {+}  {\itshape {}-jaa}  {‘person’}  {>}  {ju.mjaa [ju.mʲɑ̟ː]}  {Affix}\\
b. & {\itshape jum-}  {‘read’}  {+}  {\itshape {}-jagacinaa}  {(\textsc{sim})}  {>}  {ju.mja.ga.ci.naa [ju.mʲɑ̟.gɑ̟.t͡ɕi.nɑ̟ː]}  {Affix}\medskip\\
   & \multicolumn{3}{l}{Non-palatalization} \\
c. & {\itshape mun}  {(\textsc{advrs})}  {+}  {\itshape jaa}  {(\textsc{sol})}  {>}  {mun.jaa [muɴ.jɑ̟ː]}  {Clitic}\\
d. & {\itshape jum-∅}  {(read-\textsc{inf})}  {+}  {\itshape jass-sa}  {(easy-\textsc{adj})}  {>}  {jum.jas.sa [juɴ.jɑ̟s.sɑ̟]}  {Root}\\
e. & {\itshape nɨkan}  {‘orange’}  {+}  {\itshape jama}  {‘mountain’}  {>}  {nɨ.kan.ja.ma [nɨ.kɑ̟ɴ.jɑ̟.mɑ̟]}  {Root}\\
\end{tabular}
\z

These examples show that if the following morpheme (the morphological status of the following morphemes is shown in the right-most column labeled “Latter”) is a clitic or a root, palatalization does not occur. However, if it is an affix, palatalization necessarily occurs. In this grammar, the syllable boundary between /m/ and /j/ in \textit{jum-∅+jass-sa} (read-\textsc{inf}+easy-\textsc{adj}) ‘easy to read’ is expressed by a period mark such as /jum.jassa/ in the surface form level.

\subsubsection{Monosyllabic words}
\begin{table}
\caption{Monosyllabic (and monomorphemic) grammatical words}
\begin{tabular}{l@{ }l@{ }llccc} 
\lsptoprule
                 & &   & C  & G &  V  &  V (or C)\\\midrule
/ai/ & [ʔɑ̟i] & ‘No’    &    &   &   a  & i\\
/an/ & [ʔɑ̟ɴ] & ‘that’  &    &   & a   & n\\
/jaa/ & [jɑ̟ː] & ‘house’  & j &  &  a &  a\\
/wan/ & [wɑ̟n] & ‘I’  & w  & &  a &  n\\
/naa/ & [nɑ̟ː] & ‘name’  & n  & &  a &  a\\
/mja/ & [mʲɑ̟] & ‘k.o.shellfish’  & m &  j &  a  \\
/mjaa/ & [mʲɑ̟ː] & ‘cat’ &  m &  j &  a &  a\\
/nan/ & [nɑ̟ɴ] & ‘you.\textsc{hon}’  & n  &  &  a &  n\\
/cjan/ & [tɕɑ̟ɴ] & ‘coal tar’  & c  & j  & a &  n\\
/mˀa/ & [Ɂmɑ̟] & ‘horse’  & mˀ  & &  a & \\
/wˀaa/ & [Ɂwɑ̟ː] & ‘pig’  & wˀ &  &  a &  a\\
/kˀjaa/ & [kˀʲɑ̟ː] & ‘Kikai island’ &  kˀ &  j &  a &  a\\
/cˀjan/ & [t͡ɕˀɑ̟ɴ] & ‘father’  & cˀ &  j &  a &  n\\
\lspbottomrule
\end{tabular}
\end{table}

\subsubsection{Polysyllabic phonological words}
\label{bkm:Ref347179283}\hypertarget{RefHeadingToc395696973}{}\label{bkm:Ref347178311}
In principle, the phonotactics of polysyllabic phonological words are the same as those of monomorphemic ones, but there is an important difference in terms of the phonemes that can fill coda slots. In monosyllabic words, the coda slots in word-final position can only be filled by /n/. However, in polysyllabic words, the coda slots in word-internal position can be filled by many kinds of consonants. The possible combinations of consonants around a syllable boundary are shown below, including the total number of monomorphemic lexemes that have such a sequence (out of approximately 1,000 lexemes). In the following table, /N/ indicates the archiphoneme (see also “Transcription” in the beginning of this grammar and §\ref{bkm:Ref347174390} for more details).

\begin{sidewaystable}
\caption{/C.C/ combination in polysyllabic phonological words (monomorphemic)}
\begin{tabular}{ *{10}{l} S[table-format=1] }
\lsptoprule
\multicolumn{4}{l}{} &   C & G & V & C  & .C  & &  {Number}\\\midrule
{/p.p/:}     & {/sip.poo/}       & [ɕip.po̞ː]          &  {‘blunt’}             &  s  &  &    i  &  p   &  .p  & oo & 6\\
{/b.b/:}     & {/cɨb.bi.da.ci/}  & [t͡sɨb.bi.dɑ̟.t͡ɕi] &  {‘rut (of animal)’}   &  c  &  &    ɨ  &  b   &  .b  & idaci &  1\\
{/t.t/:}     & {/at.ta.kəə/}     & [ʔɑ̟ttɑ̟kɜː]        &   {‘everything’}       & &  & a   &     t &  .t  &   akəə  & 16\\
{/k.k/:}     & {/juk.ka.dɨ/}     & [jukkɑ̟dɨ]          &  {‘throughout’}        &  j  &  &    u  &  k   &  .k &  adɨ & 14\\
{/g.g/:}     & {/hɨg.gɨ/}        & [xɨggɨ]             & {‘(place name)’}       & h   &  &   ɨ   & g    & .g &  ɨ &  1\\
{/c.c/:}     & {/gac.cɨn/}       & [gɑ̟tt͡sɨɴ]         &  {‘saurel’}            &  g  &  &    a  &  c   &  .c &  ɨn &  7\\
{/s.s/:}     & {/kas.sa/}        & [kɑ̟ssɑ̟]           &   {‘like this’}        &   k &  &     a &   s  &   .s &  a &  9\\
{/z.z/:}     & {/az.zjəə/}       & [ʔɑ̟dd͡ʑɜː]         &  {‘grandfather’}      & & &  a   &    z  & .z   &  jəə &  1\\
{/N/ + /p/:} & {/an.pəə/}        & [ʔɑ̟m.pɜː]          &  {‘appearance’}       & & & a   &    n  & .p   &  əə &  2\\
{/N/ + /b/:} & {/gan.boo/}       & [gɑ̟m.bo̞ː]         &   {‘naughty boy/girl’} &   g &   &     a &   n  &   .b & oo &  1\\
{/N/ + /t/:} & {/nin.təə/}       & [nin.tɜː]           & {‘group’}              & n   & &   i   & n    & .t  & əə & 2\\
{/N/ + /d/:} & {/cɨn.dai/}       & [t͡sɨn.dɑ̟i]        &  {‘snail’}             &  c  &  &    ɨ  &  n   &  .d & ai & 7\\
{/N/ + /k/:} & {/in.ku.zjaa/}    & [ʔiŋ.ku.(d͡)ʑɑ̟ː]   &  {‘(place name)’}      & & & i &    n  & .k   &  uzjaa & 5\\
{/N/ + /g/:} & {/jin.ga/}        & [iŋ.gɑ̟]            &  {‘man’}               &  j  & &    i  &  n   &  .g & a & 10\\
{/N/ + /c/:} & {/kan.cɨmɨ/}      & [kɑ̟n.t͡sɨ.mɨ]      &  {‘(name of person)’}  &  k  & &    a  &  n   &  .c & ɨmɨ & 1\\
{/N/ + /s/:} & {/han.sɨ/}        & [hɑ̟ɴ.sɨ]           &  {‘sweet potato’}      &  h  & &    a  &  n   &  .s & ɨ & 4\\
{/N/ + /z/:} & {/hin.zjaa/}      & [çin.(d͡)ʑɑ̟ː]      &  {‘goat’}              &  h  & &    i  &  n   &  .z & jaa & 5\\
{/N/ + /m/:} & {/an.maa/}        & [ʔɑ̟m.mɑ̟ː]         &   {‘mother’}           &  &   &a   &     n &   .m  &  aa & 8\\
{/N/ + /n/:} & {/han.njəə/}      & [hɑ̟n.njɜː]         &  {‘grandmother’}       &  h  &  &    a  &  n    & .n & jəə & 6\\
\lspbottomrule
\end{tabular}
\end{sidewaystable}

There are no monomorphemic words with the sequences of /dd/, /hh/, or /rr/ in Yuwan. The data show that the number of monomorphemic lexemes that have C\textsubscript{coda}.C\textsubscript{onset} sequences are very small; however, this sequence is not uncommon in the case of polymorphemic phonological words, such as \textit{ar-} ‘exist’ + \textit{doo} (\textsc{ass}) > /at.too/ and \textit{ar} ‘exist’ + \textit{ba} (\textsc{csl}) > /ap.pa/. These sequences are formed by the (morpho)phonological rules (see §\ref{bkm:Ref347178914} and §\ref{bkm:Ref347177096}). In monomorphemic words, it is impossible to determine the (morpho)phoneme of the nasal that fills the C\textsubscript{coda} slot in the C\textsubscript{coda}.C sequence, but it is possible to do so in polymorphemic phonological words, as shown below.

\begin{table}\footnotesize
\caption{/Nasal + C/ combination in polysyllabic phonological words (polymorphemic)}
\begin{tabular}{ *{10}{l} }
\lsptoprule
 \multicolumn{4}{l}{} &  {C} & {G} & {V} & {C} & {.C} & {}\\\midrule
{/m.b/:} & {/jum.ba/}     & [jum.bɑ̟]      &  {(read.\textsc{csl})}          &   j   & &  u & m & .b &a\\
{/m.d/:} & {/jum.doo/}    & [jun.do̞ː]     &  {(read.\textsc{inf}.\textsc{ass})}      &   j   & &  u & m & .d &oo\\
{/m.k/:} & {/kam.kai/}    & [kɑ̟ŋ.kɑ̟i]    &   {(eat.\textsc{du}B)}          &    k  & &   a&  m&  .k& ai\\
{/m.g/:} & {/jum.ga.dɨ/}  & [juŋ.gɑ̟.dɨ]   &  {(read.until)}        &   j   & &  u & m & .g &adɨ\\
{/m.c/:} & {/jum.cja.sa/} & [jun.t͡ɕɑ̟.sɑ̟]&   {(read.\textsc{inf}.want.\textsc{adj})}&    j  &  &   u&  m&  .c& jasa\\
{/m.n/:} & {/jum.nja/}    & [junʲ.nʲɑ̟]    &  {(read.\textsc{inf}.\textsc{top})}      &   j   & &  u & m & .n &ja\\
{/m.j/:} & {/jum.jas.sa/} & [juɴ.jɑ̟s.sɑ̟] &   {(read.\textsc{inf}.easy.\textsc{adj})}&    j  & &   u&  m&  .j& assa\\
{/n.b/:} & {/nɨ.kan.ba/}  & [nɨ.kɑ̟m.bɑ̟]  &   {(orange.\textsc{acc})}       &    nɨ.   k  & &  a & n & .b & a\\
{/n.t/:} & {/nan.tu/}     & [nɑ̟n.tu]      &  {(you.\textsc{hon}.\textsc{com})}       &   n &   &  a  &  n& .t& u\\
{/n.d/:} & {/kin.du/}     & [kˀin.du]      & {(clothes.\textsc{foc})}        &  k &   & i   & n &.d &u\\
{/n.k/:} & {/un.ka.ci/}   & [ʔuŋ.kɑ̟.t͡ɕi] &  {(sea.\textsc{all})}           & & &   u   &  n  &  .k & aci\\
{/n.g/:} & {/wan.ga/}     & [wɑ̟ŋ.gɑ̟]     &   {(1\textsc{sg}.\textsc{nom})}          &    w  & &   a &    n & .g & a\\
{/n.n/:} & {/wan.na/}     & [wɑ̟n.nɑ̟]     &   {(1\textsc{sg}.\textsc{top})}          &    w  & &   a &   n & .n & a\\
{/n.j/:} & {/mun.jaa/}    & [muɴ.jɑ̟ː]     &  {(\textsc{advrs}.\textsc{sol})}         &   m   & &  u  &  n & .j & aa\\
\lspbottomrule
\end{tabular}
\end{table}

As mentioned in (\ref{ex:2.8}h) in §\ref{bkm:Ref302599510}, a sequence of C\textsubscript{coda}.C\textsubscript{onset} (C\textsubscript{coda} is nasal, C\textsubscript{onset} is /j/) never appears in monomorphemic grammatical words; however, it can appear in polymorphemic phonological words (see the examples of /m.j/ and /n.j/ above). There are four morphemes able to make this sequence: \textit{jass} ‘easy,’ \textit{jaa} (\textsc{sol}), \textit{joo} (\textsc{cfm}1), and \textit{jukkuma} (\textsc{cmp}).

\subsubsection{Glottalized consonants}
\label{bkm:Ref347180773}\hypertarget{RefHeadingToc395696974}{}
Phonologically, glottalized consonants are contrastive only at stem-initial positions. Phonetically, they require laryngeal intension and may be divided into two types: glottalized obstruents [tˀ, t͡ɕˀ, kˀ] and glottalized sonorants [ʔm, ʔn, ʔj, ʔw]. The former group sounds like unaspirated obstruents in Chinese or unaspirated tense obstruents in Korean, and a more detailed phonetic comparison should be done in the future. The latter group has the following two characteristics (compared with non-glottalized sonorants [m, n, j, w]): \textsc{ref}{ex:key:1} relatively larger amplitude in the onset, \REF{ex:key:2} relatively shorter duration in the onset, which indicates their coarticulation with the glottal stop in the onset position \citep{NiinagaEtAl2011}. Word initial /p/, /cɨ/, and /ki/ are basically phonetically glottalized, and they appear to have developed from historical changes (cf. \citealt{HirayamaEtAl1966}: 22-23), but the details of their development are beyond the scope of this grammar.

Glottalized consonants are proposed to have developed from two phonological processes: \textsc{ref}{ex:key:1} syllable omission and \REF{ex:key:2} retainment of a distinction affected by vowel merger (\citealt{HirayamaEtAl1966}: 22-23). An example of the former is */hutari/ > /tˀai/ ‘human’ (/ri/ > /i/ is also a synchronic phonological rule in §\ref{bkm:Ref304225942}). An example of the latter is */kome/ > /kumɨ/, and */kura/ > /kˀura/, where */o/ is merged with */u/ and both become /u/ (the change of */e/ > /ɨ/ is another historical change that is not addressed here). Previous research has shown that */ku/ became /kˀu/ in order to retain a difference from /ku/ (made of */ko/) (\citealt{HirayamaEtAl1966}: 23). Almost all of the current tokens of /kˀ/ in Yuwan have developed from */ku/. Additionally, /kˀjaa/ [kˀʲɑ̟ː] ‘Kikai-zima,’ which is the name of an island, appears to have developed from syllable omission. There are a number of lexicon that has /kˀ/ in modern Yuwan. The other glottalized phonemes seem to have developed as a result of syllable omission. This process does not seem to have been common, so there are only a few lexemes that have these glottalized phonemes. The following table shows the number of lexemes that have word-initial glottalized phonemes (and their examples) compared with that of non-glottalized initial phonemes.

\begin{table}
\caption{Lexemes that have word-initial glottalized phonemes (out of approximately 1,000 lexemes)}
\begin{tabularx}{\textwidth}{ ll S[table-format=1] Q l S[table-format=2]}
\lsptoprule
{Phonemes} & {Allophones} & {Number} & Examples   &    cf.  & {Number}\\\midrule
{/wˀ/}  & {[ʔw]}     & 2  & [ʔwɑ̟ː]  ‘pig’         \newline\relax  [ʔwɑ̟bijɑ̟ː]  ‘instep’ & /w/  & 18\\
{/tˀ/}  & {[tˀ]}     & 3  & [tˀɑ̟i]  ‘two persons’ \newline\relax  [tˀɨɨ]  ‘one thing’  & /t/  & 59\\
{/nˀj/} &  {[ʔnʲ]}   & 3  & [ʔnʲut͡ɕi]  ‘life’    \newline\relax [ʔnʲɨ]  ‘rice plant’ & /nj/ & 2\\
{/kˀj/} &  {[kˀʲ]}   & 5  & [kˀʲɑ̟ː]  ‘Kikai-zima’ \newline\relax  [kˀʲubiː]  ‘band’    & /kj/ & 7\\
{/mˀ/}  & {[ʔm]}     & 4  & [ʔmɑ̟]  ‘horse’        \newline\relax  [ʔmɑ̟t͡sɨ]  ‘fire’    & /m/  & 96\\
{/cˀj/} &  {[t͡ɕˀ]}  & 5  & [t͡ɕˀɑ̟ɴ]  ‘father’    \newline\relax  [t͡ɕˀu]  ‘person’    & /cj/ & 5\\
{/jˀ/}  & {[ʔj]}     & 5  & [ʔju]  ‘fish’         \newline\relax [ʔjɑ̟]  ‘arrow’       & /j/  & 63\\
{/kˀ/}  & {[kˀ]}     & 35 & [kˀubi]  ‘neck’       \newline\relax [kˀuru(ː)]  ‘black’  & /k/  & 81\\\midrule
\end{tabularx}\smallskip\\
\raggedright
\textit{Note:}\\
\begin{enumerate}[label=\alph*.,nosep]
\item The number of /C\textit{\textsubscript{i} }/ and /C\textit{\textsubscript{i} }j/ is not redundant. For example, the number of /k/ excluded the number of /kj/;
\item The number of lexemes that have non-glottalized initial /k/ excludes that of /ki/ [kˀi].
\end{enumerate}
\begin{tabularx}{\textwidth}{X}\lspbottomrule\end{tabularx}
\end{table}

The data show there are fewer lexemes that have word-initial glottalized phonemes than non-glottalized ones; however, the number of lexemes with /Cˀj/ and /Cj/ does not follow this pattern. In fact, the number of combinations where a consonant is followed by /j/ in these examples is relatively small, so it is not meaningful to compare these particular consonants.

Since there are fewer lexemes that have word-initial glottalized phonemes than non-glottalized ones, we propose that the former are “marked” phonemes. Therefore, if a “phonetically” word-initial glottalized consonant does not have a “phonemic” contrast with a non-glottalized one, we regard it as a “phonemically non-glottalized” phoneme. For example, Yuwan has only [pˀ], but this phoneme is interpreted as /p/ in this grammar. Moreover, there are no word-internal contrasts with glottalization in Yuwan, so word-internal phonemes are always phonemically non-glottalized even if they might be phonetically glottalized (with the exception of the case of compounds, see §\ref{bkm:Ref302599510}). The combination of velar stop and /w/ is always realized as [kˀʷ], but we will interpret it as /kˀw/ with the exception of the case of \textit{{}-kkwa} (\textsc{dim}) and /joikwa/ ‘silently’ (see \sectref{sec:key:7.7}) against the markedness principle because the interpretation as /kˀw/ makes it easier to explain a prosodic phenomenon discussed in §\ref{bkm:Ref347180628}.

\subsubsection{Interpretation of /C/ + /j/ combination}
\label{bkm:Ref347180720}\hypertarget{RefHeadingToc395696975}{}
Yuwan has a contrast between [ɕ] and [s]: [kɑ̟ɕɑ̟] ‘wrapping leaf’ vs. [kɑ̟sɑ̟] ‘bamboo hat.’ In this grammar, [ɕ] is interpreted as /sj/ (except for the case of [ɕi]\footnote{[ɕi] is regarded as /si/ (not */sji/) to keep the full set of combinations with /s/ and vowels, since /\textsc{cv}/ is a more productive combination than /CjV/. For example, /b/ can precede any vowel, but /bj/ can only precede /a/ and /u/ (see §\ref{bkm:Ref347180694}).}). There are two reasons why we do not assign a new phoneme /ɕ/: \textsc{ref}{ex:key:1} the overall number of phonemes, and \REF{ex:key:2} morphology.

First, we do not need another new phoneme if we interpret [ɕ] as /sj/, so this interpretation is more economical than the other.

Second, Yuwan has an affix \textit{{}-jaa} ‘person,’ which can nominalize verbal roots (see \sectref{sec:key:7.6}). For example, if the affix follows \textit{hɨmɨkas-} ‘get drunk,’ it becomes [xɨmɨkɑ̟ɕɑ̟ː] ‘drunken person.’ In this case, there would be two interpretations: \textsc{ref}{ex:key:1} /hɨmɨkasjaa/, or \REF{ex:key:2} /hɨmɨkaɕaa/. The first interpretation is transparent, but the second is not because it needs an alternation rule, i.e., //s// + //j// > /ɕ/. The affix \textit{{}-jaa} is fairly productive, such as \textit{tug-} ‘whet’ + \textit{{}-jaa} ‘person’ > /tugjaa/ [tugʲɑ̟ː] ‘a person who whet cutlery professionally’ and \textit{kik-} ‘hear’ + \textit{{}-jaa} ‘person’ > /kikjaa/ [kikʲɑ̟ː] ‘audience.’ Thus, it is (paradigmatically) natural to regard [ɸumukɑ̟ɕɑ̟ː] as /humukasjaa/. Therefore, we adopt the interpretation of [ɕ] as /sj/ in Yuwan (cf., \citet[79-81]{Shimoji2008} for a similar argument in Irabu Ryukyuan).

The same argument can be applied to /cj/ [t͡ɕ]: \textit{ut-} ‘hit’ + \textit{{}-jaa} ‘person’ > /ucjaa/ [ʔut͡ɕɑ̟ː] ‘a person who plays a role to hit someone,’ where an alternation rule from //t// to /c/ is applied (see §\ref{bkm:Ref347180796} for more details). In this case, the merit of regarding [t͡ɕ] not as a new phoneme but as a combination of two existing phonemes remains to be valid. Yuwan has no verbal roots that end with /z/, but there is no reason to treat /zj/ differently from /cj/, so we interpret [d͡ʑ] as /zj/.

\subsubsection{Combination of consonants and vowels}
\hypertarget{RefHeadingToc395696976}{}\label{bkm:Ref347177989}\label{bkm:Ref347180694}\label{bkm:Ref347181003}
The combinations of consonants and vowels, followed by examples, are shown in the following tables.

Pre-notes: 

\begin{enumerate}[label=\alph*.]
\item It might be possible to find combinations for the blank cells, but they have not yet been found so far.
\item If a plausible phonetic combination in one cell (e.g., /t/ + /ja/ > [t͡ɕɑ̟]) is regarded as a combination in another cell (e.g., /cja/), it will be shown in this way “[t͡ɕɑ̟]=/cja/” (cf. §\ref{bkm:Ref347180720}).
\item N/A means such a combination is prohibited by either phonological rules (see §\ref{bkm:Ref302723494}) or the syllable structure (see §\ref{bkm:Ref301830963}).
\item Parenthesized phones mostly appear in stem-initial position (cf. §\ref{bkm:Ref347180773}).
\item Glottalization of the second phoneme of a geminate is not taken into consideration.
\end{enumerate}

\begin{sidewaystable}
\caption{Combinations of \textsc{cv} and CjV showing allophones}
\footnotesize
\begin{tabular}{@{}*{12}{l@{\hspace{1em}}}l@{}} 
\lsptoprule
 & a & i & u & ɨ & ə & o & ja & ji & ju & jɨ & jə & jo\\\midrule
\textminus\footnote{This means there is no consonant in the onset C slot.} & [(ʔ)ɑ̟] & [(ʔ)i] & [(ʔ)u] & [(ʔ)ɨ] & [(ʔ)ɜ] & [(ʔ)o̞] & N/A & N/A & N/A & N/A & N/A & N/A\\
p & [p(ˀ)ɑ̟] & [pʲ(ˀ)i] & [p(ˀ)u] & [p(ˀ)ɨ] & [p(ˀ)ɜ] & [p(ˀ)o̞] & [p(ˀ)ʲɑ̟] &  & [p(ˀ)ʲu] &  &  & \\
b & [bɑ̟] & [bʲi] & [bu] & [bɨ] & [bɜ] & [bo̞] & [bʲɑ̟] &  & [bʲu] &  &  & \\
t & [tɑ̟] & [t͡ɕi]=/ci/ & [tu] & [tɨ] & [tɜ] & [to̞] & [t͡ɕɑ̟]=/cja/ &  & [t͡ɕu]=/cju/ & [t͡ɕɨ]=/cjɨ/ & [t͡ɕɜ]=/cjə/ & [t͡ɕo̞]=/cjo/\\
tˀ & [tˀɑ̟] &  &  & [tˀɨ] &  & [tˀo̞] &  &  &  &  &  & \\
d & [dɑ̟] & [d͡ʑi]=/zi/ & [du] & [dɨ] & [dɜ] & [do̞] & [d͡ʑɑ̟]=/zja/ &  & [d͡ʑu]=/zju/ & [d͡ʑɨ]=/zjɨ/ & [d͡ʑɜ]=/zjə/ & [d͡ʑo̞]=/zjo/\\
k & [kɑ̟] & [kʲ(ˀ)i] & [ku] & [kɨ] & [kɜ] & [ko̞] & [kʲɑ̟] &  & [kʲu] & [kʲɨ] &  & [kʲo̞]\\
kˀ &  & [kʲˀi]=/ki/ & [kˀu] &  &  &  & [kˀʲɑ̟] &  & [kʲu] &  &  & [kˀʲo̞]\\
g & [gɑ̟] & [gʲi] & [gu] & [gɨ] & [gɜ] & [go̞] & [gʲɑ̟] &  & [gʲu] & [gʲɨ] &  & [gʲo̞]\\
c &  & [t͡ɕ(ˀ)i] & [t͡su] & [t͡s(ˀ)ɨ] & [t͡sɜ] &  & [t͡ɕɑ̟] &  & [t͡ɕu] & [t͡ɕɨ] & [t͡ɕɜ] & [t͡ɕo̞]\\
cˀ &  & [t͡ɕˀi]=/ci/ &  & { [t͡sˀɨ]=/cɨ/} &  &  & [t͡ɕˀɑ̟] &  & [t͡ɕˀu] & [t͡ɕˀɨ] & [t͡ɕˀɜ] & [t͡ɕˀo̞]\\
s & [sɑ̟] & [ɕi] & [su] & [sɨ] & [sɜ] & [so̞] & [ɕɑ̟] &  & [ɕu] & [ɕɨ] & [ɕɜ] & [ɕo̞]\\
z & [(d͡)zɑ̟] & [(d͡)ʑi] &  & [(d͡)zɨ] & [(d͡)zɜ] &  & [(d͡)ʑɑ̟] &  & [(d͡)ʑu] & [(d͡)ʑɨ] & [(d͡)ʑɜ] & [(d͡)ʑo̞]\\
h & [hɑ̟] & [çi] & [ɸu] & [xɨ] & [hɜ] & [ho̞] &  &  & [çu] &  &  & \\
m & [mɑ̟] & [mʲi] & [mu] & [mɨ] & [mɜ] & [mo̞] & [mʲɑ̟] & [mʲi] & [mʲu] & [mʲɨ] &  & [mʲo̞]\\
mˀ & [ʔmɑ̟] &  &  & [ʔmɨ] &  & [ʔmo̞] &  &  &  &  &  & \\
n & [nɑ̟] & [nʲi] & [nu] & [nɨ] & [nɜ] & [no̞] & [nʲɑ̟] &  & [nʲu] & [nʲɨ] & [nʲɜ] & [nʲo̞]\\
nˀ &  &  &  &  &  &  &  &  & [ʔnʲu] & [ʔnʲɨ] & [ʔnʲɜ] & \\
w & [wɑ̟] & N/A & [wu] & [wɨ] & [wɜ] & [wo̞] & N/A & N/A & N/A & N/A & N/A & N/A\\
wˀ & [ʔwɑ̟] &  &  &  &  &  &  &  &  &  &  & \\
j & [jɑ̟] & [i] & [ju] & [jɨ] & [jɜ] & [jo̞] & N/A & N/A & N/A & N/A & N/A & N/A\\
jˀ & [ʔjɑ̟] & [ʔi] & [ʔju] & [ʔjɨ] &  & [ʔjo̞] & N/A & N/A & N/A & N/A & N/A & N/A\\
r & [ɾɑ̟] & N/A & [ɾu] & [ɾɨ] & [ɾɜ] & [ɾo̞] & N/A & N/A & N/A & N/A & N/A & N/A\\
\lspbottomrule
\end{tabular}
\end{sidewaystable}

\begin{sidewaystable}
\caption{Examples of \textsc{cv}}
\footnotesize
\begin{tabular}{@{} l@{\hspace{.75em}} *{5}{l@{ }l@{\hspace{.75em}}} l@{ }l@{}}
\lsptoprule
  & a &  & i &  & u &  & ɨ &  & ə &  & o & \\\midrule
\textminus & aasa & ‘red’ & isi & ‘stone’ & uma & ‘there’ & ɨn & ‘dog’ & əəcɨrɨ & ‘classmate’ & oonazi & ‘k.o.sneak’\\
p & gappaa & ‘fist’ & piri & ‘tail end’ & roppu & ‘rope’ & pɨɨ & ‘(ass)hole’ & anpəə & ‘state’ & ponwata & ‘big belly’\\
b & naba & ‘mushroom’ & bija & ‘leek’ & habu & ‘k.o. snake’ & warabɨ & ‘child’ & ɨbəəsa & ‘narrow’ & zɨboo & ‘tail’\\
t & tanɨ & ‘seed’ &  &  & tui & ‘bird’ & tɨn & ‘sky’ & nɨntəə & ‘members’ & bottobotto & ‘lazily’\\
tˀ & tˀai & ‘two people’ &  &  &  &  & tˀɨɨ & ‘one’ &  &  & tˀoomu.nii & ‘Tsutomu’\\
d & kada & ‘smell’ &  &  & dusi & ‘friend’ & dɨru & ‘which’ & kjoodəə & ‘brother’ & dookunɨɨ & ‘white radish’\\
k & kabi & ‘paper’ & kin & ‘clothes’ & kuma & ‘here’ & kɨɨ & ‘tree’ & kəənja & ‘arm’ & koo & ‘skin’\\
kˀ &  &  &  &  & kˀura & ‘storehouse’ &  &  &  &  &  & \\
g & gan & ‘crab’ & ginməə & ‘contract’ & wunagu & ‘woman’ & hagɨr & ‘bald’ & kugəər & ‘tumble’ & kagoo & ‘basket’\\
c &  &  & cikjara & ‘power’ & cubusi & ‘knee’ & cɨmɨ & ‘nail’ & miicəə & (three.\textsc{top}) &  & \\
s & sataa & ‘sugar’ & siju & ‘soup’ & sura & ‘treetop’ & sɨba & ‘tongue’ & səə & ‘alcohol’ & soo & ‘stem’\\
z & sijuzataa & ‘white sugar’ & ziju & ‘cooking stove’ &  &  & kazɨ & ‘wind’ & kazəə & (wing.\textsc{top}) &  & \\
h & hana & ‘nose’ & hindjaa & ‘goat’ & hunɨ & ‘ship’ & hɨnma & ‘day’ & həəsa & ‘quick’ & hoorasja & ‘happy’\\
m & mamɨ & ‘bean’ & min & ‘ear’ & munɨ & ‘breast’ & mɨzɨ & ‘water’ & məə & ‘front’ & umoor & (move.\textsc{hon})\\
mˀ & mˀa & ‘horse’ &  &  &  &  & mˀɨɨ & ‘k.o. fruit’ &  &  & mˀoo & (horse.\textsc{top})\\
n & nama & ‘now’ & nissja & ‘similar’ & nudu & ‘throat’ & nɨzɨn & ‘mouse’ & junəə & ‘evening’ & noo & ‘fishing line’\\
w & wan & ‘I’ &  &  & wutu & ‘husband’ & wɨɨ & ‘tub’ & juwəə & ‘celebration’ & tawoo & (plain.\textsc{top})\\
wˀ & wˀaa & ‘pig’ &  &  &  &  &  &  &  &  &  & \\
j & jama & ‘mountain’ & jinga & ‘man’ & juru & ‘night’ & jɨɨ & ‘grip’ & kawajəə & ‘substitute’ & joikwa & ‘silently’\\
jˀ & jˀa & ‘arrow’ & jˀii & (say.\textsc{inf}) & jˀu & ‘fish’ & jˀɨ & (say.\textsc{imp}) &  &  & jˀoo & (say.\textsc{int})\\
r & warabɨ & ‘child’ &  &  & dɨru & ‘which’ & kurɨ & ‘this’ & kurəə & (this.\textsc{top}) & sɨroo & ‘lie’\\
\lspbottomrule
\end{tabular}
\end{sidewaystable}

\begin{sidewaystable}
\caption{Examples of CjV}
\footnotesize
\begin{tabular}{@{}l @{\hspace{.5em}}  *{5}{l@{ }l@{\hspace{.5em}}} l@{ } l@{}} 
\lsptoprule
 & ja &  & ji &  & ju &  & jɨ &  & jə &  & jo & \\\midrule
p & appjaganaa & (play.\textsc{sim}) &  &  & appjur & (play.\textsc{umrk}) &  &  &  &  &  & \\
b & jurukubjaganaa & (glad.\textsc{sim}) & & &  asɨbjur & (play.\textsc{umrk}) &  &  &  &  &  & \\
k & kjaaganaa & (come.\textsc{sim}) &  &  & kjuu & ‘today’ & ikjɨ & (go.\textsc{imp}) &  &  & kjoodəə & ‘brother’\\
kˀ & kˀjaa & ‘Kikai-zima’ &  &  & kˀjubii & ‘band’ &  &  &  &  & kˀjoos & ‘break’\\
g & asigja & ‘k.o. sandal’ &  &  & higjussa & ‘cold’ & uigjɨ & (swim.\textsc{imp}) &  &  & uigjoo & (swim.\textsc{int})\\
c & cjaa & ‘tea’ &  &  & cjukaa & ‘kettle’ & kacjɨ & (write.\textsc{seq}) & məəhucjəə & ‘forehead’ & cjoo & ‘just’\\
cˀ & cˀjan & ‘father’ &  &  & cˀju & ‘person’ & cˀjɨ & (come.\textsc{seq}) & cˀjəəra & (come.SEQ.after) & cˀjoo & (person.\textsc{top})\\
s & sja & ‘below’ &  &  & sjuukɨɨ & ‘feast’ & sjɨ & (do.\textsc{seq}) & kasjəə & ‘help’ & isjoobiki & ‘whistle’\\
z & zjaraa & ‘piggyback’ &  &  & zjuu & ‘father’ & izjɨ & (go.\textsc{seq}) & azzjəə & ‘grandfather’ & zjootoo & ‘good’\\
h &  &  &  &  & hjuusɨ & ‘bulbul’ &  &  &  &  &  & \\
m & mjaa & ‘cat’ & mjicja & (see.\textsc{pst})  & mjuuna & (see.\textsc{proh}) & mjɨ & (see.\textsc{imp}) &  &  & mjoo & (see.\textsc{int})\\
n & kəənja & ‘arm’ &  &  & kinju & ‘yesterday’ & njɨɨ & ‘load’ & hannjəə & ‘grandmother’ & anjoo & ‘elder brother’\\
nˀ &  &  &  &  & nˀjuci & ‘life’ & nˀjɨ & ‘rice plant’ & nˀjəə & (rice.plant.\textsc{top}) &  & \\
\lspbottomrule
\end{tabular}
\end{sidewaystable}


\begin{table}
\caption{Combinations of CwV showing allophones\label{bkm:Ref365009143}}
\begin{tabularx}{\textwidth}{lQQQQ}
\lsptoprule
   & wa &  wo &  wɨ & wə\\\midrule
kˀ &  [kˀʷɑ̟] & [kˀʷo̞]   & & [kˀʷɜ]\\
k  &  [kʷɑ̟]      \\
h  &          &      & [φɨ]  \\
\lspbottomrule
\end{tabularx}
\end{table}


\begin{table}
\caption{Examples of CwV}
\begin{tabularx}{\textwidth}{lllQl}
\lsptoprule
& {wa}    & {wo}   & {wɨ}   & {wə}  \\\midrule
{kˀ} & {kˀwa}  {‘child’} & {kˀwoo}  {(child.\textsc{top})}  &    & {kˀwəər}  {‘get fat’}\\
{k} & {joikwa}  {‘silently’}            \\
{h}  &                       &                        & {hwɨɨ}  {‘fart’}    \\
\lspbottomrule
\end{tabularx}
\end{table}

\section{Phonological rules}
\hypertarget{RefHeadingToc395696977}{}\label{bkm:Ref302723494}
Every phonological rule is applied at the morpheme boundaries within phonological words (see §\ref{bkm:Ref347179371}). In this grammar, the following dimensions are distinguished: phonetic, phonological (surface level), and morphophonemic (underlying level). Possible phonetic realization was shown in §\ref{bkm:Ref347181003}, the details of which are beyond the scope of this grammar. Thus, what is called the ‘surface’ level in this grammar represents the phonological level, and the ‘underlying’ level represents the morphophonemic level, against the Bloomfieldians’ convention of merging phonetic and phonological levels (cf. \citealt{Lass1984}: 59-62). The morphophonemic level is abstracted from the information about the morphosyntactic (i.e. paradigmatic and syntagmatic) variation of lexemes. In other words, surface variations of phonemes (i.e. allomorphs) are synthesized into abstract morphophonemes, which are determined by the following criteria: (1) phonemes that are not affected by assimilation, (2) phonemes that are relatively unrestricted by the phonological environments (e.g., the environment before vowels is regarded as “relatively unrestricted” in this grammar), or (3) phonemes that are unmarked cross-linguistically (e.g., oral is more unmarked than nasal, etc.). Needless to say, phonemes at the surface level are considered to contrast with one another, which is different from the variation at the phonetic level.

There are phonological rules and morphophonological rules, both of which are applied within phonological words (see §\ref{bkm:Ref347179371}). The phonological rules are not affected by the surrounding morphosyntactic or lexical information; however, this information is necessary for morphophonological rules; cf., the terms “morphophonological” (\citealt{HaspelmathSims2010}: 214) or “morphophonemic” \citep[23-24]{Payne1997} are used for the alternations that require lexical (and morphosyntactic) information in order to apply the alternation rules. Please note that morphophonological rules precede phonological rules in situations where both rules can apply since morphophonological rules are more specific than phonological rules by definition. Thus, if I encountered a phenomenon which could not be explained by general rules (i.e. phonological rules) already established by other linguistic phenomena, I postulated a special rule (i.e. a morphophonological rule) that would explain the phenomenon and would be applied before the general rule.

  Both of the phonological and morphophonological rules are described as processes, but this does not mean that these processes actually occur in the speaker’s mind. Rather, this style is used because it is easily understandable (cf., \citealt{HaspelmathSims2010}: 211-212).

  In the following subsections, I will present the phonological rules. The first three sections (see §\ref{bkm:Ref304225942}--§\ref{bkm:Ref347173344}) deal with obligatory rules, while the latter two (see §\ref{bkm:Ref347178914}{}-§\ref{bkm:Ref301832441}) deal with rules that are not obligatory but are merely tendencies. The morphophonological rules will be presented in the sections where the relevant morphemes are discussed, e.g., the fusion of the preceding nominal and the topic marker \textit{ja} will be discussed in §\ref{bkm:Ref367134300}.

\subsection{Tap and bilabial approximant deletion}
\label{bkm:Ref304225942}\hypertarget{RefHeadingToc395696978}{}\label{bkm:Ref381399452}
There are no sequences such as /wi/ or /ri/ in Yuwan (except for the three cases discussed later). If this type of sequence occurs at a morpheme boundary, a bilabial approximant //w// or a tap //r// are deleted.

\ea $\left\{\begin{array}{c} \text{w}\\\text{r}\end{array}\right\}$ > ∅ / \_ i 

\ex
\ea \textit{w}{}-deletion\\
     koow\footnote{Strictly speaking, some \textit{w}{}-final verbal roots have \textit{r}{}-final variants (see §\ref{bkm:Ref356245430}), which constitutes free alternation. For example, \textit{koow-} ‘buy’ may be realized as /koor/. If we propose that only the latter could appear before /i/, it is the deletion of //r// (not //w//); however, there is no beneficial reason to propose such a restriction, so we also assume \textit{w}{}-deletion.}   ‘buy’   +  i  (\textsc{inf})  >  koi\footnote{Phonological rule (see §\ref{bkm:Ref301832441}): (koow + i >) kooi > koi.}  (*koowi)\\
\ex \textit{r}{}-deletion\\
      ar   ‘exist’   +   i   (\textsc{inf})   >   ai   (*ari)\\
\z
\z

There are, however, three items in the lexicon that have the sequence of /ri/: \textit{piri} ‘tail end,’ \textit{rikkoo} ‘(by) foot,’ and \textit{kiri} ‘fog.’ The first word is regarded as Standard Japanese by the speaker \textsc{tm}, although the plausible equivalent in Standard Japanese is /biri/. The second word \textit{rikkoo} is considered a recent loan word from modern Japanese because there are no other words with word-initial /r/ in Yuwan. It is not clear whether the last word, \textit{kiri} ‘fog,’ existed originally in Yuwan, or was borrowed from Standard Japanese.

\subsection{Alveolar stop affrication (or palatalization)}
\label{bkm:Ref347180796}\hypertarget{RefHeadingToc395696979}{}
The alveolar stop //t// becomes /c/ if it precedes //i// or //j//, which may be called “palatalization” in the broader sense. The reason why we do not assume the combination of /ti/ [t͡ɕi] is argued in §\ref{bkm:Ref347181344}.

\ea t   >   c  /  \_ $\left\{\begin{array}{c} \text{i}\\\text{j}\end{array}\right\}$ 

\ex
\ea Before //i//\\
    ut   ‘hit’   +   i   (\textsc{inf})   >   uci\\
\ex Before //j//\\
    ut   ‘hit’   +   jaa   ‘person’   >   ucjaa\\
\z
\z

\subsection{Epenthetic vowel /u/}
\label{bkm:Ref301838720}\hypertarget{RefHeadingToc395696980}{}\label{bkm:Ref347173344}
A syllable should have a nucleus filled by a vowel (see §\ref{bkm:Ref301830963}), so if a syllable does not satisfy this condition at morpheme boundaries, an epenthetic vowel /u/ is inserted at the morpheme boundaries and serves as a nucleus.

\ea  ∅   >   u   /   \#\footnote{‘\#’ indicates a syllable boundary.}   \_   C\#
\ex
\ea  mun   ‘thing’   +   n   ‘also’   >   mu.nun   (*mun.n or *mun.nu)\\
\ex  +   nkja   (\textsc{appr})   >   mu.nun.kja   (*mun.nkja or *mun.nu.kja)\\
\ex  +   kkwa   (\textsc{dim})   >   mu.nuk.kwa   (*mun.kkwa or *mun.ku.kwa)\\
\z
\z

Further, there are no sequences of C\textsubscript{coda}.V in Yuwan. If such a sequence occurs around a morpheme boundary, an epenthetic vowel /u/ is inserted at the morpheme boundary.

\ea   ∅   >   u   /   C\#   \_   V
\ex  \gll tankan   ‘k.o. orange’   +   i   (\textsc{plq})   >   tan.ka.nui [tɑ̟ŋ.kɑ̟.nui]   (*tan.kan.i [tɑ̟ŋ.kɑ̟ɴ.i])\\
           {}         {}           {}  {}  {}      {}    {}          {}          (*tan.ka.ni [tɑ̟ŋ.kɑ̟.ni])\\
\z

These examples show that the forbidden sequence /n.i/ [ɴ.i] is not realized and /nui/ appears instead. Interestingly, a simple combination of /ni/ [ni] does not appear, which may imply that the epenthetic vowel /u/ is inserted not only to stabilize the syllable construction but also to leave a trace of the previous morpheme boundary.

\subsection{Geminate devoicing}
\label{bkm:Ref347178914}\hypertarget{RefHeadingToc395696981}{}
Almost all of the geminates within monomorphemic words in Yuwan are voiceless (see \ref{ex:2.8d} in §\ref{bkm:Ref302599510}). Moreover, if a voiced geminate occurs at a morpheme boundary, it tends to be voiceless.

\begin{exe}
\ex \gll C\textit{\textsubscript{i}}  C\textit{\textsubscript{i}}  >  C\textit{\textsubscript{i}}  C\textit{\textsubscript{i}}\footnotemark\\
 {[+v]}  [+v] {}   [-v]  [-v]\\
\ex\label{ex:2.20}
\begin{xlist}
\ex bb > pp\\
    ar   ‘exist’   +   ba   (\textsc{csl})   >   appa\footnote{Morphophonological rule (see §\ref{bkm:Ref347177096}): ar +ba > abba (> appa)}
\ex dd > tt\\
    ar   ‘exist’   +   doo   (\textsc{ass})   >   attoo\footnote{Morphophonological rule (see §\ref{bkm:Ref347177096}): ar +doo > addoo (> attoo)}
\ex gg > kk\\
    ar   ‘exist’   +   ga   (\textsc{cfm}3)   >   akka\footnote{Morphophonological rule (see §\ref{bkm:Ref347177096}): ar +ga > agga (> akka)}\\
\end{xlist}
\end{exe}
\footnotetext{The small italic \textit{i} means they have the same articulatory place and manner. Supplemental information is provided in square brackets under the rule schema.}

\subsection{Vowel deletion}
\label{bkm:Ref301832441}\hypertarget{RefHeadingToc395696982}{}
A monomorphemic word has a sequence with at most two vowels (see \ref{ex:2.8e} in §\ref{bkm:Ref302599510}) and it does not have a V\textit{\textsubscript{i}}V\textit{\textsubscript{i}}C\textsubscript{coda} sequence (see \ref{ex:2.8f} in §\ref{bkm:Ref302599510}). If this sequence occurs around a morpheme boundary, one of the preceding vowels tends to be deleted.

\ea  {V\textit{\textsubscript{i}} V\textit{\textsubscript{i}}}  {>}  $\left\{\begin{array}{c} \text{V\textit{\textsubscript{i}}} \\ C \end{array}  \right\}$  {/}  {\_} V    {\#}
\ex
\ea Before a vowel\\
   {koow}  {‘buy’}  {+}  {i}  {(\textsc{inf})}  {>}  {koi\footnote{Phonological rule (see §\ref{bkm:Ref304225942}): koow + i > kooi (> koi)}}\\
\ex Before a consonant\\
   \gll {attaa}  {‘they’}  {+}  {n}    {‘also’}   {>}  {attan}\\
        {}         {}      {+}  {nkja}  {(\textsc{appr})}  {>}  {attankja}\\
\z
\z

Interestingly, though three-vowel sequences tend to be avoided at morpheme boundaries, four-vowel sequences are not. (If we suppose that a syllable dislikes having three morae cosidering \textsc{ref}{ex:2.20}, the acceptability of /kooii/ may mean the existence of a syllable boundary, such as /koo.ii/.) See the example below; for convenience, the surface form is shown from the beginning in this example (see §\ref{bkm:Ref364174852} for the lengthened form of the infinitive).

\ea koow  ‘buy’  +  ii  (\textsc{inf})  >  kooii\footnote{Phonological rule (see §\ref{bkm:Ref304225942}): koow + ii > kooii} \z

Yuwan has few lexemes where the vowel /o/ is short (see the note “\ref{bkm:Ref347176670}.” of \tabref{tab:key:4}), and when /o/ appears, its syllable is frequently heavy, i.e., it is /oi/, /oo/ or /oC\textsubscript{coda}/. Otherwise, these lexemes are onomatopoeia such as \textit{botto+botto} ‘lazily,’ interjections such as \textit{ido} ‘hey,’ or seem to be relatively modern loan words from standard Japanese such as \textit{itoko} ‘cousin.’ Those facts may indicate that the /o/ that is short in surface level is long, i.e. /oo/, in underlying level, and that the underlying /oo/ becomes /o/ by the vowel deletion rule in \textsc{ref}{ex:2.20}. The same argument can be applied to /ə/.

\section{Prosody}\label{bkm:Ref301560567}\hypertarget{RefHeadingToc395696983}{}
\subsection{Three pitch patterns}\label{bkm:Ref303982713}\hypertarget{RefHeadingToc395696984}{}

There is lexical prosody in Yuwan. That is, each root has its own prosodic pattern, and these patterns fall into three types.

\begin{enumerate}[label=\Roman*.]
\item Falling after the penultimate mora of a phonological word
\item Falling after the syllable including the second mora of a phonological word
\item Rising at the final mora of a phonological word
\end{enumerate}

(If the falling position is located word-finally, then falling is realized after the penultimate mora.)

In Tables~\ref{tab:2:20}--\ref{tab:2:22}, both “H” (high pitch) and “L” (low pitch) are counted as a mora respectively.

\begin{table}
\caption{Pitch patterns in Yuwan\label{tab:2:20}}
\begin{tabular}{ l >{\itshape}l *{5}{l} }  
\lsptoprule
&  \normalfont Form & Gloss & \multicolumn{4}{c}{Pitch pattern}\\\cmidrule(lr){4-7} 
     &                     & & Isolation &  x=\textit{nu} &  x=\textit{n} &  x=\textit{gadɨ}\\
     &                     & &          &   (\textsc{nom}) &  ‘also’ &  (\textsc{lmt})\\\midrule
I & \itshape haa &  ‘leaf’ &  HL &  HHL &  HL\footnote{(Optional) phonological rule (see §\ref{bkm:Ref301832441}): haa + n > han} &  HHHL\\
  & judai &  ‘saliva’ &  HHL &  HHHL &  HHHL &  HHHHL\\
II & \itshape haa &  ‘teeth’ &  HL &  HHL &  HL &  HHLL\\
  & sɨkama &  ‘morning’ &  HHL &  HHLL &  HHLL &  HHLLL\\
  & məərabɨ &  ‘lady’ &  HHLL &  HHLLL &  HHLLL &  HHLLLL\\
  & hizjai &  ‘left’ &  HHL &  HHHL &  HHLL &  HHHLL\\
III & \itshape naa &  ‘inside’ &  LH &  LLH &  LLH &  LLLH\\
  & nabɨ &  ‘pan’ &  LH &  LLH &  LLH &  LLLH\\
  & usagi &  ‘rabbit’ &  LLH &  LLLH &  LLLH &  LLLLH\\
\lspbottomrule
\end{tabular}
\end{table}

\tabref{tab:2:20} shows that in order to determine the type II pitch pattern, it is necessary to count both syllables and morae.

Most of the lexicon belonging to type II is realized with falling after the second mora, such as /sɨ.ka.ma.nu/ \textit{sɨkama=nu} (morning=\textsc{nom}) produced as HHLL and /məə.ra.bɨ.nu/ \textit{məərabɨ=nu} (lady=NOM) produced as HHLLL. However, if the second syllable contains a vowel sequence, the falling occurs after the third mora, such as /hi.zjai.nu/ \textit{hizjai=nu} (left=NOM) produced as HHHL, which means type II represents falling not after the second mora, but after the second syllable including the second mora. Furthermore, if you only allow that “type II represents falling after the second syllable,” you cannot explain why /məə.ra.bɨ.nu/ \textit{məərabɨ=nu} (lady=NOM) is produced as HHLLL.

The prosodic behavior discussed above helps us think about the long vowels and dipthongs in Yuwan. In short, we cannot assume a long vowel phoneme, such as /aː/, or a diphthong phoneme, such as /a\textsuperscript{i}/, because we presuppose the following three points:

\begin{enumerate}[label=\alph*.]
\item A mora is assigned not to a phoneme but to a slot;
\item A slot may have maximally one mora;
\item One phoneme can fill only one slot.
\end{enumerate}

(Note: ‘slot’ in the above means C, G, or V in a syllable. See \sectref{sec:key:2.3.1} for more details.)

That is, we do not propose that one slot has two morae, that one phoneme has two morae, or that one phoneme can fill two moraic slots in a syllable. From the point of view of prosody, long vowels and diphthongs in Yuwan have two morae, so we do not assume a long vowel phoneme, such as /aː/, or a diphthong phoneme, such as /a\textsuperscript{i}/. A similar problem was discussed in \citet[196-199]{Dixon2010} where “in Fijian - a mora-counting language - a long vowel can be usefully regarded as a sequence of two short vowels.”

\subsection{Some notes on initial glottalized consonants}
\label{bkm:Ref347180628}\hypertarget{RefHeadingToc395696985}{}
In Yuwan, there seems to be irregular pitch patterns if the initial consonant of words is glottalized.

\begin{table}
\caption{Pitch patterns of words beginning with a glottalized consonant (part 1)}
\begin{tabular}{>{\itshape}llllll}
\lsptoprule
\normalfont Form & Gloss & \multicolumn{4}{c}{Pitch pattern}\\\cmidrule(lr){3-6}
     &       & Isolation & x=\textit{nu} & x=\textit{n} & x=\textit{gadɨ}\\
     &       &           & (\textsc{nom}) & ‘also’ & (\textsc{lmt})\\\midrule
nˀjɨ    &  ‘rice plant’ &  H   & HL   & HL   &  HLL\\
mˀa     &  ‘horse’       &  H   & HL   & HL   &  HLL\\
nˀjuti  &  ‘life’     &  HL  & HLL  & HLL  &  HLLL\\
mˀacɨ   &  ‘fire’      &  HL  & HLL  & HLL  &  HLLL\\
kˀwagɨ  &  ‘mulberry’ &  HL  & HLL  & HLL  &  HLLL\\
kˀjubii &  ‘belt’    &  HLL & HLLL & HLLL &  HLLLL\\
\lspbottomrule
\end{tabular}
\end{table}

In these words, falling seems to occur after the first mora, and such a pitch pattern is not found elsewhere (see §\ref{bkm:Ref303982713}). There are two possible analyses to explain this finding:

\begin{description}
\item[Analysis 1:] Glottalized phonemes have one mora by themselves.
\item[Analysis 2:] Glottalized resonants or glottalized stops with approximants create a subcategory of pitch patterns.
\end{description}

Analysis 1, however, immediately turns out to be false, because there is a case where a glottalized phoneme does not seem to have one mora.

\begin{table}
\caption{Pitch patterns of words beginning with a glottalized consonant (part 2)\label{tab:key:22}}
\begin{tabular}{>{\itshape}llllll}
\lsptoprule
\normalfont Form & Gloss & \multicolumn{4}{c}{Pitch pattern}\\\cmidrule(lr){3-6}
     &       & Isolation & x=\textit{nu} & x=\textit{n} & x=\textit{gadɨ}\\
     &       &           & (\textsc{nom}) & ‘also’ & (\textsc{lmt})\\\midrule
kˀura &  ‘storehouse’ &  HL &  HHL &  HHL &  HHLL\\
\lspbottomrule
\end{tabular}
\end{table}

\tabref{tab:key:22} shows that glottalized /kˀ/ does not have a mora because the falling is realized not after /kˀu/ but after /ra/ (when it precedes clitics). In other words, it behaves regularly as the type II pitch pattern (see §\ref{bkm:Ref303982713}). Since we cannot regard the glottalized consonant /kˀ/ as having one mora, Analysis 1 cannot be accepted.

Analysis 2 assumes that the type II pitch pattern has two subcategories:

\begin{description}
\item[Subcategory I:] If initial consonants are glottalized resonants such as /nˀ/, or the glottalized velar stop /kˀ/ plus an approximant such as /kˀw/ or /kˀj/, then the falling occurs after the initial mora.
\item[Subcategory II:] Otherwise, the falling occurs after the syllable including the second mora.
\end{description}

These subcategories can be explained by phonotactics, which means their differences need not be assigned to the lexicon. Following these points, we will take up Analysis 2. Additionally, many of the glottalized consonants were the result of syllable omission (see §\ref{bkm:Ref347180773}). Therefore, the retaining of a mora by a glottal phoneme can also be explained from a historical perspective.

\subsection{Further research}
\label{bkm:Ref347175621}\hypertarget{RefHeadingToc395696986}{}\label{bkm:Ref381418627}
In the previous section, we discussed the prosody of nominals in Yuwan; however, the data set is very limited. In fact, we only dealt with 207 words. The breakdown of the pitch patterns of these words are shown in \tabref{tab:2:33}.

\begin{table}
\caption{Breakdown of pitch patterns of nominals\label{tab:2:33}}
\begin{tabular}{lS[table-format=3]S[table-format=2]}
\lsptoprule
Pattern & {Number of words} & {\%}\\\midrule
I     & 99  & 48\\
II    & 56  & 27\\
III   & 52  & 25\\\midrule
Total & 207 & 100\\
\lspbottomrule
\end{tabular}
\end{table}

It is important to note that there are many cases where the falling or rising of the three accent patterns is not realized. In other words, there are many cases where a phonological word keeps a flat pitch throughout, and this makes it difficult to fully know the accurate pitch patterns of words in Yuwan. In the above data, we excluded these data and only focused on words that have pitch movement; however, we need to clarify this omission for future research.

Although research into the prosody of Yuwan is not yet sufficient, our current data and analysis make it possible to propose the following points. First, we propose that verbs and adjectives seem to have the same pitch patterns as nominals, although the details of their proportions are different. Second, compounds seem to retain the pitch patterns of the preceding stem. Third, the most recent loan words (from English loan words in Standard Japanese) tend to have the type I pitch pattern.



\chapter{Inter-clausal phenomena}

This chapter describes several inter-clausal phenomena. In \sectref{sec:11.1}, we will discuss the subordinate clauses, which can modify another clause. There are four types in the subordinate clauses: adverbial clause (where the subordinate clause functions as an adverb) (see \sectref{sec:11.1.1}); adnominal clause (where the subordinate clause functions as an adnominal) (see \sectref{sec:11.1.2}); nominal clause (where the subordinate clause functions as a nominal) (see \sectref{sec:11.1.3}); and complement clause (where the subordinate clause fills the complement slot of the verbal predicate phrase) (see \sectref{sec:11.1.4}). Some of the subordinate clauses can be used without their superordinate clauses. The conventionalized omission of the superordinate clause is called “insubordination” \citep{Evans2007}, which will be discussed in \sectref{sec:11.2}. In \sectref{sec:11.3}, I will present the phenomena that are related with the focus markers, especially the phenomenon called “kakari-musubi” (i.e. ‘government-predication’) in Japanese and Ryukyuan linguistics.

\section{Subordinate clauses}\label{sec:11.1}

Yuwan has four types of subordinate clauses: adverbial clauses (see \sectref{sec:11.1.1}); adnominal clauses (see \sectref{sec:11.1.2}); nominal clauses (see \sectref{sec:11.1.3}); and complement clauses (see \sectref{sec:11.1.4}). The dependency of the subordinate clauses on the superordinate clause is different from one to another. Many of the subordinate clauses can take their own subjects different from those in the superordinate clauses. However, the adverbial clauses headed by the converbs \textit{{}-tai} (\textsc{lst}) and \textit{{}-jagacinaa} (\textsc{sim}) and the nominal clauses headed by the infinitives (not accompanied with \textit{n} (\textsc{dat}1)) cannot take their own subjects (see \sectref{sec:key:8.4.3} and \sectref{sec:key:8.4.4.2} for more details).

\subsection{Adverbial clause}\label{sec:11.1.1}

The adverbial clause is the subordinate clause that functions as an adverb. The adverbial clause precedes its superordinate claue in principle. The adverbial clause can be expressed in two ways. First, the adverbial clause can be expressed by the converbal affixes. For example, \textit{{}-ba} (\textsc{csl}) following the verbal stem can express a causal meaning as in (11-1 a) (see \sectref{sec:key:8.4.3} for more details). Secondly, the adverbial clause can also be expressed by the conjunctive particles as in (11-1 b) (see \sectref{sec:key:10.2} for more details).

\ea\label{ex:11-1}  Adverbial clauses in Yuwan
  \ea Using a converb [= (8-86 a)] [Context: \textsc{my} asked \textsc{tm} if TM had made the pickles; TM: ‘(I) don’t know. How (was it)?’]\\
%   \textsc{tm}
  \glll nɨɨzinnu  appa,  arandaroo.\\
    [\textit{nɨɨzin=nu}  \textit{ar-\Highlight{ba}}]\textsubscript{Adverbial clause}  \textit{ar-an=daroo}\\
    carrot=\textsc{nom}  exist-\textsc{csl}  \textsc{cop}-\textsc{neg}=\textsc{supp}\\
    \glt ‘There are (pieces of) a carrot, so maybe (the pickles) are not (mine).’ [Co: 101023\_01.txt]

  \ex Using a conjunctive particle [= (4-20 b)]\\
%   \textsc{tm}
   \glll wanna  honami-{\textbar}cjan{\textbar}  naaja  siccjunban, naakjaa  jumɨnu  naaja  sijandoojaa.\\
    [\textit{wan=ja}  \textit{honami-cjan}  \textit{naa=ja}  \textit{sij-tur-n=\Highlight{ban}}]\textsubscript{Adverbial clause} \textit{naakjaa}  \textit{jumɨ=nu}  \textit{naa=ja}  \textit{sij-an=doo=jaa}\\
    1\textsc{sg}=\textsc{top}  Honami-\textsc{dim}  name=TOP  know-\textsc{prog}-\textsc{ptcp}=\textsc{advrs} 2\textsc{pl}.\textsc{hon}.\textsc{adn}Z  daughter.in.law=\textsc{gen}  name=TOP  know-\textsc{neg}=\textsc{ass}=\textsc{sol}\\
    \glt ‘I know Honami’s name, but don’t know the name of your daughter in law.’ [Co: 110328\_00.txt]
\z
\z

All of the converbal affixes and some of the conjunctive particles are restricted in their choice of tense markers. However, a few conjunctive particles, i.e. \textit{ban} (\textsc{advrs}), \textit{kara} (\textsc{csl}) and \textit{mun} (ADVRS), are not restricted in their choice of tense markers.

It is common in Yuwan that the adverbial clauses (especially including \textit{{}-tɨ} (\textsc{seq})) are used sequentially, which is called clause-chaining (cf. \citealt{Payne1997}: 321-325). In that case, the adverbial clauses do not seem to be embedded in the superordinate clauses as adverbs, and it is natural to translate the meanings of the relations among the clauses into ‘and then’ as in \textsc{ref}{ex:11-2}.

\ea\label{ex:11-2}  Clause-chaining in Yuwan [= (8-102 b)]\\
%   \textsc{tm}
    \glll idocjɨ  jˀicjɨ,  (an)  mata  (an)  agan izjibatɨ  izjɨ,  amanan  sawakotankja minakotankjaga  wutattu,\\
    [\textit{ido=ccjɨ}  \textit{jˀ-\Highlight{tɨ}}]\textsubscript{Adverbial clause}  \textit{a-n}  \textit{mata}  \textit{a-n}  [\textit{aga-n} \textit{izir-i+bar-tɨ}  \textit{ik-\Highlight{tɨ}}]\textsubscript{Adverbial clause}\textbf{  }[\textit{a-ma=nan}  \textit{sawako-taa=nkja}  \textit{minako-taa=nkja=ga}  \textit{wur-tar-tu}]\textsubscript{Adverbial clause}\\
    oh=\textsc{qt}  say-\textsc{seq}  \textsc{dist}-\textsc{adn}Z  again  DI\textsc{st}-\textsc{adnz}  DIST-\textsc{advz} go.out-\textsc{inf}+?-SEQ  go-SEQ  DIST-place=\textsc{loc}1  Sawako-\textsc{pl}=\textsc{appr} Minako-PL=\textsc{app}R=\textsc{nom}  exist-\textsc{pst}-\textsc{csl}\\
    \glt ‘Saying that “Oh!” (I) went out there again, and there were Sawako, Minako and their friends, so ...’ [Co: 101020\_01.txt]
\z

Interestingly, some clauses headed by converbs can be used without their superordinate clauses. The conventionalized omission of the superordinate clauses is called “insubordination” (see \sectref{sec:11.2} for more details).

\subsection{Adnominal clause}\label{sec:11.1.2}

The adnominal clause is the subordinate clause that functions as an adnominal. The adnominal clause always precedes its head nominal. The predicate of the adnominal clause is always filled by the participles that end with \textit{{}-n} (\textsc{ptcp}) as in (11-3 a) or \textit{{}-an} (\textsc{neg}) as in (11-3 b) (see \sectref{sec:key:8.4.2} for more details), but not vice versa since the participle followed by the conjunctive particles function as the adverbial clauses as in (11-1 b) in \sectref{sec:11.1.1} (see also \sectref{sec:key:10.2}).

\ea\label{ex:11-3}  Adnominal clauses in Yuwan
  \ea Using the participial affix \textit{{}-n} (\textsc{ptcp}) [= (8-80 a)]\\
%   \textsc{tm}
   \glll sakkiija  (hinzjaa)  xxx  hinzjaaba  succjun cˀjunu  atooradu  cˀjanmun.\\
    \textit{sakkii=ja}  \textit{hinzjaa}    [\textit{hinzjaa=ba}  \textit{sukk-tur-{n}}]\textsubscript{Adnominal clause}   \textit{cˀju=nu}  \textit{atu=kara=du}  \textit{k-tar-n=mun}\\
    a.short.while.ago  goat    goat=\textsc{acc}  pull-\textsc{prog}-\textsc{ptcp}  person=\textsc{nom}  after=\textsc{abl}=\textsc{foc}  come-\textsc{pst}-PTCP=\textsc{advrs}\\
   \glt ‘A short while ago, the person who was pulling a goat came afterward, but (this time he came beforehand).’    [\textsc{pf}: 090827\_02.txt]
  \ex Using the participial affix \textit{{}-an} (\textsc{neg}) [= (8-83 b)]\\
%   \textsc{tm}
   \glll kˀwaga  dɨkɨran  cˀju  natɨ,\\
    [\textit{kˀwa=ga}  \textit{dɨkɨr-\Highlight{an}}]\textsubscript{Adnominal clause}  \textit{cˀju}  \textit{nar-tɨ}\\
    child=\textsc{nom}  be.born-\textsc{neg}  person  \textsc{cop}-\textsc{seq}\\
    ‘Since (the woman) was a person who cannot have a baby, ...’    [Co: 120415\_00.txt]
\z
\z

If the constituent of a clause is focused by \textit{du} (\textsc{foc}), the predicate-final verb may take the participle without the following head NP, which is called the focus construction (or “kakari-musubi”) (see \sectref{sec:11.3} for more details).

\subsection{Nominal clause}\label{sec:11.1.3}

The nominal clause is the subordinate clause that functions as a nominal. The nominal clause can be expressed in three ways. First, the nominal clause can be expressed by the compound. For example, \textit{mai} (\textsc{obl}) is compounded with the preceding verbal stem: /ikimai/ \textit{ik-i+mai} (go-\textsc{inf}+OBL) ‘to have to go’ (see \sectref{sec:key:4.2.3.2} for more details) as in (11-4 a). Secondly, the nominal clause can be expressed by the infinitival affix \textit{{}-i}/\textit{{}-∅} as in (11-4 b) (see \sectref{sec:key:8.4.4.2} for more details). Thirdly, the nominal clause can be expressed by the formal noun \textit{sɨ}, which can directly follow the bound verbal stem and forms a nominal clause as in (11-4 c) (see \sectref{sec:key:6.2.2.1} for more details).

\ea\label{ex:11-4}  Nominal clauses in Yuwan

  \ea Using a nominal compound [= (4-35 d)]
  
      %\textsc{tm}
      \glll    wanna  urɨba  kakimaidoo.\\
    \textit{wan=ja}  [\textit{u-rɨ=ba}  \textit{kak-i+mai}]\textsubscript{Nominal clause}\textit{=doo}\\
    1\textsc{sg}=\textsc{top}  \textsc{mes}-\textsc{nlz}=\textsc{acc}  write-\textsc{inf}+\textsc{obl}=\textsc{ass}\\
    \glt     ‘I have to write it.’ [El: 130816]

  \ex Using an infinitive [= (8-113 a)]  [Context: Remembering the days when people send off the people who went to mainland Japan]\\
  
      %\textsc{tm}
      \glll    umanan  sanbasinu  atɨ,  umantɨ  cɨkɨ  jatattu.\\
    \textit{u-ma=nan}  \textit{sanbasi=nu}  \textit{ar-tɨ}    [\textit{u-ma=nantɨ}  \textit{\Highlight{cɨkɨr-∅}}]\textsubscript{Nominal clause}  \textit{jar-tar-tu}\\
    \textsc{mes}-place=\textsc{loc}1  pier=\textsc{nom}  exist-\textsc{seq}  MES-place=LOC2  attach-\textsc{inf}  \textsc{cop}-\textsc{pst}-\textsc{csl}\\
    \glt  ‘There is a pier there, and (the ship) came alongside there [lit. (the ship) was to dock there].’    [Co: 120415\_00.txt]

  \ex Using the formal noun \textit{sɨ}   [Context: Talking about the present author] = (6-13 a)\\
  
      %\textsc{tm}
      \glll    an  nɨsəə  muccjɨ  ikjusəə  nun   nənba,  jakkəə.\\
    [\textit{a-n}  \textit{nəɨsəə}  \textit{mut-tɨ}  \textit{\Highlight{ik-jur=sɨ}}]\textsubscript{Nominal clause} \textit{=ja}  \textit{nuu=n}   \textit{nə-an-ba}  \textit{jakkəə}\\
    \textsc{dist}-\textsc{adn}Z  young.man  have-\textsc{seq}  go-\textsc{umrk}=\textsc{fn}=\textsc{top}  what=any  exist-\textsc{neg}-\textsc{csl}  trouble\\
    \glt    ‘There is not anything [i.e. any food] the young man can take (for meals), so it’s pity.’    [Co: 101023\_01.txt]
\z
\z

All of the above strategies can make the nominal clause, but the degree of the nominal characteristic and the verbal characteristic (or “clause-hood”) is different from one another. Their differences are summrized in the following \tabref{tab:102}.

\begin{table}
\caption{Comparison among the clauses headed by \textit{mai} (\textsc{obl}), -\textit{i}/-\textit{∅} (\textsc{inf}), or \textit{sɨ} (\textsc{fn}). Note: (+) means that there are a few cases where \textit{-i}\slash\textit{-∅} (INF) can satisfy the nominal\slash verbal characteristics.\label{bkm:Ref365475191}\label{tab:102}}
\begin{tabularx}{\textwidth}{lQccc}
\lsptoprule
\multicolumn{2}{l}{Nominal characteristics} &  \\\midrule
a. & May be follwed by the copula verbs & + & + & +\\
b. & May be followed by case particles & \textminus & (+) & +\\\midrule
\multicolumn{2}{l}{Verbal characteristics (or “clause-hood”)} &   \\\midrule
c. & Retains the internal syntax & + & + & +\\
d. & May take the subject different from that of the superordinate clause & \textminus & (+) & +\\
\lspbottomrule
\end{tabularx}
\end{table}

About the nominal characteristics in \tabref{tab:102}, all of the nominal clauses headed by (the compound including) \textit{mai} (\textsc{obl}), the infinitive, and \textit{sɨ} (\textsc{fn}) may be followed by the copula verbs. In this respect, they behave like nominals. However, the compound including \textit{mai} (OBL) cannot take any case particle. In other words, it cannot become an argument. Similarly, the infinitive cannot take any case particles with the exception of the nominative case \textit{ga} and the dative case 1 \textit{n} (see \sectref{sec:key:8.4.4.2} for more details). On the contrary, \textit{sɨ} (FN) has more freedom to take case particle than the others. Thus, the clause headed by \textit{sɨ} (FN) has more nominal characteristics than those headed by \textit{mai} (OBL) or \textit{{}-i}/\textit{{}-∅} (\textsc{inf}). About the verbal characteristics in \tabref{tab:102}, all of the verbal stems that are followed by \textit{mai} (OBL), \textit{{}-i}/\textit{{}-∅} (INF), and \textit{sɨ} (FN) may retain their internal syntax. In this respect, these words behave like verbs. However, the clause headed by (the compound including) \textit{mai} (OBL) cannot have its own subject different from the superordinate (i.e. modified) clause. The clause headed by the infinitive also cannot take its own subject with the exception of the case where the infitive takes \textit{n} (\textsc{dat}1) as in (8-114) - (8-115) in \sectref{sec:key:8.4.4.2}. On the contrary, the clause headed by \textit{sɨ} (FN) can take its own subject different from the superordinate clause. Thus, the clause headed by \textit{sɨ} (FN) has more verbal characteristics (or “clause-hood”) than those headed by \textit{mai} (OBL) or \textit{{}-i}/\textit{{}-∅} (INF). From another point of view, it is probable that the clause headed by \textit{sɨ} (FN) has the status sufficient to be called the nominal clause, but that the clauses headed by (the compound that includes) \textit{mai} (OBL) or the infinitives are better analyzed as the components of the complex predicate (with the copula verb in a single clause).

\subsection{Complement clause}\label{sec:11.1.4}

The complement clause in Yuwan is the subordinate clause that functions as a complement of the verbal predicate phrase (see \sectref{sec:key:9.1} about the complement slot). A complement clause ends with one of the utterance-final particles A, i.e. \textit{ccjɨ} (\textsc{qt}), \textit{ka} (\textsc{du}B), \textit{gajaaroo} (\textsc{dub}), and \textit{nən} ‘such as.’ I present an example of \textit{ccjɨ} in \textsc{ref}{ex:11-5} (see \sectref{sec:key:10.4} for more details).

\ea\label{ex:11-5}  Complement clause in Yuwan [= (10-63 c)]
  
      %\textsc{tm}
      \glll    isaburootaa,  tomokkotaaga  atai  jatancjɨ   jˀicjɨ,\\
    [\textit{isaburoo-taa}  \textit{tomokko-taa=ga}  \textit{atai}  \textit{jar-tar-n=\Highlight{ccjɨ}}]\textsubscript{Complement clause}    \textit{\Highlight{jˀ}-tɨ}\\
    Isaburo-\textsc{pl}  Tomohiko-PL=\textsc{nom}  50.years.old  \textsc{cop}-\textsc{pst}-\textsc{ptcp}=\textsc{qt}  say-\textsc{seq}\\
    \glt ‘(People) said that Isaburo (and) Tomohiko were fifty years old, and ...’    [Co: 120415\_01.txt]
\z

Other examples of complement clauses were shown in (9-23 b-e) in \sectref{sec:key:9.1.2.1} and (9-39) in \sectref{sec:key:9.1.2.2}.

In fact, the clause followed by \textit{ccjɨ} (\textsc{qt}) is similar to the nominal clause (in \sectref{sec:11.1.3}), since it may be followed by the copula verb, may take the genitive case \textit{nu}, and can retain the internal syntax including its own subject (see \sectref{sec:key:10.4.1.5} for more details). However, I propose that the clause followed by \textit{ccjɨ} (QT) is different from the nominal clause since it does not take any argument case (i.e. the cases other than the genitive). In fact, the clause headed by (the compound including) \textit{mai} (\textsc{obl}) does not take any argument case as well as the clause followed by \textit{ccjɨ} (QT). However, the former, i.e. the clause headed by \textit{mai} (OBL), only fills the predicate phrase of the superordinate clause, but the latter, i.e. the clause followed by \textit{ccjɨ} (QT), can (and frequently) fill the slot other than the head of the predicate phrase of the superordinate clasue. In other words, the clause followed by \textit{ccjɨ} (QT) fills the complement slot of the verbal predicate phrase. The components in the complement slot do not take any argument case since they are not the arguments of the clause (see \sectref{sec:key:9.1}). Thus, it is more appropriate to call the clause followed by \textit{ccjɨ} (QT) the “complement clause” (not the nominal clause).

\section{Insubordination}\label{sec:11.2}

Insubordination is defined by \citet[367]{Evans2007} as follows: “I will apply the term “insubordination” to \textit{the} \textit{conventionalized} \textit{main} \textit{clause} \textit{use} \textit{of} \textit{what,} \textit{on} \textit{prima} \textit{facie} \textit{grounds,} \textit{appear} \textit{to} \textit{be} \textit{formally} \textit{subordinate} \textit{clauses}” (italic in original). As \citet[367]{Evans2007} said, the insubordination is a phenomenon strongly related with the diachronic linguistic change. Therefore, it is probable that there is a case where the subordinate use is very rare and also the main-clause use dominats in the modern language. In fact, the affix \textit{{}-ɨba} (\textsc{sugs}) in Yuwan is a good candidate for that (see \sectref{sec:key:8.4.1.5} for more details). In Yuwan, the omission of the main clause is very common, where the (meaning of the) omitted clause can be often restored by the context. However, there are a few cases where the restoring is difficult. In those cases, the predicates have gained some grammatical functions different from the functions in the original subordinate clauses. In the following sections, I will present four examples: \textit{-tɨ} (\textsc{seq}) in \sectref{sec:11.2.1}, \textit{{}-ba} (\textsc{csl}) in \sectref{sec:11.2.2}, \textit{ccjɨ=joo} (\textsc{qt}=\textsc{cfm}1) in \sectref{sec:11.2.3}, and \textit{{}-an-boo} (\textsc{neg}-\textsc{cnd}) in \sectref{sec:11.2.4}.

\subsection{\textit{{}-tɨ} (\textsc{seq}) as insubordination}\label{sec:11.2.1}

Non-finite uses of the converbal affix \textit{-tɨ} (\textsc{seq}) are found in the adverbial clause expressing sequential meaning as in \sectref{sec:key:8.4.3.5} or in the auxiliary verb construction as in \sectref{sec:key:9.1.1}. However, there is a finite use of the converbal affix \textit{-tɨ} (SEQ), which expresses the past tense as in (11-6 a-c).

\ea\label{ex:11-6}  \textit{{}-tɨ} (\textsc{seq}) expressing the past tense as the insubordination

  \ea  
      %\textsc{tm}
      \glll    naakjoo  injasainnja  dantɨ  asɨbjutɨ?\\    
      \textit{naakja=ja}  \textit{inja-sa+ar-i=n=ja}  \textit{daa=nantɨ} \textit{asɨb-jur-\Highlight{tɨ}}\\   
      2.\textsc{hon}.\textsc{pl}=\textsc{top}  small-\textsc{adj}+\textsc{st}V-\textsc{inf}=\textsc{dat}1=TOP  where=\textsc{loc}2   play-\textsc{umrk}-\textsc{seq}\\
      \glt ‘Where did you used to play when (you) were in your childhood?’       [Co: 110328\_00.txt]

  \ex  
      %\textsc{tm}
      \glll    gazimarugɨɨnu  sjantɨ  asɨbantɨ?\\
      \textit{gazimaru+kɨɨ=nu}  \textit{sja=nantɨ}  \textit{asɨb-an-\Highlight{tɨ}}\\
      bayan.tree+tree=\textsc{gen}  under=\textsc{loc}2  play-\textsc{neg}-\textsc{seq}\\
      \glt       ‘Didn’t you play under the banyan tree?’ [Co: 110328\_00.txt]

  \ex  
      %\textsc{tm}
      \glll    jadunkjoo  akɨtɨdoo.\\
      \textit{jaduu=nkja=ja}  \textit{akɨr-\Highlight{tɨ}=doo}\\
      door=\textsc{appr}=\textsc{top}  open-\textsc{seq}=\textsc{ass}\\
      \glt       ‘(We) opened the doors (on New Year’s Eve in the old days).’ [Co: 111113\_02.txt]
\z
\z

  In fact, the finite-form affix \textit{{}-tar} (\textsc{pst}) cannot appear in the interrogative clause (see also \sectref{sec:key:8.4.1.1}). In that case, \textit{{}-tɨ} (\textsc{seq}) is used to express the past tense as in (11-6 a-b). Therefore, the particle that expresses the polar question, e.g., \textit{na} (\textsc{plq}), cannot co-occur with \textit{{}-tar} (P\textsc{st}) as in (11-7 b), but can with \textit{{}-tɨ} (SEQ) as in (11-7 a).

\ea\label{ex:11-7}  \textit{na} (\textsc{plq}) in the past tense

  \ea  
      %\textsc{tm}
      \glll    waatɨna?\\
      \textit{waar-\Highlight{tɨ=na}}\\
      understand-\textsc{seq}=\textsc{plq}\\
      \glt       ‘(Did you) understand?’ [El: 090830]

  \ex  
      %\textsc{tm}
      \glll    *waatana?\\
       \textit{waar-\Highlight{tar=na}}\\
      understand-\textsc{pst}=\textsc{plq}\\
      \glt       (Intended meaning) ‘(Did you) understand?’ [El: 090830]
\z
\z

  It should be noted that \textit{{}-tar} (\textsc{pst}) can appear in the interrogative clasue if it is followed by \textit{{}-u} (\textsc{pfc}) as in (11-18 a-b) in \sectref{sec:11.3.2}, or if it is followed by \textit{{}-mɨ} (\textsc{plq}), although the combination of \textit{{}-tar-mɨ} (P\textsc{st}-\textsc{pl}Q) has not yet appeared in the text data (it only appears in elicitation). Additionally, if the alleged interrogative clause is used to express the speaker’s wondering to herself, \textit{{}-tar} (PST) can be used as in \textsc{ref}{ex:11-8} (see also \sectref{sec:key:10.3.6}).

\ea\label{ex:11-8}  \textit{nuu} ‘what’ co-occuring with \textit{{}-tar} (\textsc{pst}) because of \textit{kai} (\textsc{du}B) [= (10-50)]   [Context: \textsc{ms} asked \textsc{tm} whether the place in the picture used to be called “Yubinhana.”]\\
  
      %\textsc{tm}
      \glll    nuucjɨga  jutakaijaa?\\
    \textit{\Highlight{nuu}=ccjɨ=ga}  \textit{jˀ-jur-\Highlight{tar=kai}=jaa}\\
    what=\textsc{qt}=\textsc{foc}  call-\textsc{umrk}-\textsc{pst}=\textsc{du}B=\textsc{sol}\\
    \glt     ‘(I) wonder what (people) used to call (the place).’ [Co: 120415\_00.txt]
\z

\subsection{\textit{{}-ba} (\textsc{csl}) as the insubordination}\label{sec:11.2.2}

Non-finite uses of the converbal affix \textit{-ba} (\textsc{csl}) are found in the adverbial clause expressing causal meaning as in \sectref{sec:key:8.4.3.1}. However, there is a finite use of the converbal affix \textit{-ba} (CSL), which expresses the speaker’s request to the hearer as in (11-9 a-c). In that case, \textit{{}-ba} (CSL) always appears in the \textsc{av}C following the auxiliary verbs \textit{kurɨr-} (\textsc{ben}) or \textit{taboor-} (BEN.\textsc{hon}).

\ea\label{ex:11-9}  \textit{kurɨr-} (\textsc{ben}) \textit{+-ba} (\textsc{csl})

  \ea  
      %\textsc{tm}
      \gllll    hanacjɨ  kurɨppa.  dooka.\\
      \textit{hanas-tɨ}  \textit{kurɨr-\Highlight{ba}}  \textit{dooka}\\
      talk-\textsc{seq}  \textsc{ben}-\textsc{csl}  please\\
      [Lex. verb  Aux. verb]\textsubscript{\textsc{av}C}  \\
      \glt ‘Please, talk (to me).’      [Co: 120415\_01.txt]

  \ex  
      %\textsc{tm}
      \gllll    naa  hazimɨtɨ  kurɨppajoo.\\
      \textit{naa}  \textit{hazimɨr-tɨ}  \textit{kurɨr-\Highlight{ba}=joo}\\
      \textsc{fil}  begin-\textsc{seq}  \textsc{ben}-\textsc{csl}=\textsc{cfm}1\\
        [Lex. verb  Aux. verb]\textsubscript{\textsc{av}C}\\
     \glt  ‘(Please) begin (the training for the traditional dance for our community).’      [Co: 120415\_01.txt]

  \textit{taboor-} (\textsc{ben}.\textsc{hon}) \textit{+-ba} (\textsc{csl})

  \ex  
      %\textsc{tm}
      \gllll    umoojaganaa,  abɨtɨ  tabooppajoo.\\
      \textit{umoor-jaganaa}  \textit{abɨr-tɨ}  \textit{taboor-\Highlight{ba}=joo}\\
      come.\textsc{hon}-\textsc{sim}  call-\textsc{seq}  \textsc{ben}.HON-\textsc{csl}=\textsc{cfm}1\\
        [Lex. verb  Aux. verb]\textsubscript{\textsc{av}C}\\
      \glt ‘Coming (here), call (the person for me please).’      [El: 120930]
\z
\z

\subsection{\textit{ccjɨ=joo} (\textsc{qt}=\textsc{cfm}1) as the insubordination}\label{sec:11.2.3}

\textit{ccjɨ} (\textsc{qt}) embeds any utterance into the complement of the superordinate clause in principle. For example, an imperative clause is embedded into the complement of \textit{jˀ-} ‘say’ as in \textsc{ref}{ex:11-10}.

\ea\label{ex:11-10}  \textit{ccjɨ} (\textsc{qt}) in the complement clause [= (8-148 g)]\\
  
      %\textsc{tm}
      \glll    kanɨcɨboja  urakja  tuikurawɨcjɨ  jˀicjɨ,\\
    [\textit{kanɨ+cɨbo=ja}  \textit{urakja}  \textit{tur-i+kuraw-ɨ=\Highlight{ccjɨ}}]\textsubscript{Complement clause}  \textit{jˀ-tɨ}\\
    gold+pot=\textsc{top}  2.N\textsc{hon}.\textsc{pl}  take-\textsc{inf}+\textsc{drg}-\textsc{imp}=\textsc{qt}  say-\textsc{seq}\\
    \glt     ‘(The man) said that, “You take (this) damn gold pot!” and ...’ [Fo: 090307\_00.txt]
\z

However, if it is followed by \textit{joo} (\textsc{cfm}1), it always expresses an objective (not hearsay) information without any superordinate clause as in \textsc{ref}{ex:11-11}.

\ea\label{ex:11-11}  \textit{ccjɨ} (\textsc{qt}) in the insubordination [= (10-75 a)]   [Context: The speaker explains the story of the Pear Film to the hearer.]\\
  
      %\textsc{tm}
      \glll    tuutɨ  izjancjɨjoo.\\
    \textit{tuur-tɨ}  \textit{ik-tar-n=\Highlight{ccjɨ=joo}}\\
    pass-\textsc{seq}  go-\textsc{pst}-\textsc{ptcp}=\textsc{qt}=\textsc{cfm}1\\
    \glt     ‘(A young man who pulls a goat) passed by.’ [\textsc{pf}: 090305\_01.txt]
\z

The more detail discussion was done in \sectref{sec:key:10.4.1.7}.

\subsection{\textit{{}-an-boo} (\textsc{neg}-\textsc{cnd}) as the pre-insubordination}\label{sec:11.2.4}

The converbal affix \textit{{}-boo} (\textsc{cnd}) expresses the conditional meaning. Interestingly, the combination of \textit{{}-an-boo} (\textsc{neg}-CND) in the adverbial clause and \textit{nar-an} (become-NEG) in the main clause expresses the obligative meaning as in \textsc{ref}{ex:11-12}, where the obligative meaning is expressed in the adverbial clause.

\ea\label{ex:11-12}  Obligation expressed by \textit{{}-an-boo} (\textsc{neg}-\textsc{cnd}) plus \textit{nar-an} (become-NEG) [=(9-40)]\\
  
      %\textsc{tm}
      \glll    waasan  ucjəə,  ganba  hatarakanboo, naranbajaa.\\
    \textit{waa-sa+ar-n}  \textit{uci=ja}  \textit{ganba}  \textit{hatarak-\Highlight{an-boo}} \textit{\Highlight{nar-an}-ba=jaa}\\
    young-\textsc{adj}+\textsc{st}V-\textsc{ptcp}  period=\textsc{top}  therefore  work-\textsc{neg}-\textsc{cnd}  become-NEG-\textsc{csl}=\textsc{sol}\\
    \glt ‘While (one) is young, (one) has to work.’     [Co: 120415\_01.txt]
\z

The above collocation has an idiomatic meaning (i.e. obligation), and it is difficult to construct the meaning from the literal meaning of each morpheme. The idiomatic meaning is frequently expressed without the main clause, which is the “conventionalization of ellipsis” \citep[372-373]{Evans2007} as in (11-13 a-d).

\ea\label{ex:11-13}  Obligation expressed only by \textit{{}-an-boo} (\textsc{neg}-\textsc{cnd})

  \ea\relax  [= (8-122 b)]

    
      %\textsc{tm}
      \glll    nan  umoorasanboocjɨ  umutɨ,\\
      \textit{nan}  \textit{umoor-as-\Highlight{an-boo}=ccjɨ}  \textit{umuw-tɨ}\\
      2.\textsc{hon}.\textsc{sg}  come.HON-\textsc{caus}-\textsc{neg}-\textsc{cnd}=\textsc{qt}  think-\textsc{seq}\\
      \glt       ‘(I) thought that (I) have to make you come, and ...’ [Co: 110328\_00.txt]

  \ex\relax[= (10-33)]

    
      %\textsc{tm}
      \glll    jazin  kjunmuncjɨ  umutɨ  kurɨranboo.\\
      \textit{jazin}  \textit{k-jur-n=mun=ccjɨ}  \textit{umuw-tɨ}  \textit{kurɨr-\Highlight{an-boo}}\\
      necessarily  come-\textsc{umrk}-\textsc{ptcp}=\textsc{advrs}=\textsc{qt}  think-\textsc{seq}  \textsc{ben}-\textsc{neg}-\textsc{cnd}\\
      \glt       ‘(You) have to think that necessarily (you) will come.’ [Co: 101023\_01.txt]

  \ex\relax[= (4-57)]

    
      %\textsc{tm}
      \glll    ude,  naa,  ganboo,  urakjoo  ude,  ude,  kamanboo, udeccjɨdu  xxx  jutattujaa.\\
      \textit{ude}  \textit{naa}  \textit{ganboo}  \textit{urakja=ja}  \textit{ude}  \textit{ude}  \textit{kam-\Highlight{an-boo}} \textit{ude=ccjɨ=du}  N/A  \textit{jˀ-jur-tar-tu=jaa}\\
      well  \textsc{fil}  if.so  2.N\textsc{hon}.\textsc{sg}=\textsc{top}  well  well  eat-\textsc{neg}-\textsc{cnd}  well=\textsc{qt}=\textsc{foc}  N/A  say-\textsc{umrk}-\textsc{pst}-\textsc{csl}=\textsc{sol}\\
    \glt       ‘(The old people) would say, ‘Well, now, then, you have to eat (more).’’ [Co: 120415\_01.txt]

  \ex  
      %\textsc{tm}
      \glll    uraba  həəku  tɨmɨranbooccjɨga.\\
      \textit{ura=ba}  \textit{həə-ku}  \textit{tɨmɨr-\Highlight{an-boo}=ccjɨ=ga}\\
      2.N\textsc{hon}.\textsc{sg}=\textsc{acc}  quick-\textsc{advz}  find-\textsc{neg}-\textsc{cnd}=\textsc{qt}=\textsc{foc}\\
      \glt       ‘(I think) that (I) have to find you quickly.’ [Co: 101023\_01.txt]
\z
\z

In the above examples, \textit{{}-an-boo} (\textsc{neg}-\textsc{cnd}) expresses obligation without \textit{nar-an} (become-NEG). In other words, the subordiante clauses headed by (the verb that includes) \textit{{}-an-boo} (NEG-CND) has obtained the grammatical meaning of obligation.

\section{Focus construction (or “Kakari-musubi”)}\label{sec:11.3}

It is famous that there are a kind of focus constructions (i.e. constructions that include focus particles) that are traditoinally called \textit{kakari-musubi} (i.e. ‘government-predication’) in Japanese linguistics and Ryukyuan linguistics (cf. \citealt{Shimoji2008}: 565-570). The characteristics of the focus constructions in Yuwan can be summarized as follows.

\ea\label{ex:11-14}   Focus construction (or “Kakari-musubi”) in Yuwan

\ea   \textit{{}-n} (\textsc{ptcp}) is in the predicate of the main clause

> \textit{du} (\textsc{foc}) is in the clause, but not vice versa;

\ex   \textit{{}-u} (\textsc{pfc}) is in the predicate

> \textit{du} (\textsc{foc}) or an interrogative word is in the clause, but not vice versa.
\z
\z

The argumentation for \textsc{ref}{ex:11-14} is shown in the following sections. First, I will present examples of the focus construction of \textit{du} (\textsc{foc}) in \sectref{sec:11.3.1}. Then, I will present examples of the focus construction of \textit{ga} (FOC) in \sectref{sec:11.3.2}.

\subsection{Focus construction of \textit{du} (\textsc{foc})}\label{sec:11.3.1}

In Yuwan, the participle that has \textit{{}-n} (\textsc{ptcp}) fills the predicate of the adnominal clause, and it cannot fill the predicate of the main claue in principle (see also \sectref{sec:11.1.2}). However, if the focus particle \textit{du} appears in the same clause, the participle can fill the predicate of the main claue as in (\ref{ex:11-14}a--d).

\ea\label{ex:11-15}  \textit{du} (\textsc{foc}) co-occuring with \textit{{}-n} (\textsc{ptcp}) in the main clause

  \ea\relax[= (6-108 a)]

    
      %\textsc{tm}
      \glll    nuunu  nangikaicjɨdu  umujun.\\
      \textit{nuu=nu}  \textit{nangi=kai=ccjɨ=\Highlight{du}}  \textit{umuw-jur-\Highlight{n}}\\
      what=\textsc{gen}  trouble=\textsc{du}B=\textsc{qt}=\textsc{foc}  think-\textsc{umrk}-\textsc{ptcp}\\
\glt       ‘(I) wonder what (kinds) of trouble (I took).’ [i.e. ‘I didn’t want to take such trouble.’] [Co: 120415\_01.txt]

  \ex  
      %\textsc{tm}
      \glll    kadɨdu,  cikjaranu  izijun.\\
      \textit{kam-tɨ=\Highlight{du}}  \textit{cikjara=nu}  \textit{izir-jur-\Highlight{n}}\\
      eat-\textsc{seq}=\textsc{foc}  power=\textsc{nom}  go.out-\textsc{umrk}-\textsc{ptcp}\\
\glt       ‘(One) eat (food), and then the power goes out.’ [i.e. ‘One can become powerful after eating a meal.’] [Co: 120415\_01.txt]

  \ex  
      %\textsc{tm}
      \glll    dujasankutubəidu  siccjun.\\
      \textit{duja-sa+ar-n=kutu=bəi=\Highlight{du}}  \textit{sij-tur-\Highlight{n}}\\
      rich-\textsc{adj}+\textsc{st}V-\textsc{ptcp}=fact=only=\textsc{foc}  know-\textsc{prog}-PTCP\\
    \glt       ‘(I) know only the fact that (your grandparents) were rich.’ [Co: 120415\_01.txt]

  \ex\relax[Context: \textsc{tm} has been taught to chew her food well, but her stomach was not good until two or three years before.]

    
      %\textsc{tm}
      \glll    naa,  kunugurudu  jiccjan.\\
      \textit{naa}  \textit{kunuguru=\Highlight{du}}  \textit{jiccj-sa+ar-\Highlight{n}}\\
      \textsc{fil}  recently=\textsc{foc}  good-\textsc{adj}+\textsc{st}V-\textsc{ptcp}\\
    \glt       ‘(My stomach) has been good recently.’ [Co: 120415\_01.txt]
\z
\z

The above examples show that \textit{-n} (\textsc{ptcp}) can fill the predicate of the main clause if there is \textit{du} (\textsc{foc}) in the same clause. However, its opposite is not necessarily true. For example, \textit{{}-u} (\textsc{pfc}) can also fill the predicate of the main clause if there is \textit{du} (FOC) in the same clause as in (11-16 a-b).

\ea\label{ex:11-16}  \textit{du} (\textsc{foc}) co-occuring with \textit{{}-u} (\textsc{pfc}) [= (8-77 a)]

  \ea  
      %\textsc{tm}
      \glll    utuzjoobasanna  un  cˀjunu  samisjentudu  utoo  (sii..)  sɨrarɨɨru.  \\
      \textit{utuzjo+obasan=ja}  \textit{u-n}  \textit{cˀju=nu}  \textit{samisjen=tu=\Highlight{du}}  \textit{uta=ja}  \textit{sɨr-i}  \textit{sɨr-arɨr-\Highlight{u}}  \\
      Utujo+old.woman=\textsc{top}  \textsc{mes}-\textsc{adn}Z  person=\textsc{gen}  samisen=\textsc{com}=\textsc{foc} song=TOP  do-\textsc{inf}  do-\textsc{cap}-\textsc{pfc}  \\
      \glt       ‘Utujo can sing a song [lit. do a song] just with that person’s samisen. (Otherwise, she cannot sing a song.)’ [Co: 120415\_00.txt]

  \ex  
      %\textsc{tm}
      \glll    tacuu{\textbar}toka{\textbar}ga  juubadu,  jˀarɨɨru.\\
      \textit{tacuu=toka=ga}  \textit{jˀ-ba=\Highlight{du}}  \textit{jˀ-arɨr-\Highlight{u}}\\
      Tatsu=\textsc{appr}=\textsc{nom}  say-\textsc{csl}=\textsc{foc}  say-\textsc{cap}-\textsc{pfc}\\
      \glt       ‘(People) can say (a piece of advice to her), since (it is) Tatsu (who) says (it). (Otherwise, no one can give any advice to her.)’ [Co: 101023\_01.txt]
\z
\z

  Furtheremore, other inflectional affixes (or affix-like clitics) can co-occur with \textit{du} (\textsc{foc}) in the same clause as in (11-17 a-g).

\ea\label{ex:11-17}  \ea \textit{du} (\textsc{foc}) co-occuring with \textit{{}-i} (\textsc{npst})

  [Context: Mutsu went away saying that she would stop in an electric appliance store.]

  
      %\textsc{tm}
      \glll    muccuuja  jaakacidu  izjəijaa.\\
    \textit{muccuu=ja}  \textit{jaa=kaci=\Highlight{du}}  \textit{ik-təər-\Highlight{i}=jaa}\\
    Mutsu=\textsc{top}  house=\textsc{all}=\textsc{foc}  go-\textsc{rsl}-\textsc{npst}=\textsc{sol}\\
\glt     ‘Mutsu has gone (back) home.’ [Co: 110328\_00.txt]

  \ex \textit{du} (\textsc{foc}) co-occuring with \textit{doo} (\textsc{ass})
  [Context: \textsc{tm} said that there were no people who were able to make a wooden boat in Yuwan.]
  
      %\textsc{tm}
      \glll    kusinandu  wutattoo.\\
    \textit{kusi=nan=\Highlight{du}}  \textit{wur-tar=\Highlight{doo}}\\
    Kushi=\textsc{loc}1=\textsc{foc}  exist-\textsc{pst}=\textsc{ass}\\
\glt     ‘(People who can make a wooden boat) were in Kushi.’ [Co: 111113\_01.txt]

  \ex \textit{du} (\textsc{foc}) co-occuring with \textit{{}-tar} (\textsc{pst}) [= (8-134 a)]
  
      %\textsc{tm}
      \glll    kunuguru\Highlight{du}  kurəə  mucjɨ\footnotemark[1]{}  kjuu\Highlight{ta}.\\
    \textit{kunuguru=du}  \textit{ku-rɨ=ja}  \textit{mut-tɨ}  \textit{k-jur-ta}\\
    recently=\textsc{foc}  \textsc{prox}-\textsc{nlz}=\textsc{top}  have-\textsc{seq}  come-\textsc{umrk}-\textsc{pst}\\
\glt     ‘(Satsue’s child) brought this (picture) recently.’ [Co: 120415\_00.txt]

  \ex \textit{du} (\textsc{foc}) co-occuring with \textit{{}-ba} (\textsc{csl}) or \textit{-tɨ} (\textsc{seq}) [= (10-9 c)]

  
      %\textsc{tm}
      \glll    naa{\textbar}nihon{\textbar}bəidu  appa,  {\textbar}hacikiro{\textbar}naadu kinmɨ  sjɨ,  haatɨ,\\
    \textit{naa+nihon=bəi=\Highlight{du}}  \textit{ar-\Highlight{ba}}  \textit{hacikiro+naa=\Highlight{du}}  \textit{kinmɨ}  \textit{sɨr-\Highlight{tɨ}}  \textit{haar-\Highlight{tɨ}}\\
    another+two.\textsc{clf}=about=\textsc{foc}  exist-\textsc{csl}  eight.kilogram+each=FOC  measure  do-\textsc{seq}  measure-SEQ\\
    ‘There are other two white radishes, so (one) measures eight kilograms (of the materials) for each, and ...’    [Co: 101023\_01.txt]

  \ex \textit{du} (\textsc{foc}) co-occuring with \textit{{}-tu} (\textsc{csl})

  
      %\textsc{tm}
      \glll    kamɨccjɨdu  jutattu.\\
    \textit{kam-ɨ=ccjɨ=\Highlight{du}}  \textit{jˀ-tar-\Highlight{tu}}\\
    eat-\textsc{imp}=\textsc{qt}=\textsc{foc}  say-\textsc{pst}-\textsc{csl}\\
\glt     ‘(The people in the past) said (roughly to children), “Eat!”’ [Co: 120415\_01.txt]

  \ex \textit{du} (\textsc{foc}) co-occuring with \textit{{}-i} (\textsc{inf})

  
      %\textsc{tm}
      \glll    iccjaijaacjɨdu  umuii.\\
    \textit{jiccj}\footnotemark[2]{}\textit{{}-sa+ar-i=jaa=ccjɨ=\Highlight{du}  umuw-\Highlight{i}}\\
    good-\textsc{adj}+\textsc{st}V-\textsc{npt}=\textsc{qt}=\textsc{foc}  think-\textsc{inf}\\
    \glt     ‘(I) think that (it) is good.’ [Co: 120415\_01.txt]
\z
\z
\footnotetext[1]{\textit{mut-tɨ} (have-\textsc{seq}) usually becomes /muccjɨ/ according to the rule in \sectref{sec:key:8.3.1.2}. However, it becomes /mucjɨ/ in this example.}
\footnotetext[2]{\textit{jiccj-sa} (good-\textsc{adj}) usually becomes /jiccja/ [itt͡ɕɑ̟], but it becomes /iccja/ [ʔitt͡ɕɑ̟] in this example.}

The above examples show that \textit{du} (\textsc{foc}) does not necessarily induce \textit{{}-n} (\textsc{ptcp}) or \textit{{}-u} (\textsc{pfc}) in the predicate in the same clause. \textit{du} (FOC) can occur not only in the main clause, but also in the adverbial clause as in (11-17 d). Furthermore, \textit{du} (FOC) can occur in the adnominal clause in the literal meaning (i.e. the clasue that modifies an NP in effect) as in (10-9 d) in \sectref{sec:key:10.1.2.1}.

\subsection{Focus construction of \textit{ga} (\textsc{foc})}\label{sec:11.3.2}

The finite-form affix \textit{{}-u} (\textsc{pfc}) only appears in the clauses that include \textit{du} (\textsc{foc}) or in the interrogative clauses of information question (see also \sectref{sec:key:8.4.1.6}). The interrogative words are often followed by \textit{ga} (FOC) (see also \sectref{sec:key:5.3.1}). I will present examples of \textit{{}-u} (\textsc{pf}C) co-occuring with \textit{ga} (FOC) as in (11-18 a-d). The examples of \textit{-u} (PFC) co-occuring with \textit{du} (FOC) were already shown in \textsc{ref}{ex:11-16} in \sectref{sec:11.3.1}.

\ea\label{ex:11-18}  \textit{ga} (\textsc{foc}) co-occuring with \textit{{}-u} (\textsc{pfc}) and the interrogative word

  \ea\relax[Context: \textsc{tm} was surprised that US brought a lot of foods to TM’s house.] = (6-101 a)

    
      %\textsc{tm}
      \glll    nunkjabaga  mata  muccjɨ  moocjaru?\\
      \textit{\Highlight{nuu}=nkja=ba=ga}  \textit{mata}  \textit{mut-tɨ}  \textit{moor-tar-\Highlight{u}}\\
      what=\textsc{appr}=\textsc{acc}=\textsc{foc}  again  have-\textsc{seq}  \textsc{hon}-\textsc{pst}-\textsc{pfc}\\
      \glt       ‘What did (you) bring (here) again?’ [Co: 110328\_00.txt]

  \ex  
      %\textsc{tm}
      \glll    nuu  sjɨga,  asɨbjutaru?\\
      \textit{\Highlight{nuu}}  \textit{sɨr-tɨ=\Highlight{ga}}  \textit{asɨb-jur-tar-\Highlight{u}}\\
      what  do-\textsc{seq}=\textsc{foc}  play-\textsc{umrk}-\textsc{pst}-\textsc{pfc}\\
      \glt       ‘What did (you) do for play (in your childhood)?’ [lit. ‘Doing what, did (you) play?’] [Co: 110328\_00.txt]

  \ex  
      %\textsc{tm}
      \glll    kurəə  nuu{\textbar}sjooten{\textbar}cjɨga  kacjəəru?\\
      \textit{ku-rɨ=ja}  \textit{\Highlight{nuu}+sjooten=ccjɨ=\Highlight{ga}}  \textit{kak-təər-\Highlight{u}}\\
      \textsc{prox}-\textsc{nlz}=\textsc{top}  what+shop=\textsc{qt}=\textsc{foc}  write-\textsc{rsl}-\textsc{pfc}\\
      \glt        ‘What was written on this shop(’s signboard in the picture)?’ [lit. ‘What shop have (people) written on this?’]  [Co: 120415\_00.txt]

  \ex  
      %\textsc{tm}
      \glll    nuucjɨga  arɨboo  juru?\\
      \textit{\Highlight{nuu}=ccjɨ=\Highlight{ga}}  \textit{a-rɨ=ba=ja}  \textit{jˀ-jur-\Highlight{u}}\\
      what=\textsc{qt}=\textsc{foc}  \textsc{dist}-\textsc{nlz}=\textsc{acc}=\textsc{top}  say-\textsc{umrk}-\textsc{pfc}\\
      \glt       ‘What is that person called?’ [i.e. ‘What is his name?’] [Co: 120415\_00.txt]
\z
\z

In (11-18 a-d), \textit{{}-u} (\textsc{pfc}) co-occurs with \textit{ga} (\textsc{foc}). However, the existense of \textit{ga} (FOC) does not induce that of \textit{{}-u} (\textsc{pf}C). For example, \textit{ga} (FOC) in the (alleged) interrogative clause can appear without \textit{-u} (PFC) if it is followed by \textit{kai} (\textsc{du}B) as in \textsc{ref}{ex:11-8} in \sectref{sec:11.2.1}. Moreover, \textit{ga} (FOC) can be used in the non-interrogative clauses, where \textit{ga} (FOC) does not take \textit{{}-u} (FOC) as in \REF{ex:11-19} (see \sectref{sec:key:10.1.2.2} for more details).

\ea\label{ex:11-19}  \textit{ga} (\textsc{foc}) not co-occuring with \textit{-u} (\textsc{pfc}) [= (10-14 b)]
  
      %\textsc{tm}
      \glll    kunəədaga  waakja  dusinu,  asikendusinu,  wutɨ,\\
    \textit{kunəəda=\Highlight{ga}}  \textit{waakja-a}  \textit{dusi=nu}  \textit{asiken+dusi=nu}  \textit{wur-\Highlight{tɨ}}\\
    the.other.day=\textsc{foc}  1\textsc{pl}-\textsc{adn}Z  friend=\textsc{nom}  Ashiken+frend=NOM  exist{}-\textsc{seq}\\
    \glt     ‘The other day, there is my friend, (i.e.) a friend in Ashiken, and ...’ [Co: 120415\_00.txt]
\z

In the above example, \textit{ga} (\textsc{foc}) co-occurs with \textit{{}-tɨ} (\textsc{seq}).

\appendix
\chapter{Morphophonological alternations of verbs}
\hypertarget{RefHeadingToc395697264}{}

Pre-notes:

(a)  The following lists correspond to the stem classes (Stems No. 1-17 and irregular verbal stems) and affix classes (Types A-D) discussed in \sectref{sec:key:8.2};

(b)  The affixes shown below exclude the Group II inflectional affixes discussed in \tabref{tab:key:55} in \sectref{sec:key:8.1} since they do not directly follow verbal roots and the morphophonological alternation caused by them are very transparent;

(c)  All of the non-italic verbal forms express the surface forms;

(d)  Examples other than those marked by the asterisk (*) were actually pronounced (in the elicitation or the natural discourse) by the speaker TM;

(e)  The examples marked by the asterisk (*) are created by the present author using synchronic (morpho)phonological rules in Yuwan;

(f)  The question mark (?) means that the speaker TM never uttered the form and that the present author could not find the natural context where the form can be used in elicitation;

(g)  Infinitives (simple forms and lengthened forms) are shown in the final page.

Stem No. 1 (ending with //V\textsubscript{non-back}r//): \textit{hingir-} ‘escape’

\tablefirsthead{}

\tabletail{}
\tablelasttail{}
\begin{tabularx}{\textwidth}{XXXXXXXXXXXXXXXXXXXXXXX}
\lsptoprule
\multicolumn{23}{X}{{\bfseries Type-A affixes}}\\
{ \textit{{}-an} (NEG)} & \multicolumn{5}{X}{{ \textit{{}-ar(ɨr)} (PASS)}} & \multicolumn{4}{X}{{ \textit{{}-ar(ɨr)} (CAP)}} & \multicolumn{4}{X}{{ \textit{{}-as} (CAUS)}} & \multicolumn{3}{X}{{ \textit{{}-azɨi} (NEG.PLQ)}} & \multicolumn{2}{X}{{ \textit{{}-ɨ} (IMP)}} & \multicolumn{2}{X}{{ \textit{{}-ɨba} (SUGS)}} & { \textit{{}-oo}(INT)} & \\
{ hingir-an} & \multicolumn{5}{X}{{ hingir-at-ta}} & \multicolumn{4}{X}{{ hingir-arɨk=kai}} & \multicolumn{4}{X}{{ N/A\footnotemark{}}} & \multicolumn{3}{X}{{ hingir-azɨi}} & \multicolumn{2}{X}{{ hingir-ɨ}} & \multicolumn{2}{X}{{ hingir-ɨba}} & { hingir-oo} & \\
escape-NEG & \multicolumn{5}{X}{escape-PASS-PST} & \multicolumn{4}{X}{escape-CAP=DUB} & \multicolumn{4}{X}{} & \multicolumn{3}{X}{escape-NEG.PLQ} & \multicolumn{2}{X}{escape-IMP} & \multicolumn{2}{X}{escape-SUGS} & escape-INT & \\
\multicolumn{23}{X}{}\\
\multicolumn{23}{X}{{\bfseries Type-B affixes}}\\
\multicolumn{3}{X}{{ \textit{{}-tar} (PST)}} & \multicolumn{5}{X}{{ \textit{{}-tuk} (PRPR)}} & \multicolumn{4}{X}{{ \textit{{}-tur} (PROG)}} & \multicolumn{3}{X}{{ \textit{{}-təər} (RSL)}} & \multicolumn{3}{X}{{ \textit{{}-tɨ} (SEQ)}} & \multicolumn{2}{X}{{ \textit{{}-tai} (LST)}} & \multicolumn{3}{X}{{ \textit{{}-təəra} ‘after’}}\\
\multicolumn{3}{X}{{ hingi-tat=too}} & \multicolumn{5}{X}{{ hingi-tuk-ɨ}} & \multicolumn{4}{X}{{ hingi-tu-i}} & \multicolumn{3}{X}{{ hingi-təəp-pa}} & \multicolumn{3}{X}{{ *hingi-tɨ}} & \multicolumn{2}{X}{{ *hingi-tai}} & \multicolumn{3}{X}{{ *hingi-təəra}}\\
\multicolumn{3}{X}{escape-PST=ASS} & \multicolumn{5}{X}{escape-PRPR-IMP} & \multicolumn{4}{X}{escape-PROG-NPST} & \multicolumn{3}{X}{escape-RSL-CSL} & \multicolumn{3}{X}{escape-SEQ} & \multicolumn{2}{X}{escape-LST} & \multicolumn{3}{X}{escape-after}\\
\multicolumn{3}{X}{} & \multicolumn{5}{X}{} & \multicolumn{4}{X}{} & \multicolumn{3}{X}{} & \multicolumn{3}{X}{} & \multicolumn{2}{X}{} & \multicolumn{3}{X}{}\\
\multicolumn{23}{X}{{\bfseries Type-C affixes}}\\
\multicolumn{4}{X}{{ \textit{{}-jawur} (POL)}} & \multicolumn{3}{X}{{ \textit{{}-jaa} ‘person’}} & \multicolumn{4}{X}{{ \textit{{}-jur} (UMRK)}} & \multicolumn{5}{X}{{ \textit{{}-jagacinaa} (SIM)}} & \multicolumn{7}{X}{}\\
\multicolumn{4}{X}{{ *hingi-jawu-i}} & \multicolumn{3}{X}{{ hingi-jaa}} & \multicolumn{4}{X}{{ hingi-jus-sa}} & \multicolumn{5}{X}{{ hingi-jagacinaa}} & \multicolumn{7}{X}{}\\
\multicolumn{4}{X}{escape-POL-NPST} & \multicolumn{3}{X}{escape-person} & \multicolumn{4}{X}{escape-UMRK-POL} & \multicolumn{5}{X}{escape-SIM} & \multicolumn{7}{X}{}\\
\multicolumn{23}{X}{}\\
\multicolumn{23}{X}{{\bfseries Type-D affixes and clitic}}\\
\multicolumn{2}{X}{{ \textit{{}-ba} (CSL)}} & \multicolumn{3}{X}{{ \textit{{}-boo} (CND)}} & \multicolumn{4}{X}{{ \textit{{}-gadɨ} ‘until’}} & \multicolumn{4}{X}{{ \textit{{}-na} (PROH)}} & \multicolumn{4}{X}{{ \textit{kai} (DUB)}} & \multicolumn{6}{X}{}\\
\multicolumn{2}{X}{{ hingip-pa}} & \multicolumn{3}{X}{{ *hingip-poo}} & \multicolumn{4}{X}{{ *hingik-kadɨ}} & \multicolumn{4}{X}{{ hingin-na}} & \multicolumn{4}{X}{{ *hingik=kai}} & \multicolumn{6}{X}{}\\
\multicolumn{2}{X}{escape-CSL} & \multicolumn{3}{X}{escape-CND} & \multicolumn{4}{X}{escape-until} & \multicolumn{4}{X}{escape-PROH} & \multicolumn{4}{X}{escape=DUB} & \multicolumn{6}{X}{}\\
\lspbottomrule
\end{tabularx}
\footnotetext{ A different root is used with \textit{{}-as} (CAUS), i.e. /hing{}-jas-ju-i/ \textit{hing{}-as-jur-i} (escape-CAUS-UMRK-NPST).}

Stem No. 1 (ending with //V\textsubscript{non-back}r//): \textit{abɨr-} ‘call’

\tablefirsthead{}

\tabletail{}
\tablelasttail{}
\begin{tabularx}{\textwidth}{XXXXXXXXXXXXXXXXXXXXXX}
\lsptoprule
\multicolumn{22}{X}{{\bfseries Type-A affixes}}\\
{ \textit{{}-an} (NEG)} & \multicolumn{3}{X}{{ \textit{{}-ar(ɨr)} (PASS)}} & \multicolumn{5}{X}{{ \textit{{}-ar(ɨr)} (CAP)}} & \multicolumn{3}{X}{{ \textit{{}-as} (CAUS)}} & \multicolumn{5}{X}{{ \textit{{}-azɨi} (NEG.PLQ)}} & { \textit{{}-ɨ} (IMP)} & \multicolumn{2}{X}{{ \textit{{}-ɨba} (SUGS)}} & { \textit{{}-oo}(INT)} & \\
{ abɨr-an} & \multicolumn{3}{X}{{ abɨr-at-tɨ}} & \multicolumn{5}{X}{{ abɨr-arɨ-n=nja}} & \multicolumn{3}{X}{{ abɨr-as-ɨ}} & \multicolumn{5}{X}{{ abɨr-azɨi}} & { abɨr-ɨ} & \multicolumn{2}{X}{{ abɨr-ɨba}} & { abɨr-oo} & \\
call-NEG & \multicolumn{3}{X}{call-PASS-SEQ} & \multicolumn{5}{X}{call-CAP-NPST=PLQ} & \multicolumn{3}{X}{call-CAUS-IMP} & \multicolumn{5}{X}{call-NEG.PLQ} & call-IMP & \multicolumn{2}{X}{call-SUGS} & call-INT & \\
\multicolumn{22}{X}{}\\
\multicolumn{22}{X}{{\bfseries Type-B affixes}}\\
\multicolumn{2}{X}{{ \textit{{}-tar} (PST)}} & \multicolumn{3}{X}{{ \textit{{}-tuk} (PRPR)}} & \multicolumn{3}{X}{{ \textit{{}-tur} (PROG)}} & \multicolumn{5}{X}{{ \textit{{}-təər} (RSL)}} & \multicolumn{3}{X}{{ \textit{{}-tɨ} (SEQ)}} & \multicolumn{3}{X}{{ \textit{{}-tai} (LST)}} & \multicolumn{3}{X}{{ \textit{{}-təəra} ‘after’}}\\
\multicolumn{2}{X}{{ abɨ-ta}} & \multicolumn{3}{X}{{ abɨ-tuk-ɨ}} & \multicolumn{3}{X}{{ abɨ-tu-i}} & \multicolumn{5}{X}{{ abɨ-təət=too}} & \multicolumn{3}{X}{{ *abɨ-tɨ}} & \multicolumn{3}{X}{{ *abɨ-tai}} & \multicolumn{3}{X}{{ *abɨ-təəra}}\\
\multicolumn{2}{X}{call-PST} & \multicolumn{3}{X}{call-PRPR-IMP} & \multicolumn{3}{X}{call-PROG-NPST} & \multicolumn{5}{X}{call-RSL=ASS} & \multicolumn{3}{X}{call-SEQ} & \multicolumn{3}{X}{call-LST} & \multicolumn{3}{X}{{ call-after}}\\
\multicolumn{2}{X}{} & \multicolumn{3}{X}{} & \multicolumn{3}{X}{} & \multicolumn{5}{X}{} & \multicolumn{3}{X}{} & \multicolumn{3}{X}{} & \multicolumn{3}{X}{}\\
\multicolumn{22}{X}{{\bfseries Type-C affixes}}\\
\multicolumn{3}{X}{{ \textit{{}-jawur} (POL)}} & \multicolumn{3}{X}{{ \textit{{}-jaa} ‘person’}} & \multicolumn{4}{X}{{ \textit{{}-jur} (UMRK)}} & \multicolumn{4}{X}{{ \textit{{}-jagacinaa} (SIM)}} & \multicolumn{8}{X}{}\\
\multicolumn{3}{X}{{ *abɨ-jawu-i}} & \multicolumn{3}{X}{{ abɨ-jaa}} & \multicolumn{4}{X}{abɨ-ju-i} & \multicolumn{4}{X}{{ abɨ-jagacinaa}} & \multicolumn{8}{X}{}\\
\multicolumn{3}{X}{call-POL-NPST} & \multicolumn{3}{X}{call-person} & \multicolumn{4}{X}{call-UMRK-NPST} & \multicolumn{4}{X}{call-SIM} & \multicolumn{8}{X}{}\\
\multicolumn{22}{X}{}\\
\multicolumn{22}{X}{{\bfseries Type-D affixes and clitic}}\\
\multicolumn{2}{X}{{ \textit{{}-ba} (CSL)}} & \multicolumn{3}{X}{{ \textit{{}-boo} (CND)}} & \multicolumn{2}{X}{{ \textit{{}-gadɨ} ‘until’}} & \multicolumn{4}{X}{{ \textit{{}-na} (PROH)}} & \multicolumn{4}{X}{{ \textit{kai} (DUB)}} & \multicolumn{7}{X}{}\\
\multicolumn{2}{X}{{ abɨp-pa}} & \multicolumn{3}{X}{{ abɨp-poo}} & \multicolumn{2}{X}{{ *abɨk-kadɨ}} & \multicolumn{4}{X}{{ abɨn-na}} & \multicolumn{4}{X}{{ *abɨk=kai}} & \multicolumn{7}{X}{}\\
\multicolumn{2}{X}{call-CSL} & \multicolumn{3}{X}{call-CND} & \multicolumn{2}{X}{call-until} & \multicolumn{4}{X}{call-PROH} & \multicolumn{4}{X}{call=DUB} & \multicolumn{7}{X}{}\\
\lspbottomrule
\end{tabularx}
Stem No. 1 (ending with //V\textsubscript{non-back}r//): \textit{kəər-} ‘exchange’

\tablefirsthead{}

\tabletail{}
\tablelasttail{}
\begin{tabularx}{\textwidth}{XXXXXXXXXXXXXXXXXXXX}
\lsptoprule
\multicolumn{20}{X}{{\bfseries Type-A affixes}}\\
{ \textit{{}-an} (NEG)} & \multicolumn{3}{X}{{ \textit{{}-ar(ɨr)} (PASS)}} & \multicolumn{4}{X}{{ \textit{{}-ar(ɨr)} (CAP)}} & \multicolumn{4}{X}{{ \textit{{}-as} (CAUS)}} & \multicolumn{3}{X}{{ \textit{{}-azɨi} (NEG.PLQ)}} & \multicolumn{2}{X}{{ \textit{{}-ɨ} (IMP)}} & \multicolumn{2}{X}{{ \textit{{}-ɨba} (SUGS)}} & { \textit{{}-oo}(INT)}\\
{ kəər-an} & \multicolumn{3}{X}{{ kəər-at-tup-pa}} & \multicolumn{4}{X}{{ kəər-ar-an}} & \multicolumn{4}{X}{{ kəər-as-oo}} & \multicolumn{3}{X}{{ kəər-azɨi}} & \multicolumn{2}{X}{{ kəər-ɨ}} & \multicolumn{2}{X}{{ kəər-ɨba}} & { kəər-oo}\\
exchange-NEG & \multicolumn{3}{X}{exchange-PASS-PROG-CSL} & \multicolumn{4}{X}{exchange-CAP-NEG} & \multicolumn{4}{X}{exchange-CAUS-INT} & \multicolumn{3}{X}{exchange-NEG.PLQ} & \multicolumn{2}{X}{exchange-IMP} & \multicolumn{2}{X}{exchange-SUGS} & exchange-INT\\
\multicolumn{20}{X}{}\\
\multicolumn{20}{X}{{\bfseries Type-B affixes}}\\
\multicolumn{2}{X}{{ \textit{{}-tar} (PST)}} & \multicolumn{4}{X}{{ \textit{{}-tuk} (PRPR)}} & \multicolumn{5}{X}{{ \textit{{}-tur} (PROG)}} & \multicolumn{3}{X}{{ \textit{{}-təər} (RSL)}} & \multicolumn{2}{X}{{ \textit{{}-tɨ} (SEQ)}} & \multicolumn{2}{X}{{ \textit{{}-tai} (LST)}} & \multicolumn{2}{X}{{ \textit{{}-təəra} ‘after’}}\\
\multicolumn{2}{X}{{ kəə-tat=too}} & \multicolumn{4}{X}{{ kəə-tuk-ɨ}} & \multicolumn{5}{X}{{ kəə-tu-i}} & \multicolumn{3}{X}{{ kəə-təəp-pa}} & \multicolumn{2}{X}{{ *kəə-tɨ}} & \multicolumn{2}{X}{{ *kəə-tai}} & \multicolumn{2}{X}{{ *kəə-təəra}}\\
\multicolumn{2}{X}{exchange-PST=ASS} & \multicolumn{4}{X}{exchange-PRPR-IMP} & \multicolumn{5}{X}{exchange-PROG-NPST} & \multicolumn{3}{X}{exchange-RSL-CSL} & \multicolumn{2}{X}{exchange-SEQ} & \multicolumn{2}{X}{exchange-LST} & \multicolumn{2}{X}{{ exchange-after}}\\
\multicolumn{2}{X}{} & \multicolumn{4}{X}{} & \multicolumn{5}{X}{} & \multicolumn{3}{X}{} & \multicolumn{2}{X}{} & \multicolumn{2}{X}{} & \multicolumn{2}{X}{}\\
\multicolumn{20}{X}{{\bfseries Type-C affixes}}\\
\multicolumn{3}{X}{{ \textit{{}-jawur} (POL)}} & \multicolumn{2}{X}{{ \textit{{}-jaa} ‘person’}} & \multicolumn{4}{X}{{ \textit{{}-jur} (UMRK)}} & \multicolumn{11}{X}{{ \textit{{}-jagacinaa} (SIM)}}\\
\multicolumn{3}{X}{{ *kəə-jawu-i}} & \multicolumn{2}{X}{{ kəə-jaa}} & \multicolumn{4}{X}{kəə-ju-i} & \multicolumn{11}{X}{{ kəə-jagacinaa}}\\
\multicolumn{3}{X}{exchange-POL-NPST} & \multicolumn{2}{X}{exchange-person} & \multicolumn{4}{X}{exchange-UMRK-NPST} & \multicolumn{11}{X}{exchange-SIM}\\
\multicolumn{20}{X}{}\\
\multicolumn{20}{X}{{\bfseries Type-D affixes and clitic}}\\
{ \textit{{}-ba} (CSL)} & \multicolumn{3}{X}{{ \textit{{}-boo} (CND)}} & \multicolumn{3}{X}{{ \textit{{}-gadɨ} ‘until’}} & \multicolumn{3}{X}{{ \textit{{}-na} (PROH)}} & \multicolumn{3}{X}{{ \textit{kai} (DUB)}} & \multicolumn{7}{X}{}\\
{ kəəp-pa} & \multicolumn{3}{X}{{ kəəp-poo}} & \multicolumn{3}{X}{{ *kəək-kadɨ}} & \multicolumn{3}{X}{{ kəən-na}} & \multicolumn{3}{X}{{ *kəək=kai}} & \multicolumn{7}{X}{}\\
exchange-CSL & \multicolumn{3}{X}{exchange-CND} & \multicolumn{3}{X}{exchange-until} & \multicolumn{3}{X}{exchange-PROH} & \multicolumn{3}{X}{exchange=DUB} & \multicolumn{7}{X}{}\\
\lspbottomrule
\end{tabularx}
Stem No. 2 (ending with //V\textsubscript{back}r//): \textit{kˀuur-} ‘close’

\tablefirsthead{}

\tabletail{}
\tablelasttail{}
\begin{tabularx}{\textwidth}{XXXXXXXXXXXXXXXXXXXXXXX}
\lsptoprule
\multicolumn{23}{X}{{\bfseries Type-A affixes}}\\
{ \textit{{}-an} (NEG)} & \multicolumn{6}{X}{{ \textit{{}-ar(ɨr)} (PASS)}} & \multicolumn{4}{X}{{ \textit{{}-ar(ɨr)} (CAP)}} & \multicolumn{4}{X}{{ \textit{{}-as} (CAUS)}} & \multicolumn{3}{X}{{ \textit{{}-azɨi} (NEG.PLQ)}} & \multicolumn{2}{X}{{ \textit{{}-ɨ} (IMP)}} & \multicolumn{2}{X}{{ \textit{{}-ɨba} (SUGS)}} & { \textit{{}-oo}(INT)}\\
{ kˀuur-an} & \multicolumn{6}{X}{{ kˀuur-at-tat-tu}} & \multicolumn{4}{X}{{ kˀuur-arɨ-n=nja}} & \multicolumn{4}{X}{{ kˀuur-as-oo}} & \multicolumn{3}{X}{{ kˀuur-azɨi}} & \multicolumn{2}{X}{{ kˀuur-ɨ}} & \multicolumn{2}{X}{{ kˀuur-ɨba}} & { kˀuur-oo}\\
close-NEG & \multicolumn{6}{X}{close-PASS-PST-CSL} & \multicolumn{4}{X}{close-CAP-NPST=PLQ} & \multicolumn{4}{X}{close-CAUS-INT} & \multicolumn{3}{X}{close-NEG.PLQ} & \multicolumn{2}{X}{close-IMP} & \multicolumn{2}{X}{close-SUGS} & close-INT\\
\multicolumn{23}{X}{}\\
\multicolumn{23}{X}{{\bfseries Type-B affixes}}\\
\multicolumn{3}{X}{{ \textit{{}-tar} (PST)}} & \multicolumn{5}{X}{{ \textit{{}-tuk} (PRPR)}} & \multicolumn{4}{X}{{ \textit{{}-tur} (PROG)}} & \multicolumn{4}{X}{{ \textit{{}-təər} (RSL)}} & \multicolumn{3}{X}{{ \textit{{}-tɨ} (SEQ)}} & \multicolumn{2}{X}{{ \textit{{}-tai} (LST)}} & \multicolumn{2}{X}{{ \textit{{}-təəra} ‘after’}}\\
\multicolumn{3}{X}{{ kˀuu-tat=too}} & \multicolumn{5}{X}{{ kˀuu-tuk-ɨ}} & \multicolumn{4}{X}{{ kˀuu-tut=too}} & \multicolumn{4}{X}{{ kˀuu-təə-tat-tu}} & \multicolumn{3}{X}{{ kˀuu-tɨ}} & \multicolumn{2}{X}{{ *kˀuu-tai}} & \multicolumn{2}{X}{{ *kˀuu-təəra}}\\
\multicolumn{3}{X}{close-PST=ASS} & \multicolumn{5}{X}{close-PRPR-IMP} & \multicolumn{4}{X}{close-PROG=ASS} & \multicolumn{4}{X}{close-RSL-PST-CSL} & \multicolumn{3}{X}{close-SEQ} & \multicolumn{2}{X}{close-LST} & \multicolumn{2}{X}{{ close-after}}\\
\multicolumn{3}{X}{} & \multicolumn{5}{X}{} & \multicolumn{4}{X}{} & \multicolumn{4}{X}{} & \multicolumn{3}{X}{} & \multicolumn{2}{X}{} & \multicolumn{2}{X}{}\\
\multicolumn{23}{X}{{\bfseries Type-C affixes}}\\
\multicolumn{4}{X}{{ \textit{{}-jawur} (POL)}} & \multicolumn{2}{X}{{ \textit{{}-jaa} ‘person’}} & \multicolumn{4}{X}{{ \textit{{}-jur} (UMRK)}} & \multicolumn{4}{X}{{ \textit{{}-jagacinaa} (SIM)}} & \multicolumn{9}{X}{}\\
\multicolumn{4}{X}{{ *kˀuu-jawu-i}} & \multicolumn{2}{X}{{ ?}} & \multicolumn{4}{X}{kˀuu-ju-i} & \multicolumn{4}{X}{{ kˀuu-jagacinaa}} & \multicolumn{9}{X}{}\\
\multicolumn{4}{X}{close-POL-NPST} & \multicolumn{2}{X}{} & \multicolumn{4}{X}{close-UMRK-NPST} & \multicolumn{4}{X}{close-SIM} & \multicolumn{9}{X}{}\\
\multicolumn{23}{X}{}\\
\multicolumn{23}{X}{{\bfseries Type-D affixes and clitic}}\\
\multicolumn{2}{X}{{ \textit{{}-ba} (CSL)}} & \multicolumn{3}{X}{{ \textit{{}-boo} (CND)}} & \multicolumn{4}{X}{{ \textit{{}-gadɨ} ‘until’}} & \multicolumn{4}{X}{{ \textit{{}-na} (PROH)}} & \multicolumn{4}{X}{{ \textit{kai} (DUB)}} & \multicolumn{6}{X}{}\\
\multicolumn{2}{X}{{ kˀuup-pa}} & \multicolumn{3}{X}{{ *kˀuup-poo}} & \multicolumn{4}{X}{{ *kˀuuk-kadɨ}} & \multicolumn{4}{X}{{ kˀuun-na}} & \multicolumn{4}{X}{{ *kˀuuk=kai}} & \multicolumn{6}{X}{}\\
\multicolumn{2}{X}{close-CSL} & \multicolumn{3}{X}{close-CND} & \multicolumn{4}{X}{close-until} & \multicolumn{4}{X}{close-PROH} & \multicolumn{4}{X}{close=DUB} & \multicolumn{6}{X}{}\\
\lspbottomrule
\end{tabularx}
Stem No. 2 (ending with //V\textsubscript{back}r//): \textit{koor-} ‘buy’

\tablefirsthead{}

\tabletail{}
\tablelasttail{}
\begin{tabularx}{\textwidth}{XXXXXXXXXXXXXXXXXXXXXXX}
\lsptoprule
\multicolumn{23}{X}{{\bfseries Type-A affixes}}\\
{ \textit{{}-an} (NEG)} & \multicolumn{5}{X}{{ \textit{{}-ar(ɨr)} (PASS)}} & \multicolumn{3}{X}{{ \textit{{}-ar(ɨr)} (CAP)}} & \multicolumn{4}{X}{{ \textit{{}-as} (CAUS)}} & \multicolumn{4}{X}{{ \textit{{}-azɨi} (NEG.PLQ)}} & \multicolumn{2}{X}{{ \textit{{}-ɨ} (IMP)}} & { \textit{{}-ɨba} (SUGS)} & \multicolumn{2}{X}{{ \textit{{}-oo}(INT)}} & \\
{ koor-an-ta} & \multicolumn{5}{X}{{ koor-at-ta}} & \multicolumn{3}{X}{{ koor-arɨk=kai}} & \multicolumn{4}{X}{{ koor-as-oo}} & \multicolumn{4}{X}{{ koor-azɨi}} & \multicolumn{2}{X}{{ koor-ɨ}} & { koor-ɨba} & \multicolumn{2}{X}{{ koor-oo}} & \\
buy-NEG-PST & \multicolumn{5}{X}{buy-PASS-PST} & \multicolumn{3}{X}{buy-CAP=DUB} & \multicolumn{4}{X}{buy-CAUS-INT} & \multicolumn{4}{X}{buy-NEG.PLQ} & \multicolumn{2}{X}{buy-IMP} & buy-SUGS & \multicolumn{2}{X}{buy-INT} & \\
\multicolumn{23}{X}{}\\
\multicolumn{23}{X}{{\bfseries Type-B affixes}}\\
\multicolumn{2}{X}{{ \textit{{}-tar} (PST)}} & \multicolumn{5}{X}{{ \textit{{}-tuk} (PRPR)}} & \multicolumn{4}{X}{{ \textit{{}-tur} (PROG)}} & \multicolumn{4}{X}{{ \textit{{}-təər} (RSL)}} & \multicolumn{3}{X}{{ \textit{{}-tɨ} (SEQ)}} & \multicolumn{3}{X}{{ \textit{{}-tai} (LST)}} & \multicolumn{2}{X}{{ \textit{{}-təəra} ‘after’}}\\
\multicolumn{2}{X}{{ koo-ta-n}} & \multicolumn{5}{X}{{ koo-tuk-an-boo}} & \multicolumn{4}{X}{{ koo-tut=too}} & \multicolumn{4}{X}{{ koo-tə-n}} & \multicolumn{3}{X}{{ koo-tɨ}} & \multicolumn{3}{X}{{ *koo-tai}} & \multicolumn{2}{X}{{ *koo-təəra}}\\
\multicolumn{2}{X}{buy-PST-PTCP} & \multicolumn{5}{X}{buy-PRPR-NEG-CND} & \multicolumn{4}{X}{buy-PROG=ASS} & \multicolumn{4}{X}{buy-RSL-PTCP} & \multicolumn{3}{X}{buy-SEQ} & \multicolumn{3}{X}{buy-LST} & \multicolumn{2}{X}{{ buy-after}}\\
\multicolumn{2}{X}{} & \multicolumn{5}{X}{} & \multicolumn{4}{X}{} & \multicolumn{4}{X}{} & \multicolumn{3}{X}{} & \multicolumn{3}{X}{} & \multicolumn{2}{X}{}\\
\multicolumn{23}{X}{{\bfseries Type-C affixes}}\\
\multicolumn{3}{X}{{ \textit{{}-jawur} (POL)}} & \multicolumn{2}{X}{{ \textit{{}-jaa} ‘person’}} & \multicolumn{5}{X}{{ \textit{{}-jur} (UMRK)}} & \multicolumn{4}{X}{{ \textit{{}-jagacinaa} (SIM)}} & \multicolumn{9}{X}{}\\
\multicolumn{3}{X}{{ *koo-jawu-i}} & \multicolumn{2}{X}{{ koo-jaa}} & \multicolumn{5}{X}{koo-ju-n} & \multicolumn{4}{X}{{ koo-jagacinaa}} & \multicolumn{9}{X}{}\\
\multicolumn{3}{X}{buy-POL-NPST} & \multicolumn{2}{X}{buy-person} & \multicolumn{5}{X}{buy-UMRK-PTCP} & \multicolumn{4}{X}{buy-SIM} & \multicolumn{9}{X}{}\\
\multicolumn{23}{X}{}\\
\multicolumn{23}{X}{{\bfseries Type-D affixes and clitic}}\\
{ \textit{{}-ba} (CSL)} & \multicolumn{3}{X}{{ \textit{{}-boo} (CND)}} & \multicolumn{4}{X}{{ \textit{{}-gadɨ} ‘until’}} & \multicolumn{4}{X}{{ \textit{{}-na} (PROH)}} & \multicolumn{4}{X}{{ \textit{kai} (DUB)}} & \multicolumn{7}{X}{}\\
{ koop-pa} & \multicolumn{3}{X}{{ *koop-poo}} & \multicolumn{4}{X}{{ *kook-kadɨ}} & \multicolumn{4}{X}{{ koon-na}} & \multicolumn{4}{X}{{ *kook=kai}} & \multicolumn{7}{X}{}\\
buy-CSL & \multicolumn{3}{X}{buy-CND} & \multicolumn{4}{X}{buy-until} & \multicolumn{4}{X}{buy-PROH} & \multicolumn{4}{X}{buy=DUB} & \multicolumn{7}{X}{}\\
\lspbottomrule
\end{tabularx}
Stem No. 2 (ending with //V\textsubscript{back}r//): \textit{tur-} ‘take’

\tablefirsthead{}

\tabletail{}
\tablelasttail{}
\begin{tabularx}{\textwidth}{XXXXXXXXXXXXXXXXXXXXXXX}
\lsptoprule
\multicolumn{23}{X}{{\bfseries Type-A affixes}}\\
{ \textit{{}-an} (NEG)} & \multicolumn{4}{X}{{ \textit{{}-ar(ɨr)} (PASS)}} & \multicolumn{4}{X}{{ \textit{{}-ar(ɨr)} (CAP)}} & \multicolumn{4}{X}{{ \textit{{}-as} (CAUS)}} & \multicolumn{4}{X}{{ \textit{{}-azɨi} (NEG.PLQ)}} & \multicolumn{2}{X}{{ \textit{{}-ɨ} (IMP)}} & { \textit{{}-ɨba} (SUGS)} & \multicolumn{2}{X}{{ \textit{{}-oo}(INT)}} & \\
{ tur-an} & \multicolumn{4}{X}{{ tur-arɨ-Ø}} & \multicolumn{4}{X}{{ tur-ar-an}} & \multicolumn{4}{X}{{ tur-as-an-tat-tu}} & \multicolumn{4}{X}{{ tur-azɨi}} & \multicolumn{2}{X}{{ tur-ɨ}} & { tur-ɨba} & \multicolumn{2}{X}{{ tur-oo}} & \\
take-NEG & \multicolumn{4}{X}{take-PASS-INF} & \multicolumn{4}{X}{take-CAP-NEG} & \multicolumn{4}{X}{take-CAUS-PST-CSL} & \multicolumn{4}{X}{take-NEG.PLQ} & \multicolumn{2}{X}{take-IMP} & take-SUGS & \multicolumn{2}{X}{take-INT} & \\
\multicolumn{23}{X}{}\\
\multicolumn{23}{X}{{\bfseries Type-B affixes}}\\
\multicolumn{3}{X}{{ \textit{{}-tar} (PST)}} & \multicolumn{5}{X}{{ \textit{{}-tuk} (PRPR)}} & \multicolumn{3}{X}{{ \textit{{}-tur} (PROG)}} & \multicolumn{4}{X}{{ \textit{{}-təər} (RSL)}} & \multicolumn{3}{X}{{ \textit{{}-tɨ} (SEQ)}} & \multicolumn{3}{X}{{ \textit{{}-tai} (LST)}} & \multicolumn{2}{X}{{ \textit{{}-təəra} ‘after’}}\\
\multicolumn{3}{X}{{ tu-ta}} & \multicolumn{5}{X}{{ tu-tuk-ii}} & \multicolumn{3}{X}{{ tu-tu-ta}} & \multicolumn{4}{X}{{ tu-təəp-pa}} & \multicolumn{3}{X}{{ tu-tɨ}} & \multicolumn{3}{X}{{ *tu-tai}} & \multicolumn{2}{X}{{ *tu-təəra}}\\
\multicolumn{3}{X}{take-PST} & \multicolumn{5}{X}{take-PRPR-INF} & \multicolumn{3}{X}{take-PROG-PST} & \multicolumn{4}{X}{take-RSL-CSL} & \multicolumn{3}{X}{take-SEQ} & \multicolumn{3}{X}{take-LST} & \multicolumn{2}{X}{{ take-after}}\\
\multicolumn{3}{X}{} & \multicolumn{5}{X}{} & \multicolumn{3}{X}{} & \multicolumn{4}{X}{} & \multicolumn{3}{X}{} & \multicolumn{3}{X}{} & \multicolumn{2}{X}{}\\
\multicolumn{23}{X}{{\bfseries Type-C affixes}}\\
\multicolumn{4}{X}{{ \textit{{}-jawur} (POL)}} & \multicolumn{3}{X}{{ \textit{{}-jaa} ‘person’}} & \multicolumn{3}{X}{{ \textit{{}-jur} (UMRK)}} & \multicolumn{4}{X}{{ \textit{{}-jagacinaa} (SIM)}} & \multicolumn{9}{X}{}\\
\multicolumn{4}{X}{{ *tu-jawu-i}} & \multicolumn{3}{X}{tu-jaa} & \multicolumn{3}{X}{tu-ju-n} & \multicolumn{4}{X}{{ tu-jagacinaa}} & \multicolumn{9}{X}{}\\
\multicolumn{4}{X}{take-POL-NPST} & \multicolumn{3}{X}{take-person} & \multicolumn{3}{X}{take-UMRK-PTCP} & \multicolumn{4}{X}{take-SIM} & \multicolumn{9}{X}{}\\
\multicolumn{23}{X}{}\\
\multicolumn{23}{X}{{\bfseries Type-D affixes and clitic}}\\
\multicolumn{2}{X}{{ \textit{{}-ba} (CSL)}} & \multicolumn{4}{X}{{ \textit{{}-boo} (CND)}} & \multicolumn{3}{X}{{ \textit{{}-gadɨ} ‘until’}} & \multicolumn{3}{X}{{ \textit{{}-na} (PROH)}} & \multicolumn{4}{X}{{ \textit{kai} (DUB)}} & \multicolumn{7}{X}{}\\
\multicolumn{2}{X}{{ tup-pa}} & \multicolumn{4}{X}{{ tup-poo}} & \multicolumn{3}{X}{{ *tuk-kadɨ}} & \multicolumn{3}{X}{{ tun-na}} & \multicolumn{4}{X}{{ *tuk=kai}} & \multicolumn{7}{X}{}\\
\multicolumn{2}{X}{take-CSL} & \multicolumn{4}{X}{take-CND} & \multicolumn{3}{X}{take-until} & \multicolumn{3}{X}{take-PROH} & \multicolumn{4}{X}{take=DUB} & \multicolumn{7}{X}{}\\
\lspbottomrule
\end{tabularx}
Stem No. 3 (ending with //pp//): \textit{app-} ‘play’

\tablefirsthead{}

\tabletail{}
\tablelasttail{}
\begin{tabularx}{\textwidth}{XXXXXXXXXXXXXXXXXXXXXXX}
\lsptoprule
\multicolumn{23}{X}{{\bfseries Type-A affixes}}\\
{ \textit{{}-an} (NEG)} & \multicolumn{6}{X}{{ \textit{{}-ar(ɨr)} (PASS)}} & \multicolumn{3}{X}{{ \textit{{}-ar(ɨr)} (CAP)}} & \multicolumn{4}{X}{{ \textit{{}-as} (CAUS)}} & \multicolumn{4}{X}{{ \textit{{}-azɨi} (NEG.PLQ)}} & \multicolumn{2}{X}{{ \textit{{}-ɨ} (IMP)}} & { \textit{{}-ɨba} (SUGS)} & { \textit{{}-oo}(INT)} & \\
{ app-an} & \multicolumn{6}{X}{{ app-at-tat-tu}} & \multicolumn{3}{X}{{ app-arɨ-n=nja}} & \multicolumn{4}{X}{{ app-as-an}} & \multicolumn{4}{X}{{ app-azɨi}} & \multicolumn{2}{X}{{ app-ɨ}} & { app-ɨba} & { app-oo} & \\
play-NEG & \multicolumn{6}{X}{play-PASS-PST-CSL} & \multicolumn{3}{X}{play-CAP-NPST=PLQ} & \multicolumn{4}{X}{play-CAUS-NEG} & \multicolumn{4}{X}{play-NEG.PLQ} & \multicolumn{2}{X}{play-IMP} & play-SUGS & play-INT & \\
\multicolumn{23}{X}{}\\
\multicolumn{23}{X}{{\bfseries Type-B affixes}}\\
{ \textit{{}-tar} (PST)} & \multicolumn{3}{X}{{ \textit{{}-tuk} (PRPR)}} & \multicolumn{5}{X}{{ \textit{{}-app} (PROG)}} & \multicolumn{4}{X}{{ \textit{{}-təər} (RSL)}} & \multicolumn{3}{X}{{ \textit{{}-tɨ} (SEQ)}} & \multicolumn{3}{X}{{ \textit{{}-tai} (LST)}} & \multicolumn{4}{X}{{ \textit{{}-təəra} ‘after’}}\\
{ at-ta} & \multicolumn{3}{X}{{ ?}} & \multicolumn{5}{X}{{ at-tur-ɨ}} & \multicolumn{4}{X}{{ at-tə-i}} & \multicolumn{3}{X}{{ at-tɨ}} & \multicolumn{3}{X}{{ *at-tai}} & \multicolumn{4}{X}{{ *at-təəra}}\\
play-PST & \multicolumn{3}{X}{} & \multicolumn{5}{X}{play-PROG-IMP} & \multicolumn{4}{X}{play-RSL-NPST} & \multicolumn{3}{X}{play-SEQ} & \multicolumn{3}{X}{play-LST} & \multicolumn{4}{X}{{ play-after}}\\
& \multicolumn{3}{X}{} & \multicolumn{5}{X}{} & \multicolumn{4}{X}{} & \multicolumn{3}{X}{} & \multicolumn{3}{X}{} & \multicolumn{4}{X}{}\\
\multicolumn{23}{X}{{\bfseries Type-C affixes}}\\
\multicolumn{3}{X}{{ \textit{{}-jawur} (POL)}} & \multicolumn{3}{X}{{ \textit{{}-jaa} ‘person’}} & \multicolumn{5}{X}{{ \textit{{}-jur} (UMRK)}} & \multicolumn{4}{X}{{ \textit{{}-jagacinaa} (SIM)}} & \multicolumn{8}{X}{}\\
\multicolumn{3}{X}{{ *app-jawu-i}} & \multicolumn{3}{X}{?} & \multicolumn{5}{X}{app-jur-u} & \multicolumn{4}{X}{{ app-jagacinaa}} & \multicolumn{8}{X}{}\\
\multicolumn{3}{X}{play-POL-NPST} & \multicolumn{3}{X}{} & \multicolumn{5}{X}{play-UMRK-PFC} & \multicolumn{4}{X}{play-SIM} & \multicolumn{8}{X}{}\\
\multicolumn{23}{X}{}\\
\multicolumn{23}{X}{{\bfseries Type-D affixes and clitic}}\\
\multicolumn{2}{X}{{ \textit{{}-ba} (CSL)}} & \multicolumn{3}{X}{{ \textit{{}-boo} (CND)}} & \multicolumn{3}{X}{{ \textit{{}-gadɨ} ‘until’}} & \multicolumn{4}{X}{{ \textit{{}-na} (PROH)}} & \multicolumn{5}{X}{{ \textit{kai} (DUB)}} & \multicolumn{6}{X}{}\\
\multicolumn{2}{X}{{ app-uba}} & \multicolumn{3}{X}{{ app-uboo}} & \multicolumn{3}{X}{{ *app-ugadɨ}} & \multicolumn{4}{X}{{ app-una}} & \multicolumn{5}{X}{{ *app=ukai}} & \multicolumn{6}{X}{}\\
\multicolumn{2}{X}{play-CSL} & \multicolumn{3}{X}{play-CND} & \multicolumn{3}{X}{play-until} & \multicolumn{4}{X}{play-PROH} & \multicolumn{5}{X}{play=DUB} & \multicolumn{6}{X}{}\\
\lspbottomrule
\end{tabularx}
Stem No. 4 (ending with //b//): \textit{narab-} ‘line up’

\tablefirsthead{}

\tabletail{}
\tablelasttail{}
\begin{tabularx}{\textwidth}{XXXXXXXXXm{-2.4015456E-4in}XXXXXXXXXXXX}
\lsptoprule
\multicolumn{22}{X}{{\bfseries Type-A affixes}}\\
{ \textit{{}-an} (NEG)} & \multicolumn{5}{X}{{ \textit{{}-ar(ɨr)} (PASS)}} & \multicolumn{3}{X}{{ \textit{{}-ar(ɨr)} (CAP)}} & \multicolumn{4}{X}{{ \textit{{}-as} (CAUS)}} & \multicolumn{4}{X}{{ \textit{{}-azɨi} (NEG.PLQ)}} & \multicolumn{2}{X}{{ \textit{{}-ɨ} (IMP)}} & \multicolumn{2}{X}{{ \textit{{}-ɨba} (SUGS)}} & { \textit{{}-oo}(INT)}\\
{ narab-an} & \multicolumn{5}{X}{{ narab-at-ta}} & \multicolumn{3}{X}{{ narab-arɨk=kai}} & \multicolumn{4}{X}{{ narab-as-oo}} & \multicolumn{4}{X}{{ narab-azɨi}} & \multicolumn{2}{X}{{ narab-ɨ}} & \multicolumn{2}{X}{{ narab-ɨba}} & { narab-oo}\\
line.up-NEG & \multicolumn{5}{X}{line.up-PASS-PST} & \multicolumn{3}{X}{line.up-CAP=DUB} & \multicolumn{4}{X}{line.up-CAUS-INT} & \multicolumn{4}{X}{line.up-NEG.PLQ} & \multicolumn{2}{X}{line.up-IMP} & \multicolumn{2}{X}{line.up-SUGS} & line.up-INT\\
\multicolumn{22}{X}{}\\
\multicolumn{22}{X}{{\bfseries Type-B affixes}}\\
\multicolumn{3}{X}{{ \textit{{}-tar} (PST)}} & \multicolumn{4}{X}{{ \textit{{}-tuk} (PRPR)}} & \multicolumn{4}{X}{{ \textit{{}-tur} (PROG)}} & \multicolumn{4}{X}{{ \textit{{}-təər} (RSL)}} & \multicolumn{3}{X}{{ \textit{{}-tɨ} (SEQ)}} & \multicolumn{2}{X}{{ \textit{{}-tai} (LST)}} & \multicolumn{2}{X}{{ \textit{{}-təəra} ‘after’}}\\
\multicolumn{3}{X}{{ nara-da}} & \multicolumn{4}{X}{{ nara-duk-ɨ}} & \multicolumn{4}{X}{{ nara-du-i}} & \multicolumn{4}{X}{{ nara-dəəp-pa}} & \multicolumn{3}{X}{{ *nara-dɨ}} & \multicolumn{2}{X}{{ *nara-dai}} & \multicolumn{2}{X}{{ *nara-dəəra}}\\
\multicolumn{3}{X}{line.up-PST} & \multicolumn{4}{X}{line.up-PRPR-IMP} & \multicolumn{4}{X}{line.up-PROG-NPST} & \multicolumn{4}{X}{line.up-RSL-CSL} & \multicolumn{3}{X}{line.up-SEQ} & \multicolumn{2}{X}{line.up-LST} & \multicolumn{2}{X}{{ line.up-after}}\\
\multicolumn{3}{X}{} & \multicolumn{4}{X}{} & \multicolumn{4}{X}{} & \multicolumn{4}{X}{} & \multicolumn{3}{X}{} & \multicolumn{2}{X}{} & \multicolumn{2}{X}{}\\
\multicolumn{22}{X}{{\bfseries Type-C affixes}}\\
\multicolumn{4}{X}{{ \textit{{}-jawur} (POL)}} & \multicolumn{2}{X}{{ \textit{{}-jaa} ‘person’}} & \multicolumn{4}{X}{{ \textit{{}-jur} (UMRK)}} & \multicolumn{4}{X}{{ \textit{{}-jagacinaa} (SIM)}} & \multicolumn{8}{X}{}\\
\multicolumn{4}{X}{{ *narab-jawu-i}} & \multicolumn{2}{X}{{ narab-jaa}} & \multicolumn{4}{X}{narab-ju-i} & \multicolumn{4}{X}{{ narab-jagacinaa}} & \multicolumn{8}{X}{}\\
\multicolumn{4}{X}{line.up-POL-NPST} & \multicolumn{2}{X}{line.up-person} & \multicolumn{4}{X}{line.up-UMRK-NPST} & \multicolumn{4}{X}{line.up-SIM} & \multicolumn{8}{X}{}\\
\multicolumn{22}{X}{}\\
\multicolumn{22}{X}{{\bfseries Type-D affixes and clitic}}\\
\multicolumn{2}{X}{{ \textit{{}-ba} (CSL)}} & \multicolumn{3}{X}{{ \textit{{}-boo} (CND)}} & \multicolumn{3}{X}{{ \textit{{}-gadɨ} ‘until’}} & \multicolumn{4}{X}{{ \textit{{}-na} (PROH)}} & \multicolumn{4}{X}{{ \textit{kai} (DUB)}} & \multicolumn{6}{X}{}\\
\multicolumn{2}{X}{{ narab-uba}} & \multicolumn{3}{X}{{ narab-uboo}} & \multicolumn{3}{X}{{ *narab-ugadɨ}} & \multicolumn{4}{X}{{ narab-una}} & \multicolumn{4}{X}{{ *narab=ukai}} & \multicolumn{6}{X}{}\\
\multicolumn{2}{X}{line.up-CSL} & \multicolumn{3}{X}{line.up-CND} & \multicolumn{3}{X}{line.up-until} & \multicolumn{4}{X}{line.up-PROH} & \multicolumn{4}{X}{line.up=DUB} & \multicolumn{6}{X}{}\\
\lspbottomrule
\end{tabularx}
Stem No. 5 (ending with //Vm//): \textit{jum-} ‘read’

\tablefirsthead{}

\tabletail{}
\tablelasttail{}
\begin{tabularx}{\textwidth}{XXXXXXXXXXXXXXXXXXXXXXX}
\lsptoprule
\multicolumn{23}{X}{{\bfseries Type-A affixes}}\\
{ \textit{{}-an} (NEG)} & \multicolumn{4}{X}{{ \textit{{}-ar(ɨr)} (PASS)}} & \multicolumn{5}{X}{{ \textit{{}-ar(ɨr)} (CAP)}} & \multicolumn{4}{X}{{ \textit{{}-as} (CAUS)}} & \multicolumn{3}{X}{{ \textit{{}-azɨi} (NEG.PLQ)}} & { \textit{{}-ɨ} (IMP)} & \multicolumn{2}{X}{{ \textit{{}-ɨba} (SUGS)}} & \multicolumn{2}{X}{{ \textit{{}-oo}(INT)}} & \\
{ jum-an} & \multicolumn{4}{X}{{ jum-at-ta}} & \multicolumn{5}{X}{{ jum-arɨ-i}} & \multicolumn{4}{X}{{ jum-as-oo}} & \multicolumn{3}{X}{{ jum-azɨi}} & { jum-ɨ} & \multicolumn{2}{X}{{ jum-ba}} & \multicolumn{2}{X}{{ jum-oo}} & \\
read-NEG & \multicolumn{4}{X}{read-PASS-PST} & \multicolumn{5}{X}{read-CAP-NPST} & \multicolumn{4}{X}{read-CAUS-INT} & \multicolumn{3}{X}{read-NEG.PLQ} & read-IMP & \multicolumn{2}{X}{read-SUGS} & \multicolumn{2}{X}{read-INT} & \\
\multicolumn{23}{X}{}\\
\multicolumn{23}{X}{{\bfseries Type-B affixes}}\\
\multicolumn{3}{X}{{ \textit{{}-tar} (PST)}} & \multicolumn{5}{X}{{ \textit{{}-tuk} (PRPR)}} & \multicolumn{4}{X}{{ \textit{{}-tur} (PROG)}} & \multicolumn{4}{X}{{ \textit{{}-təər} (RSL)}} & \multicolumn{3}{X}{{ \textit{{}-tɨ} (SEQ)}} & \multicolumn{2}{X}{{ \textit{{}-tai} (LST)}} & \multicolumn{2}{X}{{ \textit{{}-təəra} ‘after’}}\\
\multicolumn{3}{X}{{ ju-da}} & \multicolumn{5}{X}{{ ju-duk-ɨba}} & \multicolumn{4}{X}{{ ju-dur-ɨba}} & \multicolumn{4}{X}{{ ju-dəəp-pa}} & \multicolumn{3}{X}{{ *ju-dɨ}} & \multicolumn{2}{X}{{ *ju-dai}} & \multicolumn{2}{X}{{ *ju-dəəra}}\\
\multicolumn{3}{X}{read-PST} & \multicolumn{5}{X}{read-PRPR-SUGS} & \multicolumn{4}{X}{read-PROG-SUGS} & \multicolumn{4}{X}{read-RSL-CSL} & \multicolumn{3}{X}{read-SEQ} & \multicolumn{2}{X}{read-LST} & \multicolumn{2}{X}{{ read-after}}\\
\multicolumn{3}{X}{} & \multicolumn{5}{X}{} & \multicolumn{4}{X}{} & \multicolumn{4}{X}{} & \multicolumn{3}{X}{} & \multicolumn{2}{X}{} & \multicolumn{2}{X}{}\\
\multicolumn{23}{X}{{\bfseries Type-C affixes}}\\
\multicolumn{4}{X}{{ \textit{{}-jawur} (POL)}} & \multicolumn{3}{X}{{ \textit{{}-jaa} ‘person’}} & \multicolumn{4}{X}{{ \textit{{}-jur} (UMRK)}} & \multicolumn{4}{X}{{ \textit{{}-jagacinaa} (SIM)}} & \multicolumn{8}{X}{}\\
\multicolumn{4}{X}{{ *jum-jawu-i}} & \multicolumn{3}{X}{{ jum-jaa}} & \multicolumn{4}{X}{{ jum-ju-n}} & \multicolumn{4}{X}{{ jum-jagacinaa}} & \multicolumn{8}{X}{}\\
\multicolumn{4}{X}{read-POL-NPST} & \multicolumn{3}{X}{read-person} & \multicolumn{4}{X}{read-UMRK-PTCP} & \multicolumn{4}{X}{read-SIM} & \multicolumn{8}{X}{}\\
\multicolumn{23}{X}{}\\
\multicolumn{23}{X}{{\bfseries Type-D affixes and clitic}}\\
\multicolumn{2}{X}{{ \textit{{}-ba} (CSL)}} & \multicolumn{4}{X}{{ \textit{{}-boo} (CND)}} & \multicolumn{3}{X}{{ \textit{{}-gadɨ} ‘until’}} & \multicolumn{4}{X}{{ \textit{{}-na} (PROH)}} & \multicolumn{4}{X}{{ \textit{kai} (DUB)}} & \multicolumn{6}{X}{}\\
\multicolumn{2}{X}{{ jum-ba}} & \multicolumn{4}{X}{{ jum-boo}} & \multicolumn{3}{X}{{ jum-gadɨ}} & \multicolumn{4}{X}{{ jum-na}} & \multicolumn{4}{X}{{ *jum=kai}} & \multicolumn{6}{X}{}\\
\multicolumn{2}{X}{read-CSL} & \multicolumn{4}{X}{read-CND} & \multicolumn{3}{X}{read-until} & \multicolumn{4}{X}{read-PROH} & \multicolumn{4}{X}{read=DUB} & \multicolumn{6}{X}{}\\
\lspbottomrule
\end{tabularx}
Stem No. 6 (ending with //nm//): \textit{tanm-} ‘ask’

\tablefirsthead{}

\tabletail{}
\tablelasttail{}
\begin{tabularx}{\textwidth}{XXXXXXXXXXXXXXXXXXXXX}
\lsptoprule
\multicolumn{21}{X}{{\bfseries Type-A affixes}}\\
{ \textit{{}-an} (NEG)} & \multicolumn{4}{X}{{ \textit{{}-ar(ɨr)} (PASS)}} & \multicolumn{4}{X}{{ \textit{{}-ar(ɨr)} (CAP)}} & \multicolumn{4}{X}{{ \textit{{}-as} (CAUS)}} & \multicolumn{3}{X}{{ \textit{{}-azɨi} (NEG.PLQ)}} & \multicolumn{2}{X}{{ \textit{{}-ɨ} (IMP)}} & { \textit{{}-ɨba} (SUGS)} & { \textit{{}-oo}(INT)} & \\
{ tanm-an} & \multicolumn{4}{X}{{ tanm-ar-ɨ}} & \multicolumn{4}{X}{{ tanm-ar-an}} & \multicolumn{4}{X}{{ tanm-as-oo}} & \multicolumn{3}{X}{{ tanm-azɨi}} & \multicolumn{2}{X}{{ tanm-ɨ}} & { tanm-ɨba} & { tanm-oo} & \\
ask-NEG & \multicolumn{4}{X}{ask-PASS-IMP} & \multicolumn{4}{X}{ask-CAP-NEG} & \multicolumn{4}{X}{ask-CAUS-INT} & \multicolumn{3}{X}{ask-NEG.PLQ} & \multicolumn{2}{X}{ask-IMP} & ask-SUGS & ask-INT & \\
\multicolumn{21}{X}{}\\
\multicolumn{21}{X}{{\bfseries Type-B affixes}}\\
\multicolumn{2}{X}{{ \textit{{}-tar} (PST)}} & \multicolumn{3}{X}{{ \textit{{}-tuk} (PRPR)}} & \multicolumn{4}{X}{{ \textit{{}-tur} (PROG)}} & \multicolumn{3}{X}{{ \textit{{}-təər} (RSL)}} & \multicolumn{3}{X}{{ \textit{{}-tɨ} (SEQ)}} & \multicolumn{2}{X}{{ \textit{{}-tai} (LST)}} & \multicolumn{4}{X}{{ \textit{{}-təəra} ‘after’}}\\
\multicolumn{2}{X}{{ tan-da}} & \multicolumn{3}{X}{{ tan-duk-ɨ}} & \multicolumn{4}{X}{{ tan-du-tɨ}} & \multicolumn{3}{X}{{ tan-də-i}} & \multicolumn{3}{X}{{ *tan-dɨ}} & \multicolumn{2}{X}{{ *tan-dai}} & \multicolumn{4}{X}{{ *tan-dəəra}}\\
\multicolumn{2}{X}{ask-PST} & \multicolumn{3}{X}{ask-PRPR-IMP} & \multicolumn{4}{X}{ask-PROG-SEQ} & \multicolumn{3}{X}{ask-RSL-NPST} & \multicolumn{3}{X}{ask-SEQ} & \multicolumn{2}{X}{ask-LST} & \multicolumn{4}{X}{{ ask-after}}\\
\multicolumn{2}{X}{} & \multicolumn{3}{X}{} & \multicolumn{4}{X}{} & \multicolumn{3}{X}{} & \multicolumn{3}{X}{} & \multicolumn{2}{X}{} & \multicolumn{4}{X}{}\\
\multicolumn{21}{X}{{\bfseries Type-C affixes}}\\
\multicolumn{4}{X}{{ \textit{{}-jawur} (POL)}} & \multicolumn{3}{X}{{ \textit{{}-jaa} ‘person’}} & \multicolumn{3}{X}{{ \textit{{}-jur} (UMRK)}} & \multicolumn{4}{X}{{ \textit{{}-jagacinaa} (SIM)}} & \multicolumn{7}{X}{}\\
\multicolumn{4}{X}{{ *tanm-jawu-i}} & \multicolumn{3}{X}{{ ?}} & \multicolumn{3}{X}{tanm-jut=too} & \multicolumn{4}{X}{{ tanm-jagacinaa}} & \multicolumn{7}{X}{}\\
\multicolumn{4}{X}{ask-POL-NPST} & \multicolumn{3}{X}{} & \multicolumn{3}{X}{ask-UMRK=ASS} & \multicolumn{4}{X}{ask-SIM} & \multicolumn{7}{X}{}\\
\multicolumn{21}{X}{}\\
\multicolumn{21}{X}{{\bfseries Type-D affixes and clitic}}\\
\multicolumn{3}{X}{{ \textit{{}-ba} (CSL)}} & \multicolumn{3}{X}{{ \textit{{}-boo} (CND)}} & \multicolumn{2}{X}{{ \textit{{}-gadɨ} ‘until’}} & \multicolumn{3}{X}{{ \textit{{}-na} (PROH)}} & \multicolumn{5}{X}{{ \textit{kai} (DUB)}} & \multicolumn{5}{X}{}\\
\multicolumn{3}{X}{{ tanm-uba}} & \multicolumn{3}{X}{{ *tanm-uboo}} & \multicolumn{2}{X}{{ *tanm-ugadɨ}} & \multicolumn{3}{X}{{ tanm-una}} & \multicolumn{5}{X}{{ *tanm=ukai}} & \multicolumn{5}{X}{}\\
\multicolumn{3}{X}{ask-CSL} & \multicolumn{3}{X}{ask-CND} & \multicolumn{2}{X}{ask-until} & \multicolumn{3}{X}{ask-PROH} & \multicolumn{5}{X}{ask=DUB} & \multicolumn{5}{X}{}\\
\lspbottomrule
\end{tabularx}
Stem No. 7 (ending with //V\textsubscript{non-}\textit{\textsubscript{i} }k//): \textit{kak-} ‘write’

\tablefirsthead{}

\tabletail{}
\tablelasttail{}
\begin{tabularx}{\textwidth}{XXXXXXm{-2.4015456E-4in}XXXXXXXXXXXXXXXX}
\lsptoprule
\multicolumn{23}{X}{{\bfseries Type-A affixes}}\\
\multicolumn{3}{X}{{ \textit{{}-an} (NEG)}} & \multicolumn{4}{X}{{ \textit{{}-ar(ɨr)} (PASS)}} & \multicolumn{3}{X}{{ \textit{{}-ar(ɨr)} (CAP)}} & \multicolumn{5}{X}{{ \textit{{}-as} (CAUS)}} & \multicolumn{3}{X}{{ \textit{{}-azɨi} (NEG.PLQ)}} & \multicolumn{2}{X}{{ \textit{{}-ɨ} (IMP)}} & \multicolumn{2}{X}{{ \textit{{}-ɨba} (SUGS)}} & { \textit{{}-oo}(INT)}\\
\multicolumn{3}{X}{{ kak-an-ta}} & \multicolumn{4}{X}{{ kak-at-ta}} & \multicolumn{3}{X}{{ kak-arɨk=kai}} & \multicolumn{5}{X}{{ kak-as-i+gjaa}} & \multicolumn{3}{X}{{ kak-azɨi}} & \multicolumn{2}{X}{{ kak-ɨ}} & \multicolumn{2}{X}{{ kak-ɨba}} & { kak-oo}\\
\multicolumn{3}{X}{write-NEG-PST} & \multicolumn{4}{X}{write-PASS-PST} & \multicolumn{3}{X}{write-CAP=DUB} & \multicolumn{5}{X}{write-CAUS-INF+PURP} & \multicolumn{3}{X}{write-NEG.PLQ} & \multicolumn{2}{X}{write-IMP} & \multicolumn{2}{X}{write-SUGS} & write-INT\\
\multicolumn{23}{X}{}\\
\multicolumn{23}{X}{{\bfseries Type-B affixes}}\\
\multicolumn{2}{X}{{ \textit{{}-tar} (PST)}} & \multicolumn{6}{X}{{ \textit{{}-tuk} (PRPR)}} & \multicolumn{4}{X}{{ \textit{{}-tur} (PROG)}} & \multicolumn{4}{X}{{ \textit{{}-təər} (RSL)}} & \multicolumn{3}{X}{{ \textit{{}-tɨ} (SEQ)}} & \multicolumn{2}{X}{{ \textit{{}-tai} (LST)}} & \multicolumn{2}{X}{{ \textit{{}-təəra} ‘after’}}\\
\multicolumn{2}{X}{{ ka-cja}} & \multicolumn{6}{X}{{ ka-cjuk-ɨ}} & \multicolumn{4}{X}{{ ka-cjur-an-ta}} & \multicolumn{4}{X}{{ ka-cjə-i}} & \multicolumn{3}{X}{{ ka-cjɨ}} & \multicolumn{2}{X}{{ *ka-cjai}} & \multicolumn{2}{X}{{ *ka-cjəəra}}\\
\multicolumn{2}{X}{write-PST} & \multicolumn{6}{X}{write-PRPR-IMP} & \multicolumn{4}{X}{write-PROG-NEG-PST} & \multicolumn{4}{X}{write-RSL-NPST} & \multicolumn{3}{X}{write-SEQ} & \multicolumn{2}{X}{write-LST} & \multicolumn{2}{X}{{ write-after}}\\
\multicolumn{2}{X}{} & \multicolumn{6}{X}{} & \multicolumn{4}{X}{} & \multicolumn{4}{X}{} & \multicolumn{3}{X}{} & \multicolumn{2}{X}{} & \multicolumn{2}{X}{}\\
\multicolumn{23}{X}{{\bfseries Type-C affixes}}\\
\multicolumn{4}{X}{{ \textit{{}-jawur} (POL)}} & \multicolumn{2}{X}{{ \textit{{}-jaa} ‘person’}} & \multicolumn{5}{X}{{ \textit{{}-jur} (UMRK)}} & \multicolumn{3}{X}{{ \textit{{}-jagacinaa} (SIM)}} & \multicolumn{9}{X}{}\\
\multicolumn{4}{X}{{ *kak-jawu-i}} & \multicolumn{2}{X}{kak-jaa} & \multicolumn{5}{X}{kak-ju-mɨ} & \multicolumn{3}{X}{{ kak-jagacinaa}} & \multicolumn{9}{X}{}\\
\multicolumn{4}{X}{write-POL-NPST} & \multicolumn{2}{X}{write-person} & \multicolumn{5}{X}{write-UMRK-PLQ} & \multicolumn{3}{X}{write-SIM} & \multicolumn{9}{X}{}\\
\multicolumn{23}{X}{}\\
\multicolumn{23}{X}{{\bfseries Type-D affixes and clitic}}\\
{ \textit{{}-ba} (CSL)} & \multicolumn{4}{X}{{ \textit{{}-boo} (CND)}} & \multicolumn{4}{X}{{ \textit{{}-gadɨ} ‘until’}} & \multicolumn{4}{X}{{ \textit{{}-na} (PROH)}} & \multicolumn{4}{X}{{ \textit{kai} (DUB)}} & \multicolumn{6}{X}{}\\
{ kak-uba} & \multicolumn{4}{X}{{ kak-uboo}} & \multicolumn{4}{X}{{ kak-ugadɨ}} & \multicolumn{4}{X}{{ kak-una}} & \multicolumn{4}{X}{{ kak=ukai}} & \multicolumn{6}{X}{}\\
write-CSL & \multicolumn{4}{X}{write-CND} & \multicolumn{4}{X}{write-until} & \multicolumn{4}{X}{write-PROH} & \multicolumn{4}{X}{write=DUB} & \multicolumn{6}{X}{}\\
\lspbottomrule
\end{tabularx}
Stem No. 8 (ending with //V\textsubscript{non-}\textit{\textsubscript{i} }kk//): \textit{sukk-} ‘pull’

\tablefirsthead{}

\tabletail{}
\tablelasttail{}
\begin{tabularx}{\textwidth}{XXXXXXXXXXXXXXXXXXXXXXXX}
\lsptoprule
\multicolumn{24}{X}{{\bfseries Type-A affixes}}\\
{ \textit{{}-an} (NEG)} & \multicolumn{4}{X}{{ \textit{{}-ar(ɨr)} (PASS)}} & \multicolumn{5}{X}{{ \textit{{}-ar(ɨr)} (CAP)}} & \multicolumn{4}{X}{{ \textit{{}-as} (CAUS)}} & \multicolumn{4}{X}{{ \textit{{}-azɨi} (NEG.PLQ)}} & \multicolumn{2}{X}{{ \textit{{}-ɨ} (IMP)}} & \multicolumn{2}{X}{{ \textit{{}-ɨba} (SUGS)}} & { \textit{{}-oo}(INT)} & \\
{ sukk-an} & \multicolumn{4}{X}{{ *sukk-arɨ-i}} & \multicolumn{5}{X}{{ *sukk-arɨ-i}} & \multicolumn{4}{X}{{ *sukk-as-oo}} & \multicolumn{4}{X}{{ *sukk-azɨi}} & \multicolumn{2}{X}{{ *sukk-ɨ}} & \multicolumn{2}{X}{{ *sukk-ɨba}} & { *sukk-oo} & \\
pull-NEG & \multicolumn{4}{X}{pull-PASS-NPST} & \multicolumn{5}{X}{pull-CAP-NPST} & \multicolumn{4}{X}{pull-CAUS-INT} & \multicolumn{4}{X}{pull-NEG.PLQ} & \multicolumn{2}{X}{pull-IMP} & \multicolumn{2}{X}{pull-SUGS} & pull-INT & \\
\multicolumn{24}{X}{}\\
\multicolumn{24}{X}{{\bfseries Type-B affixes}}\\
\multicolumn{3}{X}{{ \textit{{}-tar} (PST)}} & \multicolumn{5}{X}{{ \textit{{}-tuk} (PRPR)}} & \multicolumn{3}{X}{{ \textit{{}-sukk} (PROG)}} & \multicolumn{4}{X}{{ \textit{{}-təər} (RSL)}} & \multicolumn{2}{X}{{ \textit{{}-tɨ} (SEQ)}} & \multicolumn{4}{X}{{ \textit{{}-tai} (LST)}} & \multicolumn{3}{X}{{ \textit{{}-təəra} ‘after’}}\\
\multicolumn{3}{X}{{ *suc-cja}} & \multicolumn{5}{X}{{ *suc-cjuk-ɨ}} & \multicolumn{3}{X}{{ *suc-cju-i}} & \multicolumn{4}{X}{{ *suc-cjə-i}} & \multicolumn{2}{X}{{ suc-cjɨ}} & \multicolumn{4}{X}{{ *suc-cjai}} & \multicolumn{3}{X}{{ *suc-cjəəra}}\\
\multicolumn{3}{X}{pull-PST} & \multicolumn{5}{X}{pull-PRPR-IMP} & \multicolumn{3}{X}{pull-PROG-NPST} & \multicolumn{4}{X}{pull-RSL-NPST} & \multicolumn{2}{X}{pull-SEQ} & \multicolumn{4}{X}{pull-LST} & \multicolumn{3}{X}{{ pull-after}}\\
\multicolumn{3}{X}{} & \multicolumn{5}{X}{} & \multicolumn{3}{X}{} & \multicolumn{4}{X}{} & \multicolumn{2}{X}{} & \multicolumn{4}{X}{} & \multicolumn{3}{X}{}\\
\multicolumn{24}{X}{{\bfseries Type-C affixes}}\\
\multicolumn{4}{X}{{ \textit{{}-jawur} (POL)}} & \multicolumn{3}{X}{{ \textit{{}-jaa} ‘person’}} & \multicolumn{5}{X}{{ \textit{{}-jur} (UMRK)}} & \multicolumn{4}{X}{{ \textit{{}-jagacinaa} (SIM)}} & \multicolumn{8}{X}{}\\
\multicolumn{4}{X}{{ *sukk-jawu-i}} & \multicolumn{3}{X}{*sukk-jaa} & \multicolumn{5}{X}{sukk-ju-i} & \multicolumn{4}{X}{{ *sukk-jagacinaa}} & \multicolumn{8}{X}{}\\
\multicolumn{4}{X}{pull-POL-NPST} & \multicolumn{3}{X}{pull-person} & \multicolumn{5}{X}{pull-UMRK-NPST} & \multicolumn{4}{X}{pull-SIM} & \multicolumn{8}{X}{}\\
\multicolumn{24}{X}{}\\
\multicolumn{24}{X}{{\bfseries Type-D affixes and clitic}}\\
\multicolumn{2}{X}{{ \textit{{}-ba} (CSL)}} & \multicolumn{4}{X}{{ \textit{{}-boo} (CND)}} & \multicolumn{3}{X}{{ \textit{{}-gadɨ} ‘until’}} & \multicolumn{4}{X}{{ \textit{{}-na} (PROH)}} & \multicolumn{6}{X}{{ \textit{kai} (DUB)}} & \multicolumn{5}{X}{}\\
\multicolumn{2}{X}{{ *sukk-uba}} & \multicolumn{4}{X}{{ *sukk-uboo}} & \multicolumn{3}{X}{{ *sukk-ugadɨ}} & \multicolumn{4}{X}{{ sukk-una}} & \multicolumn{6}{X}{{ *sukk=ukai}} & \multicolumn{5}{X}{}\\
\multicolumn{2}{X}{pull-CSL} & \multicolumn{4}{X}{pull-CND} & \multicolumn{3}{X}{pull-until} & \multicolumn{4}{X}{pull-PROH} & \multicolumn{6}{X}{pull=DUB} & \multicolumn{5}{X}{}\\
\lspbottomrule
\end{tabularx}
Stem No. 9 (ending with //Vs//): \textit{us-} ‘push’

\tablefirsthead{}

\tabletail{}
\tablelasttail{}
\begin{tabularx}{\textwidth}{XXm{4.5984238E-4in}XXXXXXXXXXXXXXXXXXXX}
\lsptoprule
\multicolumn{23}{X}{{\bfseries Type-A affixes}}\\
\multicolumn{3}{X}{{ \textit{{}-an} (NEG)}} & \multicolumn{3}{X}{{ \textit{{}-ar(ɨr)} (PASS)}} & \multicolumn{4}{X}{{ \textit{{}-ar(ɨr)} (CAP)}} & \multicolumn{4}{X}{{ \textit{{}-as} (CAUS)}} & \multicolumn{3}{X}{{ \textit{{}-azɨi} (NEG.PLQ)}} & \multicolumn{2}{X}{{ \textit{{}-ɨ} (IMP)}} & \multicolumn{2}{X}{{ \textit{{}-ɨba} (SUGS)}} & { \textit{{}-oo}(INT)} & \\
\multicolumn{3}{X}{{ us-an-boo}} & \multicolumn{3}{X}{{ us-at-ta}} & \multicolumn{4}{X}{{ us-arɨk=kai}} & \multicolumn{4}{X}{{ us-as-oo}} & \multicolumn{3}{X}{{ us-azɨi}} & \multicolumn{2}{X}{{ us-ɨ}} & \multicolumn{2}{X}{{ us-ɨba}} & { us-oo} & \\
\multicolumn{3}{X}{push-NEG-CND} & \multicolumn{3}{X}{push-PASS-PST} & \multicolumn{4}{X}{push-CAP=DUB} & \multicolumn{4}{X}{push-CAUS-INT} & \multicolumn{3}{X}{push-NEG.PLQ} & \multicolumn{2}{X}{push-IMP} & \multicolumn{2}{X}{push-SUGS} & push-INT & \\
\multicolumn{23}{X}{}\\
\multicolumn{23}{X}{{\bfseries Type-B affixes}}\\
\multicolumn{2}{X}{{ \textit{{}-tar} (PST)}} & \multicolumn{6}{X}{{ \textit{{}-tuk} (PRPR)}} & \multicolumn{3}{X}{{ \textit{{}-tur} (PROG)}} & \multicolumn{4}{X}{{ \textit{{}-təər} (RSL)}} & \multicolumn{3}{X}{{ \textit{{}-tɨ} (SEQ)}} & \multicolumn{2}{X}{{ \textit{{}-tai} (LST)}} & \multicolumn{3}{X}{{ \textit{{}-təəra} ‘after’}}\\
\multicolumn{2}{X}{{ u-cja}} & \multicolumn{6}{X}{{ u-cjuk-ɨ}} & \multicolumn{3}{X}{{ u-cjut=too}} & \multicolumn{4}{X}{{ u-cjəəp-pa}} & \multicolumn{3}{X}{{ *u-cjɨ}} & \multicolumn{2}{X}{{ *u-cjai}} & \multicolumn{3}{X}{{ *u-cjəəra}}\\
\multicolumn{2}{X}{push-PST} & \multicolumn{6}{X}{push-PRPR-IMP} & \multicolumn{3}{X}{push-PROG=ASS} & \multicolumn{4}{X}{push-RSL-CSL} & \multicolumn{3}{X}{push-SEQ} & \multicolumn{2}{X}{push-LST} & \multicolumn{3}{X}{{ push-after}}\\
\multicolumn{2}{X}{} & \multicolumn{6}{X}{} & \multicolumn{3}{X}{} & \multicolumn{4}{X}{} & \multicolumn{3}{X}{} & \multicolumn{2}{X}{} & \multicolumn{3}{X}{}\\
\multicolumn{23}{X}{{\bfseries Type-C affixes}}\\
\multicolumn{4}{X}{{ \textit{{}-jawur} (POL)}} & \multicolumn{3}{X}{{ \textit{{}-jaa} ‘person’}} & \multicolumn{5}{X}{{ \textit{{}-jur} (UMRK)}} & \multicolumn{4}{X}{{ \textit{{}-jagacinaa} (SIM)}} & \multicolumn{7}{X}{}\\
\multicolumn{4}{X}{{ *us-jawu-i}} & \multicolumn{3}{X}{us-jaa} & \multicolumn{5}{X}{us-jut=too} & \multicolumn{4}{X}{{ us-jagacinaa}} & \multicolumn{7}{X}{}\\
\multicolumn{4}{X}{push-POL-NPST} & \multicolumn{3}{X}{push-person} & \multicolumn{5}{X}{push-UMRK=ASS} & \multicolumn{4}{X}{push-SIM} & \multicolumn{7}{X}{}\\
\multicolumn{23}{X}{}\\
\multicolumn{23}{X}{{\bfseries Type-D affixes and clitic}}\\
{ \textit{{}-ba} (CSL)} & \multicolumn{4}{X}{{ \textit{{}-boo} (CND)}} & \multicolumn{4}{X}{{ \textit{{}-gadɨ} ‘until’}} & \multicolumn{4}{X}{{ \textit{{}-na} (PROH)}} & \multicolumn{4}{X}{{ \textit{kai} (DUB)}} & \multicolumn{6}{X}{}\\
{ us-ɨba} & \multicolumn{4}{X}{{ us-ɨboo}} & \multicolumn{4}{X}{{ *us-ɨgadɨ}} & \multicolumn{4}{X}{{ us-ɨna}} & \multicolumn{4}{X}{{ *us=ɨkai}} & \multicolumn{6}{X}{}\\
push-CSL & \multicolumn{4}{X}{push-CND} & \multicolumn{4}{X}{push-until} & \multicolumn{4}{X}{push-PROH} & \multicolumn{4}{X}{push=DUB} & \multicolumn{6}{X}{}\\
\lspbottomrule
\end{tabularx}
Stem No. 10 (ending with //ss//): \textit{kuss-} ‘kill’

\tablefirsthead{}

\tabletail{}
\tablelasttail{}
\begin{tabularx}{\textwidth}{XXm{4.5984238E-4in}XXXXXXXXXXXXXXXXXXXX}
\lsptoprule
\multicolumn{23}{X}{{\bfseries Type-A affixes}}\\
\multicolumn{3}{X}{{ \textit{{}-an} (NEG)}} & \multicolumn{3}{X}{{ \textit{{}-ar(ɨr)} (PASS)}} & \multicolumn{4}{X}{{ \textit{{}-ar(ɨr)} (CAP)}} & \multicolumn{4}{X}{{ \textit{{}-as} (CAUS)}} & \multicolumn{3}{X}{{ \textit{{}-azɨi} (NEG.PLQ)}} & \multicolumn{2}{X}{{ \textit{{}-ɨ} (IMP)}} & \multicolumn{2}{X}{{ \textit{{}-ɨba} (SUGS)}} & { \textit{{}-oo}(INT)} & \\
\multicolumn{3}{X}{{ kuss-an}} & \multicolumn{3}{X}{{ kuss-at-ta}} & \multicolumn{4}{X}{{ kuss-ar-an}} & \multicolumn{4}{X}{{ kuss-as-oo}} & \multicolumn{3}{X}{{ kuss-azɨi}} & \multicolumn{2}{X}{{ kuss-ɨ}} & \multicolumn{2}{X}{{ kuss-ɨba}} & { kuss-oo} & \\
\multicolumn{3}{X}{kill-NEG} & \multicolumn{3}{X}{kill-PASS-PST} & \multicolumn{4}{X}{kill-CAP-NEG} & \multicolumn{4}{X}{kill-CAUS-INT} & \multicolumn{3}{X}{kill-NEG.PLQ} & \multicolumn{2}{X}{kill-IMP} & \multicolumn{2}{X}{kill-SUGS} & kill-INT & \\
\multicolumn{23}{X}{}\\
\multicolumn{23}{X}{{\bfseries Type-B affixes}}\\
\multicolumn{2}{X}{{ \textit{{}-tar} (PST)}} & \multicolumn{6}{X}{{ \textit{{}-tuk} (PRPR)}} & \multicolumn{3}{X}{{ \textit{{}-tur} (PROG)}} & \multicolumn{4}{X}{{ \textit{{}-təər} (RSL)}} & \multicolumn{3}{X}{{ \textit{{}-tɨ} (SEQ)}} & \multicolumn{2}{X}{{ \textit{{}-tai} (LST)}} & \multicolumn{3}{X}{{ \textit{{}-təəra} ‘after’}}\\
\multicolumn{2}{X}{{ kuc-cja}} & \multicolumn{6}{X}{{ kuc-cjuk-ɨ}} & \multicolumn{3}{X}{{ kuc-cju-i}} & \multicolumn{4}{X}{{ kuc-cjə-i}} & \multicolumn{3}{X}{{ *kuc-cjɨ}} & \multicolumn{2}{X}{{ *kuc-cjai}} & \multicolumn{3}{X}{{ *kuc-cjəəra}}\\
\multicolumn{2}{X}{kill-PST} & \multicolumn{6}{X}{kill-PRPR-IMP} & \multicolumn{3}{X}{kill-PROG-NPST} & \multicolumn{4}{X}{kill-RSL-NPST} & \multicolumn{3}{X}{kill-SEQ} & \multicolumn{2}{X}{kill-LST} & \multicolumn{3}{X}{{ kill-after}}\\
\multicolumn{2}{X}{} & \multicolumn{6}{X}{} & \multicolumn{3}{X}{} & \multicolumn{4}{X}{} & \multicolumn{3}{X}{} & \multicolumn{2}{X}{} & \multicolumn{3}{X}{}\\
\multicolumn{23}{X}{{\bfseries Type-C affixes}}\\
\multicolumn{4}{X}{{ \textit{{}-jawur} (POL)}} & \multicolumn{3}{X}{{ \textit{{}-jaa} ‘person’}} & \multicolumn{5}{X}{{ \textit{{}-jur} (UMRK)}} & \multicolumn{4}{X}{{ \textit{{}-jagacinaa} (SIM)}} & \multicolumn{7}{X}{}\\
\multicolumn{4}{X}{{ *kuss-jawu-i}} & \multicolumn{3}{X}{kuss-jaa} & \multicolumn{5}{X}{{ kuss-jur-oo}} & \multicolumn{4}{X}{{ kuss-jagacinaa}} & \multicolumn{7}{X}{}\\
\multicolumn{4}{X}{kill-POL-NPST} & \multicolumn{3}{X}{kill-person} & \multicolumn{5}{X}{kill-UMRK-SUPP} & \multicolumn{4}{X}{kill-SIM} & \multicolumn{7}{X}{}\\
\multicolumn{23}{X}{}\\
\multicolumn{23}{X}{{\bfseries Type-D affixes and clitic}}\\
{ \textit{{}-ba} (CSL)} & \multicolumn{4}{X}{{ \textit{{}-boo} (CND)}} & \multicolumn{4}{X}{{ \textit{{}-gadɨ} ‘until’}} & \multicolumn{4}{X}{{ \textit{{}-na} (PROH)}} & \multicolumn{4}{X}{{ \textit{kai} (DUB)}} & \multicolumn{6}{X}{}\\
{ kuss-ɨba} & \multicolumn{4}{X}{{ *kuss-ɨboo}} & \multicolumn{4}{X}{{ *kuss-ɨgadɨ}} & \multicolumn{4}{X}{{ kuss-ɨna}} & \multicolumn{4}{X}{{ *kuss=ɨkai}} & \multicolumn{6}{X}{}\\
kill-CSL & \multicolumn{4}{X}{kill-CND} & \multicolumn{4}{X}{kill-until} & \multicolumn{4}{X}{kill-PROH} & \multicolumn{4}{X}{kill=DUB} & \multicolumn{6}{X}{}\\
\lspbottomrule
\end{tabularx}
Stem No. 11 (ending with //t//): \textit{ut-} ‘hit’

\tablefirsthead{}

\tabletail{}
\tablelasttail{}
\begin{tabularx}{\textwidth}{XXm{4.5984238E-4in}XXXXXXXXXXXXXXXXXXXX}
\lsptoprule
\multicolumn{23}{X}{{\bfseries Type-A affixes}}\\
\multicolumn{3}{X}{{ \textit{{}-an} (NEG)}} & \multicolumn{3}{X}{{ \textit{{}-ar(ɨr)} (PASS)}} & \multicolumn{4}{X}{{ \textit{{}-ar(ɨr)} (CAP)}} & \multicolumn{4}{X}{{ \textit{{}-as} (CAUS)}} & \multicolumn{3}{X}{{ \textit{{}-azɨi} (NEG.PLQ)}} & \multicolumn{2}{X}{{ \textit{{}-ɨ} (IMP)}} & \multicolumn{2}{X}{{ \textit{{}-ɨba} (SUGS)}} & { \textit{{}-oo}(INT)} & \\
\multicolumn{3}{X}{{ ut-an}} & \multicolumn{3}{X}{{ ut-at-tɨ}} & \multicolumn{4}{X}{{ ut-arɨk=kai}} & \multicolumn{4}{X}{{ ut-as-oo}} & \multicolumn{3}{X}{{ ut-azɨi}} & \multicolumn{2}{X}{{ ut-ɨ}} & \multicolumn{2}{X}{{ ut-ɨba}} & { ut-oo} & \\
\multicolumn{3}{X}{hit-NEG} & \multicolumn{3}{X}{hit-PASS-SEQ} & \multicolumn{4}{X}{hit-CAP=DUB} & \multicolumn{4}{X}{hit-CAUS-INT} & \multicolumn{3}{X}{hit-NEG.PLQ} & \multicolumn{2}{X}{hit-IMP} & \multicolumn{2}{X}{hit-SUGS} & hit-INT & \\
\multicolumn{23}{X}{}\\
\multicolumn{23}{X}{{\bfseries Type-B affixes}}\\
\multicolumn{2}{X}{{ \textit{{}-tar} (PST)}} & \multicolumn{6}{X}{{ \textit{{}-tuk} (PRPR)}} & \multicolumn{3}{X}{{ \textit{{}-tur} (PROG)}} & \multicolumn{4}{X}{{ \textit{{}-təər} (RSL)}} & \multicolumn{3}{X}{{ \textit{{}-tɨ} (SEQ)}} & \multicolumn{2}{X}{{ \textit{{}-tai} (LST)}} & \multicolumn{3}{X}{{ \textit{{}-təəra} ‘after’}}\\
\multicolumn{2}{X}{{ uc-cja}} & \multicolumn{6}{X}{{ uc-cjuk-ɨ}} & \multicolumn{3}{X}{{ uc-cju-tɨ}} & \multicolumn{4}{X}{{ uc-cjəəp-pa}} & \multicolumn{3}{X}{{ uc-cjɨ}} & \multicolumn{2}{X}{{ *uc-cjai}} & \multicolumn{3}{X}{{ *uc-cjəəra}}\\
\multicolumn{2}{X}{hit-PST} & \multicolumn{6}{X}{hit-PRPR-IMP} & \multicolumn{3}{X}{hit-PROG-SEQ} & \multicolumn{4}{X}{hit-RSL-CSL} & \multicolumn{3}{X}{hit-SEQ} & \multicolumn{2}{X}{hit-LST} & \multicolumn{3}{X}{{ hit-after}}\\
\multicolumn{2}{X}{} & \multicolumn{6}{X}{} & \multicolumn{3}{X}{} & \multicolumn{4}{X}{} & \multicolumn{3}{X}{} & \multicolumn{2}{X}{} & \multicolumn{3}{X}{}\\
\multicolumn{23}{X}{{\bfseries Type-C affixes}}\\
\multicolumn{4}{X}{{ \textit{{}-jawur} (POL)}} & \multicolumn{3}{X}{{ \textit{{}-jaa} ‘person’}} & \multicolumn{5}{X}{{ \textit{{}-jur} (UMRK)}} & \multicolumn{4}{X}{{ \textit{{}-jagacinaa} (SIM)}} & \multicolumn{7}{X}{}\\
\multicolumn{4}{X}{{ *uc-jawu-i}} & \multicolumn{3}{X}{uc-jaa} & \multicolumn{5}{X}{uc-ju-i} & \multicolumn{4}{X}{{ uc-jagacinaa}} & \multicolumn{7}{X}{}\\
\multicolumn{4}{X}{hit-POL-NPST} & \multicolumn{3}{X}{hit-person} & \multicolumn{5}{X}{hit-UMRK-NPST} & \multicolumn{4}{X}{hit-SIM} & \multicolumn{7}{X}{}\\
\multicolumn{23}{X}{}\\
\multicolumn{23}{X}{{\bfseries Type-D affixes and clitic}}\\
{ \textit{{}-ba} (CSL)} & \multicolumn{4}{X}{{ \textit{{}-boo} (CND)}} & \multicolumn{4}{X}{{ \textit{{}-gadɨ} ‘until’}} & \multicolumn{4}{X}{{ \textit{{}-na} (PROH)}} & \multicolumn{4}{X}{{ \textit{kai} (DUB)}} & \multicolumn{6}{X}{}\\
{ uc-ɨba} & \multicolumn{4}{X}{{ uc-ɨboo}} & \multicolumn{4}{X}{{ *uc-ɨgadɨ}} & \multicolumn{4}{X}{{ uc-ɨna}} & \multicolumn{4}{X}{{ *uc=ɨkai}} & \multicolumn{6}{X}{}\\
hit-CSL & \multicolumn{4}{X}{hit-CND} & \multicolumn{4}{X}{hit-until} & \multicolumn{4}{X}{hit-PROH} & \multicolumn{4}{X}{hit=DUB} & \multicolumn{6}{X}{}\\
\lspbottomrule
\end{tabularx}
Stem No. 12 (ending with //\$C(G)//): \textit{jˀ-} ‘say’

\tablefirsthead{}

\tabletail{}
\tablelasttail{}
\begin{tabularx}{\textwidth}{XXm{4.5984238E-4in}XXXXXXXXXXXXXXXXXXXX}
\lsptoprule
\multicolumn{23}{X}{{\bfseries Type-A affixes}}\\
\multicolumn{3}{X}{{ \textit{{}-an} (NEG)}} & \multicolumn{3}{X}{{ \textit{{}-ar(ɨr)} (PASS)}} & \multicolumn{4}{X}{{ \textit{{}-ar(ɨr)} (CAP)}} & \multicolumn{4}{X}{{ \textit{{}-as} (CAUS)}} & \multicolumn{3}{X}{{ \textit{{}-azɨi} (NEG.PLQ)}} & \multicolumn{2}{X}{{ \textit{{}-ɨ} (IMP)}} & \multicolumn{2}{X}{{ \textit{{}-ɨba} (SUGS)}} & { \textit{{}-oo}(INT)} & \\
\multicolumn{3}{X}{{ jˀ-an-tɨ}} & \multicolumn{3}{X}{{ jˀ-at-tɨ}} & \multicolumn{4}{X}{{ jˀ-arɨɨr-u}} & \multicolumn{4}{X}{{ jˀ-as-oo}} & \multicolumn{3}{X}{{ jˀ-azɨi}} & \multicolumn{2}{X}{{ jˀ-ɨ}} & \multicolumn{2}{X}{{ jˀ-ɨba}} & { jˀ-oo} & \\
\multicolumn{3}{X}{say-NEG-SEQ} & \multicolumn{3}{X}{say-PASS-SEQ} & \multicolumn{4}{X}{say-CAP-PFC} & \multicolumn{4}{X}{say-CAUS-INT} & \multicolumn{3}{X}{say-NEG.PLQ} & \multicolumn{2}{X}{say-IMP} & \multicolumn{2}{X}{say-SUGS} & say-INT & \\
\multicolumn{23}{X}{}\\
\multicolumn{23}{X}{{\bfseries Type-B affixes}}\\
\multicolumn{2}{X}{{ \textit{{}-tar} (PST)}} & \multicolumn{6}{X}{{ \textit{{}-tuk} (PRPR)}} & \multicolumn{3}{X}{{ \textit{{}-tur} (PROG)}} & \multicolumn{4}{X}{{ \textit{{}-təər} (RSL)}} & \multicolumn{3}{X}{{ \textit{{}-tɨ} (SEQ)}} & \multicolumn{2}{X}{{ \textit{{}-tai} (LST)}} & \multicolumn{3}{X}{{ \textit{{}-təəra} ‘after’}}\\
\multicolumn{2}{X}{{ jˀi-cja}} & \multicolumn{6}{X}{{ jˀi-cjuk-ɨ}} & \multicolumn{3}{X}{{ jˀi-cju-tɨ}} & \multicolumn{4}{X}{{ jˀi-cjə-n}} & \multicolumn{3}{X}{{ jˀi-cjɨ}} & \multicolumn{2}{X}{{ *jˀi-cjai}} & \multicolumn{3}{X}{{ *jˀi-cjəəra}}\\
\multicolumn{2}{X}{say-PST} & \multicolumn{6}{X}{say-PRPR-IMP} & \multicolumn{3}{X}{say-PROG-SEQ} & \multicolumn{4}{X}{say-RSL-PTCP} & \multicolumn{3}{X}{say-SEQ} & \multicolumn{2}{X}{say-LST} & \multicolumn{3}{X}{{ say-after}}\\
\multicolumn{2}{X}{} & \multicolumn{6}{X}{} & \multicolumn{3}{X}{} & \multicolumn{4}{X}{} & \multicolumn{3}{X}{} & \multicolumn{2}{X}{} & \multicolumn{3}{X}{}\\
\multicolumn{23}{X}{{\bfseries Type-C affixes}}\\
\multicolumn{4}{X}{{ \textit{{}-jawur} (POL)}} & \multicolumn{3}{X}{{ \textit{{}-jaa} ‘person’}} & \multicolumn{5}{X}{{ \textit{{}-jur} (UMRK)}} & \multicolumn{4}{X}{{ \textit{{}-jagacinaa} (SIM)}} & \multicolumn{7}{X}{}\\
\multicolumn{4}{X}{{ *jˀ-awu-i}} & \multicolumn{3}{X}{jˀ-aa} & \multicolumn{5}{X}{jˀ-ur-u} & \multicolumn{4}{X}{{ jˀ-aagacinaa}} & \multicolumn{7}{X}{}\\
\multicolumn{4}{X}{say-POL-NPST} & \multicolumn{3}{X}{say-person} & \multicolumn{5}{X}{say-UMRK-PFC} & \multicolumn{4}{X}{say-SIM} & \multicolumn{7}{X}{}\\
\multicolumn{23}{X}{}\\
\multicolumn{23}{X}{{\bfseries Type-D affixes and clitic}}\\
{ \textit{{}-ba} (CSL)} & \multicolumn{4}{X}{{ \textit{{}-boo} (CND)}} & \multicolumn{4}{X}{{ \textit{{}-gadɨ} ‘until’}} & \multicolumn{4}{X}{{ \textit{{}-na} (PROH)}} & \multicolumn{4}{X}{{ \textit{kai} (DUB)}} & \multicolumn{6}{X}{}\\
{ jˀ-uuba} & \multicolumn{4}{X}{{ jˀ-uuboo}} & \multicolumn{4}{X}{{ *jˀ-uugadɨ}} & \multicolumn{4}{X}{{ jˀ-uuna}} & \multicolumn{4}{X}{{ *jˀ=uukai}} & \multicolumn{6}{X}{}\\
say-CSL & \multicolumn{4}{X}{say-CND} & \multicolumn{4}{X}{say-until} & \multicolumn{4}{X}{say-PROH} & \multicolumn{4}{X}{say=DUB} & \multicolumn{6}{X}{}\\
\lspbottomrule
\end{tabularx}
Stem No. 12 (ending with //\$C(G)//): \textit{mj-} ‘see’

\tablefirsthead{}

\tabletail{}
\tablelasttail{}
\begin{tabularx}{\textwidth}{XXXXXXXXXXXXXXXXXXXXXX}
\lsptoprule
\multicolumn{22}{X}{{\bfseries Type-A affixes}}\\
\multicolumn{2}{X}{{ \textit{{}-an} (NEG)}} & \multicolumn{4}{X}{{ \textit{{}-ar(ɨr)} (PASS)}} & \multicolumn{3}{X}{{ \textit{{}-ar(ɨr)} (CAP)}} & \multicolumn{4}{X}{{ \textit{{}-as} (CAUS)}} & \multicolumn{3}{X}{{ \textit{{}-azɨi} (NEG.PLQ)}} & \multicolumn{2}{X}{{ \textit{{}-ɨ} (IMP)}} & \multicolumn{2}{X}{{ \textit{{}-ɨba} (SUGS)}} & { \textit{{}-oo}(INT)} & \\
\multicolumn{2}{X}{{ mj-an}} & \multicolumn{4}{X}{{ mj-at-ta}} & \multicolumn{3}{X}{{ mj-ar-an-ba}} & \multicolumn{4}{X}{{ mj-as-oo}} & \multicolumn{3}{X}{{ mj-azɨi}} & \multicolumn{2}{X}{{ mj-ɨ}} & \multicolumn{2}{X}{{ mj-ɨba}} & { mj-oo} & \\
\multicolumn{2}{X}{see-NEG} & \multicolumn{4}{X}{see-PASS-PST} & \multicolumn{3}{X}{see-CAP-NEG-CSL} & \multicolumn{4}{X}{see-CAUS-INT} & \multicolumn{3}{X}{see-NEG.PLQ} & \multicolumn{2}{X}{see-IMP} & \multicolumn{2}{X}{see-SUGS} & see-INT & \\
\multicolumn{22}{X}{}\\
\multicolumn{22}{X}{{\bfseries Type-B affixes}}\\
\multicolumn{3}{X}{{ \textit{{}-tar} (PST)}} & \multicolumn{5}{X}{{ \textit{{}-tuk} (PRPR)}} & \multicolumn{2}{X}{{ \textit{{}-tur} (PROG)}} & \multicolumn{4}{X}{{ \textit{{}-təər} (RSL)}} & \multicolumn{3}{X}{{ \textit{{}-tɨ} (SEQ)}} & \multicolumn{2}{X}{{ \textit{{}-tai} (LST)}} & \multicolumn{3}{X}{{ \textit{{}-təəra} ‘after’}}\\
\multicolumn{3}{X}{{ mji-cja}} & \multicolumn{5}{X}{{ mji-cjuk-ɨ}} & \multicolumn{2}{X}{{ mji-cju-tɨ}} & \multicolumn{4}{X}{{ mji-cjəəp-pa}} & \multicolumn{3}{X}{{ mji-cjɨ}} & \multicolumn{2}{X}{{ *mji-cjai}} & \multicolumn{3}{X}{{ *mji-cjəəra}}\\
\multicolumn{3}{X}{see-PST} & \multicolumn{5}{X}{see-PRPR-IMP} & \multicolumn{2}{X}{see-PROG-SEQ} & \multicolumn{4}{X}{see-RSL-CSL} & \multicolumn{3}{X}{see-SEQ} & \multicolumn{2}{X}{see-LST} & \multicolumn{3}{X}{{ see-after}}\\
\multicolumn{3}{X}{} & \multicolumn{5}{X}{} & \multicolumn{2}{X}{} & \multicolumn{4}{X}{} & \multicolumn{3}{X}{} & \multicolumn{2}{X}{} & \multicolumn{3}{X}{}\\
\multicolumn{22}{X}{{\bfseries Type-C affixes}}\\
\multicolumn{4}{X}{{ \textit{{}-jawur} (POL)}} & \multicolumn{3}{X}{{ \textit{{}-jaa} ‘person’}} & \multicolumn{4}{X}{{ \textit{{}-jur} (UMRK)}} & \multicolumn{4}{X}{{ \textit{{}-jagacinaa} (SIM)}} & \multicolumn{7}{X}{}\\
\multicolumn{4}{X}{{ *mj-awu-i}} & \multicolumn{3}{X}{?} & \multicolumn{4}{X}{mj-u-i} & \multicolumn{4}{X}{{ mj-aagacinaa}} & \multicolumn{7}{X}{}\\
\multicolumn{4}{X}{see-POL-NPST} & \multicolumn{3}{X}{} & \multicolumn{4}{X}{see-UMRK-NPST} & \multicolumn{4}{X}{see-SIM} & \multicolumn{7}{X}{}\\
\multicolumn{22}{X}{}\\
\multicolumn{22}{X}{{\bfseries Type-D affixes and clitic}}\\
{ \textit{{}-ba} (CSL)} & \multicolumn{4}{X}{{ \textit{{}-boo} (CND)}} & \multicolumn{4}{X}{{ \textit{{}-gadɨ} ‘until’}} & \multicolumn{3}{X}{{ \textit{{}-na} (PROH)}} & \multicolumn{10}{X}{{ \textit{kai} (DUB)}}\\
{ mj-uuba} & \multicolumn{4}{X}{{ mj-uuboo}} & \multicolumn{4}{X}{{ mj-uugadɨ / mjik-kadɨ}} & \multicolumn{3}{X}{{ mj-uuna}} & \multicolumn{10}{X}{{ mj=uukai / mjik=kai}}\\
see-CSL & \multicolumn{4}{X}{see-CND} & \multicolumn{4}{X}{see-until} & \multicolumn{3}{X}{see-PROH} & \multicolumn{10}{X}{see=DUB}\\
\lspbottomrule
\end{tabularx}
Stem No. 13 (ending with //ij//): \textit{kij-} ‘cut’

\tablefirsthead{}

\tabletail{}
\tablelasttail{}
\begin{tabularx}{\textwidth}{XXXXXXXXXXXXXXXXXXXXXX}
\lsptoprule
\multicolumn{22}{X}{{\bfseries Type-A affixes}}\\
{ \textit{{}-an} (NEG)} & \multicolumn{4}{X}{{ \textit{{}-ar(ɨr)} (PASS)}} & \multicolumn{4}{X}{{ \textit{{}-ar(ɨr)} (CAP)}} & \multicolumn{4}{X}{{ \textit{{}-as} (CAUS)}} & \multicolumn{4}{X}{{ \textit{{}-azɨi} (NEG.PLQ)}} & \multicolumn{2}{X}{{ \textit{{}-ɨ} (IMP)}} & { \textit{{}-ɨba} (SUGS)} & { \textit{{}-oo}(INT)} & \\
{ kij-an} & \multicolumn{4}{X}{{ kij-at-tɨ}} & \multicolumn{4}{X}{{ kij-ar-an}} & \multicolumn{4}{X}{{ kij-as-oo}} & \multicolumn{4}{X}{{ kij-azɨi}} & \multicolumn{2}{X}{{ kij-ɨ}} & { kij-ɨba} & { kij-oo} & \\
cut-NEG & \multicolumn{4}{X}{cut-PASS-SEQ} & \multicolumn{4}{X}{cut-CAP-NEG} & \multicolumn{4}{X}{cut-CAUS-INT} & \multicolumn{4}{X}{cut-NEG.PLQ} & \multicolumn{2}{X}{cut-IMP} & cut-SUGS & cut-INT & \\
\multicolumn{22}{X}{}\\
\multicolumn{22}{X}{{\bfseries Type-B affixes}}\\
\multicolumn{2}{X}{{ \textit{{}-tar} (PST)}} & \multicolumn{4}{X}{{ \textit{{}-tuk} (PRPR)}} & \multicolumn{4}{X}{{ \textit{{}-tur} (PROG)}} & \multicolumn{4}{X}{{ \textit{{}-təər} (RSL)}} & \multicolumn{2}{X}{{ \textit{{}-tɨ} (SEQ)}} & \multicolumn{2}{X}{{ \textit{{}-tai} (LST)}} & \multicolumn{4}{X}{{ \textit{{}-təəra} ‘after’}}\\
\multicolumn{2}{X}{{ ki-cja}} & \multicolumn{4}{X}{{ ki-cjuk-ɨ}} & \multicolumn{4}{X}{{ ki-cjut=too}} & \multicolumn{4}{X}{{ ki-cjəəp-pa}} & \multicolumn{2}{X}{{ ki-cjɨ}} & \multicolumn{2}{X}{{ *ki-cjai}} & \multicolumn{4}{X}{{ *ki-cjəəra}}\\
\multicolumn{2}{X}{cut-PST} & \multicolumn{4}{X}{cut-PRPR-IMP} & \multicolumn{4}{X}{cut-PROG=ASS} & \multicolumn{4}{X}{cut-RSL-CSL} & \multicolumn{2}{X}{cut-SEQ} & \multicolumn{2}{X}{cut-LST} & \multicolumn{4}{X}{{ cut-after}}\\
\multicolumn{2}{X}{} & \multicolumn{4}{X}{} & \multicolumn{4}{X}{} & \multicolumn{4}{X}{} & \multicolumn{2}{X}{} & \multicolumn{2}{X}{} & \multicolumn{4}{X}{}\\
\multicolumn{22}{X}{{\bfseries Type-C affixes}}\\
\multicolumn{4}{X}{{ \textit{{}-jawur} (POL)}} & \multicolumn{3}{X}{{ \textit{{}-jaa} ‘person’}} & \multicolumn{4}{X}{{ \textit{{}-jur} (UMRK)}} & \multicolumn{4}{X}{{ \textit{{}-jagacinaa} (SIM)}} & \multicolumn{7}{X}{}\\
\multicolumn{4}{X}{{ *ki-jawu-i}} & \multicolumn{3}{X}{ki-jaa} & \multicolumn{4}{X}{ki-ju-mɨ} & \multicolumn{4}{X}{{ ki-jagacinaa}} & \multicolumn{7}{X}{}\\
\multicolumn{4}{X}{cut-POL-NPST} & \multicolumn{3}{X}{cut-person} & \multicolumn{4}{X}{cut-UMRK-PLQ} & \multicolumn{4}{X}{cut-SIM} & \multicolumn{7}{X}{}\\
\multicolumn{22}{X}{}\\
\multicolumn{22}{X}{{\bfseries Type-D affixes and clitic}}\\
\multicolumn{3}{X}{{ \textit{{}-ba} (CSL)}} & \multicolumn{3}{X}{{ \textit{{}-boo} (CND)}} & \multicolumn{2}{X}{{ \textit{{}-gadɨ} ‘until’}} & \multicolumn{4}{X}{{ \textit{{}-na} (PROH)}} & \multicolumn{5}{X}{{ \textit{kai} (DUB)}} & \multicolumn{5}{X}{}\\
\multicolumn{3}{X}{{ kip-pa}} & \multicolumn{3}{X}{{ kip-poo}} & \multicolumn{2}{X}{{ kig-gadɨ}} & \multicolumn{4}{X}{{ kin-na}} & \multicolumn{5}{X}{{ *kik=kai}} & \multicolumn{5}{X}{}\\
\multicolumn{3}{X}{cut-CSL} & \multicolumn{3}{X}{cut-CND} & \multicolumn{2}{X}{cut-until} & \multicolumn{4}{X}{cut-PROH} & \multicolumn{5}{X}{cut=DUB} & \multicolumn{5}{X}{}\\
\lspbottomrule
\end{tabularx}
Stem No. 14 (ending with //V\textsubscript{non-}\textit{\textsubscript{i}} g//): \textit{tug-} ‘whet’

\tablefirsthead{}

\tabletail{}
\tablelasttail{}
\begin{tabularx}{\textwidth}{XXXXXXXXXXXXXXXXXXXX}
\lsptoprule
\multicolumn{20}{X}{{\bfseries Type-A affixes}}\\
{ \textit{{}-an} (NEG)} & \multicolumn{3}{X}{{ \textit{{}-ar(ɨr)} (PASS)}} & \multicolumn{3}{X}{{ \textit{{}-ar(ɨr)} (CAP)}} & \multicolumn{4}{X}{{ \textit{{}-as} (CAUS)}} & \multicolumn{4}{X}{{ \textit{{}-azɨi} (NEG.PLQ)}} & \multicolumn{2}{X}{{ \textit{{}-ɨ} (IMP)}} & { \textit{{}-ɨba} (SUGS)} & { \textit{{}-oo}(INT)} & \\
{ tug-an} & \multicolumn{3}{X}{{ tug-at-ta}} & \multicolumn{3}{X}{{ tug-arɨk=kai}} & \multicolumn{4}{X}{{ tug-as-oo}} & \multicolumn{4}{X}{{ tug-azɨi}} & \multicolumn{2}{X}{{ tug-ɨ}} & { tug-ɨba} & { tug-oo} & \\
whet-NEG & \multicolumn{3}{X}{whet-PASS-PST} & \multicolumn{3}{X}{whet-CAP=DUB} & \multicolumn{4}{X}{whet-CAUS-INT} & \multicolumn{4}{X}{whet-NEG.PLQ} & \multicolumn{2}{X}{whet-IMP} & whet-SUGS & whet-INT & \\
\multicolumn{20}{X}{}\\
\multicolumn{20}{X}{{\bfseries Type-B affixes}}\\
{ \textit{{}-tar} (PST)} & \multicolumn{3}{X}{{ \textit{{}-tuk} (PRPR)}} & \multicolumn{3}{X}{{ \textit{{}-tur} (PROG)}} & \multicolumn{3}{X}{{ \textit{{}-təər} (RSL)}} & \multicolumn{3}{X}{{ \textit{{}-tɨ} (SEQ)}} & \multicolumn{3}{X}{{ \textit{{}-tai} (LST)}} & \multicolumn{4}{X}{{ \textit{{}-təəra} ‘after’}}\\
{ tu-zja} & \multicolumn{3}{X}{{ tu-zjuk-ɨ}} & \multicolumn{3}{X}{{ tu-zjut=too}} & \multicolumn{3}{X}{{ tu-zjəəp-pa}} & \multicolumn{3}{X}{{ *tu-zjɨ}} & \multicolumn{3}{X}{{ *tu-zjai}} & \multicolumn{4}{X}{{ *tu-zjəəra}}\\
whet-PST & \multicolumn{3}{X}{whet-PRPR-IMP} & \multicolumn{3}{X}{whet-PROG=ASS} & \multicolumn{3}{X}{whet-RSL-CSL} & \multicolumn{3}{X}{whet-SEQ} & \multicolumn{3}{X}{whet-LST} & \multicolumn{4}{X}{{ whet-after}}\\
& \multicolumn{3}{X}{} & \multicolumn{3}{X}{} & \multicolumn{3}{X}{} & \multicolumn{3}{X}{} & \multicolumn{3}{X}{} & \multicolumn{4}{X}{}\\
\multicolumn{20}{X}{{\bfseries Type-C affixes}}\\
\multicolumn{3}{X}{{ \textit{{}-jawur} (POL)}} & \multicolumn{2}{X}{{ \textit{{}-jaa} ‘person’}} & \multicolumn{3}{X}{{ \textit{{}-jur} (UMRK)}} & \multicolumn{4}{X}{{ \textit{{}-jagacinaa} (SIM)}} & \multicolumn{8}{X}{}\\
\multicolumn{3}{X}{{ *tug-jawu-i}} & \multicolumn{2}{X}{{ tug-jaa}} & \multicolumn{3}{X}{tug-ju-mɨ} & \multicolumn{4}{X}{{ tug-jagacinaa}} & \multicolumn{8}{X}{}\\
\multicolumn{3}{X}{whet-POL-NPST} & \multicolumn{2}{X}{whet-person} & \multicolumn{3}{X}{whet-UMRK-PLQ} & \multicolumn{4}{X}{whet-SIM} & \multicolumn{8}{X}{}\\
\multicolumn{20}{X}{}\\
\multicolumn{20}{X}{{\bfseries Type-D affixes and clitic}}\\
\multicolumn{2}{X}{{ \textit{{}-ba} (CSL)}} & \multicolumn{2}{X}{{ \textit{{}-boo} (CND)}} & \multicolumn{2}{X}{{ \textit{{}-gadɨ} ‘until’}} & \multicolumn{3}{X}{{ \textit{{}-na} (PROH)}} & \multicolumn{5}{X}{{ \textit{kai} (DUB)}} & \multicolumn{6}{X}{}\\
\multicolumn{2}{X}{{ tug-uba}} & \multicolumn{2}{X}{{ tug-uboo}} & \multicolumn{2}{X}{{ *tug-ugadɨ}} & \multicolumn{3}{X}{{ tug-una}} & \multicolumn{5}{X}{{ *tug=ukai}} & \multicolumn{6}{X}{}\\
\multicolumn{2}{X}{whet-CSL} & \multicolumn{2}{X}{whet-CND} & \multicolumn{2}{X}{whet-until} & \multicolumn{3}{X}{whet-PROH} & \multicolumn{5}{X}{whet=DUB} & \multicolumn{6}{X}{}\\
\lspbottomrule
\end{tabularx}
Stem No. 15 (ending with //ik//): \textit{kik-} ‘hear’

\tablefirsthead{}

\tabletail{}
\tablelasttail{}
\begin{tabularx}{\textwidth}{XXXXXXXXXXXXXXXXXXXXXXX}
\lsptoprule
\multicolumn{23}{X}{{\bfseries Type-A affixes}}\\
{ \textit{{}-an} (NEG)} & \multicolumn{4}{X}{{ \textit{{}-ar(ɨr)} (PASS)}} & \multicolumn{4}{X}{{ \textit{{}-ar(ɨr)} (CAP)}} & \multicolumn{4}{X}{{ \textit{{}-as} (CAUS)}} & \multicolumn{5}{X}{{ \textit{{}-azɨi} (NEG.PLQ)}} & \multicolumn{2}{X}{{ \textit{{}-ɨ} (IMP)}} & { \textit{{}-ɨba} (SUGS)} & { \textit{{}-oo}(INT)} & \\
{ kik-jan} & \multicolumn{4}{X}{{ kik-jar-an}} & \multicolumn{4}{X}{{ kik-jarɨ-i}} & \multicolumn{4}{X}{{ kik-jas-i}} & \multicolumn{5}{X}{{ kik-jazɨi}} & \multicolumn{2}{X}{{ kik-jɨ}} & { kik-jɨba} & { kik-joo} & \\
hear-NEG & \multicolumn{4}{X}{hear-PASS-NEG} & \multicolumn{4}{X}{hear-CAP-NPST} & \multicolumn{4}{X}{hear-CAUS-INF} & \multicolumn{5}{X}{hear-NEG.PLQ} & \multicolumn{2}{X}{hear-IMP} & hear-SUGS & hear-INT & \\
\multicolumn{23}{X}{}\\
\multicolumn{23}{X}{{\bfseries Type-B affixes}}\\
\multicolumn{2}{X}{{ \textit{{}-tar} (PST)}} & \multicolumn{4}{X}{{ \textit{{}-tuk} (PRPR)}} & \multicolumn{4}{X}{{ \textit{{}-tur} (PROG)}} & \multicolumn{4}{X}{{ \textit{{}-təər} (RSL)}} & \multicolumn{3}{X}{{ \textit{{}-tɨ} (SEQ)}} & \multicolumn{2}{X}{{ \textit{{}-tai} (LST)}} & \multicolumn{4}{X}{{ \textit{{}-təəra} ‘after’}}\\
\multicolumn{2}{X}{{ ki-cja}} & \multicolumn{4}{X}{{ ki-cjuk-ɨ}} & \multicolumn{4}{X}{{ ki-cju-tɨ}} & \multicolumn{4}{X}{{ ki-cjəəp-pa}} & \multicolumn{3}{X}{{ *ki-cjɨ}} & \multicolumn{2}{X}{{ *ki-cjai}} & \multicolumn{4}{X}{{ *ki-cjəəra}}\\
\multicolumn{2}{X}{hear-PST} & \multicolumn{4}{X}{hear-PRPR-IMP} & \multicolumn{4}{X}{hear-PROG-SEQ} & \multicolumn{4}{X}{hear-RSL-CSL} & \multicolumn{3}{X}{hear-SEQ} & \multicolumn{2}{X}{hear-LST} & \multicolumn{4}{X}{{ hear-after}}\\
\multicolumn{2}{X}{} & \multicolumn{4}{X}{} & \multicolumn{4}{X}{} & \multicolumn{4}{X}{} & \multicolumn{3}{X}{} & \multicolumn{2}{X}{} & \multicolumn{4}{X}{}\\
\multicolumn{23}{X}{{\bfseries Type-C affixes}}\\
\multicolumn{3}{X}{{ \textit{{}-jawur} (POL)}} & \multicolumn{4}{X}{{ \textit{{}-jaa} ‘person’}} & \multicolumn{4}{X}{{ \textit{{}-jur} (UMRK)}} & \multicolumn{4}{X}{{ \textit{{}-jagacinaa} (SIM)}} & \multicolumn{8}{X}{}\\
\multicolumn{3}{X}{{ *kik-jawu-i}} & \multicolumn{4}{X}{kik-jaa} & \multicolumn{4}{X}{{ kik-ju-n}} & \multicolumn{4}{X}{{ kik-jagacinaa}} & \multicolumn{8}{X}{}\\
\multicolumn{3}{X}{hear-POL-NPST} & \multicolumn{4}{X}{hear-person} & \multicolumn{4}{X}{hear-UMRK-PTCP} & \multicolumn{4}{X}{hear-SIM} & \multicolumn{8}{X}{}\\
\multicolumn{23}{X}{}\\
\multicolumn{23}{X}{{\bfseries Type-D affixes and clitic}}\\
\multicolumn{2}{X}{{ \textit{{}-ba} (CSL)}} & \multicolumn{2}{X}{{ \textit{{}-boo} (CND)}} & \multicolumn{4}{X}{{ \textit{{}-gadɨ} ‘until’}} & \multicolumn{4}{X}{{ \textit{{}-na} (PROH)}} & \multicolumn{4}{X}{{ \textit{kai} (DUB)}} & \multicolumn{7}{X}{}\\
\multicolumn{2}{X}{{ kik-uba}} & \multicolumn{2}{X}{{ *kik-uboo}} & \multicolumn{4}{X}{{ *kik-ugadɨ}} & \multicolumn{4}{X}{{ kik-una}} & \multicolumn{4}{X}{{ *kik=ukai}} & \multicolumn{7}{X}{}\\
\multicolumn{2}{X}{hear-CSL} & \multicolumn{2}{X}{hear-CND} & \multicolumn{4}{X}{hear-until} & \multicolumn{4}{X}{hear-PROH} & \multicolumn{4}{X}{hear=DUB} & \multicolumn{7}{X}{}\\
\lspbottomrule
\end{tabularx}
Stem No. 16 (ending with //i(n)g//): \textit{uig-} ‘swim’

\tablefirsthead{}

\tabletail{}
\tablelasttail{}
\begin{tabularx}{\textwidth}{XXXXXXXXXXXXXXXXXXXX}
\lsptoprule
\multicolumn{20}{X}{{\bfseries Type-A affixes}}\\
{ \textit{{}-an} (NEG)} & \multicolumn{3}{X}{{ \textit{{}-ar(ɨr)} (PASS)}} & \multicolumn{3}{X}{{ \textit{{}-ar(ɨr)} (CAP)}} & \multicolumn{4}{X}{{ \textit{{}-as} (CAUS)}} & \multicolumn{4}{X}{{ \textit{{}-azɨi} (NEG.PLQ)}} & \multicolumn{2}{X}{{ \textit{{}-ɨ} (IMP)}} & { \textit{{}-ɨba} (SUGS)} & { \textit{{}-oo}(INT)} & \\
{ uig-jan} & \multicolumn{3}{X}{{ uig-jat-ta}} & \multicolumn{3}{X}{{ uig-jarɨk=kai}} & \multicolumn{4}{X}{{ uig-jas-oo}} & \multicolumn{4}{X}{{ uig-jazɨi}} & \multicolumn{2}{X}{{ uig-jɨ}} & { uig-iba} & { uig-joo} & \\
swim-NEG & \multicolumn{3}{X}{swim-PASS-PST} & \multicolumn{3}{X}{swim-CAP=DUB} & \multicolumn{4}{X}{swim-CAUS-INT} & \multicolumn{4}{X}{swim-NEG.PLQ} & \multicolumn{2}{X}{swim-IMP} & swim-SUGS & swim-INT & \\
\multicolumn{20}{X}{}\\
\multicolumn{20}{X}{{\bfseries Type-B affixes}}\\
{ \textit{{}-tar} (PST)} & \multicolumn{3}{X}{{ \textit{{}-tuk} (PRPR)}} & \multicolumn{3}{X}{{ \textit{{}-tur} (PROG)}} & \multicolumn{3}{X}{{ \textit{{}-təər} (RSL)}} & \multicolumn{3}{X}{{ \textit{{}-tɨ} (SEQ)}} & \multicolumn{3}{X}{{ \textit{{}-tai} (LST)}} & \multicolumn{4}{X}{{ \textit{{}-təəra} ‘after’}}\\
{ ui-zja} & \multicolumn{3}{X}{{ ui-zjuk-ɨ}} & \multicolumn{3}{X}{{ ui-zju-i}} & \multicolumn{3}{X}{{ ui-zjəəp-pa}} & \multicolumn{3}{X}{{ *ui-zjɨ}} & \multicolumn{3}{X}{{ *ui-zjai}} & \multicolumn{4}{X}{{ *ui-zjəəra}}\\
swim-PST & \multicolumn{3}{X}{swim-PRPR-IMP} & \multicolumn{3}{X}{swim-PROG-NPST} & \multicolumn{3}{X}{swim-RSL-CSL} & \multicolumn{3}{X}{swim-SEQ} & \multicolumn{3}{X}{swim-LST} & \multicolumn{4}{X}{{ swim-after}}\\
& \multicolumn{3}{X}{} & \multicolumn{3}{X}{} & \multicolumn{3}{X}{} & \multicolumn{3}{X}{} & \multicolumn{3}{X}{} & \multicolumn{4}{X}{}\\
\multicolumn{20}{X}{{\bfseries Type-C affixes}}\\
\multicolumn{3}{X}{{ \textit{{}-jawur} (POL)}} & \multicolumn{2}{X}{{ \textit{{}-jaa} ‘person’}} & \multicolumn{3}{X}{{ \textit{{}-jur} (UMRK)}} & \multicolumn{4}{X}{{ \textit{{}-jagacinaa} (SIM)}} & \multicolumn{8}{X}{}\\
\multicolumn{3}{X}{{ *uig-jawu-i}} & \multicolumn{2}{X}{{ uig-jaa}} & \multicolumn{3}{X}{uig-ju-n} & \multicolumn{4}{X}{{ uig-jagacinaa}} & \multicolumn{8}{X}{}\\
\multicolumn{3}{X}{swim-POL-NPST} & \multicolumn{2}{X}{swim-person} & \multicolumn{3}{X}{swim-UMRK-PTCP} & \multicolumn{4}{X}{swim-SIM} & \multicolumn{8}{X}{}\\
\multicolumn{20}{X}{}\\
\multicolumn{20}{X}{{\bfseries Type-D affixes and clitic}}\\
\multicolumn{2}{X}{{ \textit{{}-ba} (CSL)}} & \multicolumn{2}{X}{{ \textit{{}-boo} (CND)}} & \multicolumn{2}{X}{{ \textit{{}-gadɨ} ‘until’}} & \multicolumn{3}{X}{{ \textit{{}-na} (PROH)}} & \multicolumn{5}{X}{{ \textit{kai} (DUB)}} & \multicolumn{6}{X}{}\\
\multicolumn{2}{X}{{ uig-uba}} & \multicolumn{2}{X}{{ uig-uboo}} & \multicolumn{2}{X}{{ uig-ugadɨ}} & \multicolumn{3}{X}{{ uig-una}} & \multicolumn{5}{X}{{ *uig=ukai}} & \multicolumn{6}{X}{}\\
\multicolumn{2}{X}{swim-CSL} & \multicolumn{2}{X}{swim-CND} & \multicolumn{2}{X}{swim-until} & \multicolumn{3}{X}{swim-PROH} & \multicolumn{5}{X}{swim=DUB} & \multicolumn{6}{X}{}\\
\lspbottomrule
\end{tabularx}
Stem No. 16 (ending with //i(n)g//): \textit{ming-} ‘grab’

\tablefirsthead{}

\tabletail{}
\tablelasttail{}
\begin{tabularx}{\textwidth}{XXXXXXXXXXXXXXXXXXXXXXX}
\lsptoprule
\multicolumn{23}{X}{{\bfseries Type-A affixes}}\\
{ \textit{{}-an} (NEG)} & \multicolumn{4}{X}{{ \textit{{}-ar(ɨr)} (PASS)}} & \multicolumn{4}{X}{{ \textit{{}-ar(ɨr)} (CAP)}} & \multicolumn{4}{X}{{ \textit{{}-as} (CAUS)}} & \multicolumn{3}{X}{{ \textit{{}-azɨi} (NEG.PLQ)}} & \multicolumn{3}{X}{{ \textit{{}-ɨ} (IMP)}} & \multicolumn{2}{X}{{ \textit{{}-ɨba} (SUGS)}} & { \textit{{}-oo}(INT)} & \\
{ ming-jan} & \multicolumn{4}{X}{{ ming-jat-ta}} & \multicolumn{4}{X}{{ ming-jar-an}} & \multicolumn{4}{X}{{ ming-jas-oo}} & \multicolumn{3}{X}{{ ming-jazɨi}} & \multicolumn{3}{X}{{ ming-jɨ}} & \multicolumn{2}{X}{{ ming-jɨba / ming-iba}} & { ming-joo} & \\
grab-NEG & \multicolumn{4}{X}{grab-PASS-PST} & \multicolumn{4}{X}{grab-CAP-NEG} & \multicolumn{4}{X}{grab-CAUS-INT} & \multicolumn{3}{X}{grab-NEG.PLQ} & \multicolumn{3}{X}{grab-IMP} & \multicolumn{2}{X}{grab-SUGS} & grab-INT & \\
\multicolumn{23}{X}{}\\
\multicolumn{23}{X}{{\bfseries Type-B affixes}}\\
\multicolumn{3}{X}{{ \textit{{}-tar} (PST)}} & \multicolumn{5}{X}{{ \textit{{}-tuk} (PRPR)}} & \multicolumn{2}{X}{{ \textit{{}-tur} (PROG)}} & \multicolumn{4}{X}{{ \textit{{}-təər} (RSL)}} & \multicolumn{3}{X}{{ \textit{{}-tɨ} (SEQ)}} & \multicolumn{3}{X}{{ \textit{{}-tai} (LST)}} & \multicolumn{3}{X}{{ \textit{{}-təəra} ‘after’}}\\
\multicolumn{3}{X}{{ min-zjat=too}} & \multicolumn{5}{X}{{ min-zjuk-ɨ}} & \multicolumn{2}{X}{{ min-zjur-ɨ}} & \multicolumn{4}{X}{{ min-zjəəp-pa}} & \multicolumn{3}{X}{{ *min-zjɨ}} & \multicolumn{3}{X}{{ *min-zjai}} & \multicolumn{3}{X}{{ *min-zjəəra}}\\
\multicolumn{3}{X}{grab-PST=ASS} & \multicolumn{5}{X}{grab-PRPR-IMP} & \multicolumn{2}{X}{grab-PROG-IMP} & \multicolumn{4}{X}{grab-RSL-CSL} & \multicolumn{3}{X}{grab-SEQ} & \multicolumn{3}{X}{grab-LST} & \multicolumn{3}{X}{{ grab-after}}\\
\multicolumn{3}{X}{} & \multicolumn{5}{X}{} & \multicolumn{2}{X}{} & \multicolumn{4}{X}{} & \multicolumn{3}{X}{} & \multicolumn{3}{X}{} & \multicolumn{3}{X}{}\\
\multicolumn{23}{X}{{\bfseries Type-C affixes}}\\
\multicolumn{4}{X}{{ \textit{{}-jawur} (POL)}} & \multicolumn{3}{X}{{ \textit{{}-jaa} ‘person’}} & \multicolumn{4}{X}{{ \textit{{}-jur} (UMRK)}} & \multicolumn{4}{X}{{ \textit{{}-jagacinaa} (SIM)}} & \multicolumn{8}{X}{}\\
\multicolumn{4}{X}{{ *ming-jawu-i}} & \multicolumn{3}{X}{{ ?}} & \multicolumn{4}{X}{ming-ju-i} & \multicolumn{4}{X}{{ ming-jagacinaa}} & \multicolumn{8}{X}{}\\
\multicolumn{4}{X}{grab-POL-NPST} & \multicolumn{3}{X}{} & \multicolumn{4}{X}{grab-UMRK-NPST} & \multicolumn{4}{X}{grab-SIM} & \multicolumn{8}{X}{}\\
\multicolumn{23}{X}{}\\
\multicolumn{23}{X}{{\bfseries Type-D affixes and clitic}}\\
\multicolumn{2}{X}{{ \textit{{}-ba} (CSL)}} & \multicolumn{4}{X}{{ \textit{{}-boo} (CND)}} & \multicolumn{3}{X}{{ \textit{{}-gadɨ} ‘until’}} & \multicolumn{3}{X}{{ \textit{{}-na} (PROH)}} & \multicolumn{6}{X}{{ \textit{kai} (DUB)}} & \multicolumn{5}{X}{}\\
\multicolumn{2}{X}{{ ming-uba}} & \multicolumn{4}{X}{{ *ming-uboo}} & \multicolumn{3}{X}{{ *ming-ugadɨ}} & \multicolumn{3}{X}{{ ming-una}} & \multicolumn{6}{X}{{ *ming=ukai}} & \multicolumn{5}{X}{}\\
\multicolumn{2}{X}{grab-CSL} & \multicolumn{4}{X}{grab-CND} & \multicolumn{3}{X}{grab-until} & \multicolumn{3}{X}{grab-PROH} & \multicolumn{6}{X}{grab=DUB} & \multicolumn{5}{X}{}\\
\lspbottomrule
\end{tabularx}
Stem No. 17 (ending with //in//): \textit{sin-} ‘die’

\tablefirsthead{}

\tabletail{}
\tablelasttail{}
\begin{tabularx}{\textwidth}{XXXXXXXXXXXXXXXXXXXXXX}
\lsptoprule
\multicolumn{22}{X}{{\bfseries Type-A affixes}}\\
{ \textit{{}-an} (NEG)} & \multicolumn{4}{X}{{ \textit{{}-ar(ɨr)} (PASS)}} & \multicolumn{3}{X}{{ \textit{{}-ar(ɨr)} (CAP)}} & \multicolumn{5}{X}{{ \textit{{}-as} (CAUS)}} & \multicolumn{4}{X}{{ \textit{{}-azɨi} (NEG.PLQ)}} & \multicolumn{2}{X}{{ \textit{{}-ɨ} (IMP)}} & { \textit{{}-ɨba} (SUGS)} & { \textit{{}-oo}(INT)} & \\
{ sin-jan} & \multicolumn{4}{X}{{ sin-jat-tɨ}} & \multicolumn{3}{X}{{ sin-jarɨp-poo}} & \multicolumn{5}{X}{{ sin-ja-cja-n}} & \multicolumn{4}{X}{{ sin-jazɨi}} & \multicolumn{2}{X}{{ sin-jɨ}} & { sin-ba} & { sin-joo} & \\
dile-NEG & \multicolumn{4}{X}{dile-PASS-SEQ} & \multicolumn{3}{X}{dile-CAP-CND} & \multicolumn{5}{X}{dile-CAUS-PST-PTCP} & \multicolumn{4}{X}{dile-NEG.PLQ} & \multicolumn{2}{X}{dile-IMP} & dile-SUGS & dile-INT & \\
\multicolumn{22}{X}{}\\
\multicolumn{22}{X}{{\bfseries Type-B affixes}}\\
\multicolumn{2}{X}{{ \textit{{}-tar} (PST)}} & \multicolumn{3}{X}{{ \textit{{}-tuk} (PRPR)}} & \multicolumn{4}{X}{{ \textit{{}-tur} (PROG)}} & \multicolumn{3}{X}{{ \textit{{}-təər} (RSL)}} & \multicolumn{3}{X}{{ \textit{{}-tɨ} (SEQ)}} & \multicolumn{3}{X}{{ \textit{{}-tai} (LST)}} & \multicolumn{4}{X}{{ \textit{{}-təəra} ‘after’}}\\
\multicolumn{2}{X}{{ si-zja}} & \multicolumn{3}{X}{{ ?}} & \multicolumn{4}{X}{{ si-zjup-pa}} & \multicolumn{3}{X}{{ si-zjəəp-pa}} & \multicolumn{3}{X}{{ si-zjɨ}} & \multicolumn{3}{X}{{ *si-zjai}} & \multicolumn{4}{X}{{ *si-zjəəra}}\\
\multicolumn{2}{X}{dile-PST} & \multicolumn{3}{X}{} & \multicolumn{4}{X}{dile-PROG-CSL} & \multicolumn{3}{X}{dile-RSL-CSL} & \multicolumn{3}{X}{dile-SEQ} & \multicolumn{3}{X}{dile-LST} & \multicolumn{4}{X}{{ dile-after}}\\
\multicolumn{2}{X}{} & \multicolumn{3}{X}{} & \multicolumn{4}{X}{} & \multicolumn{3}{X}{} & \multicolumn{3}{X}{} & \multicolumn{3}{X}{} & \multicolumn{4}{X}{}\\
\multicolumn{22}{X}{{\bfseries Type-C affixes}}\\
\multicolumn{4}{X}{{ \textit{{}-jawur} (POL)}} & \multicolumn{3}{X}{{ \textit{{}-jaa} ‘person’}} & \multicolumn{3}{X}{{ \textit{{}-jur} (UMRK)}} & \multicolumn{4}{X}{{ \textit{{}-jagacinaa} (SIM)}} & \multicolumn{8}{X}{}\\
\multicolumn{4}{X}{{ *sin-jawu-i}} & \multicolumn{3}{X}{{ ?}} & \multicolumn{3}{X}{sin-juk=kai} & \multicolumn{4}{X}{{ ?}} & \multicolumn{8}{X}{}\\
\multicolumn{4}{X}{dile-POL-NPST} & \multicolumn{3}{X}{} & \multicolumn{3}{X}{dile-UMRK=DUB} & \multicolumn{4}{X}{} & \multicolumn{8}{X}{}\\
\multicolumn{22}{X}{}\\
\multicolumn{22}{X}{{\bfseries Type-D affixes and clitic}}\\
\multicolumn{3}{X}{{ \textit{{}-ba} (CSL)}} & \multicolumn{3}{X}{{ \textit{{}-boo} (CND)}} & \multicolumn{2}{X}{{ \textit{{}-gadɨ} ‘until’}} & \multicolumn{3}{X}{{ \textit{{}-na} (PROH)}} & \multicolumn{5}{X}{{ \textit{kai} (DUB)}} & \multicolumn{6}{X}{}\\
\multicolumn{3}{X}{{ sin-ba}} & \multicolumn{3}{X}{{ *sin-boo}} & \multicolumn{2}{X}{{ *sin-gadɨ}} & \multicolumn{3}{X}{{ sin-na}} & \multicolumn{5}{X}{{ sin=kai}} & \multicolumn{6}{X}{}\\
\multicolumn{3}{X}{dile-CSL} & \multicolumn{3}{X}{dile-CND} & \multicolumn{2}{X}{dile-until} & \multicolumn{3}{X}{dile-PROH} & \multicolumn{5}{X}{dile=DUB} & \multicolumn{6}{X}{}\\
\lspbottomrule
\end{tabularx}
Irregular type verbal stems (a): \textit{sɨr-} ‘do’

\tablefirsthead{}

\tabletail{}
\tablelasttail{}
\begin{tabularx}{\textwidth}{XXXXXXXXXXXXXXXXXXXXX}
\lsptoprule
\multicolumn{21}{X}{{\bfseries Type-A affixes}}\\
{ \textit{{}-an} (NEG)} & \multicolumn{4}{X}{{ \textit{{}-ar(ɨr)} (PASS)}} & \multicolumn{5}{X}{{ \textit{{}-ar(ɨr)} (CAP)}} & \multicolumn{4}{X}{{ \textit{{}-as} (CAUS)}} & \multicolumn{3}{X}{{ \textit{{}-azɨi} (NEG.PLQ)}} & \multicolumn{2}{X}{{ \textit{{}-ɨ} (IMP)}} & { \textit{{}-ɨba} (SUGS)} & { \textit{{}-oo}(INT)}\\
{ sɨr-an} & \multicolumn{4}{X}{{ sɨr-at-ta}} & \multicolumn{5}{X}{{ sɨr-arɨ-i}} & \multicolumn{4}{X}{{ sɨr-as-oo}} & \multicolumn{3}{X}{{ sɨr-azɨi}} & \multicolumn{2}{X}{{ sɨr-ɨ}} & { sɨr-ɨba} & { sɨr-oo}\\
do-NEG & \multicolumn{4}{X}{do-PASS-PST} & \multicolumn{5}{X}{do-CAP-NPST} & \multicolumn{4}{X}{do-CAUS-INT} & \multicolumn{3}{X}{do-NEG.PLQ} & \multicolumn{2}{X}{do-IMP} & do-SUGS & do-INT\\
\multicolumn{21}{X}{}\\
\multicolumn{21}{X}{{\bfseries Type-B affixes}}\\
\multicolumn{3}{X}{{ \textit{{}-tar} (PST)}} & \multicolumn{4}{X}{{ \textit{{}-tuk} (PRPR)}} & \multicolumn{4}{X}{{ \textit{{}-tur} (PROG)}} & \multicolumn{4}{X}{{ \textit{{}-təər} (RSL)}} & { \textit{{}-tɨ} (SEQ)} & \multicolumn{2}{X}{{ \textit{{}-tai} (LST)}} & \multicolumn{3}{X}{{ \textit{{}-təəra} ‘after’}}\\
\multicolumn{3}{X}{{ sjat=too}} & \multicolumn{4}{X}{{ sjuk-uba}} & \multicolumn{4}{X}{{ sju-i}} & \multicolumn{4}{X}{{ sjəə=sɨ}} & { sjɨ} & \multicolumn{2}{X}{{ sjai}} & \multicolumn{3}{X}{{ *sjəəra}}\\
\multicolumn{3}{X}{do.PST=ASS} & \multicolumn{4}{X}{do.PRPR-CSL} & \multicolumn{4}{X}{do.PROG-NPST} & \multicolumn{4}{X}{do.RSL=FN} & do.SEQ & \multicolumn{2}{X}{do.LST} & \multicolumn{3}{X}{{ do.after}}\\
\multicolumn{3}{X}{} & \multicolumn{4}{X}{} & \multicolumn{4}{X}{} & \multicolumn{4}{X}{} &  & \multicolumn{2}{X}{} & \multicolumn{3}{X}{}\\
\multicolumn{21}{X}{{\bfseries Type-C affixes}}\\
\multicolumn{4}{X}{{ \textit{{}-jawur} (POL)}} & \multicolumn{4}{X}{{ \textit{{}-jaa} ‘person’}} & \multicolumn{4}{X}{{ \textit{{}-jur} (UMRK)}} & \multicolumn{9}{X}{{ \textit{{}-jagacinaa} (SIM)}}\\
\multicolumn{4}{X}{{ *s-jawu-i}} & \multicolumn{4}{X}{{ s-jaa}} & \multicolumn{4}{X}{s-ju-i} & \multicolumn{9}{X}{{ s-jaagacinaa}}\\
\multicolumn{4}{X}{do-POL-NPST} & \multicolumn{4}{X}{do-person} & \multicolumn{4}{X}{do-UMRK-NPST} & \multicolumn{9}{X}{do-SIM}\\
\multicolumn{21}{X}{}\\
\multicolumn{21}{X}{{\bfseries Type-D affixes and clitic}}\\
\multicolumn{2}{X}{{ \textit{{}-ba} (CSL)}} & \multicolumn{4}{X}{{ \textit{{}-boo} (CND)}} & \multicolumn{3}{X}{{ \textit{{}-gadɨ} ‘until’}} & \multicolumn{4}{X}{{ \textit{{}-na} (PROH)}} & \multicolumn{8}{X}{{ \textit{kai} (DUB)}}\\
\multicolumn{2}{X}{{ sɨp-pa}} & \multicolumn{4}{X}{{ sɨp-poo}} & \multicolumn{3}{X}{{ sɨk-kadɨ}} & \multicolumn{4}{X}{{ sɨn-na}} & \multicolumn{8}{X}{{ sɨk=kai}}\\
\multicolumn{2}{X}{do-CSL} & \multicolumn{4}{X}{do-CND} & \multicolumn{3}{X}{do-until} & \multicolumn{4}{X}{do-PROH} & \multicolumn{8}{X}{do=DUB}\\
\lspbottomrule
\end{tabularx}
Notes: \textit{sɨr-} ‘do’ and \textit{moosɨr-} (die.HON) behave like the verbal stem No. 1 (ending with //V\textsubscript{non-back}r//) except for the following cases.

(i)  Type-B affixes are fused with the preceding verbal root, e.g. \textit{sɨr-tar} (do-PST) > /sja/ (not /sɨ-ta/);

(ii)  Before the type-C affixes, \textit{sɨr-} ‘do’ becomes /s/, and \textit{moosɨr-} (die.HON) becomes /moos/;

(iii)  Before the infinitival affix, \textit{sɨr-} ‘do’ becomes /s/, and \textit{moosɨr-} (die.HON) becomes /moos/ (see also the final page of the appendix).

Irregular type verbal stems (b): \textit{k-} ‘come’

\tablefirsthead{}

\tabletail{}
\tablelasttail{}
\begin{tabularx}{\textwidth}{XXXXXXXXXXXXXXXXXXXXX}
\lsptoprule
\multicolumn{21}{X}{{\bfseries Type-A affixes}}\\
\multicolumn{2}{X}{{ \textit{{}-on} (NEG)}} & \multicolumn{5}{X}{{ \textit{{}-oor(ɨr)} (PASS)}} & \multicolumn{5}{X}{{ \textit{{}-oor(ɨr)} (CAP)}} & \multicolumn{3}{X}{{ \textit{{}-oos} (CAUS)}} & \multicolumn{3}{X}{{ \textit{{}-oozɨi} (NEG.PLQ)}} & { \textit{{}-oo} (IMP)} & { \textit{{}-ooba} (SUGS)} & { \textit{{}-oo}(INT)}\\
\multicolumn{2}{X}{{ k-on}} & \multicolumn{5}{X}{{ k-oorɨp-poo}} & \multicolumn{5}{X}{{ k-oorɨ-n=nja}} & \multicolumn{3}{X}{{ k-oos-an}} & \multicolumn{3}{X}{{ k-oozɨi}} & { k-oo} & { k-ooba} & { k-oo}\\
\multicolumn{2}{X}{come-NEG} & \multicolumn{5}{X}{come-PASS-CND} & \multicolumn{5}{X}{come-CAP-NPST=PLQ} & \multicolumn{3}{X}{come-CAUS-NEG} & \multicolumn{3}{X}{come-NEG.PLQ} & come-IMP & come-SUGS & come-INT\\
\multicolumn{21}{X}{}\\
\multicolumn{21}{X}{{\bfseries Type-B affixes}}\\
{ \textit{{}-tar} (PST)} & \multicolumn{4}{X}{{ \textit{{}-tuk} (PRPR)}} & \multicolumn{5}{X}{{ \textit{{}-tur} (PROG)}} & \multicolumn{4}{X}{{ \textit{{}-təər} (RSL)}} & \multicolumn{2}{X}{{ \textit{{}-tɨ} (SEQ)}} & { \textit{{}-tai} (LST)} & \multicolumn{4}{X}{{ \textit{{}-təəra} ‘after’}}\\
{ cˀja} & \multicolumn{4}{X}{{ ?}} & \multicolumn{5}{X}{{ cˀjup-pa}} & \multicolumn{4}{X}{{ cˀjən}} & \multicolumn{2}{X}{{ cˀjɨ}} & { cˀjai} & \multicolumn{4}{X}{{ *cˀjəəra}}\\
come.PST & \multicolumn{4}{X}{} & \multicolumn{5}{X}{come.PROG-CSL} & \multicolumn{4}{X}{come.RSL-PTCP} & \multicolumn{2}{X}{come.SEQ} & come.LST & \multicolumn{4}{X}{{ come.after}}\\
& \multicolumn{4}{X}{} & \multicolumn{5}{X}{} & \multicolumn{4}{X}{} & \multicolumn{2}{X}{} &  & \multicolumn{4}{X}{}\\
\multicolumn{21}{X}{{\bfseries Type-C affixes}}\\
\multicolumn{4}{X}{{ \textit{{}-jawur} (POL)}} & \multicolumn{4}{X}{{ \textit{{}-jaa} ‘person’}} & \multicolumn{3}{X}{{ \textit{{}-jur} (UMRK)}} & \multicolumn{10}{X}{{ \textit{{}-jagacinaa} (SIM)}}\\
\multicolumn{4}{X}{{ *k-jawu-i}} & \multicolumn{4}{X}{{ ?}} & \multicolumn{3}{X}{k-ju-i} & \multicolumn{10}{X}{{ k-jaagacinaa}}\\
\multicolumn{4}{X}{come-POL-NPST} & \multicolumn{4}{X}{} & \multicolumn{3}{X}{come-UMRK-NPST} & \multicolumn{10}{X}{come-SIM}\\
\multicolumn{21}{X}{}\\
\multicolumn{21}{X}{{\bfseries Type-D affixes and clitic}}\\
\multicolumn{3}{X}{{ \textit{{}-ba} (CSL)}} & \multicolumn{3}{X}{{ \textit{{}-boo} (CND)}} & \multicolumn{3}{X}{{ \textit{{}-gadɨ} ‘until’}} & \multicolumn{4}{X}{{ \textit{{}-na} (PROH)}} & \multicolumn{8}{X}{{ \textit{kai} (DUB)}}\\
\multicolumn{3}{X}{{ kˀ-uuba}} & \multicolumn{3}{X}{{ kˀ-uuboo}} & \multicolumn{3}{X}{{ kˀ-uugadɨ}} & \multicolumn{4}{X}{{ kˀ-uuna}} & \multicolumn{8}{X}{{ *kˀ=uukai}}\\
\multicolumn{3}{X}{come-CSL} & \multicolumn{3}{X}{come-CND} & \multicolumn{3}{X}{come-until} & \multicolumn{4}{X}{come-PROH} & \multicolumn{8}{X}{come=DUB}\\
\lspbottomrule
\end{tabularx}
Notes: \textit{k-} ‘come’ and \textit{tɨkk-} ‘bring’ behave like the verbal stem No. 7 (ending with //V\textsubscript{non-}\textit{\textsubscript{i}} k//) except for the following cases.

(i)  The initial vowel of the type-A affixes is //oo// (or //o//);

(ii)  Type-B affixes are fused with the preceding verbal root \textit{k-} ‘come,’ e.g. \textit{k-tar} (do-PST) > /cˀja/;

(iii)  Before the type-D affixes, \textit{k-} ‘come’ becomes /kˀ/.

Irregular type verbal stems (c): \textit{ik-} ‘go’

\tablefirsthead{}

\tabletail{}
\tablelasttail{}
\begin{tabularx}{\textwidth}{XXXXXXXXXXXXXXXXXXXX}
\lsptoprule
\multicolumn{20}{X}{{\bfseries Type-A affixes}}\\
\multicolumn{2}{X}{{ \textit{{}-an} (NEG)}} & \multicolumn{4}{X}{{ \textit{{}-ar(ɨr)} (PASS)}} & \multicolumn{5}{X}{{ \textit{{}-ar(ɨr)} (CAP)}} & \multicolumn{4}{X}{{ \textit{{}-as} (CAUS)}} & \multicolumn{2}{X}{{ \textit{{}-azɨi} (NEG.PLQ)}} & { \textit{{}-ɨ} (IMP)} & { \textit{{}-ɨba} (SUGS)} & { \textit{{}-oo}(INT)}\\
\multicolumn{2}{X}{{ ik-jan}} & \multicolumn{4}{X}{{ ik-jat-tɨ}} & \multicolumn{5}{X}{{ ik-jarɨ-n=nja}} & \multicolumn{4}{X}{{ ik-jas-ju-i}} & \multicolumn{2}{X}{{ ik-jazɨi}} & { ik-jɨ} & { ik-jɨba} & { ik-joo}\\
\multicolumn{2}{X}{go-NEG} & \multicolumn{4}{X}{go-PASS-SEQ} & \multicolumn{5}{X}{go-CAP-NPST=PLQ} & \multicolumn{4}{X}{go-CAUS-UMRK-NPST} & \multicolumn{2}{X}{go-NEG.PLQ} & go-IMP & go-SUGS & go-INT\\
\multicolumn{20}{X}{}\\
\multicolumn{20}{X}{{\bfseries Type-B affixes}}\\
{ \textit{{}-tar} (PST)} & \multicolumn{4}{X}{{ \textit{{}-tuk} (PRPR)}} & \multicolumn{4}{X}{{ \textit{{}-tur} (PROG)}} & \multicolumn{4}{X}{{ \textit{{}-təər} (RSL)}} & { \textit{{}-tɨ} (SEQ)} & \multicolumn{2}{X}{{ \textit{{}-tai} (LST)}} & \multicolumn{4}{X}{{ \textit{{}-təəra} ‘after’}}\\
{ i-zja} & \multicolumn{4}{X}{{ ?}} & \multicolumn{4}{X}{{ i-zjur-ɨ}} & \multicolumn{4}{X}{{ i-zjəəp-pa}} & { i-zjɨ} & \multicolumn{2}{X}{{ i-zjai}} & \multicolumn{4}{X}{{ *i-zjəəra}}\\
go-PST & \multicolumn{4}{X}{} & \multicolumn{4}{X}{go-PROG-IMP} & \multicolumn{4}{X}{go-RSL-CSL} & go-SEQ & \multicolumn{2}{X}{go-LST} & \multicolumn{4}{X}{{ go-after}}\\
& \multicolumn{4}{X}{} & \multicolumn{4}{X}{} & \multicolumn{4}{X}{} &  & \multicolumn{2}{X}{} & \multicolumn{4}{X}{}\\
\multicolumn{20}{X}{{\bfseries Type-C affixes}}\\
\multicolumn{3}{X}{{ \textit{{}-jawur} (POL)}} & \multicolumn{4}{X}{{ \textit{{}-jaa} ‘person’}} & \multicolumn{3}{X}{{ \textit{{}-jur} (UMRK)}} & \multicolumn{10}{X}{{ \textit{{}-jagacinaa} (SIM)}}\\
\multicolumn{3}{X}{{ *ik-jawu-i}} & \multicolumn{4}{X}{*ik-jaa} & \multicolumn{3}{X}{ik-ju-i} & \multicolumn{10}{X}{{ ik-jagacinaa}}\\
\multicolumn{3}{X}{go-POL-NPST} & \multicolumn{4}{X}{go-person} & \multicolumn{3}{X}{go-UMRK-NPST} & \multicolumn{10}{X}{go-SIM}\\
\multicolumn{20}{X}{}\\
\multicolumn{20}{X}{{\bfseries Type-D affixes and clitic}}\\
{ \textit{{}-ba} (CSL)} & \multicolumn{3}{X}{{ \textit{{}-boo} (CND)}} & \multicolumn{4}{X}{{ \textit{{}-gadɨ} ‘until’}} & \multicolumn{4}{X}{{ \textit{{}-na} (PROH)}} & \multicolumn{8}{X}{{ \textit{kai} (DUB)}}\\
{ ik-uba} & \multicolumn{3}{X}{{ ik-uboo}} & \multicolumn{4}{X}{{ ik-ugadɨ}} & \multicolumn{4}{X}{{ ik-una}} & \multicolumn{8}{X}{{ *ik=ukai}}\\
go-CSL & \multicolumn{3}{X}{go-CND} & \multicolumn{4}{X}{go-until} & \multicolumn{4}{X}{go-PROH} & \multicolumn{8}{X}{go=DUB}\\
\lspbottomrule
\end{tabularx}
Note: \textit{ik-} ‘go’ behaves like the verbal stem No. 15 (ending with //ik//) except for the following case.

(i)  The initial consonant of the type-B affixes becomes /zj/ (not /cj/) after \textit{ik-} ‘go.’

Irregular type verbal stems (d): \textit{umoor-} (move.HON)

\tablefirsthead{}

\tabletail{}
\tablelasttail{}
\begin{tabularx}{\textwidth}{XXXXXXXXXXXXXXXXXXXX}
\lsptoprule
\multicolumn{20}{X}{{\bfseries Type-A affixes}}\\
\multicolumn{2}{X}{{ \textit{{}-an} (NEG)}} & \multicolumn{5}{X}{{ \textit{{}-ar(ɨr)} (PASS)}} & \multicolumn{2}{X}{{ \textit{{}-ar(ɨr)} (CAP)}} & \multicolumn{3}{X}{{ \textit{{}-as} (CAUS)}} & \multicolumn{3}{X}{{ \textit{{}-azɨi} (NEG.PLQ)}} & \multicolumn{2}{X}{{ \textit{{}-ɨ} (IMP)}} & \multicolumn{2}{X}{{ \textit{{}-ɨba} (SUGS)}} & { \textit{{}-oo}(INT)}\\
\multicolumn{2}{X}{{ umoor-an}} & \multicolumn{5}{X}{{ umoor-at-tat-tu}} & \multicolumn{2}{X}{{ umoor-arɨ-n=nja}} & \multicolumn{3}{X}{{ umoor-as-an-boo}} & \multicolumn{3}{X}{{ umoor-azɨi}} & \multicolumn{2}{X}{{ umoor-ɨ}} & \multicolumn{2}{X}{{ umoor-ɨba}} & { umoor-oo}\\
\multicolumn{2}{X}{move.HON-NEG} & \multicolumn{5}{X}{move.HON-PASS-PST-CSL} & \multicolumn{2}{X}{move.HON-CAP-NPST=PLQ} & \multicolumn{3}{X}{move.HON-CAUS-NEG-CND} & \multicolumn{3}{X}{move.HON-NEG.PLQ} & \multicolumn{2}{X}{move.HON-IMP} & \multicolumn{2}{X}{move.HON-SUGS} & move.HON-INT\\
\multicolumn{20}{X}{}\\
\multicolumn{20}{X}{{\bfseries Type-B affixes}}\\
\multicolumn{3}{X}{{ \textit{{}-tar} (PST)}} & \multicolumn{3}{X}{{ \textit{{}-tuk} (PRPR)}} & \multicolumn{5}{X}{{ \textit{{}-tur} (PROG)}} & \multicolumn{3}{X}{{ \textit{{}-təər} (RSL)}} & \multicolumn{2}{X}{{ \textit{{}-tɨ} (SEQ)}} & \multicolumn{2}{X}{{ \textit{{}-tai} (LST)}} & \multicolumn{2}{X}{{ \textit{{}-təəra} ‘after’}}\\
\multicolumn{3}{X}{{ umoo-cja}} & \multicolumn{3}{X}{{ ?}} & \multicolumn{5}{X}{{ umoo-cjuk=ka}} & \multicolumn{3}{X}{{ umoo-cjə-i}} & \multicolumn{2}{X}{{ umoo-cjɨ}} & \multicolumn{2}{X}{{ *umoo-cjai}} & \multicolumn{2}{X}{{ *umoo-cjəəra}}\\
\multicolumn{3}{X}{move.HON-PST} & \multicolumn{3}{X}{} & \multicolumn{5}{X}{move.HON-PROG=DUB} & \multicolumn{3}{X}{move.HON-RSL-NPST} & \multicolumn{2}{X}{move.HON-SEQ} & \multicolumn{2}{X}{move.HON-LST} & \multicolumn{2}{X}{{ move.HON-after}}\\
\multicolumn{3}{X}{} & \multicolumn{3}{X}{} & \multicolumn{5}{X}{} & \multicolumn{3}{X}{} & \multicolumn{2}{X}{} & \multicolumn{2}{X}{} & \multicolumn{2}{X}{}\\
\multicolumn{20}{X}{{\bfseries Type-C affixes}}\\
{ \textit{{}-jawur} (POL)} & \multicolumn{4}{X}{{ \textit{{}-jaa} ‘person’}} & \multicolumn{6}{X}{{ \textit{{}-jur} (UMRK)}} & \multicolumn{9}{X}{{ \textit{{}-jagacinaa} (SIM)}}\\
{ ?} & \multicolumn{4}{X}{?} & \multicolumn{6}{X}{umoo-ju-i} & \multicolumn{9}{X}{{ umoo-jagacinaa}}\\
& \multicolumn{4}{X}{} & \multicolumn{6}{X}{move.HON-UMRK-NPST} & \multicolumn{9}{X}{move.HON-SIM}\\
\multicolumn{20}{X}{}\\
\multicolumn{20}{X}{{\bfseries Type-D affixes and clitic}}\\
\multicolumn{4}{X}{{ \textit{{}-ba} (CSL)}} & \multicolumn{4}{X}{{ \textit{{}-boo} (CND)}} & \multicolumn{2}{X}{{ \textit{{}-gadɨ} ‘until’}} & \multicolumn{3}{X}{{ \textit{{}-na} (PROH)}} & \multicolumn{7}{X}{{ \textit{kai} (DUB)}}\\
\multicolumn{4}{X}{{ umoop-pa}} & \multicolumn{4}{X}{{ *umoop-poo}} & \multicolumn{2}{X}{{ *umook-kadɨ}} & \multicolumn{3}{X}{{ umoon-na}} & \multicolumn{7}{X}{{ *umook=kai}}\\
\multicolumn{4}{X}{move.HON-CSL} & \multicolumn{4}{X}{move.HON-CND} & \multicolumn{2}{X}{move.HON-until} & \multicolumn{3}{X}{move.HON-PROH} & \multicolumn{7}{X}{move.HON=DUB}\\
\lspbottomrule
\end{tabularx}
Note: The honorific verbs such as \textit{umoor-} (move.HON) behaves like the verbal stem No. 2 (ending with //V\textsubscript{back}r//) except for the following case.

(i)  The initial consonant of the type-B affixes become /cj/ (not /t/) after honorific verbs (although \textit{moosɨr-} (die.HON) behaves like \textit{sɨr-} ‘do’).

Irregular type verbal stems (e): \textit{hijaw-} ‘pick up’

\tablefirsthead{}

\tabletail{}
\tablelasttail{}
\begin{tabularx}{\textwidth}{XXXXXXXm{4.5984238E-4in}XXXXXXXXXXX}
\lsptoprule
\multicolumn{19}{X}{{\bfseries Type-A affixes}}\\
{ \textit{{}-an} (NEG)} & \multicolumn{5}{X}{{ \textit{{}-ar(ɨr)} (PASS)}} & \multicolumn{3}{X}{{ \textit{{}-ar(ɨr)} (CAP)}} & \multicolumn{4}{X}{{ \textit{{}-as} (CAUS)}} & \multicolumn{2}{X}{{ \textit{{}-azɨi} (NEG.PLQ)}} & \multicolumn{2}{X}{{ \textit{{}-ɨ} (IMP)}} & { \textit{{}-ɨba} (SUGS)} & { \textit{{}-oo}(INT)}\\
{ hijaw-an} & \multicolumn{5}{X}{{ hijoo-t-tat-tu}} & \multicolumn{3}{X}{{ hijoo-r-an-ta}} & \multicolumn{4}{X}{{ hijoo-s-oo}} & \multicolumn{2}{X}{{ hijaw-azɨi}} & \multicolumn{2}{X}{{ hijaw-ɨ}} & { hijaw-ɨba} & { hijaw-oo}\\
pick.up-NEG & \multicolumn{5}{X}{pick.up-PASS-PST-CSL} & \multicolumn{3}{X}{pick.up-CAP-NEG-PST} & \multicolumn{4}{X}{pick.up-CAUS-INT} & \multicolumn{2}{X}{pick.up-NEG.PLQ} & \multicolumn{2}{X}{pick.up-IMP} & pick.up-SUGS & pick.up-INT\\
\multicolumn{19}{X}{}\\
\multicolumn{19}{X}{{\bfseries Type-B affixes}}\\
{ \textit{{}-tar} (PST)} & \multicolumn{3}{X}{{ \textit{{}-tuk} (PRPR)}} & \multicolumn{4}{X}{{ \textit{{}-tur} (PROG)}} & \multicolumn{4}{X}{{ \textit{{}-təər} (RSL)}} & \multicolumn{2}{X}{{ \textit{{}-tɨ} (SEQ)}} & \multicolumn{2}{X}{{ \textit{{}-tai} (LST)}} & \multicolumn{3}{X}{{ \textit{{}-təəra} ‘after’}}\\
{ hija-ta} & \multicolumn{3}{X}{{ hija-tuk-ɨ}} & \multicolumn{4}{X}{{ hija-tut=too}} & \multicolumn{4}{X}{{ hija-təəp-pa}} & \multicolumn{2}{X}{{ hija-tɨ}} & \multicolumn{2}{X}{{ *hija-tai}} & \multicolumn{3}{X}{{ *hija-təəra}}\\
pick.up-PST & \multicolumn{3}{X}{pick.up-PRPR-IMP} & \multicolumn{4}{X}{pick.up-PROG=ASS} & \multicolumn{4}{X}{pick.up-RSL-CSL} & \multicolumn{2}{X}{pick.up-SEQ} & \multicolumn{2}{X}{pick.up-LST} & \multicolumn{3}{X}{{ pick.up-after}}\\
& \multicolumn{3}{X}{} & \multicolumn{4}{X}{} & \multicolumn{4}{X}{} & \multicolumn{2}{X}{} & \multicolumn{2}{X}{} & \multicolumn{3}{X}{}\\
\multicolumn{19}{X}{{\bfseries Type-C affixes}}\\
\multicolumn{3}{X}{{ \textit{{}-jawur} (POL)}} & \multicolumn{2}{X}{{ \textit{{}-jaa} ‘person’}} & \multicolumn{5}{X}{{ \textit{{}-jur} (UMRK)}} & \multicolumn{9}{X}{{ \textit{{}-jagacinaa} (SIM)}}\\
\multicolumn{3}{X}{{ *hija-jawu-i}} & \multicolumn{2}{X}{hija-jaa} & \multicolumn{5}{X}{hija-ju=sə=ə} & \multicolumn{9}{X}{{ hijəə-jagacinaa}}\\
\multicolumn{3}{X}{pick.up-POL-NPST} & \multicolumn{2}{X}{pick.up-person} & \multicolumn{5}{X}{pick.up-UMRK=FN=TOP} & \multicolumn{9}{X}{pick.up-SIM}\\
\multicolumn{19}{X}{}\\
\multicolumn{19}{X}{{\bfseries Type-D affixes and clitic}}\\
\multicolumn{2}{X}{{ \textit{{}-ba} (CSL)}} & \multicolumn{3}{X}{{ \textit{{}-boo} (CND)}} & \multicolumn{2}{X}{{ \textit{{}-gadɨ} ‘until’}} & \multicolumn{4}{X}{{ \textit{{}-na} (PROH)}} & \multicolumn{8}{X}{{ \textit{kai} (DUB)}}\\
\multicolumn{2}{X}{{ hijəə-ba}} & \multicolumn{3}{X}{{ *hijəə-boo}} & \multicolumn{2}{X}{{ hijəə-gadɨ}} & \multicolumn{4}{X}{{ hijəə-na}} & \multicolumn{8}{X}{{ *hijəə=kai}}\\
\multicolumn{2}{X}{pick.up-CSL} & \multicolumn{3}{X}{pick.up-CND} & \multicolumn{2}{X}{pick.up-until} & \multicolumn{4}{X}{pick.up-PROH} & \multicolumn{8}{X}{pick.up=DUB}\\
\lspbottomrule
\end{tabularx}
Notes: The verbal stems that end with //aw// behave like the verbal stem No. 2 (ending with //V\textsubscript{back}w//) except for the following cases.

(i)  The stem-final //aw// becomes /oo/ before \textit{{}-ar(ɨr)} (PASS), \textit{{}-ar(ɨr)} (CAP) or \textit{{}-as} (CAUS), and also these affixes delete their initial vowels;

(ii)  The stem-final //aw// becomes /əə/ before \textit{{}-jagacinaa} (SIM), the type-D affixes and clitic, or the infinitival affix (see aslo the final page of the appendix).

Irregular type verbal stems (f): \textit{sij-} ‘know’

\tablefirsthead{}

\tabletail{}
\tablelasttail{}
\begin{tabularx}{\textwidth}{XXXXXXXXXXXXXXXXXXXXXX}
\lsptoprule
\multicolumn{22}{X}{{\bfseries Type-A affixes}}\\
{ \textit{{}-an} (NEG)} & \multicolumn{7}{X}{{ \textit{{}-ar(ɨr)} (PASS)}} & \multicolumn{2}{X}{{ \textit{{}-ar(ɨr)} (CAP)}} & \multicolumn{4}{X}{{ \textit{{}-as} (CAUS)}} & \multicolumn{4}{X}{{ \textit{{}-azɨi} (NEG.PLQ)}} & \multicolumn{2}{X}{{ \textit{{}-ɨ} (IMP)}} & { \textit{{}-ɨba} (SUGS)} & { \textit{{}-oo}(INT)}\\
{ sij-an} & \multicolumn{7}{X}{{ sij-at-təəp-pa}} & \multicolumn{2}{X}{{ ?}} & \multicolumn{4}{X}{{ sij-as-oo}} & \multicolumn{4}{X}{{ sij-azɨi}} & \multicolumn{2}{X}{{ ?}} & { ?} & { sij-oo}\\
know-NEG & \multicolumn{7}{X}{know-PASS-RSL-CSL} & \multicolumn{2}{X}{} & \multicolumn{4}{X}{know-CAUS-INT} & \multicolumn{4}{X}{know-NEG.PLQ} & \multicolumn{2}{X}{} &  & know-INT\\
\multicolumn{22}{X}{}\\
\multicolumn{22}{X}{{\bfseries Type-B affixes}}\\
\multicolumn{3}{X}{{ \textit{{}-tar} (PST)}} & \multicolumn{3}{X}{{ \textit{{}-tuk} (PRPR)}} & \multicolumn{5}{X}{{ \textit{{}-tur} (PROG)}} & \multicolumn{4}{X}{{ \textit{{}-təər} (RSL)}} & \multicolumn{2}{X}{{ \textit{{}-tɨ} (SEQ)}} & \multicolumn{2}{X}{{ \textit{{}-tai} (LST)}} & \multicolumn{3}{X}{{ \textit{{}-təəra} ‘after’}}\\
\multicolumn{3}{X}{{ sic-cjat=too}} & \multicolumn{3}{X}{{ ?}} & \multicolumn{5}{X}{{ sic-cju-i}} & \multicolumn{4}{X}{{ sic-cjə-n}} & \multicolumn{2}{X}{{ *sic-cjɨ}} & \multicolumn{2}{X}{{ *sic-cjai}} & \multicolumn{3}{X}{{ *sic-cjəəra}}\\
\multicolumn{3}{X}{know-PST=ASS} & \multicolumn{3}{X}{} & \multicolumn{5}{X}{know-PROG-NPST} & \multicolumn{4}{X}{know-RSL-PTCP} & \multicolumn{2}{X}{know-SEQ} & \multicolumn{2}{X}{know-LST} & \multicolumn{3}{X}{{ know-after}}\\
\multicolumn{3}{X}{} & \multicolumn{3}{X}{} & \multicolumn{5}{X}{} & \multicolumn{4}{X}{} & \multicolumn{2}{X}{} & \multicolumn{2}{X}{} & \multicolumn{3}{X}{}\\
\multicolumn{22}{X}{{\bfseries Type-C affixes}}\\
\multicolumn{4}{X}{{ \textit{{}-jawur} (POL)}} & \multicolumn{3}{X}{{ \textit{{}-jaa} ‘person’}} & \multicolumn{5}{X}{{ \textit{{}-jur} (UMRK)}} & \multicolumn{4}{X}{{ \textit{{}-jagacinaa} (SIM)}} & \multicolumn{6}{X}{}\\
\multicolumn{4}{X}{{ ?}} & \multicolumn{3}{X}{?} & \multicolumn{5}{X}{?} & \multicolumn{4}{X}{{ ?}} & \multicolumn{6}{X}{}\\
\multicolumn{4}{X}{} & \multicolumn{3}{X}{} & \multicolumn{5}{X}{} & \multicolumn{4}{X}{} & \multicolumn{6}{X}{}\\
\multicolumn{22}{X}{}\\
\multicolumn{22}{X}{{\bfseries Type-D affixes and clitic}}\\
\multicolumn{2}{X}{{ \textit{{}-ba} (CSL)}} & \multicolumn{3}{X}{{ \textit{{}-boo} (CND)}} & \multicolumn{4}{X}{{ \textit{{}-gadɨ} ‘until’}} & \multicolumn{4}{X}{{ \textit{{}-na} (PROH)}} & \multicolumn{4}{X}{{ \textit{kai} (DUB)}} & \multicolumn{5}{X}{}\\
\multicolumn{2}{X}{?} & \multicolumn{3}{X}{{ ?}} & \multicolumn{4}{X}{{ ?}} & \multicolumn{4}{X}{{ ?}} & \multicolumn{4}{X}{{ ?}} & \multicolumn{5}{X}{}\\
\multicolumn{2}{X}{} & \multicolumn{3}{X}{} & \multicolumn{4}{X}{} & \multicolumn{4}{X}{} & \multicolumn{4}{X}{} & \multicolumn{5}{X}{}\\
\lspbottomrule
\end{tabularx}
Notes: \textit{sij-} ‘know’ behaves like the verbal stem No. 13 (ending with //ij//) except for the following case.

(i)  The stem-final consonant //j// of \textit{sij-} ‘know’ becomes /c/ before the type-B affixes, e.g. \textit{sij-tar} (know-PST) > /sic-cja/ (not /si-cja/).

Irregular type verbal stems (g): \textit{jurukub-} ‘happy’

\tablefirsthead{}

\tabletail{}
\tablelasttail{}
\begin{tabularx}{\textwidth}{XXXXXXXXXm{-2.4015456E-4in}XXXXXXXXX}
\lsptoprule
\multicolumn{19}{X}{{\bfseries Type-A affixes}}\\
\multicolumn{2}{X}{{ \textit{{}-an} (NEG)}} & \multicolumn{4}{X}{{ \textit{{}-ar(ɨr)} (PASS)}} & \multicolumn{3}{X}{{ \textit{{}-ar(ɨr)} (CAP)}} & \multicolumn{3}{X}{{ \textit{{}-as} (CAUS)}} & \multicolumn{2}{X}{{ \textit{{}-azɨi} (NEG.PLQ)}} & \multicolumn{2}{X}{{ \textit{{}-ɨ} (IMP)}} & \multicolumn{2}{X}{{ \textit{{}-ɨba} (SUGS)}} & { \textit{{}-oo}(INT)}\\
\multicolumn{2}{X}{{ jurukub-an}} & \multicolumn{4}{X}{{ jurukub-at-ta}} & \multicolumn{3}{X}{{ jurukub-ar-an}} & \multicolumn{3}{X}{{ jurukub-as-oo}} & \multicolumn{2}{X}{{ jurukub-azɨi}} & \multicolumn{2}{X}{{ jurukub-ɨ}} & \multicolumn{2}{X}{{ ?}} & { jurukub-oo}\\
\multicolumn{2}{X}{happy-NEG} & \multicolumn{4}{X}{happy-PASS-PST} & \multicolumn{3}{X}{happy-CAP-NEG} & \multicolumn{3}{X}{happy-CAUS-INT} & \multicolumn{2}{X}{happy-NEG.PLQ} & \multicolumn{2}{X}{happy-IMP} & \multicolumn{2}{X}{} & happy-INT\\
\multicolumn{19}{X}{}\\
\multicolumn{19}{X}{{\bfseries Type-B affixes}}\\
\multicolumn{3}{X}{{ \textit{{}-tar} (PST)}} & \multicolumn{4}{X}{{ \textit{{}-tuk} (PRPR)}} & \multicolumn{4}{X}{{ \textit{{}-tur} (PROG)}} & \multicolumn{2}{X}{{ \textit{{}-təər} (RSL)}} & \multicolumn{2}{X}{{ \textit{{}-tɨ} (SEQ)}} & \multicolumn{2}{X}{{ \textit{{}-tai} (LST)}} & \multicolumn{2}{X}{{ \textit{{}-təəra} ‘after’}}\\
\multicolumn{3}{X}{{ juruku-da}} & \multicolumn{4}{X}{{ ?}} & \multicolumn{4}{X}{{ juruku-dup-pa}} & \multicolumn{2}{X}{{ juruku-də-i}} & \multicolumn{2}{X}{{ juruku-dɨ}} & \multicolumn{2}{X}{{ *juruku-dai}} & \multicolumn{2}{X}{{ *juruku-dəəra}}\\
\multicolumn{3}{X}{happy-PST} & \multicolumn{4}{X}{} & \multicolumn{4}{X}{happy-PROG-CSL} & \multicolumn{2}{X}{happy-RSL-NPST} & \multicolumn{2}{X}{happy-SEQ} & \multicolumn{2}{X}{happy-LST} & \multicolumn{2}{X}{{ happy-after}}\\
\multicolumn{3}{X}{} & \multicolumn{4}{X}{} & \multicolumn{4}{X}{} & \multicolumn{2}{X}{} & \multicolumn{2}{X}{} & \multicolumn{2}{X}{} & \multicolumn{2}{X}{}\\
\multicolumn{19}{X}{{\bfseries Type-C affixes}}\\
\multicolumn{4}{X}{{ \textit{{}-jawur} (POL)}} & \multicolumn{2}{X}{{ \textit{{}-jaa} ‘person’}} & \multicolumn{4}{X}{{ \textit{{}-jur} (UMRK)}} & \multicolumn{9}{X}{{ \textit{{}-jagacinaa} (SIM)}}\\
\multicolumn{4}{X}{{ *jurukub-jawu-i}} & \multicolumn{2}{X}{{ ?}} & \multicolumn{4}{X}{?} & \multicolumn{9}{X}{{ jurukub-jagacinaa}}\\
\multicolumn{4}{X}{happy-POL-NPST} & \multicolumn{2}{X}{} & \multicolumn{4}{X}{} & \multicolumn{9}{X}{happy-SIM}\\
\multicolumn{19}{X}{}\\
\multicolumn{19}{X}{{\bfseries Type-D affixes and clitic}}\\
{ \textit{{}-ba} (CSL)} & \multicolumn{4}{X}{{ \textit{{}-boo} (CND)}} & \multicolumn{3}{X}{{ \textit{{}-gadɨ} ‘until’}} & \multicolumn{3}{X}{{ \textit{{}-na} (PROH)}} & \multicolumn{8}{X}{{ \textit{kai} (DUB)}}\\
{ jurukun-ba} & \multicolumn{4}{X}{{ jurukun-boo}} & \multicolumn{3}{X}{{ *jurukun-gadɨ}} & \multicolumn{3}{X}{{ jurukun-na}} & \multicolumn{8}{X}{{ *jurukun=kai}}\\
happy-CSL & \multicolumn{4}{X}{happy-CND} & \multicolumn{3}{X}{happy-until} & \multicolumn{3}{X}{happy-PROH} & \multicolumn{8}{X}{happy=DUB}\\
\lspbottomrule
\end{tabularx}
Notes: \textit{jurukub-} ‘happy’ behaves like the verbal stem No. 4 (ending with //b//) except for the following case.

(i)  The stem-final consonant //b// of \textit{jurukub-} ‘happy’ becomes /n/ (strictly speaking, the archiphoneme /N/) before the type-D affixes and clitic, e.g. \textit{jurukub-ba} (happy-CSL) > /jurukun-ba/ (not /jurukub-uba/).

Irregular type verbal stems (h): \textit{hənk-} ‘enter’

\tablefirsthead{}

\tabletail{}
\tablelasttail{}
\begin{tabularx}{\textwidth}{XXXXXXXXXXXXXXXXXXXXXX}
\lsptoprule
\multicolumn{22}{X}{{\bfseries Type-A affixes}}\\
{ \textit{{}-an} (NEG)} & \multicolumn{4}{X}{{ \textit{{}-ar(ɨr)} (PASS)}} & \multicolumn{4}{X}{{ \textit{{}-ar(ɨr)} (CAP)}} & \multicolumn{4}{X}{{ \textit{{}-as} (CAUS)}} & \multicolumn{5}{X}{{ \textit{{}-azɨi} (NEG.PLQ)}} & \multicolumn{2}{X}{{ \textit{{}-ɨ} (IMP)}} & { \textit{{}-ɨba} (SUGS)} & { \textit{{}-oo}(INT)}\\
hənk-jan & \multicolumn{4}{X}{hənk-jat-ta} & \multicolumn{4}{X}{hənk-jarɨk=kai} & \multicolumn{4}{X}{hənk-jas-oo} & \multicolumn{5}{X}{hənk-jazɨi} & \multicolumn{2}{X}{hənk-jɨ} & hənk-jɨba & hənk-joo\\
enter-NEG & \multicolumn{4}{X}{enter-PASS-PST} & \multicolumn{4}{X}{enter-CAP=DUB} & \multicolumn{4}{X}{enter-CAUS-INT} & \multicolumn{5}{X}{enter-NEG.PLQ} & \multicolumn{2}{X}{enter-IMP} & enter-SUGS & enter-INT\\
\multicolumn{22}{X}{}\\
\multicolumn{22}{X}{{\bfseries Type-B affixes}}\\
\multicolumn{2}{X}{{ \textit{{}-tar} (PST)}} & \multicolumn{4}{X}{{ \textit{{}-tuk} (PRPR)}} & \multicolumn{4}{X}{{ \textit{{}-tur} (PROG)}} & \multicolumn{4}{X}{{ \textit{{}-təər} (RSL)}} & \multicolumn{3}{X}{{ \textit{{}-tɨ} (SEQ)}} & \multicolumn{2}{X}{{ \textit{{}-tai} (LST)}} & \multicolumn{3}{X}{{ \textit{{}-təəra} ‘after’}}\\
\multicolumn{2}{X}{{ hən-cja}} & \multicolumn{4}{X}{{ hən-cjuk-ɨ}} & \multicolumn{4}{X}{{ hən-cjut=too}} & \multicolumn{4}{X}{{ hən-cjəəp-pa}} & \multicolumn{3}{X}{{ *hən-cjɨ}} & \multicolumn{2}{X}{{ *hən-cjai}} & \multicolumn{3}{X}{{ *hən-cjəəra}}\\
\multicolumn{2}{X}{enter-PST} & \multicolumn{4}{X}{enter-PRPR-IMP} & \multicolumn{4}{X}{enter-PROG=ASS} & \multicolumn{4}{X}{enter-RSL-CSL} & \multicolumn{3}{X}{enter-SEQ} & \multicolumn{2}{X}{enter-LST} & \multicolumn{3}{X}{{ enter-after}}\\
\multicolumn{2}{X}{} & \multicolumn{4}{X}{} & \multicolumn{4}{X}{} & \multicolumn{4}{X}{} & \multicolumn{3}{X}{} & \multicolumn{2}{X}{} & \multicolumn{3}{X}{}\\
\multicolumn{22}{X}{{\bfseries Type-C affixes}}\\
\multicolumn{3}{X}{{ \textit{{}-jawur} (POL)}} & \multicolumn{4}{X}{{ \textit{{}-jaa} ‘person’}} & \multicolumn{4}{X}{{ \textit{{}-jur} (UMRK)}} & \multicolumn{4}{X}{{ \textit{{}-jagacinaa} (SIM)}} & \multicolumn{7}{X}{}\\
\multicolumn{3}{X}{{ *hənk-jawu-i}} & \multicolumn{4}{X}{?} & \multicolumn{4}{X}{{ hənk-ju-n}} & \multicolumn{4}{X}{{ hənk-jagacinaa}} & \multicolumn{7}{X}{}\\
\multicolumn{3}{X}{enter-POL-NPST} & \multicolumn{4}{X}{} & \multicolumn{4}{X}{enter-UMRK-PTCP} & \multicolumn{4}{X}{enter-SIM} & \multicolumn{7}{X}{}\\
\multicolumn{22}{X}{}\\
\multicolumn{22}{X}{{\bfseries Type-D affixes and clitic}}\\
\multicolumn{2}{X}{{ \textit{{}-ba} (CSL)}} & \multicolumn{2}{X}{{ \textit{{}-boo} (CND)}} & \multicolumn{4}{X}{{ \textit{{}-gadɨ} ‘until’}} & \multicolumn{4}{X}{{ \textit{{}-na} (PROH)}} & \multicolumn{4}{X}{{ \textit{kai} (DUB)}} & \multicolumn{6}{X}{}\\
\multicolumn{2}{X}{{ hənk-uba}} & \multicolumn{2}{X}{{ *hənk-uboo}} & \multicolumn{4}{X}{{ hənk-ugadɨ}} & \multicolumn{4}{X}{{ hənk-una}} & \multicolumn{4}{X}{{ *hənk=ukai}} & \multicolumn{6}{X}{}\\
\multicolumn{2}{X}{enter-CSL} & \multicolumn{2}{X}{enter-CND} & \multicolumn{4}{X}{enter-until} & \multicolumn{4}{X}{enter-PROH} & \multicolumn{4}{X}{enter=DUB} & \multicolumn{6}{X}{}\\
\lspbottomrule
\end{tabularx}
Notes: \textit{hənk-} ‘enter’ behaves like the verbal stem No. 7 (ending with //V\textsubscript{non-}\textit{\textsubscript{i}} k//) except for the following case.

(i)  /j/ is inserted between \textit{hənk-} ‘enter’ and the type-A affixes, e.g. \textit{hənk-an} (enter-NEG) > /hənk-jan/. In other words, \textit{hənk-} ‘enter’ behaves like the verbal stem No. 15 (ending with //ik//) although it does not include //i// in the stem-final syllable.

Infinitives (simple forms and lengthened forms)

\tablefirsthead{}

\tabletail{}
\tablelasttail{}
\begin{tabularx}{\textwidth}{XXXXXXX}
\lsptoprule

\raggedleft Stem No. & \multicolumn{3}{X}{{ 1. V\textsubscript{non-back}r}} & \multicolumn{3}{X}{{ 2. V\textsubscript{back}r, V\textsubscript{back}w}}\\
\raggedleft ex. & {\itshape hingir-} & {\itshape abɨr-} & {\itshape kəər-} & {\itshape ˀkuur-} & {\itshape nugoor-} & {\itshape koow-}\\
& ‘escape’ & ‘call’ & ‘exchange’ & ‘close’ & ‘don’t do’ & ‘buy’\\
Simple & hingi & abɨ & kəə & ˀkuu-i & nugoo-i & koo-i / ko-i\\
Lengthened & hingii & abɨɨ & kəə & ˀkuu-ii & nugoo-ii & koo-ii\\
\raggedleft Stem No. & { 2. V\textsubscript{back}r} & 3. pp & 4. b & 5. Vm & 6. nm & { 7. V\textsubscript{non-}\textit{\textsubscript{i} }k}\\
\raggedleft ex. & {\itshape tur-} & {\itshape app-} & {\itshape narab-} & {\itshape jum-} & {\itshape tanm-} & {\itshape kak-}\\
& ‘take’ & ‘play’ & ‘line up’ & ‘read’ & ‘ask’ & ‘write’\\
Simple & tu-i & app-i & narab-i & jum / jum-i & tanm-i & kak-i\\
Lengthened & tu-ii & app-ii & narab-ii & jum / jum-ii & tanm-ii & kak-ii\\
\raggedleft Stem No. & 8. V\textsubscript{non-}\textit{\textsubscript{i} }kk & 9. Vs & 10. ss & 11. t & \multicolumn{2}{X}{12. Only C(G)}\\
\raggedleft ex. & {\itshape sukk-} & {\itshape us-} & {\itshape kuss-} & {\itshape ut-} & {\itshape jˀ-} & {\itshape mj-}\\
& ‘pull’ & ‘push’ & ‘kill’ & ‘hit’ & ‘say’ & ‘see’\\
Simple & sukk-i & us-i & kuss-i & uc-i & jˀ-ii & m-ii\\
Lengthened & sukk-ii & us-ii & kuss-ii & uc-ii & jˀ-ii & m-ii\\
\raggedleft Stem No. & 13. ij & 14. V\textsubscript{non-}\textit{\textsubscript{i}} g & 15. ik & 16. i(n)g &  & 17. in\\
\raggedleft ex. & {\itshape kij-} & {\itshape tug-} & {\itshape kik-} & {\itshape uig-} & {\itshape ming-} & {\itshape sin-}\\
& ‘cut’ & ‘whet’ & ‘hear’ & ‘swim’ & ‘grab’ & ‘die’\\
Simple & ki-i & tug-i & kik-i & uig-i & ming-i & sin / sin-i\\
Lengthened & ki-i & tug-ii & kik-ii & uig-ii & ming-ii & N/A\\
Irregular stems & a. & b. & c. & d. & e. & f.\\
ex. & {\itshape sɨr-} & {\itshape k-} & {\itshape ik-} & {\itshape umoor-} & {\itshape hijaw-} & {\itshape sij-}\\
& ‘do’ & ‘come’ & ‘go’ & (move.HON) & ‘pick up’ & ‘know’\\
Simple & s-ii & k-ii & ik-i & umoo-i & hijəə-Ø & si-i\\
Lengthened & s-ii & k-ii & ik-ii & umoo-ii & hijəə-Ø & ?\\
Irregular stems & g. & h. &  &  &  & \\
ex. & {\itshape jurkub-} & {\itshape hənk-} &  &  &  & \\
& ‘happy’ & ‘enter’ &  &  &  & \\
Simple & jurukub-i & hənk-i &  &  &  & \\
Lengthened & jurukub-ii & hənk-ii &  &  &  & \\
\lspbottomrule
\end{tabularx}


% % copy the lines above and adapt as necessary

%%%%%%%%%%%%%%%%%%%%%%%%%%%%%%%%%%%%%%%%%%%%%%%%%%%%
%%%             Backmatter                       %%%
%%%%%%%%%%%%%%%%%%%%%%%%%%%%%%%%%%%%%%%%%%%%%%%%%%%%

\is{some term| see {some other term}}
\il{some language| see {some other language}}
\issa{some term with pages}{some other term also of interest}
\ilsa{some language with pages}{some other lect also of interest}
% There is normally no need to change the backmatter section
\backmatter
\phantomsection%this allows hyperlink in ToC to work
{\sloppy\printbibliography[heading=references]}
\cleardoublepage

\phantomsection 
\addcontentsline{toc}{chapter}{\lsIndexTitle} 
\addcontentsline{toc}{section}{\lsNameIndexTitle}
\ohead{\lsNameIndexTitle} 
\printindex 
\cleardoublepage
  
\phantomsection 
\addcontentsline{toc}{section}{\lsLanguageIndexTitle}
\ohead{\lsLanguageIndexTitle} 
\printindex[lan] 
\cleardoublepage
  
\phantomsection 
\addcontentsline{toc}{section}{\lsSubjectIndexTitle}
\ohead{\lsSubjectIndexTitle} 
\printindex[sbj]
\ohead{} 

\end{document}

% you can create your book by running
% xelatex main.tex
