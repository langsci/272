\author{Yuto Niinaga}
\title{A grammar of Yuwan}
\subtitle{}
\renewcommand{\lsSeries}{sidl}
\renewcommand{\lsSeriesNumber}{}
\renewcommand{\lsID}{272}

\BackBody{This grammar provides a synchronic grammatical description of Yuwan, a regional variety of Amami, a Northern Ryukyuan language in the Japonic language family. Yuwan is spoken by about a hundred people in a small community of Amami-Oshima island in Japan. The study is based on four hours of recordings of monologues and conversations among Yuwan speakers, complemented by targeted elicitation. The grammar is written in a typological framework. After a general introduction to the language, the grammar discusses the following topics: phonology, nominal phrases, verbal morphology, predicate phrases, particles, and subordinate clauses. Of special interest to linguists, typologists, and Ryukyuan specialists are the following in-depth analyses and descriptions: animacy hierarchy in NPs, singular use of plural markers, grammaticalization of a non-finite verb to a case particle, rich morphophonological alternations in verbs and some particles, finite use of subordinate clauses (so-called “insubordination”), and a restriction on the co-occurrence of some focus particles and verbal inflections (so-called “Kakari-musubi” in Japanese linguistics). This study provides a starting point of comparison for further studies on other Ryukyuan varieties.}
