\chapter{Nominal phrases}\label{chap:6}

The nominal phrase (NP) has the following construction. The round brackets mean that the contents inside are optional, and the equal sign “=” indicates a clitic boundary.

\ea{}
[(Modifier) Head]\textsubscript{NP} (=Case)
\z

An NP is made of a modifier slot and a head slot, to which a case particle may be attached to as an NP extender. I will call an NP that contains a case particle an “extended NP” following \citet[167]{Shimoji2008}. An NP can be followed by a sequence of two case particles. So far, the second case of the sequence is genitive or nominative (see \sectref{sec:key:6.1.1.} about genitive, and \sectref{sec:key:6.3.2.1} about nominative), with the exception of infinitives followed by \textit{n=kara} (\textsc{dat}1=\textsc{abl}) (see \sectref{sec:key:6.3.2.3}). An (extended) NP can function as an argument, predicate, or modifier of an NP. If an NP functions as a predicate, it does not take any case, although there are a few exceptions (see \sectref{sec:key:9.3.3}). In the following sections, we will consider Modifier (see \sectref{sec:key:6.1}), Head (see \sectref{sec:key:6.2}), and Case (see \sectref{sec:key:6.3}) respectively. In addition, the constituents that fill the slots in the NP in Yuwan are very sensitive to the animacy hierarchy, which will be addressed in \sectref{sec:key:6.4.}

\section{Modifier}

The modifier slot of an NP is not obligatory, and it can be filled by an NP itself (i.e. genitive case), adnominal word, and adnominal clause. Let us see some examples in the following sections.

\subsection{Modifier filled by an NP}

If a nominal is to modify another nominal in an NP, first it fills the head slot of an NP taking a genitive case particle, and then it fills the modifier slot of the larger NP recursively.

\ea\label{ex:6-2}
   [Context: Talking about the days when US (the hearer) sold fish]\\
{\TM}
\gll \textit{sima=nu}  \textit{jˀu=nu}  \textit{naa.}\\
community=\textsc{gen}  fish=\textsc{gen}  name\\
\glt    ‘(I asked if you know) the name of the fish of (our) community.’ [Co: 110328\_00.txt]
\z

The above NP can be analyzed as follows.

\ea\label{ex:6-3}
 <\{[\textit{sima}\textsubscript{Head}=\textit{nu}\textsubscript{Case}]\textsubscript{NP: Modifier} \textit{jˀu}\textsubscript{Head}=\textit{nu}\textsubscript{Case}\}\textsubscript{NP: Modifier} \textit{naa}\textsubscript{Head}>\textsubscript{NP}
\z


If the NP modifier is address an noun (see \sectref{sec:key:7.2}) such as \textit{anmaa} ‘mother’ or a nominal that contains \textit{-taa} (\textsc{pl}) (see \sectref{sec:key:6.4.1}), it does not take the genitive case, and only juxtaposition shows the possessive meaning as in (\ref{ex:6-4}a-b).

\ea\label{ex:6-4}
\ea\relax[Context: Remembering the day when a few students came to see \textsc{tm}’s mother]\\
{\TM}
\glll anmaa  məəci  kjuuta.\\
\textit{anmaa}  \textit{məə=kaci}  \textit{k-jur-tar}\\
mother  front=\textsc{all}  come-\textsc{umrk}-\textsc{pst}\\
\glt ‘(They) used to come to (my) mother’s place.’ [Co: 110328\_00.txt]

\ex\relax[Context: Talking about US’s grandchild, whom US had went to see]\\
{\US}
\glll uttaa  məəci  mata  {\textbar}oohuku{\textbar}  aicjɨ                                                    izjanwakejo.                                            \\
      \textit{u-rɨ-taa}  \textit{məə=kaci}  \textit{mata}  \textit{oohuku}  \textit{aik-tɨ}                   \textit{ik-tar-n=wake=joo}                              \\
      \textsc{mes}-\textsc{nlz}-\textsc{pl}  front=\textsc{all}  again  back.and.forth  walk-\textsc{seq}     go-\textsc{pst}-\textsc{ptcp}=\textsc{cfp}=\textsc{cfm}1\\
\glt ‘(I) went to their place [i.e. the family of US’s grandchild] and came back again on foot.’ [Co: 110328\_00.txt]


\ex\relax[Context: Asking a person to go to another place]\\
{\TM}
\glll  kˀwanu  məəci  cˀjɨ  kurɨrancjɨ  jˀicjattoojoo.\\
\textit{kˀwa=nu}  \textit{məə=kaci}  \textit{k-tɨ}  \textit{kurɨr-an=ccjɨ}  \textit{jˀ-tar-too=joo}\\
child=\textsc{gen}  front=\textsc{all}  come-\textsc{seq}  \textsc{ben}-\textsc{neg}=\textsc{qt}  say-\textsc{pst}-\textsc{cnd}=\textsc{cfm}1\\
\glt ‘I said (to him), “Would you please come to (my) son’s place?”’ [Co: 120415\_00.txt]
\z
\z

A nominal that is not an address noun nor followed by \textit{-taa} (\textsc{pl}) should take the genitive case to fill the modifier slot of an NP such as \textit{kˀwa=nu} (child=\textsc{gen}) in (\ref{ex:6-4}c). The constructions in (\ref{ex:6-4}a-b) are merely juxtaposition, and not compounding (see \sectref{sec:key:4.2.4} for more details).

There are a few cases where a genitive case particle \textit{nu} can follow another case particle. The sequences of case particles are underlined below.

\ea\label{ex:6-5}
\ea\relax[Context: Hearing that US’s son went somewhere]\\
{\TM}
\glll  amakacinu  {\textbar}sjokurjoo{\textbar}  muccjɨ  ikidaroo.\\
\textit{a-ma=kaci=nu}  \textit{sjokurjoo}  \textit{mut-tɨ}  \textit{ik-i=daroo}\\
\textsc{dist}-place=\textsc{all}=\textsc{gen}  food  have-\textsc{seq}  go-\textsc{inf}=\textsc{supp}\\
\glt ‘(He) would probably bring the food for that place.’ [Co: 110328\_00.txt]

\ex\relax[Context: Speaking about a ditch there used to be]\\
{\TM}
\glll  huukubumizjuukaranu  mɨzɨ  natɨ,\\
\textit{huukubu+mizjuu=kara=nu}  \textit{mɨzɨ}  \textit{nar-tɨ}\\
Hukubu+ditch=\textsc{abl}=\textsc{gen}  water  \textsc{cop}-\textsc{seq}\\
\glt ‘(It) is a water from the ditch at Hukubu, so ...’ [Co: 120415\_00.txt]

\ex\relax[Context: Seeing a photo taken in celebration of setting up the first outdoor lamps in the shopping street of the village]\\
{\TM}
\glll  un  tukinnu  juwəəja  aran?\\
\textit{u-n}  \textit{tuki=n=nu}  \textit{juwəə=ja}  \textit{ar-an}\\
\textsc{mes}-\textsc{ptcp}  time=\textsc{dat}1=\textsc{gen}  celebration=\textsc{top}  \textsc{cop}-\textsc{neg}\\
\glt ‘Is (the photo about) the celebration at that time?’ [Co: 120415\_00.txt]

\ex
{\TM}
\glll kumannu  tukinnja  {\textbar}kootookaninen{\textbar}gadɨ jappa.\\
\textit{ku-ma=nan=nu}  \textit{tuki=n=ja}  \textit{kootooka+ni+nen=gadɨ}   \textit{jar-ba}\\
\textsc{prox}-place=\textsc{loc1}=\textsc{gen}  time=\textsc{dat}1=\textsc{top}  junior.high+two+year=\textsc{lmt}   \textsc{cop}-\textsc{csl}\\
\glt ‘At the time when (we were) there [lit. at the time of at here], compulsory education was until the second grade of junior high school.’ [Co: 120415\_00.txt]

\ex
{\TM}
\glll {\textbar}sugiuradenki{\textbar}tu  {\textbar}sjuukaisjo{\textbar}tunu  əəda...  ganbəi  acjutattu.\\
\textit{sugiura+denki=tu}  \textit{sjuukaisjo=tu=nu}  \textit{əəda}  \textit{ga-n=bəi}  \textit{ak-tur-tar-tu}\\
Sugiura+electricity=\textsc{com}  meeting.place=\textsc{com}=\textsc{gen}  space  \textsc{mes}-\textsc{advz}=only   open-\textsc{prog}-\textsc{pst}-\textsc{csl}\\
\glt ‘There was a space like that between the Sugiura electric appliance shop and the meeting place.’ [Co: 111113\_02.txt]
\z
\z

\textit{nu} (\textsc{gen}) follows \textit{kaci} (\textsc{all}) as in (\ref{ex:6-5}a), \textit{kara} (\textsc{abl}) as in (\ref{ex:6-5}b), \textit{n} (\textsc{dat}1) as in (\ref{ex:6-5}c)\footnote{When \textit{nu} (\textsc{gen}) follows \textit{n} (DAT1), the head of an NP is always \textit{tuki} ‘time’ in my texts.}, \textit{nan} (\textsc{loc1}) as in (\ref{ex:6-5}d) (about the alternation from //nan// to /n/, see \sectref{sec:key:6.3.1.4}), and \textit{tu} (\textsc{com}) as in (\ref{ex:6-5}e).

\subsection{Modifier filled by adnominal word or adnominal clause}

The adnominal word fills only the modifier slot of an NP taking no genitive particle, and it obligatorily takes a specific inflectional affix, e.g. \textit{-a} (\textsc{adnz}) and \textit{-n} (\textsc{adnz}) (see \chapref{sec:5}).

\ea\label{ex:6-6}
\ea\relax[Context: Taking about the present author]\\
{\US}
\glll  waa  məəci  saki  umoocjanwake.\\
\textit{waa-a}  \textit{məə=kaci}  \textit{saki}  \textit{umoor-tar-n=wake}\\
1\textsc{sg}-\textsc{adnz}  front=\textsc{all}  first  move/stay.\textsc{hon}-\textsc{pst}-\textsc{ptcp}=\textsc{cfp}\\
\glt ‘(He) came to my place first.’ [Co: 110328\_00.txt]

\ex\relax[Context: Speaking with \textsc{my}]\\
{\TM}
\glll  ude,  kun  nɨkan  kadɨn  njɨ!\\
\textit{ude}  \textit{ku-n}  \textit{nɨkan}  \textit{kam-tɨ=n}  \textit{nj-ɨ}\\
well  \textsc{prox}-\textsc{adnz}  mikan  eat-\textsc{seq}=ever  \textsc{exp}-\textsc{imp}\\
\glt ‘Well, try to eat this \textit{mikan}!’ [Co: 101023\_01.txt]
\z
\z

/waa/ \textit{waa-a} (1\textsc{sg}-\textsc{adnz}) ‘my’ in (\ref{ex:6-6}a) fills the modifier slot of an NP, whose head is \textit{məə} ‘front.’ \textit{ku-n} (\textsc{prox}-\textsc{adnz}) ‘this’ in (\ref{ex:6-6}b) fills the modifier slot of an NP, whose head is \textit{nɨkan} ‘\textit{mikan}.’

Furthermore, a modifier slot of an NP can be filled by an adnominal clause, whose final constituent is a participle (see \sectref{sec:key:8.4.2}).

\ea\label{ex:6-7}
[Context: Speaking of the time when US was selling fish]\\
{\TM}
\glll simanantɨ  tujun  jˀudu  ujutarooga?\\
    {}[\textit{sima=nantɨ}  \textit{tur-jur-n}]\textsubscript{Adnominal clause}  \textit{jˀu=du}  \textit{ur-jur-tar-oo=ga}\\
    community=\textsc{loc2}  take-\textsc{umrk}-\textsc{ptcp}  fish=\textsc{foc}  sell-\textsc{umrk}-\textsc{pst}-\textsc{supp}=\textsc{foc}\\
\glt    ‘(You) used to sell fish which (people) caught in the community [i.e. not buying from outside the community]?’ [Co: 110328\_00.txt]
\z

In the above example, \textit{sima=nantɨ} \textit{tur-jur-n} (community=\textsc{loc2} take-\textsc{umrk}-\textsc{ptcp}) ‘catching in the community’ is an adnominal clause, which modifies its head \textit{jˀu} ‘fish’.

\section{Head}
\subsection{The structural property of head}

The head slot of an NP is obligatory, and can be filled by a nominal.

\ea\label{ex:6-8}
 Head is filled by a nominal\\{}
[Context: Talking of kinds of snails]

{\TM}
\glll arɨga  tanmjaa  jappajaa.\\
\textit{a-rɨ=ga}  \textit{tanmjaa}  \textit{jar-ba=jaa}\\
    \textsc{dist}-\textsc{nlz}=\textsc{nom}  mud.snail  \textsc{cop}-\textsc{csl}=\textsc{sol}\\
\glt    ‘That is a mud snail, you know.’ [Co: 111113\_02.txt]
\z

In \REF{ex:6-8}, \textit{tanmjaa} ‘mud snail’ fills the head slot of an NP, which is followed by a copula verb.

The head slot of an NP can be fiiled by the infinitive (see \sectref{sec:key:8.5.4}).

\ea\label{ex:6-9}
 Head is filled by an infinitive\\{}
[Context: Speaking with \textsc{my} about the present author]

{\TM}
\glll {\textbar}benkjoo{\textbar}  sjun  cˀjunkjaccjɨboo,  gan  sjɨ    sjutɨ,  {\textbar}benkjoo{\textbar}  sii  jappajaa.\\
\textit{benkjoo}  \textit{sɨr-jur-n}  \textit{cˀju=nkja=ccjɨboo}  \textit{ga-n}  \textit{sɨr-tɨ}    \textit{sɨr-jur-tɨ}  \textit{benkjoo}  \textit{sɨr-i}  \textit{jar-ba=jaa}\\
study  do-\textsc{umrk}-\textsc{ptcp}  person=\textsc{appr}=speaking.of  \textsc{mes}-\textsc{advz}  do-\textsc{seq}    do-\textsc{umrk}-\textsc{seq}  study  do-\textsc{inf}  \textsc{cop}-\textsc{csl}=\textsc{sol}\\
\glt    ‘Speaking of a person who does studies, (he) does studying like that, you know.’ [Co: 101023\_01.txt]
\z

In \REF{ex:6-9}, the infinitive /sii/ \textit{sɨr-i} (do-\textsc{inf}) ‘doing’ fills the head slot of an NP, which is followed by a copula verb.

It should be noted that an NP can have recursive structure. A head nominal followed by a genitive particle can fill the modifier slot recursively as in \REF{ex:6-2}, whose construction is as follows: “[Modifier Head]\textsubscript{Modifier} Head.” In addition, a head modified by an adnominal clause can fill the head slot recursively, which is further modified by an adnominal as in (\ref{ex:4-46}b) in \sectref{sec:key:4.2.4}, whose construction is as follows: “Modifier [Modifier Head]\textsubscript{Head}.”

\subsection{Bound head (formal nouns)}

A head of an NP is usually a free form as in the previous section. There are, however, some morphemes that are bound, i.e. cannot start an utterance by themselves, but can fill the head slot of an NP. Such morphemes are called “formal nouns” in this grammar associated with the same term used in the traditional Japanese linguistics. So far, I have found thirteen formal nouns in my texts: \textit{sɨ} ‘thing; person; fact’, \textit{kutu} ‘event’, \textit{hudu} ‘quantity’, \textit{bun} ‘share’, \textit{taməə} ‘sake’, \textit{hazɨ} ‘certainty’, \textit{nintəə} ‘people’, \textit{nagatɨɨ} ‘along’, \textit{hutəə/butəə/datəə} ‘vicinity’, \textit{turoo} ‘place’, \textit{mama} ‘still’, \textit{tui} ‘as,’ and \textit{hui} ‘pretend.’ They can be modified by at least one of adnominals, address nouns, or adnominal clauses.

\subsubsection{\textit{sɨ} ‘thing; person; fact’}

The formal noun \textit{sɨ} behaves differently from other formal nouns. For example, the semantic content is so “light” that it can indicate almost all of the substances, i.e. humans, non-humans, or events. Furthermore, \textit{sɨ} (\textsc{fn}) behaves like an affix when it follows the verbal stems, i.e., the verbal stem that precedes \textit{sɨ} (FN) does not take the participial affix \textit{-n} (\textsc{ptcp}). This phenomenon does not occur in the case of other formal nouns. I will present the details of \textit{sɨ} (FN) in turn below.

Semantically, the formal noun \textit{sɨ} can indicate either human or non-human referents. \textit{sɨ} in (\ref{ex:6-10}a) indicates a person, but \textit{sɨ} in (\ref{ex:6-10}b-c) indicates non-human referents.

\ea\label{ex:6-10}
 Human referent\\

 \ea\relax[Context: Talking about how to cook in the old days ]\\
{\TM}
\glll nanzijucjɨnkjoo  sjusəə  waakjabəi  arantakai?\\
\textit{nanziju=ccjɨ=nkja=ja}  \textit{sɨr-jur=sɨ=ja}  \textit{waakja=bəi}   \textit{ar-an-tar=kai}\\
fireplace =\textsc{qt}=\textsc{appr}=\textsc{top}  do-\textsc{umrk}=\textsc{fn}=\textsc{top}  1\textsc{pl}=only     \textsc{cop}-\textsc{neg}-\textsc{pst}=\textsc{dub}\\
\glt ‘Perhaps, (it was) only us, who did (the cooking) at fireplaces, wasn’t (it)?’ [Co: 111113\_02.txt]

\ex
  Non-human referent\\
{\TM}
\glll uraga  jˀusɨnan  (hɨntooja  sjun ..)    hɨntooja  sjussa.\\
\textit{ura=ga}  \textit{jˀ-jur=sɨ=nan}  \textit{hɨntoo=ja}  \textit{sɨr-jur-n} \textit{hɨntoo=ja}  \textit{sɨr-jur-sa}\\
2.\textsc{nhon}.\textsc{sg}=\textsc{nom}  say-\textsc{umrk}=\textsc{fn}=\textsc{loc1}  reply=\textsc{top}  do-\textsc{umrk}-\textsc{ptcp}   reply=\textsc{top}  do-\textsc{umrk}-\textsc{pol}\\
\glt ‘(I) will reply to what you say.’ [Co: 120415\_01.txt]

\ex\relax[Context: Talking about the bulletins of Yuwan made by the speaker’s son]\\
{\TM}
\glll  kurəə  {\textbar}mae{\textbar}nusɨ  zjajaa.\\
\textit{ku-rɨ=ja}  \textit{mae=nu=sɨ}  \textit{zjar=jaa}\\
\textsc{prox}-\textsc{nlz}=\textsc{top}  before=\textsc{gen}=\textsc{fn}  \textsc{cop}=\textsc{sol}\\
\glt ‘This is the thing (made) before.’ [Co: 120415\_01.txt]
\z
\z

Additionally, \textit{sɨ} can indicate an event. In other words, it can function as a so-called “complementizer” (see also \sectref{sec:key:11.1.3}).

\ea\label{ex:6-11}
\ea\relax[Context: Looking at a picture, where people older than \textsc{tm} got together.]\\
{\TM}
\glll  wakaran....  kan  sjɨ  juratasəə sijan.\\
\textit{wakar-an}  \textit{ka-n}  \textit{sɨr-tɨ}  \textit{juraw-tar=sɨ=ja} \textit{sij-an}\\
understand-\textsc{neg}  \textsc{prox}-\textsc{advz}  do-\textsc{seq}  get.together-\textsc{pst}=\textsc{fn}=\textsc{top}     know-\textsc{neg}\\
\glt ‘(I) don’t know.... (I) don’t know that (they) got together like this.’ [Co: 120415\_00.txt]

\ex\relax[Context: \textsc{tm} asked when US had come to her house.]\\
{\TM}
\glll  nanga  kunəəda  umoocjasəə  kun  cˀjunu  cˀjəərai?\\
\textit{nan=ga}  \textit{kunəəda}  \textit{umoor-tar=sɨ=ja}  \textit{ku-n}   \textit{cˀju=nu}  \textit{k-təəra=i}\\
2.\textsc{hon}.\textsc{sg}=\textsc{nom}  the.other.day  come.\textsc{hon}-\textsc{pst}=\textsc{fn}=\textsc{top}  \textsc{prox}-\textsc{adnz}   person=\textsc{nom}  come-after=\textsc{plq}\\
\glt ‘(Is it) after this person [i.e. the present author] came (to your house) that you [i.e. US] came (here) the other day?’ [Co: 110328\_00.txt]
\z
\z

In (\ref{ex:6-11}a-b), \textit{sɨ} indicates neither a human nor a non-human referent, but indicates an event as a whole.

  Within a clause, an NP headed by \textit{sɨ} can fill the argument slot as in (\ref{ex:6-10}b) or the nominal predicate slot as in (\ref{ex:6-10}c). Within an NP, \textit{sɨ} cannot fill the head slot only by itself: */sɨnu ai/ \textit{sɨ=nu} \textit{ar-i} (\textsc{fn}=\textsc{nom} exist-\textsc{npst}) [Intended meaning] ‘There is something.’ In order to fill the head slot of an NP, \textit{sɨ} has to be modified by adnominals, genitive NPs, or address nouns as in (\ref{ex:6-12}a-c). The modifiers and \textit{sɨ} (FN) are underlined below.

\ea\label{ex:6-12}
\ea Modified by an adnominal word\\{}
[Context: Talking about laundry detergent]

{\TM}
\glll uraasəə  ooja  iziran.jaa.\\
      \textit{ura-a=sɨ=ja}  \textit{oo=ja}  \textit{izir-an=jaa}\\
      2.\textsc{nhon}.\textsc{sg}-\textsc{adnz}=\textsc{fn}=\textsc{top}  bubble=\textsc{top}  go.out-\textsc{neg}=\textsc{sol}\\
\glt ‘Yours [i.e. your laundry detergent] does not make bubbles, does it?’ [El: 120928]

\ex Modified by a genitive NP\\{}
[Context: Talking about a photograph collection]
{\TM}
\glll {\textbar}taken{\textbar}nusɨga  mutu  zja.\\
      \textit{taken=nu=sɨ=ga}  \textit{mutu}  \textit{zjar}\\
      Taken=\textsc{gen}=\textsc{fn}=\textsc{nom}  original  \textsc{cop}\\
\glt ‘The things from Taken [i.e. pictures gathered in Taken] are originals (of the collection).’ [Co: 111113\_02.txt]

\ex Modified by an address noun\\
{\TM}
\glll  anmaasəə  dɨru?\\
\textit{anmaa=sɨ=ja}  \textit{dɨ-ru}\\
mother=\textsc{fn}=\textsc{top}  which-\textsc{nlz}\\
\glt ‘Which one (is) mother’s?’ [El: 140227]
\z
\z

There is a characteritic unique to the formal noun \textit{sɨ}, which differentiates \textit{sɨ} from other formal nouns. \textit{sɨ} cannot be modified by an adnominal clause (with the exception of the case where \textit{-an} (\textsc{neg}) precedes \textit{sɨ}). Rather, it behaves like a verbal affix directly following a bound verbal stem (cf. affix-like clitics in \sectref{sec:key:4.2.2.2}). Relevant examples were already shown in (6-10 a-b, 6-11 a-b). Thus, I will compare \textit{sɨ} and another formal noun, e.g. \textit{turoo} ‘place,’ in (\ref{ex:6-13}a-b).

\ea\label{ex:6-13}
\ea Head is \textit{sɨ} (\textsc{fn})\\{}
[Context: Talking about the present author]

{\TM}
\glll an  nɨsəə  muccjɨ  ikjusəə  nun  nənba,  jakkəə.\\
\textit{a-n}  \textit{nəɨsəə}  \textit{mut-tɨ}  \textit{ik-jur=sɨ=ja}  \textit{nuu=n}   \textit{nə-an-ba}  \textit{jakkəə}\\
    \textsc{dist}-\textsc{adnz}  young.man  have-\textsc{seq}  go-\textsc{umrk}=\textsc{fn}=\textsc{top}  what=any    exist-\textsc{neg}-\textsc{csl}  trouble\\
\glt    ‘There is not anything [i.e. any food] the young man can take (for meals), so it’s a pity.’ [Co: 101023\_01.txt]

\ex Head is \textit{turoo} ‘place’\\{}
  [Context: Looking at a picture, where people gathered in front of a truck]

{\TM}
\glll ikjun  turookai?\\
\textit{ik-jur-n}  \textit{turoo=kai}\\
    go-\textsc{umrk}-\textsc{ptcp}  place=\textsc{dub}\\
\glt    ‘Is (this) a scene where they go (somewhere)?’ [Co: 120415\_00.txt]
\z
\z

An adnominal clause should take a participle as its predicate in Yuwan (see \sectref{sec:key:11.1.2}). Thus, \textit{turoo} ‘place’ in (\ref{ex:6-13}b) is modified by an adnominal clause whose predicate is a participle /ikjun/ \textit{ik-jur-n} (go-\textsc{umrk}-\textsc{ptcp}). However, in (\ref{ex:6-13}a), \textit{sɨ} is not modified by an adnominal clause, but it follows directly a bound verbal stem /ikju/ \textit{ik-jur} (go-\textsc{umrk}), which does not take the participial affix \textit{-n}. Therefore, in (\ref{ex:6-13}a), we may say that the formal noun \textit{sɨ} has lost its ability to fill the head slot of an NP. Rather, it behaves as an affix, and the verbal form /ikjusɨ/ \textit{ik-jur=sɨ} (go-\textsc{umrk}=\textsc{fn}) as a whole has developed the ability to fill the head slot of an NP (see also \sectref{sec:key:11.1.3}). If \textit{sɨ} is directly preceded by the negative participial affix \textit{-an} (\textsc{neg}), the preceding clause has the same form with the adnominal clause whose head is a common noun as in (\ref{ex:6-14}a-b).

\ea\label{ex:6-14}
 Directly preceded by \textit{-an} (\textsc{neg})\\
 \ea Head is \textit{sɨ} (\textsc{fn})\\
{\TM}
\glll  kamansəə  jiccjoo  nən.\\
\textit{kam-an=sɨ=ja}  \textit{jiccj-soo}  \textit{nə-an}\\
eat-\textsc{neg}=\textsc{fn}=\textsc{top}  good-\textsc{adj}  \textsc{stv}-\textsc{neg}\\
\glt ‘The fact (you) do not eat (anything) is not good (for your health).’ [El: 100222]

\ex Head is \textit{cˀju} ‘person’\\
{\TM}
\glll  hanməəga  kaman  cˀju  natɨ  cˀjɨjoo.\\
\textit{hanməə=ga}  \textit{kam-an}  \textit{cˀju}  \textit{nar-tɨ}  \textit{k-tɨ=joo}\\
meal=\textsc{nom}  eat-\textsc{neg}  person  become-\textsc{seq}  come-\textsc{seq}=\textsc{cfm}1\\
\glt ‘(I)’ve become a person who cannot eat meal (very much).’ [Co: 120415\_01.txt]
\z
\z

In (\ref{ex:6-14}b), the predicate of the adnominal clause, i.e. \textit{kam-an} (eat-\textsc{neg}), precedes the common noun \textit{cˀju} ‘person.’ Similarly, in (\ref{ex:6-14}a), \textit{kam-an} (eat-\textsc{neg}) does not undergo any reduction before \textit{sɨ} (\textsc{fn}). In this case, we may say that the predicate \textit{kam-an} (eat-\textsc{neg}) in (\ref{ex:6-14}a) fills the predicate slot of the adnominal clause whose head is \textit{sɨ} (FN).

\subsubsection{\textit{kutu} ‘event’}

I will present examples of \textit{kutu} ‘event.’ In (\ref{ex:6-15}a), \textit{kutu} ‘event’ is modified by a genitive NP \textit{mukasi=nu} (past=\textsc{gen}), and in (\ref{ex:6-15}b) it is modified by an adnominal clause whose head is the participle /kadan/ \textit{kam-tar-n} (eat-\textsc{pst}-\textsc{ptcp}).

\ea\label{ex:6-15}
\ea With a genitive NP [= (\ref{ex:4-20}a)]\\
{\TM}
\glll  tarun  mukasinukutu  siccjun  cˀjoo  wuranbajaa.\\
\textit{ta-ru=n}  \textit{mukasi=nu=kutu}  \textit{sij-tur-n} \textit{cˀju=ja}  \textit{wur-an-ba=jaa}\\
who-\textsc{nlz}=any  past=\textsc{gen}=event  know-\textsc{prog}-\textsc{ptcp}   person=\textsc{top}  exist-\textsc{neg}-\textsc{csl}=\textsc{sol}\\
\glt ‘There is not anyone who knows the events of the past.’ [Co: 110328\_00.txt]

\ex With an adnominal clause\\
{\TM}
\glll  dookunɨɨcɨkɨmunna  urɨhudu  cɨkɨjunban,  kadankutoo  tˀɨn   nən.\\
\textit{dookunɨɨ+cɨkɨmun=ja}  \textit{u-rɨ+hudu}   \textit{cɨkɨr-jur-n=ban}  \textit{kam-tar-n=kutu=ja}  \textit{tˀɨɨ=n}  \textit{nə-an}\\
white.radish+pickles =\textsc{top}  \textsc{mes}-\textsc{nlz}+quantity  pickle-\textsc{umrk}-\textsc{ptcp}=\textsc{advrs}  eat-\textsc{pst}-\textsc{ptcp}=event=\textsc{top}  one.\textsc{clf}=even   exist-\textsc{neg}\\
\glt ‘I pickle so many white radishes, but there is no time when I ate (them).’ [Co: 101023\_01.txt]
\z
\z

\subsubsection{\textit{hudu} ‘quantity’}

I will present examples of \textit{hudu} ‘quantity.’ \textit{hudu} ‘quantity’ in \REF{ex:6-16} is modified by an adnominal clause whose head is the participle /tujun/ \textit{tur-jur-n} (take-\textsc{umrk}-\textsc{ptcp}).

\ea\label{ex:6-16}
 With an adnominal clause\\{}
[Context: Remembering a flood in the past]

{\TM}
\glll naa,  {\textbar}ikkaime{\textbar}nu  mununkjoo  sjasin    tujunhudugadəə  arannən,\\
\textit{naa}  \textit{ikkai+me=nu}  \textit{mun=nkja=ja}  \textit{sjasin} \textit{tur-jur-n=hudu=gadɨ=ja}  \textit{ar-annən}\\
    \textsc{fil}  one.\textsc{clf}+time=\textsc{gen}  thing=\textsc{appr}=\textsc{top}  picture   take-\textsc{umrk}-\textsc{ptcp}=quantity=\textsc{lmt}=\textsc{top}  \textsc{cop}-\textsc{neg}.\textsc{seq}\\
\glt    ‘Well. The first one [i.e. flood] wasn’t quite wothy of a photograph...’ [Co: 120415\_00.txt]
\z

An example of compounding of \textit{hudu} ‘quantity’ was also shown in (\ref{ex:6-15}b).

\subsubsection{\textit{bun} ‘share’}

I will present examples of \textit{bun} ‘share.’ In (\ref{ex:6-17}a), \textit{bun} ‘share’ is modified by an adnominal \textit{u-n} (\textsc{mes}-\textsc{adnz}), and in (\ref{ex:6-17}b) it is modified by an adnominal clause whose head is the participle /kikjun/ \textit{kik-jur-n} (hear-\textsc{umrk}-\textsc{ptcp}).

\ea\label{ex:6-17}
\ea With an adnominal\\{}
[Context: Explaining that there are not so many plates in \textsc{tm}’s house]
{\TM}
\glll unbundu  saran  anmun.\\
      \textit{u-n=bun=du}  \textit{sara=n}  \textit{ar-n=mun}\\
      \textsc{mes}-\textsc{ptcp}=share=\textsc{foc}  plate=also  exist-\textsc{ptcp}=\textsc{advrs}\\
\glt ‘There are so many plates as (there are).’ [Co: 110328\_00.txt]

\ex With an adnominal clause\\{}
[Context: Talking about traditional songs; {\TM} ‘If (I) hear a music tape, ...’

{\TM}
\glll samisjen  kikjunbunsjɨ  nuuutaccjəə     sɨgu  wakajuttoo.\\
      \textit{samisjen}  \textit{kik-jur-n=bun=sjɨ}  \textit{nuu+uta=ccjɨ=ja} \textit{sɨgu}  \textit{wakar-jur=doo}\\
      samisen  hear-\textsc{umrk}-\textsc{ptcp}=share=\textsc{inst}  what+song=\textsc{qt}=\textsc{top}      soon  understand-\textsc{umrk}=\textsc{ass}\\
\glt ‘Soon (I) can understand what song (it is) only by hearing (the sound of) samisen.’ [Co: 111113\_01.txt]
\z
\z

\subsubsection{\textit{taməə} ‘sake’}

I will present examples of \textit{taməə} ‘sake.’ In (\ref{ex:6-18}a), \textit{taməə} ‘sake’ is modified by an adnominal \textit{urakja-a} (2.\textsc{nhon}.\textsc{pl}-\textsc{adnz}), and in (\ref{ex:6-18}b) it is modified by an adnominal clause whose head is the participle /noosjun/ \textit{noos-jur-n} (leave-\textsc{umrk}-\textsc{ptcp}).

\ea\label{ex:6-18}
\ea With an adnominal\\
{\TM}
\glll  uraa  baasanna  jazin  magankjanu  urakjaataməəja  {\textbar}nacuwa{\textbar}  jazin      kinukkwa  jatattujaa.\\
\textit{ura-a}  \textit{baasan=ja}  \textit{jazin}  \textit{maga=nkja=nu}     \textit{urakja-a=taməə=ja}  \textit{nacu=wa}  \textit{jazin}     \textit{kin-kkwa}  \textit{jar-tar-tu=jaa} \\
2.\textsc{nhon}.\textsc{sg}-\textsc{adnz}  grandmother=\textsc{top}  necessarily  grandchild=\textsc{appr}=\textsc{gen}   2.\textsc{nhon}.\textsc{pl}-\textsc{adnz}=sake=\textsc{top}  summer=\textsc{top}  necessarily clothes-\textsc{dim}  \textsc{cop}-\textsc{pst}-\textsc{csl}=\textsc{sol}\\
\glt ‘Your grandmother necessarily prepared clothes for (her) grandchild, (i.e.) you, in summer.’ [Co: 120415\_01.txt]

\ex With an adnominal clause\\{}
[Context: Thanking \textsc{ms} for his kind cooperation to preserve the old tradition of Yuwan]

{\TM}
\glll noosjuntaməə  urakjaga  {\textbar}kjoorjoku{\textbar}     sjɨ  kurɨjun  mun  natɨ,\\
     \textit{noos-jur-n=taməə}  \textit{urakja=ga}  \textit{kjoorjoku}     \textit{sɨr-tɨ}  \textit{kurɨr-jur-n}  \textit{mun}  \textit{nar-tɨ}\\
      leave-\textsc{umrk}-\textsc{ptcp}=sake  2.\textsc{nhon}.\textsc{pl}=\textsc{nom}  cooperation   do-\textsc{seq}  \textsc{ben}-\textsc{umrk}-\textsc{ptcp}  thing  \textsc{cop}-\textsc{seq}\\
\glt ‘To preserve (the old traditions) a person like you is so kind as to cooperate (with us), so ...’ [Co: 111113\_02.txt]
\z
\z

\subsubsection{\textit{hazɨ} ‘certainty’}

I will present examples of \textit{hazɨ} ‘certainty.’ In (\ref{ex:6-19}a), \textit{hazɨ} ‘certainty’ is modified by a genitive NP \textit{u-ma=nu} (\textsc{mes}-place=\textsc{gen}), and in (\ref{ex:6-19}b) it is modified by an adnominal clause whose head is the participle /wun/ \textit{wur-n} (exist-\textsc{ptcp}).

\ea\label{ex:6-19}
\ea With a genitive NP\\{}
[Context: Looking at a picture]
{\TM}
\glll umanuhazɨ  zjaga.\\
      \textit{u-ma=nu=hazɨ}  \textit{zjar=ga}\\
      \textsc{mes}-place=\textsc{gen}=certatinty  \textsc{cop}=\textsc{cfm}3\\
\glt ‘(The place you are speaking of) must be there.’ [Co: 111113\_01.txt]

\ex With an adnominal clause\\{}
[Context: Looking at a picture]
{\TM}
\glll josihironiitaa  wunhazɨ  zjassɨgajaa.\\
      \textit{josihiro+nii-taa}  \textit{wur-n=hazɨ}  \textit{zjar-sɨga=jaa}\\
      Yoshihiro+older.brother-\textsc{pl}  exist-\textsc{ptcp}=certainty  \textsc{cop}-\textsc{pol}=\textsc{sol}\\
\glt ‘Yoshihiro must be (there).’ [Co: 120415\_00.txt]
\z
\z

In both of the examples of (\ref{ex:6-19}a-b), the NPs headed by \textit{hazɨ} ‘certainty’ fill the predicate slots with the copular verb \textit{zjar-}. In addition, the NP headed by \textit{hazɨ} ‘certainty’ can fill the modifier slot of an NP as in \REF{ex:6-20}.

\ea\label{ex:6-20}
 [Context: Talking about \textsc{tm}’s son]\\

{\TM}
\glll jˀaranhazɨnu  mungadɨ  jattɨ.\\
\textit{jˀ-ar-an=hazɨ=nu}  \textit{mun=gadɨ}  \textit{jˀ-ar-tɨ}\\
    say-P\textsc{ass}-\textsc{neg}=certainty=\textsc{gen}  thing=\textsc{lmt}  say-\textsc{pass}-\textsc{seq}\\
\glt    ‘A thing that need not be said is said (about him).’ [Co: 120415\_01.txt]
\z

In the above example, \textit{hazɨ} ‘certainty’ is modified by an adnominal clause \textit{jˀ-ar-an} (say-P\textsc{ass}-\textsc{neg}) ‘(need) not be said,’ and the NP headed by \textit{hazɨ} ‘certainty’ recursively filled the modifier slot of an NP with genitive case, whose head is \textit{mun} ‘thing.’

\subsubsection{\textit{nintəə} ‘people’}

I will present examples of \textit{nintəə} ‘people.’ In (\ref{ex:6-21}a), \textit{nintəə} ‘people’ is modified by an adnominal \textit{u-n} (\textsc{mes}-\textsc{adnz}), and in (\ref{ex:6-21}b) it is modified by an adnominal clause whose head is the participle /nacɨkasjan/ \textit{nacɨkasj-sa+ar-n} (familiar-\textsc{adj}+\textsc{stv}-\textsc{ptcp}), and in (\ref{ex:6-21}c) it undergoes compounding with \textit{juwan} ‘Yuwan.’

\ea\label{ex:6-21}
\ea With an adnominal\\{}
[Context: \textsc{tm} said that she knew some old people went to see prefectural highway.]

{\TM}
\glll un  nintəənu  hanacjattu.\\
      \textit{u-n}  \textit{nintəə=nu}  \textit{hanas-tar-tu}\\
      \textsc{mes}-\textsc{adnz}  people=\textsc{nom}  talk-\textsc{pst}-\textsc{csl}\\
\glt ‘They said (that they went there, so I know that).’ [Co: 120415\_00.txt]

\ex With an adnominal clause\\{}
[Context: Looking at a picture]

{\TM}
\glll minna  nacɨkasjannintəəbəi.\\
      \textit{minna}  \textit{nacɨkasj-sa+ar-n=nintəə=bəi}\\
      everybody  familiar-\textsc{adj}+\textsc{stv}-\textsc{ptcp}=people=only\\
\glt ‘(They are) all familiar people.’ [Co: 120415\_01.txt]

\ex Compounding\\{}
[Context: Looking at a picture where the women of Yuwan are dancing the traditional dance]

{\TM}
\glll kurəə,  juwannintəənu,  dantɨkai?\\
      \textit{ku-rɨ=ja}  \textit{juwan+nintəə=nu}  \textit{daa=nantɨ=kai}\\
      \textsc{prox}-\textsc{nlz}=\textsc{top}  Yuwan+people=\textsc{nom}  where=\textsc{loc2}=\textsc{dub}\\
\glt ‘(Where do) the people of Yuwan (dance?) Where is this?’ [Co: 111113\_01.txt]
\z
\z

\subsubsection{\textit{nagatɨɨ} ‘along’}

I will present examples of \textit{nagatɨɨ} ‘along.’ In (\ref{ex:6-21}a), \textit{nagatɨɨ} ‘along’ is modified by an adnominal \textit{u-n} (\textsc{mes}-\textsc{adnz}), and in (\ref{ex:6-22}b) it goes through compounding with \textit{koo} ‘river’. So far, there is no example where \textit{nagatɨɨ} ‘along’ is modified by an adnominal clause.

\ea\label{ex:6-22}
\ea With an adnominal\\{}
[Context: Talking about \textsc{tm}’s house in the past]

{\TM}
\glll jaaja  unnagatɨɨ  haija  buubuu  tubjakudɨ,\\
      \textit{jaa=ja}  \textit{u-n=nagatɨɨ}  \textit{hai=ja}  \textit{buu+buu}  \textit{tubjakum-tɨ}\\
      house=\textsc{top}  \textsc{mes}-\textsc{adnz}=along  ash=\textsc{top}  \textsc{red}+floating  scatter-\textsc{seq}\\
\glt ‘(In my) house, around there, ashes scattered.’ [Co: 111113\_02.txt]

\ex Compounding\\{}
[Context: Remembering how to gather wood for business in the past]

{\TM}
\glll jamanu  kɨɨ  urɨsjɨ  koonagatɨɨ  {\textbar}hora{\textbar}   sɨccjɨ  kjuuroogai?\\
      \textit{jama=nu}  \textit{kɨɨ}  \textit{u-rɨ=sjɨ}  \textit{koo+nagatɨɨ}  \textit{hora}   \textit{sɨkk-tɨ}  \textit{k-jur-oo=ga=i}\\
      mountain=\textsc{gen}  tree  \textsc{mes}-\textsc{nlz}=\textsc{inst}  river+along  hey  draw-\textsc{seq}  come-\textsc{umrk}-\textsc{supp}=\textsc{cfm}3=\textsc{plq}\\
\glt ‘(Do you remember that people) harvest the trees on the mountain along the river by that (river boat)?’ [Co: 111113\_01.txt]
\z
\z

In addition, \textit{nagatɨɨ} ‘along’ can be the head of a compound, and it means ‘while.’

\ea\label{ex:6-23}
{\TM}
\glll hudəəsinagatɨɨ,  nun  kangəəgutoo  nən.jojaa.\\
\textit{hudəəs-i+nagatɨɨ}  \textit{nuu=n}  \textit{kangəər+kutu=ja}  \textit{nə-an=joo=jaa}\\
    bring.up-\textsc{inf}+along  what=any  think.INF+event=\textsc{top}  exist-\textsc{neg}=\textsc{cfm}1=\textsc{sol}\\
\glt    ‘While (you) are bringing up (your child), there is nothing to think about [i.e. you are in a trance].’ [Co: 120415\_01.txt]
\z

The compound \textit{hudəəs-i+nagatɨɨ} (bring.up-\textsc{inf}+along) ‘while (someone) is bringing up’ is similar to the special-type compound in (\ref{ex:4-27}a) in \sectref{sec:key:4.2.3.2.} However, they are different from each other since the former heads an adverbial clause. Further research is required for this expression.

\subsubsection{\textit{hutəə}/\textit{butəə}/\textit{datəə} ‘vicinity’}

I will present the examples of \textit{hutəə}, \textit{butəə}, and \textit{datəə}, meaning ‘vicinity’. \textit{hutəə} may be replaced by \textit{butəə} freely. In (\ref{ex:6-24}a), \textit{hutəə} ‘vicinity’ is modified by an adnominal \textit{u-n} (\textsc{mes}-\textsc{adnz}), and in (\ref{ex:6-24}b) it goes through compounding with \textit{kusi} ‘Kushi.’

\ea\label{ex:6-24}
\ea With an adnominal\\{}
[Context: Talking about \textsc{my}]\\
{\TM}
\glll attaaja,  un,  unhutəənan   wutancjɨjaa.\\
      \textit{a-rɨ-taa=ja}  \textit{u-n}  \textit{u-n=hutəə=nan} \textit{wur-tar-n=ccjɨ=jaa}\\
      \textsc{dist}-\textsc{nlz}-\textsc{pl}=\textsc{top}  \textsc{mes}-\textsc{adnz}  \textsc{mes}-\textsc{adnz}=vicinity=\textsc{loc1}     exist-\textsc{pst}-\textsc{ptcp}=\textsc{qt}=\textsc{sol}\\
\glt ‘(I heard) that she and her family were around there.’ [Co: 110328\_00.txt]

\ex Compounding\\
{\TM}
\glll  kusihutəənu  cˀju  zja.\\
\textit{kusi+hutəə=nu}  \textit{cˀju}  \textit{zjar}\\
Kushi+vicinity=\textsc{gen}  person  \textsc{cop}\\
\glt ‘(The person in the picture) is a person from around Kushi.’ [Co: 111113\_02.txt]
\z
\z

Similarly, \textit{datəə} ‘vicinity’ can be modified by an adnominal or undergoes compounding. In (\ref{ex:6-25}a), \textit{datəə} ‘vicinity’ is modified by an adnominal \textit{u-n} (\textsc{mes}-\textsc{adnz}), and in (\ref{ex:6-25}b) it goes through compounding with \textit{sutu} ‘outside.’

\ea\label{ex:6-25}
\ea With an adnominal\\
{\TM}
\glll  undatəəja  nuuga  aru?\\
\textit{u-n=datəə=ja}  \textit{nuu=ga}  \textit{ar-u}\\
\textsc{mes}-\textsc{adnz}=vicinity=\textsc{top}  what=\textsc{foc}  exist-\textsc{pfc}\\
\glt ‘What is around that place?’ [El: 120919]

\ex Compounding\\
{\TM}
\glll  kazɨ  hikijassa  atoo,  gan  sjɨ  natɨ,      sutudatəə  aikjankarajaa\\
\textit{kazɨ}  \textit{hik-i+jass-sa}  \textit{ar-too}  \textit{ga-n}  \textit{sɨr-tɨ}  \textit{nar-tɨ}  \textit{sutu+datəə}  \textit{aik-an=kara=jaa}\\
cold  draw-\textsc{inf}+easy-\textsc{adj}  \textsc{stv}-\textsc{csl}  \textsc{mes}-\textsc{advz}  do-\textsc{seq}  \textsc{cop}-\textsc{seq}   outside+vicinity  walk-\textsc{neg}=after=\textsc{sol}\\
\glt ‘(I) am liable to catch a cold, so (I) do not walk around outside.’ [Co: 120415\_01.txt]
\z
\z

So far, there is no example where \textit{hutəə}/\textit{butəə}/\textit{datəə} ‘vicinity’ is modified by an adnominal clause.

\subsubsection{ \textit{turoo} ‘place’}

I will present examples of \textit{turoo} ‘place.’ In (\ref{ex:6-26}a), \textit{turoo} ‘place’ is modified by an NP \textit{sugoja-taa} (Sugoya-\textsc{pl}), which fills the modifier slot by juxtaposition, and in (\ref{ex:6-26}b) it is modified by an adnominal clause whose head is the participle /asasan/ \textit{asa-sa+ar-n} (shallow-\textsc{adj}+\textsc{stv}-\textsc{ptcp}).

\ea\label{ex:6-26}
\ea With an NP filling the modifier slot by juxtaposition\\{}
[Context: Remembering a scene around \textsc{tm}’s house in the past]

{\TM}
\glll sugojataaturoobəi  jaanu  atanwake.\\
      \textit{sugoja-taa=turoo=bəi}  \textit{jaa=nu}  \textit{ar-tar-n=wake}\\
      Sugoya-\textsc{pl}=place=only  house=\textsc{nom}  exist-\textsc{pst}-\textsc{ptcp}=\textsc{cfp}\\
\glt ‘There was a house only at the Sugoya’s place.’ [Co: 120415\_00.txt]

\ex With an adnominal clause\\{}
[Context: Talking about how to carry woods using ships along the river]

{\TM}
\glll {\textbar}sijo{\textbar}nu  asasanturoo  jatɨn,\\
      \textit{sijo=nu}  \textit{asa-sa+ar-n=turoo}  \textit{jar-tɨ=n}\\
      tide=\textsc{nom}  shallow-\textsc{adj}+\textsc{stv}-\textsc{ptcp}=place  \textsc{cop}-\textsc{seq}=even\\
\glt ‘Even if it was the place where the tide was shallow, ...’ [Co: 111113\_01.txt]
\z
\z

\subsubsection{ \textit{mama} ‘still’}

I will present examples of \textit{mama} ‘still.’ In (\ref{ex:6-27}a), \textit{mama} ‘still’ is modified by an adnominal \textit{u-n} (\textsc{mes}-\textsc{adnz}), and in (\ref{ex:6-27}b) it goes through compounding with \textit{zitensja} ‘bicycle.’

\ea\label{ex:6-27}
\ea With an adnominal\\{}
[Context: Explaining how to make the pickles of white radish]

{\TM}
\glll unnan  unmama  {\textbar}bakecu{\textbar}nan  kan    sjɨ  tatɨtɨ  ukuboo,\\
      \textit{u-n=nan}  \textit{u-n=mama}  \textit{bakecu=nan}  \textit{ka-n}   \textit{sɨr-tɨ}  \textit{tatɨr-tɨ}  \textit{uk-boo}\\
      \textsc{mes}-\textsc{adnz}=\textsc{loc1}  \textsc{mes}-\textsc{adnz}=still  bucket=\textsc{loc1}  \textsc{prox}-\textsc{advz}  do-\textsc{seq}  stand-\textsc{seq}  put-\textsc{cnd}\\
\glt ‘If (you) stand (the white radishes with seasoning) there, in the bucket, as they are, ...’ [Co: 101023\_01.txt]

\ex Compounding\\
{\TM}
\glll  {\textbar}zitensja{\textbar}mama  hankəətɨ,\\
\textit{zitensja+mama}  \textit{hankəər-tɨ}\\
bicycle+still  tumble-\textsc{seq}\\
\glt ‘(The boy) tumbled while riding on the bicycle.’ [\textsc{pf}: 090225\_00.txt]
\z
\z

So far, there is no example in texts where \textit{mama} ‘still’ is modified by an adnominal clause.

\subsubsection{ \textit{tui} ‘as’}

I will present examples of \textit{tui} ‘as.’ In \REF{ex:6-28}, \textit{tui} ‘as’ is modified by the adnominal clause whose head is the participle /jˀicjan/ \textit{jˀ-tar-n} (say-\textsc{pst}-\textsc{ptcp}).

\ea\label{ex:6-28}
 With an adnominal clause\\

{\TM}
\glll {\textbar}zibunga{\textbar}  jˀicjantuidaroogaccjɨ  un  jingoo  jˀicjɨ,\\
\textit{zibun=ga}  \textit{jˀ-tar-n=tui=daroo=ccjɨ}  \textit{u-n}  \textit{jinga=ja}  \textit{jˀ-tɨ}\\
    \textsc{rfl}=\textsc{nom}  say-\textsc{pst}-\textsc{ptcp}=as=\textsc{supp}=\textsc{qt}  \textsc{mes}-\textsc{adnz}  mam=\textsc{top}  say-\textsc{seq}\\
\glt    ‘The man said that, “(It is) just as (I) myself said”, and ...’ [Fo: 090307\_00.txt]
\z

So far, there is no example in texts where \textit{tui} ‘as’ is modified by other than adnominal clauses.

\subsubsection{ \textit{hui} ‘pretend’}

I will present examples of \textit{hui} ‘pretend.’ In \REF{ex:6-29}, \textit{hui} ‘pretend’ is modified by the adnominal clause whose head is the participle \textit{sij-an} (know-\textsc{neg}).

\ea\label{ex:6-29}
 With an adnominal clause\\

{\TM}
\glll sijanhuikkwa  sjɨ,\\
\textit{sij-an=hui-kkwa  sɨr-tɨ}\\
    know-\textsc{neg}=pretend-\textsc{dim}  do-\textsc{seq}\\
\glt    ‘Pretending not to know (about the thrown snacks), ...’ [Co: 120415\_01.txt]
\z

So far, there is no example in texts where \textit{hui} ‘pretend’ is modified by other than adnominal clauses.

\section{Case}

Yuwan has fourteen case particles, which are clitics that follow an NP. They are classified into the argument case, which marks a dependent in a clause (nominative, accusative, dative 1, dative 2, allative, locative 1, locative 2, locative 3, instrumental, ablative, comitative, limitative, and comparative) and the genitive case, which marks a modifier in an NP. Yuwan has a nominative-accusative case marking system.

\begin{table}
\caption{\label{tab:key:40}. Case particles}
\begin{tabular}{lll}
\lsptoprule
Names & Forms & Prototypical functions \\
\midrule
Nominative & \textit{ga}/\textit{nu} & S, A \\
Accusative & \textit{ba} & P \\
Dative 1 & \textit{n} & beneficiary \\
Dative 2 & \textit{nkatɨ} & recipient of information \\
Allative & \textit{kaci} & goal of locomotion \\
Locative 1 & \textit{nan}/\textit{nən} & place of contact \\
Locative 2 & \textit{nantɨ}/\textit{nəntɨ}& location \\
Locative 3 & \textit{zjɨ} & location distant from the speaker \\
Instrumental & \textit{sjɨ} & instrument \\
Ablative & \textit{kara} & source \\
Comitative & \textit{tu} & participant of association \\
Limitative & \textit{gadɨ} & limit \\
Comparative & \textit{jukkuma} standard of comparison \\
Genitive & \textit{ga}/\textit{nu} & NP modifier \\
\lspbottomrule
\end{tabular}
\end{table}
I will discuss case particles in Yuwan in the following order. First, I will present the mophophonological alternation that are found in some case particles in \sectref{sec:key:6.3.1.} Some of the case particles undergo contraction with their preceding demonstrative nominals, i.e. \textit{ku-rɨ} (\textsc{prox}-\textsc{nlz}), \textit{u-rɨ} (\textsc{mes}-\textsc{nlz}), or \textit{a-rɨ} (\textsc{dist}-\textsc{nlz}), which was already discussed in \REF{ex:5-19} and \REF{ex:5-20} in \sectref{sec:key:5.2.1.} Second, the morphosyntax and semantics of each case particle is shown in \sectref{sec:key:6.3.2.} Thirdly, case particles that have similar functions are compared with one another in \sectref{sec:key:6.3.3.} Finally, the grammaticalization found in a few case particles in Yuwan will be discussed in \sectref{sec:key:6.3.4.}

\subsection{Morphophonology of case particles}

The following morphophonological alternations are found in the case particles in Yuwan

\ea\label{ex:6-30} Morphophonological alternations of case particles\\
    \ea fusion:         \textit{kaci} (\textsc{all}) (see \sectref{sec:key:6.3.1.1}); \textit{kara} (\textsc{abl}) (see \sectref{sec:key:6.3.1.2});
    \ex epenthesis:         \textit{n} (\textsc{dat}1) and \textit{nan} (\textsc{loc1}) (see \sectref{sec:key:6.3.1.3});
    \ex deletion:         \textit{nan} (\textsc{loc1}) and \textit{nantɨ} (\textsc{loc2}) (see \sectref{sec:key:6.3.1.4}).
    \z
\z

\subsubsection{Fusion of \textit{kaci} (\textsc{all})}
\label{bkm:Ref365151806}
If the allative case \textit{kaci} follows vowels, the following fusion frequently occurs. Please note that the fusion of //ci, si, zi// and \textit{kaci} requires a little attention because it forms not /Cəəci/ but /Cjəəci/.

\ea\label{ex:6-31}
 \ea High front vowel\\
    //  C  i  //  +  \textit{kaci} (\textsc{all})  >  /Cjəəci/

    [C is //c, s, z//]

    //  C  i  //      >  /Cəəci/

    [C is not //c, s, z//]

\ex High mid vowel\footnote{If the consonant before a mid-vowel is bilabial or velar, the fused form /əəci/ often sounds like [ɜːt͡ɕi] and [ɨ̞ːt͡ɕi], and the latter may be interpreted as /ɨɨci/. Audio-instrumental research is needed on this point in the future.}\\
    //  C  ɨ  //      >  /Cəəci/

\ex High back vowel\\
    //  C  u  //      >  /Cooci/

\ex Other short vowels\\
    //  C  V\textit{\textsubscript{i}}  //      >  /\textsc{cv}\textit{\textsubscript{i}}V\textit{\textsubscript{i} }ci/

\ex Long vowels and diphthongs\\
    //  V  V  //      >  /VVci/

\ex  Elsewhere\\
    //  C  //        >  /Ckaci/
\z
\z

The fusion of //i, ɨ, u// and \textit{kaci} (\textsc{all}) changes the original vowel positions, but the other short vowels retain their original positions. I will show examples below.

\ea\label{ex:6-32}
 \ea High front vowel  \\
    \textit{kuci}  ‘mouth’  +  \textit{kaci} (\textsc{all})  >  /kucjəəci/  (*/kucəəci/)\\
    \textit{kusi}  ‘(name of place)’      >  /kusjəəci/  (*/kusəəci/)\\
    \textit{tuzi}  ‘wife’      >  /tuzjəəci/  (*/tuzəəci/)\\
    \textit{kˀubi}  ‘neck’      >  /kˀubəəci/  

\ex High mid vowel\\
    \textit{umutɨ}  ‘front’  +  \textit{kaci} (\textsc{all})  >  /umutəəci/\\

\ex High back vowel  \\
    \textit{haku}  ‘box’  +  \textit{kaci} (\textsc{all})  >  /hakooci/

\ex Other short vowels\\
    \textit{jama}  ‘mountain’  +  \textit{kaci} (\textsc{all})  >  /jamaaci/\\
    \textit{kumamoto}  ‘(place name)’      >  /kumamotooci/  

\ex Long vowels or diphthongs  \\
    \textit{naa}  ‘inside’  +  \textit{kaci} (\textsc{all})  >  /naaci/\\
    \textit{hizjai}  ‘left’      >  /hizjaici/  

\ex  Elsewhere\\
\textit{mun}  ‘thing’  +  \textit{kaci} (\textsc{all})  >  /munkaci/\\
\z
\z

\subsubsection{Fusion of \textit{kara} (\textsc{abl})}
\label{bkm:Ref365151812}
The process of fusion in the ablative case \textit{kara} is the same as that of the allative case \textit{kaci} (see \sectref{sec:key:6.3.1.1}). The only difference between them is the phonemes in their final syllables, i.e., the former is /ra/ and the latter is /ci/.

\ea\label{ex:6-33} \ea High front vowel\\
    //  C  i  //  +  \textit{kara} (\textsc{abl})  >  /Cjəəra/\\{}
    [C is //c, s, z//]\\
    //  C  i  //      >  /Cəəra/\\{}
    [C is not //c, s, z//]

\ex High mid vowel\footnote{If the consonant before a mid-vowel is bilabial or velar, the fused form /əəra/ often sounds like both [ɜːra] and [ɨ̞ːra], and the latter may be interpreted as /ɨɨra/. Audio-instrumental research is needed on this point in the future.}\\
    //  C  ɨ  //      >  /Cəəra/

\ex High back vowel\\
    //  C  u  //      >  /Coora/

\ex Other short vowels\\
    //  C  V\textit{\textsubscript{i}}  //      >  /\textsc{cv}\textit{\textsubscript{i}}V\textit{\textsubscript{i} }ra/

\ex Long vowels and diphthongs\\
    //  V  V  //      >  /VVra/

\ex  Elsewhere\\
    //  C  //        >  /Ckara/
\z
\z

The fusion of //i, ɨ, u// and \textit{kara} (\textsc{abl}) changes the original vowel positions, but the other short vowels retain their original positions. I will show examples below.

\ea\label{ex:6-34}
 \ea High front vowel  \\
    \textit{kuci}  ‘mouth’  +  \textit{kara} (\textsc{abl})  >  /kucjəəra/  (*/kucəəra/)\\
    \textit{kusi}  ‘(name of place)’      >  /kusjəəra/  (*/kusəəra/)\\
    \textit{tuzi}  ‘wife’      >  /tuzjəəra/  (*/tuzəəra/)\\
    \textit{kˀubi}  ‘neck’      >  /kˀubəəra/  

\ex High mid vowel\\
    \textit{umutɨ}  ‘front’  +  \textit{kara} (\textsc{abl})  >  /umutəəra/

\ex High back vowel  \\
    \textit{atu}  ‘later’  +  \textit{kara} (\textsc{abl})  >  /atoora/

\ex Other short vowels\\
    \textit{jama}  ‘mountain’  +  \textit{kara} (\textsc{abl})  >  /jamaara/\\
    \textit{kumamoto}  ‘(place name)’      >  /kumamotoora/  

\ex Long vowels or diphthongs  \\
    \textit{naa}  ‘inside’  +  \textit{kara} (\textsc{abl})  >  /naara/\\
    \textit{hizjai}  ‘left’      >  /hizjaira/  

\ex  Elsewhere\\
\textit{unin}  ‘that time’  +  \textit{kara} (\textsc{abl})  >  /uninkara/\\

\z
\z
\subsubsection{Epenthesis of dative case 1 \textit{n} and locative case \textit{nan} (\textsc{loc1})}

A syllable must have a nucleus filled by a vowel (see \sectref{sec:key:2.3.1}). Thus, if the dative case \textit{n} or locative case \textit{nan} (\textsc{loc1}) happens to precede a syllable filled by a single consonant at a morpheme boundary, an epenthetic vowel /i/ is inserted as a nucleus.

\ea\label{ex:6-35}
    Ø  >  /i/  /  %%[Warning: Draw object ignored]
    \textit{n}  %%[Warning: Draw object ignored]
    (\textsc{dat}1)    \_  //n\#//

    \textit{nan}  (\textsc{loc1})

    \ea\label{ex:6-36}
    Dative\\
        \textit{jinga}  ‘man’  +  \textit{n}  (\textsc{dat}1)  +  \textit{n}  ‘also’\\
        >  /jinga.      ni      n/

    \ex Locative 1\\
        \ea  \textit{kun}  (\textsc{prox}.\textsc{adnz})  +  \textit{nan}  (\textsc{loc1})  +  \textit{n}  ‘also’\\
            >  /kun      na.ni      n/
        \ex  \textit{kun}  (\textsc{prox}.\textsc{adnz})  +  \textit{nən}  (\textsc{loc1})  +  \textit{n}  ‘also’\\
            >  /kun      nə.ni      n/
        \z
    \z
\z

In cases where \textit{n} (\textsc{dat}1) follows a syllable-final //n// (instead of preceding //n// such as (\ref{ex:6-36}a)), an epenthetic vowel /u/ is inserted between them by the application of a phonological rule disucussed in \sectref{sec:key:2.4.3}, e.g. \textit{bun} ‘the Bon festival’ + \textit{n} (DAT1) > /bu.nun/. This raises the question of what happens in cases where \textit{n} (DAT1) is surrounded by //n//s. In those cases, as mentioned before (at the beginning of \sectref{sec:key:2.4}), the morphophonemic rule \REF{ex:6-35} applies first, and that is sufficient in order to adjust the syllable structure.

\ea\label{ex:6-37}
    \textit{wan}  (1\textsc{sg})  +  \textit{n}  (\textsc{dat}1)  +  \textit{n}  ‘also’\\
  >  /wan.      ni      n/  \\
    */wa.nu.      ni      n/  
\z

\subsubsection{Deletion in locative cases \textit{nan} (\textsc{loc1}) and \textit{nantɨ} (\textsc{loc2})}

The locative cases \textit{nan} (\textsc{loc1}) and \textit{nantɨ} (\textsc{loc2}) may become /n/ and /ntɨ/ respectively, i.e., //na// in their initial positoin may be deleted, when they follow a vowel.

%%[Warning: Draw object ignored]
\ea\label{ex:6-38}
  \textit{nan}  (\textsc{loc1})  > \\ %%[Warning: Draw object ignored]
/n/    /  //V//  \_\\
  \textit{nantɨ}  (\textsc{loc2})  >  /ntɨ/
  \z

\ea\label{ex:6-39}
 \ea Locative 1      \\
    \textit{kuma}  ‘here’  +  \textit{nan}  (\textsc{loc1})  +  \textit{nu}  (\textsc{gen})\\
  >  /kuma      n      nu/  \\

\ex Locative 2      \\
    sja  ‘lower side’  +  nantɨ  (\textsc{loc2})\\
  >  /sja      ntɨ/        
  \z
  \z

Additionally, if the locative case \textit{nan} (\textsc{loc1}) follows a vowel and also precedes a syllable filled by a single consonant, it becomes /ni/. In other words, //na// is deleted with \textit{i}-insertion (see \sectref{sec:key:6.3.1.3}).

\ea\label{ex:6-40}
  \textit{nan}  (\textsc{loc1})  >  /ni/  /  //V//  \_  //C\#//
\z

\ea\label{ex:6-41}
  Input form  \textit{ui}  ‘upper side’  +  \textit{nan}  (\textsc{loc1})  +  \textit{n}  ‘also’\\
  //na// deletion:  ui      n      n  \\
  /i/ insertion:  ui      ni      n  \\
  Output form  /ui.      ni      n/  
\z

When it is not followed by a syllable filled by a single consonant, it is preferred to avoid the deletion of //na//. That is, \textit{kuma} (\textsc{prox}.place) + \textit{nan} (\textsc{loc1}) > /kuma=nan/ is prefferred. In fact, /kuma=n/ is judged as possible when I asked my consultants whether it can be used, but it is rarely uttered not only in the discourse, but also in elicitation. For this reason, the /ni/ is not regarded as the dative case \textit{n}, but is regarded as the deleted (and \textit{i}-inserted) form of \textit{nan} (\textsc{loc1}). Moreover, interpreting this /n/ as the deleted form of \textit{nan} (\textsc{loc1}) makes it easy to see the correspondence between \textit{nan} (\textsc{loc1}) and \textit{nantɨ} (\textsc{loc2}).

\subsection{Syntax and semantics of case particles}

The fourteen case particles, i.e. the argument cases (nominative, accusative, dative 1, dative 2, allative, locative 1, locative 2, locative 3, instrumental, ablative, comitative, limitative, and comparative,) and the genitive case, are discussed in the following subsections in turn.

\subsubsection{Nominative case \textit{ga}/\textit{nu}}
\label{bkm:Ref366360662}
The nominative case has two morphemes \textit{ga} and \textit{nu}, and they are chosen depending on the lexical meanings (or the animacy hierarchy) of their head nominals (see also \sectref{sec:key:4.1.1} and \sectref{sec:key:6.4.3} for more details). The nominative case is used in the following environments.

\ea\label{ex:6-42}
 Nominative case is used to mark,\\

 \begin{itemize}
\item[a.]   Subject of predicates;
\item[b.] Object of transitive verb that expresses incapability;
\item[c.] Predicate NP of the subordinate clause in negative;
\item[d.] Lexical verb in the \textsc{av}C that expresses incapability or includes /nə-n/ (\textsc{rsl}-\textsc{neg});
\item[e.] Infinitives in the complement slot of \textsc{lvc} that expresses incapability;
\item[f.] Object of \textit{wakar-} ‘understand.’
\end{itemize}
I will present examples of (\ref{ex:6-42}a-f) in turn below.

With regard to (\ref{ex:6-42}a), the nominative case is used to mark the subject of intransitive verb, transitive verb, or copula verb.

\ea\label{ex:6-43}
\ea Subject of verbal predicates (intransitive verb)\\{}
[Context: Remembering \textsc{tm}’s mother who knew traditional things very much]

{\TM}
\glll anmataaga  wuppoojaa.\\
      \textit{anmaa-taa=ga}  \textit{wur-boo=jaa}\\
      mother-\textsc{pl}=\textsc{nom}  exist-\textsc{cnd}=\textsc{sol}\\
\glt ‘If (my) mother were here, (it would be good).’ [Co: 110328\_00.txt]

\ex Subject of verbal predicates (transitive verb)\\{}
[Context: Rembering a scene from the Pear Film]

{\TM}
\glll uziiga  mutɨ,  un  kˀwanu  muccjɨ  izjɨ,\\
      \textit{uzii=ga}  \textit{mur-tɨ}  \textit{u-n}  \textit{kˀwa=nu}     \textit{mut-tɨ}  \textit{ik-tɨ}\\
      old.man=\textsc{nom}  pick.up-\textsc{seq}  \textsc{mes}-\textsc{adnz}  child=\textsc{nom} have-\textsc{seq}  go-\textsc{seq}\\
\glt ‘The old man picked up (the pears), and the child brought (them), and ...’ [\textsc{pf}: 090827\_02.txt]

\ex Subject of adjectival predicates\\
{\TM}
\glll  nama  haanu  awusan  ucin,\\
\textit{nama}  \textit{haa=nu}  \textit{awu-sa+ar-n}  \textit{uci=n}\\
still  leaf=\textsc{nom}  green-\textsc{adj}+\textsc{stv}-\textsc{ptcp}  during=\textsc{dat}1\\
\glt ‘While the leaves were still green, ...’ [Co: 101023\_01.txt]

\ex Subject of nominal predicates\\{}
[Context: Looking at a picture]

{\TM}
\glll kumaga  hasi  jappa.\\
      \textit{ku-ma=ga}  \textit{hasi}  \textit{jar-ba}\\
      \textsc{prox}-place=\textsc{nom}  bridge  \textsc{cop}-\textsc{csl}\\
\glt ‘Since here is a bridge.’ [Co: 120415\_00.txt]
\z
\z

In (\ref{ex:6-43}a), /anmataa/ \textit{anmaa-taa} (mother-\textsc{pl}) is the subject of the verbal predicate (whose head is the intransitive verb \textit{wur-} ‘exist’), and it takes the nominative case particle \textit{ga}. In (\ref{ex:6-43}b), \textit{uzii} ‘old man’ is also the subject of the verbal predicate (whose head is the transitive verb \textit{mur-} ‘pick up’), and it takes the nominative case particle \textit{ga}. Similarly, \textit{u-n} \textit{kˀwa} (\textsc{mes}-\textsc{adnz} child) ‘that child’ is the subject of the verbal predicate (whose head is the transitive verb \textit{mut-} ‘have’), and it takes the nominative case particle \textit{nu}. In (\ref{ex:6-43}c), \textit{haa} ‘leaf’ is the subject of the adjectival predicate (whose head is \textit{awu-sa} (blue-\textsc{adj}) ‘blue’), and it takes the nominative case particle \textit{nu}. In (\ref{ex:6-43}d), \textit{ku-ma} (\textsc{prox}-place) ‘here’ is the subject of the nominal predicate, and it takes the nominative case particle \textit{ga}. It should be noted that there are some situations where the nominative case does not appear. For example, the subject of an imperative sentence usually does not appear, but sometimes it can appear. In that case, the subject does not take the nominative case.

\ea\label{ex:6-44}
 Subjects of imperative\\
 \ea\relax[Context: \textsc{tm} tried to make \textsc{my} pronounce the word for ‘knee’ in Yuwan.]\\
{\TM}
\glll  ura  jˀicjɨn  njɨ!\\
\textit{ura}  \textit{jˀ-tɨ=n}  \textit{nj-ɨ}\\
2.\textsc{nhon}.\textsc{sg}  say-\textsc{seq}=also  \textsc{exp}-\textsc{imp}\\
\glt ‘You try to say (it)!’ [Co: 110328\_00.txt]

\ex\relax[Context: \textsc{tm} asked \textsc{ms} to make the topic of their conversation for recording.]\\
{\TM}
\glll  ura  {\textbar}wadai{\textbar}  cɨkɨtɨ  kurɨppa.\\
\textit{ura}  \textit{wadai}  \textit{cɨkɨr-tɨ}  \textit{kurɨr-ba}\\
2.\textsc{nhon}.\textsc{sg}  topic  make-\textsc{seq}  \textsc{ben}-\textsc{csl}\\
\glt ‘Would you please make the topic (of our conversation)?’ [Co: 120415\_01.txt]
\z
\z

The subjects of the above examples, i.e. \textit{ura} ‘you’, do not take any case in imperative sentences. Moreover, if the NP is followed by \textit{ja} (\textsc{top}), \textit{du} (\textsc{foc}), \textit{ga} (\textsc{foc}), and \textit{n} ‘also; even; any’, the nominative case cannot occur (see \sectref{sec:key:10.1}).

  With regard to (\ref{ex:6-42}b), there are examples, where the nominative case does not mark the subject of the clause, but mark the object. In such a case, the clause expresses “incapability,” and it should use \textit{ga} (not \textit{nu}) with a verb containing \textit{-an} (\textsc{neg}) (see \sectref{sec:key:6.4.3.3} for more details).

\ea\label{ex:6-45}
 Objects of the transitive verbs\\

 \ea Object taking \textit{ga} (\textsc{nom})\\
{\TM}
\glll  wanna,  joo,  anmai  hanməəja,  hanməəga  kaman  cˀju  natɨ  cˀjɨjoo.\\
\textit{wan=ja}  \textit{joo}  \textit{anmai}  \textit{hanməə=ja} \textit{hanməə=ga}  \textit{kam-an}  \textit{cˀju}  \textit{nar-tɨ}  \textit{k-tɨ=joo}\\
1\textsc{sg}=\textsc{top}  \textsc{fil}  so.much  meal=\textsc{top}  meal=\textsc{nom}  eat-\textsc{neg}  person  become-\textsc{seq}  come-\textsc{seq}=\textsc{cfm}1\\
\glt ‘I, (about) the meal, came to be a (kind of) person who cannot eat the meal so much.’ [Co: 120415\_01.txt]

\ex Object taking \textit{ba} (\textsc{acc})\\
{\TM}
\glll  hanməəba  kamanboojaa\\
\textit{hanməə=ba}  \textit{kam-an-boo=jaa}\\
meal=\textsc{acc}  eat-\textsc{neg}-\textsc{cnd}=\textsc{sol}\\
\glt ‘(We) have to eat the meal.’ [Co: 101020\_01.txt]
\z
\z

In (\ref{ex:6-45}b), the verb is \textit{kam-} ‘eat’ and its object, i.e. \textit{hanməə} ‘meal’, is followed by the accusative case \textit{ba}, which is a regular case marking for the object (see \sectref{sec:key:6.3.2.2}). In (\ref{ex:6-45}a), however, the object of the same verb takes \textit{ga} (\textsc{nom}), with a meaning of incapability. Other examples are also shown below.

\ea\label{ex:6-46}
 Objects of the transitive verbs\\

\ea
{\TM}
\glll {\textbar}wadai{\textbar}ga  sɨranba.\\
\textit{wadai=ga}  \textit{sɨr-an-ba}\\
topic=\textsc{nom}  do-\textsc{neg}-\textsc{csl}\\
\glt ‘(I) cannot initiate a topic, so ...’ [Co: 120415\_01.txt]

\ex
{\TM}
\glll hanasimiciga  sijanbajaa.\\
\textit{hanas-i+mici=ga}  \textit{sij-an-ba=jaa}\\
talk-\textsc{inf}+way=\textsc{nom}  know-\textsc{neg}-\textsc{csl}=\textsc{sol}\\
\glt ‘(I) don’t know the way to talk (well), so (I cannot communicate well with the present author).’ [Co: 120415\_01.txt]
\z
\z

The clauses in \REF{ex:6-45} and \REF{ex:6-46} express incapability in spite of there being no morphemes to express capability such as \textit{-ar} (\textsc{cap}) or \textit{kij-} (CAP).

  With regard to (\ref{ex:6-42}c), an NP in the predicate phrase [i.e. the nominal predicate] usually does not take any case particle, but if it is in negative and also in the adverbial (or adnominal) clause, it takes one of the nominative case particles (see \sectref{sec:key:9.3.3.1}).

\ea\label{ex:6-47}
 [= (\ref{ex:5-9}b)]\\
{\TM}
\gllll uraga  tumainu  aran  tukin,\\
\textit{ura=ga}  \textit{tumai=nu}  \textit{ar-an}  \textit{tuki=n}\\
    2.\textsc{nhon}.\textsc{sg}=\textsc{nom}  night.duty=\textsc{nom}  \textsc{cop}-\textsc{neg}  time=\textsc{dat}1\\
    Subject  [NP  Copula verb]\textsubscript{Nomimal predicate phrase}  \\
\glt    ‘When you are not on night duty, ...’ [Co: 111113\_02.txt]
\z

The above example shows that not only the subject, i.e. \textit{ura=ga} (2.\textsc{nhon}.\textsc{sg}=\textsc{nom}), but also the NP in the predicate, i.e. \textit{tumai=nu} (night.duty=\textsc{nom}), take the nominative case.

  With regard to (\ref{ex:6-42}d), the nominative case can be used to mark the lexical verbs in the auxiliary verb construction (\textsc{av}C) that express incapability or includes /nə-n/ \textit{nə-an} (\textsc{rsl}-\textsc{neg}).

\ea\label{ex:6-48}
 Lexical verbs in \textsc{av}C expressing incapability\\

 \ea
 {\TM}
\glll kumɨnkjanu  nənboo,  kadɨga  ikjankara,\\
\textit{kumɨ=nkja=nu}  \textit{nə-an-boo}  \textit{kam-tɨ=ga}  \textit{ik-an=kara}\\
rice=\textsc{appr}=\textsc{nom}  exist-\textsc{neg}-\textsc{cnd}  eat-\textsc{seq}=\textsc{nom}  go-\textsc{neg}=\textsc{csl}\\
\glt ‘If there is no food such as rice, (we) cannot live, so ...’ [Co: 120415\_01.txt]

  Lexical verbs in \textsc{av}C whose auxiliary verb is /nə-n/ \textit{nə-an} (\textsc{rsl}-\textsc{neg})

\ex\relax[Context: Wondering whther the owner of the electric shop is there; \textsc{my}: ‘(He) may be there.’]\\
{\TM}
\glll  naa,  unmama  hanməə  kamgjaa  izjɨnu\\                                                                                        nənboo.  ikjasjɨgajaaroo.\\
      \textit{naa}  \textit{u-n=mama}  \textit{hanməə}  \textit{kam-Ø+gjaa}  \textit{ik-tɨ=nu}                                        \textit{nə-an-boo  ikja-sjɨ=gajaaroo}\\
      \textsc{fil}  \textsc{mes}-\textsc{adnz}=still  meal  eat-\textsc{inf}+\textsc{purp}  go-\textsc{seq}=\textsc{nom}              exist-\textsc{neg}-\textsc{cnd}  how-\textsc{advz}=\textsc{dub}\\
\glt ‘If (he) has not gone to eat the meal yet (and if he is not still out) that, (he may be there). (But actually I) wonder if (he is).’ [Co: 110328\_00.txt]

\ex\relax[Context: Talking about the beam in the ceiling; {\TM} ‘(The beam) of your house is very white.’; \textsc{ms}: ‘Yeah, (it) is not as black as yours.’; TM: ‘(Yours) is not black, I suppose. ...’]\\
{\TM}
\glll  məəcjɨga  nənba.\\
\textit{məəs-tɨ=ga}  \textit{nə-an-ba}\\
fire-\textsc{seq}=\textsc{nom}  exist-\textsc{neg}-\textsc{csl}\\
\glt ‘(Your family) has not burned (wood as we did in my place, where the kitchen was very close by), so (yours is white).’ [Co: 111113\_01.txt]
\z
\z

In (\ref{ex:6-48}a), the lexical verb in the \textsc{av}C, i.e. /kadɨ/ \textit{kam-tɨ} (eat-\textsc{seq}), takes \textit{ga} (\textsc{nom}). The predicate means incapability, although there is no verbal morpheme to express capability such as \textit{kij-} (\textsc{cap}) or \textit{ar-} (CAP), which is similar to the cases in \REF{ex:6-45} and \REF{ex:6-46}. In (\ref{ex:6-48}b-c), the lexical verbs in the \textsc{avc}s, i.e. /izjɨ/ \textit{ik-tɨ} (go-\textsc{seq}) and /məəcjɨ/ \textit{məəs-tɨ} (fire-\textsc{seq}), take \textit{nu} (\textsc{nom}) or \textit{ga} (\textsc{nom}) (see also \sectref{sec:key:9.1.1.1}).

With regard to (\ref{ex:6-42}e), the nominative case can be used to mark the infinitives in the complement slot of \textsc{lvc} that expresses incapability.

\ea\label{ex:6-49}
 Infinitive in the complement slot of \textsc{lvc}\\

{\TM}
\gllll aikiga  siikijanba.\\
\textit{aik-i=ga}  \textit{sɨr-i+kij-an-ba}\\
    walk-\textsc{inf}=\textsc{nom}  do-INF+\textsc{cap}-\textsc{neg}-\textsc{csl}\\
    Complement  \textsc{lv}\\
\glt    ‘(I) cannot walk [lit. do walking], so (I cannot bring the pickles from my house).’
  [Co: 120415\_01.txt]
\z

In \REF{ex:6-49}, the infinitive in the complement slot of the light verb \textit{sɨr-} ‘do,’ i.e. \textit{aik-i} (walk-\textsc{inf}), takes \textit{ga} (\textsc{nom}) (see also \sectref{sec:key:9.1.2.1}).

  With regard to (\ref{ex:6-42}f), the nominative case can be used to mark the object of \textit{wakar-} ‘understand; know.’

\ea\label{ex:6-50}
  To mark the object of \textit{wakar-} ‘understand.’
 \ea
 {\TM}
\glll  un  {\textbar}zjookjoo{\textbar}nu  wakajui?\\
\textit{u-n}  \textit{zjookjoo=nu}  \textit{wakar-jur-i}\\
\textsc{mes}-\textsc{adnz}  situation=\textsc{nom}  understand-\textsc{umrk}-\textsc{npst}\\
\glt ‘Do (you) understand the situation (that I told)?’ [\textsc{pf}: 090827\_02.txt]

\ex
{\TM}
\glll jakɨtəəranu  atuga  wakaran.\\
\textit{jakɨr-təəra=nu}  \textit{atu=ga}  \textit{wakar-an}\\
burn-after=\textsc{gen}  after=\textsc{nom}  know-\textsc{neg}\\
\glt ‘(I) don’t know (what happened) after (the houses) burnt.’ [Co: 120415\_01.txt]
\z
\z

  Before concluding this section, I will present the examples where the nominative can follow another case particle as in (\ref{ex:6-51}a-b).

\ea\label{ex:6-51}
 Nominative following another case\\

 \ea
 {\TM}
\glll  kumakaciga  asikenkai?\\
\textit{ku-ma=kaci=ga}  \textit{asiken=kai}\\
\textsc{prox}-place=\textsc{all}=\textsc{nom}  Ashiken=\textsc{dub}\\
\glt ‘(The area) from here is Ashiken?’ [Co: 111113\_01.txt]

\ex
{\TM}
\glll kun  cˀjutu  kun  cˀjutuga  dɨkɨmun.jo.\\
\textit{ku-n}  \textit{cˀju=tu}  \textit{ku-n}  \textit{cˀju=tu=ga} \textit{dɨkɨmun=joo}\\
\textsc{prox}-\textsc{adnz}  person=\textsc{com}  \textsc{prox}-\textsc{adnz}  person=\textsc{com}=\textsc{nom}   genius=\textsc{cfm}1\\

\glt ‘This person and this person are genius.’ [Co: 120415\_00.txt]
\z
\z

The above examples show that the nominative case can follow another case particle when they are the subjects of the nominal predicates.

\subsubsection{Accusative case \textit{ba}}

The accusative case \textit{ba} is normally used to mark the object of transitive verbs. In (\ref{ex:6-52}a), \textit{ura} ‘you’ is an animate pronoun and the object of a transitive verb \textit{abɨr-} ‘call.’ In (\ref{ex:6-52}b), \textit{nasi} ‘pear’ is an inanimate common noun and also the object of a transitive verb \textit{mur-} ‘pick up.’

\ea\label{ex:6-52}
\ea Object of transitive verb (animate pronoun)\\
{\TM}
\glll  mattaku  wakaranba,  uraba  abɨranboo.\\
\textit{mattaku}  \textit{wakar-an-ba}  \textit{ura=ba}  \textit{abɨr-an-boo}\\
at.all  understand-\textsc{neg}-\textsc{csl}  2.\textsc{nhon}.\textsc{sg}=\textsc{acc}  call-\textsc{neg}-\textsc{cnd}\\
\glt ‘I called you because if (I) don’t call you, (I) won’t understand (what I should do) at all.’ [Co: 101023\_01.txt]

\ex Object of transitive verb (inanimate common noun) [= (\ref{ex:4-2}a)]\\
{\TM}
\glll  nasiba  tˀɨɨ  tˀɨɨ  mutunwakejo.\\
\textit{nasi=ba}  \textit{tˀɨɨ}  \textit{tˀɨɨ}  \textit{mur-tur-n=wake=joo}\\
pear=\textsc{acc}  one.\textsc{clf}  one.CLF  pick.up-\textsc{prog}-\textsc{ptcp}=\textsc{cfp}=\textsc{cfm}1\\
\glt ‘(The old man) is picking up pears one by one.’ [\textsc{pf}: 090222\_00.txt]
\z
\z

Both object NPs in (\ref{ex:6-52}a-b) take the accusative case particle \textit{ba}. Additionally, the accusative case \textit{ba} can be omitted as follows.

\ea\label{ex:6-53}
 Patient of transitive verb (inanimate common noun)\\

{\TM}
\glll uziiga  daibangɨɨnantɨ  nasi  mutunwake.\\
\textit{uzii=ga}  \textit{daiban+kɨɨ=nantɨ}  \textit{nasi}  \textit{mur-tur-n=wake}\\
    old.man=\textsc{nom}  big+tree=\textsc{loc2}  pear  pick.up-\textsc{prog}-\textsc{ptcp}=\textsc{cfp}\\
\glt    ‘An old man is picking pears off on a big tree.’ [\textsc{pf}: 090305\_01.txt]
\z

In both (\ref{ex:6-52}b) and \REF{ex:6-53}, the NP \textit{nasi} ‘pear’ is the object argument of the verb \textit{mur-} ‘pick up.’ On the one hand, the former takes \textit{ba} (\textsc{acc}); on the other hand, the latter does not take any case. So far, such an omission of \textit{ba} (\textsc{acc}) has rarely been found when the object is a personal pronoun, a human demonstrative, or an address noun (except for the causative construction discussed in (\ref{ex:8-122}b) in \sectref{sec:key:8.5.1.1}). (The example of commoun noun, however, was found in \REF{ex:8-26} in \sectref{sec:key:8.3.1.2}, which is taken from the elicitation.) In fact, these lexical groups appeared so many times in the text, but there are only a few instances where they are used as objects. Therefore, it is difficult to know whether it is impossible that \textit{ba} (\textsc{acc}) is really unable to be omitted after these lexical groups. Mitsukaido, which is a dialect of Japanese, has two accusative forms, one of which has a phonetic form, i.e. \textit{godo}, but the other does not (zero form), and the choice of them depends on the animacy of their head NP (Sasaki: 2004: 129). In Yuwan, the choice of \textit{ba} (\textsc{acc}) is not restricted by the animacy of its head NP, but there is a possibility that the omissibility of the accusative case is influenced by the animacy of the head of an NP. The omissionability of accusative case particle after an inanimate referent NP seems to have a relation with one of the components of transitivity “I\textsc{ndividuation} \textsc{of} O” in Hopper and \citet[252]{Thompson1980}.

It should be noted that the accusative case \textit{ba} can be used to mark the goal of (deictic) locomotion verbs.

\ea\label{ex:6-54}
 Goal of a deictic locomotion verb\\

 \ea
 {\TM}
 [Context: Speaking about an aquaintance] = (\ref{ex:4-51}c)\\
\glll nasjeba  izjɨ  cˀjəəroo,  akka  taməə  naa  issai  warusoo  jantatto.\\
\textit{nasje=ba}  \textit{ik-tɨ}  \textit{k-təəra=ja}  \textit{a-rɨ=ga}  \textit{taməə}      \textit{naa}  \textit{issai}  \textit{waru-soo}  \textit{jˀ-an-tar-too}\\
Naze=\textsc{acc}  go-\textsc{seq}  come-after=\textsc{top}  \textsc{dist}-\textsc{nlz}=\textsc{gen}  sake    already  all  bad-\textsc{adj}  say-\textsc{neg}-\textsc{pst}-\textsc{csl}\\
\glt ‘After going to and returning from Naze, (she) did not say anything bad about him.’ [Co: 101023\_01.txt]

\ex
{\TM}
\glll  jama  izjɨ,\\
\textit{jama}  \textit{ik-tɨ}\\
\textit{mountain}  go-\textsc{seq}\\\\
\glt ‘(The people) go to the mountain (to get wood to make a coffin), and ...’ [Co: 111113\_01.txt]
\z
\z

In (\ref{ex:6-54}a), the locomotion verb \textit{ik-} ‘go’ takes \textit{ba} (\textsc{acc}) to mark the goal NP, i.e. \textit{nasje} ‘Naze.’ In (\ref{ex:6-54}b), the goal NP is not marked by any case particle. In fact, both of the accusative case \textit{ba} (\textsc{acc}) and the allative case \textit{kaci} (\textsc{all}) can mark the goal of locomotion verbs (see \sectref{sec:key:6.3.2.5}). Thus, it is difficult to determine the omitted case particle in (\ref{ex:6-54}b). The verbs that can take \textit{ba} (\textsc{acc}) for the goal of locomotion are all deictic locomotion verbs, i.e. \textit{ik-} ‘go,’ \textit{k-} ‘come,’ and \textit{umoor-} ‘go; come (honorific).’

  Before conclusion, it should be noted that the accusative particle \textit{ba} is different from the topic particle \textit{ja}. Therefore, they can make a sequence as in \REF{ex:10-7} in \sectref{sec:key:10.1.1.2.}

\subsubsection{Dative case 1 \textit{n}}

The dative case 1 \textit{n} has a wide range of use: beneficiary, causee, agent of passive construction, agent of verbs to express capability, and time. It is also used to mark the benefactor (in a broad sense), whose examples will be shown (\ref{ex:9-18}b) in \sectref{sec:key:9.1.1.3.}

\ea\label{ex:6-55}
\ea Beneficiary\\
{\TM}
\glll  nuu  jatɨn  sɨgu  cˀjun  \textit{kurɨcjasa}  \textit{sii}  \textit{natɨjo.}\\
\textit{nuu}  \textit{jar-tɨ=n}  \textit{sɨgu}  \textit{cˀju=n}  \textit{kurɨr-cja-sa}      \textit{sɨr-i}  \textit{nar-tɨ=joo}\\
what  \textsc{cop}-\textsc{seq}=even  soon  person=\textsc{dat}1  give-want-\textsc{adj}   do-\textsc{inf}  become-\textsc{seq}=\textsc{cfm}1\\
\glt ‘Whatever it is, (I) feel like wanting to give (it) to a person without hesitation.’ [Co: 120415\_01.txt]

\ex Causee\\
{\TM}
\glll  arɨn  karasoojəə.\\
\textit{a-rɨ=n}  \textit{kar-as-oo=jəə}\\
\textsc{dist}-\textsc{nlz}=\textsc{dat}1  borrow-\textsc{caus}-\textsc{int}=\textsc{cfm}2\\
\glt ‘(I) will make that person borrow (it).’ [El: 120921]

\ex Agent of passive construction\\{}
[Context: An old man found gold under the ground, but he did not bring it home, so his wife was surprised to hear that.]

{\TM}
\glll gan  jiccjan  mun  həəku  tutɨ   konboo,  cˀjun  tɨmɨrarɨɨdoocjɨ  jˀicjanmun,\\
      \textit{ga-n}  \textit{jiccj-sa+ar-n}  \textit{mun}  \textit{həə-ku}  \textit{tur-tɨ}  \textit{k-on-boo}  \textit{cˀju=n}  \textit{tɨmɨr-arɨr-Ø=doo}  \textit{jˀ-tar-n=mun}\\
      \textsc{mes}-\textsc{advz}  good-\textsc{adf}+\textsc{stv}-\textsc{ptcp}  thing  early-\textsc{advz}  take-\textsc{seq} come-\textsc{neg}-\textsc{cnd}  person=\textsc{dat}1  find-P\textsc{ass}-\textsc{inf}=\textsc{ass}  say-\textsc{pst}-\textsc{ptcp}=\textsc{advrs}\\
\glt ‘(The wife) said that, “If (you) don’t bring such a good thing, (it) will be found by another person,” but ...’ [Fo: 090307\_00.txt]

\ex Agent of verbs to express capability\\
{\TM}
\glll  wannin  kakarɨssa.\\
\textit{wan=n=n}  \textit{kak-arɨr-sa}\\
1\textsc{sg}=\textsc{dat}1=also  write-\textsc{cap}-\textsc{pol}\\
\glt ‘I also can write (it).’ [El: 121001]

\ex Time\\
{\TM}
\glll  ɨcɨnkuin  attu  hanasjun   tukinnja,\\
\textit{ɨcɨɨ=n=kui=n}  \textit{a-rɨ=tu}  \textit{hanas-jur-n} \textit{tuki=n=ja}\\
when=any=\textsc{indf}=any  \textsc{dist}-\textsc{nlz}=\textsc{com}  talk-\textsc{umrk}-\textsc{ptcp}    time=\textsc{dat}1=\textsc{top}\\
\glt ‘Whenever (I) talk with him, ...’ [Co: 111113\_02.txt]
\z
\z

In (\ref{ex:6-55}a), \textit{cˀju} ‘person’ is the beneficiary of the verb \textit{kurɨr-} ‘give’ and takes \textit{n} (\textsc{dat}1). In (\ref{ex:6-55}b), \textit{a-rɨ} ‘that person’ is the causee of the verb \textit{kar-as-} (borrow-\textsc{caus}) ‘make (someone) borrow’ and takes \textit{n} (DAT1). In (\ref{ex:6-55}c), \textit{cˀju} ‘person’ is the agent of the passive construction whose predicate includes the passive affix \textit{-arɨr} and it takes \textit{n} (DAT1). In (\ref{ex:6-55}d), \textit{wan} (1\textsc{sg}) is the agent of the verb \textit{kak-arɨr-} (write-\textsc{cap}) ‘can write’ and takes \textit{n} (DAT1). In (\ref{ex:6-55}e), \textit{tuki} ‘time’ takes \textit{n} (DAT1).

  The dative 1 \textit{n} can follow the verbal infinitives. This combination expresses the time of the event.

\ea\label{ex:6-56}
{\TM}
\glll amanan  wuinkara,  naa  naikwa  kawatɨ,\\
\textit{a-ma=nan}  \textit{wur-i=n=kara}  \textit{naa}  \textit{naikwa}  \textit{kawar-tɨ}\\
\textsc{dist}-place=\textsc{loc1}  exist-\textsc{inf}=\textsc{dat}1=\textsc{abl}  already  a.little  strange-\textsc{seq}\\
\glt    ‘(The person) was already strange since (the person) was there, and ...’ [Co: 120415\_01.txt]
\z

In the above example, \textit{n} (\textsc{dat}1) follows the infinitve of the \textit{wur-} ‘exist’, i.e. /wui/ \textit{wur-i} (exist-\textsc{inf}), and is followed by \textit{kara} (\textsc{abl}) meaing ‘from the time ...’. Such a phenomenon, i.e. the combination of an infinitive plus \textit{n} (DAT1) meaing the time of the event, is said to be common in Ryukyuan languages (Prof. Shigehisa Karimata, 2013 p.c.). There are no examples in my texts where \textit{n} (DAT1) is followed by \textit{kara} (ABL) if the preceding word is a nominal, e.g. *\textit{tuki=n=kara} (time=DAT1=ABL). Thus, it seems that the \textit{n} following a nominal would be different from \textit{n} following a verb. However, I will regard them as the same morpheme \textit{n} (DAT1) because of the following reasons: (a) both kinds of \textit{n} behave in the same way on morphophonological alternation; (b) \textit{n} (DAT1) following a nominal can also mean the time of the event.

\ea\label{ex:6-57}
\ea Following a nominal\\
{\US}
\glll kˀuusjuunnja  wurantancjɨ?\\
 \textit{kˀuusjuu=n=ja}  \textit{wur-an-tar-n=ccjɨ}\\
      air.raid=\textsc{dat}1=\textsc{top}  exist-\textsc{neg}-\textsc{pst}-\textsc{ptcp}=\textsc{qt}\\
\glt ‘(Did you said) that (\textsc{my}) was not living here at the time of the air raid (in the World War II)?’ [Co: 110328\_00.txt]

\ex Following a verb\\
{\TM}
\glll  usato{\textbar}obasan{\textbar}ga  wuinnja  muru  jiccja atanmuncjɨjo.\\
\textit{usato+obasan=ga}  \textit{wur-i=n=ja}  \textit{muru}  \textit{jiccj-sa} \textit{ar-tar-n=mun=ccjɨ=joo}\\
Usato+aunt=\textsc{nom}  exist-\textsc{inf}=\textsc{dat}1=\textsc{top}  very  good-\textsc{adj}  \textsc{stv}-\textsc{pst}-\textsc{ptcp}=\textsc{advrs}=\textsc{qt}=\textsc{cfm}1\\
\glt ‘The time when Usato lived (here) was very good.’ [Co: 110328\_00.txt]
\z
\z

In (\ref{ex:6-57}a-b), both instances of \textit{ja} (\textsc{top}), which follow \textit{n} (\textsc{dat}1), become /nja/. Furthermore, in (\ref{ex:6-57}a), the nominal \textit{kˀuusjuu} ‘air raid’ followed by \textit{n} (DAT1) does not mean ‘air raid’ itself but means ‘the time of air raid,’ which is similar to the use of \textit{n} (DAT1) that follows the verb /wuin/ \textit{wur-i=n} (exist-\textsc{inf}=DAT1) meaning ‘the time when (someone) exists.’

\subsubsection{Dative case 2 \textit{nkatɨ}}

The dative case 2 \textit{nkatɨ} is used to mark the recipient of information.

\ea\label{ex:6-58}
 Recipient of information\\{}
[Context: \textsc{tm} advised her son about how to treat a certain aquaintance of them]

{\TM}
\glll wanna  mata  sɨgu  arɨnkatɨ  jˀicjancjɨjo.\\
\textit{wan=ja}  \textit{mata}  \textit{sɨgu}  \textit{a-rɨ=nkatɨ}  \textit{jˀ-tar-n=ccjɨ=joo}\\
    1\textsc{sg}=\textsc{top}  again  soon  \textsc{dist}-\textsc{nlz}=\textsc{dat}2  say-\textsc{pst}-\textsc{ptcp}=\textsc{qt}=\textsc{cfm}1\\
\glt    ‘I said (it) to that person [i.e. my son] without hesitation.’ [Co: 120415\_00.txt]
\z

In the above example, \textit{a-rɨ} (\textsc{dist}-\textsc{nlz}) ‘that person’ is the addressee of the verb \textit{jˀ-} ‘say’ and takes \textit{nkatɨ} (\textsc{dat}2). \textit{nkatɨ} (\textsc{dat2}) can co-occur with \textit{jˀ-} ‘say,’ \textit{hanas-} ‘talk,’ and \textit{jusɨr-} ‘teach.’ The origin of \textit{nkatɨ} (\textsc{dat2}) is not clear so far. Although we cannot say the correct candidate for its origin, we can say a wrong candidate. The initial phoneme /n/ of \textit{nkatɨ} (\textsc{dat2}) is not made of the contraction of the genitive particle \textit{nu} (see \REF{ex:6-81} in \sectref{sec:key:6.3.2.14} for the contraction of the genitive \textit{nu}), because the demonstrative nominal does not take the genitive particle \textit{nu} if it indicates human (see \tabref{tab:key:44} in \sectref{sec:key:6.4} and \REF{ex:6-106} in \sectref{sec:key:6.4.2.1}). In \REF{ex:6-58}, the demonstrative /arɨ/ \textit{a-rɨ} (\textsc{dist}-\textsc{nlz}) clearly indicates a human referent, so it cannot take \textit{nu} (\textsc{gen}). That is, the /n/ of \textit{nkatɨ} (\textsc{dat2}) is not made of \textit{nu} (\textsc{gen}), at least considering the modern synchronic data.

\subsubsection{Allative case \textit{kaci}}

The allative case \textit{kaci} is used to mark the goal of locomotion.

\ea\label{ex:6-59}
\ea Goal of locomotion (\textit{nagɨr-} ‘throw’)\\{}
[Context: A man got angry thinking that he had been cheated by the old couple.]

{\TM}
\glll janməəkaci  nagɨtɨ,  un  jingoo  hingitancjɨ.\\
      \textit{janməə=kaci}  \textit{nagɨr-tɨ}  \textit{u-n}  \textit{jinga=ja}  \textit{hingir-tar-n=ccjɨ}\\
      garden=\textsc{all}  throw-\textsc{seq}  \textsc{mes}-\textsc{adnz}  man=\textsc{top}  run.away-\textsc{pst}-\textsc{ptcp}=\textsc{qt}\\
\glt ‘(It was said) that the man threw (mud) in their garden and ran away.’ [Fo: 090307\_00.txt]

\ex Goal of deictic locomotion (\textit{ik-} ‘go’)\\{}
[Context: Looking at a picture, \textsc{tm} was guessing where the scene was.]

{\TM}
\glll in,  in.  jaakaci  ikjunturoo  zja.\\
      \textit{in}  \textit{in}  \textit{jaa=kaci}  \textit{ik-jur-n=turoo}  \textit{zjar}\\
      yes  yes  house=\textsc{all}  go-\textsc{umrk}-\textsc{ptcp}=place  \textsc{cop}\\
\glt ‘Oh, yeah. (It) is a scene of going to the house.’ [Co: 120415\_01.txt]
\z
\z

In (\ref{ex:6-59}a), \textit{janməə} ‘garden’ is the goal of the verb \textit{nagɨr-} ‘throw’ and takes \textit{kaci} (\textsc{all}). In (\ref{ex:6-59}b), \textit{jaa} ‘house’ is the goal of the verb \textit{ik-} ‘go’ and takes \textit{kaci} (ALL) too.

Additionally, \textit{kaci} (\textsc{all}) can be used to mark the result of change with \textit{nar-} ‘become.’ However, such an example is very rare. Among 44 examples, where the predicates are \textit{nar-} ‘become,’ there are only two such examples.

\ea\label{ex:6-60}
\ea\relax[Context: A bad man threw a pot filled with mud.]\\
{\TM}
\glll  un  janməəkaci  nagɨrattəətan  cɨboga mata  kundoo  kinkakaci  natɨ,\\
\textit{u-n}  \textit{janməə=kaci}  \textit{nagɨr-ar-təər-tar-n}  \textit{cɨbo=ga} \textit{mata}  \textit{kundu=ja}  \textit{kinka=kaci}  \textit{nar-tɨ}\\
\textsc{mes}-\textsc{adnz}  garden=\textsc{all}  throw-P\textsc{ass}-\textsc{rsl}-\textsc{pst}-\textsc{ptcp}  pot=\textsc{nom} again  this.time=\textsc{top}  gold.coin=\textsc{all}  become-\textsc{seq}\\
\glt ‘The pot thrown in the garden became (filled with) golds coins again this time.’ [Fo: 090307\_00.txt]

\ex\relax[Context: Speaking about a teacher who taught at the elementary school of \textsc{tm}’s childhood]\\
{\TM}
\glll  atoo  cjuugakkookaci  natɨ,\\
\textit{atu=ja}  \textit{cjuugakkoo=kaci}  \textit{nar-tɨ}\\
after=\textsc{top}  junior.high.school=\textsc{all}  become-\textsc{seq}\\
\glt ‘After (that), (he) became (a teacher at) a junior high school, and...’ [Co: 120415\_00.txt]

\ex
{\TM}
\glll tacumianjootuzituuga  nakawudo  natɨ,\\
\textit{tacumi+anjoo+tuzituu=ga}  \textit{nakawudo}  \textit{nar-tɨ}\\
Tatsumi+older.brother+couple=\textsc{nom}  matchmaker  become-\textsc{seq}\\
\glt ‘Mr. and Mrs. Tatsumi became matchmaker, and ...’ [Co: 120415\_00.txt]

\ex\relax[Context: Taking about a tradition]\\
{\TM}
\glll  jurunkjoojoo,  hajasa  nɨbuppoo,  kuuhuu    natɨ,  urɨ  sjuncjɨ  jˀicjɨ\\
\textit{juru=nkja=ja=joo}  \textit{haja-sa}  \textit{nɨbur-boo}  \textit{kuuhuu} \textit{nar-tɨ  u-rɨ  sɨr-jur-n=ccjɨ  jˀ-tɨ}\\
night=\textsc{appr}=\textsc{top}=\textsc{cfm}1  early-\textsc{adj}  sleep-\textsc{cnd}  owl   become-\textsc{seq}  \textsc{mes}-\textsc{nlz}  do-\textsc{umrk}-\textsc{ptcp}=\textsc{qt}  say-\textsc{seq}\\
\glt ‘(Old people) said that if you go to sleep early at night, (you) become an owl, and do it, and ...’ [Co: 111113\_02.txt]
\z
\z

Both \textit{kinka} ‘gold coin’ in (\ref{ex:6-60}a) and \textit{cjuugakkoo} ‘junior high school’ in (\ref{ex:6-60}b) are the goals of change indicated by \textit{nar-} ‘become’ and marked by \textit{kaci} (\textsc{all}); however, such a goal is normally not marked by any case particle as in (\ref{ex:6-60}c-d). So far, the difference between them is not so clear, but there is a good example in another language of Ryukyuan. In Irabu (Southern Ryukyuan), there are two case particles \textit{n} (\textsc{dat}1) and \textit{nkai} (ALL), both of which can be used with \textit{nar-} ‘become’, and the allative case is used when the speaker feels that there is a long distance between the source and the goal of change \citep{Shimoji2013}. Looking back to the examples of Yuwan in (\ref{ex:6-60}a-b), it is possible to assume a long distance between the source and goal of change. In (\ref{ex:6-60}a), the source ‘mud’ became the goal ‘gold coin,’ and in (\ref{ex:6-60}b), the source ‘(a teacher at the) elementary school’ became ‘(a teacher at the) junior high school.’ There is, however, an example which does not use \textit{kaci} (ALL) in spite of there being a long distance between the source and the goal, e.g. the source ‘a child’ and the goal ‘an owl’ in (\ref{ex:6-60}d). Therefore, it may be said in Yuwan that if \textit{kaci} (ALL) is used as the goal of change, the distance between the source and goal is relatively long, but not vice versa.

\subsubsection{Locative case 1 \textit{nan}/\textit{nən}}

The locative case 1 \textit{nan} (or \textit{nən}) is used to mark the place of contact; \textit{nən} is used only after the demonstrative adnominal (see \REF{ex:5-23} in \sectref{sec:key:5.2.1}). At least, \textit{nan} (\textsc{loc1}) needs two referents, i.e. a place and something (or someone) that makes contact with the place. \textit{nan} (\textsc{loc1}) follows an NP that indicates the place, and the subject of an intransitive clause, or the object of a transitive clause indicates a referent that makes contact with the place. First, let us see intransitive (or less transitive) clauses.

\ea\label{ex:6-61}
\ea
{\TM}
\glll un  sjanan  cɨbonu  atɨ,\\
\textit{u-n}  \textit{sja=nan}  \textit{cɨbo=nu}  \textit{ar-tɨ}\\
\textsc{mes}-\textsc{adnz}  below=\textsc{loc1}  pot=\textsc{nom}  exist-\textsc{seq}\\
\glt ‘There was a pot under there, and ...’ [Fo: 090307\_00.txt]

\ex\relax[Context: Talking about \textsc{my}]\\
= (\ref{ex:6-24}a)\\

{\TM}
\glll attaaja  (un)  un  hutəənan   wutancjɨjaa.\\
      \textit{a-rɨ-taa=ja}  \textit{u-n}  \textit{u-n}  \textit{hutəə=nan} \textit{wur-tar-n=ccjɨ=jaa}\\
      \textsc{dist}-\textsc{nlz}-\textsc{pl}=\textsc{top}  \textsc{mes}-\textsc{adnz}  \textsc{mes}-\textsc{adnz}  vicinity=\textsc{loc1}      exist-\textsc{pst}-\textsc{ptcp}=\textsc{qt}=\textsc{sol}\\
\glt ‘(I heard) that she and her family were around there.’ [Co: 110328\_00.txt]

\ex\relax[Context: A boy who put a basket full of pears in front of his bicycle bumped into a stone.]\\
{\TM}
\glll  isinan  atatɨ,\\
\textit{isi=nan}  \textit{atar-tɨ}\\
stone=\textsc{loc1}  bump-\textsc{seq}\\
\glt ‘(The boy) bumped into a stone, and ...’ [\textsc{pf}: 090225\_00.txt]
\z
\z

In (\ref{ex:6-61}a), \textit{un} \textit{sja} ‘the place under there,’ which takes \textit{nan} (\textsc{loc1}), is the place where the subject \textit{cɨbo} ‘pot’ exists. In (\ref{ex:6-61}b), \textit{un} \textit{hutəə} ‘around there [lit. that vicinity]’, which takes \textit{nan} (\textsc{loc1}), is the place where the subject /attaa/ \textit{a-rɨ-taa} (\textsc{dist}-\textsc{nlz}-\textsc{pl}) ‘she and her family’ stayed. In (\ref{ex:6-61}c), \textit{isi} ‘stone’, which takes \textit{nan} (\textsc{loc1}), is the place that the subject \textit{inja+warabɨ} ‘boy [lit. small child]’, though it was omitted in the above sentence, made contact with. The period for the subject to be in contact with the place of \textit{nan} (\textsc{loc1}) differs from a relatively long instance as in (\ref{ex:6-61}a-b) to a short instance as in (\ref{ex:6-61}c). Such a difference results from the meaning of each verb and the context where it is used. In my texts, the following intransitive verbs co-occured with \textit{nan} (\textsc{loc1}): \textit{ar-} ‘exist,’ \textit{tamar-} ‘accumulate,’ \textit{hamar-} ‘get stuck,’ \textit{wur-} ‘exist,’ \textit{umoor-} ‘exist (honorific),’ \textit{tat-} ‘stand,’ \textit{nɨhur-} ‘sleep,’ \textit{tumar-} ‘stay,’ \textit{cɨk-} ‘stick to,’ \textit{kaar-} ‘relate to,’ \textit{hənkj-} ‘enter,’ and \textit{atar-} ‘bump.’

Then, I will show the examples of transitive (especially three-participant) clauses.

\ea\label{ex:6-62}
\ea
{\TM}
\glll kɨɨnu  sjanannja  kagonu  tˀaacɨ  ucjutɨ,\\
\textit{kɨɨ=nu}  \textit{sja=nan=ja}  \textit{kago=nu}  \textit{tˀaacɨ}  \textit{uk-tur-tɨ}\\
tree=\textsc{gen}  below=\textsc{loc1}=\textsc{top}  basket=\textsc{gen}  two.\textsc{clf}.thing  put-\textsc{prog}-\textsc{seq}\\
\glt ‘Under the tree, (the old man) put two baskets, and ...’ [\textsc{pf}: 090222\_00.txt]

\ex\relax[Context: Describing how the village mayor answers the questions addressed to him by  members of the village assembly]\\
{\TM}
\glll  attaaga  jun  munnan  hɨntooja     sjuppa.\\
\textit{a-rɨ-taa=ga}  \textit{jˀ-jur-n}  \textit{mun=nan}  \textit{hɨntoo=ja} \textit{sɨr-jur-ba}\\
\textsc{dist}-\textsc{nlz}-\textsc{pl}=\textsc{nom}  say-\textsc{umrk}-\textsc{ptcp}  thing=\textsc{loc1}  reply=\textsc{top}   do-\textsc{umrk}-\textsc{csl}\\
\glt ‘(He) makes a reply (smoothly) to what they say, so ...’ [Co: 120415\_01.txt]
\z
\z

In (\ref{ex:6-62}a), \textit{kɨɨ=nu} \textit{sja} ‘the place under the tree,’ which takes \textit{nan} (\textsc{loc1}), is the place where the object \textit{kago=nu} \textit{tˀaacɨ} ‘two baskets’ exists. In (\ref{ex:6-62}b), /attaaga jun mun/ \textit{a-rɨ-taa=ga} \textit{jˀ-jur-n} \textit{mun} (\textsc{dist}-\textsc{nlz}-\textsc{pl}=\textsc{nom} say-\textsc{umrk}-\textsc{ptcp} thing) ‘what they say,’ which takes \textit{nan} (\textsc{loc1}), is the place that the object \textit{hɨntoo} ‘a reply’ makes contact with, although the meaning of ‘contact’ is very abstract here. At the beginning of this section, I said that in the transitive clause the place of \textit{nan} (\textsc{loc1}) is the one that the object (not the subject) makes contact with. However, among about twenty examples of transitive clauses that include \textit{nan} (\textsc{loc1}), there is only one example where it seems that the subject (but not the object) would be the referent contacting with the place of \textit{nan} (\textsc{loc1}).

\ea\label{ex:6-63}
 [Context: Seeing a picture where a harvest festival is held and people were wandering and dancing around the community, while men only wore the cotton belts called ‘mawashi’ in order to do sumo wrestling, and women walked and danced, having the meal for festival, between the men]\\

{\TM}
\gll wunagunintən  əədanan  kurɨ  muccjɨ,\\
\textit{wunagu+nintəə=n}  \textit{əəda=nan}  \textit{ku-rɨ}  \textit{mut-tɨ}\\
    woman+people=also  between=\textsc{loc1}  \textsc{prox}-\textsc{nlz}  have-\textsc{seq}\\
\glt    ‘Also the women had this [i.e. the meal for festival] between (the men), and ...’

{\MS}
\glll   {\textbar}hai,  hai,  hai,  hai.{\textbar}\\
    \textit{hai}  \textit{hai}  \textit{hai}  \textit{hai}\\
    yes  yes  yes  yes\\
\glt    ‘Oh, yeah.’ [Co: 111113\_01.txt]
\z

In the above example, the object \textit{ku-rɨ} ‘this [i.e. the meal for festival]’ is not the referent that made contact with the place \textit{əəda} ‘the space between (the men).’ Rather, the subject \textit{wunagu+nintəə} ‘women’ made contact with the place of \textit{nan} (\textsc{loc1}). Thus, it seems that this example would be a counterexample of the generalization at the beginning of this section. However, the above sentence uttered by \textsc{tm} was stopped with the converbal form /muccjɨ/ \textit{mut-tɨ} (have-\textsc{seq}), which means that there is a possibility that TM could continue the utterance with a certain verb that can take \textit{nan} (\textsc{loc1}), say \textit{wur-} ‘exist.’ In fact, TM’s utterance was interrupted by the nodding of \textsc{ms} (and TM did not continue the preceding sentence).

Before concluding this section, I want to remark the fact that \textit{nan} (\textsc{loc1}) can directly follow demonstrative adnominals, and then \textit{nan} (\textsc{loc1}) may alternate with \textit{nən}.

\ea\label{ex:6-64}
\ea Demonstrative adnominal + \textit{nan} (\textsc{loc1})\\{}
[Context: Explaining how to make the pickles of white radish]

{\TM}
\glll unnan  un  mama  {\textbar}bakecu{\textbar}nan  kan   sjɨ  tatɨtɨ  ukuboo,\\
      \textit{u-n=nan}  \textit{u-n}  \textit{mama}  \textit{bakecu=nan}  \textit{ka-n} \textit{sɨr-tɨ}  \textit{tatɨr-tɨ}  \textit{uk-boo}\\
      \textsc{mes}-\textsc{adnz}=\textsc{loc1}  \textsc{mes}-\textsc{adnz}  still  bucket=\textsc{loc1}  \textsc{prox}-\textsc{advz}      do-\textsc{seq}  stand-\textsc{seq}  put-\textsc{cnd}\\
\glt ‘If (you) stand (the white radishes with seasoning) there, in the bucket, as they are, ...’ [Co: 101023\_01.txt]

\ex Demonstrative adnominal + \textit{nən} (\textsc{loc1})\\
{\TM}
\glll  unnən  nasinu  natunwake.\\
\textit{u-n=nən}  \textit{nasi=nu}  \textit{nar-tur-n=wake}\\
\textsc{mes}-\textsc{adnz}=\textsc{loc1}  nasi=\textsc{nom}  bear-\textsc{prog}-\textsc{ptcp}=\textsc{cfp}\\
\glt ‘There are pears there [i.e. on the big tree].’ [\textsc{pf}: 090827\_02.txt]
\z
\z

In (\ref{ex:6-64}a), \textit{nan} (\textsc{loc1}) directly follows an adnominal \textit{u-n} ‘that (one)’ and they express a place as a whole. In (\ref{ex:6-64}b), \textit{nən} (\textsc{loc1}) also directly follows an adnominal \textit{u-n} ‘that (one).’ \textit{nan} (\textsc{loc1}) can follow both nominals and demonstrative adnominals. On the other hand, \textit{nən} (\textsc{loc1}) can follow only demonstrative adnominals.

\subsubsection{Locative case 2 \textit{nantɨ}/\textit{nəntɨ}}

The locative case 2 \textit{nantɨ} is used to mark the place of dynamic action. In (\ref{ex:6-65}a), /daibangɨɨ/ \textit{daiban+kɨɨ} ‘big tree,’ which takes \textit{nantɨ} (\textsc{loc2}), is the place where the action \textit{nasi} \textit{mur-} (pear pick.up) ‘to pick up pears’ occurs. In (\ref{ex:6-65}b), \textit{jaa} ‘house,’ which takes \textit{nantɨ} (\textsc{loc2}), is the place where the action \textit{nusi=sjɨ} \textit{hanməə} \textit{sɨr-} (\textsc{rfl}=\textsc{inst} cooking do) ‘to do cooking by oneself’ occurs.

\ea\label{ex:6-65}
\ea\relax[= \REF{ex:6-53}]\\
{\TM}
\glll  uziiga  daibangɨɨnantɨ  nasi  mutunwake.\\
\textit{uzii=ga}  \textit{daiban+kɨɨ=nantɨ}  \textit{nasi}  \textit{mur-tur-n=wake}\\
old.man=\textsc{nom}  big+tree=\textsc{loc2}  pear  pick.up-\textsc{prog}-\textsc{ptcp}=\textsc{cfp}\\
\glt ‘An old man is picking pears off on a big tree.’ [\textsc{pf}: 090305\_01.txt]

\ex
{\TM}
\glll uroo  jaanantɨ  nusisjɨ  hanməə  sjɨ,  kamii?\\
\textit{ura=ja}  \textit{jaa=nantɨ}  \textit{nusi=sjɨ}  \textit{hanməə}  \textit{sɨr-tɨ}  \textit{kam-i}\\
2.\textsc{nhon}.\textsc{sg}=\textsc{top}  house=\textsc{loc2}  \textsc{rfl}=\textsc{inst}  cooking  do-\textsc{seq}  eat-\textsc{inf}\\
\glt ‘You do cooking by yourself, and eat (the meal) at home?’ [Co: 120415\_01.txt]
\z
\z

This is a mere conjecture, but \textit{nantɨ} (\textsc{loc2}) can be thought to be made of /nan wutɨ/ \textit{nan} \textit{wur-tɨ} (\textsc{loc1} exist-\textsc{seq}) ‘to exist at (somewhere), and ...,’ since normally the environment where \textit{nantɨ} (\textsc{loc2}) can be used shows complementary distribution with that of \textit{nan} (\textsc{loc1}). For example, \textit{nantɨ} (\textsc{loc2}) cannot be used with \textit{wur-} ‘exist,’ but \textit{nan} (\textsc{loc1}) can (see also \sectref{sec:key:6.3.3.2}). Furthermore, \textit{nantɨ} (\textsc{loc2}), as well as \textit{nan} (\textsc{loc1}), can directly follow demonstrative adnominals with an optional alternation with \textit{nəntɨ} as in \REF{ex:6-66}. In (\ref{ex:6-66}a), \textit{nantɨ} (\textsc{loc2}) directly follows an adnominal \textit{u-n} ‘that (one)’ and they express a place as a whole. In (\ref{ex:6-66}b), \textit{nəntɨ} (\textsc{loc2}) also directly follows an adnominal \textit{u-n} ‘that (one)’ with its vowel centralization.

\ea\label{ex:6-66}
\ea Demonstrative adnominal + \textit{nantɨ} (\textsc{loc2})\\
{\TM}
\glll  kunugurugadɨ  (kun ..)  unnantɨ cukututanmundoojaa.\\
\textit{kunuguru=gadɨ}  \textit{ku-n}  \textit{u-n=nantɨ}   \textit{cukur-tur-tar-n=mun=doo=jaa}\\
recently=\textsc{lmt}  \textsc{prox}-\textsc{adnz}  \textsc{mes}-\textsc{adnz}=\textsc{loc2}   make-\textsc{prog}-\textsc{pst}-\textsc{ptcp}=\textsc{advrs}=\textsc{ass}=\textsc{sol}\\
\glt ‘(They) were making dyed goods until recently there.’ [Co: 111113\_01.txt]

\ex Demonstrative adnominal + \textit{nəntɨ} (\textsc{loc2})\\
{\TM}
\glll  daibangɨɨnu  atɨ,  unnəntɨ  jinganu  {\textbar}hasigo{\textbar}  kɨɨtɨ,\\
\textit{daiban+kɨɨ=nu}  \textit{ar-tɨ}  \textit{u-n=nəntɨ}  \textit{jinga=nu}  \textit{hasigo}  \textit{kɨɨr-tɨ}\\
big+tree=\textsc{nom}  exist-\textsc{seq}  \textsc{mes}-\textsc{adnz}=\textsc{loc2}  man=\textsc{nom}  ladder  put-\textsc{seq}\\
\glt ‘There was a big tree, and there a man put a ladder (against it), and ...’ [\textsc{pf}: 090222\_00.txt]
\z
\z

Thus, it is reasonable to think that the initial syllable /nan/ of \textit{nantɨ} (\textsc{loc2}) has the same origin with \textit{nan} (\textsc{loc1}).

\subsubsection{Locative case 3 \textit{zjɨ}}

The locative case 3 \textit{zjɨ} is used to mark the location of an action, which is distant from the speaker. It is probable that \textit{zjɨ} (\textsc{loc3}) was grammaticalized from the converb /izjɨ/ \textit{ik-tɨ} (go-\textsc{seq}) ‘to go, and ...’ (see \sectref{sec:key:6.3.4}). The head verb of \textit{zjɨ} (\textsc{loc3}) must have an animate subject (except for the metaphorical expression).

\ea\label{ex:6-67}
\ea
{\TM}
\glll usjəə  amanu ...  kusabutuuzjɨ  cɨnazjɨ   koojaccjɨ  jˀicjɨ,\\
\textit{usi=ja}  \textit{a-ma=nu}  \textit{kusabutuu=zjɨ}  \textit{cɨnag-tɨ}   \textit{k-oo=jaa=ccjɨ}  \textit{jˀ-tɨ}\\
ox=\textsc{top}  \textsc{dist}-place=\textsc{gen}  thick.grass=\textsc{loc3}  hitch-\textsc{seq}    come-\textsc{int}=\textsc{sol}=\textsc{qt}  say-\textsc{seq}\\
\glt ‘‘Let’s go to hitch the ox to the thick grass there’, said (the man), and ...’ [Fo: 090307\_00.txt]

\ex\relax[= (\ref{ex:4-54}b)]\\
{\TM}
\glll  sabiisabi  aikikippoo,  cɨkɨmununkja  jaazjɨ   tɨkkoorɨnmun.\\
\textit{sabi+sabi}  \textit{aik-i+kij-boo}  \textit{cɨkɨr+mun=nkja}  \textit{jaa=zjɨ}    \textit{tɨkk-oorɨ-n=mun}\\
\textsc{red}+smoothly  walk-\textsc{inf}+\textsc{cap}-\textsc{cnd}  pickle.INF+thing=\textsc{appr}  house=\textsc{loc3} bring-\textsc{cap}-\textsc{ptcp}=\textsc{advrs}\\
\glt ‘If (I) could walk smoothly, (I) could go home and bring the pickles, but (I cannot).’ [Co: 120415\_01.txt]
\z
\z

In (\ref{ex:6-67}a), \textit{a-ma=nu} \textit{kusabutuu} ‘thick grass there,’ which takes \textit{zjɨ} (\textsc{loc3}), is the goal where the subject goes and takes the action \textit{usi} (ox) + \textit{cɨnag-tɨ} \textit{k-} (hitch-\textsc{seq} come) ‘to go to hitch the ox.’ In this example, the subject is ‘the man,’ although it is not overtly expressed in the example. In (\ref{ex:6-67}b), \textit{jaa} ‘house,’ which takes \textit{zjɨ} (\textsc{loc3}), is the goal where the subject goes and takes the action \textit{cɨkɨr+mun=nkja} (pickle.\textsc{inf}+thing=\textsc{appr}) + \textit{tɨkk-} (bring) ‘to bring the pickles.’ In this example, the subject is ‘I’ [i.e. the speaker \textsc{tm}], although it is not overtly expressed in the example. In both of the examples, the places indicated by (NPs followed by) \textit{zjɨ} (\textsc{loc3}) are distant from the speaker, which is the main characteristic specific to \textit{zjɨ} (\textsc{loc3}) (see also \sectref{sec:key:6.3.4}).

\subsubsection{Instrumental case \textit{sjɨ}}

The instrumental \textit{sjɨ}, which is used to mark primarily an instrument, but in fact it can be used to mark a very broad meaning, e.g. material, reason, and membership of agent. First, let us see examples of instrumental \textit{sjɨ}.

\ea\label{ex:6-68}
 Instrument\\{}
[Context: Complaining about an acquaintance’s slander]

{\TM}
\glll wanga  kucisjɨ  nusiboo  jamacjuncjɨ,\\
\textit{wan=ga}  \textit{kuci=sjɨ}  \textit{nusi=ba=ja}  \textit{jam-as-tur-n=ccjɨ}\\
    1\textsc{sg}=\textsc{nom}  mouth=\textsc{inst}  \textsc{rfl}=\textsc{acc}=\textsc{top}  have.a.pain-\textsc{caus}-\textsc{prog}-\textsc{ptcp}=\textsc{qt}\\
\glt    ‘(The person\textit{\textsubscript{} }said) that I was making the person ill using (my) mouth, and ...’ [Co: 120415\_01.txt]
\z

In the above example, \textit{kuci} ‘mouth’ is the instrument used to criticize someone, and it takes \textit{sjɨ} (\textsc{inst}). The next examples are used to mean material, where the NP marked by \textit{sjɨ} (IN\textsc{st}) becomes a part of the result of action.

\ea\label{ex:6-69}
  Material

 \ea\relax[Context: Hearing that US spoke to the present author in the standard Japanese]\\
{\TM}
\glll  {\textbar}hoogen{\textbar}sjɨ  jˀanboo.\\
\textit{hoogen=sjɨ}  \textit{jˀ-an-boo}\\
dialect=\textsc{inst}  say-\textsc{neg}-\textsc{cnd}\\
\glt ‘(You) have to speak in the dialect [i.e. Yuwan].’ [Co: 110328\_00.txt]

\ex
{\TM}
\glll cˀjasuguu  kusasjɨ  mata  usatɨ\\
\textit{cˀjasuguu}  \textit{kusa=sjɨ}  \textit{mata}  \textit{usaw-tɨ}\\
soon  grass=\textsc{inst}  again  cover-\textsc{seq}\\
\glt ‘Soon (the man) covered (the pot filled with gold coins) with grass again.’ [Fo: 090307\_00.txt]
\z
\z

In (\ref{ex:6-69}a), \textit{hoogen} ‘dialect’ is the material to make an uttrance, and it takes \textit{sjɨ} (\textsc{inst}). In (\ref{ex:6-69}b), \textit{kusa} ‘grass’ is also the material to cover the pot, and it takes \textit{sjɨ} (IN\textsc{st}) too.

Next, let us look at examples of \textit{sjɨ} used to give a reason.

\ea\label{ex:6-70}
  Reason
 \ea\relax[Context: Talking about students who participate in the training camp held in the village]\\
{\TM}
\glll  hasijaankjanu  {\textbar}gassjuku{\textbar}sjɨ  kjuuroogai?\\
\textit{hasij-jaa=nkja=nu}  \textit{gassjuku=sjɨ}  \textit{k-jur-oo=ga=i}\\
run-person=\textsc{appr}=\textsc{nom}  training.camp=\textsc{inst}  come-\textsc{umrk}-\textsc{supp}=\textsc{cfm}3=\textsc{plq}\\
\glt ‘Runners would come for training camp, you know.’ [Co: 110328\_00.txt]

\ex\relax[Context: Remembering the days of the World War II]\\
{\TM}
\glll  kˀuusjuusjɨ  attakəə  jakɨtattujaa.\\
\textit{kˀuusjuu=sjɨ}  \textit{attakəə}  \textit{jakɨr-tar-tu=jaa}\\
air.raid=\textsc{inst}  everything  be.burnt-\textsc{pst}-\textsc{csl}=\textsc{sol}\\
\glt ‘Everything was burnt by the air raid, so (there are no houses from that time).’ [Co: 110328\_00.txt]
\z
\z

In (\ref{ex:6-70}a), \textit{gassjuku} ‘training camp’ is the reason that the runners come to the village, and it takes \textit{sjɨ} (\textsc{inst}). In (\ref{ex:6-70}b), \textit{kˀuusjuu} ‘air raid’ is also the reason that everything was burnt in the village, and it takes \textit{sjɨ} (IN\textsc{st}) as well.

Finally, I will show examples of an agent made up of multiple members, where the NP marked by \textit{sjɨ} (\textsc{inst}) expresses how many people or what kind of people composed of the membership of a collective agent.

\ea\label{ex:6-71}
  Membership of agent

 \ea\relax[Context: There are three boys who saw another boy bumping against a stone by bicycle, and the pears fell off the front basket; {\TM} ‘The three (happened to) pass the way, and standed the bicycle of the boy who bumped (there), and ...’]\\
{\TM}
\glll  micjaisjɨ  (ka)  kasjəə  sjɨ,  kagokaci   ɨrɨjunwake.\\
\textit{micjai=sjɨ}    \textit{kasjəə}  \textit{sɨr-tɨ}  \textit{kago=kaci}   \textit{ɨrɨr-jur-n=wake}\\
three.\textsc{clf}=\textsc{inst}    help  do-\textsc{seq}  basket=\textsc{all}  put.in-\textsc{umrk}-\textsc{ptcp}=\textsc{cfp}\\
\glt ‘The three (of them), helped (the boy), and put (the pears) in the basket.’ [\textsc{pf}: 090222\_00.txt]

\ex\relax[Context: Speaking to \textsc{ms}]\\
{\TM}
\glll  uroo  jaanantɨ  nusisjɨ  hanməə  sjɨ,  kamii?\\
\textit{ura=ja}  \textit{jaa=nantɨ}  \textit{nusi=sjɨ}  \textit{hanməə}  \textit{sɨr-tɨ}  \textit{kam-i}\\
2.\textsc{nhon}.\textsc{sg}=\textsc{top}  house=\textsc{loc2}  \textsc{rfl}=\textsc{inst}  meal  do-\textsc{seq}  eat-\textsc{inf}\\
\glt ‘You cook by yourself and eat (the meal) at home?’ [Co: 120415\_01.txt]

\ex
{\TM}
\glll burakusjɨ  sjən  {\textbar}suidoo{\textbar}  jatɨkai?\\
\textit{buraku=sjɨ}  \textit{sɨr-təər-n}  \textit{suidoo}  \textit{jar-tɨ=kai}\\
community=\textsc{inst}  do-\textsc{rsl}-\textsc{ptcp}  water.conduit  \textsc{cop}-\textsc{seq}=\textsc{dub}\\
\glt ‘(It) was the water conduit that has been set up by the community?’ [Co: 110328\_00.txt]
\z
\z

In (\ref{ex:6-71}a), \textit{micjai} ‘three people’ is the membership of agent who helped the boy, and it takes \textit{sjɨ} (\textsc{inst}). In (\ref{ex:6-71}b), \textit{nusi} (\textsc{rel}) ‘oneself’ is the membership of agent who makes the meal, and it takes \textit{sjɨ} (IN\textsc{st}). In (\ref{ex:6-71}c), \textit{buraku} ‘community’ is also the membership of agent who has set up the water conduit, and it takes \textit{sjɨ} (INST) too. These NPs marked by \textit{sjɨ} (INST) add some pieces of information about the membership of agents. In other words, there may be another NP that indicates the agent itself, e.g. \textit{ura} ‘you’ in (\ref{ex:6-71}b), which is the subject of the sentence. The form of the instrumental case, i.e. \textit{sjɨ}, is the same with a converbal form of \textit{sɨr-} ‘do’, i.e. \textit{sjɨ} (do.\textsc{seq}). It is probable that \textit{sjɨ} (INST) originates from /sjɨ/ \textit{sɨr-tɨ} (do-\textsc{seq}). However, the two forms are different from each other in modern Yuwan, since \textsc{ref}{ex:key:1} \textit{sjɨ} (INST) in the environments discussed above cannot take other inflection as the verb, e.g. one cannot say */nusi sjuttoo/ \textit{nusi} \textit{sɨr-jur=doo} (\textsc{rfl} do-\textsc{umrk}=\textsc{ass}) [Intended meaning] ‘(I) will do by myself’; \REF{ex:key:2} the NP before \textit{sjɨ} (INST) cannot take another case particle, e.g. one cannot say */nusinu sjɨ/ \textit{nusi=nu} \textit{sɨr-tɨ} (RFL=\textsc{nom} do-\textsc{seq}) instead of \textit{nusi=sjɨ} (RFL=INST) in (\ref{ex:6-71}b).

\subsubsection{ Ablative case \textit{kara}}

The ablative \textit{kara} is used to mark a source, which is a starting point of an action (or event) in space or time as in (\ref{ex:6-72}a-b). There are also examples of semantic extension of these as in (\ref{ex:6-72}c-d).

\ea\label{ex:6-72}
 Spatial source\\

 \ea\relax[Context: Talking about the staff of the village office, who went to help the people after the earthquake disaster on 11 \citealt{March2011}]\\
{\TM}
\glll  kumakara  kinju  jakubakara,  naa,  an   sɨmɨnu  mɨzɨnkja  nunkuin  cɨnkudɨ,\\
\textit{ku-ma=kara}  \textit{kinju}  \textit{jakuba=kara}  \textit{naa}  \textit{a-n}   \textit{sɨmɨ=nu}  \textit{mɨzɨ=nkja}  \textit{nu=n=kui=n}  \textit{cɨnkum-tɨ}\\
\textsc{prox}-place=\textsc{abl}  yesterday  village.office=ABL  \textsc{fil}  \textsc{dist}-\textsc{adnz} Sumiyo=\textsc{gen}  water=\textsc{appr}  what=any=\textsc{indf}=any  load-\textsc{seq}\\
\glt ‘From here, yesterday, from the village office, (they) loaded (a truck) with that water from Sumiyo and other things [lit. anything], and ...’ [Co: 110328\_00.txt]

  Temporal source

\ex
{\TM}
\glll waakjaa  anmataa  məəkacjəə  mukasikara kjuutattoo.\\
\textit{waakja-a}  \textit{anmaa-taa}  \textit{məə=kaci=ja}  \textit{mukasi=kara}  \textit{k-jur-tar=doo}\\
1\textsc{pl}-\textsc{adnz}  mother-\textsc{pl}  front=\textsc{all}=\textsc{top}  past=\textsc{abl}  come-\textsc{umrk}-\textsc{pst}=\textsc{ass}\\
\glt ‘From the past, (people who want to learn the traditional songs) would come to my mother’s place.’ [Co: 110328\_00.txt]

  Semantic extension

\ex
{\TM}
\glll arəə  attaa  məəra  muratən  jaa    jappa.\\
\textit{a-rɨ=ja}  \textit{a-rɨ-taa}  \textit{məə=kara}  \textit{muraw-təər-n}  \textit{jaa}    \textit{jar-ba}\\
\textsc{dist}-\textsc{nlz}=\textsc{top}  \textsc{dist}-\textsc{nlz}-\textsc{pl}  front=\textsc{abl}  receive-\textsc{rsl}-\textsc{ptcp}  house \textsc{cop}-\textsc{csl}\\
\glt ‘Since that is the house (he) has received from them.’ [Co: 111113\_01.txt]

\ex
{\TM}
\glll  urakjaa  (mm)  ziisan  məəradu narajutancjɨ.\\
\textit{urakja-a}    \textit{ziisan}  \textit{məə=kara=du}     \textit{naraw-jur-tar-n=ccjɨ}\\
2.\textsc{nhon}.\textsc{pl}-\textsc{adnz}    grandfather  front=\textsc{abl}=\textsc{foc}   learn-\textsc{umrk}-\textsc{pst}-\textsc{ptcp}=\textsc{qt}\\
\glt ‘(My mother said) that (she) learned (the traditional songs) from your grandfather.’ [Co: 111113\_01.txt]
\z
\z

In (\ref{ex:6-72}a), \textit{ku-ma} ‘here’ and \textit{jakuba} ‘the village office’ are spatial sources, from which the truck loaded with relief supplies would set off. In (\ref{ex:6-72}b), \textit{mukasi} ‘the past’ is a temporal source, from which the people started to come to see \textsc{tm}’s mother in order to learn the traditional songs. The next two examples are semantic extension from spatio-temporal uses. In (\ref{ex:6-72}c), /attaa məə/ \textit{a-rɨ-taa} \textit{məə} ‘them [lit. thier front]’ is the source from which the ownership of the house is transferred. In (\ref{ex:6-72}d), /urakjaa ziisan/ ‘your grandfather’ is the source from which the knowledge of the traditional songs is transmitted.

\subsubsection{ Comitative case \textit{tu}}

The comitative \textit{tu} is used to mark a participant of association. The participant of association is an added member of situation indicated by verbal predicate, nominal predicate, or adjective predicate. In (\ref{ex:6-73}a), \textit{nan} ‘you (honorific)’ is the participant associated with the speaker, and it takes \textit{tu} (\textsc{com}). In (\ref{ex:6-73}b), \textit{u-n=nintəə} ‘those people’ are the participants associated with \textit{muhaa+anjoo-taa} ‘Muha and his friends’ and takes \textit{tu} (\textsc{com}). Finally, in (\ref{ex:6-73}c), \textit{urakja-a} \textit{ziisan} ‘your grandfather’ is the participant associated with the speaker’s mother, and also takes \textit{tu} (\textsc{com}).

\ea\label{ex:6-73}
\ea With verbal predicate\\
{\TM}
\glll  injasainnja,  nantoo  asɨbantajaa.\\
\textit{inja-sa+ar-i-n=ja}  \textit{nan=tu=ja}  \textit{asɨb-an-tar=jaa}\\
small-\textsc{adj}+\textsc{stv}-\textsc{inf}-time=\textsc{top}  2.\textsc{hon}=\textsc{com}=\textsc{top}  play-\textsc{neg}-\textsc{pst}=\textsc{sol}\\
\glt ‘(I) did not play with you when (we) were young.’ [Co: 110328\_00.txt]

\ex With nominal predicate\\
{\TM}
\glll  muhaaanjootaa  unnintəətu  əəcɨrɨ  natɨ,     muru  dusi  jata.\\
\textit{muhaa+anjoo-taa}  \textit{u-n=nintəə=tu}  \textit{əəcɨrɨ}  \textit{nar-tɨ}   \textit{muru}  \textit{dusi}  \textit{jar-tar}\\
Muha+older.brother-\textsc{pl}  \textsc{mes}-\textsc{adnz}=people=\textsc{com}  classmate  \textsc{cop}-\textsc{seq}   very  friend  \textsc{cop}-\textsc{pst}\\
\glt ‘Muha and his friends were classmates with those people, and (they) were very friendly.’ [Co: 120415\_00.txt]

\ex With adjectival predicate\\{}
[Context: Talking of \textsc{tm}’s mother]

{\TM}
\glll urakjaa  ziisantu  nissja  ata.\\
      \textit{urakja-a}  \textit{ziisan=tu}  \textit{nissj-sa}  \textit{ar-tar}\\
      2.\textsc{nhon}.\textsc{pl}-\textsc{adnz}  grandfather=\textsc{com}  similar-\textsc{adj}  \textsc{stv}-\textsc{pst}\\
\glt ‘(My mother) was similar to your grandfather.’ [Co: 111113\_02.txt]
\z
\z

In the above examples, \textit{tu} (\textsc{com}) follows only one NP. On the other hand, \textit{tu} (\textsc{com}) can connect two (or more) NPs together, and there are twenty such examples in my texts. It can be said from the data of text that if the combined NPs are the subject (except for that of nominal predicate), only the first NP is followed by \textit{tu} (\textsc{com}), i.e. NP1\textit{=tu} NP2.

\ea\label{ex:6-74}
\ea Subject of an intransitive verb\\
{\TM}
\gllll  an  saeetu  ujuribəidu  kjun.\\
\textit{a-n}  \textit{saee=tu}  \textit{ujuri=bəi=du}  \textit{k-jur-n}\\
{}[\textsc{dist}-\textsc{adnz}  Sae=tu  Uyuri=only=\textsc{foc}]  [come-\textsc{umrk}-\textsc{ptcp}]\\
      {}[Subject]  [Intransitive verb]\\
\glt ‘Only Sae and Uyuri come (to the day-care center).’ [Co: 120415\_01.txt]

\ex Subject of a transitive verb\\{}
[Context: Remembering the days when \textsc{tm}’s son took her to sightseeing]

{\TM}
\gllll masajukitaatu  ataankjaga  xxx  nkja  sɨmɨtɨ,\\
      \textit{masajuki-taa=tu}  \textit{a-rɨ-taa=nkja=ga}    \textit{=nkja}     \textit{sɨmɨr-tɨ}\\
      {}[Masayuki-\textsc{pl}=\textsc{com}  \textsc{dist}-\textsc{nlz}-\textsc{pl}=\textsc{appr}=\textsc{nom}]    \textsc{appr}      [do.\textsc{caus}-\textsc{seq}]\\
      {}[Subject]          [Transitive verb]\\
\glt ‘Masayuki (and his family) and they had (me) do xxx, and ...’ [Co: 120415\_01.txt]
\z
\z

In (\ref{ex:6-74}a), \textit{a-n} \textit{saee} ‘(that) Sae,’ which is the first NP of the subject, takes \textit{tu} (\textsc{com}). In (\ref{ex:6-74}b), \textit{masajuki-taa} ‘Masayuki (and his family),’ which is the first NP of the subject, also takes \textit{tu} (\textsc{com}).

  However, if the combined NPs are the subject of a nominal predicate or the object of a transitive clause, not only the first NP but also the second NP is followed by \textit{tu} (\textsc{com}), i.e. NP1\textit{=tu} NP2\textit{=tu}.

\ea\label{ex:6-75}
 Subject of nominal predicates\\

 \ea
 {\TM}
\gllll hamaicɨuziitu  waakjaa  torataroouziitudu     kjoodəə  janmun.\\
\textit{hamaicɨ+uzii=tu}  \textit{waakja-a}  \textit{torataroo+uzii=tu=du}    \textit{kjoodəə}  \textit{jar-n=mun}\\
[Hamaitsu+grandfather=\textsc{com}  1\textsc{pl}-\textsc{adnz}  Torataro+grandfather=\textsc{com}=\textsc{foc}]     [brother  \textsc{cop}-\textsc{ptcp}=\textsc{advrs}]\\
      {}[Subject]      [Nominal predicate]\\
\glt ‘Hamaitsu and my grandfather Torataro are brothers.’ [Co: 111113\_01.txt]

\ex
{\TM}
\gllll kun  cˀjutu  kun  cˀjutuga      dɨkɨmun.jo.\\
\textit{ku-n}  \textit{cˀju=tu}  \textit{ku-n}  \textit{cˀju=tu=ga}    \textit{dɨkɨmun=joo}\\
[\textsc{prox}-\textsc{adnz}  person=\textsc{com}  \textsc{prox}-\textsc{adnz}  person=\textsc{com}]     [genius]=\textsc{cfm}1\\
      {}[Subject]      [Nominal predicate]\\
\glt ‘This person\textit{\textsubscript{i}} and this person\textit{\textsubscript{j}}are genius.’ [Co: 120415\_00.txt]

  Object of transitive verbs

\ex\relax[Context: Remembering that the present author asked \textsc{tm} to pronounce ‘head’ and ‘knee’ in Yuwan]\\
{\TM}
\gllll  cuburutu  cɨbusitu  jˀicjutɨga,  warəəcjɨjo.\\
\textit{cuburu=tu}  \textit{cɨbusi=tu}  \textit{jˀ-tur-tɨ=ga}  \textit{waraw-i=ccjɨ=joo}\\
{}[head=\textsc{com}  knee=\textsc{com}]  [say-\textsc{prog}-\textsc{seq}]=\textsc{foc}  laugh-\textsc{inf}=\textsc{qt}=\textsc{cfm}1\\
      {}[Object]  [Transitive verb]  \\
\glt ‘(We) were saying ‘head’ and ‘knee’ (in Yuwan), and laughed.’ [Co: 110328\_00.txt]

\ex
{\TM}
\gllll  ittannu  kinsjɨ  {\textbar}haori{\textbar}tu  kintu  nuuwarɨɨtattu.\\
\textit{ittan=nu}  \textit{kin=sjɨ}  \textit{haori=tu}  \textit{kin=tu}  \textit{nuuw-arɨɨr-tar-tu}\\
one.\textsc{clf}=\textsc{gen}  cloth=\textsc{inst}  [haori=\textsc{com}  cloth=\textsc{com}]  [sew-\textsc{cap}-\textsc{pst}-\textsc{csl}]\\
          {}[Object]  [Transitive verb]\\
\glt ‘From a roll of cloth (about ten meters in length), (we) could sew a haori [i.e. a short Japanese overgarment] and a (light cotton) kimono.’ [Co: 120415\_01.txt]
\z
\z

In (\ref{ex:6-75}a), each NP, i.e. /hamaicu+uzii/ ‘Hamaitsu’ and /waakjaa torataroouzii/ ‘my grandfather Torataroo’ being the subject of nominal predicate, is followed by \textit{tu} (\textsc{com}). Similarly, in (\ref{ex:6-75}b), each NP, i.e. /kun cˀju/ ‘this person\textit{\textsubscript{i}}’ and /kun cˀju/ ‘this person\textit{\textsubscript{j}}’ being the subject of nominal predicate, is followed by \textit{tu} (\textsc{com}). In (\ref{ex:6-75}c), each NP, i.e. \textit{cuburu} ‘head’ and \textit{cɨbusi} ‘knee’ being the object of transitive verb, is followed by \textit{tu} (\textsc{com}). Similarly, in (\ref{ex:6-75}d), each NP, i.e. \textit{haori} ‘haori’ and \textit{kin} ‘cloth’ being the object of transitive verb, is followed by \textit{tu} (\textsc{com}).

\subsubsection{ Limitative case \textit{gadɨ}}

The limitative \textit{gadɨ} is used to mark limits, which is a limitation of action (or event) in space and time, and there are examples of semantic extension of them.

\ea\label{ex:6-76}
\ea Spatial limits\\
  {}[Context: Talking about the size in the past of \textsc{tm}’s house]
{\TM}
\glll amagadɨ,  ude,  naanai  nagasa  atanmundoo.\\
\textit{a-ma=gadɨ}  \textit{ude}  \textit{naa+nai}  \textit{naga-sa}  \textit{ar-tar-n=mun=doo}\\
    \textsc{prox}-place  well  already+little  long-\textsc{adj}  \textsc{stv}-\textsc{pst}-\textsc{ptcp}=\textsc{advrs}=\textsc{ass}\\
\glt    ‘(It) was a little longer even to reach that place.’ [Co: 111113\_01.txt]

\ex Temporal limits\\
{\TM}
\glll namagadɨ  daanan  wutattukai?\\
\textit{nama=gadɨ}  \textit{daa=nan}  \textit{wur-tar-tu=kai}\\
    now=\textsc{lmt}  where=\textsc{loc1}  exist-\textsc{pst}-\textsc{csl}=\textsc{dub}\\
\glt    ‘Where was (he) until recently?’ [Co: 120415\_01.txt]

\ex Semantic extension\\
  {}[Context: Talking about a song that used to be sung when a meeting of old people was held]

{\TM}
\glll {\textbar}tagaini{\textbar}  naa  huccjunkjoo  minna    urəə  mjantɨn  sicjutattoojaa,    {\textbar}jonban{\textbar}gadɨ.\\
\textit{tagai=ni}  \textit{naa}  \textit{huccju=nkja=ja}  \textit{minna}  \textit{u-rɨ=ja}  \textit{mj-an-tɨ=n}  \textit{sij-tur-tar=doo=jaa}    \textit{jonban=gadɨ}\\
    each.other=\textsc{dat}  already  old.person=\textsc{appr}=\textsc{top}  everyone    \textsc{mes}-\textsc{nlz}=\textsc{top}  see-\textsc{neg}-\textsc{seq}=even  know-\textsc{prog}-\textsc{pst}=\textsc{ass}=\textsc{sol} fourth=\textsc{lmt}\\
\glt    ‘Each, all of the old people already knew (the song from the first verse) to the fourth, even if (they) did not see it [i.e. a card with the lyrics].’ [Co: 120415\_01.txt]
\z
\z

In (\ref{ex:6-76}a), \textit{a-ma} ‘that place’ is the spatial limit, which constraints the size of \textsc{tm}’s old house, and it takes \textit{gadɨ} (\textsc{lmt}). In (\ref{ex:6-76}b), \textit{nama} ‘now’ is the temporal limit, until which a man had been living there, and it also takes \textit{gadɨ} (L\textsc{mt}). In (\ref{ex:6-76}c), \textit{jonban} ‘fourth’ is the limit of the number of the song’s verses, which is an example of the semantic extension of the spatio-temporal meaning of \textit{gadɨ} (LMT).

\textit{gadɨ} (\textsc{lmt}) is not only a case particle, but also a limiter particle. \textit{gadɨ} (L\textsc{mt}) in the limiter-particle use can replace the nominative case. In addition, it may follow other case particles. The limiter particle \textit{gadɨ} (LMT) can express some emphasis, e.g. the speaker’s surprise (see \sectref{sec:key:10.1.5}). I will present an example here.

\ea\label{ex:6-77}
  \textit{gadɨ} (\textsc{lmt}) as a limiter particle\\
  {}[Context: Talking about the present author]

{\TM}
\glll tookjookaragadɨ  umoocjun  cˀjuboo  kattəə   warabɨnən  sjɨ  cɨkəədu  sjunmun,  wanna.\\
\textit{tookjoo=kara=gadɨ}  \textit{umoor-tur-n}  \textit{cˀju=ba=ja}  \textit{kattəə} \textit{warabɨ=nən}  \textit{sɨr-tɨ}  \textit{cɨkaw-i=du}  \textit{sɨr-jur-n=mun}  \textit{wan=ja}\\
    Tokyo=\textsc{abl}=\textsc{lmt}  move.\textsc{hon}-\textsc{prog}-\textsc{ptcp}  person=\textsc{acc}=\textsc{top}  freely child=like  do-\textsc{seq}  use-\textsc{inf}=\textsc{foc}  do-\textsc{umrk}-\textsc{ptcp}=\textsc{advrs}  1\textsc{sg}=\textsc{top}\\
\glt    ‘I ordered even a person who came from Tokyo [i.e. the present author] freely like a child.’ [Co: 110328\_00.txt]
\z

In the above example, \textit{gadɨ} (\textsc{lmt}) follows an extended NP \textit{tookjoo=kara} (Tokyo=\textsc{abl}) ‘from Tokyo.’ That is, \textit{gadɨ} (L\textsc{mt}) does not show the (spatial) limit of anything here, but expresses the speaker’s surprise about the present author’s coming from Tokyo.

\subsubsection{Comparative case \textit{jukkuma}}

The comparative \textit{jukkuma} is used to mark the standard of comparison. (The speaker \textsc{tm} also taught me another form \textit{junma} (\textsc{cmp}), but she has never used the form in the free conversation.) An NP followed by \textit{jukkuma} (CMP) can modify an adjective, an adverb, or a nominal.

\ea\label{ex:6-78}
  Modifying an adjective\\
 \ea\relax[Context: Talking about the size of a traditional coffin; \textsc{ms}: ‘(It) is as large as a box to fill in the tea.’]\\
{\TM}
\glll  aran.  urɨjukkumoo  hɨɨsai.\\
\textit{ar-an}  \textit{u-rɨ=jukkuma=ja}  [\textit{hɨɨ-sa}]\textsubscript{Adjective}\textit{+ar-i}\\
\textsc{cop}-\textsc{neg}  \textsc{mes}-\textsc{nlz}=\textsc{cmp}=\textsc{top}  big-\textsc{adj}+\textsc{stv}-\textsc{npst}\\
\glt ‘No. (The coffin) is bigger than that [i.e. a box to fill in the tea].’ [Co: 111113\_01.txt]
\z

  Modifying an adverb

\ex
{\TM}
\glll arɨjukkumoo  həəku  hɨɨranba.\\
\textit{a-rɨ=jukkuma=ja}  [\textit{həə-ku}]\textsubscript{Adverb}  \textit{hɨɨr-an-ba}\\
\textsc{dist}-\textsc{nlz}=\textsc{com}P=\textsc{top}  early-\textsc{advz}  wake.up-\textsc{neg}-\textsc{csl}\\
\glt ‘(You) have to wake up earlier than that person.’ [El: 130816]

  Modifying a nominal

\ex
{\TM}
\glll arəə  waakjajukkuma  sja  jappajaa.\\
\textit{a-rɨ=ja}  \textit{waakja=jukkuma}  [\textit{sja}]\textsubscript{Nominal}  \textit{jar-ba=jaa}\\
\textsc{dist}-\textsc{nlz}=\textsc{top}  1\textsc{pl}=\textsc{cmp}  below  \textsc{cop}-\textsc{csl}=\textsc{sol}\\
\glt ‘He is younger than me.’ [lit: ‘That person is below than me.’]      [Co: 110328\_00.txt]

\ex
{\TM}
\glll wan.jukkuma  sɨdoo  wurandoo.\\
\textit{wan=jukkuma}  [\textit{sɨda}]\textsubscript{Nominal}\textit{=ja}  \textit{wur-an=doo}\\
1\textsc{sg}=\textsc{cmp}  over=\textsc{top}  exist-\textsc{neg}=\textsc{ass}\\
\glt ‘There is no one (who) is older than me.’
[lit. ‘(The people whose ages are) over than me do not exist.’]      [El: 130816]
\z
\z

In (\ref{ex:6-78}a), \textit{u-rɨ} ‘it’ is the standard that is compared with the traditional coffin, modifying the adjective \textit{hɨɨ-sa} ‘big.’ In (\ref{ex:6-78}b), \textit{a-rɨ} ‘that person’ is the standard that is compared with the hearer, modifying the adverb \textit{həə-ku} ‘early.’ In (\ref{ex:6-78}c), \textit{waa-kja} ‘we’ is the standard that is compared with \textit{a-rɨ} ‘he,’ modifying the nominal \textit{sja} ‘below.’ In (\ref{ex:6-78}d), \textit{wan} ‘I’ is the standard that is compared with the people in the community, modifying the nominal \textit{sɨda} ‘over.’ In all examples in (\ref{ex:6-78}a-d), the standards take \textit{jukkuma} (\textsc{cmp}).

\subsubsection{ Genitive case \textit{ga}/\textit{nu}}

The genitive has two morphemes \textit{ga} and \textit{nu}, and they are chosen depending on the lexical meaning of their head nominals (see \sectref{sec:key:6.4}). Syntactically, the genitive case follows a head of an NP, which fills the modifier slot of another larger NP recursively, i.e \{[NP=\textsc{gen}]\textsubscript{Modifier} Head\}\textsubscript{NP} (see also \sectref{sec:key:6.1.1}). The meaning of genitive case (or the semantic relation between the modifier and the head) is very wide. Here, I will present its prototypical use (i.e. the possesion) and marginal use (i.e. the apposition).

\ea\label{ex:6-79}
\ea Possession\\
{\TM}
\glll  an  cˀjunu  naaja  sijan.\\
\textit{a-n}  \textit{cˀju=nu}  \textit{naa=ja}  \textit{sij-an}\\
\{[\textsc{dist}-\textsc{adnz}  person=\textsc{gen}]\textsubscript{Modifier}  [name]\textsubscript{Head}\}\textsubscript{NP}=\textsc{top}  know-\textsc{neg}\\
\glt ‘I don’t know that person’s name.’ [Co: 110328\_00.txt]

\ex Apposition\\
{\TM}
\glll  waakjaa  cɨrɨnkjanu  kikukotankja,    attankjaga  wun  ucibəi  jappoo,\\
\textit{waakja-a}  \textit{cɨrɨ=nkja=nu}  \textit{kikuko-taa=nkja}    \textit{a-rɨ-taa=nkja=ga}  \textit{wur-n}  \textit{uci=bəi}  \textit{jar-boo}\\
\{[1\textsc{pl}-\textsc{adnz}  classmate=\textsc{appr}=\textsc{gen}]\textsubscript{ Modifier}  [Kikuko-\textsc{pl}=\textsc{appr}]\textsubscript{Head}\}\textsubscript{NP}  \textsc{dist}-\textsc{nlz}-\textsc{pl}=\textsc{appr}=\textsc{nom}  exist-\textsc{ptcp}  inside=only  \textsc{cop}-\textsc{cnd}\\
\glt ‘If it is just while there are our friends, Kikuko and her friends, (and if it is just while there are) those people, ...’ [Co: 120415\_01.txt]
\z
\z

In (\ref{ex:6-79}a), \textit{a-n} \textit{cˀju} ‘that person’ is a possessor and is followed by \textit{nu} (\textsc{gen}), and it modifies the head nominal \textit{naa} ‘name,’ which is a possessee. In (\ref{ex:6-79}b), \textit{waakja-a} \textit{cɨrɨ=nkja} ‘our friends’ and \textit{kikuko-taa=nkja} ‘Kikuko and her friends’ are in apposition, i.e., they indicate the same referents.

The genitive has two morphemes, i.e. \textit{ga} and \textit{nu}, and they are formally same with those of the nominative case (see \sectref{sec:key:6.3.2.1}). Thus, one may regard them as the same single case, i.e. “the nominative-genitive case.” I would not, however, regard them as the same case because of \textsc{ref}{ex:key:1} the differences of syntactic distribution and \REF{ex:key:2} the differences of correspondence to the animacy hierarchy.

First, an NP followed by the nominative case fills the argument slot of a clause, and its head is the predicate phrase as in (\ref{ex:6-80}a-b) (see \sectref{sec:key:4.1.1}). On the other hand, an NP followed by the genitive case fills the modifier slot of an NP, and its head is a nominal as in (\ref{ex:6-80}c-d) (see \sectref{sec:key:6.1}).

\ea\label{ex:6-80}
 Filling the argument slot of a clause\\

 \ea
 {\TM}
\glll  arɨga..,  sizuobaaga  wuppoo,  jiccja   atənmundoo.\\
\textit{a-rɨ=ga}  \textit{sizu+obaa=ga}  \textit{wur-boo}  \textit{jiccj-sa}  \textit{ar-təər-n=mun=doo}\\
\textsc{dist}-\textsc{nlz}=\textsc{nom}  Shizu+grandmother=\textsc{nom}  exist-\textsc{cnd}  good-\textsc{adj} \textsc{stv}-\textsc{rsl}-\textsc{ptcp}=\textsc{advrs}=\textsc{ass}\\
\glt ‘If Shizu were here, (it) would be good (now).’ [Co: 120415\_01.txt]

\ex
{\TM}
\gllll umoo  kan  sjɨ  kɨɨnu  atɨ,\\
\textit{u-ma=ja}  \textit{ka-n}  \textit{sɨr-tɨ}  \textit{kɨɨ=nu}  \textit{ar-tɨ}\\
\textsc{mes}-place=\textsc{top}  \textsc{prox}-\textsc{advz}  do-\textsc{seq}  tree=\textsc{nom}  exist-\textsc{seq}\\
            Argument  Predicate\\
\glt ‘There is a tree like this, and ...’ [\textsc{pf}: 120415\_01.txt]

  Filliing the modifier slot of an NP

\ex
{\TM}
\glll agga  ututunan  masuoccjɨ  jˀicjɨ,      wutɨ,\\
\textit{a-rɨ=ga}  \textit{ututu=nan}  \textit{masuo=ccjɨ}  \textit{jˀ-tɨ}      \textit{wur-tɨ}\\
\textsc{dist}-\textsc{nlz} =\textsc{gen}  younger.sibling=\textsc{loc1}  Masuo=\textsc{qt}  say-\textsc{seq}   exist-\textsc{seq}\\
\glt ‘That person has a younger sibling called Masuo, and ...’
[lit. ‘In that person’s younger sibling is (a person) called Masuo, and ...’]      [Co: 120415\_00.txt]

\ex\relax[= (\ref{ex:6-62}a)]\\
{\TM}
\gllll  kɨɨnu  sjanannja  kagonu  tˀaacɨ  ucjutɨ,\\
\textit{kɨɨ=nu}  \textit{sja=nan=ja}  \textit{kago=nu}  \textit{tˀaacɨ}  \textit{uk-tur-tɨ}\\
tree=\textsc{gen}  under=\textsc{loc1}=\textsc{top}  basket=\textsc{gen}  two.\textsc{clf}  put-\textsc{prog}-\textsc{seq}\\
      Modifier  Head      \\
\glt ‘Under the tree, (tha man) put two baskets, and ...’ [\textsc{pf}: 090222\_00.txt]
\z
\z

In the first two examples, both \textit{a-rɨ} (\textsc{dist}-\textsc{nlz}) ‘that person’ in (\ref{ex:6-80}a) and \textit{kɨɨ} ‘tree’ in (\ref{ex:6-80}b) fill the argument slots of the clauses. More specifically, they are subjects of the clasues. In the next two examples, however, the same NPs do not fill the arguments but fill the modifier slots of NPs. In (\ref{ex:6-80}c), \textit{a-rɨ} (\textsc{dist}-\textsc{nlz}) ‘that person’ modifies the head nominal \textit{ututu} ‘younger sibling’ (about the contranction from \textit{a-rɨ=ga} > /agga/, see \sectref{sec:key:5.2.1}). In (\ref{ex:6-80}d), \textit{kɨɨ} ‘tree’ modifies the head nominal \textit{sja} ‘(th place) under (something)’. It is true that each case particle in (6-80 a, c), i.e. /ga/, and those in (6-80 b, d), i.e. /nu/, have the same form respectively. However, I will propose that they should be regarded as different case particles.

Secondly, the choice of \textit{ga} and \textit{nu} depends on the lexical meaning of the head nominals. However, the lexical group that takes the nominative case particle \textit{ga} (\textsc{nom}) is different from that of the genitive case particle \textit{ga} (\textsc{gen}) as in \tabref{tab:key:41} (see \tabref{tab:key:44} in \sectref{sec:key:6.4} for more details).


\begin{table}
\caption{\label{tab:key:41} Differences between the nominative and the genitive (following singular NPs)}
\begin{tabular}{lllll}
\lsptoprule
                       & Personal pronominals   & Human demonstratives  &  Address nouns  &  The others\\
\midrule
Nominative case         & \textit{ga}           &  \textit{ga}          &  \textit{ga}      &  \textit{nu}\\
NP modifiers            & Adnominal             & \textit{ga}           &  Juxtaposition  & \textit{nu}\\
\lspbottomrule
\end{tabular}
\end{table}

The above table shows that personal pronominals, human demonstratives, and address nouns take the nominative case particle \textit{ga}, and the other nominals take \textit{nu}. On the other hand, the genitive case \textit{ga} is taken only by human demonstratives, because personal pronominals inflect as adnominals when they fill the modifier slot of an NP like [\textit{waakja-a}]\textsubscript{Modifier} [\textit{anmaa}]\textsubscript{Head} (1\textsc{pl}-\textsc{adnz} mother) ‘our mother,’ and also address nouns do not take any case (in other words, use juxtaposition) when they fill the modifier slot of an NP like [\textit{naohide+uzii}]\textsubscript{Modifier} [\textit{ututu}]\textsubscript{Head} (Naohide+grandfather younger.sibling) ‘Naohide’s younger sibling’ (see \sectref{sec:key:7.2} in detail). In fact, there is no difference when the two cases follow common nouns, e.g. \textit{kɨɨ} ‘tree’ as in (6-80 b, d). Considering the distributional difference shown in \tabref{tab:key:41}, I will propose that they should be regarded as different cases. This point of view owes to the idea of “distributional cases” in \citet{Comrie1991}.

  The genitive particle \textit{nu} often contracts to /n/ when the external head of the genitive NP, i.e. “NP\textsubscript{2}” in “NP\textsubscript{1}=\textsc{gen} NP\textsubscript{2},” indicates space.

\ea\label{ex:6-81}
 Head nominal (modified by the genitve NP) is \textit{sja} ‘under’\\

 \ea\relax[Context: Talking about the shore protection at the community]\\
{\TM}
\glll  jakuban  sjanu,  (ee)  namanu  {\textbar}sinrjoosjo{\textbar}nu    sjantɨ,\\
\textit{jakuba=nu}  \textit{sja=nu}    \textit{nama=nu}  \textit{sinrjoosjo=nu}    \textit{sja=nantɨ}\\
village.office=\textsc{gen}  under=\textsc{gen}    now=\textsc{gen}  clinic=\textsc{gen}    under=\textsc{loc2}\\
\glt ‘Down from the village office [lit. at (the place) under the village office] (that existed before), down from the clinic (that exists) now (at the same place), ...’ [Co: 111113\_02.txt]

\ex
{\TM}
\glll micin  sjanan.\\
\textit{mici=nu}  \textit{sja=nan}\\
road=\textsc{gen}  under=\textsc{loc1}\\
\glt ‘(The post office exists) down along the road [lit. at (the place) under the road].’ [Co: 120415\_00.txt]

  Head nominal (modified by the genitve NP) is \textit{nɨzɨɨ} ‘corner’

\ex
{\TM}
\glll jaman  nɨzɨɨ  natɨ.\\
\textit{jama=nu}  \textit{nɨzɨɨ}  \textit{nar-tɨ}\\
mountain=\textsc{gen}  corner  \textsc{cop}-\textsc{seq}\\
\glt ‘Since (our house) was (at) the foot of the mountain.’ [Co: 111113\_02.txt]

  Head nominal (modified by the genitve NP) is \textit{məə} ‘front’

\ex
{\TM}
\glll  un  kɨn  məəkaci  mudutɨ  kii.\\
\textit{u-n}  \textit{kɨɨ=nu}  \textit{məə=kaci}  \textit{mudur-tɨ}  \textit{k-i}\\
\textsc{mes}-\textsc{adnz}  tree=\textsc{gen}  front=\textsc{all}  return-\textsc{seq}  come-\textsc{inf}\\
\glt ‘(The boys) were back to the front of the tree.’ [\textsc{pf}: 090305\_01.txt]

\ex
{\TM}
\glll urakjaa  uman  məənu  an..   {\textbar}obasan{\textbar}ga  {\textbar}iciban{\textbar}jo.\\
\textit{urakja-a}  \textit{u-ma=nu}  \textit{məə=nu}  \textit{a-n}   \textit{obasan=ga}  \textit{iciban=joo}\\
2.\textsc{nhon}.\textsc{pl}-\textsc{adnz}  \textsc{mes}-place=\textsc{gen}  front=\textsc{gen}  \textsc{dist}-\textsc{adnz}  old.woman=\textsc{nom}  number.one=\textsc{cfm}1\\
\glt ‘That old woman who lived in front of your place [lit. of the front of your that place] is number one.’ [Co: 120415\_01.txt]

  Head nominal (modified by the genitve NP) is \textit{buci} ‘edge’

\ex
{\TM}
\glll kon  buci?\\
\textit{koo=nu}  \textit{buci}\\
      river=\textsc{gen}  edge\\
\glt ‘Near the river?’ [lit. ‘(At) the edge of the river?’]      [Co: 110328\_00.txt]

\ex  [Context: Speaking about \textsc{tm}’s mother; TM: ‘Until (she) learn (how to tap a rhythm of the traditional songs), ...’]
{\TM}
\glll zijun  buci  uccjutɨ,\\
      \textit{ziju=nu}  \textit{buci}  \textit{ut-tur-tɨ}\\
      kitchen.stove=\textsc{gen}  edge  hit-\textsc{prog}-\textsc{seq}\\
\glt ‘(My mother) was hitting the edge of the kitchen stove, and ...’
\z
\z

The contraction shown in (\ref{ex:6-81}a-g) does not occur in the case of a nominative case particle \textit{nu} (\textsc{nom}), which partly supports the appropriateness of distinguishing the genitive case particle from the nominative case particle in Yuwan.

  Finally, the genitive case may follow another case particle, which was already shown in (\ref{ex:6-5}a-e) in \sectref{sec:key:6.1.1.}

\subsection{Comparison among similar case particles}

In the following subsections, I will compare some case particles that have similar functions. In \sectref{sec:key:6.3.3.1}, dative 1, dative 2, and allative will be discussed. In \sectref{sec:key:6.3.3.2}, the locative 1, 2, and 3 will be disscussed.

\subsubsection{Dative 1, dative 2, and allative}

All of the cases \textit{n} (\textsc{dat}1), \textit{nkatɨ} (\textsc{dat2}), and \textit{kaci} (\textsc{all}) may co-occur with verbs that have a meaning related with direction. The details of their differences are not very clear, but there are restrictions on their co-occurence with their head verbs depending on the meanings of the verbs. The possibility of their co-occurence with several verbs (or verbal affixes) is shown in the following table and examples. In \tabref{tab:key:42}, “+” means that the case particle can co-occur with the verbs (or verbal affixes), and “-” means cannot.

\begin{table}
\caption{\label{tab:key:42} \textit{n} (\textsc{dat}1), \textit{kaci} (\textsc{all}), and \textit{nkatɨ} (\textsc{dat2})}
\begin{tabular}{llllllll}
\textit{-arɨr} (P\textsc{ass}) & \textit{-as} (\textsc{caus}) & \textit{kurɨr-} ‘give’  &\textit{jˀ-} ‘say’  &\textit{nagɨr-} ‘throw’ & \textit{ik-} ‘go’\\
\textit{n} &     (\textsc{dat}1) & + & + & + & + & - & -\\
\textit{kaci} &  (\textsc{all})  & - & + & + & + & + & +\\
\textit{nkatɨ} & (\textsc{dat}2) & - & - & - & + & - & -\\
\end{tabular}
\end{table}

In \REF{ex:6-82}, “*” means that the form is not grammatical in the environments.

\ea\label{ex:6-82}
\ea Co-occurence with \textit{-arɨr} (P\textsc{ass}) to mark the agent\\
{\TM}
\glll wanna  zjun/*zjuukaci/*zjunkatɨ  oosattɨdoo\\
\textit{wan=ja}  \textit{zjuu=n}/\textit{zjuu=kaci}/\textit{zjuu=nkatɨ}  \textit{oos-ar-tɨ=doo}\\
    1\textsc{sg}=\textsc{top}  father=\textsc{dat}1/father=\textsc{all}/father=\textsc{dat2}  scold-P\textsc{ass}-\textsc{seq}=\textsc{ass}\\
\glt    ‘I was scolded by (my) father.’ [El: 130820]

\ex Co-occurence with \textit{-as} (\textsc{caus}) to mark the causee\\

{\TM}
\glll arɨn/arɨkaci/*arɨnkatɨ  kakasoojəə.\\
\textit{a-rɨ=n}/\textit{a-rɨ=kaci}/\textit{a-rɨ=nkatɨ}  \textit{kak-as-oo=jəə}\\
    \textsc{dist}-\textsc{nlz}=\textsc{dat}1/\textsc{dist}-\textsc{nlz}=\textsc{all}/DIST-\textsc{nlz}=\textsc{dat2}  write-\textsc{caus}-\textsc{int}=\textsc{cfm}2\\
\glt    ‘(I) will make that person write (it).’ [El: 130820]

\ex Co-occurence with \textit{kurɨr-} ‘give’ to mark the recepient\\

{\TM}
\glll arɨn/arɨkaci/*arɨnkatɨ  kurɨroojəə.\\
\textit{a-rɨ=n}/\textit{a-rɨ=kaci}/\textit{a-rɨ=nkatɨ}  \textit{kurɨr-oo=jəə}\\
    \textsc{dist}-\textsc{nlz}=\textsc{dat}1/\textsc{dist}-\textsc{nlz}=\textsc{all}/DIST-\textsc{nlz}=\textsc{dat2}  give-\textsc{int}=\textsc{cfm}2\\
\glt    ‘(I) will give (it) to that person.’ [El: 130820]

\ex Co-occurence with \textit{jˀ-} ‘say’ to mark the recepient of the information\\

{\TM}
\glll uroo  tarun/tarukaci/tarunkatɨ  jˀicjɨ?\\
\textit{ura=ja}  \textit{ta-ru=n}/\textit{ta-ru=kaci}/\textit{ta-ru=nkatɨ}  \textit{jˀ-tɨ}\\
    2.\textsc{nhon}.\textsc{sg}=\textsc{top}  who-\textsc{nlz}=\textsc{dat}1/who-\textsc{nlz}=\textsc{all}/who-\textsc{nlz}=\textsc{dat2}  say-\textsc{seq}\\
\glt    ‘To whom did you talk to?’ [El: 130820]

\ex Co-occurence with \textit{nagɨr-} ‘throw’ to mark the goal\\

{\TM}
\glll *dan/daakaci/*dankatɨ  nagɨtɨ?\\
\textit{daa=n}/\textit{daa=kaci}/\textit{daa=nkatɨ}  \textit{nagɨr-tɨ}\\
     where=\textsc{dat}1/where=\textsc{all}/where=\textsc{dat2}  throw-\textsc{seq}\\
 \glt    ‘Where did (you) throw (it)?’ [El: 130820]

\ex Co-occurence with \textit{ik-} ‘go’ to mark the goal\\

{\TM}
\glll uroo  *dan/daaci/*dankatɨ  ikjui?\\
\textit{ura=ja}  \textit{daa=n}/\textit{daa=kaci}/\textit{daa=nkatɨ}  \textit{ik-jur-i}\\
    2.\textsc{nhon}.\textsc{sg}=\textsc{top}  where=\textsc{dat}1/where=\textsc{all}/where=\textsc{dat2}  go-\textsc{umrk}-\textsc{npst}\\
 \glt    ‘Where do (you) go?’ [El: 130820]
\z
\z

As far as the verbs (and the verbal affixes) in \tabref{tab:key:42} are concerned, we can say the following things. First, \textit{n} (\textsc{dat}1) can co-occur with several verbs or verbal affixes with the exception of \textit{nagɨr-} ‘throw’ and \textit{ik-} ‘go.’ Thus, \textit{n} (DAT1) seems not to be used to mark the goal in a narrow sense. In other words, the “goal” marked by \textit{n} (DAT1) is the recepient or causee. Secondly, \textit{kaci} (\textsc{all}) can co-occur with almost all of the verbs or verbal affixes with the exception of \textit{-arɨr} (P\textsc{ass}). In fact, \textit{-arɨr} (\textsc{pass}) has little meaning strongly related with direction. Thus, it may be possible to say that \textit{kaci} (ALL) can be used with verbs that have a meaning related with direction. Finally, \textit{nkatɨ} (\textsc{dat2}) can be used only with \textit{jˀ-} ‘say.’ As mentioned in \sectref{sec:key:6.3.2.4}, \textit{nkatɨ} (\textsc{dat2}) can be used only to mark the recepient of the information.

\subsubsection{Locative 1, locative 2, and locative 3}

All of the cases \textit{nan} (\textsc{loc1}), \textit{nantɨ} (\textsc{loc2}), and \textit{zjɨ} (\textsc{loc3}) can express the place where the action (or event) (indicated by the head verb) occurs. The details of their differences are not very clear, but there are restrictions on co-occurence with verbs or the context where they are used. The possibility of co-occurence with a few verbs and a nominal is shown in the following table and examples. In \tabref{tab:key:43}, “+” means that the case particle can co-occur with the verbs (or the nominals), and “-” means cannot.

\begin{table}
\caption{\label{tab:key:43} \textit{nan} (\textsc{loc1}),
                            \textit{nantɨ} (\textsc{loc2}), and
                            \textit{zjɨ} (\textsc{loc3})
                            }

\begin{tabular}{lllll}
Co-occurence with  && Verbs &        Nominal\\
\textit{wur-} ‘exist (animate)’  &\textit{ar-} ‘exist (inanimate)’ & \textit{udur-} ‘dance’    &\textit{ku-ma} ‘here’\\
\textit{nan}  (\textsc{loc1})    &+ & + &  -  &  + \\
\textit{nantɨ}  (\textsc{loc2})  &- & - &  +  &  + \\
\textit{zjɨ}  (\textsc{loc3})    &+ & - &  +  &  - \\
\end{tabular}
\end{table}

In \REF{ex:6-83}, “*” means that the form is not grammatical in the environment.

\ea\label{ex:6-83}
 \ea Co-occurence with \textit{wur-} ‘exist (animate)’\\

{\TM}
\glll wanna  amanan/*amanantɨ/amazjɨ  wuroojəə.\\
\textit{wan=ja}  \textit{a-ma=nan}/\textit{a-ma=nantɨ}/\textit{a-ma=zjɨ}  \textit{wur-oo=jəə}\\
    1\textsc{sg}=\textsc{top}  \textsc{dist}-place=\textsc{loc1}/\textsc{dist}-place=\textsc{loc2}/DIST-place=\textsc{loc3}  exist-\textsc{int}=\textsc{cfm}2\\
\glt    ‘I will be there.’ [El: 130817]
\ex Co-occurence with \textit{ar-} ‘exist (inanimate)’\\

{\TM}
\glll tɨganna  amanandu/*amanantɨdu/*amazjɨdu    attoo.  \\
\textit{tɨgan=ja}  \textit{a-ma=nan=du}/\textit{a-ma=nantɨ=du}/\textit{a-ma=zjɨ=du}   \textit{ar=doo}  \\
    letter=\textsc{top}  \textsc{dist}-place=\textsc{loc1}=\textsc{foc}/\textsc{dist}-place=\textsc{loc2}=\textsc{foc}/DIST-place=\textsc{loc3}=\textsc{foc}   exist=\textsc{ass}\\
\glt    ‘The letter is there.’ [El: 130817]

\ex Co-occurence with \textit{udur-} ‘dance’\\
{\TM}
\glll *amanan/amanantɨ/amazjɨ  wuduroojəə.\\
\textit{a-ma=nan}/\textit{a-ma=nantɨ}/\textit{a-ma=zjɨ}  \textit{wudur-oo=jəə}\\
    \textsc{dist}-place=\textsc{loc1}/\textsc{dist}-place=\textsc{loc2}/DIST-place=\textsc{loc3}  dance-\textsc{int}=\textsc{cfm}2\\
 \glt    ‘(I) will dance there.’ [El: 130817]
\z
\z

If the clause is used to mean that the subject of the intransitive verb (or the object of the transitive verb) stays (or contacts) somewhere, \textit{nantɨ} (\textsc{loc2}) cannot be used, but \textit{nan} (\textsc{loc1}) and \textit{zjɨ} (\textsc{loc3}) can as in (\ref{ex:6-83}a) (see also \sectref{sec:key:6.3.2.6}). Because of the same reason, \textit{ar-} ‘exist’ can be used with \textit{nan} (\textsc{loc1}), but cannot be used with \textit{nantɨ} (\textsc{loc2}) as in (\ref{ex:6-83}b). Additionally, \textit{ar-} ‘exist’ must have an inanimate subject (strictly speaking, an inanimate “core argument,” see \sectref{sec:key:8.3.2.2} for more details). On the contrary, \textit{zjɨ} (\textsc{loc3}) always has an animate subject (see \sectref{sec:key:6.3.2.8}). Therefore, \textit{zjɨ} (\textsc{loc3}) cannot be used with \textit{ar-} ‘exist’ as in (\ref{ex:6-83}b). If the head verb expresses a dynamic action, the place of action cannot be marked by \textit{nan} (\textsc{loc1}), but can be marked by \textit{nantɨ} (\textsc{loc2}) and \textit{zjɨ} (\textsc{loc3}) as in (\ref{ex:6-83}c).

  Furthermore, \textit{zjɨ} (\textsc{loc3}) has a restriction; it cannot follow an NP that indicates a place where the speaker exists at the time of utterance (see \sectref{sec:key:6.3.4} for more details). Thus, \textit{zjɨ} (\textsc{loc3}) cannot follow \textit{ku-ma} (\textsc{prox}-place) ‘here.’

\ea\label{ex:6-84}
 Co-occurence with \textit{ku-ma} ‘here’\\

 \ea \textit{nan} (\textsc{loc1})\\
{\TM}
\glll  wanna  kumanan  wuroojəə.\\
\textit{wan=ja}  \textit{ku-ma=nan}  \textit{wur-oo=jəə}\\
1\textsc{sg}=\textsc{top}  \textsc{prox}-place=\textsc{loc1}  exist-\textsc{int}=\textsc{cfm}2\\
\glt ‘I will be here.’ [El: 130817]

\ex \textit{nantɨ} (\textsc{loc2})\\
{\TM}
\glll  wanna  kumanantɨ  wuduroojəə\\
\textit{wan=ja}  \textit{ku-ma=nantɨ}  \textit{wudur-oo=jəə}\\
1\textsc{sg}=\textsc{top}  \textsc{prox}-place=\textsc{loc2}  dance-\textsc{int}=\textsc{cfm}2\\
\glt ‘I will dance here.’ [El: 130817]

\ex \textit{zjɨ} (\textsc{loc3})\\
{\TM}
\glll  *wanna  kumazjɨ  wuroojəə.\\
\textit{wan=ja}  \textit{ku-ma=zjɨ}  \textit{wur-oo=jəə}\\
1\textsc{sg}=\textsc{top}  \textsc{prox}-place=\textsc{loc3}  exist-\textsc{int}=\textsc{cfm}2 \\
{}[El: 130817]
\z
\z

\textit{nan} (\textsc{loc1}) and \textit{nantɨ} (\textsc{loc2}) can be used with \textit{ku-ma} ‘here’ as in (\ref{ex:6-84}a-b), but \textit{zjɨ} (\textsc{loc3}) cannot as in (\ref{ex:6-84}c), which made a clear contrast with (\ref{ex:6-83}a), where a similar expression, i.e. \textit{wan=ja} \textit{a-ma=zjɨ} \textit{wur-oo=jəə} (1\textsc{sg}=\textsc{top} \textsc{dist}-place=\textsc{loc3} exist-\textsc{int}=\textsc{cfm}2) ‘I will be there’ is grammatical.

\subsection{Grammaticalization of case particles}

In Ryukyuan languages, some case particles are said to have been created through grammaticalization of a certain verbal form (\citealt{NishiokaNakahara2000}: 87, \citealt{Shimoji2008}: 207). Yuwan also has a few case particles which seem to have come from grammaticalization. For example, it is possible that the instrumental case \textit{sjɨ} has come from /sjɨ/ \textit{sɨr-tɨ} (do-\textsc{seq}) (see \sectref{sec:key:6.3.2.9}). The locative case 2 \textit{nantɨ} may have come from the combination of \textit{nan} (\textsc{loc1}) plus /wutɨ/ \textit{wur-tɨ} (exist-\textsc{seq}) (see \sectref{sec:key:6.3.2.7}). Additionally, the locative case 3 \textit{zjɨ} seems to have come from /izjɨ/ \textit{ik-tɨ} (go-\textsc{seq}). All of these case particles include, as their putative origin, the same converbal affix, i.e. \textit{-tɨ} (\textsc{seq}), which makes an adverbial clause that precedes the main clause (see also \sectref{sec:key:11.1.1}). Thus, it is reasonable that such a clause becomes an argument of the predicate of the main clause considering the verb-final word order in Yuwan. In the remainder of this section, we will look at \textit{zjɨ} (\textsc{loc3}) in detail.

There are two reasons why we can say that \textit{zjɨ} (\textsc{loc3}) and /izjɨ/ (go.\textsc{seq}) have the same origin; (a) resemblance between the two forms; (b) the same restriction on the reference point, or the “deictic center” (cf. \citealt{Fillmore1971} [1997]). With regard to (a), there is no problem since \textit{zjɨ} (\textsc{loc3}) and /izjɨ/ \textit{ik-tɨ} (go-\textsc{seq}) has the same form excluding the existence of the initial vowel /i/. With respect to (b), neither form allows their goals to be the place where the speaker exists at the time of utterance. Briefly speaking, neither can be used with \textit{ku-ma} (\textsc{prox}-place) ‘here.’ First, let us see the examples that have no problem because of the correct context.

\ea\label{ex:6-85}
  [Context: The speaker has not arrived at the goal yet.]

 \ea /izjɨ/ (go.\textsc{seq})\\
{\TM}
\glll  ama  izjɨ,  asɨboojaa.\\
\textit{ama}  \textit{ik-tɨ}  \textit{asɨb-oo=jaa}\\
there  go-\textsc{seq}  play-\textsc{int}=\textsc{sol}\\
\glt ‘Let’s go there, and play (together)!’ [El: 130816]

\ex /zjɨ/ (\textsc{loc3})\\
{\TM}
\glll  amazjɨ  asɨboojaa.\\
\textit{ama=zjɨ}  \textit{asɨb-oo=jaa}\\
there=\textsc{loc3}  play-\textsc{int}=\textsc{sol}\\
\glt ‘Let’s go and play there (together)!’ [El: 130816]
\z
\z

As mentioned in \sectref{sec:key:6.3.2.2}, the deictic locomotion verb \textit{ik-} ‘go’ can take accusative case \textit{ba} to mark its goal, and also can easily omit such \textit{ba} (\textsc{acc}) as in (\ref{ex:6-85}a). Both of the above examples are grammatical, but similar sentences cannot be acceptable as in \REF{ex:6-86}. The sentence-initial “\textsuperscript{\#}” means that the context is not acceptable to produce the sentence.

\ea\label{ex:6-86}
  [Context: The speaker has already arrived at the goal.]

 \ea /izjɨ/ (go.\textsc{seq})\\
{\TM}
\glll  \textsuperscript{\#}kuma  izjɨ,  asɨboojaa.\\
\textit{kuma}  \textit{ik-tɨ}  \textit{asɨb-oo=jaa}\\
here  go-\textsc{seq}  play-\textsc{int}=\textsc{sol}\\
{}[Expressed meaning] ‘Let’s go here, and play (together)!’ [El: 130816]

\ex /zjɨ/ (\textsc{loc3})\\
{\TM}
\glll  \textsuperscript{\#}kumazjɨ  asɨboojaa.\\
\textit{kuma=zjɨ}  \textit{asɨb-oo=jaa}\\
here=\textsc{loc3}  play-\textsc{int}=\textsc{sol}\\
{}    [Expressed meaning] ‘Let’s go and play here (together)!’ [El: 130816]
\z
\z

In (\ref{ex:6-85}a-b), the spearker has not arrived yet at the goal. Thus, both /izjɨ/ (go.\textsc{seq}) and /zjɨ/ (\textsc{loc3}) are grammatical. However, in (\ref{ex:6-86}a-b), the speaker has already arrived at the goal, so both /izjɨ/ (go.\textsc{seq}) and /zjɨ/ (\textsc{loc3}) become unacceptable. In other words, /izjɨ/ (go.\textsc{seq}) and /zjɨ/ (\textsc{loc3}) cannot take the place where the speaker exists at the time of utterance as their deictic center.

  I would not, however, like to regard the two forms are absolutely indentical. Rather, it is more appropriate to regard that there has been a grammaticalization from /izjɨ/ \textit{ik-tɨ} (go-\textsc{seq}) to \textit{zjɨ} (\textsc{loc3}), since the latter has (c) the loss of initial vowel, (d) the impossibility of insertion of another case particle, and (e) the capability to take directly a human referent as the goal of (deictic) locomotion. With regard to (c), /zjɨ/ (\textsc{loc3}) seems to have dropped the initial vowel /i/ of /izjɨ/ \textit{ik-tɨ} (go-\textsc{seq}). With regard to (d), \textit{ik-} ‘go’ can take the accusative case to mark the goal of deictic locomotion as in (\ref{ex:6-87}a). On the contrary, /zjɨ/ (\textsc{loc3}) cannot take (or be preceded by) it as in (\ref{ex:6-87}b).

\ea\label{ex:6-87}
  Capability of the accusative’s insertion
 \ea /izjɨ/ (go.\textsc{seq})\\
{\TM}
\glll  wanna  unba  izjɨ,  asɨdɨ  koojəə.\\
\textit{wan=ja}  \textit{un=ba}  \textit{ik-tɨ}  \textit{asɨb-tɨ}  \textit{k-oo=jəə}\\
1\textsc{sg}=\textsc{top}  sea=\textsc{acc}  go-\textsc{seq}  play-\textsc{seq}  come-\textsc{int}=\textsc{cfm}2\\
\glt ‘(I) will go (to) the sea, and play (there) and come (back).’ [El: 130817]

\ex /zjɨ/ (\textsc{loc3})\\
{\TM}
\glll  *wanna  unbazjɨ  asɨdɨ  koojəə.\\
\textit{wan=ja}  \textit{un=ba=zjɨ}  \textit{asɨb-tɨ}  \textit{k-oo=jəə}\\
1\textsc{sg}=\textsc{top}  sea=\textsc{acc}=\textsc{loc3}  play-\textsc{seq}  come-\textsc{int}=\textsc{cfm}2\\
{}      [Intended meaning] ‘(I) will go (to) the sea, and play (there) and come (back).’ [El: 130817]
\z
\z

With regard to (e), \textit{zjɨ} (\textsc{loc3}) can directly take a human referent as the goal, although \textit{ik-} ‘go’ cannot.

\ea\label{ex:6-88}
  Capability of directly taking a human referent as the goal

 \ea /izjɨ/ (go.\textsc{seq})\\
{\TM}
\glll  *akira  izjɨ,  abɨtɨ  koo!\\
\textit{akira}  \textit{ik-tɨ}  \textit{abɨr-tɨ}  \textit{k-oo}\\
Akira  go-\textsc{seq}  call-\textsc{seq}  \textsc{exp}-\textsc{imp}\\
{}       [Intended meaning] ‘Go to Akira’s place and call him and come (back)!’ [El: 130817]

\ex /zjɨ/ (\textsc{loc3})\\
{\TM}
\glll  akirazjɨ  abɨtɨ  koo!\\
\textit{akira=zjɨ}  \textit{abɨr-tɨ}  \textit{k-oo}\\
Akira=\textsc{loc3}  call-\textsc{seq}  \textsc{exp}-\textsc{imp}\\
 \glt ‘Go to Akira’s place and call him and come (back)!’ [El: 130817]
\z
\z

The above three differences show almost all of the features of grammaticalization discussed in Heine and \citet[2]{Kuteva2002} as follows.

\ea\label{ex:6-89}
 Four features of grammaticalization in Heine and \citet[2]{Kuteva2002}
A.  desemanticization (or ‘semantic bleaching’) - loss in meaning content;\\
B.  extension (or context generalization) - use in new contexts;\\
C.    decategorialization - loss in morphosyntactic properties characteristic of lexical or other less gramaticalized forms;\\
D.  erosion (or ‘phonetic reduction’) - loss in phonetic substance.
\z

In the context of the above features, (6-89 B) corresponds to the above (e), i.e. the capability to take directly a human referent as the goal of (deictic) locomotion; (6-89 C) corresponds to the above (d), i.e. the impossibility of insertion of another case particle; and (6-89 D) corresponds to the above (c), i.e. the loss of initial vowel. Although Heine and \citet[3]{Kuteva2002} assume the (6-89 A) prcedes others (with a possible exception of (6-89 C)), the semantic bleaching (or loss in meaning content) does not seem to occur in the case of \textit{zjɨ} (\textsc{loc3}) in Yuwan since the restriction of goal of locomotion of \textit{ik-} ‘go’ still applies to \textit{zjɨ} (\textsc{loc3}). A particle made of the grammaticalization of a verb meaning ‘go’ is found in the another language of Ryukyuans. In \citet[207]{Shimoji2008}, there is a clitic /nkii/, which is said to be made of \textit{n} \textit{ik-i-i} (\textsc{dat} go-\textsc{ep}-\textsc{seq}), and it expresses ‘going to’ (glosses in Irabu are changed in order to correspond to those in Yuwan by the present author, and “EP” means an epenthetic vowel).

  In addition, there is a particle that also has the form /zjɨ/, but it can follow a verbal predicate.

\ea\label{ex:6-90}
  [Context: The speaker will go to somewhere.]

{\TM}
\glll wanun  səəba  numoozjɨjəə.\\
\textit{wan=n}  \textit{səə=ba}  \textit{num-oo=zjɨ=jəə}\\
    1\textsc{sg}=also  alcohol=\textsc{acc}  drink-\textsc{int}=\textsc{dirc}=\textsc{cfm}2\\
\glt    ‘I will also go to drink alcohol.’ [El: 130817]
\z

The above sentence, however, becomes unacceptable if the context is different.

\ea\label{ex:6-91}
  [Context: The speaker will not go to anywhere, but drinks at the place where she is.]

{\TM}
\glll \textsuperscript{\#}wanun  səəba  numoozjɨjəə.\\
\textit{wan=n}  \textit{səə=ba}  \textit{num-oo=zjɨ=jəə}\\
    1\textsc{sg}=also  alcohol=\textsc{acc}  drink-\textsc{int}=\textsc{dirc}=\textsc{cfm}2\\
    {}[Expressed meaning] ‘I will go to drink alcohol.’ [El: 130817]
    \z

The above example shows that if the speaker will not be apart from the place where she exists at the time of utterance, the particle \textit{zjɨ}, which is glossed “\textsc{dirc}” here meaning “directional,” cannot be used. The restriction is the same with that of the case particle \textit{zjɨ} (\textsc{loc3}) (and \textit{ik-} ‘go’). Thus, it is probable that both of \textit{zjɨ} (\textsc{loc3}) and \textit{zjɨ} (D\textsc{ir}C) have the same origin. They are, however, cannot be regarded as the same morpheme in the present Yuwan since their syntactic circumstances are different from each other. That is, \textit{zjɨ} (DIRC) follows a verb in the predicate slot, but \textit{zjɨ} (\textsc{loc3}) follows an NP in an argument slot.

\section{Animacy hierarchy}

Yuwan has several phenomena which are concerned with the animacy hierarchy in linguistic typology (about the animacy hierarchy, see \citealt{Silverstein1976}, \citealt{Comrie1989}, \citealt{Dixon1994}, \citealt{Whaley1997}, \citealt{Corbett2000}, and \citealt{Croft2003} [1990] among many others). For example, only personal pronouns have dual forms in Yuwan (see \sectref{sec:key:5.1}). Additionally, there are four other phenomena that are correlated with the animacy hierarchy: the choice of plural markers, the choice of tactics used in the modifier slot of an NP, the choice of the nominative case forms, and the choice of the existential verbs. See the following table (\tabref{tab:key:44}), where “address nouns” include mainly elder kinship terms and personal names, both of which can be used to address the hearer (see \sectref{sec:key:7.2}). “Human demonstratives” in the following table mean that the demonstrative nominals are used to indicate human referents (see \sectref{sec:key:5.2}). The rightmost column (“the other nominals”) also includes non-human demonstratives (i.e. the demonstrative nominals used to indicate non-human referents).


\begin{table}
\caption{\label{tab:key:44}Animacy hierarchy in Yuwan}
\begin{tabular}{lllll}
\lsptoprule
  Personal pronominals  & Human interrogatives  & Human demonstratives  & Address nouns  & The other nominals\\
  \midrule
  1st/2nd  3rd        Animate  Inanimate\\
  \lspbottomrule
  \todo[inline]{fix tabular}
\end{tabular}

  \textbf{Number}

Singular markers\footnote{If a word ends with \textit{-ru} (\textsc{nlz}) or \textit{-rɨ} (\textsc{nlz}), it expresses the singularity, at least in natural discourse.}  \textit{-n} / \textit{-Ø}  N/A  \textit{-ru  -rɨ} N/A  N/A

Dual marker     \textit{-ttəə} N/A  N/A  N/A  N/A

Plural markers\footnote{This alignment depends on the text data. In the elicitation data, human demonstratives may take \textit{nkja} (\textsc{appr}), and non-human demonstratives may take \textit{-taa} (\textsc{pl}) (see §\ref{bkm:Ref361694047} for more details).}  \textit{-kja} N/A  \textit{-taa  -taa  -taa  nkja}

\textbf{NP} \textbf{modifiers}          

Singular  Adnominal  N/A  Adnominal  \textit{ga}  Juxtaposition  \textit{nu}

Dual     \textit{ga}  N/A  N/A  N/A  N/A

Plural  Adnominal  N/A  Juxtaposition  Juxtaposition  Juxtaposition  \textit{nu}

\textbf{Case} \textbf{particles}          

S/A     \textit{ga}  N/A\footnote{If the subject of a clause is an interrogative word, it does not take the nominative case particle, but takes the focus particle \textit{ga} (which is different from the nominative \textit{ga}). See §\ref{bkm:Ref367267321} and \sectref{sec:key:10.1} for more details.}  \textit{ga}  \textit{ga}  \textit{nu}

P     \textit{ba}  (Not found)  \textit{ba}  \textit{ba}  \textit{ba}  \textit{ba} / \textit{Ø}

\textbf{Existential} \textbf{verbs}      \textit{wur-}  \textit{wur-}  \textit{wur-}  \textit{wur-}  \textit{wur-}  \textit{ar-} / \textit{nə-}
\end{table}

Generally, human interrogatives, e.g. \textit{ta-ru} (who-\textsc{nlz}) ‘who’ in Yuwan, does not come up for discussion of animacy hierarchy (at least in the papers introduced above). The data of Yuwan shows that the distribution of human interrogatives is partly similar to personal pronominals with regard to the singular form as an NP modifier, e.g. /ta-a/ (who-\textsc{adnz}) ‘whose’ and /ura-a/ (2.\textsc{nhon}.\textsc{sg}-\textsc{adnz}) ‘your.’ It is also partly similar to human demonstratives and address nouns with regard to the plural marker (and the plural form as an NP modifier), e.g. /ta-t-taa/ (who-\textsc{nlz}-\textsc{pl}) ‘who (plural)’ and /a-t-taa/ (\textsc{dist}-\textsc{nlz}-\textsc{pl}) ‘those people.’ A possible reason why the human interrogative behaves in the same way with the personal pronominals is as follows. Human interrogatives and personal pronominals are literally “pronominal,” and also they obligatorily indicate human referents. On the other hand, the demonstrative nominals (and also the reflexive pronouns to be discussed in \sectref{sec:key:7.3}) may indicate non-human referents (see \sectref{sec:key:5.2}). Thus, the pronominal characteristic and the obligatoriness of indicating human referents may differentiate the personal pronominals and the human interrogatives from the others.

In the following subsections, we will see the details of the plural markers (see \sectref{sec:key:6.4.1}), the NP modifiers (see \sectref{sec:key:6.4.2}), and the nominative case (see \sectref{sec:key:6.4.3}). The accusative case was already discussed in \sectref{sec:key:6.3.2.2.} About existential verbs, see \sectref{sec:key:8.3.2.}

\subsection{Plural (or approximative) markers}
\subsubsection{Semantics of plural (or approximative) markers}

Yuwan has three morphemes that can express a kind of plural meaning: \textit{-kja}, \textit{-taa}, and \textit{nkja}. These morphemes can be used to indicate more than one referent, which is a function of both of the ordinary plural and the “associative plural” in other languages (cf. \citealt{Corbett2000}: 101-111). However, the “plural” markers in Yuwan can be used in another situation. They can indicate a virtually single referent. I will present the relevant examples of \textit{-kja}, \textit{-taa}, and \textit{nkja} in turn below.

First, \textit{-kja} (\textsc{pl}) can indicate not only plural specific referents, but also a single specific referent as in (\ref{ex:6-92}a-b). It can be translated into ‘a person like me.’

\ea\label{ex:6-92}
  \textit{-kja} (\textsc{pl})

 \ea\relax[Context: Speaking to \textsc{ms} about the tuna fishing in old days]\\
{\TM}
\glll  wanna  sijan.  waakjoo  sijandoo.\\
\textit{waa-n=ja}  \textit{sij-an}  \textit{waa-kja=ja}  \textit{sij-an=doo}\\
1-\textsc{sg}=\textsc{top}  know-\textsc{neg}  1-\textsc{pl}=\textsc{top}  know-\textsc{neg}=\textsc{ass}\\
\glt ‘I don’t know. I don’t know (the detail of the tuna fishing).’ [Co: 120415\_01.txt]

\ex\relax[Context: US told \textsc{tm} and \textsc{my} that TM knew everything, but TM said she knew nothing herself, but that her mother had known everything important.]\\
= \REF{ex:5-8}\\

{\TM}
\glll waakjan  sijanmun.\\
      \textit{waa-kja=n}  \textit{sij-an=mun}\\
      1\textsc{pl}=also  know-\textsc{neg}=\textsc{advrs}\\
\glt ‘I don’t know anything either.’ (or ‘A person like me doesn’t know anything either.’)      [Co: 110328\_00.txt]
\z
\z

In (\ref{ex:6-92}a), \textsc{tm} and \textsc{ms} were talking alone about the tuna fishing in old days, and TM said she did not know about it in detail. Here, the \textit{waa-kja} (1-\textsc{pl}) in this example indicates the speaker herself alone as an instance of people who are not familiar with the tuna fishing. The semantic “non-plurality” of the referent can be implied by the singular pronoun /wan/ \textit{waa-n} (1-\textsc{sg}), which precedes and is paraphrased by the following \textit{waa-kja} (1-\textsc{pl}). In (\ref{ex:6-92}b), there are only four participants in the scene, and TM told US that she (i.e. TM) did not know anything showing her modesty. In this case, the expression \textit{waa-kja} (1-pL) did not indicate a referent other than TM (see also the discussion about \REF{ex:5-8} in \sectref{sec:key:5.1.1}). In order to specify the ability to indicate a single referent using the form \textit{waa-kja} (1-\textsc{pl}), I did an elicitation as in \REF{ex:6-93}, where the singularity of the agent is stressed by the extended NP \textit{cˀjui=sjɨ} (one.person.\textsc{clf}=\textsc{inst}) ‘alone.’ Both of \textit{-kja} (\textsc{pl}) and \textit{cˀjui=sjɨ} ‘alone’ are underlined below.

\ea\label{ex:6-93}
  [Context: There are only two people, and one talks to the other.]

{\TM}
\glll urəə  mucɨkasjanu,  waakjoo  cˀjuisjəə    siikijandoo.\\
\textit{u-rɨ=ja}  \textit{mucɨkasj-sa=nu}  \textit{waa-kja=ja}  \textit{cˀjui=sjɨ=ja}    \textit{sɨr-i+kij-an=doo}\\
    \textsc{mes}-\textsc{nlz}=\textsc{top}  difficult-\textsc{adj}=\textsc{csl}  1-\textsc{pl}=\textsc{top}  one.person.\textsc{clf}=\textsc{inst}=\textsc{top} do-\textsc{inf}+\textsc{cap}-\textsc{neg}=\textsc{ass}\\
\glt    ‘That is difficult, so I cannot do (it) alone.’ [El: 130820]
\z

In \REF{ex:6-93}, the speaker uses \textit{waa-kja} (1-\textsc{pl}) in order to pick up herself as an instance who cannot do the difficult thing.

  These uses of \textit{-kja} (\textsc{pl}) are very frequent in Yuwan. One may remember the so-called “associative plural” (or “group plural”) in other languages (cf. \citealt{Corbett2000}: 101-111). However, there is a crucial difference between the functoin of the “plural” in Yuwan and that of the associative plural in other languages. On the one hand, the common usage of the associative plural markers in other languages is to indicate a specific group. In other words, wherether or not there are a number of unspecific referents in the group, the group itself must be specific. For example, if you are a pupil of an elementary school and school lunches are provided, you can say something like: \textit{We} \textit{don’t} \textit{need} \textit{to} \textit{bring} \textit{lunch} \textit{by} \textit{ourselves}. Here, the plural form \textit{we} indicates a specific referent (i.e. the speaker), and the remaining referents may be specific or unspecific. Anyway, the group indicated by \textit{we}, i.e. the pupils of the school as a whole, must be specific. On the other hand, the plural markers of Yuwan can indicate a certain group that is \textit{not} specific in itself. For example, \textit{waa-kja} (1-\textsc{pl}) in (\ref{ex:6-92}a) does not indicate any specific group. If we dare to identify the group in the context, it might be a group where the members are not familiar with the tuna fishing in those days. In the case of (\ref{ex:6-92}b), it seems more difficult (or impossible) to identify such a group indicated by \textit{waa-kja} (1-\textsc{pl}). The “group” mentioned here is very different from that of \textit{we} in English in terms of specificity. In fact, the unspecificity of the group indicated by \textit{-kja} (\textsc{pl}) is not the sufficient condition to distinguish it from the plural forms in other languages. For example, the “\textit{houses}” in \textit{I} \textit{suppose} \textit{there} \textit{are} \textit{many} \textit{houses} \textit{in} \textit{the} \textit{city} in English can indicate an unspecific group. Thus, I have to mention another difference between \textit{-kja} (\textsc{pl}) and the plural forms in other languages. On the one hand, \textit{-kja} (\textsc{pl}) can be used to indicate a single referent as an example (to illustrate the proposition expressed by the clause where \textit{-kja} (\textsc{pl}) is included). For example, \textit{waa-kja} (1-\textsc{pl}) in (\ref{ex:6-92}a-b) indicates the speaker alone as an example (to illustrate the proposition expressed by the clause where \textit{-kja} (\textsc{pl}) is included). On the other hand, \textit{-s} in \textit{houses} in English does not have a meaning like that.

The above argumentation is summarized as follows.

\ea\label{ex:6-94}
  The difference between \textit{-kja} (\textsc{pl}) and the plural markers in other languages;\\

 \ea \textit{-kja} (\textsc{pl}) can indicate an unspecific group (which is different from the associative plural);\\
\ex \textit{-kja} (\textsc{pl}) can indicate a singel referent as an example (to illustrate the proposition expressed by the clause where \textit{-kja} (\textsc{pl}) is included).\\

\z
\z

The above characteristics also found in the other plural markers in Yuwan, i.e. \textit{-taa} (\textsc{pl}) and \textit{nkja} (\textsc{appr}).

I will present examples of \textit{-taa} (\textsc{pl}). (\ref{ex:6-95}a) is a conversation of \textsc{tm} with US. (\ref{ex:6-95}b) is a conversation of TM with \textsc{ms}.

\ea\label{ex:6-95}
  \textit{-taa} (\textsc{pl})\\
 \ea\relax[Context: \textsc{tm} is speaking to US about the present author. (US’s reply is omitted from the convesation for convenience.)]\\
{\TM}
\glll  jonesigetaa  cˀjantu  attaa  ziisantugajoo    {\textbar}itoko{\textbar}bəi  najuncjɨ.\\
\textit{jonesige-taa}  \textit{cˀjan=tu}  \textit{a-rɨ-taa}  \textit{ziisan=tu=ga=joo}   \textit{itoko=bəi}  \textit{nar-jur-n=ccjɨ}\\
Yoneshige-\textsc{pl}  father=\textsc{com}  \textsc{dist}-\textsc{nlz}-\textsc{pl}  grandfather=\textsc{com}=\textsc{nom}=\textsc{cfm}1      cousin=only  become-\textsc{umrk}-\textsc{ptcp}=\textsc{qt}\\
\glt ‘Yoneshige’s father and his [i.e. the present speaker’s] grandfather are cousin, (I heard).’ [Co: 110328\_00.txt]

\ex\relax[Context: There was a bell used to tell time, and it used to be rung by a subordinate who was working under the chief of the Yuwan district.]\\
{\TM}
\glll  kucjoo-san=nu  sja=nan.  mata,  a-t-taa=ja,     cˀju=ja  cɨ-cju-tat-tu.\\
\textit{kucjoo-san=nu}  \textit{sja=nan}  \textit{mata}  \textit{a-rɨ-taa=ja}     \textit{cˀju=ja}  \textit{cɨk-tur-tar-tu}\\
chief.of.a.ward-\textsc{hon}=\textsc{gen}  below=\textsc{loc1}  again  \textsc{dist}-\textsc{nlz}-\textsc{pl}=\textsc{top} person=\textsc{top}  accompany-\textsc{prog}-\textsc{pst}-\textsc{csl}\\
\glt ‘A subordinate was working under the man, (who was) the chief of our ward, so ...’ [Co: 111113\_02.txt]
\z
\z

In (\ref{ex:6-95}a), \textsc{tm} and US had not seen the other members of the present author’s family. Thus, it is natural to think that /attaa/ \textit{a-rɨ-taa} (\textsc{dist}-\textsc{nlz}-\textsc{pl}) in this example indicates specifically the present author alone. At least, it is difficult to translate TM’s second utterance into ‘their grandfather’ in this context. One might think that the plurality of the modifier is induced by the head nominal, i.e. \textit{ziisan} ‘grandfather,’ because kin terms are always related with a broad kinship relation. However, it is not the case at least in the case of Yuwan. For example, a singular form (i.e. /akka/ \textit{a-rɨ=ga} (\textsc{dist}-\textsc{nlz}=\textsc{gen})) can fill the modifier slot of an NP whose head is the same kinship term (i.e. \textit{ziisan} ‘grandfather’) as in (\ref{ex:9-36}b) in \sectref{sec:key:9.1.2.2.} Next, in (\ref{ex:6-95}b), /attaa/ \textit{a-rɨ-taa} (DIST-\textsc{nlz}-\textsc{pl}) indicates the chief of the Yuwan district. One district has one chief. Thus, /attaa/ \textit{a-rɨ-taa} (DIST-\textsc{nlz}-\textsc{pl}) in this example should be interpreted as indicating only one referent.

In both of the examples above, \textit{-taa} (\textsc{pl}) is preceded by the demonstrative stem \textit{a-rɨ} (\textsc{dist}-\textsc{nlz}). \textit{-taa} (\textsc{pl}) can also follow address nouns (see \sectref{sec:key:7.2}). An address noun followed by \textit{-taa} (\textsc{pl}) can also indicate a single referent as in \REF{ex:6-96}.

\ea\label{ex:6-96}
  [Context: \textsc{tm} said that she used to practice the traditional dance until someone visited her.]

{\TM}
\glll minakotaa,  akka  kˀuugadɨ,\\
\textit{minako-taa}  \textit{a-rɨ=ga}  \textit{k-gadɨ}\\
    Minako-\textsc{pl}  \textsc{dist}-\textsc{nlz}=\textsc{nom}  come-until\\
\glt    ‘Minako\textit{\textsubscript{i}}, until she\textit{\textsubscript{i}} come (here), ...’ [Co: 120415\_01.txt]
\z

In \REF{ex:6-96}, \textit{minako-taa} (Minako-\textsc{pl}) indicates only one referent, i.e. ‘Minako.’ The semantic “non-plurality” of the referent can be implied by the singular pronoun \textit{a-rɨ} (\textsc{dist}-\textsc{nlz}) ‘she,’ which followed and paraphrased the preceding \textit{minako-taa} (Minako-\textsc{pl}), which is very similar to the case in (\ref{ex:6-92}a). In order to specify the ability to indicate a single referent using \textit{-taa} (\textsc{pl}), I did an elicitation research as in \REF{ex:6-97}, where the singularity of the agent is stressed by the extended NP \textit{cˀjui=sjɨ} (one.person.\textsc{clf}=\textsc{inst}) ‘alone.’ Both \textit{-taa} (\textsc{pl}) and \textit{cˀjui=sjɨ} ‘alone’ are underlined below.

\ea\label{ex:6-97}
  \textit{-taa} (\textsc{pl})\\{}
  [Context: \textsc{tm} is talking about a person, and the person is the only candidate who is assumed by the speaker.]

{\TM}
\glll urəə  mucɨkasjanu,  attaa cˀjuisjəə  siikijandoo.\\
\textit{u-rɨ=ja}  \textit{mucɨkasj-sa=nu}  \textit{a-rɨ-taa}    \textit{cˀjui=sjɨ=ja}  \textit{sɨr-i+kij-an=doo}\\
    \textsc{mes}-\textsc{nlz}=\textsc{top}  difficult-\textsc{adj}=\textsc{csl}  \textsc{dist}-\textsc{nlz}-\textsc{pl} one.person.\textsc{clf}=\textsc{inst}=\textsc{top}  do-\textsc{inf}+\textsc{cap}-\textsc{neg}=\textsc{ass}\\
\glt    ‘That is difficult, so he cannot do (it) alone.’ [El: 130820]
\z


In \REF{ex:6-97}, /attaa/ \textit{a-rɨ-taa} (\textsc{dist}-\textsc{nlz}-\textsc{pl}) is used to indicate a person as an example who cannot do the difficult thing mentioned, which can be translated into ‘a person like him.’

  Finally, I will present examples of \textit{nkja} (\textsc{appr}). In (\ref{ex:6-98}a), \textsc{tm} and \textsc{ms} were looking at a picture, and she said that she did not know such a scene on it. Here, \textit{ku-rɨ=nkja} (\textsc{prox}-\textsc{nlz}=\textsc{appr}) did not indicate plural pictures in the photographic collection, but indicated a single specific picture that they were looking at (perhaps with unspecific pictures that were also unfamiliar to TM). In (\ref{ex:6-98}b), there is only a house where the speaker lived, and \textit{nkja} (APPR) is used to indicate the house as an example of the old houses where there is no papered sliding door.

\ea\label{ex:6-98}
  \textit{nkja} (\textsc{appr})\\{}
  [Context: \textsc{tm} and \textsc{ms} were looking at a picture (in a photographic collection), where was a scene TM had not seen before]

 {\TM}
 \ea
\glll  sijan,  kurɨnkjoo.\\
\textit{sij-an}  \textit{ku-rɨ=nkja=ja}\\
know-\textsc{neg}  \textsc{prox}-\textsc{nlz}=\textsc{appr}=\textsc{top}\\
\glt ‘(I) don’t know this [i.e. the picture].’ [Co: 120415\_00.txt]

{\TM}
\ex
\glll waakjaa  jankjoo  {\textbar}husumasjoozi{\textbar}n  nənba,\\
 \textit{waa-kja-a}  \textit{jaa=nkja=ja}  \textit{husuma+sjoozi=n}  \textit{nə-an-ba}\\
1-\textsc{pl}-\textsc{adnz}  house=\textsc{appr}=\textsc{top}  k.o.door+k.o.door=also  exist-\textsc{neg}-\textsc{csl}\\
\glt ‘Our house did not have \textit{fusuma} [i.e. thick papered sliding door] and also \textit{shōji} [i.e. thin papered sliding door], so ...’ [Co: 111113\_02.txt]
\z
\z

The characteristics of these examples correspond to those in (\ref{ex:6-94}a-b).

  The above uses of the “plural” markers in Yuwan do not seem to be similar to the uses of the plural markers in other languages. At least, they are different from the so-called associative plural. It is probable that a use of the plural markers that is named “approximative” by \citet[239-240]{Corbett2000} may be the candidate. For example, \citet[239]{Corbett2000} cited the use of the plural markers in Dogon (spoken in Mari): \textit{isu} \textit{mbe} \textit{nie} \textit{mbe} (fish \textsc{pl} oil \textsc{pl}) ‘fish, oil, and similar things’ [‘du poisson, de l’huile et cetera’ in the original text in \citet[11]{Plungian1995}]. According to \citet[240]{Corbett2000}, “(t)he approximative requires more research. There is evidence only for the use of the plural.” Therefore, the more elaborated research of the plural markers in Yuwan will present the good examples for the approximative.

  For the reader’s convenience, I glossed both of \textit{-kja} and \textit{-taa} as “\textsc{pl}” (i.e. plural). On the other hand, I glossed \textit{nkja} as “\textsc{appr}” (i.e. approximative) considering its capability to follow not only nominals but also verbs (see \sectref{sec:key:10.1.6} for more details).

\subsubsection{Morphosyntax of plural (or approximative) markers}

The three plural markers -\textit{kja} (\textsc{pl}), \textit{-taa} (\textsc{pl}), and \textit{nkja} (\textsc{appr}) are chosen in this order corresponding to the lexical meaning of their preceding nominals, which is subject to the animacy hierarchy of Yuwan (see \tabref{tab:key:44}). A similar phenomenon, where more than one plural marker correspond to the animacy hierarchy, is found in other Ryukyuan languages, e.g. Ogami (Southern Ryukyuan) \citep[133]{Pellard2010}, and also in other languages, e.g. Eastern Huasteca Nahuatl \citep[77-78]{Corbett2000}. The verb in Yuwan do not show any number agreement with the arguments.

First, personal pronominals use \textit{-kja} (\textsc{pl}) to express the plural (or approximative) meaning (see also \sectref{sec:key:5.1}). In (\ref{ex:6-99}a), the first person pronoun has its plural form \textit{waa-kja} (1-\textsc{pl}). In (\ref{ex:6-99}b), the second person honorific pronoun has its plural form \textit{naa-kja} (2.\textsc{hon}-\textsc{pl}). In (\ref{ex:6-99}c), the second person non-honorific pronoun has its plural form as \textit{ura-kja} (2.\textsc{nhon}-\textsc{pl}).

\ea\label{ex:6-99}
 \ea Personal pronominal (1\textsuperscript{st} person)\\{}
[Context: Remembering her childfood after looking at a relatively new picture, where children wore clothes of Western style]

{\TM}
\glll waakjaga  warabɨ  sjuininkjoo,  ganba      hukunkjoo  tˀɨn  nənba.\\
      \textit{waa-kja=ga}  \textit{warabɨ}  \textit{sɨr-tur-i-n=nkja=ja}  \textit{ganba}      \textit{huku=nkja=ja}  \textit{tˀɨɨ=n}  \textit{nə-an-ba}\\
      1-\textsc{pl}=\textsc{nom}  child  do-\textsc{prog}-\textsc{inf}-time=\textsc{appr}=\textsc{top}  therefore      clothes.of.Western.style=\textsc{appr}=\textsc{top}  one.\textsc{clf}=even  exist-\textsc{neg}-\textsc{csl}\\
\glt ‘When we were children, there were no Western style clothes.’ [Co: 111113\_01.txt]

\ex Personal pronominal (2\textsuperscript{nd} person honorific)\\{}
[Context: Speaking to US, whose family used to deal in fish]

{\TM}
\glll naakjaga  sjɨ  moojuinnja,  simanu      jˀudarooga?\\
      \textit{naa-kja=ga}  \textit{sɨr-tɨ}  \textit{moor-jur-i-n=ja}  \textit{sima=nu}     \textit{jˀu=daroo=ga}\\
      2.\textsc{hon}-\textsc{pl}=\textsc{nom}  do-\textsc{seq}  \textsc{hon}-\textsc{umrk}-\textsc{inf}=\textsc{top}  island=\textsc{gen}      fish=\textsc{supp}=\textsc{cfm}3\\
\glt ‘When you dealt in (fish), (they were) probably fish from the community [i.e. fish taken around the community].’ [Co: 110328\_00.txt]

\ex Personal pronominal (2\textsuperscript{nd} person non-honorific)\\{}
[Context: Talking about a riverboat of the \textsc{ms}’s family]

{\TM}
\glll urakjoo  nusinkjanu  atattudu,  siccjuro.\\
      \textit{ura-kja=ja}  \textit{nusi=nkja=nu}  \textit{ar-tar-tu=du}  \textit{sij-tur-oo}\\
      2.\textsc{nhon}.\textsc{pl}=\textsc{top}  \textsc{rfl}=\textsc{appr}=\textsc{nom}  exist-\textsc{pst}-\textsc{csl}=\textsc{foc}      know-\textsc{prog}-\textsc{supp}\\
\glt ‘You probably know (it), because you have a riverboat of your own.’ [Co: 111113\_01.txt]
\z
\z


  Second, human interrogatives, human demonstratives and address nouns (i.e. elder kinships and personal names) use \textit{-taa} (\textsc{pl}) to express the plural (or approximative) meaning. In (\ref{ex:6-100}a), the human interrogative root \textit{ta-} ‘who’ has its plural form /tattaa/ \textit{ta-ru-taa} (who-\textsc{nlz}-\textsc{pl}). In (\ref{ex:6-100}b), a human demonstrative root \textit{u-} (\textsc{mes}) has its plural form /uttaa/ \textit{u-rɨ-taa} (\textsc{mes}-\textsc{nlz}-\textsc{pl}). In (\ref{ex:6-100}c), an address noun (elder kinship) \textit{anmaa} ‘mother’ has its plural form /anmataa/ \textit{anmaa-taa} (mother-\textsc{pl}). Finally, in (\ref{ex:6-100}d), an address noun (personal name) \textit{nobuari} ‘Nobuari’ has its plural form \textit{nobuari-taa} (Nobuari-\textsc{pl}).

\ea\label{ex:6-100}
 \ea Human interrogtive\\
{\US}
\glll tattaaga  umoojuru?\\
      \textit{ta-ru-taa=ga}  \textit{umoor-jur-u}\\
      who-\textsc{nlz}-\textsc{pl}=\textsc{nom}  exist.\textsc{hon}-\textsc{umrk}-\textsc{pfc}\\
\glt ‘Who would (still) be alive (over ninty years old)?’ [Co: 110328\_00.txt]

\ex Human demonstrative\\{}
[Context: Looking for a picture, where a rutual in marriage called ‘Sansankudo’ was held]

{\TM}
\glll uttaaga  {\textbar}sansankudo{\textbar}  sjun  turonkjanu      izituttɨjaa.\\
      \textit{u-rɨ-taa=ga}  \textit{sansankudo}  \textit{sɨr-tur-n}  \textit{turoo=nkja=nu}    \textit{izir-tur-tɨ=jaa}\\
      \textsc{mes}-\textsc{nlz}-\textsc{pl}=\textsc{nom}  k.o.ritual  do-\textsc{prog}-\textsc{ptcp}  place=\textsc{appr}=\textsc{nom}      go.out-\textsc{prog}-\textsc{seq}=\textsc{sol}\\
\glt ‘There was a scene where they were doing Sansankudo.’ [Co: 120415\_00.txt]

\ex Address noun (elder kinship)\\{}
[Context: \textsc{tm} and US said that it would be nice if there were TM’s mother.]

{\TM}
\glll anmataaga  wuppoojaa.\\
      \textit{anmaa-taa=ga}  \textit{wur-boo=jaa}\\
      mother-\textsc{pl}=\textsc{nom}  exist-\textsc{cnd}=\textsc{sol}\\
\glt ‘If there were (a kind of person like my) mother.’ [Co: 110328\_00.txt]

\ex Address noun (personal name)\\{}
[Context: Talking about a riverboat in old days]

{\TM}
\glll naa  nobuaritaakaroo  siccjukkai?\\
      \textit{naa}  \textit{nobuari-taa=kara=ja}  \textit{sij-tur=kai}\\
      already  Nobuari-\textsc{pl}=\textsc{abl}=\textsc{top}  know-\textsc{prog}=\textsc{dub}\\
\glt ‘I wonder if (the generation) after Nobuari already know (it).’ [Co: 111113\_01.txt]
\z
\z

  Finally, the other nominals use \textit{nkja} (\textsc{appr}) to express the plural (or approximative) meaning. If indefinite pronouns or demonstrative pronouns do not indicate human referents, they express the plurality using \textit{nkja} (\textsc{appr}) as in (\ref{ex:6-101}a-b). On the other hand, the reflexive pronoun \textit{nusi} (\textsc{rfl}) also exploits \textit{nkja} (APPR) to indicate the plurarity, although the referent is a human, i.e. the hearer, as in (\ref{ex:6-101}c). Common nouns always exploit \textit{nkja} (APPR) despite the referents being humans or non-humans as in (\ref{ex:6-101}d-e).

\ea\label{ex:6-101}
 \ea Non-human interrogative\\{}
[Context: \textsc{tm} was surprised that US brought a lot of foods to TM’s house.]

{\TM}
\glll nunkjabaga  mata  muccjɨ  moocjaru?\\
      \textit{nuu=nkja=ba=ga}  \textit{mata}  \textit{mut-tɨ}  \textit{moor-tar-u}\\
      what=\textsc{appr}=\textsc{acc}=\textsc{foc}  again  have-\textsc{seq}  \textsc{hon}-\textsc{pst}-\textsc{pfc}\\
\glt ‘What did (you) bring (here) again?’ [Co: 110328\_00.txt]

\ex Non-human demonstrative\\{}
[Context: Looking at a picture]

{\TM}
\glll kurɨnkjoo  daakai?\\
      \textit{ku-rɨ=nkja=ja}  \textit{daa=kai}\\
      \textsc{prox}-\textsc{nlz}=\textsc{appr}=\textsc{top}  where=\textsc{dub}\\
\glt ‘Where (is) this [i.e. the scene of the picture]?’ [Co: 120415\_00.txt]

\ex Human reflexive pronoun [= (\ref{ex:6-99}c)]\\{}
[Context: Talking about a riverboat of the \textsc{ms}’s family]

{\TM}
\glll urakjoo,  nusinkjanu  atattudu,    siccjuro.\\
      \textit{urakja=ja}  \textit{nusi=nkja=nu}  \textit{ar-tar-tu=du}    \textit{sij-tur-oo}\\
      2.\textsc{nhon}.\textsc{pl}=\textsc{top}  \textsc{rfl}=\textsc{appr}=\textsc{nom}  exist-\textsc{pst}-\textsc{csl}=\textsc{foc}      know-\textsc{prog}-\textsc{supp}\\
\glt ‘You probably know (it), because you have a riverboat of your own.’ [Co: 111113\_01.txt]

\ex Human common nouns\\
{\TM}
\glll  mata  namanujoo  warabɨnkjoojoo,   huccjunkjaboo  sɨkandoojaa.\\
\textit{mata}  \textit{nama=nu=joo}  \textit{warabɨ=nkja=ja=joo}  \textit{huccju=nkja=ba=ja}  \textit{sɨk-an=doo=jaa}\\
moreover  now=\textsc{gen}=\textsc{cfm}1  child=\textsc{appr}=\textsc{top}=\textsc{cfm1}      old.person=\textsc{appr}=\textsc{acc}=\textsc{top}  like-\textsc{neg}=\textsc{ass}=\textsc{sol}\\
\glt ‘Moreover, the children in these days do not like the old people.’ [Co: 120415\_01.txt]

\ex Non-human commoun noun\\{}
[Context: Looking at a picture]

{\TM}
\glll kuzɨnkjoo  nənbajaa.\\
      \textit{kuzɨ=nkja=ja}  \textit{nə-an-ba=jaa}\\
      shoe=\textsc{appr}=\textsc{top}  exist-\textsc{neg}-\textsc{csl}=\textsc{sol}\\
\glt ‘There were not any shoes (in those days).’ [Co: 110328\_00.txt]
\z
\z

  \textit{nkja} (\textsc{appr}) can follow other plural markers, i.e. \textit{-kja=nkja} (\textsc{pl}=\textsc{appr}) and \textit{-taa=nkja} (\textsc{pl}=APPR). In those cases, \textit{nkja} (APPR) ignores the correspondence with the animacy hierarchy. First, let us see examples of \textit{-kja}=\textit{nkja} (\textsc{pl}=APPR).

\ea\label{ex:6-102}
  Double plural marking

 \ea Personal pronominal (1\textsuperscript{st} person)\\{}
[Context: Looking at a pictue, where there were a few men]

{\TM}
\glll waakjankjoo  waasa  asaa.\footnotemark\\
      \textit{waakja=nkja=ja}  \textit{waa-sa}  \textit{ar-sa}\\
      1\textsc{pl}=\textsc{appr}=\textsc{top}  young-\textsc{adj}  \textsc{stv}-\textsc{pol}\\
\glt ‘I am young(er than them).’ [Co: 111113\_02.txt]
\footnotetext{The regular process is \textit{ar-sa} (\textsc{stv}-\textsc{pol}) > /assa/ (see §\ref{bkm:Ref347177096}), but it realizes as /asaa/ in this example.}

\ex Personal pronominal (2\textsuperscript{nd} person non-honorific)\\{}
[Context: Talking about riverboats]

{\TM}
\glll urakjankja,  josidanu  ozisantankja (..tankja)ga     mucjutakai?\\
      \textit{ura-kja=nkja}  \textit{josida=nu}  \textit{ozisan-ta=nkja=ga}   \textit{mut-tur-tar=kai}\\
      2.\textsc{nhon}-\textsc{pl}=\textsc{appr}  Yoshida=\textsc{gen}  unlce-\textsc{pl}=\textsc{appr}=\textsc{nom}  have-\textsc{prog}-\textsc{pst}=\textsc{dub}\\
\glt ‘(I) wonder if you all [i.e. your family] (and) Yoshida’s uncle and his family had (riverboats).’ [Co: 111113\_01.txt]
\z
\z

In fact, the combinations of \textit{-kja} (\textsc{pl}) and \textit{nkja} (\textsc{appr}) as in (\ref{ex:6-102}a-b) are very rare.

On the other hand, the combinations of \textit{-taa} (\textsc{pl}) and \textit{nkja} (\textsc{appr}) are very common in Yuwan.

\ea\label{ex:6-103}
  Double plural marking

 \ea Human interrogtive\\
{\TM}
\glll  urakjaa  tˀɨɨuicjɨboo,  tattankja?\\
\textit{urakja-a}  \textit{tˀɨɨ+ui=ccjɨboo}  \textit{ta-ru-taa=nkja}\\
2.\textsc{nhon}.\textsc{pl}-\textsc{adnz}  one.\textsc{clf}+above= speaking.of  who-\textsc{nlz}-\textsc{pl}=\textsc{appr}\\
\glt ‘Speaking of (the people who are) one (year) older (than) you, who (were they)?’ [Co: 120415\_00.txt]

\ex Address noun (personal name) \& Human demonstrative\\{}
[Context: Remembering the days when people practiced the traditional dances]

{\TM}
\glll sugojaga  arɨ  sjuinnja,  kijomitankja,   attankja,  muru...  sjutanmun,\\
      \textit{sugoja=ga}  \textit{a-rɨ}  \textit{sɨr-tur-i=n=ja}  \textit{kijomi-taa=nkja}    \textit{a-rɨ-taa=nkja}  \textit{muru}  \textit{sɨr-jur-tar-n=mun}\\
      Sugoya=\textsc{nom}  \textsc{dist}-\textsc{nlz}  do-\textsc{prog}-\textsc{inf}=\textsc{dat}1=\textsc{top}  Kiyomi-\textsc{pl}=\textsc{appr}      \textsc{dist}-\textsc{nlz}-\textsc{pl}=\textsc{appr}  very  do-\textsc{umrk}-\textsc{pst}-\textsc{ptcp}=\textsc{advrs}\\
\glt ‘When Sugoya was doing that [i.e. the practice of their traditional dances], Kiyomi and her friends, they used to do [i.e. participate in] (the practice) eagerly, but ...’ [Co: 120415\_01.txt]

\ex Address noun (elder kinship)\\{}
[Context: Looking at a picture where a formal opening of a prefectural road was held]

{\TM}
\glll waakjaa  anmatankjaga  izjɨ  cˀjancjɨ   jˀicjɨ,\\
      \textit{waakja-a}  \textit{anmaa-taa=nkja=ga}  \textit{ik-tɨ}  \textit{k-tar-n=ccjɨ}  \textit{jˀ-tɨ}\\
      1\textsc{pl}-\textsc{adnz}  mother-\textsc{pl}=\textsc{appr}=\textsc{nom}  go-\textsc{seq}  come-\textsc{pst}-\textsc{ptcp}=\textsc{qt}  say-\textsc{seq}\\
\glt ‘My mother and her friends said that (they) had been [i.e. participated in] (the formal opening), and ...’ [Co: 120415\_01.txt]
\z
\z

In my texts, there are more than thirty examples that have the combination of \textit{-taa=nkja} (\textsc{appr}).

Finally, there is also an example of double marking of \textit{nkja} (\textsc{appr}). However, it seems unproductive, since there is only one such example in my texts.

\ea\label{ex:6-104}
  Double plural marking\\{}
  Common noun\\{}
  [Context: Remebering the old days when Amami Ōshima was occupied by the US military]

{\TM}
\glll unininkjoo, ..\textsubscript{} {\textbar}gakkoosjeito{\textbar}nkjankjagajaa.    arɨ  natɨ,\\
\textit{unin}{\footnotemark}\textit{=nkja=ja}  \textit{gakkoo+sjeito=nkja=nkja=ga=jaa}    \textit{a-rɨ}  \textit{nar-tɨ}\\
    that.time=\textsc{appr}=\textsc{top}  school+pupil=\textsc{appr}=APPR=\textsc{nom}=\textsc{sol}    \textsc{dist}-\textsc{nlz}  \textsc{cop}-\textsc{seq}\\
\glt    ‘In those days, (the teachers felt that) the pupils were that [i.e. in danger], so ...’ [Co: 120415\_00.txt]
\footnotetext{\textit{unin} ‘that time’ must take the allomorph /unini/ before a consonant that fills a coda slot of a syllable.}
\z

\textit{nkja} (\textsc{appr}) has a freer distribution than \textit{-kja} (\textsc{pl}) and \textit{-taa} (\textsc{pl}). Such a fact clearly correlates with the fact that it can follow not only nominals but also verbs, e.g. /mudutɨnkja/ \textit{mudur-tɨ=nkja} (return-\textsc{seq}=\textsc{appr}) (see\sectref{sec:key:10.1.6} for more details). \textit{nkja} (APPR) is a form usually taken by nominals in the lowest (or the rightmost) of the animacy hierarchy in Yuwan. Therefore, it may be possible to say that the above possibility of double plural marking, where the following plural morpheme must be \textit{nkja} (APPR), indicates that the plurality itself decreses the “animacy” of NP, since the personal pronominals, human interrogatives, and human demonstratives in the singular do not take \textit{nkja} (APPR) directly (at least in the texts), but those in the plural can take it. Such a characteristic of the plural forms to decrease the “animacy” of an NP is found also in Polish, although the converse phenomenon is found in Russian \citep[188]{Comrie1989}.

  Before concluding this section, I present the differences between \textit{-kja} (\textsc{pl}) and \textit{nkja} (\textsc{appr}). It is probable that the two forms are cognate, and that /n/ of \textit{nkja} (\textsc{appr}) was *\textit{nu} (\textsc{gen}) in the past. However, they have to be regarded as different morphemes in modern Yuwan because of the following three reasons. First, \textit{nkja} (APPR) can follow the converbal affix \textit{-tɨ} (\textsc{seq}), but \textit{nu} (\textsc{gen}) never follows \textit{-tɨ} (\textsc{seq}). Second, /n/ of \textit{nkja} (APPR) cannot be paraphrased as /nu/, which is different from the contracted genitive particle /n/ discussed in \REF{ex:6-81} in \sectref{sec:key:6.3.2.14.} Third, the plural form of \textit{ura} (2.\textsc{nhon}.\textsc{sg}) ‘you’ is /urakja/ (not /uraakja/), which means that the morpheme preceding \textit{kja} is not the adnominal \textit{ura-a} (2.\textsc{nhon}-\textsc{adnz}) ‘your.’

\subsection{NP modifiers}

The words which can fill the modifier slot of an NP use different morphosyntacitc means to modifiy their head nominal depending on their lexical meanings, which are subject to the animacy hierarchy of Yuwan (see \tabref{tab:key:44}). The distribution of means in the singular is partly different from that in the plural, which is caused by a plural affix \textit{-taa}, which can attach to human interrogatives, human demonstrative, and address nouns. If these three lexical groups take \textit{-taa} (\textsc{pl}), they fill the modifier slot of an NP without any other morpheme, i.e. juxtaposition. As mentioned before, the description of the rightmost nominals (“the other nominals”) in \tabref{tab:key:44} is a little simplified. In fact, non-human demonstratives in the singular, e.g. \textit{a-rɨ} ‘that’, can take not only \textit{nu} (\textsc{gen}) but also \textit{ga} (\textsc{gen}) in an environment, the detail of which is explained at the last of 6.4.2.1.

In the following subsections, we will see examples in the singular (see \sectref{sec:key:6.4.2.1}). Next, we will see the examples in the plural (see \sectref{sec:key:6.4.2.2}). Only the personal pronouns have the dual forms, e.g. /wa-ttəə/ (1-\textsc{du}) ‘the two of us,’ and they take \textit{ga} (\textsc{gen}) when they fill the modifier slot of an NP, which is briefly discussed in \sectref{sec:key:6.4.2.3.}

\subsubsection{NP modifiers in the singular}

An NP modifier in the singular chooses one of the following four means in this order, i.e. affixing of \textit{-a} (\textsc{adnz}), taking \textit{ga} (\textsc{gen}), juxtaposition, and taking \textit{nu} (\textsc{gen}), corresponding to the animacy hierarchy of Yuwan (see \tabref{tab:key:44}).

First, personal pronominals and human interrogatives in the singular become adnominals using an adnominalizer \textit{-a} when they fill the modifier slot of an NP (see also \sectref{sec:key:5.1} and \sectref{sec:key:5.3}). In (\ref{ex:6-105}a), the first-person pronominal takes its adnominal form /waa/ \textit{waa-a} (1.\textsc{sg}-\textsc{adnz}) ‘my.’ In (\ref{ex:6-105}b), the second-person honorific pronominal takes its adnominal form /naa/ \textit{naa-a} (2.\textsc{hon}.SG-\textsc{adnz}) ‘your (honorific).’ In (\ref{ex:6-105}c), the second-person non-honorific pronominal takes its adnominal form \textit{ura-a} (2.\textsc{nhon}.SG-ADNZ) ‘your (non-honorific).’ Finally, in (\ref{ex:6-105}d), the human interrogative takes its adnominal form \textit{ta-a} (who-ADNZ) ‘whose.’

\ea\label{ex:6-105}
  Adnominals

 \ea Personal pronominal (1st person)\\{}
[Context: Talking about a man who used to dub tapes of songs voluntarily for villagers;

{\TM} ‘He said his recorder was not useful these days, and...’]

{\TM}
\glll waa  injasan  {\textbar}kasetto{\textbar}kkwagadɨ      muccjɨ  izjɨ,\\
      \textit{waa-a}  \textit{inja-sa+ar-n}\textsubscript{} \textit{kasetto-kkwa=gadɨ}    \textit{mut-tɨ}  \textit{ik-tɨ}\\
      1\textsc{sg}-\textsc{adnz}  small-\textsc{adj}+\textsc{stv}-\textsc{ptcp}  cassette.recorder-\textsc{dim}=\textsc{lmt}      have-\textsc{seq}  go-\textsc{seq}\\
\glt ‘(He) took even my small cassette recorder, and...’ [Co: 120415\_01.txt]

\ex Personal pronominal (2nd person honorific)\\
{\TM}
\glll  naa  məəkaci  cˀjəəradu,\\
\textit{naa-a}  \textit{məə=kaci}  \textit{k-təəra=du}\\
2.\textsc{hon}.\textsc{sg}-\textsc{adnz}  front=\textsc{all}  come-after =\textsc{foc}\\
\glt ‘After (the present author) came to your place, ...’ [Co: 110328\_00.txt]

\ex Personal pronominal (2nd person non-honorific)\\
{\TM}
\glll  uraa  {\textbar}boosi{\textbar}dooccjɨ  jˀicjɨ,\\
\textit{ura-a}  \textit{boosi=doo=ccjɨ}  \textit{jˀ-tɨ}\\
2.\textsc{nhon}.\textsc{sg}-\textsc{adnz}  hat=\textsc{ass}=\textsc{qt}  say-\textsc{seq}\\
\glt ‘(The boy) said, “(It’s) your hat.”’ [\textsc{pf}: 090827\_02.txt]

\ex  Human interrogative

{\TM}
\glll ude,  umanu  nɨkan  taa  nɨkan  xxx\\
      \textit{ude}  \textit{u-ma=nu}  \textit{nɨkan}  \textit{ta-a}  \textit{nɨkan}  \\
      well  \textsc{mes}-place=\textsc{gen}  mikan  who-\textsc{adnz}  orange\\
\glt ‘Well, whose \textit{mikan} is (this) one [lit. \textit{mikan}] there?’ [Co: 101023\_01.txt]
\z
\z

  Second, human demonstratives in the singular take the genitive case particle \textit{ga} when they fill the modifier slot of an NP as in \REF{ex:6-106} (about the contraction \textit{-rɨ=ga} > /kka/, see \REF{ex:5-20} in \sectref{sec:key:5.2.1}).

\ea\label{ex:6-106}
  Genitive case particle \textit{ga}\\
  Human demonstratives\\

  {\TM}
\glll akka  naa  nuucjɨ?\\
\textit{a-rɨ=ga}  \textit{naa}  \textit{nuu=ccjɨ}\\
    \textsc{dist}-\textsc{nlz}=\textsc{gen}  name  what=\textsc{qt}\\
\glt    ‘What is that person’s name?’ [Co: 110328\_00.txt]
\z

  Third, address nouns (elder kinships or personal names) in the singular can fill the modifier slot of an NP by themselves; in other words, they use juxtaposition to function as NP modifier. In (\ref{ex:6-107}a), the elder kinship term \textit{anmaa} ‘mother’ fills directly the modifier slot of an NP. In (\ref{ex:6-107}b), the personal name \textit{kacumi} ‘Katsumi’ fills directly the modifier slot of an NP too.

\ea\label{ex:6-107}
 Juxtapostion\\

 \ea Address noun (elder kinship)\\{}
[Context: Remembering the day when a few students came to see \textsc{tm}’s mother]

{\TM}
\glll anmaa  məəci  kjuuta.\\
    \textit{anmaa}  \textit{məə=kaci}  \textit{k-jur-tar}\\
      mother  front=\textsc{all}  come-\textsc{umrk}-\textsc{pst}\\
\glt ‘(They) used to come to (my) mother’s place.’ [Co: 110328\_00.txt]

\ex Address noun (personal name)\\
{\TM}
\glll  kun  sɨgu  kaduja  namanu    kacumi  jaa  jappa.\\
\textit{ku-n}  \textit{sɨgu}  \textit{kadu=ja}  \textit{nama=nu}   \textit{kacumi}  \textit{jaa}  \textit{jar-ba}\\
\textsc{prox}-\textsc{adnz}  immediately  corner=\textsc{top}  now=\textsc{gen}  Katsumi  house  \textsc{cop}-\textsc{csl}\\
\glt ‘This one at this corner is Katsumi’s house now.’ [Co: 120415\_00.txt]
\z
\z

  Fourth, most of the other nominals in the singular take the genitive case particle \textit{nu} when they fill the modifier slot of an NP. In (\ref{ex:6-108}a), the non-human interrogative \textit{nuu} ‘what’ takes a genitive particle \textit{nu}. In (\ref{ex:6-108}b), the non-human demonstrative \textit{a-rɨ} ‘that’ takes a genitive particle \textit{nu}. In (\ref{ex:6-108}c), both common nouns \textit{zii} ‘ground’ and \textit{micja} ‘soil’ take genitive particle \textit{nu}.

\ea\label{ex:6-108}
  Genitive case particle \textit{nu}\\
 \ea Non-human interrogative\\
{\TM}
\glll  nuunu  nangikaicjɨdu  umujun.\\
\textit{nuu=nu}  \textit{nangi=kai=ccjɨ=du}  \textit{umuw-jur-n}\\
what=\textsc{gen}  trouble=\textsc{dub}=\textsc{qt}=\textsc{foc}  think-\textsc{umrk}-\textsc{ptcp}\\
\glt ‘(I) wonder what (kinds) of trouble (I took).’ [i.e. ‘I didn’t want to take such a trouble.’]      [Co: 120415\_01.txt]

\ex Non-human demonstrative\\
{\TM}
\glll  {\textbar}sjenkjo{\textbar}nu,  arɨnu  tukin,  naajoo,\\
\textit{sjenkjo=nu}  \textit{a-rɨ=nu}  \textit{tuki=n}  \textit{naa=joo}\\
election=\textsc{gen}  \textsc{dist}-\textsc{nlz}=\textsc{gen}  time=\textsc{dat}1  already=\textsc{cfm}1\\
\glt ‘(At) the time of election, (at the time) of that [i.e. the election], you know, ...’ [Co: 120415\_00.txt]

\ex Common nouns\\{}
[Context: Remembering a lesson told by \textsc{tm}’s aquaintance]

{\TM}
\glll ziinu  micjanu  naanan  dɨkɨjun  munna      gaija  tˀɨn  nəncjɨ.\\
      \textit{zii=nu}  \textit{micja=nu}  \textit{naa=nan}  \textit{dɨkɨr-jur-n}  \textit{mun=ja}      \textit{gai=ja}  \textit{tˀɨɨ=n}  \textit{nə-an=ccjɨ}\\
      ground=\textsc{gen}  soil=\textsc{gen}  inside=\textsc{loc1}  be.born-\textsc{umrk}-\textsc{ptcp}  thing=\textsc{top}      harm=\textsc{top}  one.\textsc{clf}=even  exist-\textsc{neg}=\textsc{qt}\\
\glt ‘(He said) that the things that were made in the soil of the ground are not dangerous at all.’ [Fo: 090307\_00.txt]
\z
\z

It should be noted here that the choice of genitive particles is decided by the lexical meaning of the head within the modifier NP, not by the modifier NP as a whole. This is shown by the following example.

\ea\label{ex:6-109}
  Common noun\\{}
  [Context: \textsc{tm} and US had been talking about an acquaintance, whose nickname they knew, but they did not know his full name.]

{\TM}
\glll an  cˀjunu  naaja  sijan.\\
\textit{a-n}  \textit{cˀju=nu}  \textit{naa=ja}  \textit{sij-an}\\
    \textsc{dist}-\textsc{adnz}  person=\textsc{gen}  name=\textsc{top}  know-\textsc{neg}\\
\glt    ‘(I) don’t know that person’s name.’ [Co: 110328\_00.txt]
\z

In \REF{ex:6-109}, the common noun \textit{cˀju} ‘person’ indicates a human and is modified by a demonstrative \textit{a-n} (\textsc{dist}-\textsc{adnz}) ‘that.’ Thus, the whole NP \textit{a-n} \textit{cˀju=nu} (\textsc{dist}-\textsc{adnz} person=\textsc{gen}) ‘that person’s’ seems to have the same definiteness and “humanness” with the human demonstrative \textit{a-rɨ=ga} (DIST-\textsc{nlz}=\textsc{gen}) ‘that person’s’ in \REF{ex:6-106}. The former, i.e. \textit{a-n} \textit{cˀju=nu} ‘that person’s,’ however, still takes \textit{nu} (\textsc{gen}), while the latter, i.e. \textit{a-rɨ=ga} ‘that person’s’ takes \textit{ga} (\textsc{gen}). These facts mean that the genitive case does not take care of the lexical meaning of the modifier NP as a whole, but only takes care of the head nominal within it. Interestingly, the nominative case behaves differently from the genitive case in this point (see \sectref{sec:key:6.4.3.6} for more details).

  Lastly, it should be mentioned that non-human demonstratives can take either \textit{nu} (\textsc{gen}) as in (\ref{ex:6-108}b) or \textit{ga} (\textsc{gen}) as in (\ref{ex:6-110}a-b), and the former is the usual choice. This fact makes the correspondence of non-human demonstratives within the animacy hierarchy a little complicated.

\ea\label{ex:6-110}
  Non-human demonstrative\\

 \ea\relax[Context: Talking about a famous big banyan tree that used to be there]\\
{\TM}
\glll  naakjoo  ukka  sjantɨ  asɨbantɨ?\\
\textit{naakja=ja}  \textit{u-rɨ=ga}  \textit{sja=nantɨ}  \textit{asɨb-an-tɨ}\\
2.\textsc{hon}.\textsc{pl}=\textsc{top}  \textsc{mes}-\textsc{nlz}=\textsc{gen}  under=\textsc{loc2}  play-\textsc{neg}-\textsc{seq}\\
\glt ‘Didn’t you play at the place under that [i.e. the banyan tree]?’ [Co: 110328\_00.txt]

\ex\relax[Context: \textsc{tm} heard that \textsc{my} put an egg into the miso soup in the every morning.]\\
{\TM}
\glll  ugga  naakaci  ɨrɨppoo,  jiccjai.\\
\textit{u-rɨ=ga}  \textit{naa=kaci}  \textit{ɨrɨr-boo}  \textit{jiccj-sa+ar-i}\\
\textsc{mes}-\textsc{nlz}=\textsc{gen}  inside=\textsc{all}  put.in-\textsc{cnd}  good-\textsc{adj}+\textsc{stv}-\textsc{npst}\\
\glt ‘If (you) put (it) inside that [i.e. the soup], (it will) be good.’ [Co: 101023\_01.txt]
\z
\z

The above demonstratives do not indicate humans, but they can take \textit{ga} (\textsc{gen}). The flexible correspondence with the animacy hierarchy found in the above examples was not found in the behavior of plural markers in the text corpus, where human demonstratives always take \textit{-taa} (\textsc{pl}), and non-human demonstratives do not take it (see \sectref{sec:key:5.2.1} about the data from elicitation).

  The behaviour of words in the singular to fill the modifier slot of an NP was shown above; then, we will see that in the plural in the following section.

\subsubsection{NP modifiers in the plural}

An NP modifier in the plural chooses one of the following three means in this order, i.e. affixing \textit{-a} (\textsc{adnz}), juxtaposition, and taking \textit{nu} (\textsc{gen}), corresponding to the animacy hierarchy of Yuwan (see \tabref{tab:key:44}).

First, personal pronominals in the plural, as well as in the singular, become adnominals using an adnominalizer \textit{-a} when they fill the modifier slot of an NP. In (\ref{ex:6-111}a), the first-person pronominal takes its plural adnominal form \textit{waakj-a} (1\textsc{pl}-\textsc{adnz}) ‘our.’ In (\ref{ex:6-111}b), the second-person honorific pronominal takes its plural adnominal form \textit{naakja-a} (2.\textsc{hon}.\textsc{pl}-\textsc{adnz}) ‘your (plural honorific).’ In (\ref{ex:6-111}c), the second-person non-honorific pronominal takes its plural adnominal form \textit{urakj-a} (2.\textsc{nhon}.\textsc{pl}-ADNZ) ‘your (plural non-honorific).’

\ea\label{ex:6-111}
  Adnominals

 \ea Personal pronominal (1\textsuperscript{st} person)\\
{\TM}
\glll  waakjaa  uziitaaga  gan  sjɨ    jatassɨga.\\
\textit{waakja-a}  \textit{uzii-taa=ga}  \textit{ga-n}  \textit{sɨr-tɨ} \textit{jar-tar-sɨga}\\
1\textsc{pl}-\textsc{adnz}  grandfather-\textsc{pl}=\textsc{nom}  \textsc{mes}-\textsc{advz}  do-\textsc{seq} \textsc{cop}-\textsc{pst}-\textsc{pol}\\
\glt ‘My husband [lit. our grandfather (in the perspective of \textsc{tm}’s grandchildren)] did so.’ [Co: 101023\_01.txt]

\ex Personal pronominal (2\textsuperscript{nd} person honorific)\\
{\TM}
\glll  naakjaa  jaakacinkjoo  {\textbar}nenzjuu{\textbar} ikjutanban,\\
\textit{naakja-a}  \textit{jaa=kaci=nkja=ja}  \textit{nenzjuu}   \textit{ik-jur-tar-n=ban}\\
2.\textsc{hon}.\textsc{pl}-\textsc{adnz}  house=\textsc{all}=\textsc{appr}=\textsc{top}  always go-\textsc{umrk}-\textsc{pst}-\textsc{ptcp}=\textsc{advrs}\\
\glt ‘(I) used to go to your house, but ...’ [Co: 110328\_00.txt]

\ex Personal pronominal (2\textsuperscript{nd} person non-honorific)\\
{\TM}
\glll  urakjaa  jaaga,  uinu  jaaga      mukasinu  jaaja.\\
\textit{urakja-a}  \textit{jaa=ga}  \textit{ui=nu}  \textit{jaa=ga}  \textit{mukasi=nu}  \textit{jaa=jaa}\\
2.\textsc{nhon}.\textsc{pl}-\textsc{adnz}  house=\textsc{nom}  above=\textsc{gen}  house=\textsc{nom} past=\textsc{nom}  house=\textsc{sol}\\
\glt ‘Your house, the house above, (is) a traditional house, you know.’ [Co: 111113\_01.txt]
\z
\z

  Second, human interrogatives, human demonstratives, and address nouns in the plural can fill the modifier slot of an NP by themselves. In other words, they use juxtaposition to function as an NP modifier. In (\ref{ex:6-112}a), the human interrogative plural form /tattaa/ \textit{ta-ru-taa} (who-\textsc{nlz}-\textsc{pl}) directly fills the modifier slot of an NP. In (\ref{ex:6-112}b), the human demonstrative plural form /attaa/ \textit{a-rɨ-taa} (\textsc{dist}-\textsc{nlz}-\textsc{pl}) directly fills the modifier slot of an NP. In (\ref{ex:6-112}c), the address noun (elder kinship) plural form \textit{baasan-taa} (grandmothr-\textsc{pl}) directly fills the modifier slot of an NP. In (\ref{ex:6-112}d), the address noun (personal name) plural form \textit{minoe-taa} (Minoe-\textsc{pl}) directly fills the modifier slot of an NP.

\ea\label{ex:6-112}
  Juxtaposition

 \ea Human interrogative\\
{\TM}
\glll  kurəə  tattaa  cɨrɨkai?\\
\textit{ku-rɨ=ja}  \textit{ta-ru-taa}  \textit{cɨrɨ=kai}\\
\textsc{prox}-\textsc{nlz}=\textsc{top}  who-\textsc{nlz}-\textsc{pl}  classmate=\textsc{dub}\\
\glt ‘Whose classmate is this person?’ [Co: 120415\_00.txt]

\ex Human demonstrative\\
{\TM}
\glll  attaa  jaaga  nama  (an)  acjurooga.\\
\textit{a-rɨ-taa}  \textit{jaa=ga}  \textit{nama}    \textit{ak-tur-oo=ga}\\
\textsc{dist}-\textsc{nlz}-\textsc{pl}  house=\textsc{nom}  now    open-\textsc{prog}-\textsc{supp}=\textsc{cfm}3\\
\glt ‘Their house is probably unoccupied now.’ [Co: 120415\_00.txt]

\ex Address noun (elder kinship)\\
{\US}
\gll baasantaa  məə  kˀuranu  atarooga.\\
      \textit{baasan-taa}  \textit{məə}  \textit{kˀura=nu}  \textit{ar-tar-oo=ga}\\
      grandmother-\textsc{pl}  front  storehouse=\textsc{nom}  exist-\textsc{pst}-\textsc{supp}=\textsc{cfm}3\\
\glt ‘There was probably a storehouse (in) front of (my) grandmother(’s house).’ [Co: 110328\_00.txt]

\ex Address noun (personal name)\\
{\TM}
\glll  arəə  minoetaa  cˀjantaaga  cɨkɨtən  {\textbar}suidoo{\textbar}  jatɨkai?\\
\textit{a-rɨ=ja}  \textit{minoe-taa}  \textit{cˀjan-taa=ga}  \textit{cɨkɨr-təər-n}  \textit{suidoo}  \textit{jar-tɨ=kai}\\
\textsc{dist}-\textsc{nlz}=\textsc{top}  Minoe-\textsc{pl}  father-\textsc{pl}=\textsc{nom}  make-\textsc{rsl}-\textsc{ptcp} water.conduit  \textsc{cop}-\textsc{seq}=\textsc{dub}\\
\glt ‘Was that the water conduit which was made by Minoe (and her family)’s father (and his friends)?’ [Co: 110328\_00.txt]
\z
\z

The means of human interrogative and human demonstratives in the plural is different from that in the singular (see \sectref{sec:key:6.4.2.1}). Such a difference is clearly caused by the plural affix \textit{-taa} (\textsc{pl}), which forces the means to fill the modifier slot of an NP to become juxtaposition. It is possible to think that \textit{-taa} (\textsc{pl}) decreases the “animacy” of the above NPs. For example, human interrogatives change the means from \textit{-a} (\textsc{adnz}), which is exploited by the nominals in the higher (or left side) rank of the animacy hierarchy, to juxtaposition, which is used by the nominals in the relatively lower rank of the animacy hierarchy. Considering these facts, the plurality seems to decrease the animacy of the relevant NPs (see also the remark on the double plural marking in \sectref{sec:key:6.4.1.2}).

  Third, the other nominals in the plural take the genitive case particle \textit{nu} when they fill the modifier slot of an NP. So far, there is no use of non-human plural interrogatives in the modifier slot of an NP. In (\ref{ex:6-113}a), the non-human demonstrative in the plural \textit{a-rɨ=nkja} (\textsc{dist}-\textsc{nlz}=\textsc{appr}) takes a genitive particle \textit{nu}. In (\ref{ex:6-113}b), the common noun in the plural \textit{dusi=nkja} (friend=\textsc{appr}) also takes the genitive particle \textit{nu}.

\ea\label{ex:6-113}
  Genitive case particle \textit{nu}

 \ea Non-human demonstrative\\{}
[Context: Talking about a person who was in the picture of an inn of neighborhood]

{\TM}
\glll arɨnkjanu  huccjunu  sjasinnan      nututtojaa.\\
      \textit{a-rɨ=nkja=nu}  \textit{huccju=nu}  \textit{sjasin=nan}    \textit{nur-tur=doo=jaa}\\
      \textsc{dist}-\textsc{nlz}=\textsc{appr}=\textsc{gen}  old.person=\textsc{gen}  photo=\textsc{loc1} appear/ride-\textsc{prog}=\textsc{ass}=\textsc{sol}\\
\glt ‘(The person) appears in the photo of old people who lived in that [i.e. the inn].’ [Co: 120415\_01.txt]

\ex Common noun\\{}
[Context: After speaking about \textsc{ms}’s father, \textsc{tm} began to speak about the cousin of the friend of MS’s father.]

{\TM}
\glll dusinkjanu  zikinu  {\textbar}itoko{\textbar}nu  muhacianjootaa,     attankjoo,  cunekoccjɨnkjoo  jˀicjan   kutoo  nəntanmun.\\
      \textit{dusi=nkja=nu}  \textit{ziki=nu}  \textit{itoko=nu}  \textit{muhaci+anjoo-taa}   \textit{a-rɨ-taa=nkja=ja}  \textit{cuneko=ccjɨ=nkja=ja}  \textit{jˀ-tar-n}   \textit{kutu=ja}  \textit{nə-an-tar-n=mun}\\
      friend=\textsc{appr}=\textsc{gen}  direct=\textsc{gen}  cousin=\textsc{gen}  Muhachi+older.brother-\textsc{pl} \textsc{dist}-\textsc{nlz}-\textsc{pl}=\textsc{appr}=\textsc{top}  Tsuneko=\textsc{qt}=\textsc{appr}=\textsc{top}  say-\textsc{pst}-\textsc{ptcp}      event=\textsc{top}  exist-\textsc{neg}-\textsc{pst}-\textsc{ptcp}=\textsc{advrs}\\
\glt ‘The direct cousin [i.e. a cousin as a near relative (not by marriage)] of the friend (of your father), Muhachi, he never called (me) Tsuneko (without any honorific title).’ [Co: 120415\_01.txt]
\z
\z

In fact, there are few examples where nominals both in the plural and in the lowest side of animacy hierarchy in \tabref{tab:key:44} fill the modifier slot of an NP. Therefore, I have not found any example where a non-human demonstrative in the plural takes \textit{ga} (\textsc{gen}), which is clearly different from the case of non-human demonstratives in the singular discussed in \REF{ex:6-110} in \sectref{sec:key:6.4.2.1.}

In \sectref{sec:key:6.4.1.2}, we have seen the combination of plural morphemes \textit{-taa=nkja} (\textsc{pl}=\textsc{appr}). However, there is only one example in my texts, where the combination occurs in the modifier slot of an NP. It uses juxtaposition to fill the modifier slot of an NP.

\ea\label{ex:6-114}
  Address noun (elder kinship) with \textit{-taa=nkja} (\textsc{pl}=\textsc{appr})

{\TM}
\glll urakjaa  ziisantaankja  kjoodəə  janban,\\
    \{[\textit{urakja-a}  \textit{ziisan-taa=nkja}]\textsubscript{Modifier}  [\textit{kjoodəə}]\textsubscript{Head}\}\textsubscript{NP}  \textit{jar-n=ban}\\
    2.\textsc{nhon}.\textsc{pl}-\textsc{adnz}  grandfather-\textsc{pl}=\textsc{appr}  brother  \textsc{cop}-\textsc{ptcp}=\textsc{advrs}\\
\glt    ‘(My grandfather) is a brother of your grandfather (and his siblings), but ...’ [Co: 120415\_01.txt]
\z

The NP \textit{urakja-a} \textit{ziisan-taa=nkja} (2.\textsc{nhon}.\textsc{pl}-\textsc{adnz} grandfather-\textsc{pl}=\textsc{appr}) ‘your grandfather (and his siblings)’ directly fills the modifier slot of the larger NP, whose head is \textit{kjoodəə} ‘brother.’ It is probable that juxtaposition is chosen here because the head within the modifier NP is an address noun (elder kinship), i.e. \textit{ziisan} ‘grandfather,’ and also it contains \textit{-taa} (\textsc{pl}).

\subsubsection{NP modifiers in the dual}

Only the personal pronouns have the dual forms, i.e. \textit{wattəə} (1\textsc{du}) ‘the two of us,’ \textit{nattəə} (2.\textsc{hon}.DU) ‘the two of you (honorific), \textit{urattəə} (2.\textsc{nhon}.DU) ‘the two of you (non-honorific),’ and \textit{nattəə} (3DU) ‘the two of them’ (see also \sectref{sec:key:5.1}). These dual forms take \textit{ga} (\textsc{gen}) when they fill the modifier slot of an NP as in (\ref{ex:6-115}a-d).

\ea\label{ex:6-115}
  Genitive case particle \textit{ga}

 \ea Personal pronoun (1st person)\\
{\TM}
\glll  kurəə  wattəəga  mundoo.\\
\textit{ku-rɨ=ja}  \textit{wattəə=ga}  \textit{mun=doo}\\
\textsc{prox}-\textsc{nlz}=\textsc{top}  1\textsc{du}=\textsc{gen}  thing=\textsc{ass}\\
\glt ‘These are ours.’ [lit. ‘These are the two of us’s things.’]      [El: 130812]

\ex Personal pronoun (2nd person honorific)\\
{\TM}
\glll  urəə  nattəəga  mundoo.\\
\textit{u-rɨ=ja}  \textit{nattəə=ga}  \textit{mun=doo}\\
\textsc{mes}-\textsc{nlz}=\textsc{top}  2.\textsc{hon}.\textsc{du}=\textsc{gen}  thing=\textsc{ass}\\
\glt ‘These are yours.’ [lit. ‘These are the two of you’s things.’]       [El: 130812]

\ex Personal pronoun (2nd person non-honorific)\\
{\TM}
\glll  urəə  urattəəga  mundoo.\\
\textit{u-rɨ=ja}  \textit{urattəə=ga}  \textit{mun=doo}\\
\textsc{mes}-\textsc{nlz}=\textsc{top}  2.\textsc{nhon}.\textsc{du}=\textsc{gen}  thing=\textsc{ass}\\
\glt ‘These are yours.’ [lit. ‘These are the two of you’s things.’]       [El: 130812]

\ex Personal pronoun (3nd person)\\
{\TM}
\glll  nattəəga  mun  janban,  muratɨ,  kamɨ!\\
\textit{nattəə=ga}  \textit{mun}  \textit{jar-n=ban}  \textit{muraw-tɨ}  \textit{kam-ɨ}\\
3\textsc{du}=\textsc{gen}  thing  \textsc{cop}-\textsc{ptcp}=\textsc{advrs}  receive-\textsc{seq}  eat-\textsc{imp}\\
\glt ‘(These sweets) are theirs, but receive and eat (them)!’ [lit. ‘(These sweets) are the two of them’s, but receive and eat (them)!’]      [El: 130814]
\z
\z

In the above contexts, the dual genitive forms may be replaced by the plural adnominals. For example, \textit{wattəə=ga} (1\textsc{du}=\textsc{gen}) ‘the two of us’s’ in (\ref{ex:6-115}a) may be replaced by \textit{waakja-a} (1\textsc{pl}-\textsc{adnz}) ‘our.’

\subsection{Nominative case}

The nominative case has two morphemes \textit{ga} and \textit{nu} (see \sectref{sec:key:6.3.2.1} about the grammatical function of the nominative case). We choose one of them depending on the lexical meaning of the preceding nominals, which subject to the animacy hierarchy in Yuwan (see \tabref{tab:key:44}). On the one hand, the nominals other than the lowest (or rightmost) position in the animacy hierarchy (except for human interrogatives), i.e. personal pronominals, human demonstratives, and address nouns must take \textit{ga} (\textsc{nom}). On the other hand, the nominals in the lowest basically take \textit{nu} (\textsc{nom}). We could not know the nominative form of interrogatives, since it should be replaced by the focus marker \textit{ga} (\textsc{foc}) (see \sectref{sec:key:5.3.1} and \sectref{sec:key:10.1}).

The nominals in the lowest of the animacy hierarchy, e.g. common nouns, basically take \textit{nu} (\textsc{nom}). However, they also take \textit{ga} (\textsc{nom}) in the following environments.

\ea\label{ex:6-116}
 \textit{ga} (\textsc{nom}) prevails\\
  Obligatorily if\\

 \ea Clause has a nominal predicate; or\\
\ex Clause expresses incapability;\\
  Frequently if\\
\ex Clause has an adjectival predicate; or\\
\ex Predicate expresses non-existence;\\
  Sometimes if\\
\ex Subject indicates a definite human.\\
\z
\z

In the above five environments, the first two environments, i.e. (\ref{ex:6-116}a-b), obligatorily cause the NP to take \textit{ga} (\textsc{nom}), but the others just tend to cause it. I will present examples in the following subsections, where only the relevant examples, i.e. examples of nominals belonging to the lowest (or rightmost) rank of the animacy hierarchy (\tabref{tab:key:44}), are shown.

  First, we will look at the basic alignment of \textit{ga} (\textsc{nom}) and \textit{nu} (\textsc{nom}) (see \sectref{sec:key:6.4.3.1}). Then, I will present the conditions where \textit{ga} (\textsc{nom}) prevails over \textit{nu} (\textsc{nom}) (see \sectref{sec:key:6.4.3.2} - \sectref{sec:key:6.4.3.6}).

\subsubsection{Basic alignment}

Basically, the nominals in the higher rank of the animacy hierarchy of \tabref{tab:key:44}, must take \textit{ga} (\textsc{nom}), and the nominals in the lowest take \textit{nu} (\textsc{nom}).

First, I will present examples of nominals that must take \textit{ga} (\textsc{nom}). There is no difference of choice of case particles between the nominals in the singular and those in the plural, so they are simply shown together below.

\ea\label{ex:6-117}
  Personal pronominals (1st person)\\
 \ea Singular\\
{\TM}
\glll  naokonnəəcjɨ  wanga  jˀicjaroogai?\\
\textit{naoko+nəə=ccjɨ}  \textit{wan=ga}  \textit{jˀ-tar-oo=ga=i}\\
Naoko+older.sister=\textsc{qt}  1\textsc{sg}=\textsc{nom}  say-\textsc{pst}-\textsc{supp}=\textsc{cfm}3=\textsc{plq}\\
\glt ‘Do (you remember that) I spoke of Naoko?’ [Co: 120415\_00.txt]

\ex Plural\\
{\TM}
\glll  un  hasinantɨ, ...  waakjaga  wutattoo.\\
\textit{u-n}  \textit{hasi=nantɨ}  \textit{waakja=ga}  \textit{wur-tar=doo}\\
\textsc{mes}-\textsc{adnz}  bridge=\textsc{loc2}  1\textsc{pl}=\textsc{nom}  exist-\textsc{pst}=\textsc{ass}\\
\glt ‘We were [i.e. gathered] at the bridge.’ [Co: 110328\_00.txt]


Personal pronominals (2nd person honorific)\\

\ex Singular\\
{\TM}
\glll  nanga  jˀujaa  sjutarooga?\\
\textit{nan=ga}  \textit{jˀu+jaa}  \textit{sɨr-tur-tar-oo=ga}\\
2.\textsc{hon}.\textsc{sg}=\textsc{nom}  fish+house  do-\textsc{prog}-\textsc{pst}-\textsc{supp}=\textsc{cfm}3\\
\glt ‘You were probably running [lit. doing] a fish shop, right?’ [Co: 110328\_00.txt]

\ex Plural\\
{\TM}
\glll  naakjaga  {\textbar}socugjoo{\textbar}  sjəəraga  waakjoo  {\textbar}gakkoo{\textbar}kai?\\
\textit{naakja=ga}  \textit{socugjoo}  \textit{sɨr-təəra=ga}  \textit{waakja=ja}  \textit{gakkoo=kai}\\
2.\textsc{hon}.\textsc{pl}=\textsc{nom}  graduation  do-after=\textsc{foc}  1\textsc{pl}=\textsc{top}  school=\textsc{dub}\\
\glt ‘(Is it) after you had graduated (from the elementary school, when) I (began to go to) school?’ [Co: 110328\_00.txt]


Personal pronominals (2nd person non-honorific)\\

\ex Singular\\
{\TM}
\glll  nobuari  kunuguroo,  uraga  cjəəraga    naa  (mm)  muru  (mm)  urɨdoojaa.\\
\textit{nobuari}  \textit{kunuguru=ja}  \textit{ura=ga}  \textit{k-təəra=ga}    \textit{naa}    \textit{muru}    \textit{u-rɨ=doo=jaa}\\
Nobuari  recently=\textsc{top}  2.\textsc{nhon}.\textsc{sg}=\textsc{nom}  come-after=\textsc{foc}  \textsc{fil}    very    \textsc{mes}-\textsc{nlz}=\textsc{ass}=\textsc{sol}\\
\glt ‘Nobuari (is) recently that [i.e. feels good] after you came (back to Yuwan).’ [Co: 111113\_02.txt]

\ex  Plural\\{}
    [Context: Talking about a freind of \textsc{tm}]\\

{\TM}
\glll urakjaga  konboo,  tudɨnnasanuccjɨ  juuboo,\\
      \textit{urakja=ga}  \textit{k-on-boo}  \textit{tudɨnna-sa=nu=ccjɨ}  \textit{jˀ-boo}\\
      2.\textsc{nhon}.\textsc{pl}=\textsc{nom}  come-\textsc{neg}-\textsc{cnd}  lonely-\textsc{adj}=\textsc{csl}=\textsc{qt}  say-\textsc{cnd}\\
\glt ‘(When the friend) said that, “(I) feel lonly if you do not come, so (come here),” ...’ [Co: 120415\_01.txt]


  Human demonstratives\\

\ex  Singular [= \REF{ex:6-96}]\\

{\TM}
\glll minakotaa,  akka  kˀuugadɨ,\\
      \textit{minako-taa}  \textit{a-rɨ=ga}  \textit{k-gadɨ}\\
      Minako-\textsc{pl}  \textsc{dist}-\textsc{nlz}=\textsc{nom}  come-until\\
\glt ‘Minako, until she come (here), ...’ [Co: 120415\_01.txt]

\ex  Plural\\

{\TM}
\glll attaaga  sjɨ  kəə  sjunban,\\
      \textit{a-rɨ-taa=ga}  \textit{sɨr-tɨ}  \textit{k-i=ja}  \textit{sɨr-jur-n=ban}\\
      \textsc{dist}-\textsc{nlz}-\textsc{pl}=\textsc{nom}  do-\textsc{seq}  come-\textsc{inf}=\textsc{top}  do-\textsc{umrk}-\textsc{ptcp}=\textsc{advrs}\\
\glt ‘They (actually would) do (make lunch there) and come (here with it), but ...’ [Co: 101023\_01.txt]


  Address nouns (elder kinship)\\

\ex  Singular [= \REF{ex:6-53}]\\

{\TM}
\glll uziiga  daibangɨɨnantɨ  nasi  mutunwake.\\
      \textit{uzii=ga}  \textit{daiban+kɨɨ=nantɨ}  \textit{nasi}  \textit{mur-tur-n=wake}\\
      old.man=\textsc{nom}  big+tree=\textsc{loc2}  pear  pick.up-\textsc{prog}-\textsc{ptcp}=\textsc{cfp}\\
\glt ‘An old man is picking pears off on a big tree.’ [\textsc{pf}: 090305\_01.txt]

\ex  Plural\\

{\TM}
\glll daidai  sunaobikija  nagaiki(ikii)bikiccjɨdu    waakjaa  anmataaga  jutattu.\\
      \textit{daidai}  \textit{sunao-biki=ja}  \textit{nagaiki-biki=ccjɨ=du}   \textit{waakja-a}  \textit{anmaa-taa=ga}  \textit{jˀ-jur-tar-tu}\\
      for.generations  Sunao-pedigree=\textsc{top}  long.life-pedigree=\textsc{qt}=\textsc{foc}      1\textsc{pl}-\textsc{adnz}  mother-\textsc{pl}=\textsc{nom}  say-\textsc{umrk}-\textsc{pst}-\textsc{csl}\\
\glt ‘My mother used to say that (the members of) Sunao’s pedigree (has had) long life for generations.’ [Co: 111113\_02.txt]


  Address nouns (personal name)\\

\ex  Singular\\

{\TM}
\glll atoora  nobuariga  jappai  {\textbar}kaacjan{\textbar}ga  jˀicjan   tui,  gan  sjɨ  jatəəttoocjɨ.\\
      \textit{atu=kara}  \textit{nobuari=ga}  \textit{jappai}  \textit{kaacjan=ga}  \textit{jˀ-tar-n}   \textit{tui}  \textit{ga-n}  \textit{sɨr-tɨ}  \textit{jar-təər=doo=ccjɨ}\\
      after=\textsc{abl}  Nobuari=\textsc{nom}  after.all  mother=\textsc{nom}  say-\textsc{pst}-\textsc{ptcp}      as  \textsc{mes}-\textsc{advz}  do-\textsc{seq}  \textsc{cop}-\textsc{rsl}=\textsc{ass}=\textsc{qt}\\
\glt ‘After (that), Nobuari (said) that, “After all, as mother said, (it) was like that.”’ [Co: 120415\_00.txt]

\ex  Plural\\

{\TM}
\glll nobuaritaaga,  joo,  naikwoo ..  ujaja  ujacjɨ  joo ..   ikjasjɨgacjɨnkja  ido  zjen .. zjen  munna  jˀan.\\
      \textit{nobuari-taa=ga}  \textit{joo}  \textit{naikwa=ja}  \textit{uja=ja}  \textit{uja=ccjɨ}  \textit{joo}  \textit{ikja-sjɨ=ga=ccjɨ=nkja}  \textit{ido}  \textit{zjenzjen}  \textit{mun=ja}  \textit{jˀ-an}\\
      Nobuari-\textsc{pl}=\textsc{nom}  \textsc{fil}  a.little=\textsc{top}  parent=\textsc{top}  parent=\textsc{qt}  \textsc{fil}\\
      how-\textsc{advz}=\textsc{foc}=\textsc{qt}=\textsc{appr}  well  at.all  thing=\textsc{top}  say-\textsc{neg}\\
\glt ‘Nobuari (said that) parents (are) parents [i.e. the ways of parents are different from his], (and) do not say anything (like) “How (do you do, mom?)” at all.’ [Co: 120415\_01.txt]
\z
\z

In all of the above examples, the nominals in the higher (or left side) ranks of the animacy hierarchy (except for human interrogatives), i.e. personal pronominals, human demonstratives, and address nouns, take \textit{ga} (\textsc{nom}).

  Next, we will see example of the other nominals.

\ea\label{ex:6-118}
\ea Non-human demonstrative (animate)\\{}
[Context: Talking about silkworms that were in the silk-reeling factory in the community]

{\TM}
\glll namanu  cjoodo  an ...  kˀurusan   cjoocjonu,  (mmm)  arɨnu  wuncjɨjo.\\
      \textit{nama=nu}  \textit{cjoodo}  \textit{a-n}  \textit{kˀuru-sa+ar-n}   \textit{cjoocjo=nu}    \textit{a-rɨ=nu}  \textit{wur-n=ccjɨ=joo}\\
      now=\textsc{gen}  just  \textsc{dist}-\textsc{adnz}  black-\textsc{adj}+\textsc{stv}-\textsc{ptcp}\\  butterfly=\textsc{nom}    \textsc{dist}-\textsc{nlz}=\textsc{nom}  exist-\textsc{ptcp}=\textsc{qt}=\textsc{cfm}1\\
\glt ‘(In those days) there were (moths of silkworms) just (like) that black butterfly (in these days), (and actually, such) that [i.e. the moths] existed.’ [Co: 111113\_01.txt]

\ex Non-human demonstrative (inanimate)\\
{\TM}
\glll  namanu  ({\textbar}taiku{\textbar})  arɨnu  an  turoo.\\
\textit{nama=nu}  \textit{taiku}  \textit{a-rɨ=nu}  \textit{a-n}  \textit{turoo}\\
now=\textsc{gen}  sport  \textsc{dist}-\textsc{nlz}=\textsc{nom}  exist-\textsc{adnz}  place\\
\glt ‘(It is) the place, where that one [i.e. the sport gym] exists.’ [Co: 111113\_01.txt]

\ex Common nouns (innanimate; human)\\
{\TM}
\glll  daibangɨɨnu  atɨ,  unnəntɨ  jinganu  {\textbar}hasigo{\textbar}   kɨɨtɨ,\\
\textit{daiban+kɨɨ=nu}  \textit{ar-tɨ}  \textit{u-n=nəntɨ}  \textit{jinga=nu}  \textit{hasigo}      \textit{kɨɨr-tɨ}\\
big+tree=\textsc{nom}  exist-\textsc{seq}  \textsc{mes}-\textsc{adnz}=\textsc{loc2}  man=\textsc{nom}  ladder  put-\textsc{seq}\\
\glt ‘There was a big tree, and there a man put a ladder (against it), and ...’ [\textsc{pf}: 090222\_00.txt]

\ex Common noun (human)\\{}
[Context: \textsc{tm} was surprised there was a boy with short hair on the picture, for boys in the past usullay have their heads shaven.]

{\TM}
\glll naa,  kurəə,  kamacinkja  muijacjun     kˀwanu  wutɨ.\\
      \textit{naa}  \textit{ku-rɨ=ja}  \textit{kamaci=nkja}  \textit{muij-as-tur-n}    \textit{kˀwa=nu}  \textit{wur-tɨ}\\
      \textsc{fil}  \textsc{prox}-\textsc{nlz}=\textsc{top}  head=\textsc{appr}  grow-\textsc{casu}-\textsc{prog}-\textsc{ptcp}      child=\textsc{nom}  exist-\textsc{seq}\\
\glt ‘(Look at) this, (and) there is a child who grows (the hair of his) head.’ [Co: 120415\_00.txt]
\z
\z

In (\ref{ex:6-118}a-d), the nominals in the lowest (or rightmost) rank of the animacy hierarchy take \textit{nu} (\textsc{nom}).

  In the last of \sectref{sec:key:6.4.1.2}, it was mentioned that there can be a sequence of plural markers, i.e. \textit{-taa=nkja} (\textsc{pl}=\textsc{appr}), where the choice of nominative particle does not change as in (\ref{ex:6-74}b) or (\ref{ex:6-103}c).

\subsubsection{\textit{ga} (\textsc{nom}) prevails obligatorily if the clause has a nominal predicate}

As we have seen in the last of the previous section, usually the nominals in the lowest (or rightmost) rank of the animacy hierarchy take \textit{nu} (\textsc{nom}). There are, however, several cases where such a view is not the case. First of all, I will present the case where the predicate is filled by NPs, i.e. nominal predicates. In that case, the subject NP always takes \textit{ga} (not \textit{nu}).

\ea\label{ex:6-119}
 Non-human demonstratives\\

 \ea\relax[Context: Talking about kinds of snails]\\
{\TM}
\glll  arɨga  tanmjaa  jappajaa.\\
\textit{a-rɨ=ga}  [\textit{tanmjaa}  \textit{jar-ba}]\textsubscript{Nominal predicate}\textit{=jaa}\\
\textsc{dist}-\textsc{nlz}=\textsc{nom}  mud.snail  \textsc{cop}-\textsc{csl}=\textsc{sol}\\
\glt ‘That is a mud snail, you know.’ [Co: 111113\_02.txt]

\ex\relax[Context: Wondering where the place in the picture is; {\TM} ‘(It) may be Nogusuku.’]\\
{\TM}
\glll  kurɨga  jadui  jappa.\\
\textit{ku-rɨ=ga}  [\textit{jadui}  \textit{jar-ba}]\textsubscript{Nominal predicate}\\
\textsc{prox}-\textsc{nlz}=\textsc{nom}  cottage  \textsc{cop}-\textsc{csl}\\
\glt ‘This is the cottage, so (it is probably Nogusuku).’ [Co: 120415\_01.txt]

  Common nouns

\ex\relax[Context: \textsc{tm} asked \textsc{my} where the words \textit{cuburu} and \textit{cubusi} in Yuwan indicate.]\\
{\TM}
\glll  cuburuga  kumadarooga?\\
\textit{cuburu=ga}  [\textit{ku-ma}]\textsubscript{Nominal predicate}\textit{=daroo=ga}\\
head=\textsc{nom}  \textsc{prox}-place=\textsc{supp}=\textsc{cfm}3\\
\glt ‘(The place indicated by the term) \textit{cuburu} is here, right?’ [Co: 110328\_00.txt]

\ex
{\TM}
\gll jaaga  arɨ  jatattu.  bonsan.\\
      \textit{jaa=ga}  [\textit{a-rɨ}  \textit{jar-tar-tu}]\textsubscript{Nominal predicate}  \textit{bonsan}\\
      house=\textsc{nom}  \textsc{dist}-\textsc{nlz}  \textsc{cop}-\textsc{pst}-\textsc{csl}  Buddhist.monk\\
\glt ‘(Since the person’s) house was that. (That is, ) the Buddhist monk.’ [Co: 120415\_00.txt]
\z
\z

The subjects of nominal predicates, i.e. \textit{a-rɨ} ‘that’ in (\ref{ex:6-119}a), \textit{ku-rɨ} ‘this’ in (\ref{ex:6-119}b), \textit{cuburu} ‘head’ in (\ref{ex:6-119}c), and \textit{jaa} ‘house’ in (\ref{ex:6-119}d), take \textit{ga} (\textsc{nom}), inspite of their being non-human demonstratives or common nouns.

A nominal predicate can be filled by an infinitive (or verbal noun) as follows (see \sectref{sec:key:8.4.4.2} for more details).

\ea\label{ex:6-120}
 Head of a nominal predicate being the infinitive\\

 \ea\relax[Context: A couple tied an ox to the grass bound tightly, but the ox ran out.]\\
{\TM}
\glll  mingin  oosiran.  un ...  kusabutuuga\    bukuccjɨ  hazɨrɨ.\\
\textit{ming-i=n}  \textit{oosir-an}  \textit{u-n}  \textit{kusabutuu=ga}    \textit{buku=ccjɨ}  [\textit{hazɨrɨr-Ø}]\textsubscript{Nominal predicate}\\
grab-\textsc{ren}=even  have.time-\textsc{neg}  \textsc{mes}-\textsc{adnz}  grass=\textsc{nom}      disconnected=\textsc{qt}  be.free-\textsc{inf}\\
\glt ‘(They) don’t have time to grab (the ox). The bundled grass came out (of the ground).’ [Fo: 090307\_00.txt]

\ex
{\TM}
\glll kun  {\textbar}ike{\textbar}karanu  mizjuuga  agan  iki.\\
\textit{ku-n}  \textit{ike=kara=nu}  \textit{mizjuu=ga}  \textit{aga-n}  [\textit{ik-i}]\textsubscript{Nominal predicate}\\
\textsc{prox}-\textsc{adnz}  pond=\textsc{abl}=\textsc{gen}  ditch=\textsc{nom}  \textsc{dist}-\textsc{advz}  go-\textsc{inf}\\
\glt ‘The ditch from this pond goes [i.e. extends] there.’ [Co: 120415\_00.txt]
\z
\z

These examples show that the subjects of the nominal predicates filled by the infinitive also take \textit{ga} (\textsc{nom}) inspite of their being common nouns, i.e. \textit{kusabutuu} ‘grass’ in (\ref{ex:6-120}a) or \textit{mizjuu} ‘ditch’ in (\ref{ex:6-120}b).

\subsubsection{\textit{ga} (\textsc{nom}) prevails obligatorily if the the clause expresses incapability}

If all of the following conditions are satisfied, the NP is necessarily marked by \textit{ga} (\textsc{nom}).


\ea\label{ex:6-121}
 Conditions to mark an NP with \textit{ga} (\textsc{nom}):\\

\ea  The clause, which includes the NP, expresses incapability as a whole;\\
\ex The NP is a “core argument” (other than the subject);\\
\ex There is a strong semantic relationship between the NP and its head VP.\\
\z
\z

The “core argument” here tends to be the object of a transitive verb, or the argument that has strong semantic relationship with the head verbs, e.g. \textit{mɨɨ} ‘eye’ and \textit{mj-} ‘look at,’ or \textit{mimi} ‘ear’ and \textit{kik-} ‘hear.’ It is difficult to call the “core arguments” subjects as in (\ref{ex:6-122}a-b), where the subjects are \textit{a-n} \textit{sinsjei} ‘the teacher’ or \textit{a-n} \textit{warabɨ} ‘the child,’ not \textit{mɨɨ} ‘eye.’

\ea\label{ex:6-122}
\ea
{\TM}
\gllll an  sinsjeija  mɨɨga  mjicjɨ  moorancjɨdoo.\\
\textit{a-n}  \textit{sinsjei=ja}  \textit{mɨɨ=ga}  \textit{mj-tɨ}  \textit{moor-an=ccjɨ=doo}\\
      {}[\textsc{dist}-\textsc{adnz}  teacher]=\textsc{top}  eye=\textsc{nom}  see-\textsc{seq}  [\textsc{hon}-\textsc{neg}]=\textsc{qt}=\textsc{ass}\\
      {}[Subject]      [Honorific Aux. verb]\\
\glt ‘(I heard) that the teacher cannot see (with his) eyes.’ [El: 130816]

\ex
  {\TM}
  \glll \textsuperscript{\#}an  warabəə  mɨɨga  mjicjɨ  moorancjɨdoo.\\
       \textit{a-n}  \textit{warabɨ=ja}  \textit{mɨɨ=ga}  \textit{mj-tɨ}  \textit{moor-an=ccjɨ=doo}\\
      {}[\textsc{dist}-\textsc{adnz}  child]=\textsc{top}  eye=\textsc{nom}  see-\textsc{seq}  [\textsc{hon}-\textsc{neg}]=\textsc{qt}=\textsc{ass}\\
      {}[Subject]      [Honorific Aux. verb]\\
\glt      [Intended meaning] ‘(I heard) that the child cannot see (with his) eyes.’ [El: 130816]
\z
\z

In (\ref{ex:6-122}a-b), \textit{mɨɨ} ‘eye’ is not the subject of the clauses, since the acceptability of the use of the auxiliary honorific verb is determined by its preceding NPs, i.e. \textit{a-n} \textit{sinsjei} ‘that teacher’ in (122 a) or \textit{a-n} \textit{warabɨ} ‘that child’ in (\ref{ex:6-122}b), both of which are the subjects of the above sentences (see also \chapref{sec:3}).

  I will present other examples below.

\ea\label{ex:6-123}
 Expressing incapability\\
\ea{}  [= (\ref{ex:5-36}a)]

{\TM}
\glll dɨru?  naa  mɨɨga  mjanba.\\
      \textit{dɨ-ru}  \textit{naa}  \textit{mɨɨ=ga}  \textit{mj-an-ba}\\
      which-\textsc{nlz}  yet  eye=\textsc{nom}  see-\textsc{neg}-\textsc{csl}\\
\glt ‘Which one? (I) cannot see (with my) eyes yet, so (it is difficult to see the picture).’ [Co: 111113\_01.txt]
\ex
{\TM}
\gll mɨɨga  mjan  nata.\\
      \textit{mɨɨ=ga}  \textit{mj-an}  \textit{nar-tar}\\
      eye=\textsc{nom}  see-\textsc{neg}  become-\textsc{pst}\\
\glt ‘(I) lost my sight.’ [lit. ‘(My) eyes became unable to see (anything).’]      [Co: 120415\_00.txt]

\ex
{\TM}
\glll mimiga  kikjanba.\\
 \textit{mimi=ga}  \textit{kik-an-ba}\\
ear=\textsc{nom}  hear-\textsc{neg}-\textsc{csl}\\
\glt ‘(They) cannot hear (with their) ears, so (they are difficult to talk with).’ [Co: 120415\_01.txt]
\z
\z

In (\ref{ex:6-123}a-b), \textit{mɨɨ} ‘eye’ is a common noun, but takes \textit{ga} (\textsc{nom}) and the clauses as a whole mean the incapability of the experiencer. In (\ref{ex:6-123}c), \textit{mimi} ‘ear’ is also a common noun, but takes \textit{ga} (\textsc{nom}) and the clause as a whole means the incapability of the experiencer. The verbal roots themselves in (123 a-c), i.e. \textit{mj-} ‘see’ and \textit{kik-} ‘hear,’ can express capability, even though they do not include any morpheme that especially means capability (see also \REF{ex:6-45} and \REF{ex:6-46} in \sectref{sec:key:6.3.2.1}). In fact, \textit{kik-} ‘hear’ can express capability when it does not follow \textit{mimi=ga} (ear=\textsc{nom}) as in \REF{ex:8-103} in \sectref{sec:key:8.4.3.5.}

The predicates may optionally take the morpheme that expresses capability. The following example is similar to the environment of (\ref{ex:6-123}a), but the predicate takes a morpheme that means capability, i.e. \textit{-ar} (\textsc{cap}). In \REF{ex:6-124}, the common noun \textit{mɨɨ} ‘eye’ also takes \textit{ga} (\textsc{nom}).

\ea\label{ex:6-124}
 Expressing incapability with \textit{ar-} (\textsc{cap})\\

{\TM}
\glll mɨɨga  mjaranba,  naa  taruccjəə  wakaran.\\
\textit{mɨɨ=ga}  \textit{mj-ar-an-ba  naa  ta-ru=ccjɨ=ja  wakar-an}\\
    eye=\textsc{nom}  see-\textsc{cap}-\textsc{neg}-\textsc{csl}  yet  who-\textsc{nlz}=\textsc{qt}=\textsc{top}  understand-\textsc{neg}\\
\glt    ‘(I) cannot see (with my) eyes, so (I) can’t recognize who (it is in the picture) yet.’ [Co: 120415\_00.txt]
\z

It should be noted that \textit{ga} (\textsc{nom}) occurs even after “verbs” if the clause expresses incapability as in (\ref{ex:6-125}a-b).

\ea\label{ex:6-125}
 \ea Lexical verb in \textsc{av}C expressing incapability [= (\ref{ex:6-48}a)]\\

 {\TM}
\glll kumɨnkjanu  nənboo,  kadɨga  ikjankara,\\
\textit{kumɨ=nkja=nu}  \textit{nə-an-boo}  \textit{kam-tɨ=ga}  \textit{ik-an=kara}\\
    rice=\textsc{appr}=\textsc{nom}  exist-\textsc{neg}-\textsc{cnd}  eat-\textsc{seq}=\textsc{nom}  go-\textsc{neg}=\textsc{csl}\\
        Lex. verb  Aux. verb\\
\glt    ‘If there is no food such as rice, (we) cannot live, so ...’ [Co: 120415\_01.txt]

\ex Infinitive in the complement slot of \textsc{lvc} expressing incapability [= \REF{ex:6-49}]\\

{\TM}
\glll aikiga  siikijanba.\\
\textit{aik-i=ga}  \textit{sɨr-i+kij-an-ba}\\
    walk-\textsc{inf}=\textsc{nom}  do-INF+\textsc{cap}-\textsc{neg}-\textsc{csl}\\
    Complement  \textsc{lv}\\
\glt    ‘(I) cannot walk [lit. do walking], so (I cannot bring the pickles from my house).’  [Co: 120415\_01.txt]
\z
\z

These verbs are not “core arguments” since they are not nominals. However, the environements where \textit{ga} (\textsc{nom}) appears in (\ref{ex:6-125}a-b) are very similar to those of nominals as in \REF{ex:6-123}. One may think that the \textit{ga} (\textsc{nom}) in this section is the focus particle \textit{ga} in \sectref{sec:key:10.1.2.2.} In fact, I cannot deny this possibility (see also \sectref{sec:key:6.4.3.7}).

\subsubsection{\textit{ga} (\textsc{nom}) prevails frequently if the clause has an adjectival predicate}

If a clause has an adjectival predicate, the core arguments tends to choose \textit{ga} (\textsc{nom}) rather than \textit{nu} (\textsc{nom}). The “core arguments” here tend to be the subject of the clause, but sometimes it is difficult to call them subject as in (\ref{ex:6-126}a-b), where the subjects are \textit{naakjaa} \textit{anmaa-taa} ‘your mother’ or \textit{an} \textit{warabɨ} ‘that child,’ not \textit{kui} ‘voice.’

\ea\label{ex:6-126}
\ea
{\TM}
\gllll naakjaa  anmataaja  kuinu  kjurasa  atɨ\\
\textit{naakja-a}  \textit{anmaa-taa=ja}  \textit{kui=nu}  \textit{kjura-sa}  \textit{ar-tɨ}    moojutɨ?\\
      {}[2.\textsc{hon}.\textsc{pl}-\textsc{adnz}  mother-\textsc{pl}]=\textsc{top}  voice=\textsc{nom}  beautiful-\textsc{adj}  \textsc{stv}-\textsc{seq}    \textit{moor-jur-tɨ}\\
      {}[Subject]            [\textsc{hon}-\textsc{umrk}-\textsc{seq}]      [Honorific Aux. verb]\\
\glt ‘Did your mother have a beautiful voice?’ [El: 130816]

\ex
{\TM}
\gll \textsuperscript{\#}an  warabəə  kuinu  kjurasa  atɨ   moojutɨ?\\
       \textit{a-n}  \textit{warabɨ=ja}  \textit{kui=nu}  \textit{kjura-sa}  \textit{ar-tɨ}    \textit{moor-jur-tɨ}\\
       {}[\textsc{dist}-\textsc{adnz}  child=\textsc{top}]  voice=\textsc{nom}  beautiful-\textsc{adj}  \textsc{stv}-\textsc{seq}     [\textsc{hon}-\textsc{umrk}-\textsc{seq}]\\
       {}[Subject]            [Honorific Aux. verb]\\
\glt      [Intended meaning] ‘Did that child have a beautiful voice?’ [El: 130816]
\z
\z


In (\ref{ex:6-126}a-b), \textit{kui} ‘voice’ is not the subject of the clauses, since the acceptability of the use of the auxiliary honorific verb \textit{moor-} is determined by its preceding NPs, i.e. \textit{naakjaa} \textit{anmaa-taa} ‘your mother’ or \textit{an} \textit{warabɨ} ‘that child,’ which are the subjects of the above sentences (see also \chapref{sec:3}). If a clause has an adjectival predicate, the core arguments tends to choose \textit{ga} (\textsc{nom}) rather than \textit{nu} (\textsc{nom}) as in (\ref{ex:6-127}a-d). However, the adjectival predicate in the honorific \textsc{av}C does not induce such preference, and the core argument takes \textit{nu} (not \textit{ga}) as in (\ref{ex:6-126}a), at least in elicitation.

Examples that take \textit{ga} (not \textit{nu}) are shown below.

\ea\label{ex:6-127}
 Non-human demonstratives\\
 \ea
 {\TM}
\glll waakjaa  cˀjantaaja  kurɨga  nagasa  atɨ,\\
\textit{waakja-a}  \textit{cˀjan-taa=ja}  \textit{ku-rɨ=ga}  [\textit{naga-sa}  \textit{ar-tɨ}]\textsubscript{Adjectival} \textsubscript{predicate}\\
1\textsc{pl}-\textsc{adnz}  father-\textsc{pl}=\textsc{top}  \textsc{prox}-\textsc{nlz}=\textsc{nom}  long-\textsc{adj}  \textsc{stv}-\textsc{seq}\\
\glt ‘My father was long in this [i.e. stature], so ...’ [i.e. ‘My father was tall, so ...’]      [Co: 111113\_01.txt]

\ex\relax[Context: Talking about silkworms that were in the silk-reeling factory in the community, and the moths are similar to black butterflies that sometimes appear around \textsc{tm}’s house]\\
{\TM}
\glll  arɨnu  wuncjɨjo.  arɨga     nissjagadɨ.\\
\textit{a-rɨ=nu}  \textit{wur-n=ccjɨ=joo}  \textit{a-rɨ=ga}  [\textit{nissj-sa=gadɨ}]\textsubscript{Adjectival predicate}\\
\textsc{dist}-\textsc{nlz}=\textsc{nom}  exist-\textsc{ptcp}=\textsc{qt}=\textsc{cfm}1  \textsc{dist}-\textsc{nlz}=\textsc{nom}  similar-\textsc{adj}=\textsc{lmt}\\
\glt ‘There is that [i.e. black butterflies]. That is very similar (to the moths).’ [Co: 111113\_01.txt]

  Common nouns\\

\ex
{\TM}
\glll haruotaanintəəja  kjoodənkjaga  zjanasa  atɨ,\\
\textit{haruo-taa=nintəə=ja}  \textit{kjoodəə=nkja=ga}  [\textit{zjana-sa}  \textit{ar-tɨ}]\textsubscript{Adjectival predicate}\\
Haruo-\textsc{pl}=people=\textsc{top}  brother=\textsc{appr}=\textsc{nom}  many-\textsc{adj}  \textsc{stv}-\textsc{seq}\\
\glt ‘Haruo and his family have many brothers (and relatives).’[lit. ‘About Haruo and his family, brothers (and relatives) are many.’]      [Co: 120415\_01.txt]

\ex
{\TM}
\glll jaaga  injasankara,\\
\textit{jaa=ga}  [\textit{inja-sa+ar-n}]\textsubscript{Adjectival predicate}\textit{=kara}\\
house=\textsc{nom}  small-\textsc{adj}+\textsc{stv}-\textsc{ptcp}=\textsc{csl}\\
\glt ‘The house is small, so ...’ [Co: 120415\_00.txt]
\z
\z


The core arguments, i.e. \textit{ku-rɨ} ‘this [i.e. stature]’ as in (\ref{ex:6-127}a), \textit{a-rɨ} ‘that (butterfly)’ as in (\ref{ex:6-127}b), \textit{kjoodəə=nkja} ‘brothers (and relatives)’ as in (\ref{ex:6-127}c), and \textit{jaa} ‘house’ as in (\ref{ex:6-127}d), take \textit{ga} (\textsc{nom}) inspite of thier being non-human demonstratives or common nouns. I have not yet found any example in my text data where the non-human demonstrative takes \textit{nu} (\textsc{nom}) with adjectival predicates.

The prior uses of \textit{ga} (\textsc{nom}) as in (\ref{ex:6-127}a-d) are actually seen in Yuwan, but there are still a few examples where the arguments do not take \textit{ga} (\textsc{nom}), but take \textit{nu} (\textsc{nom}) even if their predicates are filled by adjectives.

\ea\label{ex:6-128}
 Common nouns\\

 \ea
 {\TM}
\glll  agaraa  munna  kisjoonu  cjussanu.\\
\textit{aga-raa}  \textit{mun=ja}  \textit{kisjoo=nu}  [\textit{cjuss-sa}]\textsubscript{Adjectival predicate}\textit{=nu}\\
\textsc{dist}-\textsc{drg}.\textsc{adnz}  thing=\textsc{top}  temper=\textsc{nom}  strong-\textsc{adj}=\textsc{csl}\\
\glt ‘That awful man has a strong [i.e hot] temper.’[lit. ‘About the awful man, the temper is strong.’]      [Co: 120415\_01.txt]

\ex\relax[Context: Looking at a man on the picture]\\
{\TM}
\glll  {\textbar}iro{\textbar}nu  kˀurusajaa.\\
\textit{iro=nu}  [\textit{kˀuru-sa}]\textsubscript{Adjectival predicate}\textit{=jaa}\\
color=\textsc{nom}  black-\textsc{adj}=\textsc{sol}\\
\glt ‘(He) looks black.’ [lit. ‘(About him), the color is black.’]     [Co: 120415\_00.txt]
\z
\z

The core arguments in the above examples take \textit{nu} (\textsc{nom}), although they have adjectival predicates.

\subsubsection{\textit{ga} (\textsc{nom}) prevails frequently if the predicate expresses non-existence}

If the predicate expresses non-existence, the core arguments frequenly choose \textit{ga} (\textsc{nom}). In other words, if the predicate is filled by any one of these, i.e. \textit{wur-an} (exist-\textsc{neg}), \textit{nə-n} (exist-\textsc{neg}), \textit{umoor-an} (exist.\textsc{hon}-\textsc{neg}), or \textit{ar-tɨ} \textit{moor-an} (exist-\textsc{seq} \textsc{hon}-\textsc{neg}), the core arguments tend to choose \textit{ga} (\textsc{nom}). The “core arguments” here tend to be the subjects of the clauses, but sometimes it is difficult to call them subjects as in (\ref{ex:6-129}a-b), where the subjects are \textit{a-n} \textit{sinsjei} ‘that teacher’ or \textit{a-n} \textit{warabɨ} ‘that child,’ and not \textit{kanɨ} ‘money’.

\ea\label{ex:6-129}
\ea
{\TM}
\gllll an  sinsjeija  kanɨga  atɨ  mooransjutɨ,     injasan  jaanan  sɨdɨ  moojuncjɨ.\\
\textit{a-n}  \textit{sinsjei=ja}  \textit{kanɨ=ga}  \textit{ar-tɨ}  \textit{moor-an=sjutɨ}     \textit{inja-sa+ar-n}  \textit{jaa=nan}  \textit{sɨm-tɨ}  \textit{moor-jur-n=ccjɨ}\\
      {}[\textsc{dist}-\textsc{adnz}  teacher]=\textsc{top}  money=\textsc{nom}  exist-\textsc{seq}  [\textsc{hon}-\textsc{neg}]=\textsc{seq}   small-\textsc{adj}+\textsc{stv}-\textsc{ptcp}  house=\textsc{loc}  live-\textsc{seq}  \textsc{hon}-\textsc{umrk}-\textsc{ptcp}=\textsc{qt}\\
      {}[Subject]      [Honorific Aux. verb]\\
\glt ‘That teacher does not have money, so (he) lives in a small house.’
[lit. ‘About the teacher, there is no money, so (he) lives in a small house.’]       [El: 130816]

\ex

{\TM}
\glll \textsuperscript{\#}an  warabəə  kanɨga  atɨ  mooransjutɨ,   injasan  jaanan  sɨdɨ  moojuncjɨ.\\
      \textit{a-n}  \textit{warabɨ=ja}  \textit{kanɨ=ga}  \textit{ar-tɨ}  \textit{moor-an=sjutɨ}     \textit{inja-sa+ar-n}  \textit{jaa=nan}  \textit{sɨm-tɨ}  \textit{moor-jur-n=ccjɨ}\\
      {}[\textsc{dist}-\textsc{adnz}  child]=\textsc{top}  money=\textsc{nom}  exist-\textsc{seq}  [\textsc{hon}-\textsc{neg}]=\textsc{seq}      small-\textsc{adj}+\textsc{stv}-\textsc{ptcp}  house=\textsc{loc}  live-\textsc{seq}  \textsc{hon}-\textsc{umrk}-\textsc{ptcp}=\textsc{qt}\\
      {}[Subject]      [Honorific Aux. verb]\\
\glt{}      [Intended meaning] ‘That child does not have money, so (he) lives in a small house.’ [El:]
\z
\z

In (\ref{ex:6-129}a-b), \textit{kanɨ} ‘money’ is not the subject of the clauses, since the acceptability of the use of the auxiliary honorific verb \textit{moor-} is determined by its preceding NPs, i.e. \textit{a-n} \textit{sinsjei} ‘that teacher’ or \textit{a-n} \textit{warabɨ} ‘that child,’ which are the subjects of the above sentences (see also chpater 3).

Other examples are shown below.

\ea\label{ex:6-130}
 Non-human demonstrative and common noun (inanimate)\\
 \ea
 {\TM}
 \glll kumannja  arɨga  nəntattujaa.    {\textbar}zaisan{\textbar}ga  anmai  nəntattu.\\
 \textit{ku-ma=nan=ja}  \textit{a-rɨ=ga}  \textit{nə-an-tar-tu=jaa}     \textit{zaisan=ga}  \textit{anmai}  \textit{nə-an-tar-tu}\\
\textsc{prox}-place=\textsc{loc1}=\textsc{top}  \textsc{dist}-\textsc{nlz}=\textsc{nom}  exist-\textsc{neg}-\textsc{pst}-\textsc{csl}=\textsc{sol}   fortune=\textsc{nom}  so.much  exist-\textsc{neg}-\textsc{pst}-\textsc{csl}\\
\glt ‘(The person) did not have that [i.e. fortune] here. (He) did not have so much money.’ [lit. ‘There was not that [i.e. fortune]. There was not so much money (for him).’]       [Co: 120415\_00.txt]

Common noun (inanimate)\\

\ex
{\TM}
\glll un  sɨcɨzɨbatɨga  tˀɨn  nən  natɨjaa.\\
\textit{u-n}  \textit{sɨcɨzɨ+hatɨɨ=ga}  \textit{tˀɨɨ=n}  \textit{nə-an}     \textit{nar-tɨ=jaa}\\
\textsc{mes}-\textsc{adnz}  cycad+field=\textsc{nom}  one.\textsc{clf}=even  exist-\textsc{neg}  become-\textsc{seq}=\textsc{sol}\\
\glt ‘(It) has become (that) there is no such cycad field.’ [Co: 111113\_02.txt]

Common nouns (human)\\

\ex
{\TM}
\glll siccjun  cˀjuga  wuran.\\
\textit{sij-tur-n}  \textit{cˀju=ga}  \textit{wur-an}\\
know-\textsc{prog}-\textsc{ptcp}  person=\textsc{nom}  exist-\textsc{neg}\\
\glt ‘There is not any person whom I know.’ [Co: 120415\_01.txt]
\z
\z

The above examples show that the core arguments, i.e. \textit{a-rɨ} ‘that [i.e. the fortune]’ and \textit{zaisan} ‘fortune’ in (\ref{ex:6-130}a), \textit{sɨcɨzi+hatɨɨ} ‘cycas field’ in (\ref{ex:6-130}b), and \textit{cˀju} ‘person’ in (\ref{ex:6-130}c) take \textit{ga} (\textsc{nom}) inspite of thier being non-human demonstrative or common nouns. The prior use of \textit{ga} (\textsc{nom}) is actually seen in Yuwan, but there are still several examples where the arguments do not take \textit{ga} (\textsc{nom}), but take \textit{nu} (\textsc{nom}) even if their predicates express non-existence.

\ea\label{ex:6-131}
 Common nouns\\

 \ea
 {\TM}
\glll  ude,  gan  sjan  mununkja  sicjun     cˀjunu  wuranbaccjɨ  jˀicjutɨga,\\
\textit{ude}  \textit{ga-n}  \textit{sɨr-tar-n}  \textit{mun=nkja}  \textit{sij-tur-n  \textit{cˀju=nu}  \textit{wur-an-ba=ccjɨ}  \textit{jˀ-tur-tɨ=ga}}\\
well  \textsc{mes}-\textsc{adnz}  know-\textsc{pst}-\textsc{ptcp}  thing=\textsc{appr}  know-\textsc{prog}-\textsc{ptcp}  person=\textsc{nom}  exist-\textsc{neg}-\textsc{csl}=\textsc{qt}  say-\textsc{prog}-\textsc{seq}=\textsc{foc}\\
\glt ‘Well, (I) said that there is not any person who knows such (a kind of) things, and ...’ [Co: 111113\_02.txt]

\ex\relax[= (\ref{ex:6-48}a)]\\
{\TM}
\glll  kumɨnkjanu  nənboo,  kadɨga  ikjarankara,\\
\textit{kumɨ=nkja=nu}  \textit{nə-an-boo}  \textit{kam-tɨ=ga}  \textit{ik-ar-an=kara}\\
rice=\textsc{appr}=\textsc{nom}  exist-\textsc{neg}-\textsc{csl}  eat-\textsc{seq}=\textsc{foc}  go-\textsc{cap}-\textsc{neg}=CSL\\
\glt ‘If there is no food such as rice, (we) cannot live, so ...’ [Co: 120415\_01.txt]
\z
\z

The core arguments in the above examples take \textit{nu} (\textsc{nom}), although their predicates express non-existence.

\subsubsection{\textit{ga} (\textsc{nom}) prevails sometimes if the subject indicates a definite human}

If the subject NP indicates a referent that is both definite and human, it sometimes chooses \textit{ga} (\textsc{nom}).

\ea\label{ex:6-132}
 Common nouns (human)\\

 \ea
 {\TM}
\glll  un  kˀwaga  umanan  {\textbar}boosi{\textbar}  utucjəətattu,\\
\textit{u-n}  \textit{kˀwa=ga}  \textit{u-ma=nan}  \textit{boosi}  \textit{utus-təər-tar-tu}\\
\textsc{mes}-\textsc{adnz}  child=\textsc{nom}  \textsc{mes}-place=\textsc{loc1}  hat  drop-\textsc{rsl}-\textsc{pst}-\textsc{csl}\\
\glt ‘That boy had left [lit. dropped] (his) hat there, so ...’ [\textsc{pf}: 090222\_00.txt]

\ex
{\TM}
\glll an  wunaguga  siimiciga  sijansjutɨ,\\
\textit{a-n}  \textit{wunagu=ga}  \textit{sɨr-i+mici=ga}  \textit{sij-an=sjutɨ}\\
\textsc{dist}-\textsc{adnz}  woman=\textsc{nom}  do-\textsc{inf}+way=\textsc{nom}  know-\textsc{negseQ}\\
\glt ‘That woman don’t know the way to do (it), and ...’ [Co: 101023\_01.txt]

\ex
{\TM}
\glll un  cˀjuga  jukkadɨ  humɨjutassɨga.\\
\textit{u-n}  \textit{cˀju=ga}  \textit{jukkadɨ}  \textit{humɨr-jur-tar-sɨga}\\
\textsc{mes}-\textsc{adnz}  person=\textsc{nom}  always  praise-\textsc{umrk}\\
\glt ‘That person always praised (you).’ [Co: 120415\_01.txt]
\z
\z

The subject NPs in the above examples indicate definite humans, as \textit{u-n} \textit{kˀwa} (\textsc{mes}-\textsc{adnz} child) ‘that child’ in (\ref{ex:6-132}a), \textit{a-n} \textit{wunagu} (\textsc{dist}-\textsc{adnz} woman) ‘that woman’ in (\ref{ex:6-132}b), and \textit{u-n} \textit{cˀju} (\textsc{mes}-ADNZ person) ‘that person,’ and all of them take \textit{ga} (\textsc{nom}). The definiteness of these examples are clarified by the demonstrative adnominals, i.e. \textit{u-n} (\textsc{mes}-ADNZ) or \textit{a-n} (\textsc{dist}-ADNZ). These examples show that the nominative case is very sensitive to the definiteness of the NP (not only the definiteness of its head), and such a sensitivity is a crucial difference between the nominative case and the genitive case (see \REF{ex:6-109} in \sectref{sec:key:6.4.2.1}).

  Additionally, there are examples that do not take any overt form to express definiteness, but can be analyzed as definite referents. Those examples appear in the monologue of a folk tale.

\ea\label{ex:6-133}
 \ea Reflexive pronoun\\{}
[Context: A man eavesdropped on the couple, and discovered that the husband found a pot filled with gold coins but did not bring it home.]

{\TM}
\glll mookɨta.  nusiga  izjɨ,  tɨkkonbaccjɨ  jˀicjɨ,\\
      \textit{mookɨr-tar}  \textit{nusi=ga}  \textit{ik-tɨ}  \textit{tɨkk-on-ba=ccjɨ}  \textit{jˀ-tɨ}\\
      earn.money-\textsc{pst}  \textsc{rfl}=\textsc{nom}  go-\textsc{seq}  bring-\textsc{neg}-\textsc{csl}=\textsc{qt}  say-\textsc{seq}\\
\glt ‘(The man) said that, “(I) earned money. (I) myself have to go and bring (it),” and ...’ [Fo: 090307\_00.txt]

\ex Common noun (human)\\{}
[Context: The man who eavesdropped on the couple went to the place where the pot was, but found a pot filled with mud, so he brought it back and threw it to the couple’s house. Then, the pot became filled with gold coins again.]

{\TM}
\glll jingaga,  jaaci  nusarɨja  nusisjɨ  kan   sjɨ  həncjɨ  kjunmuncjɨ,\\
      \textit{jinga=ga}  \textit{jaa=kaci}  \textit{nusarɨ=ja}  \textit{nusi=sjɨ}  \textit{ka-n}      \textit{sɨr-tɨ}  \textit{hənk-tɨ}  \textit{k-jur-n=mun=ccjɨ}\\
      man=\textsc{nom}  house=\textsc{all}  happiness=\textsc{top}  \textsc{rfl}=\textsc{inst}  \textsc{prox}-\textsc{advz}      do-\textsc{seq}  enter-\textsc{seq}  come-\textsc{umrk}-\textsc{ptcp}=\textsc{advrs}=\textsc{qt}\\
\glt ‘The man (said) that, “Happiness comes to the house by itself like this.”, (and ...)’ [Fo: 090307\_00.txt]
\z
\z

In (\ref{ex:6-133}a), the antecedent of the reflexive \textit{nusi} has already introduced in the story, so it must be difinite. Additionally, the referent indicated by \textit{jinga} ‘man’ in (\ref{ex:6-133}b) has already introduced in the story. There are only three persons that were introduced in the story, i.e. a couple of a man and a woman that are said to be honest, and a man who is sly. It is clear from the context that the nominal \textit{jinga} ‘man’ in (\ref{ex:6-133}b) indicates the husband of the couple, so it must be definite too. Thus, these nominals in (\ref{ex:6-133}a-b) took \textit{ga} (\textsc{nom}).

The same phenomenon is also found in the case of the family name. The family name is actually a kind of personal name, but it cannot be used to address someone, which is different from address nouns. Thus, it must take a genitive particle \textit{nu} if it fills in the modifier slot of an NP as in (\ref{ex:6-134}b). However, the family name can take \textit{ga} (\textsc{nom}) when it is the subject of a clause as in (\ref{ex:6-134}a), probably because the family name can also indicate definite humans.

\ea\label{ex:6-134}
  Common nouns (family name)

 \ea Taking \textit{ga} (\textsc{nom}) as the subject\\
{\TM}
\glll  {\textbar}ittoki{\textbar}  motojamaga  misje  katuta.\\
\textit{ittoki}  \textit{motojama=ga}  \textit{misje}  \textit{kar-tur-tar}\\
for.a.while  Motoyama=\textsc{nom}  shop  rent-\textsc{prog}-\textsc{pst}\\
\glt ‘For a while, Motoyama was renting the shop.’ [Co: 120415\_00.txt]

\ex Taking \textit{nu} (\textsc{gen}) as the NP modifier\\
{\TM}
\glll  {\textbar}hai,  hai,  hai{\textbar}.  cjoodo  motojamanu  misje.\\
\textit{hai}  \textit{hai}  \textit{hai}  \textit{cjoodo}  \textit{motojama=nu}  \textit{misje}\\
yes  yes  yes  just  Motoyama=\textsc{gen}  shop\\
\glt ‘Yes, yes, yes, (that’s right). (It is) just (near) Motoyama’s shop.’ [Co: 120415\_00.txt]
\z
\z

  All of the above examples show that the definite human NPs may take \textit{ga} (\textsc{nom}), but there are also examples where they can still take \textit{nu} (\textsc{nom}).

\ea\label{ex:6-135}
  Common nouns

 \ea\relax[Context: \textsc{tm} asked when US had come to her house.]\\
 = (\ref{ex:6-11}b)\\

{\TM}
\glll nanga  kunəəda  umoocjasəə  kun    cˀjunu  cˀjəərai?\\
      \textit{nan=ga}  \textit{kunəəda}  \textit{umoor-tar=sɨ=ja}  \textit{ku-n}   \textit{cˀju=nu}  \textit{k-təəra=i}\\
      2.\textsc{hon}.\textsc{sg}=\textsc{nom}  the.other.day  come.\textsc{hon}-\textsc{pst}=\textsc{fn}=\textsc{top}  \textsc{prox}-\textsc{adnz}      person=\textsc{nom}  come-after=\textsc{plq}\\
\glt ‘(Is it) after this person [i.e. the present author] came (to your house) that you [i.e. US] came (here) the other day?’ [Co: 110328\_00.txt]

\ex\relax[Context: Three children were walking along the way.]\\
{\TM}
\glll  un  kˀwanu,  cˀjuinu  kˀwanu  isjoobiki      hucjɨ,\\
\textit{u-n}  \textit{kˀwa=nu}  \textit{cˀjui=nu}  \textit{kˀwa=nu}  \textit{isjoobiki}   \textit{huk-tɨ}\\
\textsc{mes}-\textsc{adnz}  child=\textsc{nom}  one.\textsc{clf}=\textsc{gen}  child=\textsc{nom}  whistle  blow-\textsc{seq}\\
\glt ‘That child, the child (who is) one (of them) whistled, and ...’ [\textsc{pf}: 090305\_01.txt]

\ex\relax[Context: The Motoyama family borrowd a shop that had been closed.]\\
{\TM}
\glll  {\textbar}hora{\textbar},  umanan  motojamanu  (ka ...)  kˀuutəətattu,      katɨ,  unnən  nunkuin.\\
\textit{hora}  \textit{u-ma=nan}  \textit{motojama=nu}  \textit{kar}  \textit{kˀuur-təər-tar-tu}   \textit{kar-tɨ}  \textit{u-n=nən}  \textit{nuu-nkuin}\\
hey  \textsc{mes}-place=\textsc{loc1}  Motoyama=\textsc{nom}  borrow  close-\textsc{rsl}-\textsc{pst}-\textsc{csl}   borrow-\textsc{seq}  \textsc{mes}-\textsc{adnz}=\textsc{loc1}  what-\textsc{indf}\\
\glt ‘Hey, at the place, Motoyama, since (the shop) had been closed, rented (it), and (they sold) things [lit. anything] there.’ [Co: 120415\_00.txt]
\z
\z

The relevant NPs in (\ref{ex:6-135}a-c) indicate definite humans, but still take \textit{nu} (\textsc{nom}). The difference of frequency between \textit{ga} (\textsc{nom}) and \textit{nu} (\textsc{nom}) after definite human NPs is not very large. Therefore, it can be said that their alternation is merely optional one.

  Before concluding this section, I will present a case where an indefinite person takes \textit{ga} (\textsc{nom}).

\ea\label{ex:6-136}
  [Context: The very beginning of the monologue. {\TM} ‘(I will) start from the scene (where a man) picks up the pears. There is a pear tree, (i.e.) a big tree, ...’]

{\TM}
\glll unnəntɨ  uziiga  cˀjui  joonasi    mutunwake.\\
\textit{u-n=nəntɨ}  \textit{uzii=ga}  \textit{cˀjui}  \textit{joonasi} \textit{mur-tur-n=wake}\\
    \textsc{mes}-\textsc{adnz}=\textsc{loc2}  old.man=\textsc{nom}  one.\textsc{clf}.person  pear   pick.up-\textsc{prog}-\textsc{ptcp}=\textsc{cfp}\\
\glt    ‘There, an old man is picking up pears.’ [\textsc{pf}: 090225\_00.txt]
\z

As will be mentioned in \sectref{sec:key:7.2}, elder kinship terms can be used even if the referents are not actual relatives of the speaker. In \REF{ex:6-136}, \textit{uzii}, which can mean ‘grandfather’ as an address noun, indicates a man who appeared in the Pear Film. That is, it is not the real grandfather of the speaker \textsc{tm}. Additionally, it is the first time to indicate the man in the monologue. Thus, the \textit{uzii} must be indefinite, but it takes \textit{ga} (\textsc{nom}), not \textit{nu} (\textsc{nom}). The above fact means that a certain nominal that is higher in the animacy hierarchy (in \tabref{tab:key:44}) obligatorily takes \textit{ga} (\textsc{nom}) even if it actually indicates an indefinite referent.

\subsubsection{Concluding remarks on the environments where \textit{ga} (\textsc{nom}) prevails}

The environments shown above, where \textit{ga} (\textsc{nom}) prevails over \textit{nu} (\textsc{nom}), can be separated into two large groups: on the one hand, the environments influenced by the characteristic of the predicates as in \sectref{sec:key:6.4.3.2} - \sectref{sec:key:6.4.3.5}; on the other hand, the environment influenced by the characteristic of the argument NPs as in \sectref{sec:key:6.4.3.6.}

The alignment of the plural markers and NP modifiers in the animacy hierarchy is less flexible than that of the nominative case. The plural markers are concerned with the plurality of the head of an NP. The NP modifiers are also concerned with the relation within the NPs. Thus, both the plural markers and NP modifiers are parameters whose value is determined only within the NP. However, the nominative case is different from them, since it is concerned with the relation between the NP and the predicate. Those differences are considered to result in the differences in flexibility among them. Interestingly, the characteristics discussed in \sectref{sec:key:6.4.3.2} - \sectref{sec:key:6.4.3.5} are all concerned with low transitivity. Both the nominal predicate (in \sectref{sec:key:6.4.3.2}) and the adjectival predicate (in \sectref{sec:key:6.4.3.4}) have less (prototypical) transitivity, because they do not cause any change on any opponent (cf. \citealt{Tsunoda1991}: 72). Additionally, the negative pole, i.e. incapability as in \sectref{sec:key:6.4.3.3} and non-existence as in \sectref{sec:key:6.4.3.5}, is thought to have less transitivity (\citealt{HopperThompson1980}: 252).

However, it should be noted that all of the prior use of \textit{ga} (\textsc{nom}) in \sectref{sec:key:6.4.3.2} - 6.4.3.6 may be regarded as the focus particle \textit{ga} (\textsc{foc}) (see \sectref{sec:key:10.1.2.2}). As mentioned in \sectref{sec:key:6.4.3.3}, I could not completely deny this possibility. We need to clarify the details of this problem in future research.

Comparing with plural markers and NP modifiers, the nominative case is very sensitive to the definiteness of the NP. The example \REF{ex:6-109} in \sectref{sec:key:6.4.2.1} showed that NP modifiers are not sensitive to the definiteness of the whole NP, but that they are sensitive to the definiteness of the head nominal of the NP. Similarly, the plural markers are not sensitive to the definiteness of the whole NP, which is shown below.

\ea\label{ex:6-137}
  [Context: Talking about the Bon festival, and some people in Ashiken said that the way taken by the people in Yuwan on the Bon festival was the actually traditional way.]

{\TM}
\glll un  cˀjunkjoo  jutattujaa.\\
\textit{u-n}  \textit{cˀju=nkja=ja}  \textit{jˀ-jur-tar-tu=jaa}\\
    \{[\textsc{mes}-\textsc{adnz}]  [person]\}=\textsc{appr}=\textsc{top}  say-\textsc{umrk}-\textsc{pst}-\textsc{csl}=\textsc{sol}\\
    \{[Modifier]  [Head]\}\textsubscript{NP}  
\glt    ‘Those people used to say (so).’ [Co: 111113\_01.txt]
\z

In the above example, the NP, i.e. \textit{u-n} \textit{cˀju} (\textsc{mes}-\textsc{adnz} person) ‘that person,’ is definite since it has the demonstrative \textit{u-n} (\textsc{mes}-\textsc{adnz}) ‘that (one)’ in the modifier slot. However, the plural marker that follows the NP is \textit{nkja} (\textsc{appr}), which is on the lowest position on the animacy hierarchy in Yuwan. In other words, such forms as *\textit{u-n} \textit{cˀju-kja} (\textsc{mes}-ADNZ person-\textsc{pl}) or *\textit{u-n} \textit{cˀju-taa} (\textsc{mes}-ADNZ person-\textsc{pl}) are not grammatical. However, the nominative case is sensitive to the definiteness of the whole NP, as discussed in \sectref{sec:key:6.4.3.6} (especially, see (\ref{ex:6-132}c)).

  In conclusion, the form /ga/ comes to be used exclusively as the nominative case, which results in the form /nu/ to be used exclusively as the genitive case. A similar tendency is found in the nominative case and the genitive case in Irabu (southern Ryukyuan) (Michinori \citealt{Shimoji2013} p.c.). There are actually a few examples that do not fit with the environments shown in the above subsections, but still take \textit{ga} (\textsc{nom}). I merely show them without any explanation.

\ea\label{ex:6-138}
\ea\relax[Context: A bad man threw a pot filled with mud.]\\
= (\ref{ex:6-60}a)\\

{\TM}
\glll un  janməəkaci  nagɨrattəətan  cɨboga   mata  kundoo  kinkakaci  natɨ,\\
      \textit{u-n}  \textit{janməə=kaci}  \textit{nagɨr-ar-təər-tar-n}  \textit{cɨbo=ga}      \textit{mata}  \textit{kundu=ja}  \textit{kinka=kaci}  \textit{nar-tɨ}\\
      \textsc{mes}-\textsc{adnz}  garden=\textsc{all}  throw-P\textsc{ass}-\textsc{rsl}-\textsc{pst}-\textsc{ptcp}  pot=\textsc{nom}      again  this.time=\textsc{top}  gold.coin=\textsc{all}  become-\textsc{seq}\\
\glt ‘The pot thrown into the garden became (filled with) gold this time again.’ [Fo: 090307\_00.txt]

\ex\relax[Context: Talking about an acquaintance; {\TM} ‘The village office did the procedure (needed for the person), so...’]\\
{\TM}
\glll  kanɨga  {\textbar}goso{\textbar}cjɨ  həncjɨ.\\
\textit{kanɨ=ga}  \textit{goso=ccjɨ}  \textit{hənk-tɨ}\\
money=\textsc{nom}  a.lot=\textsc{qt}  enter-\textsc{seq}\\
\glt ‘A lot of the money entered (his wallet).’ [Co: 120415\_00.txt]

\ex\relax[Context: Talking about an acquaintance]\\
{\US}
\gll un  ziisanbəiga  atanwake,     kanɨga.\\
      \textit{u-n}  \textit{ziisan=bəi=ga}  \textit{ar-tar-n=wake}      \textit{kanɨ=ga}\\
      \textsc{mes}-\textsc{adnz}  old.man=only=\textsc{nom}  exist-\textsc{pst}-\textsc{ptcp}=\textsc{cfp}    money=\textsc{nom}\\
\glt ‘Only the old man had the money.’ [Co: 110328\_00.txt]
\z
\z

