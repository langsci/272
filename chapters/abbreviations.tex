% \addchap{\lsAbbreviationsTitle}
\addchap{Abbreviations and symbols}
\section*{Abbreviations}\enlargethispage{\baselineskip}
\begin{multicols}{2}
\begin{tabbing}
ECS verbs\=inflectional adjectival affix\kill
A \>  agent-like argument of \\ \> transitive verb; adjective \\
ABL \>  ablative \\
ACC \>  accusative \\
ADJ \>  inflectional adjectival affix \\
ADNZ \>  adnominalizer \\
ADVRS \>  adversative \\
ADVZ \>  adverbializer \\
ALL \>  allative \\
APPR \>  approximative \\
ASS \>  assertive \\
Aux. V \>  auxiliary verb \\
AVC \>  auxiliary verb construction \\
BEN \>  benefactive \\
C \>  any consonant \\
CAP \>  capability \\
CAUS \>  causative \\
CFM \>  confirmation \\
CFP \>  clause-final particle \\
CLF \>  classifier \\
CMP \>  comparative \\
CND \>  conditional \\
Co \>  data from the conversation \\
COM \>  comitative \\
CSL \>  causal \\
DAT \>  dative \\
DIM \>  diminutive \\
DIRC \>  directional \\
DIST \>  distal \\
DRG \>  derogative \\
DUB \>  dubitative \\
DU \>  dual \\
ECS \>  the existential, copula, \\ \> and stative verb \\
El \>  elicitational data \\
FN \>  formal nouns \\
FOC \>  focus \\
Fo \>  data from the folktale \\
GEN \>  genitive \\
G \>  glide slot in a syllable \\
IMP \>  imperative \\
INDFZ \>  indefinitizer \\
INGR \>  ingressive \\
INST \>  instrumental \\
INT \>  intentional \\
k.o. \>  a kind of \\
Lex. V \>  lexical verb \\
LF \>  lengthened (infinitival) form \\
lit. \>  literally \\
LMT \>  limitative \\
LOC \>  locative \\
LST \>  listing \\
LVC \>  light verb construction \\
LV \>  light verb \\
MES \>  mesial \\
MMC \>  Mermaid construction \\
N/A \>  not applicable \\
NEG \>  negative \\
NHON \>  non-honorific \\
NLZ \>  nominalizer \\
NOM \>  nominative \\
NP \>  nominal phrase \\
NPST \>  non-past \\
OBL \>  obligative \\
ODN \>  ordinary number \\
PASS \>  passive \\
PFC \>  predicate of focus \\ \> construction \\
PF \>  pear film \\
PL \>  plural\\
PLQ \>  polar question\\
POL \>  politeness\\
POS \>  possibility\\
P \>  patient-like argument of \\ \> transitive verb \\
PROG \>  progressive\\
PROX \>  proximal\\
PRPR \>  preparative\\
PST \>  past\\
PTCP \>  participle\\
PURP \>  purposive\\
QT \>  quotation\\
RED \>  redupulicant\\
RFL \>  reflexive\\
RSL \>  resultative\\
S \>  an argument of \\ \> intransitive verb\\
SF \>  simple (infinitival) form\\
SG \>  singular\\
SIM \>  simultaneous\\
SOL \>  solidarity\\
STV \>  stative verb\\
SUGS \>  suggessive\\
SUPP \>  suppositional\\
TOP \>  topic\\
UMRK \>  unmarked verbal affix\\
V \>  any vowel; verb\\
VP \>  verbal phrase\\
V\textsubscript{back} \>  back vowels\\
V\textsubscript{non-back} \>  non-back vowels\\
V\textsubscript{non-\textit{i}} \>  vowels excluding //i//\\
X \>  an anonymous \\ \> personal name
\end{tabbing}
\end{multicols}

\section*{Symbols}\enlargethispage{\baselineskip}
\begin{tabularx}{\textwidth}{@{}lQ@{}}
\#    & syllable boundary\\
\textsuperscript{\#}    & context is unnatural\\
\$    & word boundary\\
*    & ungrammatical expression ancestoral form (see also ‘Pre-note (b)’ in appendix)\\
+    & boundary of a compound boundary of reduplication boundary of a contracted adjectival predicate, boudary of the fusion of \textit{ccjɨ} (QT) and \textit{jˀ}{}- ‘say’\\
{}-    & affix boundary\\
=    & clitic boundary\\
A/B    & A or B\\
//A//    & “A” is a morphophoneme (or underlying form)\\
/A/    & “A” is a phoneme (or surface form)
\end{tabularx}
