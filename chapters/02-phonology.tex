\chapter{Phonology}\label{chap:2}
\hypertarget{RefHeadingToc395696960}{}
In this chapter, I will present the phonology in Yuwan. The composition of grammatical words and phonological words will be shown in §\ref{bkm:Ref381193780}. The inventories of vowels and consonants will be shown in §\ref{bkm:Ref381193872}. The syllable structures and phonotactics will be discussed in §\ref{bkm:Ref302599307}. The phonological rules will be presented in §\ref{bkm:Ref302723494}. Finally, the nominal prosody will be discussed in §\ref{bkm:Ref301560567}.

\section{Segmentation}
\label{bkm:Ref381193780}\hypertarget{RefHeadingToc395696961}{}\label{bkm:Ref347179371}
A grammatical word (GW, henceforth simply “word” unless an explicit distinction between a grammatical word and a phonological word is necessary) is a morphosyntactic unit minimally consisting of a root, or it can consist of a root (or roots) plus an affix (or affixes) \citep[cf.][]{DixonAikhenvald2002}. In other cases, a grammatical word may consist of a single clitic. The above description is briefly summarized as follows.

\ea Grammatical words: \glllll\relax
[Root]\textsubscript{GW} [Root-Affix]\textsubscript{GW} [Root-Affix]\textsubscript{GW}=[Clitic]\textsubscript{GW}\\
anmaa anmataa anmatankja\footnotemark{}\\
{\itshape anmaa} {\itshape anmaa-taa} {\itshape anmaa-taa=nkja}\\
mother mother-PL mother-PL=APPR\\
‘mother’ ‘mother and her fellow(s)’ ‘mother and her fellow(s)’\\
\z
\footnotetext{A sequence with the same vowel becomes a single vowel before a consonant that does not have a nucleus (see §\ref{bkm:Ref301832441} in detail). \textit{anmaa} ‘mother’ frequnetly becomes /anma/ when it is follwed by \textit{{}-taa} (PL).}

\noindent Taking the above distinction into consideration, we can recognize another unit, i.e., a phonological word.

\ea Phonological word: [Root (-Affix(es))]\textsubscript{GW} ([=Clitic(s)]\textsubscript{GW}) \z

\noindent A phonological word consists of a grammatical word optionally followed by a clitc (or clitics). A phonological word is the domain in which the following three rules apply: (A) phonological rule (see §\ref{bkm:Ref302723494}); (B) morphophonological rule (see §\ref{bkm:Ref356245430} and other relevant sections); and (C) prosodic rule (see §\ref{bkm:Ref301560567}), although the third criterion is in need of further research (see §\ref{bkm:Ref381418627}).

\section{Phonemes}
\label{bkm:Ref381193872}\hypertarget{RefHeadingToc395696962}{}\label{bkm:Ref347174415}\subsection{Vowels}
\label{bkm:Ref312504424}\hypertarget{RefHeadingToc395696963}{}\subsubsection{Short vowels}
\hypertarget{RefHeadingToc395696964}{}
Vowels are phonologically distinguished as below. Long vowels are treated as vowel sequences (see §\ref{bkm:Ref303982713}).

\begin{table}[H]
\caption{Inventory of vowels}
\begin{tabular}{lccc} 
\lsptoprule
     & Front & Central & Back\\
High &  i    & ɨ       & u\\
Mid  & (e)   & ə [ɜ]   & o [o̞]\\
Low  &       &         & a [ɑ̟]\\
\lspbottomrule
\end{tabular}
\end{table}

Notes:
\begin{enumerate}[label=\alph*.]
\item High vowels: only /i/, /ɨ/, and /u/ are used as epenthetic vowels (see §\ref{bkm:Ref301838720}, §\ref{bkm:Ref347175824}, and \sectref{sec:key:8.2.1.4}). These vowels become voiceless between voiceless consonants or after a voiceless consonant at word-final positions;
\item \label{bkm:Ref347176670}Mid vowels: /e/, /ə/, and /o/ rarely appear as a single short vowel except for the case of vowel deletion (see §\ref{bkm:Ref301832441}). Within the total number of 1014 lexemes, the single short vowel /a/ appears in 468 lexemes, /u/ in 400, /ɨ/ in 260, /i/ in 200, /o/ in 16, and /ə/ in 4 (see the note “e” about /e/);
\item Front and central vowels: /i/ and /ɨ/ are contracted with \textit{ja} (TOP) into /əə/ (see §\ref{bkm:Ref367134300}); verbal stems that end with front or central vowels form a single stem class (see §\ref{bkm:Ref356245430});
\item Back vowels: /u/, /o/, and /a/ are contracted with \textit{ja} (TOP) into /oo/ (see §\ref{bkm:Ref367134300}); verbal stems that end with /ur/, /or/, and /ar/ form a single stem class (see §\ref{bkm:Ref356245430});
\item /e/ is used for a small number of loanwords from Standard Japanese (e.g., /sinsjei/ ‘teacher’) or interjections (e.g., /ude/ ‘hey’).
\end{enumerate}

The minimal contrasts of vowels are shown below. (The majority of the examples in this chapter are from elicited data, so the source information (see §\ref{bkm:Ref347173399}) is omitted.)

\ea 
\ea /i/ vs /ɨ/ vs /ə/ vs /u/\\
 /mii/  vs  /mɨɨ/  vs  /məə/  vs  /muu/\\
 ‘fruit’ {} ‘eye’ {}  ‘front’ {}  ‘alga’\\
\ex /i/ vs /o/\\
/kii/  vs  /koo/\\
‘yellow’ {}  ‘skin’\\
\ex /i/ vs /ɨ/ vs /a/\\
/jii/  vs  /jɨɨ/  vs  /jaa/\\
‘rush’ {}  ‘grip’ {}  ‘house’\\
\ex /ɨ/ vs /o/ vs /ə/\\
/sɨɨ/  vs  /soo/  vs  /səə/\\
‘vinegar’ {}  ‘stem’ {}  ‘alcohol’\\
\ex /u/ vs /o/ vs /ə/ vs /a/\\
/nuu/  vs  /noo/  vs  /nəə/  vs  /naa/\\
‘what’ {}  ‘fishing line’ {}  ‘elder sister’ {}  ‘name’\\
\z
\z

\subsubsection{Long vowels and diphthongs}
\label{bkm:Ref367392475}\hypertarget{RefHeadingToc395696965}{}\label{bkm:Ref347178348}
Every vowel in Yuwan can be lengthened, and this is treated as a vowel sequence (see also §\ref{bkm:Ref303982713}). All diphthongs in Yuwan are combinations of a particular vowel plus /i/.

\begin{table}
\caption{Long vowels and diphthongs}
\begin{tabular}{ *{8}{c} }
\lsptoprule
 V\textit{\textsubscript{1}} & V\textit{\textsubscript{2}} & /a/ & /u/ &  /i/ & /ɨ/ & /ə/ & /o/\\
 /a/  &   & aa & & ai      \\
 /u/  &   & & uu & ui      \\
 /i/  &   & & & ii         \\
 /ɨ/  &   & & & ɨi & ɨɨ    \\
 /ə/  &   & & & əi & & əə  \\
 /o/  &   & & & oi & & & oo\\
\lspbottomrule
\end{tabular}
\end{table}


\begin{table}
\caption{Examples of long vowels and diphthongs}
\begin{tabular}{ *{5}{l} }
\lsptoprule
 & \multicolumn{2}{l}{Long vowels}  & \multicolumn{2}{l}{Diphthongs}\\\midrule
{/a/} & {jaa} & {‘house’}  & {mai}  & {‘hip’}\\
{/u/} & {juu} & {‘boiled water’} & {jui} & {‘lily’}\\
{/i/} & {jii} & {‘rush’} & \multicolumn{2}{l}{(= long vowel)}\\
{/ɨ/} & {jumarɨɨ}  & {(read.PASS.INF)} & jumarɨi & {(read.PASS.NPST)}\\
{/ə/} & {jəəci} & {‘Yakeuchi’} & {jəito} &   {‘well’}\\
{/o/} & {joosɨ} & {‘atmosphere’}  & {joikwa} &  {‘silently’}\\
\lspbottomrule
\end{tabular}
\end{table}


In diphthongs, /ɨi/ is very rare and it occurs only in the combination of \textit{-arɨr} (PASS) and \textit{-i} (NPST), i.e. \textit{-arɨr-i} (PASS-NPST) > /-arɨi/, and the lexeme \textit{jɨɨi} ‘brother.’

\begin{table}
\caption{(Quasi-)minimal pairs of long and short vowels}
\begin{tabular}{ *{5}{l} } 
\lsptoprule
 & \multicolumn{2}{l}{Long vowels} & \multicolumn{2}{l}{Short vowels}\\\midrule
{/a/} & {mjaa} & {‘cat’} & {mja} & {‘k.o. shellfish’}\\
{/u/} & {tuuta} & {(pass.PST)} & {tuta} & {(take.PST)}\\
{/i/} & {jˀiicjasa} & {(say.want.ADJ)} & {jˀicja} & {(say.PST)}\\
{/ɨ/} & {cɨmɨɨ} & {‘k.o. shellfish’} & cɨmɨ & {‘nail’}\\
{/ə/} & {məərabɨ} & {‘young lady’} & {məngaa} & {‘good boy/girl’}\\
{/o/} & {goroogoro} & {‘growling’} & {gooruu} & {‘circle’}\\
\lspbottomrule
\end{tabular}
\end{table}

There are few lexemes where the vowels /ə/ or /o/ is short (see the note “\ref{bkm:Ref347176670}.” of \tabref{tab:2:8}). There are reasons to believe that they are underlyingly /əə/ or /oo/ (see §\ref{bkm:Ref301832441}).

Yuwan has a few morphemes that contain sounds such as [ɑ̟u] ([tɑ̟u] ‘plain,’ [ɑ̟uː] ‘blue,’ [jɑ̟ut͡ɕikkʷɜː] ‘naughty child,’ and [jɑ̟ur] (POL)); however, the vowel sequence [ɑ̟u] can be regarded as /awu/ (not /au/) because of the morphophonological rule in §\ref{bkm:Ref367134300}. It suffices to note that the topic marker \textit{ja} retains its form after a long vowel or diphthong, but loses its form after a short vowel (by combining with the preceding short vowel).

\ea Rule for \textit{ja} (TOP)
\ea After a long vowel or diphthong\\
\begin{tabbing}
    wunagu  \= ‘boiled water’ \=  +   ja   (TOP)   > \=  juuja\kill
    juu  \> ‘boiled water’ \>  +   ja   (TOP)   > \>  juuja\\
    mai  \> ‘hip’          \>  +   ja   (TOP)   > \>  maija
\end{tabbing}
\ex After a short vowel\\
\begin{tabbing}
    wunagu  \= ‘boiled water’ \=  +   ja   (TOP)   > \=  juuja\kill
    wunagu \>  ‘woman’  \> +  ja (TOP)  >  \> wunagoo
\end{tabbing}
\z
\ex The case of [tɑ̟u] ‘plain’\\
\begin{tabbing}
    Phonologically: \= wunagu  \= ‘boiled water’ \=  +   ja   (TOP)   > \=  juuja\kill
    Phonetically:  \> [tɑ̟u] \>  +   ja (TOP)   >  \> [tɑ̟.ʷo̞ː]   (*[tɑ̟u.jɑ̟])\\
    Phonologically: \>   tawu \>  +   ja (TOP)   >  \> tawoo   (*tauja)
\end{tabbing}
\z

In terms of the other morphemes with [ɑ̟u], such as [ɑ̟uː] ‘blue,’ we could not fully determine whether it is /auu/ or /awuu/. However, we do not assume there is a combination of a vowel plus /u/ (besides a vowel plus /i/) for diphthongs since there is no positive indication (considering the case of \textit{tawu} ‘plain’). Thus, we regard [ɑ̟u] in other morphemes as /awu/; that is, /awuu/ ‘blue,’ /jawucikkwəə/ ‘naughty boy,’ and /jawur/ (POL).

\subsection{Consonants}
\subsubsection{The inventory of consonant phonemes}

Yuwan has 22 consonants, listed in \tabref{tab:2:8}.

\begin{table}
\caption{Inventory of consonants\label{tab:2:8}\todo[inline]{please check 1st column}}
\begin{tabular}{rccccc}
\lsptoprule
     & Bilabial &  Alveolar &  Palatal &  Velar & Glottal\\\midrule
voiceless non-glottalized Stops & p & t  & &  k  \\
glottalized Stops  &  & tˀ & &   kˀ  \\
voiced  Stops  &  b  & d  & &  g  \\
voiceless non-glottalized Affricates &   & c      \\
glottalized Affricates  &  & cˀ      \\
voiceless Fricatives   &   & s  & &     h\\
voiced Fricatives  &   & z      \\
non-glottalized Nasals & m & n      \\
{glottalized Nasals} & mˀ & nˀ      \\
{non-glottalized Approximants} & w  & &  j    \\
{glottalized Approximants} & wˀ & &   jˀ    \\
Tap   & & r      \\
\lspbottomrule
\end{tabular}
\end{table}

Notes:
\begin{enumerate}[label=\alph*.]
\item Stops and fricatives have voice opposition;
\item Stops (except for /p/), affricates, nasals, and approximants have glottalization opposition;
\item Alveolar affricates and fricatives behave similarly in terms of morphophonological rules (see \sectref{sec:key:6.3.1.1}, \sectref{sec:key:6.3.1.2}, §\ref{bkm:Ref347177096}, and \sectref{sec:key:10.1.1.1});
\item Approximants and the tap behave similarly in terms of (morpho)phonological rules (§\ref{bkm:Ref304225942} and §\ref{bkm:Ref347177096}).
\end{enumerate}

The phoneme /p/ often appears as a geminate in the combination of a stem and affixes (or clitics). Yuwan has a very restricted number of lexical items that have /p/ (12 lexemes so far), where non-geminated lexemes are \textit{pon+wata} ‘big belly,’ \textit{anpəə} ‘appearance,’ \textit{piri} ‘tail end,’ and \textit{mai=nu} \textit{pɨɨ} (hip=GEN hole) ‘anus,’ excluding onomatopoeia and alleged modern loan words. Additionally, /z/ can be realized as [(d͡)z] (or [(d͡)ʑ]) in Yuwan. However, we regard it as a voiced counterpart of the fricative /s/ since /s/ can precede all the vowels that /z/ can precede, but the affricate /c/ cannot precede all of these vowels. For example, there are phoneme sequences such as /za/ or /sa/, but not /ca/ (see the table in §\ref{bkm:Ref347177989}).

The glottalized phonemes could be analyzed as /ʔC/, reducing the total number of phonemes. This analysis would assume double onset slots for the word-initial syllable. However, it is difficult to propose that there is a slot for /ʔ/, since /ʔ/ cannot precede all the consonants. For example, it cannot precede fricatives or /r/. In addition, this analysis destroys the commonality of syllable structures within a word (see §\ref{bkm:Ref301830963}). Thus, I propose the analysis of /Cˀ/. Furtheremore, I do not assume [ʔ] that precedes word-initial vowel as a phoneme, i.e., [ʔɑ̟mɨ] ‘rain’ is /amɨ/ (not /ʔamɨ/), since the occurence of [ʔ] can be predicted by the phonological environments, i.e. a word-initial position preceding a vowel.

The minimal or quasi-minimal contrasts of consonants are shown below.

\ea Stops\\
\ea /t/ vs /tˀ/ vs /d/\\
\gll /tɨɨ/ vs /tˀɨɨ/ vs /dɨɨ/\\
‘hand’ {} ‘one (thing)’ {} /bamboo/\\
\ex /k/ vs /kˀ/ vs /g/\\
\gll /kuran/ vs /kˀura/ vs /gurusa/\\
‘Kuran’ {} ‘storehouse’ {} ‘fast’\\
\ex /kj/ vs /kˀj/\\
\gll /kjaaganaa/ vs /kˀjaa/\\
‘in coming’ {} ‘Kikai island’\\
\ex /p/ vs /t/ vs /k/\\
\gll /pɨɨ/ vs /tɨɨ/ vs /kɨɨ/\\
‘(ass)hole’ {} ‘hand’ {} ‘tree’\\
\ex /b/ vs /d/ vs /g/\\
\gll /baa/ vs /daa/ vs /gan/\\
‘No, thanks.’ {} ‘where’ {} ‘crab’\\
\z
\ex  Affricates and fricatives\\
\ea /c/ vs /z/ vs /s/\\
\gll /sɨcɨ/[sɨt͡sɨ] vs /sɨzɨ/[sɨ(d͡)zɨ] vs /sɨsɨ/[sɨsɨ]\\
‘coffin’ {} ‘tendon’ {} ‘soot’\\
\ex /cj/ vs /cˀj/\\
\gll /cjan/ [t͡ɕɑ̟ɴ] vs /cˀjan/[t͡ɕˀɑ̟ɴ]\\
‘coal tar’ {} ‘father’\\
\ex /s/ vs /h/\\
\gll /sɨɨsa/ vs /hɨɨsa/\\
‘sour’ {} ‘large’\\
\z
\ex  Nasals\\
\ea /m/ vs /mˀ/\\
\gll /mɨɨ/ vs /mˀɨɨ/\\
‘eye’ {}  ‘k.o.fruit’\\
\ex /n/ vs /nˀ/\\
\gll /njɨɨ/ vs /nˀjɨ/\\
‘load’ {}  ‘rice plant’\\
\ex /m/ vs /n/\\
\gll /mai/ vs /nai/\\
‘hip’ {}  ‘seed of cyad’\\
\z
\ex Approximants\\
\ea /w/ vs /wˀ/\\
\gll /waa/ vs /wˀaa/\\
‘my’ {}  ‘pig’\\
\ex /j/ vs /jˀ/\\
\gll /juu/ vs /jˀu/\\
‘boiled water’ {}  ‘fish’\\
\ex /w/ vs /j/\\
\gll /wɨɨ/ vs /jɨɨ/\\
‘tub’ {}  ‘handle’\\
\ex /r/ vs /d/\\
\gll /nuru/[nuɾu] vs /nudu/[nudu]\\
‘moss’ {}  ‘throat’\\
\z
\z

The minimal or quasi-minimal contrasts of geminates and single consonants are shown in \tabref{tab:2:9}.

\begin{table}
\caption{(Quasi-)minimal contrasts of geminates and single consonants\label{tab:2:9}}
\begin{tabular}{ *{5}{l} } 
\lsptoprule
& \multicolumn{2}{l}{Single} &  \multicolumn{2}{l}{Geminate} \\\midrule
/p/ & pocjoopocjo & ‘dripping’ & sippoo & ‘dull (sword)’\\
/b/ & cɨba  & ‘saliva’ & cɨbban & (copulate.NEG)\\
/t/ & utu   & ‘sound’ & uttui & ‘the day before yesterday’\\
/k/ & sikjan & (spread.NEG) &  sikkjan & (sink.NEG)\\
/g/ & hɨgu & ‘k.o. tree’ &  hɨggɨ &  ‘(place name)’\\
/c/ & ucja  & (put.PST)  & uccja  & (hit.PST)\\
/s/ & kusan & ‘k.o. bamboo’ & kussan & (kill.NEG)\\
/z/ & azjəə & (taste.TOP) &  azzjəə & ‘grandfather’\\
/m/ & hɨma & ‘spare time’ & hɨnma & ‘daytime’\\
/n/ & sɨna   & ‘sand’ &  sɨnna &  (do.PROH)\\
\lspbottomrule
\end{tabular}
\end{table}

Geminate in the right-side column includes the case of archiphoneme /N/ plus /n/ (or /m/) (see §\ref{bkm:Ref347178311}).

\subsubsection{Homorganic nasals}

/n/ and /m/ are homorganic nasals; that is, they assimilate with the place of the following consonants.

\begin{table}\footnotesize
\caption{Homorganic nasals\todo[inline]{transpose table}}
\begin{tabular}{ *{6}{l} } 
\lsptoprule
  & Isolation  &  Before bilabials &  Before alveolars&   Before velars &  Before vowels\\\midrule
 /n/  & un [ʔuɴ]  & un=ba [ʔum.bɑ̟] &  un=doo [ʔun.do̞ː]  & un=gadɨ [ʔuŋ.gɑ̟.dɨ]  & un=un [ʔu.nuɴ]\\
      &sea        & sea=ACC        &   sea=ASS          &  sea=LMT             & sea=also\\
 /m/  & N/A      & jum-boo [jum.bo̞ː]  & jum-cja [jun.t͡ɕɑ̟] &  jum-gadɨ [juŋ.gɑ̟.dɨ] &  jum-an [ju.mɑ̟ɴ]\\
      & read-CND &  read-want  & read-until &  read-NEG\\
\lspbottomrule
\end{tabular}
\end{table}

In these cases, the underlying forms of the root-final homorganic nasals, i.e., \textit{un} ‘sea’ or \textit{jum-} ‘read,’ can be hypothesized by making use of the phones preceding vowels, such as /un=un/ [ʔu.nuɴ] ‘sea=also’ and /jum-an/ [ju.mɑ̟ɴ] ‘read-NEG.’ However, we could not determine the underlying form of nasals that do not occur in morpheme boundaries, such as [ʔɑ̟m.mɑ̟ː] ‘mother,’ [tɨn.no̞ː.gi] ‘rainbow,’ and [iŋ.gɑ̟] ‘man.’ In these cases, we think these ostensible homorganic nasals are “archiphonemes” (\citealt{Lass1984}: 46-49, \citealt{Dixon2010}: 272). In this grammar, we use the letter \textit{n} for the orthographic representation of the archiphonemes, i.e., \textit{anmaa} ‘mother,’ \textit{tɨnnoogi} ‘rainbow,’ and \textit{jinga} ‘man’ (see also “Orthography” in the “Transcrption” in the beginning of this grammar).

\section{Syllable structure and phonotactics}
\label{bkm:Ref302599307}\hypertarget{RefHeadingToc395696969}{}\subsection{The syllable structure and morae}
\label{bkm:Ref301830963}\hypertarget{RefHeadingToc395696970}{}\label{bkm:Ref381399409}
Yuwan has the following syllable structures, and the corresponding morae are also shown. Parentheses indicate the slots are optional. In the syllables in Yuwan, the slot obligatorily filled by a phoneme is only V\textit{\textsubscript{1}}.

\begin{figure}
\caption{\todo[inline]{Please provide a caption}}
\begin{tabular}{cccc}
 (C\textit{\textsubscript{1}} & (G)) & V\textit{\textsubscript{1}} & (V\textit{\textsubscript{2}})  or  (C\textit{\textsubscript{2}})\\
 {}- & {}- & μ  & μ\\
\end{tabular}
\end{figure}

Notes: 
\begin{description}[font=\normalfont]
\item[C\textit{\textsubscript{1}}:] All consonants can fill this slot;
\item[G:] Only /w/ and /j/ can fill this slot;
\item[V\textit{\textsubscript{1}}:] All vowels can fill this slot;
\item[V\textit{\textsubscript{2}}:] The same vowel as V\textit{\textsubscript{1}} can fill this slot; /i/ can also fill this slot (see §\ref{bkm:Ref347178348});
\item[C\textit{\textsubscript{2}}:] Only /n/ can fill this slot at the final position of a phonological word; consonants, except for /h, r/, can fill this slot elsewhere.
\end{description}


Prosody tells us that V\textit{\textsubscript{1}} and V\textit{\textsubscript{2}} cannot be analyzed as /V\textit{\textsubscript{1}}.V\textit{\textsubscript{2}}/ (see §\ref{bkm:Ref301560567}). In addition, morphophonological behavior may also support this analysis (see §\ref{bkm:Ref367134300}). Both the syllable and mora are indispensable units in Yuwan.

There is a strong tendency for a phonological word to have two (or more) morae. The following words do not follow this tendency.

\ea 
\ea Nouns:\\
    /sja/ ‘below,’ /mja/\footnotemark{} ‘snail,’ /cˀju/ ‘person,’ /mˀa/ ‘horse,’ /jˀu/ ‘fish,’ /nˀjɨ/ ‘rice plant’
\ex Verbs:
\ea imperative forms: /mjɨ/ (see.IMP), /jˀɨ/ (say.IMP), /jˀɨ/ (sit.IMP), /njɨ/ (boil.IMP)
\ex past forms: /sja/ (do.PST), /cˀja/ (come.PST)
\ex sequential converbs:  /sjɨ/ (do.SEQ), /cˀjɨ/ (come.SEQ)
\z
\z
\z
\footnotetext{This word is pronounced as /mjaa/ with two morae by the speaker MT.}

It is probable that all of the examples had two syllables in the past considering their plausible counterparts in modern Japanese. Take, for example, the following nouns: /sita/ ‘below,’ /mina/ ‘snail’ (in old Japanese), /hito/ ‘person,’ /uma/ ‘horse,’ /iwo/ ‘fish’ (in old Japanese), and /ine/ ‘rice plant.’ Concerning verbs, it is difficult to do such a comparison. Nevertheless, all the plausible counterparts in Japanese have /i/ in the place of /j/ (or /jˀ/); for example, /sita/ (do.PST) and /kita/ (come.PST). Furthermore, there is a phenomenon which shows the bimoraic tendency applying to some verbal stems as if they were phonological words by themselves, i.e., the verbal stems preceding type D affixes (see the footnote Error: Reference source not found in §\ref{bkm:Ref347177096}).

\subsection{Phonotactics}
\label{bkm:Ref302599510}\hypertarget{RefHeadingToc395696971}{}
The following constraints (or tendencies) are determined from the behavior of monomorphemic and polymorphemic phonological words.

\ea Phonotactic constraints (or tendencies):\label{ex:2.8}
\ea Non-nasal resonants cannot be followed by approximants, i.e., /*rj/, /*jj/, and /*wj/ (see \sectref{sec:key:8.2.1.3});\label{ex:2.8a}
\ex Glottalized consonants can appear only at stem-initial positions (see below);\label{ex:2.8b}
\ex A sequence of consonants is geminate or its first consonant is nasal;\label{ex:2.8c}
\ex A monomorphemic word does not have voiced geminates (with the exception of the three lexemes /cɨbb/ ‘copulate,’ /azzjəə/ ‘grandfather,’ and /hɨggɨ/ ‘(place name)’). In addition, a phonological word made of polymorphemes tends to avoid voiced geminates (see §\ref{bkm:Ref347178914});\label{ex:2.8d}
\ex A monomorphemic word has a sequence with at most two vowels (with the exception of the three lexemes /jɨɨi/ ‘brother,’ /dooi/ ‘reason’ (sometimes pronounced as /doi/), and /tuuii/ ‘(place name)’); a phonological word made of polymorphemes tends to restrict a sequence made of three vowels (see §\ref{bkm:Ref301832441});\label{ex:2.8e}
\ex A monomorphemic word does not have the VVC\textsubscript{coda} sequence (with the exception of /koonmja/ ‘k.o. shellfish living in the river’\footnote{It creates a minimal pair with /konmja/ ‘a kind of shellfish living in the beach.’} and /sjoogoin/ ‘k.o. white radish,’ the latter thought to be a loan word from Modern Japanese); a phonological word made of polymorphemes tends to restrict the V\textit{\textsubscript{i}}V\textit{\textsubscript{i} }C\textsubscript{coda} sequence (see §\ref{bkm:Ref301832441});\label{ex:2.8f}
\ex A sequence of C\textsubscript{coda}.V never appears (see §\ref{bkm:Ref347173344});\label{ex:2.8g}
\ex A monomorphemic word does not have a sequence of a nasal coda followed by an onset /j/, i.e., */n.j/ and */m.j/; however, a phonological word consisting of more than one morpheme may have this sequence (see §\ref{bkm:Ref347179283});\label{ex:2.8h}
\ex The consonants that can precede /w/ filled in G slot are only /kˀ/, /k/ and /h/ (\tabref{tab:key:18} in \sectref{sec:key:2.3.2.5});\label{ex:2.8i}
\z
\z

Phonotactics determine the possible combinations of phonemes in a phonological word (see §\ref{bkm:Ref347179371}), and we have to pay attention to the following two types of sounds: (A) glottalized consonants, i.e., /Cˀ/ and (B) non-glottalized palatal approximant, i.e., /j/.

First, glottalized consonants can appear in a word-initial position such as \textit{jˀu} ‘fish,’ but cannot appear in a non-word-initial position in a simple word. For example, there is no word made of /VCˀV/; however, in the case of compounds, glottalized consonants can appear in a non-word-initial position, e.g., \textit{aa+jˀu} (red+fish) ‘red fish.’ In other words, glottalized consonants can appear in a stem-initial position. If we adopt the possibility of the occurence of glottalized consonants as a criterion of the phological word, there would be a mismatch among the criterion about glottalized consonants and that mentioned in \sectref{sec:key:2.1.} This type of mismatch between the criteria of phonological words, however, is not uncommon. In fact, \citet[18]{DixonAikhenvald2002} wrote that “(d)ifferent types of criteria are relevant to defining the phonological word in different languages. And the relative importance and weighting of criteria differ from language to language.” In this grammar, the possibility of the occurence of glottalized consonants is not adopted as the criterion of the phonological word, and I only mention its mismatch with other criteria.

Second, there are two types of morphemes beginning with /j/: one type palatalizes the preceding phoneme, as in (\ref{ex:2.9}a--b), while another type does not, as in (\ref{ex:2.9}c--e).

\ea Palatalization\\\label{ex:2.9}
\begin{tabular}{lllllllll}
   &   \multicolumn{2}{l}{Former}  & &  \multicolumn{2}{l}{Latter}   & &   {Latter}\\
a. & {\itshape jum-}  {‘read’}  {+}  {\itshape {}-jaa}  {‘person’}  {>}  {ju.mjaa [ju.mʲɑ̟ː]}  {Affix}\\
b. & {\itshape jum-}  {‘read’}  {+}  {\itshape {}-jagacinaa}  {(SIM)}  {>}  {ju.mja.ga.ci.naa [ju.mʲɑ̟.gɑ̟.t͡ɕi.nɑ̟ː]}  {Affix}\medskip\\
   & \multicolumn{3}{l}{Non-palatalization} \\
c. & {\itshape mun}  {(ADVRS)}  {+}  {\itshape jaa}  {(SOL)}  {>}  {mun.jaa [muɴ.jɑ̟ː]}  {Clitic}\\
d. & {\itshape jum-∅}  {(read-INF)}  {+}  {\itshape jass-sa}  {(easy-ADJ)}  {>}  {jum.jas.sa [juɴ.jɑ̟s.sɑ̟]}  {Root}\\
e. & {\itshape nɨkan}  {‘orange’}  {+}  {\itshape jama}  {‘mountain’}  {>}  {nɨ.kan.ja.ma [nɨ.kɑ̟ɴ.jɑ̟.mɑ̟]}  {Root}\\
\end{tabular}
\z

These examples show that if the following morpheme (the morphological status of the following morphemes is shown in the right-most column labeled “Latter”) is a clitic or a root, palatalization does not occur. However, if it is an affix, palatalization necessarily occurs. In this grammar, the syllable boundary between /m/ and /j/ in \textit{jum-∅+jass-sa} (read-INF+easy-ADJ) ‘easy to read’ is expressed by a period mark such as /jum.jassa/ in the surface form level.

\subsubsection{Monosyllabic words}
\begin{table}[H]
\caption{Monosyllabic (and monomorphemic) grammatical words}
\begin{tabular}{l@{ }l@{ }llccc} 
\lsptoprule
                 & &   & C  & G &  V  &  V (or C)\\\midrule
/ai/ & [ʔɑ̟i] & ‘No’    &    &   &   a  & i\\
/an/ & [ʔɑ̟ɴ] & ‘that’  &    &   & a   & n\\
/jaa/ & [jɑ̟ː] & ‘house’  & j &  &  a &  a\\
/wan/ & [wɑ̟n] & ‘I’  & w  & &  a &  n\\
/naa/ & [nɑ̟ː] & ‘name’  & n  & &  a &  a\\
/mja/ & [mʲɑ̟] & ‘k.o.shellfish’  & m &  j &  a  \\
/mjaa/ & [mʲɑ̟ː] & ‘cat’ &  m &  j &  a &  a\\
/nan/ & [nɑ̟ɴ] & ‘you.HON’  & n  &  &  a &  n\\
/cjan/ & [tɕɑ̟ɴ] & ‘coal tar’  & c  & j  & a &  n\\
/mˀa/ & [Ɂmɑ̟] & ‘horse’  & mˀ  & &  a & \\
/wˀaa/ & [Ɂwɑ̟ː] & ‘pig’  & wˀ &  &  a &  a\\
/kˀjaa/ & [kˀʲɑ̟ː] & ‘Kikai island’ &  kˀ &  j &  a &  a\\
/cˀjan/ & [t͡ɕˀɑ̟ɴ] & ‘father’  & cˀ &  j &  a &  n\\
\lspbottomrule
\end{tabular}
\end{table}

\subsubsection{Polysyllabic phonological words}
\label{bkm:Ref347179283}\hypertarget{RefHeadingToc395696973}{}\label{bkm:Ref347178311}
In principle, the phonotactics of polysyllabic phonological words are the same as those of monomorphemic ones, but there is an important difference in terms of the phonemes that can fill coda slots. In monosyllabic words, the coda slots in word-final position can only be filled by /n/. However, in polysyllabic words, the coda slots in word-internal position can be filled by many kinds of consonants. The possible combinations of consonants around a syllable boundary are shown below, including the total number of monomorphemic lexemes that have such a sequence (out of approximately 1,000 lexemes). In the following table, /N/ indicates the archiphoneme (see also “Transcription” in the beginning of this grammar and §\ref{bkm:Ref347174390} for more details).

\begin{sidewaystable}
\caption{/C.C/ combination in polysyllabic phonological words (monomorphemic)}
\begin{tabular}{ *{10}{l} S[table-format=1] }
\lsptoprule
\multicolumn{4}{l}{} &   C & G & V & C  & .C  & &  {Number}\\\midrule
{/p.p/:}     & {/sip.poo/}       & [ɕip.po̞ː]          &  {‘blunt’}             &  s  &  &    i  &  p   &  .p  & oo & 6\\
{/b.b/:}     & {/cɨb.bi.da.ci/}  & [t͡sɨb.bi.dɑ̟.t͡ɕi] &  {‘rut (of animal)’}   &  c  &  &    ɨ  &  b   &  .b  & idaci &  1\\
{/t.t/:}     & {/at.ta.kəə/}     & [ʔɑ̟ttɑ̟kɜː]        &   {‘everything’}       & &  & a   &     t &  .t  &   akəə  & 16\\
{/k.k/:}     & {/juk.ka.dɨ/}     & [jukkɑ̟dɨ]          &  {‘throughout’}        &  j  &  &    u  &  k   &  .k &  adɨ & 14\\
{/g.g/:}     & {/hɨg.gɨ/}        & [xɨggɨ]             & {‘(place name)’}       & h   &  &   ɨ   & g    & .g &  ɨ &  1\\
{/c.c/:}     & {/gac.cɨn/}       & [gɑ̟tt͡sɨɴ]         &  {‘saurel’}            &  g  &  &    a  &  c   &  .c &  ɨn &  7\\
{/s.s/:}     & {/kas.sa/}        & [kɑ̟ssɑ̟]           &   {‘like this’}        &   k &  &     a &   s  &   .s &  a &  9\\
{/z.z/:}     & {/az.zjəə/}       & [ʔɑ̟dd͡ʑɜː]         &  {‘grandfather’}      & & &  a   &    z  & .z   &  jəə &  1\\
{/N/ + /p/:} & {/an.pəə/}        & [ʔɑ̟m.pɜː]          &  {‘appearance’}       & & & a   &    n  & .p   &  əə &  2\\
{/N/ + /b/:} & {/gan.boo/}       & [gɑ̟m.bo̞ː]         &   {‘naughty boy/girl’} &   g &   &     a &   n  &   .b & oo &  1\\
{/N/ + /t/:} & {/nin.təə/}       & [nin.tɜː]           & {‘group’}              & n   & &   i   & n    & .t  & əə & 2\\
{/N/ + /d/:} & {/cɨn.dai/}       & [t͡sɨn.dɑ̟i]        &  {‘snail’}             &  c  &  &    ɨ  &  n   &  .d & ai & 7\\
{/N/ + /k/:} & {/in.ku.zjaa/}    & [ʔiŋ.ku.(d͡)ʑɑ̟ː]   &  {‘(place name)’}      & & & i &    n  & .k   &  uzjaa & 5\\
{/N/ + /g/:} & {/jin.ga/}        & [iŋ.gɑ̟]            &  {‘man’}               &  j  & &    i  &  n   &  .g & a & 10\\
{/N/ + /c/:} & {/kan.cɨmɨ/}      & [kɑ̟n.t͡sɨ.mɨ]      &  {‘(name of person)’}  &  k  & &    a  &  n   &  .c & ɨmɨ & 1\\
{/N/ + /s/:} & {/han.sɨ/}        & [hɑ̟ɴ.sɨ]           &  {‘sweet potato’}      &  h  & &    a  &  n   &  .s & ɨ & 4\\
{/N/ + /z/:} & {/hin.zjaa/}      & [çin.(d͡)ʑɑ̟ː]      &  {‘goat’}              &  h  & &    i  &  n   &  .z & jaa & 5\\
{/N/ + /m/:} & {/an.maa/}        & [ʔɑ̟m.mɑ̟ː]         &   {‘mother’}           &  &   &a   &     n &   .m  &  aa & 8\\
{/N/ + /n/:} & {/han.njəə/}      & [hɑ̟n.njɜː]         &  {‘grandmother’}       &  h  &  &    a  &  n    & .n & jəə & 6\\
\lspbottomrule
\end{tabular}
\end{sidewaystable}

There are no monomorphemic words with the sequences of /dd/, /hh/, or /rr/ in Yuwan. The data show that the number of monomorphemic lexemes that have C\textsubscript{coda}.C\textsubscript{onset} sequences are very small; however, this sequence is not uncommon in the case of polymorphemic phonological words, such as \textit{ar-} ‘exist’ + \textit{doo} (ASS) > /at.too/ and \textit{ar} ‘exist’ + \textit{ba} (CSL) > /ap.pa/. These sequences are formed by the (morpho)phonological rules (see §\ref{bkm:Ref347178914} and §\ref{bkm:Ref347177096}). In monomorphemic words, it is impossible to determine the (morpho)phoneme of the nasal that fills the C\textsubscript{coda} slot in the C\textsubscript{coda}.C sequence, but it is possible to do so in polymorphemic phonological words, as shown below.

\begin{table}\footnotesize
\caption{/Nasal + C/ combination in polysyllabic phonological words (polymorphemic)}
\begin{tabular}{ *{10}{l} }
\lsptoprule
 \multicolumn{4}{l}{} &  {C} & {G} & {V} & {C} & {.C} & {}\\\midrule
{/m.b/:} & {/jum.ba/}     & [jum.bɑ̟]      &  {(read.CSL)}          &   j   & &  u & m & .b &a\\
{/m.d/:} & {/jum.doo/}    & [jun.do̞ː]     &  {(read.INF.ASS)}      &   j   & &  u & m & .d &oo\\
{/m.k/:} & {/kam.kai/}    & [kɑ̟ŋ.kɑ̟i]    &   {(eat.DUB)}          &    k  & &   a&  m&  .k& ai\\
{/m.g/:} & {/jum.ga.dɨ/}  & [juŋ.gɑ̟.dɨ]   &  {(read.until)}        &   j   & &  u & m & .g &adɨ\\
{/m.c/:} & {/jum.cja.sa/} & [jun.t͡ɕɑ̟.sɑ̟]&   {(read.INF.want.ADJ)}&    j  &  &   u&  m&  .c& jasa\\
{/m.n/:} & {/jum.nja/}    & [junʲ.nʲɑ̟]    &  {(read.INF.TOP)}      &   j   & &  u & m & .n &ja\\
{/m.j/:} & {/jum.jas.sa/} & [juɴ.jɑ̟s.sɑ̟] &   {(read.INF.easy.ADJ)}&    j  & &   u&  m&  .j& assa\\
{/n.b/:} & {/nɨ.kan.ba/}  & [nɨ.kɑ̟m.bɑ̟]  &   {(orange.ACC)}       &    nɨ.   k  & &  a & n & .b & a\\
{/n.t/:} & {/nan.tu/}     & [nɑ̟n.tu]      &  {(you.HON.COM)}       &   n &   &  a  &  n& .t& u\\
{/n.d/:} & {/kin.du/}     & [kˀin.du]      & {(clothes.FOC)}        &  k &   & i   & n &.d &u\\
{/n.k/:} & {/un.ka.ci/}   & [ʔuŋ.kɑ̟.t͡ɕi] &  {(sea.ALL)}           & & &   u   &  n  &  .k & aci\\
{/n.g/:} & {/wan.ga/}     & [wɑ̟ŋ.gɑ̟]     &   {(1SG.NOM)}          &    w  & &   a &    n & .g & a\\
{/n.n/:} & {/wan.na/}     & [wɑ̟n.nɑ̟]     &   {(1SG.TOP)}          &    w  & &   a &   n & .n & a\\
{/n.j/:} & {/mun.jaa/}    & [muɴ.jɑ̟ː]     &  {(ADVRS.SOL)}         &   m   & &  u  &  n & .j & aa\\
\lspbottomrule
\end{tabular}
\end{table}

As mentioned in (\ref{ex:2.8}h) in §\ref{bkm:Ref302599510}, a sequence of C\textsubscript{coda}.C\textsubscript{onset} (C\textsubscript{coda} is nasal, C\textsubscript{onset} is /j/) never appears in monomorphemic grammatical words; however, it can appear in polymorphemic phonological words (see the examples of /m.j/ and /n.j/ above). There are four morphemes able to make this sequence: \textit{jass} ‘easy,’ \textit{jaa} (SOL), \textit{joo} (CFM1), and \textit{jukkuma} (CMP).

\subsubsection{Glottalized consonants}
\label{bkm:Ref347180773}\hypertarget{RefHeadingToc395696974}{}
Phonologically, glottalized consonants are contrastive only at stem-initial positions. Phonetically, they require laryngeal intension and may be divided into two types: glottalized obstruents [tˀ, t͡ɕˀ, kˀ] and glottalized sonorants [ʔm, ʔn, ʔj, ʔw]. The former group sounds like unaspirated obstruents in Chinese or unaspirated tense obstruents in Korean, and a more detailed phonetic comparison should be done in the future. The latter group has the following two characteristics (compared with non-glottalized sonorants [m, n, j, w]): \REF{ex:key:1} relatively larger amplitude in the onset, \REF{ex:key:2} relatively shorter duration in the onset, which indicates their coarticulation with the glottal stop in the onset position \citep{NiinagaEtAl2011}. Word initial /p/, /cɨ/, and /ki/ are basically phonetically glottalized, and they appear to have developed from historical changes (cf. \citealt{HirayamaEtAl1966}: 22-23), but the details of their development are beyond the scope of this grammar.

Glottalized consonants are proposed to have developed from two phonological processes: \REF{ex:key:1} syllable omission and \REF{ex:key:2} retainment of a distinction affected by vowel merger (\citealt{HirayamaEtAl1966}: 22-23). An example of the former is */hutari/ > /tˀai/ ‘human’ (/ri/ > /i/ is also a synchronic phonological rule in §\ref{bkm:Ref304225942}). An example of the latter is */kome/ > /kumɨ/, and */kura/ > /kˀura/, where */o/ is merged with */u/ and both become /u/ (the change of */e/ > /ɨ/ is another historical change that is not addressed here). Previous research has shown that */ku/ became /kˀu/ in order to retain a difference from /ku/ (made of */ko/) (\citealt{HirayamaEtAl1966}: 23). Almost all of the current tokens of /kˀ/ in Yuwan have developed from */ku/. Additionally, /kˀjaa/ [kˀʲɑ̟ː] ‘Kikai-zima,’ which is the name of an island, appears to have developed from syllable omission. There are a number of lexicon that has /kˀ/ in modern Yuwan. The other glottalized phonemes seem to have developed as a result of syllable omission. This process does not seem to have been common, so there are only a few lexemes that have these glottalized phonemes. The following table shows the number of lexemes that have word-initial glottalized phonemes (and their examples) compared with that of non-glottalized initial phonemes.

\begin{table}
\caption{Lexemes that have word-initial glottalized phonemes (out of approximately 1,000 lexemes)}
\begin{tabularx}{\textwidth}{ ll S[table-format=1] Q l S[table-format=2]}
\lsptoprule
{Phonemes} & {Allophones} & {Number} & Examples   &    cf.  & {Number}\\\midrule
{/wˀ/}  & {[ʔw]}     & 2  & [ʔwɑ̟ː]  ‘pig’         \newline\relax  [ʔwɑ̟bijɑ̟ː]  ‘instep’ & /w/  & 18\\
{/tˀ/}  & {[tˀ]}     & 3  & [tˀɑ̟i]  ‘two persons’ \newline\relax  [tˀɨɨ]  ‘one thing’  & /t/  & 59\\
{/nˀj/} &  {[ʔnʲ]}   & 3  & [ʔnʲut͡ɕi]  ‘life’    \newline\relax [ʔnʲɨ]  ‘rice plant’ & /nj/ & 2\\
{/kˀj/} &  {[kˀʲ]}   & 5  & [kˀʲɑ̟ː]  ‘Kikai-zima’ \newline\relax  [kˀʲubiː]  ‘band’    & /kj/ & 7\\
{/mˀ/}  & {[ʔm]}     & 4  & [ʔmɑ̟]  ‘horse’        \newline\relax  [ʔmɑ̟t͡sɨ]  ‘fire’    & /m/  & 96\\
{/cˀj/} &  {[t͡ɕˀ]}  & 5  & [t͡ɕˀɑ̟ɴ]  ‘father’    \newline\relax  [t͡ɕˀu]  ‘person’    & /cj/ & 5\\
{/jˀ/}  & {[ʔj]}     & 5  & [ʔju]  ‘fish’         \newline\relax [ʔjɑ̟]  ‘arrow’       & /j/  & 63\\
{/kˀ/}  & {[kˀ]}     & 35 & [kˀubi]  ‘neck’       \newline\relax [kˀuru(ː)]  ‘black’  & /k/  & 81\\\midrule
\end{tabularx}\smallskip\\
\raggedright
\textit{Note:}\\
\begin{enumerate}[label=\alph*.,nosep]
\item The number of /C\textit{\textsubscript{i} }/ and /C\textit{\textsubscript{i} }j/ is not redundant. For example, the number of /k/ excluded the number of /kj/;
\item The number of lexemes that have non-glottalized initial /k/ excludes that of /ki/ [kˀi].
\end{enumerate}
\begin{tabularx}{\textwidth}{X}\lspbottomrule\end{tabularx}
\end{table}

The data show there are fewer lexemes that have word-initial glottalized phonemes than non-glottalized ones; however, the number of lexemes with /Cˀj/ and /Cj/ does not follow this pattern. In fact, the number of combinations where a consonant is followed by /j/ in these examples is relatively small, so it is not meaningful to compare these particular consonants.

Since there are fewer lexemes that have word-initial glottalized phonemes than non-glottalized ones, we propose that the former are “marked” phonemes. Therefore, if a “phonetically” word-initial glottalized consonant does not have a “phonemic” contrast with a non-glottalized one, we regard it as a “phonemically non-glottalized” phoneme. For example, Yuwan has only [pˀ], but this phoneme is interpreted as /p/ in this grammar. Moreover, there are no word-internal contrasts with glottalization in Yuwan, so word-internal phonemes are always phonemically non-glottalized even if they might be phonetically glottalized (with the exception of the case of compounds, see §\ref{bkm:Ref302599510}). The combination of velar stop and /w/ is always realized as [kˀʷ], but we will interpret it as /kˀw/ with the exception of the case of \textit{{}-kkwa} (DIM) and /joikwa/ ‘silently’ (see \sectref{sec:key:7.7}) against the markedness principle because the interpretation as /kˀw/ makes it easier to explain a prosodic phenomenon discussed in §\ref{bkm:Ref347180628}.

\subsubsection{Interpretation of /C/ + /j/ combination}
\label{bkm:Ref347180720}\hypertarget{RefHeadingToc395696975}{}
Yuwan has a contrast between [ɕ] and [s]: [kɑ̟ɕɑ̟] ‘wrapping leaf’ vs. [kɑ̟sɑ̟] ‘bamboo hat.’ In this grammar, [ɕ] is interpreted as /sj/ (except for the case of [ɕi]\footnote{[ɕi] is regarded as /si/ (not */sji/) to keep the full set of combinations with /s/ and vowels, since /CV/ is a more productive combination than /CjV/. For example, /b/ can precede any vowel, but /bj/ can only precede /a/ and /u/ (see §\ref{bkm:Ref347180694}).}). There are two reasons why we do not assign a new phoneme /ɕ/: \REF{ex:key:1} the overall number of phonemes, and \REF{ex:key:2} morphology.

First, we do not need another new phoneme if we interpret [ɕ] as /sj/, so this interpretation is more economical than the other.

Second, Yuwan has an affix \textit{{}-jaa} ‘person,’ which can nominalize verbal roots (see \sectref{sec:key:7.6}). For example, if the affix follows \textit{hɨmɨkas-} ‘get drunk,’ it becomes [xɨmɨkɑ̟ɕɑ̟ː] ‘drunken person.’ In this case, there would be two interpretations: \REF{ex:key:1} /hɨmɨkasjaa/, or \REF{ex:key:2} /hɨmɨkaɕaa/. The first interpretation is transparent, but the second is not because it needs an alternation rule, i.e., //s// + //j// > /ɕ/. The affix \textit{{}-jaa} is fairly productive, such as \textit{tug-} ‘whet’ + \textit{{}-jaa} ‘person’ > /tugjaa/ [tugʲɑ̟ː] ‘a person who whet cutlery professionally’ and \textit{kik-} ‘hear’ + \textit{{}-jaa} ‘person’ > /kikjaa/ [kikʲɑ̟ː] ‘audience.’ Thus, it is (paradigmatically) natural to regard [ɸumukɑ̟ɕɑ̟ː] as /humukasjaa/. Therefore, we adopt the interpretation of [ɕ] as /sj/ in Yuwan (cf., \citet[79-81]{Shimoji2008} for a similar argument in Irabu Ryukyuan).

The same argument can be applied to /cj/ [t͡ɕ]: \textit{ut-} ‘hit’ + \textit{{}-jaa} ‘person’ > /ucjaa/ [ʔut͡ɕɑ̟ː] ‘a person who plays a role to hit someone,’ where an alternation rule from //t// to /c/ is applied (see §\ref{bkm:Ref347180796} for more details). In this case, the merit of regarding [t͡ɕ] not as a new phoneme but as a combination of two existing phonemes remains to be valid. Yuwan has no verbal roots that end with /z/, but there is no reason to treat /zj/ differently from /cj/, so we interpret [d͡ʑ] as /zj/.

\subsubsection{Combination of consonants and vowels}
\hypertarget{RefHeadingToc395696976}{}\label{bkm:Ref347177989}\label{bkm:Ref347180694}\label{bkm:Ref347181003}
The combinations of consonants and vowels, followed by examples, are shown in the following tables.

Pre-notes: 

\begin{enumerate}[label=\alph*.]
\item It might be possible to find combinations for the blank cells, but they have not yet been found so far.
\item If a plausible phonetic combination in one cell (e.g., /t/ + /ja/ > [t͡ɕɑ̟]) is regarded as a combination in another cell (e.g., /cja/), it will be shown in this way “[t͡ɕɑ̟]=/cja/” (cf. §\ref{bkm:Ref347180720}).
\item N/A means such a combination is prohibited by either phonological rules (see §\ref{bkm:Ref302723494}) or the syllable structure (see §\ref{bkm:Ref301830963}).
\item Parenthesized phones mostly appear in stem-initial position (cf. §\ref{bkm:Ref347180773}).
\item Glottalization of the second phoneme of a geminate is not taken into consideration.
\end{enumerate}

\begin{sidewaystable}
\caption{Combinations of CV and CjV showing allophones}
\footnotesize
\begin{tabular}{@{}*{12}{l@{\hspace{1em}}}l@{}} 
\lsptoprule
 & a & i & u & ɨ & ə & o & ja & ji & ju & jɨ & jə & jo\\\midrule
\textminus\footnote{This means there is no consonant in the onset C slot.} & [(ʔ)ɑ̟] & [(ʔ)i] & [(ʔ)u] & [(ʔ)ɨ] & [(ʔ)ɜ] & [(ʔ)o̞] & N/A & N/A & N/A & N/A & N/A & N/A\\
p & [p(ˀ)ɑ̟] & [pʲ(ˀ)i] & [p(ˀ)u] & [p(ˀ)ɨ] & [p(ˀ)ɜ] & [p(ˀ)o̞] & [p(ˀ)ʲɑ̟] &  & [p(ˀ)ʲu] &  &  & \\
b & [bɑ̟] & [bʲi] & [bu] & [bɨ] & [bɜ] & [bo̞] & [bʲɑ̟] &  & [bʲu] &  &  & \\
t & [tɑ̟] & [t͡ɕi]=/ci/ & [tu] & [tɨ] & [tɜ] & [to̞] & [t͡ɕɑ̟]=/cja/ &  & [t͡ɕu]=/cju/ & [t͡ɕɨ]=/cjɨ/ & [t͡ɕɜ]=/cjə/ & [t͡ɕo̞]=/cjo/\\
tˀ & [tˀɑ̟] &  &  & [tˀɨ] &  & [tˀo̞] &  &  &  &  &  & \\
d & [dɑ̟] & [d͡ʑi]=/zi/ & [du] & [dɨ] & [dɜ] & [do̞] & [d͡ʑɑ̟]=/zja/ &  & [d͡ʑu]=/zju/ & [d͡ʑɨ]=/zjɨ/ & [d͡ʑɜ]=/zjə/ & [d͡ʑo̞]=/zjo/\\
k & [kɑ̟] & [kʲ(ˀ)i] & [ku] & [kɨ] & [kɜ] & [ko̞] & [kʲɑ̟] &  & [kʲu] & [kʲɨ] &  & [kʲo̞]\\
kˀ &  & [kʲˀi]=/ki/ & [kˀu] &  &  &  & [kˀʲɑ̟] &  & [kʲu] &  &  & [kˀʲo̞]\\
g & [gɑ̟] & [gʲi] & [gu] & [gɨ] & [gɜ] & [go̞] & [gʲɑ̟] &  & [gʲu] & [gʲɨ] &  & [gʲo̞]\\
c &  & [t͡ɕ(ˀ)i] & [t͡su] & [t͡s(ˀ)ɨ] & [t͡sɜ] &  & [t͡ɕɑ̟] &  & [t͡ɕu] & [t͡ɕɨ] & [t͡ɕɜ] & [t͡ɕo̞]\\
cˀ &  & [t͡ɕˀi]=/ci/ &  & { [t͡sˀɨ]=/cɨ/} &  &  & [t͡ɕˀɑ̟] &  & [t͡ɕˀu] & [t͡ɕˀɨ] & [t͡ɕˀɜ] & [t͡ɕˀo̞]\\
s & [sɑ̟] & [ɕi] & [su] & [sɨ] & [sɜ] & [so̞] & [ɕɑ̟] &  & [ɕu] & [ɕɨ] & [ɕɜ] & [ɕo̞]\\
z & [(d͡)zɑ̟] & [(d͡)ʑi] &  & [(d͡)zɨ] & [(d͡)zɜ] &  & [(d͡)ʑɑ̟] &  & [(d͡)ʑu] & [(d͡)ʑɨ] & [(d͡)ʑɜ] & [(d͡)ʑo̞]\\
h & [hɑ̟] & [çi] & [ɸu] & [xɨ] & [hɜ] & [ho̞] &  &  & [çu] &  &  & \\
m & [mɑ̟] & [mʲi] & [mu] & [mɨ] & [mɜ] & [mo̞] & [mʲɑ̟] & [mʲi] & [mʲu] & [mʲɨ] &  & [mʲo̞]\\
mˀ & [ʔmɑ̟] &  &  & [ʔmɨ] &  & [ʔmo̞] &  &  &  &  &  & \\
n & [nɑ̟] & [nʲi] & [nu] & [nɨ] & [nɜ] & [no̞] & [nʲɑ̟] &  & [nʲu] & [nʲɨ] & [nʲɜ] & [nʲo̞]\\
nˀ &  &  &  &  &  &  &  &  & [ʔnʲu] & [ʔnʲɨ] & [ʔnʲɜ] & \\
w & [wɑ̟] & N/A & [wu] & [wɨ] & [wɜ] & [wo̞] & N/A & N/A & N/A & N/A & N/A & N/A\\
wˀ & [ʔwɑ̟] &  &  &  &  &  &  &  &  &  &  & \\
j & [jɑ̟] & [i] & [ju] & [jɨ] & [jɜ] & [jo̞] & N/A & N/A & N/A & N/A & N/A & N/A\\
jˀ & [ʔjɑ̟] & [ʔi] & [ʔju] & [ʔjɨ] &  & [ʔjo̞] & N/A & N/A & N/A & N/A & N/A & N/A\\
r & [ɾɑ̟] & N/A & [ɾu] & [ɾɨ] & [ɾɜ] & [ɾo̞] & N/A & N/A & N/A & N/A & N/A & N/A\\
\lspbottomrule
\end{tabular}
\end{sidewaystable}

\begin{sidewaystable}
\caption{Examples of CV}
\footnotesize
\begin{tabular}{@{} l@{\hspace{.75em}} *{5}{l@{ }l@{\hspace{.75em}}} l@{ }l@{}} 
\lsptoprule
  & a &  & i &  & u &  & ɨ &  & ə &  & o & \\\midrule
\textminus & aasa & ‘red’ & isi & ‘stone’ & uma & ‘there’ & ɨn & ‘dog’ & əəcɨrɨ & ‘classmate’ & oonazi & ‘k.o.sneak’\\
p & gappaa & ‘fist’ & piri & ‘tail end’ & roppu & ‘rope’ & pɨɨ & ‘(ass)hole’ & anpəə & ‘state’ & ponwata & ‘big belly’\\
b & naba & ‘mushroom’ & bija & ‘leek’ & habu & ‘k.o. snake’ & warabɨ & ‘child’ & ɨbəəsa & ‘narrow’ & zɨboo & ‘tail’\\
t & tanɨ & ‘seed’ &  &  & tui & ‘bird’ & tɨn & ‘sky’ & nɨntəə & ‘members’ & bottobotto & ‘lazily’\\
tˀ & tˀai & ‘two people’ &  &  &  &  & tˀɨɨ & ‘one’ &  &  & tˀoomu.nii & ‘Tsutomu’\\
d & kada & ‘smell’ &  &  & dusi & ‘friend’ & dɨru & ‘which’ & kjoodəə & ‘brother’ & dookunɨɨ & ‘white radish’\\
k & kabi & ‘paper’ & kin & ‘clothes’ & kuma & ‘here’ & kɨɨ & ‘tree’ & kəənja & ‘arm’ & koo & ‘skin’\\
kˀ &  &  &  &  & kˀura & ‘storehouse’ &  &  &  &  &  & \\
g & gan & ‘crab’ & ginməə & ‘contract’ & wunagu & ‘woman’ & hagɨr & ‘bald’ & kugəər & ‘tumble’ & kagoo & ‘basket’\\
c &  &  & cikjara & ‘power’ & cubusi & ‘knee’ & cɨmɨ & ‘nail’ & miicəə & (three.TOP) &  & \\
s & sataa & ‘sugar’ & siju & ‘soup’ & sura & ‘treetop’ & sɨba & ‘tongue’ & səə & ‘alcohol’ & soo & ‘stem’\\
z & sijuzataa & ‘white sugar’ & ziju & ‘cooking stove’ &  &  & kazɨ & ‘wind’ & kazəə & (wing.TOP) &  & \\
h & hana & ‘nose’ & hindjaa & ‘goat’ & hunɨ & ‘ship’ & hɨnma & ‘day’ & həəsa & ‘quick’ & hoorasja & ‘happy’\\
m & mamɨ & ‘bean’ & min & ‘ear’ & munɨ & ‘breast’ & mɨzɨ & ‘water’ & məə & ‘front’ & umoor & (move.HON)\\
mˀ & mˀa & ‘horse’ &  &  &  &  & mˀɨɨ & ‘k.o. fruit’ &  &  & mˀoo & (horse.TOP)\\
n & nama & ‘now’ & nissja & ‘similar’ & nudu & ‘throat’ & nɨzɨn & ‘mouse’ & junəə & ‘evening’ & noo & ‘fishing line’\\
w & wan & ‘I’ &  &  & wutu & ‘husband’ & wɨɨ & ‘tub’ & juwəə & ‘celebration’ & tawoo & (plain.TOP)\\
wˀ & wˀaa & ‘pig’ &  &  &  &  &  &  &  &  &  & \\
j & jama & ‘mountain’ & jinga & ‘man’ & juru & ‘night’ & jɨɨ & ‘grip’ & kawajəə & ‘substitute’ & joikwa & ‘silently’\\
jˀ & jˀa & ‘arrow’ & jˀii & (say.INF) & jˀu & ‘fish’ & jˀɨ & (say.IMP) &  &  & jˀoo & (say.INT)\\
r & warabɨ & ‘child’ &  &  & dɨru & ‘which’ & kurɨ & ‘this’ & kurəə & (this.TOP) & sɨroo & ‘lie’\\
\lspbottomrule
\end{tabular}
\end{sidewaystable}

\begin{sidewaystable}
\caption{Examples of CjV}
\footnotesize
\begin{tabular}{@{}l @{\hspace{.5em}}  *{5}{l@{ }l@{\hspace{.5em}}} l@{ } l@{}} 
\lsptoprule
 & ja &  & ji &  & ju &  & jɨ &  & jə &  & jo & \\\midrule
p & appjaganaa & (play.SIM) &  &  & appjur & (play.UMRK) &  &  &  &  &  & \\
b & jurukubjaganaa & (glad.SIM) & & &  asɨbjur & (play.UMRK) &  &  &  &  &  & \\
k & kjaaganaa & (come.SIM) &  &  & kjuu & ‘today’ & ikjɨ & (go.IMP) &  &  & kjoodəə & ‘brother’\\
kˀ & kˀjaa & ‘Kikai-zima’ &  &  & kˀjubii & ‘band’ &  &  &  &  & kˀjoos & ‘break’\\
g & asigja & ‘k.o. sandal’ &  &  & higjussa & ‘cold’ & uigjɨ & (swim.IMP) &  &  & uigjoo & (swim.INT)\\
c & cjaa & ‘tea’ &  &  & cjukaa & ‘kettle’ & kacjɨ & (write.SEQ) & məəhucjəə & ‘forehead’ & cjoo & ‘just’\\
cˀ & cˀjan & ‘father’ &  &  & cˀju & ‘person’ & cˀjɨ & (come.SEQ) & cˀjəəra & (come.SEQ.after) & cˀjoo & (person.TOP)\\
s & sja & ‘below’ &  &  & sjuukɨɨ & ‘feast’ & sjɨ & (do.SEQ) & kasjəə & ‘help’ & isjoobiki & ‘whistle’\\
z & zjaraa & ‘piggyback’ &  &  & zjuu & ‘father’ & izjɨ & (go.SEQ) & azzjəə & ‘grandfather’ & zjootoo & ‘good’\\
h &  &  &  &  & hjuusɨ & ‘bulbul’ &  &  &  &  &  & \\
m & mjaa & ‘cat’ & mjicja & (see.PST)  & mjuuna & (see.PROH) & mjɨ & (see.IMP) &  &  & mjoo & (see.INT)\\
n & kəənja & ‘arm’ &  &  & kinju & ‘yesterday’ & njɨɨ & ‘load’ & hannjəə & ‘grandmother’ & anjoo & ‘elder brother’\\
nˀ &  &  &  &  & nˀjuci & ‘life’ & nˀjɨ & ‘rice plant’ & nˀjəə & (rice.plant.TOP) &  & \\
\lspbottomrule
\end{tabular}
\end{sidewaystable}


\begin{table}
\caption{Combinations of CwV showing allophones\label{bkm:Ref365009143}}
\begin{tabularx}{\textwidth}{lQQQQ}
\lsptoprule
   & wa &  wo &  wɨ & wə\\\midrule
kˀ &  [kˀʷɑ̟] & [kˀʷo̞]   & & [kˀʷɜ]\\
k  &  [kʷɑ̟]      \\
h  &          &      & [φɨ]  \\
\lspbottomrule
\end{tabularx}
\end{table}


\begin{table}
\caption{Examples of CwV}
\begin{tabularx}{\textwidth}{lllQl}
\lsptoprule
& {wa}    & {wo}   & {wɨ}   & {wə}  \\\midrule
{kˀ} & {kˀwa}  {‘child’} & {kˀwoo}  {(child.TOP)}  &    & {kˀwəər}  {‘get fat’}\\
{k} & {joikwa}  {‘silently’}            \\
{h}  &                       &                        & {hwɨɨ}  {‘fart’}    \\
\lspbottomrule
\end{tabularx}
\end{table}

\section{Phonological rules}
\hypertarget{RefHeadingToc395696977}{}\label{bkm:Ref302723494}
Every phonological rule is applied at the morpheme boundaries within phonological words (see §\ref{bkm:Ref347179371}). In this grammar, the following dimensions are distinguished: phonetic, phonological (surface level), and morphophonemic (underlying level). Possible phonetic realization was shown in §\ref{bkm:Ref347181003}, the details of which are beyond the scope of this grammar. Thus, what is called the ‘surface’ level in this grammar represents the phonological level, and the ‘underlying’ level represents the morphophonemic level, against the Bloomfieldians’ convention of merging phonetic and phonological levels (cf. \citealt{Lass1984}: 59-62). The morphophonemic level is abstracted from the information about the morphosyntactic (i.e. paradigmatic and syntagmatic) variation of lexemes. In other words, surface variations of phonemes (i.e. allomorphs) are synthesized into abstract morphophonemes, which are determined by the following criteria: (1) phonemes that are not affected by assimilation, (2) phonemes that are relatively unrestricted by the phonological environments (e.g., the environment before vowels is regarded as “relatively unrestricted” in this grammar), or (3) phonemes that are unmarked cross-linguistically (e.g., oral is more unmarked than nasal, etc.). Needless to say, phonemes at the surface level are considered to contrast with one another, which is different from the variation at the phonetic level.

There are phonological rules and morphophonological rules, both of which are applied within phonological words (see §\ref{bkm:Ref347179371}). The phonological rules are not affected by the surrounding morphosyntactic or lexical information; however, this information is necessary for morphophonological rules; cf., the terms “morphophonological” (\citealt{HaspelmathSims2010}: 214) or “morphophonemic” \citep[23-24]{Payne1997} are used for the alternations that require lexical (and morphosyntactic) information in order to apply the alternation rules. Please note that morphophonological rules precede phonological rules in situations where both rules can apply since morphophonological rules are more specific than phonological rules by definition. Thus, if I encountered a phenomenon which could not be explained by general rules (i.e. phonological rules) already established by other linguistic phenomena, I postulated a special rule (i.e. a morphophonological rule) that would explain the phenomenon and would be applied before the general rule.

  Both of the phonological and morphophonological rules are described as processes, but this does not mean that these processes actually occur in the speaker’s mind. Rather, this style is used because it is easily understandable (cf., \citealt{HaspelmathSims2010}: 211-212).

  In the following subsections, I will present the phonological rules. The first three sections (see §\ref{bkm:Ref304225942}--§\ref{bkm:Ref347173344}) deal with obligatory rules, while the latter two (see §\ref{bkm:Ref347178914}{}-§\ref{bkm:Ref301832441}) deal with rules that are not obligatory but are merely tendencies. The morphophonological rules will be presented in the sections where the relevant morphemes are discussed, e.g., the fusion of the preceding nominal and the topic marker \textit{ja} will be discussed in §\ref{bkm:Ref367134300}.

\subsection{Tap and bilabial approximant deletion}
\label{bkm:Ref304225942}\hypertarget{RefHeadingToc395696978}{}\label{bkm:Ref381399452}
There are no sequences such as /wi/ or /ri/ in Yuwan (except for the three cases discussed later). If this type of sequence occurs at a morpheme boundary, a bilabial approximant //w// or a tap //r// are deleted.

\ea $\left\{\begin{array}{c} \text{w}\\\text{r}\end{array}\right\}$ > ∅ / \_ i 

\ex
\ea \textit{w}{}-deletion\\
     koow\footnote{Strictly speaking, some \textit{w}{}-final verbal roots have \textit{r}{}-final variants (see §\ref{bkm:Ref356245430}), which constitutes free alternation. For example, \textit{koow-} ‘buy’ may be realized as /koor/. If we propose that only the latter could appear before /i/, it is the deletion of //r// (not //w//); however, there is no beneficial reason to propose such a restriction, so we also assume \textit{w}{}-deletion.}   ‘buy’   +  i  (INF)  >  koi\footnote{Phonological rule (see §\ref{bkm:Ref301832441}): (koow + i >) kooi > koi.}  (*koowi)\\
\ex \textit{r}{}-deletion\\
      ar   ‘exist’   +   i   (INF)   >   ai   (*ari)\\
\z
\z

There are, however, three items in the lexicon that have the sequence of /ri/: \textit{piri} ‘tail end,’ \textit{rikkoo} ‘(by) foot,’ and \textit{kiri} ‘fog.’ The first word is regarded as Standard Japanese by the speaker TM, although the plausible equivalent in Standard Japanese is /biri/. The second word \textit{rikkoo} is considered a recent loan word from modern Japanese because there are no other words with word-initial /r/ in Yuwan. It is not clear whether the last word, \textit{kiri} ‘fog,’ existed originally in Yuwan, or was borrowed from Standard Japanese.

\subsection{Alveolar stop affrication (or palatalization)}
\label{bkm:Ref347180796}\hypertarget{RefHeadingToc395696979}{}
The alveolar stop //t// becomes /c/ if it precedes //i// or //j//, which may be called “palatalization” in the broader sense. The reason why we do not assume the combination of /ti/ [t͡ɕi] is argued in §\ref{bkm:Ref347181344}.

\ea t   >   c  /  \_ $\left\{\begin{array}{c} \text{i}\\\text{j}\end{array}\right\}$ 

\ex
\ea Before //i//\\
    ut   ‘hit’   +   i   (INF)   >   uci\\
\ex Before //j//\\
    ut   ‘hit’   +   jaa   ‘person’   >   ucjaa\\
\z
\z

\subsection{Epenthetic vowel /u/}
\label{bkm:Ref301838720}\hypertarget{RefHeadingToc395696980}{}\label{bkm:Ref347173344}
A syllable should have a nucleus filled by a vowel (see §\ref{bkm:Ref301830963}), so if a syllable does not satisfy this condition at morpheme boundaries, an epenthetic vowel /u/ is inserted at the morpheme boundaries and serves as a nucleus.

\ea  ∅   >   u   /   \#\footnote{‘\#’ indicates a syllable boundary.}   \_   C\#
\ex
\ea  mun   ‘thing’   +   n   ‘also’   >   mu.nun   (*mun.n or *mun.nu)\\
\ex  +   nkja   (APPR)   >   mu.nun.kja   (*mun.nkja or *mun.nu.kja)\\
\ex  +   kkwa   (DIM)   >   mu.nuk.kwa   (*mun.kkwa or *mun.ku.kwa)\\
\z
\z

Further, there are no sequences of C\textsubscript{coda}.V in Yuwan. If such a sequence occurs around a morpheme boundary, an epenthetic vowel /u/ is inserted at the morpheme boundary.

\ea   ∅   >   u   /   C\#   \_   V
\ex  \gll tankan   ‘k.o. orange’   +   i   (PLQ)   >   tan.ka.nui [tɑ̟ŋ.kɑ̟.nui]   (*tan.kan.i [tɑ̟ŋ.kɑ̟ɴ.i])\\
           {}         {}           {}  {}  {}      {}    {}          {}          (*tan.ka.ni [tɑ̟ŋ.kɑ̟.ni])\\
\z

These examples show that the forbidden sequence /n.i/ [ɴ.i] is not realized and /nui/ appears instead. Interestingly, a simple combination of /ni/ [ni] does not appear, which may imply that the epenthetic vowel /u/ is inserted not only to stabilize the syllable construction but also to leave a trace of the previous morpheme boundary.

\subsection{Geminate devoicing}
\label{bkm:Ref347178914}\hypertarget{RefHeadingToc395696981}{}
Almost all of the geminates within monomorphemic words in Yuwan are voiceless (see \ref{ex:2.8d} in §\ref{bkm:Ref302599510}). Moreover, if a voiced geminate occurs at a morpheme boundary, it tends to be voiceless.

\begin{exe}
\ex \gll C\textit{\textsubscript{i}}  C\textit{\textsubscript{i}}  >  C\textit{\textsubscript{i}}  C\textit{\textsubscript{i}}\footnotemark\\
 {[+v]}  [+v] {}   [-v]  [-v]\\
\ex\label{ex:2.20}
\begin{xlist}
\ex bb > pp\\
    ar   ‘exist’   +   ba   (CSL)   >   appa\footnote{Morphophonological rule (see §\ref{bkm:Ref347177096}): ar +ba > abba (> appa)}
\ex dd > tt\\
    ar   ‘exist’   +   doo   (ASS)   >   attoo\footnote{Morphophonological rule (see §\ref{bkm:Ref347177096}): ar +doo > addoo (> attoo)}
\ex gg > kk\\
    ar   ‘exist’   +   ga   (CFM3)   >   akka\footnote{Morphophonological rule (see §\ref{bkm:Ref347177096}): ar +ga > agga (> akka)}\\
\end{xlist}
\end{exe}
\footnotetext{The small italic \textit{i} means they have the same articulatory place and manner. Supplemental information is provided in square brackets under the rule schema.}

\subsection{Vowel deletion}
\label{bkm:Ref301832441}\hypertarget{RefHeadingToc395696982}{}
A monomorphemic word has a sequence with at most two vowels (see \ref{ex:2.8e} in §\ref{bkm:Ref302599510}) and it does not have a V\textit{\textsubscript{i}}V\textit{\textsubscript{i}}C\textsubscript{coda} sequence (see \ref{ex:2.8f} in §\ref{bkm:Ref302599510}). If this sequence occurs around a morpheme boundary, one of the preceding vowels tends to be deleted.

\ea  {V\textit{\textsubscript{i}} V\textit{\textsubscript{i}}}  {>}  $\left\{\begin{array}{c} \text{V\textit{\textsubscript{i}}} \\ C \end{array}  \right\}$  {/}  {\_} V    {\#}
\ex
\ea Before a vowel\\
   {koow}  {‘buy’}  {+}  {i}  {(INF)}  {>}  {koi\footnote{Phonological rule (see §\ref{bkm:Ref304225942}): koow + i > kooi (> koi)}}\\
\ex Before a consonant\\
   \gll {attaa}  {‘they’}  {+}  {n}    {‘also’}   {>}  {attan}\\
        {}         {}      {+}  {nkja}  {(APPR)}  {>}  {attankja}\\
\z
\z

Interestingly, though three-vowel sequences tend to be avoided at morpheme boundaries, four-vowel sequences are not. (If we suppose that a syllable dislikes having three morae cosidering \REF{ex:2.20}, the acceptability of /kooii/ may mean the existence of a syllable boundary, such as /koo.ii/.) See the example below; for convenience, the surface form is shown from the beginning in this example (see §\ref{bkm:Ref364174852} for the lengthened form of the infinitive).

\ea koow  ‘buy’  +  ii  (INF)  >  kooii\footnote{Phonological rule (see §\ref{bkm:Ref304225942}): koow + ii > kooii} \z

Yuwan has few lexemes where the vowel /o/ is short (see the note “\ref{bkm:Ref347176670}.” of \tabref{tab:key:4}), and when /o/ appears, its syllable is frequently heavy, i.e., it is /oi/, /oo/ or /oC\textsubscript{coda}/. Otherwise, these lexemes are onomatopoeia such as \textit{botto+botto} ‘lazily,’ interjections such as \textit{ido} ‘hey,’ or seem to be relatively modern loan words from standard Japanese such as \textit{itoko} ‘cousin.’ Those facts may indicate that the /o/ that is short in surface level is long, i.e. /oo/, in underlying level, and that the underlying /oo/ becomes /o/ by the vowel deletion rule in \REF{ex:2.20}. The same argument can be applied to /ə/.

\section{Prosody}\label{bkm:Ref301560567}\hypertarget{RefHeadingToc395696983}{}
\subsection{Three pitch patterns}\label{bkm:Ref303982713}\hypertarget{RefHeadingToc395696984}{}

There is lexical prosody in Yuwan. That is, each root has its own prosodic pattern, and these patterns fall into three types.

\begin{enumerate}[label=\Roman*.]
\item Falling after the penultimate mora of a phonological word
\item Falling after the syllable including the second mora of a phonological word
\item Rising at the final mora of a phonological word
\end{enumerate}

(If the falling position is located word-finally, then falling is realized after the penultimate mora.)

In Tables~\ref{tab:2:20}--\ref{tab:2:22}, both “H” (high pitch) and “L” (low pitch) are counted as a mora respectively.

\begin{table}
\caption{Pitch patterns in Yuwan\label{tab:2:20}}
\begin{tabular}{ l >{\itshape}l *{5}{l} }  
\lsptoprule
&  \normalfont Form & Gloss & \multicolumn{4}{c}{Pitch pattern}\\\cmidrule(lr){4-7} 
     &                     & & Isolation &  x=\textit{nu} &  x=\textit{n} &  x=\textit{gadɨ}\\
     &                     & &          &   (NOM) &  ‘also’ &  (LMT)\\\midrule
I & \itshape haa &  ‘leaf’ &  HL &  HHL &  HL\footnote{(Optional) phonological rule (see §\ref{bkm:Ref301832441}): haa + n > han} &  HHHL\\
  & judai &  ‘saliva’ &  HHL &  HHHL &  HHHL &  HHHHL\\
II & \itshape haa &  ‘teeth’ &  HL &  HHL &  HL &  HHLL\\
  & sɨkama &  ‘morning’ &  HHL &  HHLL &  HHLL &  HHLLL\\
  & məərabɨ &  ‘lady’ &  HHLL &  HHLLL &  HHLLL &  HHLLLL\\
  & hizjai &  ‘left’ &  HHL &  HHHL &  HHLL &  HHHLL\\
III & \itshape naa &  ‘inside’ &  LH &  LLH &  LLH &  LLLH\\
  & nabɨ &  ‘pan’ &  LH &  LLH &  LLH &  LLLH\\
  & usagi &  ‘rabbit’ &  LLH &  LLLH &  LLLH &  LLLLH\\
\lspbottomrule
\end{tabular}
\end{table}

\tabref{tab:2:20} shows that in order to determine the type II pitch pattern, it is necessary to count both syllables and morae.

Most of the lexicon belonging to type II is realized with falling after the second mora, such as /sɨ.ka.ma.nu/ \textit{sɨkama=nu} (morning=NOM) produced as HHLL and /məə.ra.bɨ.nu/ \textit{məərabɨ=nu} (lady=NOM) produced as HHLLL. However, if the second syllable contains a vowel sequence, the falling occurs after the third mora, such as /hi.zjai.nu/ \textit{hizjai=nu} (left=NOM) produced as HHHL, which means type II represents falling not after the second mora, but after the second syllable including the second mora. Furthermore, if you only allow that “type II represents falling after the second syllable,” you cannot explain why /məə.ra.bɨ.nu/ \textit{məərabɨ=nu} (lady=NOM) is produced as HHLLL.

The prosodic behavior discussed above helps us think about the long vowels and dipthongs in Yuwan. In short, we cannot assume a long vowel phoneme, such as /aː/, or a diphthong phoneme, such as /a\textsuperscript{i}/, because we presuppose the following three points:

\begin{enumerate}[label=\alph*.]
\item A mora is assigned not to a phoneme but to a slot;
\item A slot may have maximally one mora;
\item One phoneme can fill only one slot.
\end{enumerate}

(Note: ‘slot’ in the above means C, G, or V in a syllable. See \sectref{sec:key:2.3.1} for more details.)

That is, we do not propose that one slot has two morae, that one phoneme has two morae, or that one phoneme can fill two moraic slots in a syllable. From the point of view of prosody, long vowels and diphthongs in Yuwan have two morae, so we do not assume a long vowel phoneme, such as /aː/, or a diphthong phoneme, such as /a\textsuperscript{i}/. A similar problem was discussed in \citet[196-199]{Dixon2010} where “in Fijian - a mora-counting language - a long vowel can be usefully regarded as a sequence of two short vowels.”

\subsection{Some notes on initial glottalized consonants}
\label{bkm:Ref347180628}\hypertarget{RefHeadingToc395696985}{}
In Yuwan, there seems to be irregular pitch patterns if the initial consonant of words is glottalized.

\begin{table}
\caption{Pitch patterns of words beginning with a glottalized consonant (part 1)}
\begin{tabular}{>{\itshape}llllll}
\lsptoprule
\normalfont Form & Gloss & \multicolumn{4}{c}{Pitch pattern}\\\cmidrule(lr){3-6}
     &       & Isolation & x=\textit{nu} & x=\textit{n} & x=\textit{gadɨ}\\
     &       &           & (NOM) & ‘also’ & (LMT)\\\midrule
nˀjɨ    &  ‘rice plant’ &  H   & HL   & HL   &  HLL\\
mˀa     &  ‘horse’       &  H   & HL   & HL   &  HLL\\
nˀjuti  &  ‘life’     &  HL  & HLL  & HLL  &  HLLL\\
mˀacɨ   &  ‘fire’      &  HL  & HLL  & HLL  &  HLLL\\
kˀwagɨ  &  ‘mulberry’ &  HL  & HLL  & HLL  &  HLLL\\
kˀjubii &  ‘belt’    &  HLL & HLLL & HLLL &  HLLLL\\
\lspbottomrule
\end{tabular}
\end{table}

In these words, falling seems to occur after the first mora, and such a pitch pattern is not found elsewhere (see §\ref{bkm:Ref303982713}). There are two possible analyses to explain this finding:

\begin{description}
\item[Analysis 1:] Glottalized phonemes have one mora by themselves.
\item[Analysis 2:] Glottalized resonants or glottalized stops with approximants create a subcategory of pitch patterns.
\end{description}

Analysis 1, however, immediately turns out to be false, because there is a case where a glottalized phoneme does not seem to have one mora.

\begin{table}
\caption{Pitch patterns of words beginning with a glottalized consonant (part 2)\label{tab:key:22}}
\begin{tabular}{>{\itshape}llllll}
\lsptoprule
\normalfont Form & Gloss & \multicolumn{4}{c}{Pitch pattern}\\\cmidrule(lr){3-6}
     &       & Isolation & x=\textit{nu} & x=\textit{n} & x=\textit{gadɨ}\\
     &       &           & (NOM) & ‘also’ & (LMT)\\\midrule
kˀura &  ‘storehouse’ &  HL &  HHL &  HHL &  HHLL\\
\lspbottomrule
\end{tabular}
\end{table}

\tabref{tab:key:22} shows that glottalized /kˀ/ does not have a mora because the falling is realized not after /kˀu/ but after /ra/ (when it precedes clitics). In other words, it behaves regularly as the type II pitch pattern (see §\ref{bkm:Ref303982713}). Since we cannot regard the glottalized consonant /kˀ/ as having one mora, Analysis 1 cannot be accepted.

Analysis 2 assumes that the type II pitch pattern has two subcategories:

\begin{description}
\item[Subcategory I:] If initial consonants are glottalized resonants such as /nˀ/, or the glottalized velar stop /kˀ/ plus an approximant such as /kˀw/ or /kˀj/, then the falling occurs after the initial mora.
\item[Subcategory II:] Otherwise, the falling occurs after the syllable including the second mora.
\end{description}

These subcategories can be explained by phonotactics, which means their differences need not be assigned to the lexicon. Following these points, we will take up Analysis 2. Additionally, many of the glottalized consonants were the result of syllable omission (see §\ref{bkm:Ref347180773}). Therefore, the retaining of a mora by a glottal phoneme can also be explained from a historical perspective.

\subsection{Further research}
\label{bkm:Ref347175621}\hypertarget{RefHeadingToc395696986}{}\label{bkm:Ref381418627}
In the previous section, we discussed the prosody of nominals in Yuwan; however, the data set is very limited. In fact, we only dealt with 207 words. The breakdown of the pitch patterns of these words are shown in \tabref{tab:2:33}.

\begin{table}
\caption{Breakdown of pitch patterns of nominals\label{tab:2:33}}
\begin{tabular}{lS[table-format=3]S[table-format=2]}
\lsptoprule
Pattern & {Number of words} & {\%}\\\midrule
I     & 99  & 48\\
II    & 56  & 27\\
III   & 52  & 25\\\midrule
Total & 207 & 100\\
\lspbottomrule
\end{tabular}
\end{table}

It is important to note that there are many cases where the falling or rising of the three accent patterns is not realized. In other words, there are many cases where a phonological word keeps a flat pitch throughout, and this makes it difficult to fully know the accurate pitch patterns of words in Yuwan. In the above data, we excluded these data and only focused on words that have pitch movement; however, we need to clarify this omission for future research.

Although research into the prosody of Yuwan is not yet sufficient, our current data and analysis make it possible to propose the following points. First, we propose that verbs and adjectives seem to have the same pitch patterns as nominals, although the details of their proportions are different. Second, compounds seem to retain the pitch patterns of the preceding stem. Third, the most recent loan words (from English loan words in Standard Japanese) tend to have the type I pitch pattern.
