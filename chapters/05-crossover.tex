\chapter{Cross-over categories}\label{chap:5}

Every word in Yuwan can be categorized into a word class (i.e. nominals, adnominals, verbs, adjectives, particles, adverbs, and interjections), as determined by some morphosyntactic criteria (see \sectref{sec:key:4.3}). The class of demonstratives, however, can crosscut several word classes, including nominal \textit{kurɨ} ‘this’ and adnominal \textit{kun} ‘this (one).’ Here, we introduce another category of words called “cross-over categories.” There are three cross-over categories: personal pronominals, demonstratives, and interrogatives. Semantically, each cross-over category has a common functional property. The personal pronominals express “person deixis” \parencite[61--62]{Fillmore1997} (i.e. the speaker, the hearer, or the other), the demonstratives express spatial deixis, and the interrogatives can be used in questions. Morphologically, all of the personal pronominals and demonstratives, and some of the interrogatives, can be divided into a root and an affix (or affixes). The relations between word classes and cross-over categories are summarized as follows.

\begin{table}
\caption{\label{tab:key:30}Word classes and cross-over categories}
\begin{tabular}{lccc}
\lsptoprule
                      & \multicolumn{3}{c}{Word classes}\\\cmidrule(lr){2-4}
Cross-over categories & Nominals & Adnominals & Adverbs\\\midrule
Personal pronominals  & +        & +          & \textminus\\
Demonstratives        & +        & +          & +\\
Interrogatives        & +        & +          & +\\
\lspbottomrule
\end{tabular}
\end{table}

The personal pronominals cannot become adverbs. There are no cross-over categories that become verbs, adjectives, particles, or interjections. The difference between cross-over categories and verbs will be discussed in the \sectref{sec:8.4.5}.

\section{Personal pronominals}

A personal pronominal in Yuwan is a deictic word that indicates chiefly the speaker or the hearer.

Morphologically, a personal pronominal word is composed of a root plus an affix (or affixes). There are three personal pronominal roots: \textit{waa-} \REF{ex:key:1}, \textit{naa-} (2.HON), and \textit{ura-} (2.NHON). All personal pronominal roots are bound forms. They can take four affixes, i.e. \textit{{}-n}/\textit{{}-∅} (SG), \textit{{}-ttəə} (DU), \textit{{}-kja} (PL), and \textit{{}-a} (ADNZ).

Semantically, the root \textit{waa-} is used for first-person reference, i.e. the speaker. The roots \textit{naa-} and \textit{ura} are used for second-person reference, i.e. the hearer; \textit{naa-} is an honorific form, used to refer to addressees who are older or have a higher status than the speaker, and \textit{ura} is used elsewhere. Deictic expression of third-person reference, i.e. non-speaker and non-hearer, is expressed in principle by demonstratives (see \sectref{sec:5.2}); however, there is a dual form to express third person, namely /nattəə/ ‘that two people,’ which is the same as the honorific dual form to express the second person (see \sectref{sec:5.1.3} for more details).

Syntactically, personal pronominal words can become two word classes: nominals such as /waakja/ ‘we’ or adnominals such as /waakjaa/ ‘our.’ In personal pronominal words, both nominals (henceforth, “personal pronouns”) and adnominals exhibit number distinctions, but there are no dual forms of adnominals. If the dual forms of the personal pronouns fill the modifier slot of an NP, they take \textit{ga} (GEN). Note that in the following examples, \textit{waa-} becomes /wa/, and \textit{naa-} becomes /na/, when they precede \textit{{}-n}, \textit{{}-ttəə}, or \textit{–a}. This vowel reduction is explained by the phonological rule in \sectref{sec:2.4.5}.

\begin{table}
\caption{\label{tab:key:31}Personal pronouns (surface forms)}
\begin{tabular}{lllll}
\lsptoprule
Person & Honorific & \multicolumn{3}{c}{Number}\\\cmidrule(lr){3-5}
       &           &  Singular & Dual & Plural\\\midrule
1\textsuperscript{st}  &                & wan   & wattəə  & waakja\\
2\textsuperscript{nd}  &  Non-honorific & ura   & urattəə & urakja\\
                       &  Honorific     & nan   & nattəə  & naakja\\
3\textsuperscript{rd}  &                & N/A   & nattəə  & N/A   \\
\lspbottomrule
\end{tabular}
\end{table}

Dual forms are relatively rare in Yuwan. The total numbers of tokens of personal pronominals (uttered by US, TM, and MY) in my texts are as follows: singular forms totaled 148 (\textit{wan/waa}: 76, \textit{ura/uraa}: 36, \textit{nan/naa}: 36); dual forms totaled 17 (\textit{wattəə}: 9, \textit{urattəə}: 3, \textit{nattəə} (2\textsuperscript{nd}): 1, \textit{nattəə} (3\textsuperscript{rd}): 4); and plural forms totaled 189 (\textit{waakja/waakjaa}: 117, \textit{urakja/urakjaa}: 57, \textit{naakja/naakjaa}: 15). 

\begin{styleBeschriftung}
\textmd{\tabref{tab:key:32}. Personal pronominal adnominals (surface forms)}
\end{styleBeschriftung}

Person  Honorific  Number

    Singular  Plural

1\textsuperscript{st} person    waa  waakjaa

2\textsuperscript{nd} person  Non-honorific  uraa  urakjaa

  Honorific  naa  naakjaa

At first glance, the morpheme boundaries in the above personal pronominal words seem relatively easy to divide, but it is actually very difficult to do that. The challenges in determining morpheme boundaries are discussed in \sectref{sec:5.1.4} in detail. In this grammar, the morpheme boundaries of personal pronominal words are not expressed (even if they are present at the underlying level) unless they need to be clearly distinguished.

Personal pronominal adnominals in the plural, i.e. /waakjaa/, /urakjaa/, and /naakjaa/, sometimes reduce their word-final long vowels to short vowels such as /waakja/, /urakja/, and /naakja/. In these cases, it may be possible to interpret them as nominals juxtaposed in the modifier slot of an NP such as address nouns (see \sectref{sec:7.2}).

  The following examples illustrate the difference between personal pronouns and personal pronominal adnominals.

\ea \label{ex:5:1}  \ea \label{ex:5:1a} Personal pronouns

    [Context: Looking at pictures considered to be taken a little after World War II]

    %TM:
\glll  waakjaga  warabɨ  sjuinkjoo,  ganba,  hukunkjoo  tˀɨn  nənba.\\
[\textit{waakja}\textsubscript{Head}]\textsubscript{NP}\textit{=ga}  \textit{warabɨ}  \textit{sɨr-tur-i-n=kja=ja}  \textit{ganba} \textit{huku=nkja=ja}  \textit{tˀɨɨ=n}  \textit{ar-an-ba}\\
1PL=NOM  child  do-PROG-INF-time=APPR=TOP  therefore   clothes=APPRT=TOP  one=even  exist-NEG-CSL\\
\glt ‘When we were children, therefore, there are no clothes.’ [Co: 111113\_01.txt]

 \ex \label{ex:5:1b} Personal pronominal adnominals

    [Context: TM talks about usual meals with the hearer MY; MY: ‘I always eat pickles after the meals.’]

    %TM:
\glll  waakjaa  uziitaaga  gansjɨ  jatassɨga.\\
[\textit{waakjaa}\textsubscript{Modifier}  \textit{uzii-taa}\textsubscript{Head}]\textsubscript{NP}\textit{=ga}  \textit{ga-nsjɨ}  \textit{jar-tar-sɨga}\\
1PL.ADNZ  old.man-PL=NOM  MES-ADVZ  COP-PST-POL\\
\glt ‘Our old man (i.e. my husband) was like that.’ [Co: 101023\_01.txt]
\z
\z

In (5-1 a), the nominal \textit{waakja} ‘we’ fills the head slot of an NP taking the nominative particle \textit{ga}, and in (5-1 b), the adnominal \textit{waakjaa} ‘our’ directly fills the modifier slot of an NP not taking the genitive particle. In other words, the forms behave differently in light of the syntactic criteria of word classes (see \sectref{sec:4.3}).

  In the following subsections, we examine each type of person reference in detail; the first person (see \sectref{sec:5.1.1.}), the second person (see \sectref{sec:5.1.2}), and the third person (see \sectref{sec:5.1.3}). In particular, we will focus on their nominal forms. For their adnominal forms, see \sectref{sec:6.4.2.} In \sectref{sec:5.1.4}, I will show an analysis of the personal pronominal paradigm.

\subsection{First person}

First-person pronominals are shown below.

\begin{styleBeschriftung}
\textmd{\tabref{tab:key:33}. First-person pronominals (surface forms)}
\end{styleBeschriftung}

Word classes  Number

  Singular  Dual  Plural

Nominals  wan  wattəə  waakja

Adnominals  waa  waakjaa

I present an example of the singular form of first-person pronouns, i.e. \textit{wan} (1SG).

\ea \label{ex:5:2}   Singular

  %TM:
\glll  wanga  agan  ikjussaccjɨ.\\
\textit{wan=ga}  \textit{aga-n}  \textit{ik-jur-sa=ccjɨ}\\
1SG=NOM  DIST-ADVZ  go-UMRK-POL=QT\\
\glt ‘(I said to the present author), “I will go there.”’ [Co: 110328\_00.txt]
\z

Yuwan does not have inclusive vs. exclusive distinctions for the first-person dual forms or plural forms. In \REF{ex:5:3}, \textit{wattəə} (1DU) is used for both inclusive and exclusive meanings.

\ea \label{ex:5:3}  \ea \label{ex:5:3a} Inclusive dual

    [Context: TM asks the hearer US of the difference in age between them.]

    %TM:
\glll  wattəə  ikjasa  cigajui?\\
\textit{wattəə}  \textit{ikja-sa}  \textit{cigaw-jur-i}\\
1DU  how-NLZ  different-UMRK-NPST\\
\glt ‘How many (years between the age of) us (i.e. you and me)?’ [Co: 110328\_00.txt]

 \ex \label{ex:5:b} Exclusive dual

    [Context: TM talks about her son with MS; TM: ‘My son doesn’t say anything to me, and I don’t say anything to him either;’ MS: ‘Maybe, you are parent and child, I think.’]

    %TM:
\glll  aran.  sjoobunga  nissjaatɨ,  wattəəja.\\
\textit{jar-an}  \textit{sjoobun=ga}  \textit{nissj-sa+ar-tɨ}  \textit{wattəə=ja}\\
COP-NEG  character=FOC  resemble-ADJ+STV-SEQ  1DU=TOP\\
\glt ‘No. (It is because of ) the character in which we (i.e. I and he) resemble (each other).’ [Co: 120415\_01.txt]
\z
\z

In (5-2 a) TM uses \textit{wattəə} (1DU) ‘the two of us’ to include the hearer US, and in (5-2 b) she uses the same form to exclude the hearer MS.

If a speaker wants to specify a referent other than the speaker of the first-person dual form, the nominal (that indicates the associate) occurs with the case particle \textit{tu} (COM) before \textit{wattəə} (1DU).

\ea \label{ex:5:4}   [Context: Speaking about the days when TM goes to the day-care center in the community]

  %TM:
\glll  kˀajoobin  ujuritu  wattəə  ikjun  tukinnja,\\
\textit{kˀwajoobi}\footnotemark\textit{=n}  \textit{ujuri=tu}  \textit{wattəə}  \textit{ik-jur-n}  \textit{tuki=n=ja}\\
Tuesday=DAT1  Uyuri=COM  1DU  go-UMRK-PTCP  time=DAT1=TOP\\
\glt ‘On Tuesday, when Uyuri and me go (there), ...’ [Co: 120415\_01.txt]
\z
\footnotetext{The speaker TM explained to the present author that ‘Tuesday’ was /kˀwajoobi/ in Yuwan during elicitation, but she said /kˀajoobi/ in this text.}

Please note that \textit{ujuri=tu} \textit{wattəə} (Uyuri=COM 1DU) does not mean ‘Uyuri and the two of us’ (i.e. three referents), but instead means ‘Uyuri and me’ (i.e. two referents). Cross-linguistically, this kind of phenomenon is not uncommon (\citealt{Jespersen1924}[1992]: 192 and \citealt{Moravcsik2003}: 475), and it is called “inclusory constructions” in \citet{Lichtenberk2000}. One may think that the example in \REF{ex:5:4} is a case of “quantifier float,” which will be discussed in \sectref{sec:7.4.1.} In fact, the dual affix \textit{{}-ttəə} seems to have some diachronic relation with the numeral \textit{tˀai} ‘two people.’ However, synchronically \textit{{}-ttəə} (DU) and \textit{tˀai} ‘two people’ are different morphemes, because they can co-occur in the same clause modifying the same referent as in \REF{ex:5:5}.

\ea \label{ex:5:5}   %TM:
\glll  wattəə  tˀai  ikiidoo.\\
\textit{wattəə}  \textit{tˀai}  \textit{ik-i=doo}\\
1DU  two.person  go-INF=ASS\\
\glt ‘The two of us will go.’ [El: 121112]
\z

Therefore, we have to recognize that the comitative nominal, i.e. \textit{ujuri=tu} ‘Uyuri and’ in \REF{ex:5:4}, does not “add” a person to \textit{wattəə} (1DU), but instead “fills” the non-speaker slot of the dual form.

The plural form \textit{waakja} (1PL) can also be used with the numeral \textit{tˀai} ‘two people,’ which means the ‘plural’ form \textit{waakja} (1PL) does not exclude dual meaning.

\ea \label{ex:5:6}   %TM:
\glll  waakjoo  tˀai  ikiidoo.\\
\textit{waakja=ja}  \textit{tˀai}  \textit{ik-i=doo}\\
1PL=TOP  two.person  go-INF=ASS\\
\glt ‘The two of us will go.’ [El: 121112]
\z

The above example is uttered by elicitation. In the natural discourse, the two referents in the first or second person are necessarily indicated by the dual forms. That is, the dual in Yuwan is not the “facultative number” in \citet{Corbett2000}, since the forms for the facultative number usually tend to be replaced by the plural form (ibid.: 45).

As mentioned above, the plural form \textit{waakja} (1PL) can express both inclusive meaning and exclusive meaning.

\ea \label{ex:5:7}  \ea \label{ex:5:7a} Inclusive plural

    [Context: There are only three people including TM, and TM asks one of them.]

    %TM:
\glll  waakjoo  ikjantɨn,  jiccja  akkaijaa.\\
\textit{waakja=ja}  \textit{ik-an-tɨ=n}  \textit{jiccj-sa}  \textit{ar=kai=jaa}\\
1PL=TOP  go-NEG-SEQ=even  no.problem-ADJ  STV=DUB=SOL\\
\glt ‘Is there no problem, even if we (all) do not go (there)?’ [El: 130812]

 \ex \label{ex:5:b} Exclusive plural

    [Context: Someone asked TM whether she and other people gathered in TM’s house yesterday.]

    %TM:
\glll  kinjoo  waakjoo  jurawantɨdoo.\\
\textit{kinju=ja}  \textit{waakja=ja}  \textit{juraw-an-tɨ=doo}\\
yesterday=TOP  1PL=TOP  gather-NEG-SEQ=ASS\\
\glt ‘We did not gather yesterday.’ [El: 130812]
\z
\z

In (5-7 a), TM uses \textit{waakja} (1PL) ‘we (all)’ including the hearer, and in (5-7 b) she uses the same form excluding the hearer.

  The plural form \textit{waakja} (1PL) is not only used to indicate genuine plurality. That is, while it may be used to indicate multiple referents including the speaker, it may also be used to virtually indicate only the speaker. The latter use of \textit{waakja} (1PL) may be paraphrased in English as “a person like me.” I will present an example below.

\ea \label{ex:5:8}   [Context: there are only four people, i.e. US, TM, MY, and the present author. US praised TM for her knowledge, but TM was modest and said that she knew nothing at all.]

  TM: \glll  waakjan  sijanmun.\\
    \textit{waakja=n}  \textit{sij-an=mun}\\
    1PL=also  know-NEG=ADVRS\\
    \glt \hspaceThis{TM:} \parbox{\linewidth-\widthof{TM:}}{‘I don’t know anything either.’ (or ‘A person like me doesn’t know anything either.’)}

MY: \glll  wanundoojaa.\\
\textit{wan=n=doo=jaa}\\
1SG =also=ASS=SOL\\
\glt \hspaceThis{MY:} ‘Niether do I.’ [Co: 110328\_00.txt]
\z

In this scene, there are only four people, i.e. US, TM, MY, and the present author. US praised TM’s knowledge in order for the present author to recognize TM’s authority as a teacher of the Yuwan language. However, TM replied that she did not know anything showing her modesty. In this case, it is difficult to interpret the \textit{waakja} (1PL) in TM’s utterance as including US, MY, or the present author. The MY’s utterance (immediately following the TM’s) also shows that the \textit{waakja} (1PL) in TM’s utterance does not include another participant, since MY said ‘Niether do I.’ In other words, MY said so because she did not think the \textit{waakja} (PL) does not include MY herself.

  This use of \textit{waakja} (1PL) is very common in Yuwan. The reason for this phenomenon might be related to the flexible meaning of \textit{{}-kja} (PL), which can indicate not only a specific group, but also an unspecific group. The figure below illustrates the potential ambiguities associated with the three possible number distinctions in first-person reference.

%%[Warning: Draw object ignored]
%%[Warning: Draw object ignored]
%%[Warning: Draw object ignored]
%%[Warning: Draw object ignored]
%%[Warning: Draw object ignored]
%%[Warning: Draw object ignored]
%%[Warning: Draw object ignored]
%%[Warning: Draw object ignored]
%%[Warning: Draw object ignored]
%%[Warning: Draw object ignored]
%%[Warning: Draw object ignored]
%%[Warning: Draw object ignored]
%%[Warning: Draw object ignored]
%%[Warning: Draw object ignored]
%%[Warning: Draw object ignored]
%%[Warning: Draw object ignored]
%%[Warning: Draw object ignored]
%%[Warning: Draw object ignored]

Singular

Plural

: speaker (specific)

: non-speaker (specific)

: non-speaker (unspecific)

: group (specific)

: group (unspecific)

or

...

...

...

Dual

\begin{styleBeschriftung}
\textmd{\figref{fig:key:7}. Three number distinctions in first-person reference}
\end{styleBeschriftung}

This figure shows that the right-most figure, i.e. the plural indicating the speaker associated with unspecific referents in an unspecific group, is very similar to the left-most figure, i.e. the singular. This similarity makes it possible to use the plural form (in the meaning of the right-most figure) like the singular form. In fact, the plural form \textit{waakja} (1PL) in \REF{ex:5:8} indicates an unspecific group as in the right-most figure in \figref{fig:key:7}. In that group, the specific referent is only the speaker, and the unspecific group is thought to be composed of “people who do not know anything important.” This kind of plural meaning is also expressed in the second-person pronominals discussed in the next section (see also the discussion in \sectref{sec:6.4.1.1}).

\subsection{Second person}

Second-person pronominals are shown below.

\begin{styleBeschriftung}
\textmd{\tabref{tab:key:34}. Second-person pronominals (surface forms)}
\end{styleBeschriftung}

Word classes  Honorific  Number

    Singular  Dual  Plural

Nominals  Honorific  nan  nattəə  naakja

  Non-honorific  ura  urattəə  urakja

Adnominals  Honorific  naa  naakjaa

  Non-honorific  uraa  urakjaa

For second-person pronominals in Yuwan, there is a distinction between honorific and non-honorific forms; the honorific forms are used for addressees who are older (or have a higher status) than the speaker and the non-honorific forms are used elsewhere.

\ea \label{ex:5:9}  \ea \label{ex:5:9a} \textit{nan} (2.HON.SG)

    [Context: TM told US that she thought the present author would not come to her place after visiting US’s place.]

    %TM:
\glll  nanga  umoocjan  un  hiija,\\
\textit{nan=ga}  \textit{umoor-tar-n}  \textit{u-n}  \textit{hii=ja}\\
2.HON.SG =NOM  say.HON-PST-PTCP  MES-ADNZ  day=TOP\\
\glt ‘About the day you said (about the visit from the present author), ...’ [Co: 110328\_00.txt]

 \ex \label{ex:5:b} \textit{ura} (2.NHON.SG)

    [Context: TM asked MS, who sometimes has to do night duty at his place of work, to help the present author with the study.]

    %TM:
\glll  uraga  tumainu  aran  tukin,\\
\textit{ura=ga}  \textit{tumar-i=nu}  \textit{ar-an}  \textit{tuki=n}\\
2.NHON.SG=NOM  stay-INF=NOM  COP-NEG  time=DAT1\\
\glt ‘When you are not on night duty, ...’ [Co: 111113\_02.txt]
\z
\z

In (5-9 a), TM is speaking to US, who is older than TM, so TM has to use the honorific form of the second-person pronoun. On the other hand, in (5-9 b), TM is speaking to MS, who is younger than TM, so TM uses the non-honorific form of the second-person pronoun.

  Both the honorific and non-honorific forms have dual nominal forms.

\ea \label{ex:5:10}  \ea \label{ex:5:10a} \textit{nattəə} (2.HON.DU)

    [Context: TM said to US that they did not play together and wondered why they did not. Then, MY suggested a plausible reason.]

    %MY:
\glll  asɨbija  sɨran.joo.  nattəə  tusiga   cigajunmun.\\
\textit{asɨb-i=ja}  \textit{sɨr-an=joo}  \textit{nattəə}  \textit{tusi=ga}  \textit{cigaw-jur-n=mun}\\
play-INF=TOP  do-NEG=CFM1  2.HON.DU  age=FOC  different-UMRK-PTCP=ADVRS\\
\glt ‘(You) would not play. The two of you were not the same age.’ [Co: 110328\_00.txt]

 \ex \label{ex:5:b} \textit{urattəə} (2.NHON.DU)

    [Context: TM had MS and the present author for lunch.]

    %TM:
\glll  urattəə  kadɨ  kurɨppa.\\
\textit{urattəə}  \textit{kam-tɨ}  \textit{kurɨr-ba}\\
2.NHON. DU  eat-SEQ  BEN-CSL\\
\glt ‘The two of you, eat (the lunches), please.’ [Co: 120415\_01.txt]
\z
\z

  As mentioned in \sectref{sec:5.1.1}, the plural affix for personal pronominals, i.e. \textit{{}-kja} (PL), can indicate not only a specific group, but also an unspecific group. These meanings are illustrated below.

%%[Warning: Draw object ignored]
%%[Warning: Draw object ignored]
%%[Warning: Draw object ignored]
%%[Warning: Draw object ignored]
%%[Warning: Draw object ignored]
%%[Warning: Draw object ignored]
%%[Warning: Draw object ignored]
%%[Warning: Draw object ignored]
%%[Warning: Draw object ignored]
%%[Warning: Draw object ignored]
%%[Warning: Draw object ignored]

Singular

Plural

...

...

or

...

Dual

%%[Warning: Draw object ignored]
%%[Warning: Draw object ignored]
%%[Warning: Draw object ignored]
%%[Warning: Draw object ignored]
%%[Warning: Draw object ignored]
%%[Warning: Draw object ignored]
%%[Warning: Draw object ignored]

: hearer (specific)

: non-hearer (specific)

: non-hearer (unspecific)

: group (specific)

: group (unspecific)

\begin{styleBeschriftung}
\textmd{\figref{fig:key:8}. Three number distinctions in second-person reference}
\end{styleBeschriftung}

This illustration shows that the right-most figure, i.e. the plural indicating the hearer associated with unspecific referents in an unspecific group, is very similar to the left-most figure, i.e. the singular. This similarity makes it possible to use the plural form (in the meaning of the right-most figure) like the singular form. The plural form in that use may be paraphrased in English as “a person like you.” The following two examples illustrate that use of plural forms.

\ea \label{ex:5:11}  \ea \label{ex:5:11a} \textit{naakja} (2.HON.PL)

    [Context: Talking to US about labor involved with carrying miscanthus from the mountain to thatch a roof in the old days.]

    TM:  naakjoo  gajaurusinkjoo  sɨrantaroo.

      \textit{naakja=ja}  \textit{gaja+urus-i=nkja=ja} \textit{sɨr-an-tar-oo}
                                                       
      2.HON.PL=TOP  miscanthus+take.down-INF=APPR=TOP  do-NEG-PST-SUPP

      ‘Probably (a person like) you would not carry the miscanthus.’

      [Co: 110328\_00.txt]

 \ex \label{ex:5:b} \textit{urakja} (2.NHON.PL)

    [Context: Seeing a picture with MS]

    TM:  urakjaga  jamatoocinkja  ikjun  {\textbar}koro{\textbar}kai  xxx  jaa.

      \textit{urakja=ga}  \textit{jamatu=kaci=nkja} \textit{ik-jur-n}  \textit{koro=kai}    \textit{=jaa}
                                                   
      2.NHON.PL=NOM  main.island.of.Japan=ALL=APPR  go-UMRK-PTCP  time=DUB    =SOL

      ‘I wonder if (the time when the picture was taken) was the time (a person like) you went to the main island of Japan (to find a job).’

      [Co: 120415\_00.txt]
      \z
      \z

Here, \textit{naakja} (2.HON.PL) in (5-11 a) indicates an unspecific group as in the right-most figure in \figref{fig:key:8}. In that group, the specific referent is only the hearer, and the unspecific group is thought to be composed of “people who would not carry the miscanthus.” Likewise, \textit{urakja} (2.NHON.PL) in (5-11 b) indicates an unspecific group as in the right-most figure in \figref{fig:key:8}. In that group, the specific referent is only the hearer, and the unspecific group is thought to be composed of “people who went to the main island of Japan (to find a job).”

\subsection{Third person}

In principle, deictic expression of third-person reference is expressed by demonstratives in Yuwan (see \sectref{sec:5.2}). However, the demonstratives in Yuwan lack the dual number, and in the case of the third person dual, the form /nattəə/ is used. In other words, the third person pronoun and the demonstratives in Yuwan are in the complementary distribution in the grammatical number. \textit{nattəə} (3.DU) has the same form as the second-person honorific dual form (see \sectref{sec:5.1.2}), but it can indicate both of honorific referents as in (5-12 a) and non-honorific referents as in (5-12 b).

\ea \label{ex:5:12}   Third-person dual
  \ea Honorific referents

  [Context: Speaking about two people who are older than TM]

  TM:  nattəə,  {\textbar}ittoki{\textbar}ja,  muru  dusi  sjɨ,  gansjɨ  jiccja atanmundoojaa.
                                                                                                                            
    \textit{nattəə}  \textit{ittoki=ja}  \textit{muru}  \textit{dusi}  \textit{sɨr-tɨ}  \textit{ga-nsjɨ}  \textit{jiccj-sa} \textit{ar-tar-n=mun=doo=jaa}
                                                                                                                            
    3.DU  while=TOP  very  friend  do-SEQ  MES-ADVZ  good-ADJ  STV-PST-PTCP=ADVRS=ASS=SOL
    
    ‘Those two people [i.e. TM’s acquaintances older than TM], for a while, were friends, and that was very good.’

    [Co: 120415\_01.txt]

  \ex Non-honorific referents

  [Context: Talking about the speaker’s daughter and son]

  %TM:
\glll  nattəəja  {\textbar}rjooribangumi{\textbar}  hanasija  muru  sɨkidoojaa.\\
\textit{nattəə=ja}  \textit{rjooribangumi}  \textit{hanas-i=ja}  \textit{muru}  \textit{sɨki=doo=jaa}\\
3.DU=TOP  cooking.show  talk-INF=TOP  very  like=ASS=SOL\\
\glt ‘Those two people [i.e. the speaker’s daughter and son] like speaking of a cooking show very much.’ [El: 130823]
\z
\z

In (5-12 a-b), /nattəə/ indicates two people not including the speaker or hearer. In (5-12 a), the referents are older than the speaker. In (5-12 b), the referents are younger than the speaker. Thus, /natəə/ in these examples is not sensitive to the social relationship between the speaker and the referent when it indicates the third-person referents. As mentioned in \sectref{sec:5.1.2}, \textit{nattəə} (2.HON.DU) and \textit{urattəə} (2.NHON.DU) can be used to indicate the second-person referents. However, /urattəə/ cannot be used to indicate the third-person referents, which is crucially different from /nattəə/.

Additionally, \textit{nattəə} (3.DU) may be repleced by another analytic expression, i.e. \textit{a-n} \textit{tˀai} (DIST-ADNZ two.CLF.person) ‘those two people,’ which is composed of a demonstrative adnominal plus a numeral as in (5-13 a-b).

\ea \label{ex:5:13}   Analytic expression to indicate two referents

  \ea Honorific referents

  [Context: Speaking with MS, who is younger than TM, about two people who are older than TM]

  %TM:
\glll  an  tˀaija  ittokəə,  naa,  {\textbar}oi{\textbar}cjɨboo, {\textbar}oi{\textbar}cjɨ  {\textbar}juujoonakanzi{\textbar}sjɨ,\\
\textit{a-n}  \textit{tˀai=ja}  \textit{ittoki=ja}  \textit{naa}  \textit{oi=ccjɨ=boo}  \textit{oi=ccjɨ}  \textit{juujoonakanzi=sjɨ}\\
DIST-ADNZ  two.person.CLF=TOP  for.a.while=TOP  FIL  hey=QT=CND  hey=QT  likely.to.say=INST\\
\glt ‘Those two people [i.e. TM’s acquaintances older than TM] (were such close that they) likely to say (roughly) “Hey” (to each other) for a while (in the past), and ...’ [Co: 120415\_01.txt]

  \ex Non-honorific referents

  [Context: Talking to MS about two people, who are younger than TM, but who have already died.]

  TM:  un.  .. hunto  an  tˀaiga  wuppoo,  muru jiccja  atanmundoo.
                                                                                                  
    \textit{un}  \textit{huntoo}  \textit{a-n}  \textit{tˀai=ga}  \textit{wur-boo}  \textit{muru} \textit{jiccj-sa}  \textit{ar-tar-n=mun=doo}
                                                                                                  
    BCH  really  DIST-ADNZ  two.person=NOM  exist-CND  very   good-ADJ  STV-PST-PTCP=ADVRS=ASS

    ‘Yeah. .. Really, if those two people [i.e. TM’s acquaintances younger than TM] were to exist [i.e. be alive], it would be very good.’

    [Co: 120415\_01.txt]
\z
\z

In the above examples, \textit{a-n} \textit{tˀai} (DIST-ADNZ two.CLF.person) ‘those two people’ indicates the referents both of older than the speaker and younger than the speaker as well as \textit{nattəə} (3.DU).

\subsection{Analysis of the personal pronominal paradigm}

As mentioned in \sectref{sec:5.1}, personal pronominals seem to contain morpheme boundaries; however, it is difficult to determine the best way to analyze them. This kind of problem is common in the languages around the world and there is likely to be more than one analysis (cf. \citealt{Comrie1989}: 49 about Hungarian). However, I propose the following analysis as the best.

\ea \label{ex:5:14}  Personal pronominal morphemes

Roots:      \textit{waa-} \REF{ex:key:1}, \textit{naa-} (2.HON), \textit{ura-} (2.NHON);

Number affixes:  \textit{{}-n}/\textit{{}-∅} (SG), \textit{{}-ttəə} (DU), \textit{{}-kja} (PL);

Adnominalizer:  \textit{{}-a} (ADNZ).
\z

Strictly speaking, the number affixes in \REF{ex:5:14} also function as nominalizers. In the above morphemes, \textit{waa-} \REF{ex:key:1} and \textit{naa-} (2.HON) must conform to the phonological rule discussed in \sectref{sec:2.4.5}, which deletes a vowel in a vowel sequence. The zero morpheme \textit{{}-∅} is ignored in the rule.

\begin{styleBeschriftung}
\textmd{\tabref{tab:key:35}. Phonological changes}
\end{styleBeschriftung}

  Underlying forms    Surface forms

a.  \textit{waa-}  (1)  +  \textit{{}-n} (SG)    >  wa-n  (*waa-n)

    +  \textit{{}-ttəə} (DU)    >  wa-ttəə  (*waa-ttəə)

    +  \textit{{}-∅} (SG)  +  \textit{{}-a} (ADNZ)  >  wa-∅-a  (*waa-∅-a)

b.  \textit{naa-}  (2.HON)  +  \textit{{}-n} (SG)    >  na-n  (*naa-n)

    +  \textit{{}-ttəə} (DU)    >  na-ttəə  (*naa-ttəə)

    +  \textit{{}-∅} (SG)  +  \textit{{}-a} (ADNZ)  >  na-∅-a  (*naa-∅-a)

Adopting the above analysis, I propose the following paradigm. (The following paradigm shows the underlying froms. About the surface form paradigm, see \tabref{tab:key:31}-32 in \sectref{sec:5.1.})

\begin{styleBeschriftung}
\textmd{\tabref{tab:key:36} Paradigm of personal pronominals following analysis 1 (underlying forms)}
\end{styleBeschriftung}

  Singular  Dual  Plural

Nominals  \textit{waa-n}  (1-SG)  \textit{waa-ttəə}  (1-DU)  \textit{waa-kja}  (1-PL)

  \textit{naa-n}   (2.HON-SG)  \textit{naa-ttəə}  (2-DU)  \textit{naa-kja}  (2-PL)

  \textit{ura-∅}   (2.NHON-SG)  \textit{ura-ttəə}  (2.NHON-DU)  \textit{ura-kja}  (2.NHON-PL)

Adnominals  \textit{waa-∅-a}   (1-SG-ADNZ)    \textit{waa-kja-a}  (1-PL-ADNZ)  

  \textit{naa-∅-a}  (2.HON-SG-ADNZ)    \textit{naa-kja-a}  (2-PL-ADNZ)  

  \textit{ura-∅-a}   (2.NHON-SG-ADNZ)    \textit{ura-kja-a}  (2.NHON-PL-ADNZ)  

For nominals, the number distinctions are expressed by -\textit{n}/\textit{{}-∅} (SG) vs. \textit{{}-ttəə} (DU) vs. \textit{{}-kja} (PL). For adnominals, the number distinctions are expressed by \textit{{}-∅} (SG) vs. \textit{{}-kja} (PL). In order to express the singular, the zero morpheme \textit{{}-∅} (SG) appears when it follows \textit{ura-} (2.NHON) or precedes \textit{{}-a} (ADVZ). Although this analysis requires a non-visible zero morpheme, it does make it possible to explain the surface forms of personal pronominals by a regular phonological rule (see \sectref{sec:2.4.5}). Thus, I suggest that this is the best analysis.

\section{Demonstrative words}

A demonstrative word in Yuwan is a deictic word that can indicate a referent that is neither the speaker nor the hearer.

  Morphologically, a demonstrative is made up of a root plus an affix (or affixes). There are six demonstrative roots, and they can be divided into two groups: \REF{ex:key:1} \textit{ku-} (PROX), \textit{u-} (MES), and \textit{a-} (DIST), and \REF{ex:key:2} \textit{ka-} (PROX), \textit{ga-} (MES), and \textit{aga-} (DIST). In both groups, the roots are all bound forms. Each group takes its own set of affixes (see \tabref{tab:key:37}).

  Semantically, demonstratives can distinguish three degrees of distance, i.e. proximal (PROX), mesial (MES), and distal (DIST). These differences correspond to whether the speaker thinks a certain referent is spatially (in a broad sense) related to the speaker (proximal), the hearer (mesial), or others (distal). In addition, the mesial forms, especially \textit{u-rɨ} (MES-NLZ) ‘it,’ have an anaphoric use as in (8-87 a), where \textit{u-rɨ} (MES-NLZ) ‘it’ indicates \textit{boosi} ‘hat’ in the preceding utterance. \textit{u-rɨ} (MES-NLZ) can also indicate an idea that the speaker thinks s/he shares with the hearer as in (9-32 b), where the idea that the occupation of wealth is not good is shared by both of the speaker and the hearer.

Syntactically, demonstrative words can become nominals, adnominals, or adverbs.

\begin{styleBeschriftung}
\textmd{\tabref{tab:key:37}}\textmd{. Demonstratives}
\end{styleBeschriftung}

Word classes    Underlying forms  Meanings    Surface forms

    Root  Affix      Proximal  Mesial  Distal

Nominals    \textit{ku-/u-/a-}  \textit{{}-rɨ} Substance (SG)    ku-rɨ  u-rɨ  a-rɨ

      \textit{{}-rɨ-taa} Substance (PL)    ku-t-taa  u-t-taa  a-t-taa

      \textit{{}-ma} Place    ku-ma  u-ma  a-ma

Adnominals      \textit{{}-n} Neutral    ku-n  u-n  a-n

Nominals    \textit{ka-/ga-/aga-}  \textit{{}-ssa} Amount    ka-ssa  ga-ssa  aga-ssa

      \textit{{}-hɨdubəi}\footnote{\textit{{}-hɨdubəi} has alternate forms: \textit{{}-hɨbəi} and \textit{{}-hɨnbəi}.}  Small amount    ka-hɨdubəi  ga-hɨdubəi  aga-hɨdubəi

Adnominals      \textit{{}-raa} Derogative    ka-raa  ga-raa  aga-raa

      \textit{{}-hɨdon} Large size    ka-hɨdon  ga-hɨdon  aga-hɨdon

Adverbs      \textit{{}-n} Way    ka-n  ga-n  aga-n

Both /rɨ/ (NLZ) and /ttaa/ (NLZ.PL) provide the possibility of expressing a somewhat rude meaning when they are used to indicate human. Thus, they are not likely to be used to refer to people older than the speaker. In that case, a personal pronominal adnominal plus the common noun \textit{cˀju} ‘person’ can be used, e.g. \textit{a-n} \textit{cˀju} (DIST-ADNZ person) ‘that person’ or \textit{a-n} \textit{cˀju=nkja} (DIST-ADNZ person=APPR) ‘those people.’

In the following subsections, I will present examples of \textit{ku-} (PROX), \textit{u-} (MES), and \textit{a-} (DIST) in \sectref{sec:5.2.1.} Next, I will present examples of \textit{ka}{}- (PROX), \textit{ga-} (MES), and \textit{aga-} (DIST) in \sectref{sec:5.2.2.}

\subsection{\textit{ku-} (PROX), \textit{u-} (MES), and \textit{a-} (DIST)}
\label{bkm:Ref361694047}
For the first group, the roots \textit{ku-} (PROX), \textit{u-} (MES), and \textit{a-} (DIST) can indicate places with \textit{{}-ma}.

\ea \label{ex:5:15}   [Context: Remembering a scene from the Pear Film]

  %TM:
\glll  tˀaija  amanan  taccjuppoo,\\
\textit{tˀai=ja}  \textit{a-ma=nan}  \textit{tat-tur-boo}\\
two.person=TOP  DIST-place=LOC1  stand-PROG-CND\\
\glt ‘when the two people were standing there [lit. on that place], ...’ [PF: 090827\_02.txt]
\z

In the above example, the demonstrative nominal \textit{a-ma} (DIST-place) ‘that place’ indicates a place distant from both of the speaker and the hearer.

Secondly, these demonstrative roots can also be nominals with \textit{{}-rɨ}, which can indicate both humans and non-humans. In principle, \textit{{}-rɨ} indicates a single referent as in (5-16 a, c). The plurality is expressed either morphologically by \textit{{}-taa} (PL) or syntactically by \textit{nkja} (APPR). The former is used for human referents as in (5-16 d), and the latter is used for non-human referents as in (5-16 b) in my texts.

\ea \label{ex:5:16}   Non-human referents

 \ea \label{ex:5:16a} Singular

    [Context: Talking about a banyan tree, which was very big but burnt down in an air raid during World War II]

    %TM:
\glll  arəə  siccjuijojaa.  gazimaruja.\\
\textit{a-rɨ=ja}  \textit{sij-tur-i=joo=jaa}  \textit{gazimaru=ja}\\
DIST-NLZ=TOP  know-PROG-NPST=CFM1=SOL  banyan.tree=TOP\\
\glt ‘(You) know that [i.e. the bayan tree], don’t you? The banyan tree.’ [Co: 110328\_00.txt]

 \ex \label{ex:5:b} Plural

    [Context: Speaking about a meeting for old people]

    %TM:
\glll  kjuuja  xxx  arɨnkja  harəə  janmun.   {\textbar}kaihi{\textbar}.\\
\textit{kjuu=ja}    \textit{a-rɨ=nkja}  \textit{haraw-i}  \textit{jar-n=mun}   \textit{kaihi}\\
today=TOP    DIST-NLZ=APPR  pay-INF  COP-PTCP=ADVRS   membership.fee\\
\glt ‘Today, (I) have to pay (things like) that. A membership fee.’ [Co: 120415\_01.txt]

  Human referents

 \ex \label{ex:5:c} Singular

    [Context: Talking about an acquaintance of TM and US]

    %TM:
\glll  arɨn  moosjattujaa.\\
\textit{a-rɨ=n}  \textit{moosɨr-tar-tu=jaa}\\
DIST-NLZ=also  die.HON-PST-CSL=SOL\\
\glt ‘Since that person also died.’ [Co: 110328\_00.txt]

  \ex \label{ex:5:d}  Plural

    [Context: TM had thought to make her daughters prepare some meal for MY and the present author, but she gave it up since she thought the present author would feel too thankful for that.]

    TM:  attankatɨ  jˀuuboo,  attaaga  sjɨ   kəə  sjunban.joo.
                                                                                     
    \textit{a-rɨ-taa=nkatɨ}  \textit{jˀ-boo}  \textit{a-rɨ-taa=ga}  \textit{sɨr-tɨ}  \textit{k-i=ja}  \textit{sɨr-jur-n=ban=joo}
                                                                                     
    DIST-NLZ-PL=DAT2  say-SEQ  DIST-NLZ-PL=NOM  do-SEQ  come-INF=TOP  do-UMRK-PTCP=ADVRS=CFM1

    ‘If (I) said to them [i.e. my daughters], they would do (it) for us, but (you don’t want it, do you?)’

    [Co: 101023\_01.txt]
    \z
\z

In (5-16 a-b), the demonstrative nominals indicate non-humans, i.e. ‘the banyan tree’ in (5-16 a), and ‘a membership fee’ in (5-16 b). The “plurality” of \textit{nkja} in (5-16 b) is similar to that of \textit{{}-kja} as in \REF{ex:5:7} in \sectref{sec:5.1.1.} That is, \textit{nkja} does not necessarily mean genuin plurality. Thus, \textit{a-rɨ=nkja} (DIST-NLZ=APPR) indicates \textit{kaihi} ‘a membership fee’ (see \sectref{sec:6.4.1.1} for more details). In (5-16 c-d), the demonstrative nominals indicate humans, i.e. ‘that person’ in (5-16 c), and ‘my daughters’ in (5-16 d). \textit{{}-rɨ} (NLZ) not followed by any affix indicates a single referent as in (5-16 c) and \textit{{}-taa} (PL) indicates more than a single referents as in (5-16 d).

  In the text data as in (5-16 a-d), \textit{{}-rɨ} (NLZ) not followed by any affix indicates a single (human and non-human) referent; \textit{{}-taa} (PL) follows only human referents, and \textit{nkja} (APPR) (directly) follows only non-human referents. In elicitation, however, there are cases where \textit{{}-rɨ} not followed by any affix indicates more than one referent as in (5-17 a); \textit{{}-taa} (PL) follows non-human referents as in (5-17 b); and \textit{nkja} (APPR) (directly) follows human referents as in (5-17 c).

\ea \label{ex:5:17}  \ea \label{ex:5:17a} \textit{{}-rɨ} (NLZ) indicates more than one (human) referent

    [Context: TM played an imaginary scene where someone (abbreviated as “SO” here) asked TM of the event held at the precedent day.]

    SO:  jubəə  kikjun  cˀjunu  ippai  mandutɨ?

      \textit{jubɨ=ja}  \textit{kik-jur-n}  \textit{cˀju=nu}  \textit{ippai}  \textit{mandur-tɨ}

      last.night=TOP  hear-UMRK-PTCP  person=NOM  many  many-SEQ

      ‘Is there a large audience last night?’

    %TM:
\glll  in,  arɨnu  manduta.\\
\textit{in}  \textit{a-rɨ=nu}  \textit{mandur-tar}\\
yes  DIST-NLZ=NOM  many-PST\\
\glt ‘Yeah, there are many of them.’ [El: 130817]

 \ex \label{ex:5:b} \textit{{}-taa} (PL) follows non-human referents

    [Context: Speaking about some oranges]

    %TM:
\glll  attaa  tutɨ,  kamɨjoo.\\
\textit{a-rɨ-taa}  \textit{tur-tɨ}  \textit{kam-ɨ=joo}\\
DIST-NLZ-PL  take-SEQ  eat-IMP=CFM1\\
\glt ‘Take those (oranges) and eat.’ [El: 130816]

 \ex \label{ex:5:c} \textit{nkja} (APPR) (directly) follows human referents

    [Context: Speaking about a person]

    %TM:
\glll  arɨnkjoo  kondaroo.\\
\textit{a-rɨ=nkja=ja}  \textit{k-on=daroo}\\
DIST-NLZ=APPR=TOP  come-NEG=SUPP\\
\glt ‘Probably, that person will not come.’ [El: 130820]
\z
\z

However, these combinations have never appeared in the text corpus so far.

It should be noted that the plural marker \textit{{}-taa} always induces the following contraction with \textit{{}-rɨ} (NLZ).

\ea \label{ex:5:18}   Contraction of \textit{{}-rɨ} (NLZ) and \textit{{}-taa} (PL) in the demonstratives

  \textit{{}-rɨ} (NLZ)  >  t  /  Demonstrative root  \_  \textit{{}-taa} (PL)
\z

The instances are shown below.

\ea \label{ex:5:19}   Examples of the contraction of \textit{{}-rɨ} (NLZ) and \textit{{}-taa} (PL) in the demonstratives

  \textit{ku-rɨ}  (PROX-NLZ)  +  \textit{{}-taa} (PL)  >  ku-t-taa

  \textit{u-rɨ}  (MES-NLZ)  +      >  ut-t-aa

  \textit{a-rɨ}  (DIST-NLZ)  +      >  at-t-aa
\z

Similarly, the case particles (except for locative case, instrumental case, and comparative case) may induce the contraction with \textit{{}-rɨ} (NLZ).

\ea \label{ex:5:20}   Contraction of \textit{{}-rɨ} (NLZ) and case particles

  \textit{{}-rɨ} (NLZ)  >  C\textit{\textsubscript{i}}  /  Demonstrative root  \_  [C\textit{\textsubscript{i}}      ]\textsubscript{case particle}

                [C\textit{\textsubscript{i}}: stop]
\z

The above rule shows that if the case particle has a stop consonant in its initial position and also follows \textit{{}-rɨ} (NLZ), the //rɨ// assimilates to the following stop of the case particles. I will present the examples where the demonstrative root is \textit{ku-} (PROX).

\ea \label{ex:5:21}   Examples of the contraction of \textit{{}-rɨ} (NLZ) and case particles

  \textit{ku-rɨ}  (PROX-NLZ)  +  \textit{ba}  (ACC)  >  kuppa  (or kubba)

      +  \textit{tu}  (COM)  >  kuttu  

      +  \textit{kaci}  (ALL)  >  kukkaci  

      +  \textit{kara}  (ABL)  >  kukkara  

      +  \textit{ga}  (NOM)  >  kukka  (or kugga)

      +  \textit{ga}  (GEN)  >  kukka  (or kugga)

      +  \textit{gadɨ}  (LMT)  >  kukkadɨ  (or kuggadɨ)
\z

The contraction before the nominative \textit{ga} (NOM) or the accusative \textit{ba} (ACC) never appeared in the text data. However, it was easily produced in elicitation. On the other hand, the contraction before the genitive \textit{ga} (GEN) is obligatory in the text data.

Next, the same demonstrative roots (\textit{ku-}/\textit{u-}/\textit{a-}) can be attached by \textit{{}-n} (ADNZ) and become adnominals.

\ea \label{ex:5:22}   [Context: Talking about an acquaintance of TM and MS] = (4-24 e)
  %TM:
\glll  an  cˀju  daacˀju  jatakai?\\
\textit{a-n}  \textit{cˀju}  \textit{daa+cˀju}  \textit{jar-tar=kai}\\
DIST-ADNZ  person  where+person  COP-PST=DUB\\
\glt ‘Where did that person come from? [lit. That person was where’s person?]’ [Co: 120415\_01.txt]
\z

In \REF{ex:5:21}, \textit{a-n} (DIST-ADNZ) ‘that (one)’ fills the modifier slot of an NP whose head is \textit{cˀju} ‘person.’ These types of demonstrative adnominals can be directly followed by locative cases (except for \textit{zjɨ}).

\ea \label{ex:5:23}   \textit{ku-n}  (PROX-ADNZ)  +  \textit{nən}/\textit{nan}  (LOC1)  >  kunnən/ kunnan

      +  \textit{nəntɨ}/\textit{nantɨ}  (LOC2)  >  kunnəntɨ/ kunnantɨ
\z

The above phenomena may be regarded as headless NPs. The same phenomenon occurs in the case of the interrogative adnominal \textit{dɨ-n} (which-ADNZ) ‘which (one)’ (see (5-40 a) in \sectref{sec:5.3.2}). Semantically, these forms express location, whose meaning is similar to that of \textit{{}-ma} ‘place.’ That is, the meaning of /kunnən/ \textit{ku-n=nən} (PROX-ADNZ=LOC1) ‘here’ (or /kunnan/ \textit{ku-n=nan} (PROX-ADNZ=LOC1) ‘here’) is almost the same as that of \textit{ku-ma=nan} (PROX-place=LOC1) ‘here’ (see also \sectref{sec:6.3.2.6}).

\subsection{\textit{ka}{}- (PROX), \textit{ga-} (MES), and \textit{aga-} (DIST)}

The roots \textit{ka-} (PROX), \textit{ga-} (MES), and \textit{aga-} (DIST) can become nominals, adnominals, and adverbs. There are two nominalizers \textit{{}-ssa} and \textit{{}-hɨdubə}i. The former means the referent is of a specified amount as in (5-24 a); the latter expresses that the referent is of a small amount as in (5-24 b).

\ea \label{ex:5:24}  \ea \label{ex:5:24a} [Context: After telling the story of the Pear Film to SM, TM asked her the extent to which SM understood it.]

    %TM:
\glll  cjoo  gassa  wakajui?\\
\textit{cjoo}  \textit{ga-ssa}  \textit{wakar-jur-i}\\
just  MES-NLZ  understand-UMRK-NPST\\
\glt ‘(Do you) understand just so much?’ [PF: 090827\_02.txt]

 \ex \label{ex:5:24b} [Context: TM shows MS how small of an appetite she has with a gesture; TM: ‘I (always) have half much of the side dish as other people have.’]

    %TM:
\glll  gahɨbəikkwa.\\
\textit{ga-hɨdubəi{}-kkwa}\\
MES-NLZ-DIM\\
\glt ‘So little like that.’ [Co: 120415\_01.txt]
\z
\z

  Moreover, there are two adnominalizers: \textit{{}-raa}, and \textit{{}-hɨdon}. The first one expresses derogative meaning and its head in an NP is always \textit{mun} ‘substance’ as in (5-25 a). The second one expresses the large size of the referents as in (5-25 b).

\ea \label{ex:5:25}  \ea \label{ex:5:25a} [Context: Speaking about an acquaintance]

    %TM:
\glll  agaraa  munna  kisjoonu  cjussanu.\\
\textit{aga-raa}  \textit{mun=ja}  \textit{kisjoo=nu}  \textit{cjus-sa=nu}\\
DIST-DRG.ADNZ  substance=TOP  temper=NOM  strong-ADJSEQ\\
\glt ‘That awful person has a temper.’ [Co: 120415\_01.txt]

 \ex \label{ex:5:b} [Context: Speaking about the community next to where TM lives]

    %TM:
\glll  gahɨdon  tankjanu  atɨ,\\
\textit{ga-hɨdon}  \textit{taa=nkja=nu}  \textit{ar-tɨ}\\
MES-ADNZ  rice.field=APPR=NOM  exist-SEQ\\
\glt ‘There is a very big rice field, and ...’ [Co: 120415\_01.txt]
\z
\z

  There is an adverbializer \textit{-n} (ADVZ), and it can express direction, manner, or quantity. First, I will present the example where \textit{{}-n} (ADVZ) indicates direction as in \REF{ex:5:26}.

\ea \label{ex:5:26}   [Context: TM told MS how she responded to the present author, when the present author had asked her to talk with US for a recording.]

  %TM:
\glll  {\textbar}obasan{\textbar}ga  jˀuuboo,  wanga  agan  ikjussaccjɨ.\\
\textit{obasan=ga}  \textit{jˀ-boo}  \textit{wan=ga}  \textit{aga-n}  \textit{ik-jur-sa=ccjɨ}\\
old.woman=NOM  say-CND  1SG=NOM  DIST-ADVZ  go-UMRK-POL=QT\\
\glt ‘(I said to the present author), “If the old woman [i.e. US] says (it’s OK), I will go there [i.e. the house of US], so (please go there and ask her).”’ [Co: 110328\_00.txt]
\z

The adverbializer \textit{{}-n} (ADVZ) indicates direction with a verb that expresses locational movement as in \textit{ik-} ‘go’ in \REF{ex:5:26}; however, it indicates manner with other types of predicates, e.g., the light verb \textit{sɨr-} ‘do’ as in (5-27 a-b) or adjectives as in (5-27 c).

\ea \label{ex:5:27}  \ea \label{ex:5:27a} [Context: TM was wondering about the place in the picture.]

    TM:  gan  sjuppoo,  kurəə  noogusu..kuja   arannən,  an,  amakai?

      \textit{ga-n}  \textit{sɨr{}-jur-boo  ku-rɨ=ja  noogusuku=ja} \textit{jar-annən}  \textit{a-n}  \textit{a-ma=kai}
                                                                    
      MES-ADVZ  do-UMRK-CND  PROX-NLZ=TOP  Nogusuku=TOP  COP-NEG.SEQ  DIST-ADNZ  DIST-place=DUB

      ‘If (it is) so, this (i.e. the place in the picture) isn’t Nogusuku, but (it) is that place?’

      [Co: 120415\_00.txt]

 \ex \label{ex:5:b} [Context: Speaking about an incident that occurred in the past]

    %TM:
\glll  agan  sjan  hanasija  jiccjaijojaa.\\
\textit{aga-n}  \textit{sɨr{}-tar-n  hanasi=ja  jiccj-sa+ar-i=joo=jaa}\\
DIST-ADVZ  do-PST-PTCP  story=TOP  good-ADJ+STV-NPST=CFM1=SOL\\
\glt ‘(It) may be no problem (to tell) a story like that.’ [Co: 120415\_01.txt]

 \ex \label{ex:5:c} [Context: Speaking about the neighborhood in the old days]

    TM:  agan  hɨɨsan  kɨnkjanu   atanmun.jaa.

      \textit{aga-n}  \textit{hɨɨ-sa+ar-n}  \textit{kɨɨ=nkja=nu}   \textit{ar-tar-n=mun=jaa}
                                                                  
      DIST-ADVZ  big-ADJ+STV-PTCP  tree=APPR=NOM exist-PST-PTCP=ADVRS=SOL

      ‘There used to be such a big tree like that.’

      [Co: 111113\_02.txt]
\z
\z

In (5-27 a-b), the demonstrative adverbs containing \textit{{}-n} (ADVZ) modify the light verb \textit{sɨr-} ‘do.’

Furthermore, there is a case where the particle \textit{bəi} ‘about’ follows the demonstrative adverbs and also \textit{sɨr-} ‘do’ follows them as in (5-28 a-b). In these examples, the adverbializer \textit{{}-n} indicates the quantity (neither direction nor manner).

\ea \label{ex:5:28}  \ea \label{ex:5:28a} [Context: Talking about a butterfly that is similar to the moth]

    TM:  arɨga  nissjagadɨ.  ganbəi  sjɨ  kucjəə  tugaracjɨ,

      \textit{a-rɨ=ga}  \textit{nissj-sa=gadɨ}  \textit{ga-n=bəi}  \textit{sɨr-tɨ}  \textit{kuci=ja}  \textit{tugaras-tɨ}
                                                                                   
      DIST-NLZ=NOM  similar-ADJ=LMT  MES-ADVZ=about  do-SEQ     mouth=TOP  pout-SEQ

      ‘That one is very similar (to the moth). (The size is) about this, and it pouted, and ...’

      [Co: 111113\_01.txt]

 \ex \label{ex:5:b} TM:  unnən  kanbəi  sjan  ...   kanoonu  atattu.    
                                                               
      \textit{u-n=nən}  \textit{ka-n=bəi}  \textit{sɨr-tar-n} \textit{kanoo=nu}  \textit{ar-tar-tu}     
                                                              
      MES-ADNZ=LOC1  PROX-ADVZ=about  do-PST-PTCP   tripod=NOM  exist-PST-CSL    

      ‘There was a tripod (set up to support a kettle) that (has the size) about this there.’

      [Co: 111113\_02.txt]
\z
\z

Interestingly, the combination compoesd of the demonstrative adverbs and the light verb \textit{sɨr-} ‘do’ can also redundantly modify another \textit{sɨr-} ‘do’ as in \REF{ex:5:28}.

\ea \label{ex:5:29}   [Context: TM was changing the angle of a picture since it was hard to see because of the reflection of sunshine.]
  %TM:
\glll  gan  sjɨ  sɨranboo.\\
\textit{ga-n}  \textit{sɨr-tɨ}  \textit{sɨr-an-boo}\\
MES-ADVZ  do-SEQ  do-NEG-CND\\
\glt ‘If (I) don’t do like that, (I cannot see the picture).’ [Co: 120415\_00.txt]
\z

In the above example, it appears that the form /gan sjɨ/ \textit{ga-n} \textit{sɨr-tɨ} (MES-ADVZ do-SEQ) functions as an adverb as if it was \textit{gansjɨ}, and it modifies the entire predicate \textit{sɨr-an-boo} (do-NEG-CND), and there are many examples like that in my text. The mono-clausality of the above example is also attested by the scope of negation. However, I do not regard them as a single adverb, since there is a case where \textit{bəi} ‘about’ intervene between the combination as in (5-28 a-b), and also the demonstrative adverb (composed of \textit{{}-n} (ADVZ)) can modify adjectives as in (5-27 c) only by itself. Therefore, I propose that the combination of a demonstrative adverb (composed of \textit{{}-n} (ADVZ)) and a verb /sjɨ/ (< \textit{sɨr-} ‘do’ + \textit{{}-tɨ} (SEQ)) is on the path towards grammaticalization. In this grammar, they are analyzed as two words, but I do not place a comma after the converb /sjɨ/ (do.SEQ).

  Finally, it should be mentioned that demonstrative roots can make compounds, but that is allowed only for the second group, i.e. \textit{ka-}/\textit{ga-}/\textit{aga-} (PROX/MES/DIST). In addition to the following example, see also (4-26 c) in \sectref{sec:4.2.3.1.}

\ea \label{ex:5:30}   [Context: After talking about a folk tale, TM remembered an utterance said by the person who originally told the folk tale.]
  %TM:
\glll  nusjəə  (kan)  kanagəə  {\textbar}genki{\textbar}ccjɨ.\\
\textit{nusi=ja}  \textit{ka-n}  \textit{ka+nagəə}  \textit{genki=ccjɨ}\\
REF=TOP  PROX-ADVZ  PROX+long  vigorous=QT\\
\glt ‘(He said), “(I) myself am very vigorous like this.”’ [Fo: 090307\_00]
\z

\section{Interrogative words}
\label{bkm:Ref367267321}
An interrogative word is used to ask the hearer an information question (i.e. a “wh-question”). However, an interrogative word also functions as an indefinite word that does not mark a question when it is followed by certain particles. The interrogative use of these words is shown in \sectref{sec:5.3.1}, and the indefinite use is shown in \sectref{sec:5.3.2.}

\subsection{Interrogative use}

Morphologically, some interrogative roots are free forms, i.e. \textit{nuu} ‘what,’ \textit{daa} ‘where,’ and \textit{ɨcɨɨ} ‘when,’ and others are bound forms, i.e. \textit{ta-} ‘who,’ \textit{dɨ-} ‘which,’ and \textit{ikja-} ‘how.’ Syntactically, the interrogatives can become nominals, adnominals, and adverbs. Moreover, interrogative nominals are frequently followed by the focus particle \textit{ga} (see \sectref{sec:10.1.2.2}).

\begin{styleBeschriftung}
\textmd{\tabref{tab:key:38}. Interrogatives (free form made of a single root)}
\end{styleBeschriftung}

Word classes    Forms  Meanings

Nominals    nuu  ‘what’

    daa  ‘where’

    ɨcɨɨ  ‘when’

The interrogative \textit{ɨcɨɨ} ‘when’ tends to be shortened like /ɨcɨ/ in elicitation, which might be influenced by Standard Japanese form /icu/ [it͡su] ‘when.’

\begin{styleBeschriftung}
\textmd{\tabref{tab:key:39}. Interrogatives (bound root + affix)}
\end{styleBeschriftung}

Word classes    Surface forms  Meanings      Underlying forms

            Roots    Affixes

Nominals    taru  ‘who’ (singular)    <  \textit{ta-}  ‘who’  +  \textit{{}-ru}  (NLZ)

    tattaa  ‘who’ (plural)    <      +  \textit{{}-ru-taa}  (NLZ-PL)

Adnominals    taa  ‘whose’    <      +  \textit{{}-a} (ADNZ)

Nominals    dɨru  ‘which’    <  \textit{dɨ-}  ‘which’  +  \textit{{}-ru} (NLZ)

Adnominals    dɨn  ‘which (one)’    <      +  \textit{{}-n} (ADNZ)

Adnominals    ikjasjan  ‘what kind of’    <  \textit{ikja-}  ‘how’  +  \textit{{}-sjan} (ADNZ)

Adverbs    ikjasjɨ  ‘how’    <      +  \textit{{}-sjɨ} (ADVZ)

    ikjasaa  ‘how much; how old’    <      +  \textit{{}-saa} (ADVZ)

In the above table, \textit{{}-ru} (NLZ) + \textit{{}-taa} (PL) is realized as /ttaa/ at the surface form level. It seems that \textit{ta-ru} (who-NLZ) in present Yuwan was *\textit{ta-rɨ} (who-NLZ) in the past. The \textit{{}-rɨ} (NLZ) form is used with demonstrative roots in present Yuwan, e.g., \textit{ku-rɨ} (PROX-NLZ) ‘this.’ There is a lot of correspondence between /ɨ/ in Amami and /e/ in Japanese, and also between /u/ in Amami and /o/ in Japanese (\citealt{HirayamaEtAl1966}: 11). Therefore, \textit{tare} ‘who’ (and \textit{kore} ‘this’) in old Japanese might have the forms corresponding to *\textit{tarɨ} ‘who’ (and *\textit{kurɨ} ‘this’) in the ancestor language of Yuwan. In the present Yuwan, however, the relevant form is \textit{ta-ru} (not \textit{ta-rɨ}). It may be possible that the singular marker \textit{-ru} was attached as an analogy to \textit{dɨ-ru} (which-NLZ), which, I suppose, was the result of metathesis of the vowels in *\textit{du-rɨ} in the ancestor language of Yuwan. The form corresponding to \textit{*du-rɨ} (which-NLZ) in old Japanese is \textit{dore} ‘which.’

  I will present examples of these interrogatives. The first example contains the interrogative \textit{nuu} ‘what,’ which is followed by \textit{ga} (FOC). The \textit{ga} (FOC) does not co-occur with a nominative particle as in \REF{ex:5:31} (see \sectref{sec:10.1}). Other case particles can co-occur with \textit{ga} (FOC) (see an example of the accusative case in (8-76 c) in \sectref{sec:8.4.1.6}).

\ea \label{ex:5:31}   [Context: Trying to remember a scene from the Pear Film]
  %TM:
\glll  ukkara  nuuga  izitakai?\\
\textit{u-rɨ=kara}  \textit{nuu=ga}  \textit{izir-tar=kai}\\
MES-NLZ=ABL  what=FOC  go.out-PST=DUB\\
\glt ‘What did appear then? [lit. What did go out from that?]’ [PF: 090225\_00.txt]
\z

This example shows that the interrogative nominal \textit{nuu} ‘what’ is immediately followed by \textit{ga} (FOC). The focus marker \textit{ga} can also be attached to an interrogative “clause.” In that case, another word may intervene, such as the verb /sjutɨ/ sɨr-jur-tɨ (do-UMRK-SEQ) in \REF{ex:5:32}.

\ea \label{ex:5:32}   [Context: Talking with US about how they played in the past]
  %TM:
\glll  nuu  sjutɨga,  asɨdutakai?\\
\textit{nuu}  \textit{sɨr-jur-tɨ=ga}  \textit{asɨb-tur-tar=kai}\\
what  do-UMRK-SEQ=FOC  play-PROG-PST=DUB\\
\glt ‘What did (we) do (when we) were playing (around here)?’ [lit. ‘Doing what, were (we) playing?’] [Co: 110328\_00.txt]
\z

\textit{nuu} ‘what’ can be used to mean ‘why’ only when it is followed by the converb /sjattu/ \textit{sɨr-tar-tu} (do-PST-CSL).

\ea \label{ex:5:33}   [Context: TM remembered that she had asked her mother about an incantation that old people used to say when an earthquake happens.]
  %TM:
\glll  nuu  sjattu  {\textbar}kjoncɨkɨ{\textbar}ccjɨ  jˀuuboo?\\
\textit{nuu}  \textit{sɨr-tar-tu}  \textit{kjoncɨkɨ=ccjɨ}  \textit{jˀ-boo}\\
what  do-PST-CSL  k.o.incantation=QT  say-CND\\
\glt ‘Why (do you) say \textit{kjoncɨkɨ}?’ [Co: 110328\_00.txt]
\z

It seems that /nuu sjattu/ (what do.PST.CSL) does not indicate the past, and no other morpheme can interveen between them. Thus, it appears to be in the process of grammaticalization to a single adverb \textit{nuusjattu} ‘why.’ In this grammar, I will analyze it as two words, but I do not place a comma after the converb.

  Next, I present examples of \textit{daa} ‘where’ and \textit{ɨcɨɨ} ‘when.’

\ea \label{ex:5:34}  \ea \label{ex:5:a} [Context: TM asked MS where the present author went.]
    %TM:
\glll  nɨsəə  mata  daaciga  izjaru?\\
\textit{nɨsəə}  \textit{mata}  \textit{daa=kaci=ga}  \textit{ik-tar-u}\\
young.man  again  where=ALL=FOC  go-PST-PFC\\
\glt ‘Where did the young man go again?’ [Co: 120415\_01.txt]

 \ex \label{ex:5:b} [Context: Looking at a picture]

    %TM:
\glll  ɨcɨɨ  ucɨcjɨkai?\\
\textit{ɨcɨɨ}  \textit{ucɨs-tɨ=kai}\\
when  take-SEQ=DUB\\
\glt ‘When did (someone) take (the picture)?’ [Co: 120415\_01.txt]
\z
\z

  I present examples of \textit{ta-} ‘who’ followed by \textit{{}-ru} (NLZ), \textit{{}-ru-taa} (NLZ-PL), and \textit{{}-a} (ADNZ) in (5-35 a-c).

\ea \label{ex:5:35}  \ea \label{ex:5:a} [Context: Talking about a picture]

    %TM:
\glll  taruga  mucjɨ\footnotemark  cˀjaru?\\
\textit{ta-ru=ga}  \textit{mut-tɨ}  \textit{k-tar-u}\\
who-NLZ=FOC  have-SEQ  come-PST-PFC\\
\glt ‘Who did bring (the picture here)?’ [Co: 120415\_00.txt]
\footnotetext{Usually, \textit{mut-} ‘have’ becomes /muc/ before \textit{t-}initial affixes (see \sectref{sec:8.2.1.2}), but it happened to become /mu/ in this example.}

 \ex \label{ex:5:b} [Context: Talking about old people who are still healthy; US: ‘About people who are older than ninety years old, ...’]

    %US:
\glll  tattaaga  umoojuru?\\
\textit{ta-ru-taa=ga}  \textit{umoor-jur-u}\\
who-NLZ-PL=FOC  exist.HON-UMRK-PFC\\
\glt ‘Who all would exist?’ [Co: 110328\_00.txt]

 \ex \label{ex:5:c} [Context: There were oranges on the table]

    %TM:
\glll  umanu  nɨkan  taa  nɨkan   xxx?\\
\textit{u-ma=nu}  \textit{nɨkan}  \textit{ta-a}  \textit{nɨkan}  \\
MES-place=GEN  orange  who-ADNZ  orange  \\
\glt ‘(About) the orange there, whose orange (is it)?’ [Co: 101023\_01.txt]
\z
\z

The plural marker \textit{-taa} in (5-35 b) is the same morpheme used with demonstrative roots (see \sectref{sec:5.2}) and address nouns (see \sectref{sec:7.2}). Further, the adnominalizer \textit{{}-a} in (5-35 c) is the same morpheme used with personal pronominal stems in \sectref{sec:5.1.}

  I present examples of \textit{dɨ-} ‘which’ followed by \textit{{}-ru} (NLZ) and \textit{{}-n} (ADNZ) in (5-36 a-b).

\ea \label{ex:5:36}  \ea \label{ex:5:a} %TM:
\glll  dɨru?  naa,  mɨɨga  mjanba.\\
\textit{dɨ-ru}  \textit{naa}  \textit{mɨɨ=ga}  \textit{mj-an-ba}\\
which-NLZ  yet  eye=NOM  see-NEG-CSL\\
\glt ‘Which one? (I) cannot see (by my) eyes yet, so (it is difficult to see the picture).’ [Co: 111113\_01.txt]

 \ex \label{ex:5:b} %TM:
\glll  dɨnnagatɨɨ  izjɨ?\\
\textit{dɨ-n=nagatɨ}  \textit{ik-tɨ}\\
which-ADNZ=neighborhood  go-SEQ\\
\glt ‘Where did (you) go? [lit. Which neighborhood did (you) go?]’ [El: 120917]
\z
\z

The adnominalizer \textit{{}-n} in (5-36 b) is the same morpheme used with demonstrative roots in \sectref{sec:5.2.}

  Finally, I present examples of \textit{ikja-} ‘how,’ followed by \textit{{}-sjan} (ADNZ), \textit{{}-sjɨ} (ADVZ), and \textit{{}-saa} (ADVZ) in (5-37 a-c).

\ea \label{ex:5:37}  \ea \label{ex:5:a} %TM:
\glll  uroo  ikjasjan  sigutu  sjɨ?\\
\textit{ura=ja}  \textit{ikja-sjan}  \textit{sigutu}  \textit{sɨr-tɨ?}\\
2SG=TOP  how-ADNZ  job  do-SEQ\\
\glt ‘What kind of job did you do?’ [El: 111105]

 \ex \label{ex:5:b} [Context: Speaking about a person, who had been to the USA]

    %TM:
\glll  {\textbar}amerika{\textbar}acjəə,  ikjasjɨ  sjɨ,  watajutakai\\
\textit{amerika=kaci=ja}  \textit{ikja-sjɨ}  \textit{sɨr-tɨ}  \textit{watar-jur-tar=kai}\\
America=ALL=TOP  how-ADVZ  do-SEQ  cross-UMRK-PST=DUB\\
\glt ‘How did (he) cross over to America?’ [Co: 110328\_00.txt]

 \ex \label{ex:5:c} %TM:
\glll  nannja  ikjasaa  natɨ  moocjɨ?\\
\textit{nan=ja}  \textit{ikja-saa}  \textit{nar-tɨ}  \textit{moor-tɨ}\\
2.HON.SG=TOP  how-ADVZ  become-SEQ  HON-SEQ\\
\glt ‘How old are you? [lit. How old would you become?]’ [El: 111105]
\z
\z

In the above examples, \textit{{}-sjan} (ADNZ) and \textit{{}-sjɨ} (ADVZ) have the same forms as the verbs /sjan/ \textit{sɨr-tar-n} (do-PST-PTCP) and /sjɨ/ \textit{sɨr-tɨ} (do-SEQ). However, we do not recognize these affixes as verbs for the following two reasons. First, the form /ikjasjɨ/ can modify another \textit{sɨr-} ‘do’ as in (5-37 b), which shows the /sjɨ/ in /ikjasjɨ/ has lost its (supposedly original) meaning of \textit{sɨr-} ‘do.’ Thus, it is in the process of grammaticalization. Second, there are no other words that can be modified only by /ikja/. Thus, /ikja/ should not be regarded as a free form (i.e. an adverb) by itself.

  In the examples presented so far, we have only considered the cases of direct questions. However, interrogative words can also be used for indirect questions. In (5-38 a), the interrogative word \textit{ikja-saa} (how-ADVZ) ‘how much’ does not express a direct question. Similarly, the interrogative word \textit{daa} ‘where’ in (5-38 b) does not express a direct question.

\ea \label{ex:5:38}   Indirect questions

 \ea \label{ex:5:a} %TM:
\glll  wanna  {\textbar}bettarazukee{\textbar}ja  naa  ikjasaa  sjakka  wakarandoo.\\
\textit{wan=ja}  \textit{bettarazuke=ja}  \textit{naa}  \textit{ikja-saa}  \textit{sɨr-tar=ka}  \textit{wakar-an=doo}\\
1SG=TOP  k.o.pickle=TOP  FIL  how-ADVZ  do-PST=DUB  know-NEG=ASS\\
\glt ‘I don’t know how much (I) did [i.e. made] the \textit{bettarazuke} [i.e. k.o. pickles].’ [Co: 101023\_01.txt]

 \ex \label{ex:5:b} [Context: Looking at a picture, TM remembered a man.]

    %TM:
\glll  daanan  wukkaroo,  wakaija  sɨranbajaa.\\
\textit{daa=nan}  \textit{wur=gajaaroo}  \textit{wakar-i=ja}  \textit{sɨr-an-ba=jaa}\\
where=LOC1  exist=DUB  understand-INF=TOP  do-NEG-CSL=SOL\\
\glt ‘(I) don’t know where (he) is.’ [Co: 120415\_01.txt]
\z
\z

In these examples, \textit{ka} (DUB) and \textit{gajaaroo} (DUB) function as the marker of indirect questions, which will be discussed in \sectref{sec:10.4.2} and \sectref{sec:10.4.3.}

\subsection{Indefinite use}

An interrogative word can function as an indefinite word when it is followed by certain particles, namely \textit{ka} (DUB), \textit{gajaaroo} (DUB), and \textit{n} ‘any.’ There are other words that express indefinite meaning, i.e. “indefinite pronouns,” which will be shown in \sectref{sec:7.5.}

  First, I present examples of \textit{ka} (DUB), which can make interrogative nominals have indefinite meaning. The interrogative words \textit{nuu} ‘what’ in (5-39 a), \textit{taru} ‘who’ in (5-39 b), and \textit{daa} ‘where’ in (5-39 c) are all followed by \textit{ka} (DUB) and do not mark an information question but instead indicate indefinite referents. In particular, the first example takes the nominative particle, as in \textit{nuu=ka=nu} (what=DUB=NOM), which does not occur when \textit{nuu} ‘what’ is used for questions since it takes the focus particle \textit{ga} (FOC) in that case, omitting the nominative particle (see \sectref{sec:5.3.1}). The interrogatives, \textit{ka} (DUB), and the corresponding expression in the free translation are underlined below.

\ea \label{ex:5:39}   Intrrogative nominals + \textit{ka} (DUB)
 \ea \label{ex:5:39a} [Context: TM said to MS that her son was always busy.]

    %TM:
\glll  {\textbar}dojoo{\textbar}.  {\textbar}nicijoo{\textbar}.  jazin  nuukanu  ai.\\
\textit{dojoo}  \textit{nicijoo}  \textit{jazin}  \textit{nuu=ka=nu}  \textit{ar-i}\\
Saturday  Sunday  necessarily  what=DUB=NOM  exist-NPST\\
\glt ‘Saturday. Sunday. There is always something.’ [Co: 120415\_01.txt]

 \ex \label{ex:5:39b} [Context: Talking about old people who are still healthy; US: ‘About people who are older than ninety years old, who all would exist?’]

    %US:
\glll  taruka  umoojumɨ?\\
\textit{ta-ru=ka}  \textit{umoor-jur-mɨ}\\
who-NLZ=DUB  exist.HON-UMRK-PLQ\\
\glt ‘Is there anyone (who is older than ninety years old)?’ [Co: 110328\_00.txt]

 \ex \label{ex:5:39c} [Context: TM explained to MY why she had called her.]

    %TM:
\glll  uran  daacika  ikjarɨncjɨga, ...\\
\textit{ura=n}  \textit{daa=kaci=ka}  \textit{ik-arɨr-n=ccjɨ=ga}\\
2.NHON.SG=DAT1  where=ALL=DUB  go-PASS-PTCP=QT=FOC\\
\glt ‘(I thought) that (I) would suffer from your going somewhere, (so I called you.)’ [Co: 101020\_01.txt]
\z
\z

It should be noted that \textit{ka} (DUB) does not need to follow directly an interrogative word. For example, it can follow a case particle \textit{kaci} (ALL) as in (5-39 c).

  Secondly, I present examples of \textit{gajaaroo} (DUB), which can also turn interrogatives into indefinite words. The interrogatives, \textit{gajaaroo} (DUB), and the corresponding expression in the free translation are underlined below.

\ea \label{ex:5:40}  \ea \label{ex:5:40a} [Context: Looking at pictures]

    %TM:
\glll  dɨnnangajaaroo  xxx  uttaaga  {\textbar}sansankudo{\textbar}   sjun  turonkjanu  izituttɨjaa.\footnotemark\\
\textit{dɨ-n=nan=gajaaroo}    \textit{u-rɨ-taa=ga}  \textit{sansankudo}   \textit{sɨr-tur-n}  \textit{turoo=nkja=nu}  \textit{izir-tur-tɨ=jaa}\\
which-ADNZ=LOC1=DUB    MES-NLZ-PL=NOM  k.o.ceremony   do-PROG-PTCP  scene=APPR=NOM  go.out-PROG-SEQ=SOL      \\
\glt ‘Somewhere, there was a scene (in the picture) where they were doing Sansankudo.’ [Co: 120415\_00.txt]
\footnotetext{The final //r// of \textit{{}-tur} (PROG) drops before \textit{{}-tɨ} (SEQ) in principle (see §\ref{bkm:Ref347175824}); however, it assimilates with the following //t// in this example.}

 \ex \label{ex:5:b} [Context: Looking at pictures of the shopping street in the village]

    %TM:
\glll  nuucjɨgajaaroo  kacjəəttujaa.\\
\textit{nuu=ccjɨ=gajaaroo}  \textit{kak-təər-tu=jaa}\\
what=QT=DUB  write-RSL-CSL=SOL\\
\glt ‘Something has been drawn (on the sign board of the store).’ [Co: 120415\_00.txt]
\z
\z

Both of the above examples include interrogative words, but they do not express questions when they are followed by \textit{gajaaroo} (DUB).

  Finally, I will show the examples of the limiter particle \textit{n} ‘any,’ which can make interrogatives have indefinite meaning (see also \sectref{sec:10.1.3}). The interrogatives, \textit{n} ‘any,’ and the corresponding expression in the free translation are underlined below.

\ea \label{ex:5:41}   Interrogatives directly followed by \textit{n} ‘any’
 \ea \label{ex:5:41a} [Context: Speaking about a person in a picture; TM: ‘There are no classmates of her here.’]

    %TM:
\glll  tarun  wuran.  dusi.\\
\textit{ta-ru=n}  \textit{wur-an}  \textit{dusi}\\
who-NLZ=any  exist-NEG  friend\\
\glt ‘There is not anyone (of her friends). (There is no) friend (of her). [Co: 120415\_00.txt]

 \ex \label{ex:5:b} [Context: Remembering the flower arrangement class]

    %TM:
\glll  ɨcɨn  waakjoo  ikjutɨ,  urɨ  sjutassɨga.\\
\textit{ɨcɨɨ=n}  \textit{waakja=ja}  \textit{ik-jur-tɨ}  \textit{u-rɨ}  \textit{sɨr-jur-tar-sɨga}\\
when=any  1PL=TOP  go-UMRK-SEQ  MES-NLZ  do-UMRK-PST-POL\\
\glt ‘Anytime I used to go (to the class) and do that.’ [Co: 120415\_01.txt]

 \ex \label{ex:5:c} [Context: Remembering a custom in the old days, where adults made children stay awake on New Year’s Eve.]

    %TM:
\glll  ikjanagən  hɨɨracjuta.\\
\textit{ikja+nagəə=n}  \textit{hɨɨr-as-tur-tar}\\
how+long=any  awake-CAUS-PROG-PST\\
\glt ‘However long (it is), (adults) were making (us) stay awake.’ [Co: 111113\_02.txt]
\z
\z

Here, /ta-ru=n/ (who-NLZ=any) means ‘anyone’ as in (5-41 a), and /ɨcɨ=n/ (when=any) means ‘anytime’ as in (5-41 b). In addition, a compounded form such as \textit{ikja+nagəə} (how+long) can be followed by \textit{n} ‘any,’ which means ‘however long (it is)’ as in (5-41 c). Furthermore, there are cases where \textit{n} ‘any’ does not directly follow an interrogative word, but it still turns the interrogative word into an indefinite word. The following three examples illustrate those cases.

\ea \label{ex:5:42}   Interrogatives indirectly followed by \textit{n} ‘any’
 \ea \label{ex:5:42a} [Context: Talking about a man who owned a river boat.]

    %TM:
\glll  daacin  ikjanba.\\
\textit{daa=kaci=n}  \textit{ik-an-ba}\\
where=ALL=any  go-NEG-CSL\\
\glt ‘(The man) did not go anywhere, so (he should have been there).’ [Co: 111113\_01.txt]

 \ex \label{ex:5:b} [Context: Remembering that flies used to swarm on the meal in the old days; MS: We didn’t feel uncomfortable about that, did you?’]

    %TM:
\glll  nuucjɨn  umuwan\\
\textit{nuu=ccjɨ=n}  \textit{umuw-an}\\
what=QT=any  think-NEG\\
\glt ‘(I) don’t think [i.e. didn’t feel] anything (uncomfortable about that).’ [Co: 111113\_02.txt]

 \ex \label{ex:5:c} %TM:
\glll  nuu  jatɨn,  sɨki  jatattu,\\
\textit{nuu}  \textit{jar-tɨ=n}  \textit{sɨki}  \textit{jar-tar-tu}\\
what  COP-SEQ=any  like  COP-PST-CSL\\
\glt ‘(My mother) likes anything, so ...’ [Co: 111113\_02.txt]
\z
\z

In (5-42 a), the allative case \textit{kaci} (ALL) intervenes between \textit{daa} ‘where’ and \textit{n} ‘any.’ In (5-42 b), the particle \textit{ccjɨ} (QT) intervenes between \textit{nuu} ‘what’ and \textit{n} ‘any.’ In (5-42 c), the verb /jatɨ/ \textit{jar-tɨ} (COP-SEQ) intervenes between \textit{nuu} ‘what’ and \textit{n} ‘any.’
