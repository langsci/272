\addchap{Transcription methods}

These transcription methods are inspired by those of Stuart \citet[7--9, 43--52]{McGill2009}.

\section*{Interlinear examples}

Each example is composed of four tiers: the surface tier (the phonemic representation), the underlying tier (the morphophonemic representation), the tier for morpheme-by-morpheme gloss, which conforms to the convention of the Leipzig Glossing Rules\footnote{These are available at \url{https://www.eva.mpg.de/lingua/pdf/Glossing-Rules.pdf}.} and the tier for free translation provided by the present author. The surface tier does not have morpheme boundaries. This way, it is possible to handle fusions and morphophonological alternations with interlinear morphemic glosses.

\ea
\glll mukasinu janagɨjaaccjəə nən.jaa.\jambox{surface tier}\\
      mukasi=nu  janagɨ+jaa=ccjɨ=ja  nə-an=jaa\jambox{underlying tier}\\
      old.days=	extsc{gen} dirty+house=	extsc{qt}=	extsc{top} exist-	extsc{neg}=	extsc{sol}\jambox{gloss tier}\\
\glt  ‘There is not (a house) like a dirty [i.e. outdated] house of the old days.’\jambox{free translation tier}
\z 

The following markers are used in a surface (if it is deleted, in an underlying) tier.

\begin{description}[font=\normalfont]
\item[,]  after an interjection or an adverbial clause; before the hearer’s nod assent; enclosing an inserted expression
\item[.]  after a sentence (not within a word); between syllable boundaries (within a word)\footnote{As mentioned in §\ref{bkm:Ref347173344}, there is no sequence [ɴ.V] (V: vowel) within a phonological word in Yuwan, so any sequence of /VnV/ within a phonological word in the surface form would be /V.nV/ [V.nV], not /Vn.V/ [Vɴ.V].}
\item[?]  after an interrogative sentence
\item[!]  after an imperative sentence
\item[..]  short pause
\item[...]  long pause
\item[xxx]  unintelligible speech
\item[(  )] enclosing a defective utterance or a misstatement
\item[{\textbar}  {\textbar}]  enclosing standard Japanese
\end{description} 

Additionally, the underlying tier is provided in \textit{italics}, the free translation is enclosed within single quotation marks, and information inferable from the context may be added with round brackets in the free translation. Some morphemes can be translated into more than one meaning (or function) in English, i.e. polysemy. In that case, we gloss it in the following order \citep[cf.][11--12]{Lehmann2004}: (1) if we can abstract the polysemous meanings into one meaning, we use the abstract meaning as its gloss; (2) if we cannot do this, we gloss the relevant meaning in each example. In the second case, I sacrificed the consistency of the glossing and the form, because it is helpful for the reader to know the correspondence between the glossing and the free translation. Finally, in the free translation, ‘...’ means there is a remaining portion of the sentence that has been left out.

In many cases, context is supplied for an example, and it is enclosed in square brackets on the upper side of examples. Paraphrases in English (with speaker 	extsc{id}) in quotation marks may follow the description of the context. In addition, if other kinds of information, e.g., syntactic constructions, are needed, another line may be added below the glossing line \citep[cf.][4--5]{Lehmann2004}.

\ea\relax
[Context: 	extsc{tm} and 	extsc{ms} were looking at the beams of TM’s house; MS: ‘There are few houses (that have the beams) like these.’]\\
	\textsc{tm}: \gllll mukasinu janagɨjaaccjəə nən.jaa.\\
          mukasi=nu janagɨ+jaa=ccjɨ=ja nə-an=jaa\\
          \{[old.days=	extsc{gen}] [dirty+house]\}=	extsc{qt}=	extsc{top} exist-	extsc{neg}=	extsc{sol}\\
          \{[Modifier] [Head]\}\textsubscript{NP}\\
   \glt\hspaceThis{\textsc{tm}:} ‘There is not (a house) like a dirty [i.e. outdated] house of the old days.’ [Co: 111113\_01.txt]

\z

Further, each example will be shown with the data of its source, i.e. genre of data and the file name of source, in the square brackets on the lower right side of examples (for more details on the abbreviations used to indicate the source data, see §\ref{bkm:Ref347173399}).

\section*{In-text example}

An in-text example is placed in the following order: surface forms in slash marks, underlying forms in \textit{italics}, morpheme-by-morpheme glosses, and free translation in single quotation marks, as in /janagɨjaaccjəə/ \textit{janagɨ+jaa=ccjɨ=ja} (dirty+house=	extsc{qt}=	extsc{top}) ‘like a dirty house.’ If we do not need to show a morpheme boundary, we will use a period in glosses to imply there are a few morphemes, such as /janagɨjaaccjəə/ (dirty.house.QT.TOP). Contrary to interlinear examples, the surface forms of in-text examples may show their morpheme boundaries if the need arises, such as /janagɨ+jaa=ccjə=ə/ (dirty+house=QT=TOP). Sometimes, IPA symbols are used to access the concrete sounds in square brackets, e.g., [jɑ̟nɑ̟gɨjɑ̟ːtt͡ɕɜː]. The underlying forms (i.e. morphophonemic) may be expressed not only with italics but also double slash marks, such as //ja//. Forms in the middle stage of morphophonemic processes are also shown in double slash marks. If the relevant form is not a grammatical word, i.e. bound roots or affixes like \textit{kam-} ‘eat’ or \textit{{}-ɨ} (	extsc{imp}), a hyphen is attached to mark the place of morpheme boundaries.

\section*{Orthography}

Yuwan has mainly six vowels [i, u, o̞, ɑ̟, ɨ, ɜ] (see §\ref{bkm:Ref312504424}). In many of the previous studies of Amami dialects (including that of Yuwan), the first four vowels have been transcribed into ‘i, u, o, a (\textit{a} in italic)’ but the last two vowels have been transcribed as ‘ï’ [ɨ] and ‘ë’ [ɜ]. In this grammar, [ɨ] and [ɜ] are transcribed as ‘ɨ’ and ‘ə’ since (1) they do not need diacritics, and (2) [ə] is closer to [ɜ] than [ë] (but we do not use ‘ɜ’ because it is not as familiar as ‘ə’).

Furthermore, Yuwan has glottalized consonants such as [ʔj, ʔw, ʔm, ʔn, ʔ͡t, ʔ͡k, ʔ͡ʨ], which have been transcribed as ‘ʔC’ or ‘Cˀ’ (C is any consonant), depending on the researcher’s interpretation of those phones. The latest IPA diacritics\footnote{Available at \url{http://www.langsci.ucl.ac.uk/ipa/IPA\_chart\_(C)2005.pdf}.} do not have ‘ˀ’ even though this diacritic is very useful to describe these consonants. In this grammar, the glottalized consonants are regarded as single phonemes (see §\ref{bkm:Ref347173461}) and transcribed as ‘jˀ, wˀ, mˀ, nˀ, tˀ, kˀ, and cˀ.’

Finally, Yuwan has homorganic nasals, and if we cannot infer their underlying form from the paradigmatic information, we recognize them as archiphonemes \citep[46--49]{Lass1984}. Yuwan has /m/ and /n/, which are homorganic. For example, in /jum-an/ [ju.mɑ̟ɴ] (read-	extsc{neg}) ‘do not read’ and /jum-gadɨ/ (read -until) [juŋ.gɑ̟.dɨ] ‘until (someone) reads,’ /m/ can be [m] or [ŋ] depending on the following phonemes. Similarly, in /ɨn=un/ [ʔɨ.nu.ɴ] (dog=also) ‘also a dog’ and /ɨn=gadɨ/ [ʔɨŋ.gɑ̟.dɨ] (dog=	extsc{lmt}) ‘as well as dogs,’ /n/ can be [n] or [ŋ] depending on the following phonemes. [ʔɑ̟m.mɑ̟ː] ‘mother,’ however, is made up of a single root, so we cannot know whether its first [m] would be /m/ or /n/. In this case, we recognize the existence of archiphoneme /N/ and avoid choosing the unique underlying phoneme. In this grammar, the archiphoneme is transcribed as ‘n,’ since the use of /N/ implies the exsistence of a phoneme other than /m/ and /n/. Thus, [ʔɑ̟m.mɑ̟ː] is \textit{anmaa} (see §\ref{bkm:Ref347174390} for more details). The other symbols used in this grammar coincide with their phonetic representations (or commonly accepted phonemic representations) (see also §\ref{bkm:Ref347174415}).
