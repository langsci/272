\chapter{Grammatical relations}\label{chap:3}\hypertarget{RefHeadingToc395696987}{}
In Yuwan, grammatical relations, i.e. subject and object, cannot be clearly defined, but there are a few phenomena that are easily explained if we assume grammatical relations. We will examine the phenomena related to subjects in §\ref{bkm:Ref350115473}, and objects in §\ref{bkm:Ref350115710}.

\section{Subject}
\label{bkm:Ref350115473}\hypertarget{RefHeadingToc395696988}{}
The subject in Yuwan is defined as the referent that receives respect indicated by honorific verbs.

{\smallex
\ea  Subjects with honorific verbs\\
\ea TM:\hfill\relax[El: 120924] \\ \gllll  an  {jaaja}  {sinsjeiga}  {umoojuncjɨdoo.}\\
                {\itshape a-n}  {\itshape jaa=ja}  \Emph{\itshape sinsjei=ga}  {\itshape umoor-jur-n=ccjɨ=doo}\\
                {\DIST-\ADNZ}  {house=\TOP}  {[teacher]=\NOM}  {[exist.\HON-\UMRK-\PTCP]=\QT=\ASS}\\
                     {[Subject]} {} {} {[Honorific verb]}\\
\glt ‘(I heard) that there was a teacher in that house.’ 
\ex TM: \hfill\relax [El: 120924]\\ \gllll {\textsuperscript{\#}an}  {jaaja}  {warabɨnu}  {umoojuncjɨdoo.}\\
               {\itshape a-n}  {\itshape jaa=ja}  \Emph{\itshape warabɨ=nu}  {\itshape umoor-jur-n=ccjɨ=doo}\\
                {\DIST-\ADNZ}  {house=\TOP}  {[child]=\NOM}  {[exist.\HON-\UMRK-\PTCP]=\QT=\ASS}\\
                {[Subject]} {}  {}  {[Honorific verb]}\\

\ex TM:\hfill\relax[El: 120924]\\ \gllll  an  {jaaja}  {warabɨnu}  {wuncjɨdoo.}\\
            {\itshape a-n}  {\itshape jaa=ja}  \Emph{\itshape warabɨ=nu}  {\itshape wur-n=ccjɨ=doo}\\
            {\DIST-\ADNZ}  {house=\TOP}  {[child]=\NOM}  {[exist-\PTCP]=\QT=\ASS}\\
             {[Subject]} {}   {}  {[Non-honorific verb]}\\
   \glt  ‘(I heard) that there was a child in that house.’ 
\z
\z}

In (3-1 a), the honorific verb \textit{umoor-} (exist.\HON) shows respect to \textit{sjensjei} ‘teacher,’ which is the subject of the sentence. In (3-1 b), the honorific verb \textit{umoor-} (exist.\HON) shows respect to \textit{warabɨ} ‘child,’ but it is not natural for TM, who is eighty-nine years old, to show respect to a child, so this sentence cannot be possible. However, if the verb is a non-honorific verb, i.e. \textit{wur-} ‘exist,’ the sentence is problem-free as in (3-1 c).

In the above examples, all of the subjects have the nominative case. Thus, one may think that we do not need the term “subject,” but only “nominative NP” instead. We need the term “subject,” however, since there is a case where the “subject” does not take the nominative case. The following examples show that case. In these examples, possessional meaning is expressed by the existential construction, where the expression that literally means ‘About X, there is Y’ actually means ‘X has Y.’

{\smallex
\ea Existential construction expressing possessional meaning
\ea {TM:} \hfill\relax[El: 120924] \\ 
        \gllll   an  {sinsjeija}  {jɨɨinu}  {umoojuncjɨdoo.}\\
                 {\itshape a-n}  {\itshape sinsjei=ja}  {\itshape jɨɨi=nu}  {\itshape umoor-jur-n=ccjɨ=doo}\\
                 [{\DIST-\ADNZ}  {teacher]=\TOP}  {brother=\NOM}  {[exist.\HON-\UMRK-\PTCP]=\QT=\ASS}\\
                 {[Subject]} {} {}   {[Honorific verb]}\\
    \glt ‘(I heard) that the teacher has a brother.’\\\relax
         [lit. ‘(I heard) that about the teacher, there is a brother.’]
\ex {TM:} \hfill\relax [El: 120924] \\
      \gllll    {\textsuperscript{\#}an}  {warabɨja}  {jɨɨinu}  {umoojuncjɨdoo.}\\
                   {\itshape a-n}  {\itshape warabɨ=ja}  {\itshape jɨɨi=nu}  {\itshape umoor-jur-n=ccjɨ=doo}\\
                   [{\DIST-\ADNZ}  {child]=\TOP}  {brother=\NOM}  {[exist.\HON-\UMRK-\PTCP]=\QT=\ASS}\\
                   [Subject]   {}   {}   {[Honorific verb]}\\
\ex  {TM:}\hfill\relax[El: 120924] \\
\gllll an  {warabɨja}  {jɨɨinu}  {wuncjɨdoo.}\\
   {\itshape a-n}  {\itshape warabɨ=ja}  {\itshape jɨɨi=nu}  {\itshape wur-n=ccjɨ=doo}\\
   [{\DIST-\ADNZ}  {child]=\TOP}  {brother=\NOM}  {[exist-\PTCP]=\QT=\ASS}\\
   [Subject] {} {}     {[Non-honorific verb]}\\
   \glt ‘(I heard) that the child has a brother.’ \\\relax
        [lit. ‘(I heard) that about the child, there is a brother.’]\\
\z
\z} 

In the above examples, the NPs that take the nominative case have the same composition, i.e. \textit{jɨɨi=nu} (brother=\NOM). However, the acceptability of those examples is different. In fact, the initial NPs that take the topic particle \textit{ja} determine the acceptability of those sentences. In (3-2 a), the honorific verb \textit{umoor-} (exist.\HON) shows respect to \textit{a-n} \textit{sinsjei} ‘the teacher,’ which is the sentence-initial NP and also the subject of the sentence. In (3-2 b), the sentence-initial NP, which is also the subject of the sentence, is \textit{a-n} \textit{warabɨ} ‘the child,’ and it is not natural for TM to show respect to a child with honorific verbs. Thus, (3-2 b) is not acceptable. However, in (3-2 c), the verb is not an honorific verb: \textit{wur-} ‘exist.’ Therefore, \textit{warabɨ} ‘child,’ which is the sentence-initial NP and also the subject of the sentence, is acceptable.

In conclusion, it is possible to recognize the existence of the grammatical category “subject” in Yuwan. Here, the term “subject” is selected because of its likelihood to become the agent of a sentence \citep[cf.][136]{Andrews2007}. We cannot, however, identify the subject in every sentence, because sentences in Yuwan do not necessarily include honorific verbs. In other words, the criterion of the subject established by the honorific verb is not an ironclad criterion.

\section{Object}
\label{bkm:Ref350115710}\hypertarget{RefHeadingToc395696989}{}
In Yuwan, the recognition of the grammatical relation “object” is much more difficult than that of the subject. However, it is very useful to use this term in order to understand the grammar of Yuwan. For example, the locative case \textit{nan} (\LOCOne) can mark the place where the subject of an intransitive verb or the object of a transitive verb exists (or contacts) (see §\ref{bkm:Ref367136821} for more details). In this case, we should recognize the grammatical relation “object,” or at least “P,” which is a patient-like argument of a transitive clause. Another example that shows the usefulness of the term “object” is shown in (6-75 c-d) in \sectref{sec:key:6.3.2.11.}
