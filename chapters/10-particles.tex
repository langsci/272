\chapter{Particles}\label{chap:10}

This chapter describes the particles in Yuwan. All of the particles are clitics, but not vice versa since the formal nouns also belong to clitics but they are nominals (see \sectref{sec:6.2.2}). Particles in Yuwan can be classified into the following groups: case particles, limiter particles, conjunctive particles, clause-final particles, utterance-final particles A, and utterance-final particles B. They are distinguished by the units that the particles attach to and by the functions of the units after the particles attached to them. Additionally, it is distinctive whether the units attached by the particles are necessarily embedded into the superordinate clause.

\begin{sidewaystable}
\caption{Particles in Yuwan. “VPP” indicates the verbal predicate phrase; “Adv.” indicates the adverbial clause; “+/\textminus” means that some particles or some clauses cannot satisfy the criteria.\label{tab:96}
% \hypertarget{RefHeadingToc395697361}{}
}
\resizebox{\textwidth}{!}{\begin{tabular}{lccccccc}
\lsptoprule
& \multicolumn{6}{c}{The units and functions of the particles’ syntactic hosts} & Embeddedness\\\cmidrule(lr){2-7}
\multicolumn{1}{r}{Unit} & \multicolumn{2}{l}{NP} & Non-final verb in VP & \multicolumn{2}{l}{Clause} & Utterance & \\\cmidrule(lr){2-7}
\multicolumn{1}{r}{Functions} & NP Modifier & Argument &  & Main & Adv. &  & \\\midrule
Case particles & + & + & \textminus\footnote{Only the nominative case can follow the lexical verb in \textsc{av}C (see §\ref{bkm:Ref366360662}).} & \textminus & \textminus & \textminus & +\\
Limiter particles & +\footnote{A few limiter particles, e.g., \textit{n} ‘also’ or \textit{nən} ‘such as,’ cannot occur with the modifier NP.} & + & + & \textminus & +/\textminus & \textminus & +\\
Conjunctive particles & \textminus & \textminus & \textminus & \textminus & + & \textminus & +\\
Clause-final particles & \textminus & \textminus & \textminus & + & +/\textminus & \textminus & \textminus\\
Utterance-final particles A & \textminus & \textminus & \textminus & \textminus & \textminus & + & +\\
Utterance-final particles B & \textminus & \textminus & \textminus & \textminus & \textminus & + & \textminus\\
\lspbottomrule
\end{tabular}} 
\end{sidewaystable}

The above table shows that case particles and limiter particles are similar to each other. However, the case particles cannot follow the verb in the verbal predicate phrase (with the exception of the nominative case), but the limiter particle can. The unit composed of the conjunctive particle and the preceding clause functions as an adverbial clause. The clause followed by the clause-final particle functions as the main claue. Both of the utterance-final particles A and the utterance-final particles B follow an utterance, and the units followed by the utterance-final particles A function as the complement of the superordinate clause, but the units followed by the utterance-final particles B do not.

The case particles were examined in \sectref{sec:6.3}. Therefore, the remaining particles will be discussed in the following sections. The limiter particles are discussed in \sectref{sec:10.1}. The conjunctive particles are discussed in \sectref{sec:10.2}. The clause-final particles are discussed in \sectref{sec:10.3}. The utterance-final particles A are discussed in \sectref{sec:10.4}. Finally, the utterance-final particles B are discussed in \sectref{sec:10.5}.

\section{Limiter particles}\label{sec:10.1}

Yuwan has the limiter particles seen in \tabref{tab:97}. The limiter particles can be hosted by NPs, verbs in the verbal predicate phrases, or adverbial clauses.

\begin{table}
\caption{Limiter particles\label{tab:97}}
\begin{tabular}{ll}
\lsptoprule
Form  & Meaning or translation\\\midrule
\textit{ja}    & Topic\\
\textit{du}    & Focus (not information question)\\
\textit{ga}    & Focus (including information question)\\
\textit{n}     & ‘also; even; ever’\\
\textit{bəi}   & ‘only; always; about’\\
\textit{gadɨ}  & Limitative\\
\textit{nkja}  & Approximative\\
\textit{kusa}  & ‘the very (one)’\\
\textit{səəka} & ‘only’\\\lspbottomrule
\end{tabular}
\end{table}

The restriction on the co-occurence with the case particles should be mentioned. \textit{ja} (\textsc{top}), \textit{du} (\textsc{foc}), \textit{ga} (FOC), and \textit{n} ‘also; evern; ever’ cannot co-occur with the nominative case. \textit{nən} ‘such as’ cannot co-occur with any case particle. In the following sections, I will present examples of each limiter particle in turn.

\subsection{Topic particle \textit{ja}}\label{sec:10.1.1}

The topic particle \textit{ja} is frequently fused with the preceding short vowel, and always assimilates to the preceding nasal consonants. These morphophonological alternations are discussed in \sectref{sec:10.1.1.1}. The syntax and semantics of \textit{ja} (\textsc{top}) will be discussed in \sectref{sec:10.1.1.2}.

\subsubsection{Morphophonology of topic particle \textit{ja}}\label{sec:10.1.1.1}

The topic particle \textit{ja} induces either fusion or nasalization depending on the morphophonological environment of the preceding stems.

First, if the topic particle \textit{ja} follows a vowel (not a vowel sequence), frequently several types of vowel fusion occur. If not, i.e. after long vowels or diphthongs, \textit{ja} retains its form. Please note that the fusion of //ci, si, zi// and \textit{ja} requires a little attention because it forms /Cjəə/ (not */Cəə/).

\begin{exe}
\ex \label{ex:10.1} Rule shemata\\
  \begin{xlist}
  \exi{}Front vowel\footnote{There is no lexeme that ends with /ə/ (see §\ref{bkm:Ref367392475}). Additionally, there is only one lexeme (excluding \textit{ude} ‘hey’ and \textit{doosje} ‘maybe’) that ends with /e/ and is fused with \textit{ja} (\textsc{top}), i.e. \textit{nazje} (or \textit{nasje}) ‘Naze (name of place).,’ However, it is difficult to decide whether the phone is [nɑ̟(d͡)ʑeː] or [nɑ̟(d͡)ʑɜː], and audio-instrumental research should be done in the future. The same point can be made about the fusion with the allative case (or ablative case) (see §\ref{bkm:Ref365151806} and §\ref{bkm:Ref365151812}).}
  \ex  //  C  i  //  +  \textit{ja} (\textsc{top})  >  /Cjəə/\\\relax     [C is //c, s, z//]
  \ex  //  C  i  //  +  \textit{ja} (\textsc{top})  >  /Cəə/\\\relax    [C is not //c, s, z//]
  \exi{} Mid vowel
  \ex  //  C  ɨ  //  +  \textit{ja} (\textsc{top})  >  /Cəə/
  \exi{}Back vowels
  \ex  //   C  \{ u o a \}  //  +  \textit{ja} (\textsc{top})  >  /Coo/
  \exi{}Long vowels or diphthongs
  \ex  //  V  V  //  +  \textit{ja} (\textsc{top})  >  /VVja/
  \end{xlist}
\end{exe}

\begin{exe}
\ex\label{ex:10.2} Examples
\ea Front and mid vowels  
    \begin{tabbing}
    \textit{wunagu}\hspace{\tabcolsep}\=‘(personal name)’\hspace{\tabcolsep}\=+  \textit{ja} (\textsc{top})\hspace{\tabcolsep}\=\hspace{\tabcolsep}>\hspace{\tabcolsep}\=/kucjəə/\hspace{\tabcolsep}\=(*/kucəə/)\kill
    \textit{kuci}  \> ‘mouth’  \> +  \textit{ja} (\textsc{top}) \> > \> /kucjəə/  \> (*/kucəə/)\\
    \textit{nusi}  \> (\textsc{rfl})    \>                      \> > \> /nusjəə/  \> (*/nusəə/)\\
    \textit{tuzi}  \> ‘wife’   \>                      \> > \> /tuzjəə/  \> (*/tuzəə/)\\
    \textit{kˀubi} \>  ‘neck’  \>                      \> > \> /kˀubəə/  \>           \\
    \textit{kurɨ}  \> ‘this’   \>                      \> > \> /kurəə/   \>           \\
    \end{tabbing}
\ex  Back vowels  
    \begin{tabbing}
    \textit{wunagu}\hspace{\tabcolsep}\=‘(personal name)’\hspace{\tabcolsep}\=+  \textit{ja} (\textsc{top})\hspace{\tabcolsep}\=\hspace{\tabcolsep}>\hspace{\tabcolsep}\=/kucjəə/\hspace{\tabcolsep}\=(*/kucəə/)\kill
    \textit{wunagu} \>  ‘woman’          \> +  \textit{ja} (\textsc{top}) \> > \> /wunagoo/ \\
    \textit{juuto}  \> ‘(personal name)’ \>                      \> > \> /juutoo/  \\
    \textit{ura}    \> ‘you’             \>                      \> > \> /uroo/    \\
    \end{tabbing}
\ex Long vowels or diphthongs  
    \begin{tabbing}
    \textit{wunagu}\hspace{\tabcolsep}\=‘(personal name)’\hspace{\tabcolsep}\=+  \textit{ja} (\textsc{top})\hspace{\tabcolsep}\=\hspace{\tabcolsep}>\hspace{\tabcolsep}\=/kucjəə/\hspace{\tabcolsep}\=(*/kucəə/)\kill
    \textit{jaa} \> ‘house’  \> +  \textit{ja} (\textsc{top}) \> > \> /jaaja/ \> (*/ja.oo/)\\
    \textit{mai} \> ‘hip’    \>                      \> > \> /maija/ \> (*/ma.əə/)\\
    \end{tabbing}
\z
\end{exe}
The above phenomenon can be paraphrased as follows: if the preceding syllable is a light syllable, it is frequently fused with \textit{ja} (\textsc{top}); if the preceding syllable is a heavy syllable, it is not fused with \textit{ja} (TOP).

  Secondly, if \textit{ja} (\textsc{top}) follows //m// or //n//, it is always realized as /na/ or /nja/, according to the morphosyntactic environments or the lexemes of the preceding words.

\begin{exe}
\ex Rule schemata\label{ex:10.3}
  \begin{xlist}
  \ex  Special \textit{n}{}-final morphemes
    \begin{tabbing}
    \textit{ja} (\textsc{top})\hspace{\tabcolsep}\=\hspace{\tabcolsep}>\hspace{\tabcolsep}\=\hspace{\tabcolsep}/nja/\hspace{\tabcolsep}\=\hspace{\tabcolsep}/\hspace{\tabcolsep}\=\hspace{\tabcolsep}[\textit{m}{}-final or \textit{n}{}-final stems]\hspace{\tabcolsep}\=\hspace{\tabcolsep}\_\kill
    \textit{ja} (\textsc{top}) \> > \> /nja/ \> / \> $\left\{\begin{tabular}{@{}ll@{}}
                                             \textit{nan}                  & (2.\textsc{hon}.\textsc{sg})\\
                                             \textit{n}                    & (\textsc{dat}1)\\
                                             \textit{nan}                  & (\textsc{loc}1)\\
                                             \textit{{}-n}                 & (\textsc{advz})\\
                                             \textit{unin}\footnotemark{}  & ‘that time’\\
                                             \end{tabular}\right\}$ \> \_\\
    \end{tabbing}
    \ex Infinitives (stem No. 6 \& 17)
    \begin{tabbing}
    \textit{ja} (\textsc{top})\hspace{\tabcolsep}\=\hspace{\tabcolsep}>\hspace{\tabcolsep}\=\hspace{\tabcolsep}/nja/\hspace{\tabcolsep}\=\hspace{\tabcolsep}/\hspace{\tabcolsep}\=\hspace{\tabcolsep}[\textit{m}{}-final or \textit{n}{}-final stems]\hspace{\tabcolsep}\=\hspace{\tabcolsep}\_\kill
    \textit{ja} (\textsc{top}) \> > \> /nja/\footnote{The allomorph /nja/ sometimes alternates with /na/.} \> / \>  \parbox[t]{\widthof{[\textit{m}{}-final or \textit{n}{}-final stems]}}{Infinitives\\\relax[\textit{m}{}-final or \textit{n}{}-final stems]} \>   \_\\
    \end{tabbing}     
    \ex  The other \textit{n}{}-final morphemes 
    \begin{tabbing}
    \textit{ja} (\textsc{top})\hspace{\tabcolsep}\=\hspace{\tabcolsep}>\hspace{\tabcolsep}\=\hspace{\tabcolsep}/nja/\hspace{\tabcolsep}\=\hspace{\tabcolsep}/\hspace{\tabcolsep}\=\hspace{\tabcolsep}[\textit{m}{}-final or \textit{n}{}-final stems]\hspace{\tabcolsep}\=\hspace{\tabcolsep}\_\kill
    \textit{ja} (\textsc{top}) \> > \> /na/ \> /  \>  //n// \> \_    \\
    \end{tabbing}
   \end{xlist}
\end{exe}
\footnotetext{*\textit{kunin} ‘this time’ or *\textit{anin} ‘that time’ do not exist in Yuwan}
\ea Examples\label{ex:10.4}
  \ea Special \textit{n}-final morphemes
    \begin{tabbing}
    \textit{uma=nan}\hspace{\tabcolsep}\=\hspace{\tabcolsep}(grandchild=\textsc{dat}1)\hspace{\tabcolsep}\=\hspace{\tabcolsep}+  \textit{ja} (\textsc{top})\hspace{\tabcolsep}\=\hspace{\tabcolsep}>\hspace{\tabcolsep}\=\hspace{\tabcolsep}/uma.nannja/\kill
    \textit{nan}     \>  (2.\textsc{hon}.\textsc{sg})         \> +  \textit{ja} (\textsc{top}) \> > \> /nannja/    \\
    \textit{maga=n}  \>  (grandchild=\textsc{dat}1)  \>                      \> > \> /magannja/  \\
    \textit{uma=nan} \>  (there=\textsc{loc}1)       \>                      \> > \> /uma.nannja/\\
    \textit{ka-n}    \>  (\textsc{prox}-\textsc{advz})        \>                      \> > \> /kannja/    \\
    \textit{unin}    \>   ‘that time’       \>                      \> > \> /uninnja/   \\
    \end{tabbing}

  \ex Infinitives
    \begin{tabbing}
    \textit{uma=nan}\hspace{\tabcolsep}\=\hspace{\tabcolsep}(grandchild=\textsc{dat}1)\hspace{\tabcolsep}\=\hspace{\tabcolsep}+  \textit{ja} (\textsc{top})\hspace{\tabcolsep}\=\hspace{\tabcolsep}>\hspace{\tabcolsep}\=\hspace{\tabcolsep}/uma.nannja/\kill
    \textit{jum-∅} \> (read-\textsc{inf}) \>  +  \textit{ja} (\textsc{top}) \> > \> /jumnja/\\
    \textit{sin-∅} \> (die-\textsc{inf})  \>                       \> > \> /sinnja/\\
    \end{tabbing}

  \ex The other n-final morphemes
    \begin{tabbing}
    \textit{uma=nan}\hspace{\tabcolsep}\=\hspace{\tabcolsep}(grandchild=\textsc{dat}1)\hspace{\tabcolsep}\=\hspace{\tabcolsep}+  \textit{ja} (\textsc{top})\hspace{\tabcolsep}\=\hspace{\tabcolsep}>\hspace{\tabcolsep}\=\hspace{\tabcolsep}/uma.nannja/\kill
    \textit{wan}    \> (1\textsc{sg})      \>    +  \textit{ja} (\textsc{top}) \> > \> /wanna/  \\
    \textit{jum-an} \> (read-\textsc{neg}) \>                         \> > \> /jumanna/\\
    \end{tabbing}
\z
\z

\subsubsection{Syntax and semantics of topic particle \textit{ja}}\label{sec:10.1.1.2}

The term topic is here used in the following meaning: “the topic of a sentence is the thing which the proposition expressed by the sentence is about” \citep[118]{Lambrecht1994}. Yuwan uses \textit{ja} (\textsc{top}) to mark the topic in a clause. I will present an example where two people are talking about a picture in front of them. In this conversation, the referent (in a picture) indicated by \textit{ku-rɨ} (\textsc{prox}-\textsc{nlz}) ‘this person’ in (10-5 b) was already mentioned by the previous utterance in (10-5 a) as \textit{ku-n} \textit{cˀju} (PROX-\textsc{adn}Z person) ‘this one.’ In other words, \textit{ku-rɨ} ‘this one’ in (10-5 b) is presupposed by the hearer and may be topicalized. Thus, it takes \textit{ja} (TOP) as in (10-5 b).

\ea\label{ex:10.5}   \textit{ku-rɨ} (\textsc{prox}-\textsc{nlz}) ‘this (one)’ being topicalized\\\relax
  [Context: Looking at a picture]\\
  \begin{xlist}[b. \textsc{tm}:]
  \exi{a. \textsc{ms}:} \glll kun  cˀjuja  utacuobasan.ja  aran?  ikjasjɨ?\\
      \textit{ku-n}  \textit{cˀju=ja}  \textit{utacu+obasan=ja}  \textit{ar-an}  \textit{ikja-sjɨ}\\
      \textsc{prox}-\textsc{adn}Z  person=\textsc{top}  Utatsu+old.lady=TOP  \textsc{cop}-\textsc{neg}  how-\textsc{advz}\\
      \glt ‘Isn’t this person Utatsu? What (do you think)?’

   \exi{b. \textsc{tm}:} \glll    aran,  aran.  kurəə  josidanu  hannjəə.\\
      \textit{ar-an}  \textit{ar-an}  \textit{\Highlight{ku-rɨ=ja}}  \textit{josida=nu}  \textit{hannjəə}\\
      \textsc{cop}-\textsc{neg}  COP-NEG  \textsc{prox}-\textsc{nlz}=\textsc{top}  Yoshida=\textsc{gen}  grandmother\\
      \glt       ‘No, no. This one is the grandmother of the Yoshida [i.e. a name of a shop].’ [Co: 120415\_00.txt]
  \end{xlist}
\z

In (10-5 a), \textsc{ms} mistook a person in the picture for another person (i.e. ‘Utatsu’). Then, \textsc{tm} corrected the misunderstanding, and told MS that it was ‘the grandmother of the Yoshida.’ In this example, the referent of \textit{ku-rɨ} ‘this one’ in (10-5 b) is presupposed by the hearer. On the other hand, if the referent indicated by \textit{ku-rɨ} (\textsc{prox}-\textsc{nlz}) ‘this one’ is not presupposed by the hearer, \textit{ku-rɨ} ‘this one’ does not take \textit{ja} (\textsc{top}) as in (10-6 b).

\ea\label{ex:10.6}   \textit{ku-rɨ} (\textsc{prox}-\textsc{nlz}) ‘this (one)’ not being topicalized\\\relax
  [Context: Looking at a picture]\\
  \ea \textsc{ms}: \glll {\textbar}koocjoo  sita{\textbar}jaa.  {\textbar}hai{\textbar}.  hirosiccjun  cˀju?\\
      \textit{koocjoo}  \textit{sita=jaa}  \textit{hai}  \textit{hirosi=ccjɨ+jˀ-jur-n}  \textit{cˀju}\\
      principal  do.\textsc{pst}=\textsc{sol}  yes  Hiroshi=\textsc{qt}+say-\textsc{umrk}-\textsc{ptcp}  person\\
      \glt ‘(He) was the principal. Yeah. (Is he) a person who (is called) Hiroshi?’

  \ex  \textsc{tm}: \glll    kurɨga  hirosi.\\
      \textit{\Highlight{ku-rɨ=ga}}  \textit{hirosi}\\
      \textsc{prox}-\textsc{nlz}=\textsc{nom}  Hiroshi\\
      \glt       ‘This one is Hiroshi.’ [Co: 120415\_00.txt]
    \z
\z

In (10-6 a), \textsc{ms} remembered a person who was the school principal, and asked \textsc{tm} if his name was Hiroshi or not. Then, in (10-6 b), TM pointed a person in the picture and told him that the person was Hiroshi. In this conversation, \textit{ku-rɨ} ‘this one’ in (10-6 b) is not presupposed by the hearer. Thus, it cannot be marked by \textit{ja} (\textsc{top}), and the nominative case, which is used to mark the subject of the nominal predicate, appears.

  The referent (of the word) that is marked by \textit{ja} (\textsc{top}) should be presupposed by the hearer. Therefore, interrogatives cannot be marked by \textit{ja} (TOP). In fact, interrogatives are frequently marked by \textit{ga} (\textsc{foc}) (see \sectref{sec:10.1.2.2}).

  The topic marker \textit{ja} cannot co-occur with the nominative case as in (10-5 b); otherwise, the subject in (10-5 b) must take \textit{ga} (\textsc{nom}) (see \sectref{sec:6.4.3.2}). The other case particles, e.g., the accusative case \textit{ba}, can co-occur with \textit{ja} (\textsc{top}) as in \textsc{ref}{ex:10.7}.

\ea\label{ex:10.7}   \textit{ba} (\textsc{acc}) + \textit{ja} (\textsc{top}) [= (6-101 d)]\\
  %\textsc{tm}:
      \glll    mata  namanujoo  warabɨnkjoojoo, huccjunkjaboo  sɨkandoojaa.\\
    \textit{mata}  \textit{nama=nu=joo}  \textit{warabɨ=nkja=ja=joo}    \textit{huccju=nkja=\Highlight{ba=ja}}  \textit{sɨk-an=doo=jaa}\\
    moreover  now=\textsc{gen}=\textsc{cfm}1  child=\textsc{appr}=\textsc{top}=CFM1  old.person=\textsc{app}R=\textsc{acc}=TOP  like-\textsc{neg}=\textsc{ass}=\textsc{sol}\\
\glt     ‘Moreover, the children in these days do not like the old people.’  [Co: 120415\_01.txt]
\z

  \textit{ja} (\textsc{top}) also appears in the nominal predicate in negative as in \textsc{ref}{ex:10.8} (except for the case in \sectref{sec:9.3.3.1}).

\ea\label{ex:10.8}   \textit{ja} (\textsc{top}) in the nominal predicate (= [8-39 a])\\
  %\textsc{tm}:
      \gllll    kurəə  {(an ..)}  kazumataaja  aranna?\\
    \textit{ku-rɨ=ja}  \textit{a-n}  \textit{kazuma-taa=\Highlight{ja}}  \textit{ar-an=na}\\
    [\textsc{prox}-\textsc{nlz}]=\textsc{top}  \textsc{dist}-\textsc{adn}Z  [Kazuma-\textsc{pl}=TOP  \textsc{cop}-\textsc{neg}]=\textsc{plq}\\
    [Subject]    [Nominal predicate]\\
    \glt ‘Isn’t this [i.e. the scene in the picture] (about) Kazuma and his friends?’ [Co: 120415\_00.txt]
\z    

In the above example, the NP in the nominal predicate in negative takes \textit{ja} (\textsc{top}).

\subsection{Focus particles \textit{du} and \textit{ga}}\label{sec:10.1.2}

The focus particle is used to mark the word where the speaker thinks that the hearer’s attention should be paid. Thus, the focus particle and the topic particle cannot co-occur, since the latter is used to mark the word that is, the speaker thinks, pressuposed by the hearer. Yuwan has two focus particles: \textit{du} and \textit{ga}. \textit{du} (\textsc{foc}) is used in the assertion or the polar question (see \sectref{sec:10.1.2.1}). \textit{ga} (FOC) is used in the information question in principle (see \sectref{sec:10.1.2.2}).

\subsubsection{\textit{du} (\textsc{foc})}\label{sec:10.1.2.1}

\textit{du} (\textsc{foc}) is used either in the assertion or the polar question. First, I will show the examples of \textit{du} (FOC) used in the assertion.

\ea\label{ex:10.9}   \textit{du} (\textsc{foc}) in the assertion\\
  \ea  %\textsc{tm}:
      \glll    takennan  umoojutankara,  {\textbar}hotondo{\textbar}  takennu   munbəidu  ucɨcjəija.\\
      \textit{taken=nan}  \textit{umoor-jur-tar-n=kara}  \textit{hotondo}  [\textit{taken=nu} \textit{mun}]\textsubscript{NP}\textit{=bəi=\Highlight{du}}  \textit{ucɨs-təər-i=jaa}\\
      Taken=\textsc{loc}1  exist-\textsc{umrk}-\textsc{pst}-\textsc{ptcp}=\textsc{csl}  almost  Taken=\textsc{gen}   thing=only=\textsc{foc}  take-\textsc{rsl}-\textsc{npst}=\textsc{sol}\\
      \glt       ‘Since (he) used to be in Taken, (he) took only the (pictures) of Taken.’ [Co: 111113\_02.txt]

  \ex  %\textsc{tm}:
      \glll    miojakunga  wutɨdu  jiccjan.\\
      [\textit{miojakun=ga}  \textit{wur-tɨ}]\textsubscript{Adverbial clause}\textit{=\Highlight{du}}  \textit{jiccj-sa+ar-n}\\
      Mioya=\textsc{nom}  exist-\textsc{seq}=\textsc{foc}  good-\textsc{adj}+\textsc{st}V-\textsc{ptcp}\\
      \glt       ‘There is Mioya, and (it) is good (for us).’ [Co: 120415\_01.txt]

  \ex  %\textsc{tm}:
      \glll    naa{\textbar}nihon{\textbar}bəidu  appa,  {\textbar}hacikiro{\textbar}naadu   kinmɨ  sjɨ,  haatɨ,\\
      [\textit{naa+nihon=bəi=\Highlight{du}}  \textit{ar-ba}]\textsubscript{Adverbial clause}  [\textit{hacikiro+naa=\Highlight{du}} \textit{kinmɨ}  \textit{sɨr-tɨ}]\textsubscript{Adverbial clause}  \textit{haar-tɨ}\\
      another+two.\textsc{clf}=about=\textsc{foc}  exist-\textsc{csl}  eight.kilogram+each=FOC      measure  do-\textsc{seq}  measure-SEQ\\
      \glt       ‘There are the other two white radishes, so (one) measures eight kilograms (of the materials) for each, and ...’ [Co: 101023\_01.txt]

  \ex  %\textsc{tm}:
      \glll    hada  natɨbəidu  wun  cˀjunu ..\\
      [\textit{hada}  \textit{nar-tɨ=bəi=\Highlight{du}}  \textit{wur-n}]\textsubscript{Adnominal clause}  \textit{cˀju=nu}\\
      naked  become-\textsc{seq}=always=\textsc{foc}  \textsc{prog}-\textsc{ptcp}  person=\textsc{nom}\\
      \glt       ‘The person who was always naked ...’ [Co: 120415\_00.txt]
    \z
\z

In (10-9 a), \textit{du} (\textsc{foc}) follows the NP \textit{taken=nu} \textit{mun} (Taken=\textsc{gen} thing) ‘the things of Taken.’ In (10-9 b), \textit{du} (FOC) follows the clause \textit{miojakun=ga} \textit{wur-tɨ} (Mioya=\textsc{nom} exist-\textsc{seq}) ‘There is Mioya.’ In this example, the sentence-final predicate takes the participle, which is usually used to fill the predicate of the adnominal clause. The correlation of \textit{du} (FOC) and the participle has been traditionally called \textit{kakari-musubi} (i.e. ‘government-predication’), which will be discussed in \sectref{sec:11.3.1}. In (10-9 c), \textit{du} (FOC) appears in the adverbial clause. In (10-9 d), \textit{du} (FOC) appears in the adnominal clause.

  Secondly, I will show the examples of \textit{du} (\textsc{foc}) used in the polar question.

\ea\label{ex:10.10}   \textit{du} (\textsc{foc}) in the polar question\\
  \ea\relax  [= (8-76 d)]\\
    %\textsc{tm}:
      \glll    kurəə  {\textbar}maiku{\textbar}du  muccjurui?\\
      \textit{ku-rɨ=ja}  \textit{maiku=\Highlight{du}}  \textit{mut-tur-u=i}\\
      \textsc{prox}-\textsc{nlz}=\textsc{top}  microphone=\textsc{foc}  hold-\textsc{prog}-\textsc{pfc}=\textsc{plq}\\
      \glt       ‘Is this person holding a microphone?’ [Co: 111113\_02.txt]

  \ex  %\textsc{tm}:
      \glll    uroo  kumaaradu  izitarui?\\
      \textit{ura=ja}  \textit{ku-ma=kara=\Highlight{du}}  \textit{izir-tar-u=i}\\
      2.N\textsc{hon}.\textsc{sg}=\textsc{top}  \textsc{prox}-place=\textsc{abl}=\textsc{foc}  go.out-\textsc{pst}-\textsc{pfc}=\textsc{plq}\\
      \glt       ‘Did you go out from here?’ [El: 121010]
\z
\z

If \textit{du} (\textsc{foc}) is used in the polar question, the verbal inflection takes \textit{{}-u} (\textsc{pfc}) with the question particle \textit{i} (\textsc{plq}) as in the above examples.

\subsubsection{\textit{ga} (\textsc{foc})}\label{sec:10.1.2.2}

In principle, \textit{ga} (\textsc{foc}) is used in the information question as in (10-11 a-b).

\ea\label{ex:10.11}   \textit{ga} (\textsc{foc}) in the information question\\
  \ea\relax [= (5-34 a)]\\
    %\textsc{tm}:
      \glll    nɨsəə  mata  daaciga  izjaru?\\
      \textit{nɨsəə}  \textit{mata}  \textit{daa=kaci=\Highlight{ga}}  \textit{ik-tar-u}\\
      young.man  again  where=\textsc{all}=\textsc{foc}  go-\textsc{pst}-\textsc{pfc}\\
      \glt       ‘Where did the young man go again?’ [Co: 120415\_01.txt]

  \ex\relax [Context: Talking with US about how they played in the past] = (5-31)

    %\textsc{tm}:
      \glll    nuu  sjutɨga,  asɨdutakai?\\
      \textit{nuu}  \textit{sɨr-jur-tɨ=\Highlight{ga}}  \textit{asɨb-tur-tar=kai}\\
      what  do-\textsc{umrk}-\textsc{seq}=\textsc{foc}  play-\textsc{prog}-\textsc{pst}=\textsc{du}B\\
      \glt       ‘What kind of play did (we) do? [lit. What did (we) use to do, and play?] [Co: 110328\_00.txt]
    \z
\z

In (10-11 a), \textit{ga} (\textsc{foc}) follows the (extended) NP \textit{daa=kaci} (where=\textsc{all}) ‘to where.’ In (10-11 b), \textit{ga} (FOC) follows the clause \textit{nuu} \textit{sɨr-jur-tɨ} (what do-\textsc{umrk}-\textsc{seq}) ‘What did (we) use to do, and ...’ Both of the examples include the intterogative words, i.e. \textit{daa} ‘where’ and \textit{nuu} ‘what,’ and express the information question (see also \sectref{sec:5.3.1}).

  However, there are a few cases where \textit{ga} (\textsc{foc}) is used not in the information question; they are summarized below.

\ea\label{ex:10.12}\textit{ga} (\textsc{foc}) is used after,\\
  \ea \textit{tuki=n} (time=\textsc{dat}1);
  \ex temporal adverbs;
  \ex locational nominals;
  \ex adverbial clauses.
  \z
\z

First, \textit{ga} (\textsc{fon}) is used after \textit{tuki=n} (time=\textsc{dat}1), even if the clause does not express an information question.

\ea\label{ex:10.13}   \textit{ga} (\textsc{foc}) is used after \textit{tuki=n} (time=\textsc{dat}1)\\
  \ea\relax  [= (4-25 c)]\\
    %\textsc{tm}:
      \glll    {\textbar}hizjoo{\textbar}nu  tukinga  gan+gan  gan+gan                  zjanaucii.\\
      \textit{hizjoo=nu}  \textit{\Highlight{tuki=n=ga}}  \textit{gan+gan}  \textit{gan+gan}        \textit{zjana+ut-i}\\
      emergency=\textsc{gen}  time=\textsc{dat}1=\textsc{foc}  \textsc{red}+clang  RED+clang                                many+hit-\textsc{inf}\\
      \glt       ‘When there was an emergency, (the person in charge) clanged (the bell) many times.’ [Co: 111113\_02.txt]

  \ex  %\textsc{tm}:
      \glll    {\textbar}cjoodo{\textbar}  un  tukinga  (anoo ..)  nasjenu    cjuugakkoo  {\textbar}socugjoo{\textbar}  sjɨ.\\
      \textit{cjoodo}  \textit{u-n}  \textit{\Highlight{tuki=n=ga}}    \textit{nasje=nu}    \textit{cjuugakkoo}  \textit{socugjoo}  \textit{sɨr-tɨ}\\
      just  \textsc{mes}-\textsc{adn}Z  time=\textsc{dat}1=\textsc{foc}    Naze=\textsc{gen}     junior.high.school  graduation  do-\textsc{seq}  \\
      \glt       ‘Just at the time, (the teacher came, who) had graduated from the junior high school in Naze.’ [Co: 120415\_00.txt]
    \z
\z

  Secondly, \textit{ga} (\textsc{foc}) is used after temporal adverbs, even if the clause does not express an information question.

\ea\label{ex:10.14}   \textit{ga} (\textsc{foc}) is used after temporal adverbs\\
  \ea  %\textsc{tm}:
      \glll    kinjuga,  (kinjuga)  cuburutu  (cuburutu)  cubusitu     jˀicjutɨga,  warəəcjɨjo.\\
      \textit{kinju=\Highlight{ga}}  \textit{kinju=\Highlight{ga}}  [\textit{cuburu=tu}  \textit{cuburu=tu}  \textit{cubusi=tu}   \textit{jˀ-tur-tɨ=ga}]\textsubscript{Adverbial clause} \textit{waraw-i=ccjɨ=joo}\\
      yesterday=\textsc{foc}  yesterday=FOC  head=\textsc{com}  head=COM  knee=COM say-\textsc{prog}-\textsc{seq}=FOC  laugh-\textsc{inf}=\textsc{qt}=\textsc{cfm}1\\
      \glt       ‘Yesterday (I) said \textit{cuburu} [i.e. ‘head’] and \textit{cubusi} [i.e. ‘knee’] (in Yuwan for the present author), and (we) laughed.’ [Co: 110328\_00.txt]

  \ex%\textsc{tm}:
      \glll    kunəədaga  waakja  dusinu,  asikendusinu,                                     wututɨ,\\
      \textit{kunəəda=\Highlight{ga}}  \textit{waakja-a}  \textit{dusi=nu}  \textit{asiken+dusi=nu}  \textit{wur-tur-tɨ}\\
      the.other.day=\textsc{foc}  1\textsc{pl}-\textsc{adn}Z  friend=\textsc{nom}  Ashiken+frend=NOM                         exist-\textsc{prog}-\textsc{seq}\\
      \glt       ‘The other day, there is my friend, (i.e.) a friend in Ashiken, and ...’ [Co: 120415\_00.txt]
    \z
\z

  Thirdly, \textit{ga} (\textsc{foc}) is used after locational nominals, even if the clause does not express an information question. Interestingly, the locational nominals followed by \textit{ga} (FOC) (in the non-information question) do not take the locative cases.

\ea\label{ex:10.15}   \textit{ga} (\textsc{foc}) is used after locational nominals\\
  \ea  %\textsc{tm}:
      \glll    umaga  atəkkamojaa.\\
      \textit{u-ma=\Highlight{ga}}  \textit{ar-təər=kamo=jaa}\\
      \textsc{mes}-place=\textsc{foc}  exist-\textsc{rsl}=\textsc{pos}=\textsc{sol}\\
      \glt       ‘(The chamber of commerce) may have been there.’ [lit. ‘(At) that place, (the chamber of commerce) may have existed.’] [Co: 120415\_00.txt]

  \ex\relax  [= (4-38 a)]\\
    %\textsc{tm}:
      \glll    umaga  naikwanu  dɨkɨppoo,   {\textbar}kamera{\textbar}  numgja  ikiiki.\\
      \textit{u-ma=\Highlight{ga}}  \textit{naikwa=nu}  \textit{dɨkɨr-boo}    \textit{kamera}  \textit{num-∅+gja}  \textit{ik-i+ik-i}\\
      \textsc{mes}-place=\textsc{foc}  department.of.internal.medicine=\textsc{nom}  be.set.up-\textsc{cnd}   camera  swallow-\textsc{inf}+\textsc{purp}  go-INF+go-INF\\
      \glt       ‘After the department of internal medicine was set up there, (I) often went (there) in order to swallow the (stomach) camera.’ [Co: 120415\_01.txt]
    \z
\z

  Finally, \textit{ga} (\textsc{foc}) is used after adverbial clauses, even if the clause does not express an information question. (10-14 a) is an example of that. Other examples are shown below.

\ea\label{ex:10.16}   \textit{ga} (\textsc{foc}) is used after adverbial clauses\\
  \ea  %\textsc{tm}:
      \glll    uninkara  hɨɨtəəraga,  uraa  məəci {\textbar}denwa{\textbar}ba  sjəəraga,  bocuubocu  cɨra  aratɨ,\\
      [\textit{unin=kara}  \textit{hɨɨr-təəra}]\textsubscript{Adverbial clause}\textit{=\Highlight{ga}}  [\textit{ura-a}  \textit{məə=kaci}    \textit{denwa=ba}  \textit{sɨr-təəra}]\textsubscript{Adverbial clause}\textit{=ga}  \textit{bocu+bocu}  \textit{cɨra}  \textit{araw-tɨ}\\
      that.time=\textsc{abl}  get.up-after=\textsc{foc}  2.N\textsc{hon}.\textsc{sg}-\textsc{adn}Z  place=\textsc{all}   phone=\textsc{acc}  do-after=FOC  \textsc{red}+step.by.step  face  wash-\textsc{seq}\\
      \glt       ‘After (I) got up since that time, and after (I) called you, (I) washed my face, and ...’ [Co: 101020\_01.txt]

  \ex\relax  [Context: \textsc{tm} complains about the injury to her feet, since it made her unable to dance.]\\
    %\textsc{tm}:
      \glll    gan  sjɨ  natɨga,  urɨ  natɨga,  sɨrarancjɨjo.\\
      [\textit{ga-n}  \textit{sɨr-tɨ}  \textit{nar-tɨ}]\textsubscript{Adverbial clause}\textit{=\Highlight{ga}}  [\textit{u-rɨ}   \textit{nar-tɨ}]\textsubscript{Adverbial clause}\textit{=\Highlight{\Highlight{ga}}}  \textit{sɨr-ar-an=ccjɨ=joo}\\
      \textsc{mes}-\textsc{advz}  do-\textsc{seq}  become-SEQ=\textsc{foc}  MES-\textsc{nlz}  become-SEQ=FOC  do-\textsc{cap}-\textsc{neg}=\textsc{qt}=\textsc{cfm}1\\
      \glt       ‘Since (it) is like that, and since (it) is that [i.e. \textsc{tm} trips over her own feet], (I) cannot do (it) [i.e. dance].’ [Co: 120415\_01.txt]
    \z
\z

\subsection{\textit{n} ‘also; even; ever’}\label{sec:10.1.3}

The limiter particle \textit{n} has several meanings, i.e. ‘also,’ ‘even,’ and ‘ever,’ which will be exemplified below in turn.

First, the limiter particle \textit{n} means ‘also’ after NPs. The NP followed by \textit{n} ‘also’ presupposes another referent that has some relationship to the referent indicated by the NP.

\ea\label{ex:10.17}   \textit{n} meaning ‘also’\\
  \ea  %\textsc{tm}:
      \glll    sumii.  uran  acjoo  xxx  cˀjɨ  kurɨrbanboo.  naa  main  kucin  wakaranmun.\\
      \textit{sumi}  \textit{ura=\Highlight{n}}  \textit{acja=ja}    \textit{k-tɨ}  \textit{kurɨr-an-boo}   \textit{naa}  \textit{mai=\Highlight{n}}  \textit{kuci=\Highlight{n}}  \textit{wakar-an=mun}\\
      Sumi  2.N\textsc{hon}.\textsc{sg}=also  tomorrow=\textsc{top}    come{}-\textsc{seq}  \textsc{ben}-\textsc{neg}-\textsc{cnd}    already  buttock=also  mouth=also  understand-NEG=\textsc{advrs}\\
      \glt       ‘Sumi. If (not only the present author but) also you do not come tomorrow (for me), (I will be in trouble). (I) already cannot distinguish (not only complex things but) also the buttock and the mouth [i.e. cannot understand anything].’ [Co: 101023\_01.txt]

  \ex  %\textsc{tm}:
      \glll    acjan  dooka  cˀjɨ  kurɨppajoo.\\
      \textit{acja=\Highlight{n}}  \textit{dooka}  \textit{k-tɨ}  \textit{kurɨr-ba=joo}\\
      tomorrow=also  please  come-\textsc{seq}  \textsc{ben}-\textsc{csl}=\textsc{cfm}1\\
      \glt       ‘Please come (for me) also tomorrow.’ [Co: 101023\_01.txt]
    \z
\z

In (10-17 a), \textit{ura=n} ‘also you’ presupposes the existence of the present author, and \textit{mai=n} \textit{kuci=n} (buttock=also mouth=also) presupposes some complex things. See the free translation of (10-17 a). In (10-17 b), \textit{n} ‘also’ follows directly a nominal that has temporal meaning such as \textit{acja} ‘tomorrow.’ However, if \textit{n} ‘also’ follows \textit{nama} ‘now,’ it has to take \textit{n} (\textsc{dat}1) as in \textsc{ref}{ex:10.18}.

\ea\label{ex:10.18}   [Context: Speaking of the outdoor lamps which was set in the past] = (9-57 b)\\
  %\textsc{tm}:
      \glll    namanin  an  aran?\\
    \textit{\Highlight{nama=n=n}}  \textit{ar-n}  \textit{ar-an}\\
    now=\textsc{dat}1=also  exist-\textsc{ptcp}  \textsc{cop}-\textsc{neg}\\
\glt     ‘There are (outdoor lamps not only in the past but) aslo now, aren’t there?’  [Co: 120415\_00.txt]
\z

  Secondly, the limiter particle \textit{n} and the preceding adverbial clause (whose head verb ends with \textit{{}-tɨ} (\textsc{seq})) means ‘even if’ (excluding the case of \textit{nj-} (\textsc{exp}), which is discussed later).

\ea\label{ex:10.19}   \textit{{}-tɨ} (\textsc{seq}) + \textit{n} ‘even’ meaning ‘even if’\\
  \ea\relax  [= (8-103)]\\
    %\textsc{tm}:
      \glll    abɨtɨn,  kikjanba.  jˀicjɨn,  kikjanba.\\
      [\textit{abɨr-tɨ}]\textsubscript{Adverbial clause}\textit{=\Highlight{n}}  \textit{kik-an-ba}  [\textit{jˀ-tɨ}]\textsubscript{Adverbial clause}\textit{=n}  \textit{kik-an-ba}\\
      call-\textsc{seq}=even  hear{}-\textsc{neg}-\textsc{csl}  say-SEQ=even  hear{}-NEG-CSL\\
      \glt       ‘Even if (I) call (her), (she) doesn’t hear. Even if (I) say (something to her), (she) doesn’t hear, so (I don’t visit her these days).’ [Co: 120415\_01.txt]

  \ex  %\textsc{tm}:
      \glll    daa  izjɨn,  {(an ..)}  {\textbar}dɨɨsaabisu{\textbar}  izjɨn,\\
      \textit{daa}  \textit{ik-tɨ=n}  [\textit{a-n}  \textit{dɨɨsaabisu}  \textit{ik-tɨ}]\textsubscript{Adverbial clause}\textit{=n}\\
      where  go-\textsc{seq}=any  \textsc{dist}-\textsc{adn}Z  day.care  go-SEQ=even\\
      \glt       ‘Wherever (I) go, and even if (I) go to day-care (center), ...’ [Co: 120415\_01.txt]
    \z
\z

Thirdly, the limiter particle \textit{n} means ‘ever’ before \textit{nj-} (\textsc{exp}) (see \sectref{sec:9.1.1.1} for more details).

\ea\label{ex:10.20}   \textit{n} ‘ever’ + \textit{nj-} (\textsc{exp})\\
  %\textsc{tm}:
  \gllll    asɨdɨn  njan.jaa.\\
    \textit{asɨb-tɨ=\Highlight{n}}  \textit{\Highlight{nj}-an=jaa}\\
    play-\textsc{seq}=ever  \textsc{exp}-\textsc{neg}=\textsc{sol}\\
    {Lex. verb}  {Aux. verb}\\
   \glt ‘(We) have never played (together), (have we?)’ [Co: 110328\_00.txt]
\z

  Finally, if the limiter particle \textit{n} follows an indefinite word (or a clause that includes an indefinite word), the questional function of the interrogative word is deleted, and the interrogative word is used as an indefinite word. For example, \textit{nuu} ‘what’ plus \textit{n} means ‘anything’ (see also \sectref{sec:5.3.2}). Tentatively, \textit{n} in this use is glossed as ‘any.’ The interrogatives and \textit{n} ‘any’ in underlying level, and their correspondents in free translation are underlined below.

\ea\label{ex:10.21}   Interrogatives + \textit{n} ‘any’\\
  \ea  %\textsc{tm}:
      \glll    nun  sɨran.joo.\\
      \textit{\Highlight{nuu=n}}  \textit{sɨr-an=joo}\\
      what=any  do-\textsc{neg}=\textsc{cfm}1\\
      \glt       ‘(That person) did not do \Highlight{anything}.’ [Co: 120415\_01.txt]
  \ex\relax [= (8-44 a)]\\
    %\textsc{tm}:
      \glll    {\textbar}reitou{\textbar}nansəəka  ucjukuboo,  ɨcɨɨgadɨ  jatɨn,  ucjukarɨi.\\
      \textit{reitou=nan=səəka}  \textit{uk-tuk-boo}  [\textit{\Highlight{ɨcɨɨ}=gadɨ}  \textit{jar-tɨ}]\textsubscript{Adverbial} \textsubscript{clause}\textit{=\Highlight{n}} \textit{uk-tuk-arɨr-i}\\
      freezer=\textsc{loc}1=just  put-\textsc{pfv}-\textsc{cnd}  when=\textsc{lmt}  \textsc{cop}-\textsc{seq}=any  put-\textsc{pf}V-\textsc{cap}-\textsc{npst}\\
      \glt       ‘If (you) put (the pickles) in the freezer, you can keep (them) \Highlight{no matter how long} (the period of preservation) was.’ [Co: 101023\_01.txt]
  \ex  %\textsc{tm}:
      \glll    daakara  mjicjɨn,  cunekocjɨ  urabjutattu.\\
      [\textit{\Highlight{daa}=kara}  \textit{mj-tɨ}]\textsubscript{Adverbial clause}\textit{=\Highlight{n}}  \textit{cuneko=ccjɨ}  \textit{urab-jur-tar-tu}\\
      where=\textsc{abl}  see-\textsc{seq}=any  Tsuneko=\textsc{qt}  call.loudly-\textsc{umrk}-\textsc{pst}-\textsc{csl}\\
      \glt       ‘\Highlight{No matter where} (he) found (me), (he) called loudly, “Tsuneko.”’ [Co: 120415\_01.txt]
    \z
\z

As mentioned in \sectref{sec:5.3.2}, another word may intervene between the interrogative words and \textit{n} ‘any’ as in (10-21 b-c), where the adverbial clauses are similar to those in (10-20 a-b).

\subsection{\textit{bəi} ‘only; always; about’}\label{sec:10.1.4}

The limiter particle \textit{bəi} means a restrictoin such as (10-22 a), constancy such as (10-9 d), or a rough estimation such as (10-22 b). Each of them is translated as ‘only,’ ‘always,’ and ‘about’ in their glosses and free translation.

\ea\label{ex:10.22} 
 \ea\textit{bəi} meaning a restriction (‘only’)\\
    %\textsc{tm}:
      \glll    {\textbar}medama{\textbar}bəidu  jakjun.\\
      \textit{medama=\Highlight{bəi}=du}  \textit{jak-jur-n}\\
      sunny.side.up=only=\textsc{foc}  bake-\textsc{umrk}-\textsc{ptcp}\\
      \glt       ‘(I) bake only (the egg that is baked) sunny-side up.’ [Co: 101023\_01.txt]

  \ex  \textit{bəi} meaning a rough estimation (‘about’)\\
    %\textsc{tm}:
      \glll    {\textbar}sanzjuunen{\textbar}bəinu  tukikamojaa.\\
      \textit{sanzjuunen=\Highlight{bəi}=nu}  \textit{tuki=kamo=jaa}\\
      the.year.30=about=\textsc{gen}  time=\textsc{pos}=\textsc{sol}\\
      \glt       ‘(The date when this picture was taken) may be about (Showa) 30.’ [Co: 120415\_00.txt]
    \z
\z

\subsection{\textit{gadɨ} (\textsc{lmt})}\label{sec:10.1.5}

\textit{gadɨ} (\textsc{lmt}) can be used as the case particle (see \sectref{sec:6.3.2.12}). Moreover, it may be used as a limiter particle as in (10-23 a-b). \textit{gadɨ} (L\textsc{mt}) is used to express the limit of the speaker’s expectation (or the limit of the hearer’s expectation that the speaker assumes).

\ea\label{ex:10.23}   \textit{gadɨ} (\textsc{lmt}) as the limiter particle\\
    \ea%\textsc{tm}:
      \glll    injahunɨkkwakacigadɨ  {\textbar}bonbon  bakudan  utusi{\textbar}tattu.\\
      \textit{inja+hunɨ-kkwa=kaci=\Highlight{gadɨ}}  \textit{bonbon}  \textit{bakudan}  \textit{utusi-tar-tu}\\
      small+ship-\textsc{dim}=\textsc{all}=\textsc{lmt}  bong  bomb  drop-\textsc{pst}-\textsc{csl}\\
      \glt       ‘(The American soldiers) dropped the bombs even on small ships.’ [Co: 110328\_00.txt]

  \ex\relax  [Context: Remembering a flood in the past when people tried to pull a house that was being flushed away]\\
    %\textsc{tm}:
      \gllll    utɨgadəə  sɨrantattu.\\
      \textit{utɨr-∅=\Highlight{gadɨ}=ja}  \textit{sɨr-an-tar-tu}\\
      fall-\textsc{inf}=\textsc{lmt}=\textsc{top}  do-\textsc{neg}-\textsc{pst}-\textsc{csl}\\
      [Complement  \textsc{lv}]\textsubscript{VP}\\
      \glt ‘(They) were unlikely to fall (in the river).’    [Co: 120415\_00.txt]
    \z
\z

In (10-23 a), \textit{gadɨ} (\textsc{lmt}) follows another case particle, i.e. \textit{kaci} (\textsc{all}). In (10-23 b), \textit{gadɨ} (L\textsc{mt}) follows the infinitive \textit{utɨr-∅} (fall-\textsc{inf}) in the complement slot in the \textsc{lvc}.

  Before concluding this section, it is appropriate to mention that Yuwan has the clasue-final particle \textit{gadɨ} (\textsc{lmt}) as in \textsc{ref}{ex:10.56} in \sectref{sec:10.3.10}, where \textit{gadɨ} (L\textsc{mt}) always follows the adjective. Additionally, there is the inflectional affix \textit{{}-gadɨ} ‘until,’ which can directly follows a verbal root (see \sectref{sec:8.4.3.4} for more details). It is probable that these morphemes have the same origin.

\subsection{\textit{nkja} (\textsc{appr})}\label{sec:10.1.6}

\textit{nkja} (\textsc{appr}) can indicate an unspecific group, and also can indicate a referent as an example (see \sectref{sec:6.4.1.1} for more details). \textit{nkja} (\textsc{app}R) can follow both nominals and verbs.

  First, I will show the examples where \textit{nkja} (\textsc{appr}) follows nominals. In (10-24 a-d), \textit{nkja} (\textsc{app}R) precedes the case particles. In (10-24 e-g), \textit{nkja} (APPR) follows the case particles.

\ea\label{ex:10.24} 
  \ea \textit{nkja} (\textsc{appr}) precedes \textit{nu} (\textsc{nom})\\
    %\textsc{tm}:
      \glll    kun  {\textbar}supiika{\textbar}nkjanu  appa.\\
      \textit{ku-n}  \textit{supiikaa=\Highlight{nkja=nu}}  \textit{ar-ba}\\
      \textsc{prox}-\textsc{adn}Z  loudspeaker=\textsc{appr}=\textsc{nom}  exist-\textsc{csl}\\
      \glt       ‘There are loudspeakers like this (in this picture), so (this picture must have been taken recently).’ [Co: 120415\_00.txt]

  \ex  \textit{nkja} (\textsc{appr}) precedes \textit{ba} (\textsc{acc})\\
    %\textsc{tm}:
      \glll    urɨnkjaba  jˀicjutɨga,  warəəcjɨjo.\\
      \textit{u-rɨ=\Highlight{nkja=ba}}  \textit{jˀ-tur-tɨ=ga}  \textit{waraw-i=ccjɨ=joo}\\
      \textsc{mes}-\textsc{nlz}=\textsc{appr}=\textsc{acc}  say-\textsc{prog}-\textsc{seq}=\textsc{foc}  laugh-\textsc{inf}=\textsc{qt}=\textsc{cfm}1\\
      \glt       ‘(We) were (always) saying a thing like that, and laughing.’ [Co: 110328\_00.txt]

  \ex  \textit{nkja} (\textsc{appr}) precedes \textit{nu} (\textsc{gen})\\

    %\textsc{tm}:
      \glll    umankjanu  cjannui.\\
      \textit{u-ma=\Highlight{nkja=nu}}  \textit{cjan+nur-i}\\
      \textsc{mes}-place=\textsc{appr}=\textsc{gen}  coal.tar+spread-\textsc{inf}\\
    \glt   ‘(The person) gave that place a coat of coal tar.’ [lit. ‘(The person was) to spread coal tar on that place.’]      [Co: 120415\_00.txt]

  \ex  \textit{nkja} (\textsc{appr}) precedes \textit{n} (\textsc{dat}1) [= (8-125 a)]\\
    %\textsc{tm}:
      \glll    {\textbar}daibu{\textbar}  an  cˀjunkjannja  {\textbar}daibu  kuroo{\textbar}  sɨmɨrasatta.\\
      \textit{daibu}  \textit{a-n}  \textit{cˀju=\Highlight{nkja=n}=ja}  \textit{daibu}  \textit{kuroo} \textit{sɨmɨr-as-ar-ta}\\
      many  \textsc{dist}-\textsc{adn}Z  person=\textsc{appr}=\textsc{dat}1=\textsc{top}  many  hardship   do.\textsc{caus}-CAUS-P\textsc{ass}-\textsc{pst}\\
      \glt       ‘(I) was made go through many hardships by that person.’ [Co: 120415\_01.txt]

  \ex \textit{nkja} (\textsc{appr}) follows \textit{n} (\textsc{dat}1) [= (9-45 f)]\\
    %\textsc{tm}:
      \glll    nobuariga  mm  kɨga  sjun  tukininkjoo  huntoo  kuwasa  ata.\\
      \textit{nobuari=ga}    \textit{kɨga}  \textit{sɨr-tur-n}  \textit{tuki=\Highlight{n=nkja}=ja}  \textit{huntoo}  \textit{kuwa-sa}  \textit{ar-tar}\\
      Nobuari=\textsc{nom}    injury  do-\textsc{prog}-\textsc{ptcp}  time=\textsc{dat}1=\textsc{appr}=\textsc{top}  really  hard-\textsc{adj}  \textsc{st}V-\textsc{pst}\\
      \glt       ‘When Nobuari was suffering injuries, (it) was really hard (for me).’ [Co: 111113\_02.txt]

   \ex  \textit{nkja} (\textsc{appr}) follows \textit{kaci} (\textsc{all})\\
    %\textsc{tm}:
      \glll    hatɨɨkacinkja  izjɨn,  naa,  kusa  musijagacinan,  jukkadɨ  uta.      \\
      \textit{hatɨɨ=\Highlight{kaci=nkja}}  \textit{ik-tɨ=n}  \textit{naa}  \textit{kusa}  \textit{muij-jagacinaa=n}  \textit{jukkadɨ}  \textit{uta}      \\
      field=\textsc{all}=\textsc{appr}  go-\textsc{seq}=even  \textsc{fil}  weed  pull-\textsc{sim}=even  always  song\\
      \glt       ‘Even if (my mother) goes to the field, and even while (she) pulls the weeds, (she) always (sings) a song.’ [Co: 111113\_01.txt]

   \ex\textit{nkja} (\textsc{appr}) follows \textit{nantɨ} (\textsc{loc}2)\\
    %\textsc{tm}:
      \glll    mukasija  umantɨnkjoo,  waakjaga   injasain,\\
      \textit{mukasi=ja}  \textit{u-ma=\Highlight{nantɨ=nkja}=ja}  \textit{waakja=ga}   \textit{inja-sa+ar-i=n}\\
      the.past=\textsc{top}  \textsc{mes}-place=\textsc{loc}2=\textsc{appr}=TOP  1\textsc{pl}=\textsc{nom}  small-\textsc{adj}+\textsc{st}V-\textsc{inf}=\textsc{dat}1\\
      \glt       ‘In the past, at that place, when we were small [i.e. children], ...’ [Co: 120415\_01.txt]
    \z
\z

The above examples show that \textit{nkja} (\textsc{appr}) follows nominals that are at the lower level in the animacy hierarchy in Yuwan, e.g., \textit{supiikaa} ‘loudspeaker’ as in (10-24 a) (see also \tabref{tab:44} in \sectref{sec:6.4}). However, if the preceding nominals have already taken a plural marker, i.e. \textit{{}-kja} (\textsc{pl}) or \textit{{}-taa} (PL), then \textit{nkja} (\textsc{app}R) can follow every kind of nominals even if the nominals are at the higer level in in the animacy hierarchy in Yuwan as in (10-25 a-b) (see (6-102) - (6-104) in \sectref{sec:6.4.1.2} for more details).

\ea\label{ex:10.25}
\ea  \textit{{}-kja} (\textsc{pl}) + \textit{nkja} (\textsc{appr})\\\relax
  [Context: Looking at a pictue, where there were a few men] = (6-102 a)\\
  %\textsc{tm}:
      \glll    waakjankjoo  waasa  asaa.\\
    \textit{\Highlight{waakja=nkja}=ja}  \textit{waa-sa}  \textit{ar-sa}\\
    1\textsc{pl}=\textsc{appr}=\textsc{top}  young-\textsc{adj}  \textsc{st}V-\textsc{pol}\\
\glt     ‘I am young(er than them).’  [Co: 111113\_02.txt]

 \ex\textit{{}-taa} (\textsc{pl}) + \textit{nkja} (\textsc{appr})\\
  %\textsc{tm}:
      \glll    nobuhito  okkantankjan  wutənban,\\
    \textit{nobuhito}  \textit{\Highlight{okkan-taa=nkja}=n}  \textit{wur-təər-n=ban}\\
    Nobuhito  mother-\textsc{pl}=\textsc{appr}=also  exist-\textsc{rsl}-\textsc{ptcp}=\textsc{advrs}\\
\glt     ‘Nobuhito’s mother and other people were also living (here), but ...’  [Co: 120415\_00.txt]
\z
\z

  Secondly, I will show the examples where \textit{nkja} (\textsc{appr}) follows verbs. In (10-26 a-d), \textit{nkja} (\textsc{app}R) follows \textit{{}-tɨ} (\textsc{seq}). In (10-26 e), \textit{nkja} (APPR) follows \textit{-tai} (\textsc{lst}).

\ea\label{ex:10.26}   
 \begin{xlist}
 \exi{} \textit{{}-tɨ=nkja} (\textsc{seq}=\textsc{appr})\\
  \ex  %\textsc{tm}:
      \glll    mata  un ..  micjaija  mudutɨnkja  cˀjattu,\\
      \textit{mata}  \textit{u-n}  \textit{micjai=ja}  \textit{mudur-\Highlight{tɨ=nkja}}  \textit{k-tar-tu}\\
      again  \textsc{mes}-\textsc{adn}Z  three.person.\textsc{clf}=\textsc{top}  return-\textsc{seq}=\textsc{appr}  come-\textsc{pst}{}-\textsc{csl}\\
      \glt       ‘The three (boys) came back again, so ...’ [\textsc{pf}: 090222\_00.txt]

  \ex  %\textsc{tm}:
      \glll    cˀjui  jinganu  hinzjaa  succjɨnkjoo,  uma   tuutɨ  cˀjancjɨjoo.\\
      \textit{cˀjui}  \textit{jinga=nu}  \textit{hinzjaa}  \textit{sukk-\Highlight{tɨ=nkja}=ja}  \textit{u-ma}  \textit{tuur-tɨ}  \textit{k-tar-n=ccjɨ=joo}\\
      one.person.\textsc{clf}  man=\textsc{nom}  goat  pull-\textsc{seq}=\textsc{appr}=\textsc{top}  \textsc{mes}-place  pass-SEQ  come-\textsc{pst}-\textsc{ptcp}=\textsc{qt}=\textsc{cfm}1\\
      \glt       ‘A man pulled a goat alone, and came and passed there.’ [\textsc{pf}: 090827\_02.txt]

  \ex  %\textsc{tm}:
      \glll    mussjuuja  hikjannənsjutɨ,  maruu  uccjutɨnkjoo,   asɨbantɨ?      \\
      \textit{mussjuu=ja}  \textit{hik-an-nən=sjutɨ}  \textit{maruu}  \textit{ut-tur-\Highlight{tɨ=nkja}=ja} \textit{asɨb-an-tɨ}\\
      straw.mat=\textsc{top}  spread-\textsc{neg}-\textsc{seq}=SEQ  ball  hit-\textsc{prog}-SEQ=\textsc{appr}=TOP  play-NEG-SEQ      \\
      \glt       ‘Not spreading a straw mat, didn’t (you) play (something) like hitting a ball?’ [Co: 110328\_00.txt]

  \ex  %\textsc{tm}:
      \glll    sɨgu  cuburunan  kan  sjɨ  nusɨtɨnkjadu, aikjutattu.\\
      \textit{sɨgu}  \textit{cuburu=nan}  \textit{ka-n}  \textit{sɨr-tɨ}  \textit{nusɨr-\Highlight{tɨ=nkja}=du}  \textit{aik-jur-tar-tu}\\
      as.soon.as  head=\textsc{loc}1  \textsc{prox}-\textsc{advz}  do-\textsc{seq}  put.on-SEQ=\textsc{appr}=\textsc{foc}   walk-\textsc{umrk}-\textsc{pst}-\textsc{csl}\\
      \glt       ‘(I) used to walk putting (the load) on the head immediately as soon as (I felt it heavy), so (our life style in the old days is similar to that of Vietnam).’ [Co: 111113\_02.txt]

  \exi{} \textit{{}-tai=nkja} (\textsc{lst}=\textsc{appr})

  \ex  %\textsc{tm}:
      \glll    minnan  kˀubatainkjan  sjanmun,\\
      \textit{minna=n}  \textit{kˀubar-\Highlight{tai=nkja}=n}  \textit{sɨr-tar-n=mun}\\
      everyone=\textsc{dat}1  distribute-\textsc{lst}=\textsc{appr}=also  do-\textsc{pst}-\textsc{ptcp}=\textsc{advrs}\\
      \glt       ‘(People) distributed (the pamphlet of songs) to everyone, but ...’ [Co: 120415\_01.txt]
  \end{xlist}
\z

  Before concluding this section, I will present a good example that exemplifies how many times \textit{nkja} (\textsc{appr}) can be used in a clause.

\ea\label{ex:10.27}   [Context: \textsc{tm} talks to \textsc{ms}. (MS’s reply is omitted from the convesation for convenience.)]\\
  %\textsc{tm}:
      \glll    koobunɨjajoo  urakjaa  cˀjantankja,  josidankja,  an  noogusukuntɨnkja  agan  sjɨ  sjun  cˀjunkjanu  kumɨ  {\textbar}hakobi{\textbar}.\\
    \textit{koo+hunɨ=ja=joo}  \textit{urakja-a}  \textit{cˀan-taa=\Highlight{nkja}}  \textit{josida=\Highlight{nkja}}  \textit{a-n}  \textit{noogusuku=nantɨ=\Highlight{nkja}}  \textit{aga-n}  \textit{sɨr-tɨ}  \textit{sɨr-jur-n}  \textit{cˀju=\Highlight{nkja}=nu}  \textit{kumɨ}  \textit{hakobi}\\
    river+boat=\textsc{top}=\textsc{cfm}1  2.N\textsc{hon}.\textsc{pl}-\textsc{adn}Z  father-PL=\textsc{appr}  Yoshida=\textsc{app}R  \textsc{dist}-\textsc{adnz}  Nogusuku=\textsc{loc}2=APPR  DI\textsc{st}-\textsc{advz}  do-\textsc{seq}  do-\textsc{umrk}-\textsc{ptcp}  person=APPR=\textsc{gen}  rice  carrying\\
\glt     ‘The river boat (was used for) the people who do things like that (e.g.,) your father (and) Yoshida (,) to carry the rice.’  [Co: 111113\_01.txt]
\z

\subsection{\textit{kusa} ‘just’}\label{sec:10.1.7}

I will show an example of \textit{kusa} ‘just’ below.

\ea\label{ex:10.28}   \textit{kusa} ‘just’ [= (8-37 a)]\\
  %\textsc{tm}:
      \glll    an  gazimarunu  appoo,  naa,  huntoo,  naa,   urɨkusa,  naa,  {\textbar}nippon.ici{\textbar}  jatəijoo.\\
    \textit{a-n}  \textit{gazimaru=nu}  \textit{ar-boo}  \textit{naa}  \textit{huntoo}  \textit{naa} \textit{u-rɨ=\Highlight{kusa}}  \textit{naa}  \textit{nippon+ici}  \textit{jar-təər-i=joo}\\
    \textsc{dist}-\textsc{adn}Z  banyan.tree=\textsc{nom}  exist-\textsc{cnd}  \textsc{fil}  real  \textsc{fi}L \textsc{mes}-\textsc{nlz}=just  FIL  Japan+one  \textsc{cop}-\textsc{rsl}-\textsc{npst}=\textsc{cfm}1\\
\glt     ‘If that banyan tree existed, that would be just the (number) one in Japan.’  [Co: 111113\_02.txt]
\z

In fact, there is only an example of \textsc{ref}{ex:10.28} that uses \textit{kusa} ‘just’ in the text data. The details of \textit{kusa} ‘just’ should be investigated in future research.

\subsection{\textit{səəka} ‘if only’}\label{sec:10.1.8}

I will show an example of \textit{səəka} ‘if only’ below.

\ea\label{ex:10.29}   \textit{səəka} ‘if only’\\
  %\textsc{tm}:
      \glll    attaaga,  hɨnmaban  sɨrɨccjɨsəəka  juuboo,  hɨnmabanunkjoo  nunkuin  sjoosjunban,\\
    \textit{a-rɨ-taa=ga}  \textit{hɨnma-ban}  \textit{sɨr-ɨ=ccjɨ=\Highlight{səəka}}  \textit{jˀ-boo}  \textit{hɨnma-ban=nkja=ja}  \textit{nuu-nkuin}  \textit{sjoos-jur-n=ban}\\
    \textsc{dist}-\textsc{nlz}-\textsc{pl}=\textsc{nom}  noon-meal  do-\textsc{imp}=\textsc{qt}=if.only  say-\textsc{cnd}   noon-meal=\textsc{appr}=\textsc{top}  what-\textsc{indfz}  prepare-\textsc{umrk}-\textsc{ptcp}=\textsc{advrs}\\
\glt     ‘If (I) say that, “Make the lunch!” (to my daughters), they will prepare anything (for) the lunch, but (I don’t say it).’  [Co: 101023\_01.txt]
\z

In fact, there is only an example of \textsc{ref}{ex:10.29} that uses \textit{səəka} ‘if only’ in the text data. The details of \textit{səəka} ‘if only’ should be investigated in future research.

\section{Conjunctive particles}\label{sec:10.2}

Yuwan has the conjuctive particles as in \tabref{tab:98}. The conjunctive particle and the clasue that precedes it function as the adverbial clause. The units connected by the conjunctive particles in Yuwan are only clauses (not words nor phrases), which is different from \textit{and} or \textit{or} in English.

\begin{table}
\caption{Conjunctive particles\label{tab:98}}
\begin{tabular}{llcccc}
\lsptoprule
& & \multicolumn{4}{c}{Preceding morphemes}\\\cmidrule(lr){3-6}
& & \multicolumn{3}{c}{Verbal} &   Adjectival\\\cmidrule(lr){3-5}\cmidrule(lr){6-6}
Form & Meaning &   \textit{-n} (\textsc{ptcp}) & \textit{-an} (\textsc{neg}) & \textit{-nən} (\textsc{seq}) & \textit{-sa} (\textsc{adj})\\\midrule
\textit{ban}   & Adversative &   +  & +   & \textminus & \textminus \\
\textit{mun}   & Adversative &   +  & +   & \textminus & \textminus \\
\textit{kara}  & Causal      &   +  & +   & \textminus & \textminus \\
\textit{sjutɨ} & Sequential  &  \textminus & +   &  +  &  \textminus\\
\textit{nu}    & Causal      &  \textminus & \textminus & \textminus & +   \\\lspbottomrule
\end{tabular}
\end{table}

The above table shows the kinds of the morphemes that immediately precede the concjunctive particles (i.e. the phonological hosts of the conjunctive particles). In the following sections, I will present examples of each conjunctive particle in turn.

\subsection{\textit{ban} (\textsc{advrs})}\label{sec:10.2.1}

The conjunctive particle \textit{ban} (\textsc{advrs}) always follows the participle, and the clause followed by \textit{ban} (ADVRS) functions as an adverbial clause expressing the adversative meaning such as ‘but.’

\ea\label{ex:10.30} 
\ea After \textit{-n} (\textsc{ptcp}) [= (4-20 b)]\\
  %\textsc{tm}:
      \glll    wanna  honami-{\textbar}cjan{\textbar}  naaja  siccjunban, naakjaa  jumɨnu  naaja  sijandoojaa.\\
    \textit{wan=ja}  \textit{honami-cjan}  \textit{naa=ja}  \textit{sij-tur-\Highlight{n=ban}}  \textit{naakjaa}  \textit{jumɨ=nu}  \textit{naa=ja}  \textit{sij-an=doo=jaa}\\
    1\textsc{sg}=\textsc{top}  Honami-\textsc{dim}  name=TOP  know-\textsc{prog}-\textsc{ptcp}=\textsc{advrs}  2\textsc{pl}.\textsc{hon}.\textsc{adn}Z  daughter.in.law=\textsc{gen}  name=TOP  know-\textsc{neg}=\textsc{ass}=\textsc{sol}\\
\glt     ‘I know Honami’s name, but don’t know the name of your daughter in law.’  [Co: 110328\_00.txt]

  \ex After \textit{-an} (\textsc{neg})\\
  %\textsc{tm}:
      \glll    gan  sjəə  jˀiija  sɨranban,  jiccjaccjɨdu  umujun.{\textbar}joonakanzi{\textbar}  jappa.\\
    \textit{ga-n}  \textit{sɨr-tɨ=ja}  \textit{jˀ-i=ja}  \textit{sɨr-\Highlight{an=ban}}     \textit{jiccj-sa=ccjɨ=du}  \textit{umuw-jur-n=joonakanzi}  \textit{jar-ba}\\
    \textsc{mes}-\textsc{advz}  do-\textsc{seq}=\textsc{top}  say-\textsc{inf}=TOP  do-\textsc{neg}=\textsc{advrs}  good-\textsc{adj}=\textsc{qt}=\textsc{foc}  think-\textsc{umrk}-\textsc{ptcp}=appearance  \textsc{cop}-\textsc{csl}\\
\glt     ‘(They) do not say like that, but (they) seems to think that (it is) not necessary [lit. good], so ...’  [Co: 111113\_02.txt]
\z
\z

\subsection{\textit{mun} (\textsc{advrs})}\label{sec:10.2.2}

The conjunctive particle \textit{mun} (\textsc{advrs}) always follows the participle, and the clause followed by \textit{mun} (ADVRS) functions as an adverbial clause expressing the adversative meaning such as ‘but.’

\ea\label{ex:10.31}  
  \ea
  \exi{} After \textit{{}-n} (\textsc{ptcp})\\
  \ex  %\textsc{tm}:
      \glll    mukkoojocjɨ  jˀicjanmun,  naa,  nənsjutɨjaa,  mukkonba.      \\
      \textit{mukk-oo=joo=ccjɨ}  \textit{jˀ-tar-\Highlight{n=mun}}  \textit{naa}  \textit{nə-an=sjutɨ=jaa}    \textit{mukk-on-ba}      \\
      bring-\textsc{imp}=\textsc{cfm}1=\textsc{qt}  say-\textsc{pst}-\textsc{ptcp}=\textsc{advrs}  \textsc{fil}  exist-\textsc{neg}=\textsc{seq}=\textsc{sol}   bring-NEG-\textsc{csl}     \\
      \glt       ‘(I) said, “Bring (the tape)!” However, (probably she) lost (it), and (she) won’t bring (it).’ [Co: 120415\_01.txt]

 \ex  %\textsc{tm}:
      \glll    waakjoo  mata  hanasiga  zjoozɨ,  urɨ  jappoo  jiccjanmun,  wanna  hanasiga  {\textbar}heta{\textbar}  jappa.      \\
      \textit{waakja=ja}  \textit{mata}  \textit{hanasi=ga}  \textit{zjoozɨ}  \textit{u-rɨ}  \textit{jar-boo}   \textit{jiccj-sa+ar-\Highlight{n=mun}}  \textit{waakja=ja}  \textit{hanasi=ga}  \textit{heta}  \textit{jar-ba}      \\
      1\textsc{pl}=\textsc{top}  well  speaking=\textsc{nom}  good.at  \textsc{mes}-\textsc{nlz}  \textsc{cop}-\textsc{cnd}   good-\textsc{adj}+\textsc{st}V-\textsc{ptcp}=\textsc{advrs}  1PL=TOP  speaking=NOM  poor.at  COP-\textsc{csl}  \\
      \glt       ‘If I am so, (i.e.) good at speaking, (it) would be good, but I am poor at speaking, so (I’m sorry).’ [Co: 120415\_01.txt]

  \exi{} After \textit{{}-an} (\textsc{neg})

  \ex\relax  [= (9-50 b)]\\
    %\textsc{tm}:
      \glll    hankəəcjakkoo  nənmun,  hankəəmai  zjajaa.\\
      \textit{hankəər-∅+cja-kkoo}  \textit{nə-\Highlight{an=mun}}  \textit{hankəə-∅+mai}  \textit{zjar=jaa}\\
      tumble-\textsc{inf}+want-\textsc{adj}  \textsc{st}V-\textsc{neg}=\textsc{advrs}  tumble-INF+\textsc{obl}  \textsc{cop}=\textsc{sol}\\
      \glt       ‘(I) don’t want to tumble, but will have to tumble (for the play).’ [El: 110917]
    \z
\z

  The conjunctive particle \textit{mun} (\textsc{advrs}) has the same form with the nominal \textit{mun} ‘substance.’ It is probable that they have the same origin. However, they are different morphemes at least in the modern Yuwan, since \textit{mun} (ADVRS) can be preceded by the copula participle /jan/ \textit{jar-n} (\textsc{cop}-\textsc{ptcp}), which cannot occur when the head of the adnominal clause is an ordinary nominal; see (9-67 b) in \sectref{sec:9.4.1} for more details.

\ea\label{ex:10.32}   After \textit{jar-n} (\textsc{cop}-\textsc{ptcp})\\
  %\textsc{tm}:
      \glll    sjoogacɨnu  məə  janmun,  ikjasjɨga  sjuruccjɨ,  nattəənkja  hanasjagacinaa,\\
    \textit{sjoogacɨ=nu}  \textit{məə}  \textit\Highlight{{jar-n=mun}}  \textit{ikja-sjɨ=ga}   \textit{sɨr-jur-u=ccjɨ}  \textit{naa-ttəə=nkja}  \textit{hanas-jagacinaa}\\
    the.New.Year’s.Day  front  \textsc{cop}-\textsc{ptcp}=\textsc{advrs}  how-\textsc{advz}=\textsc{foc}  do-\textsc{umrk}-\textsc{pfc}=\textsc{qt}  2.\textsc{hon}-\textsc{du}=\textsc{appr}  talk-\textsc{sim}\\
    \glt     ‘The couple was saying that, “(It) will be the New Year’s Day soon [lit. (It) is in front of the New Year’s Day], but how do (we) do?”’ [Fo: 090307\_00.txt]
\z

In \textsc{ref}{ex:10.32}, \textit{mun} (\textsc{advrs}) is preceded by \textit{jar-n} (\textsc{cop}-\textsc{ptcp}). That means \textit{mun} (ADVRS) can appear in a syntactic position different from the nominal proper. Thus, I propose that \textit{mun} (ADVRS) is a conjunctive particle in modern Yuwan.

  There are many examples where the superordinate clauses of the adverbial clause of \textit{mun} (\textsc{advrs}) are omitted. Usually, the superordinate clauses can be reconstructed by the contexts. However, there is a case where the reconstruction of the superordinate clause is difficult as in \textsc{ref}{ex:10.33}.

\ea\label{ex:10.33}   \textit{mun} (\textsc{advrs}) withouth the superordinate clause (at least in the phonetic level)\\
  %\textsc{tm}:
      \glll    jazin  kjunmuncjɨ  umutɨ  kurɨranboo.\\
    \textit{jazin}  \textit{k-jur-\Highlight{n=mun}=ccjɨ}  \textit{umuw-tɨ}  \textit{kurɨr-an-boo}\\
    necessarily  come-\textsc{umrk}-\textsc{ptcp}=\textsc{advrs}=\textsc{qt}  think-\textsc{seq}  \textsc{ben}-\textsc{neg}-\textsc{cnd}\\
\glt     ‘(You) have to think that necessarily (you) will come.’  [Co: 101023\_01.txt]
\z

Both of \textit{mun} (\textsc{advrs}) in this section and \textit{ban} (ADVRS) in \sectref{sec:10.2.1} can mean the adversative meaning. The semantic difference between them is not clear to me, and the more elaborated research is required in future.

\subsection{\textit{kara} (\textsc{csl})}\label{sec:10.2.3}

The conjunctive particle \textit{kara} (\textsc{csl}) always follows the participle, and the clause followed by \textit{kara} (CSL) functions as an adverbial clause expressing a causal meaning. I will present examples below.

\ea\label{ex:10.34}  
\ea After \textit{{}-n} (\textsc{ptcp}) [= (10-9 a)]\\
  %\textsc{tm}:
      \glll    takennan  umoojutankara,  {\textbar}hotondo{\textbar}  takennu   munbəidu  ucɨcjəija.\\
    \textit{taken=nan}  \textit{umoor-jur-tar-n=\Highlight{kara}}  \textit{hotondo}  \textit{taken=nu}  \textit{mun=bəi=du}  \textit{ucɨs-təər-i=jaa}\\
    Taken=\textsc{loc}1  exist-\textsc{umrk}-\textsc{pst}-\textsc{ptcp}=\textsc{csl}  almost  Taken=\textsc{gen}   thing=only=\textsc{foc}  take-\textsc{rsl}-\textsc{npst}=\textsc{sol}\\
\glt     ‘Since (he) used to be in Taken, (he) took only the (pictures) of Taken.’  [Co: 111113\_02.txt]

\ex After \textit{{}-an} (\textsc{neg})\\
  %\textsc{tm}:
      \glll    naa  ukuppoo, ..  wakarankara,  (mmm)  məəgadɨ  {\textbar}cjokusecu{\textbar} un  kˀurumanan  xxx\\
    \textit{naa}  \textit{ukur-boo}  \textit{wakar-an=\Highlight{kara}}    \textit{məə=gadɨ}  \textit{cjokusecu}    \textit{u-n}  \textit{kˀuruma=nan}  \\
    \textsc{fil}  send-\textsc{cnd}  know-\textsc{neg}=\textsc{csl}    place=\textsc{lmt}  directly   \textsc{mes}-\textsc{adn}Z  car=\textsc{loc}1  \\
\glt     ‘If (one) sends (the relief supplies there), (one) cannot know (whether they actually arrive there), so (the people in the village office decided to carry them) directly to the place (by loading them) on that car.’  [Co: 110328\_00.txt]
\z
\z

  In fact, the conjunctive particle \textit{kara} (\textsc{csl}) has the same form with the case particle \textit{kara} (\textsc{abl}) in \sectref{sec:6.3.2.10}, and it is probable that they have the same origin. Moreover, it is probable that both of \textit{kara} (CSL) and \textit{kara} (ABL) have the same origin with (the original constituent of) \textit{{}-təəra} ‘after’ (see \sectref{sec:9.3.2.2} for more details).

\subsection{\textit{sjutɨ} (\textsc{seq})}\label{sec:10.2.4}

The conjunctive particle \textit{sjutɨ} (\textsc{seq}) always follows \textit{{}-an} (\textsc{neg}) or \textit{{}-nən} (SEQ), and the clause followed by \textit{sjutɨ} (SEQ) functions as an adverbial clause expressing a sequential meaning. The example where \textit{sjutɨ} (SEQ) follows \textit{{}-nən} (SEQ) was already shown in (10-26 c) in \sectref{sec:10.1.6}. Thus, I will show an example of \textit{{}-an} (NEG) followed by \textit{sjutɨ} (SEQ).

\ea\label{ex:10.35}   After \textit{{}-an} (\textsc{neg})\\
  %\textsc{tm}:
      \glll    waakjoo  iziga  sɨransjutɨ,  sijan.\\
    \textit{waakja=ja}  \textit{izir-∅=ga}  \textit{sɨr-a\Highlight{n=sjutɨ}}  \textit{sij-an}\\
    1\textsc{pl}=\textsc{top}  go.out-\textsc{inf}=\textsc{nom}  do-\textsc{neg}=\textsc{seq}  know-NEG\\
\glt     ‘I was not able to go out (in those days), so (I) don’t know (it).’  [Co: 120415\_00.txt]
\z

  The clause followed by \textit{sjutɨ} (\textsc{seq}) can be used without its superordinate clause (at least in the phonetic level).

\ea\label{ex:10.36}   Withouth the superordinate clause (at least in the phonetic level)\\
  %\textsc{tm}:
      \glll    naa,  cjankjoo  waasannənsjutɨdoo\\
    \textit{naa}  \textit{cja=nkja=ja}  \textit{waas-an-\Highlight{nən=sjutɨ}=doo}\\
    \textsc{fil}  tea=\textsc{appr}=\textsc{top}  boil-\textsc{neg}-\textsc{seq}=SEQ=\textsc{ass}\\
\glt     ‘(I) have forgotten to brew up the tea (for you).’  [Co: 110328\_00.txt]
\z

  \textit{sjutɨ} (\textsc{seq}) has the same form with the converb /sjutɨ/ \textit{sɨr-tur-tɨ} (do-\textsc{prog}-SEQ), and it is probable that they have the same origin. However, I propose that they are different in modern Yuwan, since \textit{sjutɨ} (SEQ) always keeps its form (i.e. does not take another inflection) when it follows \textit{{}-an} (\textsc{neg}) or \textit{{}-nən} (SEQ). On the contrary, \textit{sɨr-} ‘do’ can take any inflection (not only \textit{{}-tur-tɨ} (PROG-SEQ)) if it is preceded by the morphemes other than \textit{{}-an} (NEG) or \textit{{}-nən} (SEQ) (see \sectref{sec:9.1.2.1} for more details).

\subsection{\textit{nu} (\textsc{csl})}\label{sec:10.2.5}

The conjunctive particle \textit{nu} (\textsc{seq}) always follows an adjective (whose inflection is \textit{{}-sa} (\textsc{adj})), and the clause followed by \textit{nu} (SEQ) functions as an adverbial clause expressing a causal meaning.

\ea\label{ex:10.37} 
  \ea\relax [= (9-44 c)]\\
    %\textsc{tm}:
      \glll    waakjoo  utussjanu,  aicjɨn  njanta.\\
      \textit{waakja=ja}  \textit{utussj-\Highlight{sa=nu}}  \textit{aik-tɨ=n}  \textit{nj-an-tar}\\
      1\textsc{pl}=\textsc{top}  fearful-\textsc{adj}=\textsc{csl}  walk-\textsc{seq}=ever  \textsc{exp}-\textsc{neg}-\textsc{pst}\\
      \glt       ‘I was fearful (of the American soldiers), so did not walk (around).’ [Co: 111113\_01.txt]
  \ex   %\textsc{tm}:
      \glll    dujasanu,  ikizɨmai  jatattujaa.\\
      \textit{duja-\Highlight{sa=nu}}  \textit{ikizɨmai}  \textit{jar-tar-tu=jaa}\\
      rich-\textsc{adj}=\textsc{csl}  comfortable  \textsc{cop}-\textsc{pst}-CSL=\textsc{sol}\\
      \glt       ‘(He) was rich, so (he) was comfortable.’ [Co: 110328\_00.txt]
    \z
\z

  \textit{nu} (\textsc{csl}) has the same form with \textit{nu} (\textsc{nom}) or \textit{nu} (\textsc{gen}), but it is difficult to regard the function of \textit{nu} (CSL) as that of \textit{nu} (NOM) or \textit{nu} (GEN), since a nominal cannot be used to express a causal meaning as in \textsc{ref}{ex:10.38}.

\ea\label{ex:10.38}   A nominal cannot precede \textit{nu} (\textsc{csl}) [= (9-68b)]\\
  %\textsc{tm}:
      \glll    *arəə  warabɨnu,  waarandaro.\\
     \textit{a-rɨ=ja}  \textit{warabɨ=\Highlight{nu}}  \textit{waar-an=daroo}\\
     \textsc{dist}-\textsc{nlz}=\textsc{top}  child=\textsc{csl}  understand-\textsc{neg}=\textsc{supp}\\
    \glt     (Intended meaning) ‘That (boy) is a child, so probably (he) cannot understand (it).’ [El: 130822]
\z

  There are examples where the clauses followed by \textit{nu} (\textsc{csl}) appear without their superordinate clause (at least in the phonetic level) as in \textsc{ref}{ex:10.39} (see also \sectref{sec:9.2.1}).

\ea\label{ex:10.39}   Withouth the superordinate clause (at least in the phonetic level)\\\relax
  [Context: Talking about the old days when people in Yuwan carried their loads by putting them on their heads]
  %\textsc{tm}:
      \glll    kan  sjɨ  muccjəə,  ubusanu.\\
    \textit{ka-n}  \textit{sɨr-tɨ}  \textit{mut-tɨ=ja}  \textit{ubu-sa=\Highlight{nu}}\\
    \textsc{prox}-\textsc{advz}  do-\textsc{seq}  hold-SEQ=\textsc{top}  heavy-\textsc{adj}=\textsc{csl}\\
\glt     ‘If (you) hold (the loads) like this [i.e. holding them under your arm], (they are) heavy, so (it is better to put them on your head).’  [Co: 111113\_02.txt]
\z

\section{Clause-final particles}\label{sec:10.3}

Yuwan has the clause-final particles as in \tabref{tab:84}. A clause-final particle can be hosted by a clause. The clause followed by a clause-final particle is not embedded into any superordinate clause (except for the case when it is followed by \textit{ccjɨ} (\textsc{qt}), which can embed any clause into the superordinate clause).

\begin{table}
\caption{Clause-final particles\label{tab:99}}
\begin{tabular}{lll}
\lsptoprule
Category & Form & Meaning\\\midrule
Speech act & \textit{doo} &  Assertion\\
           & \textit{na}  &  Polar question\\
           & \textit{i}   &  Polar question\\
           & \textit{jəə} &  Confirmation\\
           & \textit{ga}  &  Confirmation\\
Modality   & \textit{kai} &  Dubitative\\
           & \textit{daroo} & Supposition\\
           & \textit{kamo}  & Possibility\\
Others     & \textit{zjɨ}   & Direction\\
           & \textit{gadɨ}  & Limitative\\
           & \textit{wake}  & ?\\\lspbottomrule
\end{tabular}
\end{table}

In principle, a clause-final particle is not followed by another clause-final particle. However, there are three exceptions: \textit{zjɨ} (\textsc{dirc}) may be followed by \textit{jəə} (\textsc{cfm}2); \textit{daroo} (\textsc{supp}) may be followed by \textit{ga} (CFM3); and \textit{ga} (CFM3) may be followed by \textit{i} (\textsc{plq}). In the following sections, I will present examples of each clause-final particle in turn.

\subsection{\textit{doo} (\textsc{ass})}\label{sec:10.3.1}

\textit{doo} (\textsc{ass}) expresses that the proposition of the clause is a new information for the hearer.

\ea\label{ex:10.40}   \textit{doo} (\textsc{ass})\\
  \ea After the verbal predicate phrase [= (6-17 b)]\\
  %\textsc{tm}:
      \glll    samisjen  kikjunbunsjɨ  nuuutaccjəə  sɨgu  wakajuttoo.\\
    \textit{samisjen}  \textit{kik-jur-n=bun=sjɨ}  \textit{nuu+uta=ccjɨ=ja}  \textit{sɨgu}  \textit{wakar-jur=\Highlight{doo}}\\
    samisen  hear-\textsc{umrk}-\textsc{ptcp}=share=\textsc{inst}  what+song=\textsc{qt}=\textsc{top}  soon  understand-UMRK=\textsc{ass}\\
\glt     ‘Soon (I) can understand what song (it is) only by hearing (the sound of the) samisen.’  [Co: 111113\_01.txt]

  \ex After the adjectival predicate phrase\\
  %\textsc{tm}:
      \glll    amanu  mjoo  mˀasa  attoo.\\
    \textit{a-ma=nu}  \textit{mja=ja}  \textit{mˀa-sa}  \textit{ar=doo}\\
    \textsc{dist}-place=\textsc{gen}  k.o.shell.fish  tasty-\textsc{adj}  \textsc{st}V=\textsc{ass}\\
    \glt     ‘The shell fish of that place is tasty.’ [El: 110327]

  \ex  After the nominal predicate phrase\\
  %\textsc{tm}:
      \glll    kurɨ  minna  katakˀwasidoo.\\
    \textit{ku-rɨ}  \textit{minna}  \textit{kata+kˀwasi=doo}\\
    \textsc{prox}-\textsc{nlz}  all  model+sweet=\textsc{ass}\\
\glt     ‘All (of) these things are \textit{katakˀwasi} [i.e. a kind of sweets].’  [Co: 111113\_01.txt]
    \z
\z

\subsection{\textit{na} (\textsc{plq})}\label{sec:10.3.2}

\textit{na} (\textsc{plq}) expresses the polar question (i.e. the so-called “yes-no question”). Therefore, it cannot co-occur with an interrogative word.

First of all, I will show the morphophonological alternation of \textit{na} (\textsc{plq}) below. If \textit{na} (\textsc{pl}Q) follows the non-past affix \textit{{}-i}, both morphemes go through assimilation. First, \textit{na} (PLQ) becomes /nja/ being influenced by \textit{{}-i} (\textsc{npst}) (progressive palatalization). Then, \textit{{}-i} (N\textsc{pst}) becomes /n/ being influenced by /nja/ (PLQ) (regressive nasalization).

\begin{exe}
\ex\label{ex:10.41}
\textit{{}-i} (\textsc{npst})  +  \textit{na}  (\textsc{plq})  >  (palatalization)  //i=nja//  >  (nasalization)  >  /n=nja/

\ex\label{ex:10.42}
\begin{xlist}
\ex Assimilation occurs\\
  \textit{wakar-jur-i}  (understand-\textsc{umrk}{}-\textsc{npst})  +  \textit{na}  (\textsc{plq})  >  /waka-ju-n=nja/  (*/waka-ju-i=na/)
\ex Assimilation does not occur\\
  \textit{wakar-an}  (understand-\textsc{neg})  +  \textit{na}  (\textsc{plq})  >  /wakar-an=na/  (*/wakar-an=nja/)
\end{xlist}
\end{exe}

In the surface-form level, the verb-final phoneme that precedes /nja/ (\textsc{plq}) is /n/ as in (10-42 a). Thus, one might think that this /n/ is not made of \textit{{}-i} (\textsc{npst}), but think that it is the participial affix \textit{{}-n} from the beginning (see \sectref{sec:8.4.2.1}), and that there is another question particle such as \textit{nja} (besides \textit{na}). However, /nja/ that expresses the polar question appears only in affirmative (and also in the non-past tense). In negative, /na/ (not /nja/) appears as in (10-42 b). Thus, in order to explain this palatalization from //na// to /nja/, we have to postulate the existense of \textit{-i} (N\textsc{pst}) in the underlying-form level. That is, the verb-final /n/ in (10-42 a) is not \textit{{}-n} (\textsc{ptcp}).

  I will present other examples of \textit{na} (\textsc{plq}) below.

\ea\label{ex:10.43}   \textit{na} (\textsc{plq})\\
  \ea After the verbal predicate phrase whose final verb ends with \textit{{}-i} (\textsc{npst})\\
  %\textsc{tm}:
      \glll    ude,  uraga  wunnja?\\
    \textit{ude}  \textit{ura=ga}  \textit{wur-i=\Highlight{na}}\\
    hey  2.N\textsc{hon}.\textsc{sg}=\textsc{nom}  exist-\textsc{npst}=\textsc{plq}\\
\glt     ‘Hey, are you (in this picture)?’  [Co: 120415\_00.txt]

  \ex After the verbal predicate phrase whose final verb ends with \textit{{}-tɨ} (\textsc{seq})\\
  %\textsc{tm}:
      \glll    misjoocjɨna?\\
    \textit{misjoor-tɨ=\Highlight{na}}\\
    eat.\textsc{hon}-\textsc{seq}=\textsc{plq}\\
    \glt     ‘Did (you) eat (it)?’ [El: 121010]

  \ex After the adjectival predicate phrase whose final verb ends with \textit{{}-i} (\textsc{npst}) [= (9-69 c)]\\
  %\textsc{tm}:
      \glll    arəə  sijusannja?\\
    \textit{a-rɨ=ja}  \textit{siju-sa+ar-i=\Highlight{na}}\\
    \textsc{dist}-\textsc{nlz}=\textsc{top}  white-\textsc{adj}+\textsc{st}V-\textsc{npst}=\textsc{plq}\\
    \glt     ‘Is that white?’ [El: 130822]

  \ex After the nominal predicate phrase\\
  %\textsc{tm}:
      \glll    ututuuna?\\
    \textit{ututu}\footnotemark\textit{=na}\\
    younger=\textsc{plq}\\
  \glt     ‘(Is your uncle) younger (than your mother)?’  [Co: 110328\_00.txt]
  \z
\z\footnotetext{\textit{ututu} ‘younger’ is a nominal, and its word-final vowel is sometimes lengthened.}

The above examples show that \textit{na} (\textsc{plq}) can follow all kinds of the predicate phrases.

  Furthermore, if \textit{na} (\textsc{plq}) follows \textit{-sɨga} (\textsc{pol}), it expresses that the speaker tries to get the hearer to remember (or notice) the proposition (expressed by the clause it attaches to). In that case, \textit{na} (\textsc{pl}Q) does not function as a (polar) question in effect.

\ea\label{ex:10.44}   \textit{{}-sɨga=na} (\textsc{pol}=\textsc{plq})\\
  \ea  %\textsc{tm}:
      \glll    ukka  məəga  sanbasi  jatassɨgana.\\
      \textit{u-rɨ=ga}  \textit{məə=ga}  \textit{sanbasi}  \textit{jar-tar-\Highlight{sɨga=na}}\\
      \textsc{mes}-\textsc{nlz}=\textsc{gen}  front=\textsc{nom}  pier  \textsc{cop}-\textsc{pst}-\textsc{pol}=\textsc{plq}\\
      \glt       ‘(You should remember that there was) a pier in front of that.’ [lit. ‘The front of that was a pier.’] [Co: 111113\_01.txt]

  \ex  %\textsc{tm}:
      \glll    uroo  kunuguroo  {\textbar}cue{\textbar}  cukansɨgana.\\
      \textit{ura=ja}  \textit{kunuguru=ja}  \textit{cue}  \textit{cuk-an-\Highlight{sɨga=na}}\\
      2.N\textsc{hon}.\textsc{sg}=\textsc{top}  these.days=TOP  stick  stick-\textsc{neg}-\textsc{pol}=\textsc{plq}\\
      \glt       ‘(You should notice that) you don’t use the stick these days.’ [Co: 110328\_00.txt]
    \z
\z

These uses of \textit{na} (\textsc{plq}) in (10-44 a-b) seem to have some commonality with the combination of \textit{ga} (\textsc{cfm}3) and \textit{i} (\textsc{pl}Q), which also does not function as a (polar) question (see \sectref{sec:10.3.5} for more details).

\subsection{\textit{i} (\textsc{plq})}\label{sec:10.3.3}

\textit{i} (\textsc{plq}) expresses the polar question (i.e. the so-called “yes-no question”) as well as \textit{na} (\textsc{pl}Q). However, the words that can precede \textit{i} (PLQ) are partly different from \textit{na} (PLQ). \textit{i} (PLQ) can follow \textit{{}-oo} (\textsc{int}), \textit{{}-u} (\textsc{pfc}), \textit{{}-təəra} ‘after,’ and nominals (see aslo \sectref{sec:2.4.3}). It can also follow \textit{ga} (\textsc{cfm}3), which is another clause-final particle (see \sectref{sec:10.3.5}).

\ea\label{ex:10.45}   \textit{i} (\textsc{plq})\\
  \ea After the verbal predicate whose final verb ends with \textit{-oo} (\textsc{int})\\
  %\textsc{tm}:
      \glll    nun  nənboo,  kurɨroi?\\
    \textit{nuu=n}  \textit{nə-an-boo}  \textit{kurɨr-oo=\Highlight{i}}\\
    what=even  exist-\textsc{neg}-\textsc{cnd}  give-\textsc{int}=\textsc{plq}\\
    \glt     ‘If (you) don’t have anything, (should I) give (something to you)?’ [El: 110327]

  \ex After the verbal predicate whose final verb ends with \textit{-u} (\textsc{pfc}) [= (8-76 d)]\\
  %\textsc{tm}:
      \glll    kurəə  {\textbar}maiku{\textbar}du  muccjurui? kun  cˀjoo.\\                                                               
    \textit{ku-rɨ=ja}  \textit{maiku=du}  \textit{mut-tur-u=\Highlight{i}}  \textit{ku-n}  \textit{cˀju=ja}\\                                                               
    \textsc{prox}-\textsc{nlz}=\textsc{top}  microphone=\textsc{foc}  hold-\textsc{prog}-\textsc{pfc}=\textsc{plq}  PROX-\textsc{adn}Z  person=TOP \\
    \glt ‘About this (picture), is this person holding a microphone?’ [Co: 111113\_02.txt]

  \ex After the verbal predicate whose final verb ends with \textit{{}-təəra} ‘after’ [= (6-11 b)]\\
  %\textsc{tm}:
      \glll    nanga  kunəəda  umoocjasəə  kun   cˀjunu  cˀjəərai?\\
    \textit{nan=ga}  \textit{kunəəda}  \textit{umoor-tar=sɨ=ja}  \textit{ku-n}  \textit{cˀju=nu}  \textit{k-təəra=\Highlight{i}}\\
    2.\textsc{hon}.\textsc{sg}=\textsc{nom}  the.other.day  come.HON-\textsc{pst}=\textsc{fn}=\textsc{top}  \textsc{prox}-\textsc{adn}Z   person=NOM  come-after=\textsc{plq}\\
\glt     ‘(Is it) after this person [i.e. the present author] came (to your house) that you [i.e. US] came (here) the other day?’  [Co: 110328\_00.txt]

  \ex After the nominal predicate\\\relax
  [Context: \textsc{tm} called Umine who had just arrived in front of the TM’s house.]\\
  %\textsc{tm}:
      \glll    uminenəi?\\
    \textit{umine+nəə=\Highlight{i}}\\
    Umine+elder.sister=\textsc{plq}\\
\glt     ‘(Are you) Umine?’  [Co: 110328\_00.txt]

  \ex After \textit{ga} (\textsc{cfm}3)\\
  %\textsc{tm}:
      \glll    naokonəəcjɨ  wanga  jˀicjaroogai?\\
    \textit{naoko+nəə=ccjɨ}  \textit{wan=ga}  \textit{jˀ-tar-oo=ga=\Highlight{i}}\\
    Naoko+elder.sister=\textsc{qt}  1\textsc{sg}=\textsc{nom}  say-\textsc{pst}-\textsc{supp}=\textsc{cfm}3=\textsc{plq}\\
  \glt     ‘(You remember that) I said Naoko (before), (don’t you)?’  [Co: 120415\_00.txt]
  \z
\z

In (10-45 a), \textit{i} (\textsc{plq}) follows \textit{{}-oo} (\textsc{int}). \textit{{}-oo} (INT) expresses the speaker’s intention (see \sectref{sec:8.5.1.2}). It is unnatural to assume that the speaker asks the hearer whether the speaker herself has any attention to do the action indicated by the verbal stem. In fact, the combination of \textit{{}-oo} (INT) and \textit{i} (\textsc{pl}Q) asks the hearer whether the speaker’s intention to do the action indicated by the verbal stem is appropriate in the hearer’s view.

\subsection{\textit{jəə} (\textsc{cfm}2)}\label{sec:10.3.4}

\textit{jəə} (\textsc{cfm}2) always follows \textit{-oo} (\textsc{int}) as in \textsc{ref}{ex:10.46}. The speaker tries to make sure that the hearer agree with the speaker’s action by \textit{jəə} (CFM2). They may be intervened by \textit{zjɨ} (\textsc{dirc}), which is another clause-final particle (see \sectref{sec:10.3.9}).

\ea\label{ex:10.46}   \textit{{}-oo}=\textit{jəə} (\textsc{int}=\textsc{cfm}2) [= (8-59 b)]\\
  \ea \textsc{tm}:
      \glll    {\textbar}onigiri{\textbar}  sjɨ,  mutasoojəə.\\
      \textit{onigiri}  \textit{sɨr-tɨ}  \textit{mut-as-\Highlight{oo=jəə}}\\
      rice.ball  do-\textsc{seq}  have-\textsc{caus}-\textsc{int}=\textsc{cfm}2\\
      \glt       ‘(I) will make a rice ball, and get (the present author) to have (it).’ [Co: 101023\_01.txt]

  \ex  US: 
      \glll wanna  ikjoojəə.\\
      \textit{wan=ja}  \textit{ik-\Highlight{oo=jəə}}\\
      1\textsc{sg}=\textsc{top}  go-\textsc{int}=\textsc{cfm}2\\
      \glt       ‘I will go (back home).’ [Co: 110328\_00.txt]
    \z
\z

The verb that includs \textit{{}-oo}=\textit{jəə} (\textsc{int}=\textsc{cfm}2) necessarily excludes the hearer from the action indicated by the verbal stem. On the contrary, \textit{{}-oo=jaa} (INT=\textsc{sol}) necessarily includes the hearer from the action indicated by the verbal stem (see \sectref{sec:10.5.2.2} for more details).

\subsection{\textit{ga} (\textsc{cfm}3)}\label{sec:10.3.5}

\textit{ga} (\textsc{cfm}3) follows \textit{{}-oo} (\textsc{supp}) or \textit{daroo} (SUPP) as in \textsc{ref}{ex:10.47} with the exception where it follows a verbal root as in (10-48 a-b). Additionally, \textit{ga} (CFM3) may be followed by \textit{i} (\textsc{plq}) as in (10-47 b, d). The combinations of \textit{{}-oo=ga} (SUPP=CFM3) or \textit{daroo=ga} (SUPP=CFM3) express that the speaker wants the hearer to confrim the speaker’s suppostion (or memory).

\ea\label{ex:10.47}   
 \begin{xlist}
 \exi{} \textit{{}-oo} (\textsc{supp}) + \textit{ga} (\textsc{cfm}3)\\
  \ex  %\textsc{tm}:
      \glll    uraga  (mm  koo)  naradutɨ,  kootancjɨ    jurooga.\\
      \textit{ura=ga}    \textit{koow-}  \textit{narab-tur-tɨ}  \textit{koow-tar-n=ccjɨ}  \textit{jˀ-jur-\Highlight{oo=ga}}\\
      2.N\textsc{hon}.\textsc{sg}=\textsc{nom}    buy-  line.up-\textsc{prog}-\textsc{seq}  buy-\textsc{pst}-\textsc{ptcp}=\textsc{qt}  say-\textsc{umrk}-\textsc{supp}=\textsc{cfm}3\\
      \glt       ‘(I hope you remember that) you say that (you) lined up to buy (the lunch box).’ [Co: 101023\_01.txt]

  \ex\relax  [= (8-41)]\\
    %\textsc{tm}:
      \glll    wanga  kicjuncjɨ  umutɨdu,  urattəə   gan  sjan  aran  hanasi  sjaroogai?\\
      \textit{wan=ga}  \textit{kik-tur-n=ccjɨ}  \textit{umuw-tɨ=du}  \textit{urattəə}   \textit{ga-n}  \textit{sɨr-tar-n}  \textit{ar-an}  \textit{hanasi}  \textit{sɨr-tar-\Highlight{oo=ga=i}}\\
      1\textsc{sg}=\textsc{nom}  hear-\textsc{prog}-\textsc{ptcp}=\textsc{qt}  think-\textsc{seq}=\textsc{foc}  2.N\textsc{hon}.\textsc{du}  \textsc{mes}-\textsc{adn}Z  do-\textsc{pst}-PTCP  \textsc{cop}-\textsc{neg}  tale  do-P\textsc{st}-\textsc{supp}=\textsc{cfm}3=\textsc{plq}\\
      \glt       ‘Probably you told the unlikely tale like that since (you) thought that I was listening to (that), didn’t you?’ [Fo: 090307\_00.txt]

  \exi{} \textit{daroo} (\textsc{supp}) \textit{+} \textit{ga} (\textsc{cfm}3)

  \ex  %\textsc{tm}:
      \glll    cuburuga  kumadarooga.\\
      \textit{cuburu=ga}  \textit{ku-ma=\Highlight{daroo=ga}}\\
      head=\textsc{nom}  \textsc{prox}-place=\textsc{supp}=\textsc{cfm}3\\
      \glt       ‘(I hope you admit that the place indicated by the word) \textit{cuburu} [i.e. head] is here.’ [Co: 110328\_00.txt]

  \ex  %\textsc{tm}:
      \glll    waakja  jinganu  kˀwankjoo  wurandaroogai?\\
      \textit{waakja-a}  \textit{jinga=nu}  \textit{kˀwa=nkja=ja}  \textit{wur-an=\Highlight{daroo=ga=i}}\\
      1\textsc{pl}-\textsc{adn}Z  male=\textsc{gen}  child=\textsc{appr}=\textsc{top}  exist-\textsc{neg}=\textsc{supp}=\textsc{cfm}3=\textsc{plq}\\
      \glt       ‘Probably there aren’t my sons [lit. male children], are they?’ [Co: 120415\_00.txt]
    \end{xlist}
\z

It is probable that \textit{i} (\textsc{plq}) that follows \textit{ga} (\textsc{cfm}3) as in (10-47 b, d) does not express the polar question. Rather, it seems that \textit{i} (\textsc{pl}Q) strengthenes the function of \textit{ga} (CFM3). This is exemplified more clearly in \textsc{ref}{ex:10.73} in \sectref{sec:10.4.1.6}. In that example, the speaker told the hearer about the film that the hearer had not seen. In that case, it is natural to think that the hearer do not know the contents of the film. Furthermore, it is unnatural that the speaker, who watched the film, asks the hearer about that. Thus, \textit{i} (PLQ) in that example does not express the polar question in effect. Rather, the speaker tried hard to get the speaker to understand the story by the expression, i.e. \textit{{}-oo=ga=i} (\textsc{supp}=CFM3=PLQ).

In almost all of the examples in my texts, \textit{ga} (\textsc{cfm}3) follows \textit{-oo} (\textsc{supp}) or \textit{daroo} (SUPP). However, there is an example where \textit{ga} (CFM3) follows a verbal root as in (10-48 a). There is a similar example in elicitation as in (10-48 b).

\ea\label{ex:10.48}   Verbal root + \textit{ga} (\textsc{cfm}3)\\
  \ea  %\textsc{tm}:
      \glll    namawui  jappoo,  wukka.\\
      \textit{namawui}  \textit{jar-boo}  \textit{wur=\Highlight{ga}}\\
      now  \textsc{cop}-\textsc{cnd}  exist=\textsc{cfm}3\\
      \glt       ‘(The shopkeeper) will be there now.’ [Co: 110328\_00.txt]

  \ex  %\textsc{tm}:
      \glll    kjurasa  akka.\\
      \textit{kjura-sa}  \textit{ar=\Highlight{ga}}\\
      beautiful-\textsc{adj}  \textsc{st}V=\textsc{cfm}3\\
      \glt       ‘(It) is beautiful.’ [El: 12921]
    \z
\z

  \textit{ga} (\textsc{cfm}3) has the same form with \textit{ga} (\textsc{foc}). However, I have not yet found the diachronic relation or the synchronic commonality between these two morphemes.

\subsection{\textit{kai} (\textsc{du}B)}\label{sec:10.3.6}

\textit{kai} (\textsc{du}B) expresses the speaker’s dubitation over the proposition expressed by the clause it attaches to. It may co-occur with the interrogative word as in (10-49 d), which is different from \textit{na} (\textsc{plq}) and \textit{i} (\textsc{pl}Q). Additionally, the verbal forms that can precede \textit{kai} (\textsc{dub}) are not so restricted as those of \textit{na} (PLQ) and \textit{i} (PLQ).

\ea\label{ex:10.49}   \textit{kai} (\textsc{du}B)\\
  \ea After the verbal predicate whose final verb ends with \textit{{}-tar} (\textsc{pst})\\
  %\textsc{tm}:
      \glll    cukujun  cˀjunu  wutakai?\\
    \textit{cukur-jur-n}  \textit{cˀju=nu}  \textit{wur-tar=\Highlight{kai}}\\
    make-\textsc{umrk}-\textsc{ptcp}  person=\textsc{nom}  exist-\textsc{pst}=\textsc{du}B\\
\glt     ‘Was there a person who made (a silk from a cocoon)?’  [Co: 111113\_01.txt]

  \ex After the verbal predicate whose final verb ends with \textit{{}-tɨ} (\textsc{seq})\\
  %\textsc{tm}:
      \glll    {\textbar}hoka{\textbar}nuturookara  maju  mucjɨ  kii  jatɨkai?\\
    \textit{hoka=nu=turoo=kara}  \textit{maju}  \textit{mut-tɨ}  \textit{k-i}  \textit{jar-tɨ=\Highlight{kai}}\\
    other=\textsc{gen}=place=\textsc{abl}  silk  have-\textsc{seq}  come-\textsc{inf}  \textsc{cop}-SEQ=\textsc{du}B\\
\glt     ‘Did (people) bring the silk from another place?’  [Co: 111113\_01.txt]

  \ex After the adjectival predicate whose final verb ends with the verbal root \textit{ar-} (\textsc{st}V)\\
  %\textsc{tm}:
      \glll    arəə  sijusa  akkai?\\
    \textit{a-rɨ=ja}  \textit{siju-sa}  \textit{ar=\Highlight{kai}}\\
    \textsc{dist}-\textsc{nlz}=\textsc{top}  white-\textsc{adj}  \textsc{st}V=\textsc{du}B\\
    \glt     ‘Is that white?’ [El: 130822]

  \ex After the nominal predicate whose head is \textit{daa} ‘where’ (the interrogative word)\\
  %\textsc{tm}:
      \glll    kurəə  daakai?\\
    \textit{ku-rɨ=ja}  \textit{daa=\Highlight{kai}}\\
    \textsc{prox}-\textsc{nlz}=\textsc{top}  where=\textsc{du}B\\
\glt     ‘Where is this (place on the picture)?’  [Co: 111113\_01.txt]

  \ex After the nominal predicate whose head is \textit{gakkoo} ‘school’ (a common noun) [= (6-117 d)]\\
  %\textsc{tm}:
      \glll    naakjaga  {\textbar}socugjoo{\textbar}  sjəəraga  waakjoo  {\textbar}gakkoo{\textbar}kai?\\
    \textit{naakja=ga}  \textit{socugjoo}  \textit{sɨr-təəra=ga}  \textit{waakja=ja}  \textit{gakkoo=\Highlight{kai}}\\
    2.\textsc{hon}.\textsc{pl}=\textsc{nom}  graduation  do-after=\textsc{foc}  1PL=\textsc{top}  school=\textsc{du}B\\
    \glt     ‘(Is it) after you had graduated (from the elementary school, when) I (began to go to) school?’  [Co: 110328\_00.txt]
    \z
\z

As mentioned before, the finite-form affix \textit{{}-tar} (\textsc{pst}) cannot be used in the interrogative clause, and in that case, \textit{{}-tɨ} (\textsc{seq}) is used instead to express the past tense (see also \sectref{sec:8.4.1.1} and \sectref{sec:11.2.1} for more details). However, \textit{kai} (\textsc{du}B) can be used with \textit{{}-tar} (P\textsc{st}) as in (10-49 a), since it expresses the speaker’s wondering to herself. In other words, the clauses followed by \textit{kai} (\textsc{dub}) are not addressed to the hearer directly. In addition, \textit{kai} (DUB) can co-occur \textit{{}-tɨ} (SEQ) as in (10-49 b) as well. The function of \textit{kai} (DUB), which avoids direct question to the hearer, is more clearly shown in \textsc{ref}{ex:10.50}, where the interrogative word for the information question, i.e. \textit{nuu} ‘what,’ can co-occur with \textit{{}-tar} (PST) since the clause is followed by \textit{kai} (DUB).

\ea\label{ex:10.50}   \textit{nuu} ‘what’ co-occuring with \textit{{}-tar} (\textsc{pst}) because of \textit{kai} (\textsc{du}B) \\\relax
  [Context: \textsc{ms} asked \textsc{tm} whether the place in the picture used to be called “Yubinhana.”]\\
  %\textsc{tm}:
      \glll    nuucjɨga  jutakaijaa?\\
    \textit{nuu=ccjɨ=ga}  \textit{jˀ-jur-tar=kai=jaa}\\
    what=\textsc{qt}=\textsc{foc}  call-\textsc{umrk}-\textsc{pst}=\textsc{du}B=\textsc{sol}\\
\glt     ‘(I) wonder what (people) used to call (the place).’  [Co: 120415\_00.txt]
\z

\textit{kai} (\textsc{du}B) may be followed by the utterance-final particle B \textit{jaa} (\textsc{sol}). In that case, \textit{kai} (\textsc{dub}) may retain its form as in \textsc{ref}{ex:10.50} and (10-51 a), or may lose one of its word-final vowel, i.e., become /ka/, as in (10-51 b).

\ea\label{ex:10.51}   \textit{kai} (\textsc{du}B) + \textit{jaa} (\textsc{sol})\\
  \ea  %\textsc{tm}:
      \glll    kunnagatɨɨnu  {\textbar}sjoobainin{\textbar}na  wurantɨ\Highlight{kaijaa}.\\
      \textit{ku-n=nagatɨɨ=nu}  \textit{sjoobainin=ja}  \textit{wur-an-tɨ=\Highlight{kai=jaa}}\\
      \textsc{prox}-\textsc{adn}Z=along=\textsc{gen}  merchant=\textsc{top}  exist-\textsc{neg}-\textsc{seq}=\textsc{du}B=\textsc{sol}\\
      \glt       ‘Wasn’t there a merchant from this neighborhood?’ [Co: 111113\_01.txt]

  \ex %\textsc{tm}:
      \glll    {\textbar}sjuusjengo{\textbar}ja  aran\Highlight{kajaa}?\\
      \textit{sjuusjengo=ja}  \textit{ar-an=\Highlight{kai=jaa}}\\
      after.war=\textsc{top}  \textsc{cop}-\textsc{neg}=\textsc{du}B=\textsc{sol}\\
      \glt       ‘Isn’t (this picture taken) after the war [i.e. World War II]?’ [Co: 111113\_01.txt]
    \z
\z

\subsection{\textit{daroo} (\textsc{supp})}\label{sec:10.3.7}

\textit{daroo} (\textsc{supp}) expresses the speaker’s supposition. It sometimes becomes /daro/ before \textit{ccjɨ} (\textsc{qt}) or \textit{jaa} (\textsc{sol}). \textit{daroo} (SUPP) follows \textit{{}-an} (\textsc{neg}) as in (10-52 a), \textit{{}-tɨ} (\textsc{seq}) as in (10-52 b), or the nominal predicate as in (10-52 c).

\ea\label{ex:10.52}   \textit{daroo} (\textsc{supp})\\
  \ea After the verbal predicate whose final verb ends with \textit{{}-an} (\textsc{neg})\\
  %\textsc{tm}:
      \glll    sijandaroo.\\
    \textit{sij-an=\Highlight{daroo}}\\
    know-\textsc{neg}=\textsc{supp}\\
\glt     ‘(He) maybe does not know (the river boat).’  [Co: 111113\_01.txt]

  \ex After the verbal predicate whose final verb ends with \textit{{}-tɨ} (\textsc{seq})\\
  %\textsc{tm}:
      \glll    gan  sjɨ  natɨ,  (naa)  naa  mudutɨdaroccjɨ   umututanwakejo.\\
    \textit{ga-n}  \textit{sɨr-tɨ}  \textit{nar-tɨ}  \textit{naa}  \textit{naa}  \textit{mudur-tɨ=\Highlight{daroo}=ccjɨ}   \textit{umuw-tur-tar-n=wake=joo}\\
    \textsc{mes}-\textsc{advz}  do-\textsc{seq}  \textsc{cop}-SEQ  already  already  return-SEQ=\textsc{supp}=\textsc{qt}  think-\textsc{prog}-\textsc{pst}-\textsc{ptcp}=\textsc{cfp}=\textsc{cfm}1\\
\glt     ‘Then [lit. Since (it) does like that], (I)’ve been thinking that (the present author) had probably already returned (to Tokyo).’  [Co: 110328\_00.txt]

  \ex After the nominal predicate\\
  %\textsc{tm}:
      \glll    {\textbar}sannin{\textbar}na  mata,  naa,  uma ..  tuujun  cˀjudaroo.\\
    \textit{sannin=ja}  \textit{mata}  \textit{naa}  \textit{u-ma}  \textit{tuur-jur-n} \textit{cˀju=\Highlight{daroo}}\\
    three.person.\textsc{clf}=\textsc{top}  again  \textsc{fil}  \textsc{mes}-place  pass-\textsc{umrk}-\textsc{ptcp}  person=\textsc{supp}  \\
    \glt     ‘Probably, the three people are people who pass there.’   [\textsc{pf}: 090225\_00.txt]
\z
\z

The verbal affix \textit{{}-oo} (\textsc{supp}), which has the same function with \textit{daroo} (SUPP), cannot directly follow \textit{{}-an} (\textsc{neg}) (see \sectref{sec:8.4.1.2}). Thus, \textit{daroo} (SUPP), which can directly follow \textit{-an} (NEG), fills the blank of the combination as in (10-52 a).

  One might think that \textit{daroo} (\textsc{supp}) is composed of a copula verbal root plus \textit{{}-oo} (SUPP), i.e. \textit{dar-oo} (\textsc{cop}-SUPP). In fact, there is an example where \textit{dar-} (COP) takes another inflection, e.g., /dajoottoo/ \textit{dar-joor=doo} (COP-\textsc{pol}=\textsc{ass}) in elicitation. However, the copula does not use the morpheme \textit{dar-} in principle (see \sectref{sec:8.3.3}). Furthermore, \textit{daroo} (SUPP) can follow another copula as in \textsc{ref}{ex:10.53}.

\ea\label{ex:10.53}   \textit{daroo} (\textsc{supp}) following another copula verb [= (8-86 a)]\\
  %\textsc{tm}:
      \glll    nɨɨzinnu  appa,  arandaroo.\\
    \textit{nɨɨzin=nu}  \textit{ar-ba}  \textit{\Highlight{ar}-an=\Highlight{daroo}}\\
    carrot=\textsc{nom}  exist-\textsc{csl}  \textsc{cop}-\textsc{neg}=\textsc{supp}\\
\glt     ‘There are (pieces of ) a carrot, so maybe (the pickles) are not (mine).’  [Co: 101023\_01.txt]
\z

This example is not regarded as an example where an adnominal clause fills the head of the nominal predicate such as \{[\textit{ar-an}]\textsubscript{Adnominal clause} \textit{dar-oo}\}\textsubscript{Nominal predicate phrase} (\textsc{cop}-\textsc{neg} COP-\textsc{supp}), since the predicate-final copula verb in that case has to take the negative affix \textit{-an} (see \sectref{sec:9.3.2.1} for more details). Thus, I propose that \textit{daroo} (SUPP) is different from the copula verb, and that it has to be regarded as a clause-final particle in modern Yuwan.

\subsection{\textit{kamo} (\textsc{pos})}\label{sec:10.3.8}

\textit{kamo} (\textsc{pos}) expresses that the speaker thinks it is possible for the proposition (expressed by the clause followed by \textit{kamo} (POS)) to be true. \textit{kamo} (POS) sometimes becomes /kamu/ as in (10-54 b).

\ea\label{ex:10.54}   \textit{kamo} (\textsc{pos})\\
  \begin{xlist}
  \exi{} After the verbal predicate
  \ex  %\textsc{tm}:
      \glll    unnən  akkamo.\\
      \textit{u-n=nən}  \textit{ar=\Highlight{kamo}}\\
      \textsc{mes}-\textsc{adn}Z=\textsc{loc}1  exist=\textsc{pos}\\
      \glt       ‘(It is) possible (that it) is there.’ [Co: 120415\_00.txt]

  \ex  %\textsc{tm}:
      \glll    ziisanga  utasjaa  jatəkkamu.\\
      \textit{ziisan=ga}  \textit{uta+sɨr-jaa}  \textit{jar-təər=\Highlight{kamo}}\\
      grandfather=\textsc{nom}  song+do-person  \textsc{cop}-\textsc{rsl}=\textsc{pos}\\
      \glt       ‘(It may be true that your) grandfather was a singer.’ [Co: 111113\_01.txt]

  \exi{} After the nominal predicate

  \ex  %\textsc{tm}:
      \glll    kuduu  sjəəsɨkamo.\\
      \textit{kudu}  \textit{sɨr-təər=sɨ=\Highlight{kamo}}\\
      last.year  do-\textsc{rsl}=\textsc{fn}=\textsc{pos}\\
      \glt       ‘(It is) possible (that the pickles) are those that were made in the last year.’ [Co: 101023\_01.txt]
    \end{xlist}
\z

The example where \textit{kamo} (\textsc{pos}) follows the adjectival predicate phrase is shown in (10-62 d) in \sectref{sec:10.4.1.1.}

\subsection{\textit{zjɨ} (\textsc{dirc})}\label{sec:10.3.9}

\textit{zjɨ} (\textsc{dirc}) expresses that the action indicated by the clause (it attaches to) occurs in the place different from where the speaker exists at the utterance time. It is probable that \textit{zjɨ} (D\textsc{ir}C) was grammaticalized from /izjɨ/ \textit{ik-tɨ} (go-\textsc{seq}) as well as \textit{zjɨ} (\textsc{loc}3) (see \sectref{sec:6.3.4} for more details). \textit{zjɨ} (DIRC) intervenes between \textit{{}-oo} (\textsc{int}) and \textit{jəə} (\textsc{cfm}2) as in (10-55 a), or follows \textit{{}-ɨba} (\textsc{sugs}) as in (10-55 b).

\ea\label{ex:10.55}   \textit{zjɨ} (\textsc{dirc})\\
  \ea Between \textit{{}-oo} (\textsc{int}) and \textit{jəə} (\textsc{cfm}2)\\
  %\textsc{tm}:
      \glll    amazjɨ  nudɨ  koozjɨjəə.\\
    \textit{a-ma=zjɨ}  \textit{num-tɨ}  \textit{k-oo=\Highlight{zjɨ}=jəə}\\
    \textsc{dist}-place=\textsc{loc}3  drink-\textsc{seq}  come-\textsc{int}=\textsc{dirc}=\textsc{cfm}2\\
    \glt     ‘(I) will go to drink (alcohol) there.’ [El: 110330]

  \ex After \textit{{}-ɨba} (\textsc{sugs})\\\relax
  [Context: Talking to a child who wants to buy something he wants]\\
  %\textsc{tm}:
      \glll    narabɨbazjɨ.\\
    \textit{narab-ɨba=\Highlight{zjɨ}}\\
    line.up-\textsc{sugs}=\textsc{dirc}\\
    \glt     ‘How about lining up going there (to buy it)?’ [El: 110914]
    \z
\z

\subsection{\textit{gadɨ} (\textsc{lmt})}\label{sec:10.3.10}

The clause-final particle \textit{gadɨ} (\textsc{lmt}) always follows the adjective (taking the inflection \textit{{}-sa} (\textsc{adj})).

\ea\label{ex:10.56}   \textit{gadɨ} (\textsc{lmt})\\\relax
  [Context: Talking about a butterfly that is similar to the moth] = (5-28 a)\\
  %\textsc{tm}:
      \glll    arɨga  nissjagadɨ.  ganbəi  sjɨ   kucjəə  tugaracjɨ,\\
    \textit{a-rɨ=ga}  \textit{\Highlight{nissj-sa=gadɨ}}  \textit{ga-n=bəi}  \textit{sɨr-tɨ}   \textit{kuci=ja}  \textit{tugaras-tɨ}\\
    \textsc{dist}-\textsc{nlz}=\textsc{nom}  similar-\textsc{adj}=\textsc{lmt}  \textsc{mes}-\textsc{advz}=about  do-\textsc{seq}   mouth=\textsc{top}  pout-SEQ\\
\glt     ‘That one is very similar (to the moth). (The size is) about this, and it pouted, and ...’  [Co: 111113\_01.txt]
\z

In \textsc{ref}{ex:10.56}, \textit{gadɨ} (\textsc{lmt}) seems to have some emphatic meaning, but the detail of the function is not clear to the present author for now. It is probable that the clause-fianl particle \textit{gadɨ} (L\textsc{mt}) has the same origin with the case particle \textit{gadɨ} (LMT), the limiter particle \textit{gadɨ} (LMT), and the verbal affix \textit{{}-gadɨ} ‘until.’

\subsection{\textit{wake} (\textsc{cfp})}\label{sec:10.3.11}

It is probable that the clause-final particle \textit{wake} (\textsc{cfp}) was borrowed from standard Japanese recently, since it includes //e//, which is rarely used in the traditional morphemes in Yuwan (see note “e” of \tabref{tab:4} in \sectref{sec:2.2.1.1}). However, \textit{wake} (CFP) is frequently used in the monologue or the conversation in Yuwan. Thus, I will include it in the present paper, although its function is not very clear for the present author. Therefore, it is abbreviated only as “CFP” (i.e. clause-final particle). \textit{wake} (CFP) always follows the participle.

\ea\label{ex:10.57}   \textit{wake} (\textsc{cfp})\\
  \ea After \textit{{}-n} (\textsc{ptcp}) [= (7-12 a)]\\
  %\textsc{tm}:
      \glll    un  kagonu  tˀɨɨ  cɨdɨ  ikjunwake.\\
    \textit{u-n}  \textit{kago=nu}  \textit{tˀɨɨ}  \textit{cɨm-tɨ}  \textit{ik-jur-n=\Highlight{wake}}\\
    \textsc{mes}-\textsc{advz}  basket=\textsc{gen}  one.\textsc{clf}.thing  load-\textsc{seq}  go-\textsc{umrk}-\textsc{ptcp}=\textsc{cfp}\\
    \glt     ‘(The boy) puts the one of the baskets on (the front of his bicycle) and goes.’ [\textsc{pf}: 090222\_00.txt]

  \ex After \textit{{}-an} (\textsc{neg})\\
  %\textsc{tm}:
      \glll    kootookˀwaja  izituranwakejo.\\
    \textit{kootoo+kˀwa=ja}  \textit{izir-tur-an=wake=joo}\\
    high.level+lesson=\textsc{top}  go.out-\textsc{prog}-\textsc{neg}=\textsc{cfp}=\textsc{cfm}1\\
  \glt     ‘(She) has not graduated from the junior high school.’  [Co: 120415\_00.txt]
  \z
\z


In fact, there is only an example in the text data where \textit{wake} is followed by the copula verb as in \textsc{ref}{ex:10.58}. It is probable that \textit{wake} (\textsc{cfp}) is on the way from the formal noun to the clause-final particle, since it does not take any case particle and there is no example where it is modified by the adnominal word.

\ea\label{ex:10.58}   \textit{wake} followed by the copular verb [= (7-3 c)]\\
  %\textsc{tm}:
      \gllll    jaanu  məəninkjadu  gan  sjɨ   sagɨjutanwake  zjajaa.\\
    \textit{jaa=nu}  \textit{məə=nan=nkja=du}  \textit{ga-n}  \textit{sɨr-tɨ} \textit{sagɨr-jur-tar-n=\Highlight{wake}}  \textit{\Highlight{zjar}=jaa}\\
    house=\textsc{gen}  front=\textsc{loc}1=\textsc{appr}=\textsc{foc}  \textsc{mes}-\textsc{advz}  do-\textsc{seq}  hang-\textsc{umrk}-\textsc{pst}-\textsc{ptcp}=\textsc{fn}  \textsc{cop}=\textsc{sol}\\
    Modifier  Head\\
    \glt  ‘(They) would hang (bundles of rice) in front of (their) houses like this.’ [Co: 111113\_02.txt]
\z

\section{Utterance-final particles A}\label{sec:10.4}

Yuwan has the utterance-final particles A as in \tabref{tab:100}. The utterance-final particles A can be hosted by the utterance, and the units followed by the utterance-final particles A are always embedded into the superordinate clauses (except for the case in \sectref{sec:10.4.1.7}). The term “utterance” here is used to indicate an abstract unit that can include both the phrase and the clause.

\begin{table}
\caption{Utterance-final particles A\label{tab:100}}
\begin{tabular}{ll}
\lsptoprule
Form & Meaning\\\midrule
\textit{ccjɨ}     & Quotation \\
\textit{ka}       & Dubitation\\
\textit{gajaaroo} & Dubitation\\
\textit{nən}      & ‘such as’ \\
\lspbottomrule
\end{tabular}
\end{table}

\subsection{\textit{ccjɨ} (\textsc{qt})}\label{sec:10.4.1}

The quotative particle \textit{ccjɨ} (\textsc{qt}) can make an utterance embedded in the complement slot of the superordinate clause. First, I will show the morphophonological alternation of \textit{ccjɨ} (QT) below. If \textit{ccjɨ} (QT) follows //n// or a diphthong (“V\textit{\textsubscript{i}}V\textit{\textsubscript{j}}”), the initial morphophoneme //c// of \textit{ccjɨ} is always deleted. If \textit{ccjɨ} (QT) follows a long vowel (“V\textit{\textsubscript{i}}V\textit{\textsubscript{i}}”), the initial morphophoneme //c// of \textit{ccjɨ} tends to be deleted, but sometimes the long vowel becomes short, and furthermore, there are a few cases where the long vowel becomes short and also //c// of \textit{ccjɨ} is deleted. Otherwise, i.e. after a short vowel, \textit{ccjɨ} retains its form (although it sometimes becomes /cjɨ/).

\ea\label{ex:10.59}   Rule schemata\\
  \ea 
  \begin{tabbing}
  Elsewhere \hspace{\tabcolsep}\=\hspace{\tabcolsep} +  \textit{ccjɨ} (\textsc{qt}) \hspace{\tabcolsep}\=\hspace{\tabcolsep} > \hspace{\tabcolsep}\=\hspace{\tabcolsep} /n=cjɨ/\kill
  //n// \> + \textit{ccjɨ} (\textsc{qt}) \> > \> /n=cjɨ/
  \end{tabbing}
  \ex  \begin{tabbing}
  Elsewhere \hspace{\tabcolsep}\=\hspace{\tabcolsep} +  \textit{ccjɨ} (\textsc{qt}) \hspace{\tabcolsep}\=\hspace{\tabcolsep} > \hspace{\tabcolsep}\=\hspace{\tabcolsep} /n=cjɨ/\kill
  //V\textit{\textsubscript{i}}V\textit{\textsubscript{j} }//  \>   \> > \> /V\textit{\textsubscript{i}}V\textit{\textsubscript{j} }=cjɨ/
  \end{tabbing}
  \ex \begin{tabbing}
  Elsewhere \hspace{\tabcolsep}\=\hspace{\tabcolsep} +  \textit{ccjɨ} (\textsc{qt}) \hspace{\tabcolsep}\=\hspace{\tabcolsep} > \hspace{\tabcolsep}\=\hspace{\tabcolsep} /n=cjɨ/\kill
  //V\textit{\textsubscript{i}}V\textit{\textsubscript{i} }// \> \>     > \> /V\textit{\textsubscript{i}}V\textit{\textsubscript{i} }=cjɨ/ or /V\textit{\textsubscript{i} }=ccjɨ/ (or /V\textit{\textsubscript{i} }=cjɨ/)
  \end{tabbing}
  \ex \begin{tabbing}
  Elsewhere \hspace{\tabcolsep}\=\hspace{\tabcolsep} +  \textit{ccjɨ} (\textsc{qt}) \hspace{\tabcolsep}\=\hspace{\tabcolsep} > \hspace{\tabcolsep}\=\hspace{\tabcolsep} /n=cjɨ/\kill
  Elsewhere    \> \>  > \>  /V=ccjɨ/ (or /V=cjɨ/)
  \end{tabbing}
\z
\z

The deletion of //c// in (10-59 a-c) and the vowel deletion in (10-59 c) conform to the phonological rule in \sectref{sec:2.4.4} and \sectref{sec:2.4.5} respectively. However, the deletion of //c// in (10-59 d) (and /V\textit{\textsubscript{i} }=cjɨ/ in (10-59 c)) is not explicable by these rules.

I will present a few examples below.

\ea\label{ex:10.60}   Examples\\
  \ea //n// + \textit{ccjɨ} (\textsc{qt})\\
  \begin{tabbing}
  \textit{wur-tar-n} \hspace{\tabcolsep}\=\hspace{\tabcolsep} (exist-\textsc{pst}-\textsc{ptcp}) \hspace{\tabcolsep}\=\hspace{\tabcolsep} +  \textit{ccjɨ}  (\textsc{qt}) \hspace{\tabcolsep}\=\hspace{\tabcolsep} > \hspace{\tabcolsep}\=\hspace{\tabcolsep} /wu-ta-n=cjɨ/\kill
  \textit{wur-tar-n} \> (exist-\textsc{pst}-\textsc{ptcp}) \> +  \textit{ccjɨ}  (\textsc{qt}) \> > \> /wu-ta-n=cjɨ/\\
  \textit{gaccɨn} \> ‘saurel’  \> \>      > \> /gaccɨn=cjɨ/
  \end{tabbing}

  \ex //V\textit{\textsubscript{i}}V\textit{\textsubscript{j} }// + \textit{ccjɨ} (\textsc{qt})\\
  \begin{tabbing}
  \textit{wur-tar-n} \hspace{\tabcolsep}\=\hspace{\tabcolsep} (exist-\textsc{pst}-\textsc{ptcp}) \hspace{\tabcolsep}\=\hspace{\tabcolsep} +  \textit{ccjɨ}  (\textsc{qt}) \hspace{\tabcolsep}\=\hspace{\tabcolsep} > \hspace{\tabcolsep}\=\hspace{\tabcolsep} /wu-ta-n=cjɨ/\kill
  \textit{kai} \> (\textsc{du}B) \> +  \textit{ccjɨ}  (\textsc{qt}) \> > \> /kai=cjɨ/
  \end{tabbing}

  \ex //V\textit{\textsubscript{i}}V\textit{\textsubscript{j} }// + \textit{ccjɨ} (\textsc{qt})\\
  \begin{tabbing}
  \textit{wur-tar-n} \hspace{\tabcolsep}\=\hspace{\tabcolsep} (exist-\textsc{pst}-\textsc{ptcp}) \hspace{\tabcolsep}\=\hspace{\tabcolsep} +  \textit{ccjɨ}  (\textsc{qt}) \hspace{\tabcolsep}\=\hspace{\tabcolsep} > \hspace{\tabcolsep}\=\hspace{\tabcolsep} /wu-ta-n=cjɨ/\kill
  \textit{nuu}  \> ‘what’ \> +  \textit{ccjɨ}  (\textsc{qt}) \> > \> /nuu=cjɨ/\\
  \textit{jaa} \> (\textsc{sol}) \>   \>    > \> /jaa=cjɨ/ or /ja=ccjɨ/\\
  \textit{{}-oo} \> (\textsc{int})   \>  \>   > \> /oo=cjɨ/ or /o=ccjɨ/\\
  \textit{daroo} \> (\textsc{supp})   \>  \>   > \> /daroo=cjɨ/, /daro=ccjɨ/ or /daro=cjɨ/
  \end{tabbing}

  \ex Elsewhere\\
  \begin{tabbing}
  \textit{wur-tar-n} \hspace{\tabcolsep}\=\hspace{\tabcolsep} (exist-\textsc{pst}-\textsc{ptcp}) \hspace{\tabcolsep}\=\hspace{\tabcolsep} +  \textit{ccjɨ}  (\textsc{qt}) \hspace{\tabcolsep}\=\hspace{\tabcolsep} > \hspace{\tabcolsep}\=\hspace{\tabcolsep} /wu-ta-n=cjɨ/\kill
  \textit{{}-sa} \> (\textsc{adj})  \> +  \textit{ccjɨ}  (\textsc{qt})  \> > \> /-sa=ccjɨ/\\
  \textit{itoko} \> ‘cousin’    \> \>   > \> /itoko=cjɨ/
  \end{tabbing}
  \z
\z

Syntactically, \textit{ccjɨ} (\textsc{qt}) is used in the following environments.

\ea\label{ex:10.61}\textit{ccjɨ} (\textsc{qt}) is used,
  \ea  To form the complement of \textit{jˀ-} ‘say’;
  \ex  To form the complement of the other language-oriented verbs;
  \ex  To form the complement of \textit{sɨr-} ‘do’;
  \ex  To form a conditional adverbial clause;
  \ex  To form a clause that has a few nominal properties;
  \ex  To embed an onomatopoeia;
  \ex  Without the superordinate clause.
  \z
\z

In the following subsections, I will show examples of (10-61 a-g) in turn.

\subsubsection{To form the complement of \textit{jˀ-} ‘say’}\label{sec:10.4.1.1}

\textit{ccjɨ} (\textsc{qt}) can embed any kind of utterance into the complement of \textit{jˀ-} ‘say.’ The reported clause (i.e. the complement clause of \textit{jˀ-} ‘say’) can be formally distinguished into two types: direct speech and indirect speech (cf. \citealt{Aikhenvald2004}).

First, in the direct speech, the predicates in the complement clause can take any kind of inflection or clause-final particle as in (10-62 a-f).

\ea\label{ex:10.62}   Direct speech\\
  \begin{xlist}
  \exi{} After verbal predicate phrases

  \ex\relax [= (8-148 g)]\\
    %\textsc{tm}:
      \glll    kanɨcɨboja  urakja  tuikurawɨcjɨ  jˀicjɨ,\\
      \textit{kanɨ+cɨbo=ja}  \textit{urakja}  [\textit{tur-i+kuraw-ɨ}]\textsubscript{verbal predicate phrase}\textit{=\Highlight{ccjɨ}}  \textit{\Highlight{jˀ}-tɨ}\\
      gold+pot=\textsc{top}  2.N\textsc{hon}.\textsc{pl}  take-\textsc{inf}+\textsc{drg}-\textsc{imp}=\textsc{qt}  say-\textsc{seq}\\
      \glt       ‘(The man) said that, “You take (this) damn gold pot!” and ...’ [Fo: 090307\_00.txt]

  \ex %\textsc{tm}:
      \glll    cɨbonu  atanban,  mukkontɨdoocjɨ     jˀicjatto,\\
      \textit{cɨbo=nu}  \textit{ar-tar-n=ban}  [\textit{mukk-on-tɨ}]\textsubscript{verbal predicate phrase}\textit{=doo=\Highlight{ccjɨ}}   \textit{\Highlight{jˀ}-tar-too}\\
      pot=\textsc{nom}  exist-\textsc{pst}-\textsc{ptcp}=\textsc{advrs}  bring-\textsc{neg}-\textsc{seq}=\textsc{ass}=\textsc{qt}   say-P\textsc{st}-\textsc{csl}\\
      \glt       ‘(The husband) said, “There was a pot (filled with gold), but (I) didn’t bring (it).” And then ...’ [Fo: 090307\_00.txt]

  \exi{} After adjectival predicate phrases

  \ex  %\textsc{tm}:
      \glll    simakutuba  narəəcjasacjɨ  jˀicjɨ,\\
      \textit{sima+kububa}  [\textit{naraw-i+cja-sa}]\textsubscript{adjectival predicate phrase}\textit{=\Highlight{ccjɨ}}  \textit{\Highlight{jˀ}-tɨ}\\
      community+language  learn-\textsc{inf}+want-\textsc{adj}=\textsc{qt}  say-\textsc{seq}\\
      \glt       ‘(The present author) said, “(I) want to learn the language of the (Yuwan) community.” And then ...’ [Co: 110328\_00.txt]

  \ex  %\textsc{tm}:
      \glll    mˀasa  akkamodoojaacjɨ  jˀicjɨ,\\
      [\textit{mˀa-sa}  \textit{ar}]\textsubscript{adjectival predicate phrase}\textit{=kamo=doo=jaa=\Highlight{ccjɨ}}  \textit{\Highlight{jˀ}-tɨ}\\
      tasty-\textsc{adj}  \textsc{st}V=\textsc{pos}=\textsc{ass}=\textsc{sol}=\textsc{qt}  say-\textsc{seq}\\
      \glt       ‘(My daughter) said, “(The orange) may be tasty.” And then ...’ [Co: 101023\_01.txt]

  \exi{} After nominal predicate phrases

  \ex  %\textsc{tm}:
      \glll    daanu  Xcjɨ  jˀicjattu,\\
      \textit{daa=nu}  X\textit{=ccjɨ}  \textit{jˀ-tar-tu}\\
      where=\textsc{gen}  X=\textsc{qt}  say-\textsc{pst}-\textsc{csl}\\
      \glt       ‘(I) said, “Who are you?” [lit. “X of where?”] And then ...’ [Co: 120415\_00.txt]

  \ex %\textsc{tm}:
      \glll    uraa  {\textbar}boosi{\textbar}dooccjɨ  jˀicjɨ,\\
      [\textit{ura-a}  \textit{boosi}]\textsubscript{nominal predicate phrase}\textit{=doo=\Highlight{ccjɨ}}  \textit{\Highlight{jˀ}-tɨ}\\
      2.N\textsc{hon}.\textsc{sg}-\textsc{adn}Z  hat=\textsc{ass}=\textsc{qt}  say-\textsc{seq}\\
      \glt       ‘(The boy) said, “(This is) your hat.” And then ...’ [\textsc{pf}: 090827\_02.txt]
  \end{xlist}
\z

In (10-62 a-f), \textit{ccjɨ} (\textsc{qt}) follows all types of the predicate phrases, where there is no restriction on the kinds of inflection or clause-final particles.

On the other hand, the complement clause in the indirect speech cannot take the infection or clause-final particles freely. In this case, only the participle is allowed as the verbal form in the predicate as in (10-63 a-c).

\ea\label{ex:10.63}   Indirect speech\\
  \begin{xlist}
  \exi{} After verbal predicate phrase
    \ex  %\textsc{tm}:
      \glll    an  cˀjo  xxx  (arəə  an ..)   arɨnu ..  {\textbar}menkjo{\textbar}  muccjuncjɨ  jˀicjɨ,\\
      \textit{a-n}  \textit{cˀju=ja}    \textit{a-rɨ=ja}  \textit{a-n}   \textit{a-rɨ=nu}  \textit{menkjo}  [\textit{mut-tur-\Highlight{n}}]\textsubscript{verbal predicate phrase}\textit{=\Highlight{ccjɨ}}  \textit{\Highlight{jˀ}-tɨ}\\
      \textsc{dist}-\textsc{adn}Z  person=\textsc{top}    DI\textsc{st}-\textsc{nlz}=TOP  DIST-\textsc{adnz}   DIST-NLZ=\textsc{gen}  license  have-\textsc{prog}-\textsc{ptcp}=\textsc{qt}  say-\textsc{seq}\\
      \glt       ‘That person said that (he) had [lit. is having] the license of that [i.e. refereeing sumo wrestling], and ...’ [Co: 120415\_00.txt]

  \exi{} After adjectival predicate phrase

  \ex\relax  [Context: \textsc{tm} told US that the present author had wanted to see US.]
    %\textsc{tm}:
      \glll    nanga  hanacjɨ  moojun  mun  kikicjasancjɨ  jˀicjɨ,\\
      \textit{nan=ga}  \textit{hanas-tɨ}  \textit{moor-jur-n}  \textit{mun}   [\textit{kik-i+cja-sa+ar-\Highlight{n}}]\textsubscript{adjectival predicate phrase}\textit{=\Highlight{ccjɨ}}  \textit{\Highlight{jˀ}-tɨ}\\
      2.\textsc{hon}.\textsc{sg}=\textsc{nom}  speak-\textsc{seq}  HON-\textsc{umrk}-\textsc{ptcp}  thing  hear-\textsc{inf}+want-\textsc{adj}+\textsc{st}V-PTCP=\textsc{qt}  say-SEQ\\
      \glt       ‘(The present author) said that (he) wanted to hear what you would say, and ...’ [Co: 110328\_00.txt]

  \exi{} After nominal predicate phrase

  \ex  %\textsc{tm}:
      \glll    isaburootaa,  tomokkotaaga  atai  jatancjɨ  jˀicjɨ,\\
      \textit{isaburoo-taa}  \textit{tomokko-taa=ga}  [\textit{atai}  \textit{jar-tar-\Highlight{n}}]\textsubscript{nominal predicate phrase}\textit{=\Highlight{ccjɨ}} \textit{\Highlight{jˀ}-tɨ}\\
      Isaburo-\textsc{pl}  Tomohiko-PL=\textsc{nom}  50.years.old  \textsc{cop}-\textsc{pst}-\textsc{ptcp}=\textsc{qt}   say-\textsc{seq}\\
      \glt       ‘(People) said that Isaburo (and) Tomohiko were fifty years old, and ...’ [Co: 120415\_01.txt]
    \end{xlist}
\z

In principle, the participle cannot finish a sentence (with the exception of the focus construction discussed in \sectref{sec:11.3}). Thus, the participle in the complement clause of indirect speech cannot be the one that was uttered in the real conversation. Thus, we can formally distinguish the direct speech from the indirect speech. It should be noted that the modality that could be expressed in the direct speech by the verbal inflection or the clause-final particles are unable to be expressed in the indirect speech, since only the participle is allowed for the indirect speech.

  Furthermore, the difference between the direct speech and the indirect speech can also be distinguished semantically by the deictic center of the pronouns. In the direct speech, the deictic center of the pronoun is the person who gave the utterance (not the speaker who reported the utterance). For example, the deictic center of \textit{ura} ‘you’ in (10-62 f) is the character in the Pear Film (not the speaker \textsc{tm}). On the contrary, in the indirect speech, the deictic center of the pronoun is the speaker who reported the utterance (not the person who gave the utterance). For example, the deictic center of \textit{nan} ‘you (honorific)’ in (10-63 b) is the speaker TM (not the original speaker, i.e. the present author).

The difference between the direct speech and the indirect speech can be formally expressed by the verbal form in the predicate, i.e., whether it is the participle or not. However, the difference cannot be expressed formally in the nominal predicate if it is in the non-past tense and also in the affirmative pole, since the copula does not take the participial form in the non-past tense and the affirmative pole, i.e. *\textit{jar-n} (\textsc{cop}-\textsc{ptcp}) is not available; see (9-67 b) in \sectref{sec:9.4.1} with an exception of \textit{jar-n=mun} (COP-PTCP=\textsc{advrs}) in (8-46 a) in \sectref{sec:8.3.3.5.} Thus, in the non-past tense and the affirmative pole, the nominal predicate in the indirect speech as in \textsc{ref}{ex:10.64} has the same form with that in the direct speech as in (10-62 e).

\ea\label{ex:10.64}   Indirect speech\\
  After nominal predicate phrase (non-past and affirmative pole)\\
  %\textsc{tm}:
      \glll    usato{\textbar}obasan{\textbar}  xxx  nusinujoo  jinganənkjatu  kun  ziisantuga  {\textbar}itoko{\textbar}cjɨ   jˀuta.\\
    \textit{usato+obasan}    \textit{nusi=nu=joo}  \textit{jinga-nəə=nkja=tu}  \textit{ku-n}  \textit{ziisan=tu=ga}  [\textit{itoko}]\textsubscript{nominal predicate phrase}\textit{=\Highlight{ccjɨ}} \textit{\Highlight{jˀ}-jur-tar}\\
    Usato+old.lady    \textsc{rfl}=\textsc{gen}=\textsc{cfm}1  man-parent=\textsc{appr}=\textsc{com} this-\textsc{adn}Z  grandfather=COM=\textsc{nom}  cousin=\textsc{qt}   say-\textsc{umrk}-\textsc{pst}\\
    \glt ‘Usato said that her [lit. herself’s] father is cousin to this (person’s) grandfather.’ [Co: 110328\_00.txt]
\z

In \textsc{ref}{ex:10.64}, the nominal predicate \textit{itoko} ‘cousin’ does not take the copula participle *\textit{jar-n} (\textsc{cop}-\textsc{ptcp}). Formally, the feature of the indirect speech is not expressed, but semantically, it is expressed by the demonstrative \textit{ku-n} ‘this (one),’ whose deictic center is the speaker \textsc{tm} (not the original speaker Usato). Similar formal ambiguity occurs when the predicate in the complement ends with the negative participial affix \textit{{}-an}, since it can also finish a clause in the non-reported utterance (see \sectref{sec:8.4.2.2}).

  In fact, there is a case where there is a mixture of the strategy of the direct speech and the indirect speech as in \textsc{ref}{ex:10.65}, where the adjectival predicate before \textit{ccjɨ} (\textsc{qt}) does not take the participle \textit{ar-n} (\textsc{st}V-\textsc{ptcp}), but the deictic center of the complement clause is the speaker \textsc{tm} (not the original speaker, i.e. the present author).

\ea\label{ex:10.65}   Mixture of the strategy of the direct speech and the indirect speech\\
  After adjectival predicate phrase\\\relax
  [Context: \textsc{tm} said to US that the present author had wanted to see US for a long time.]\\
  %\textsc{tm}:
      \glll    naa  məəci  ikicjasaccjɨ  jukkadɨ  umoojutanmun,  {\textbar}mae{\textbar}gajo  {\textbar}mae{\textbar}ga  umoojutanmun,  kinju  atadan.\\                                                                                                                                                     
    \textit{\Highlight{naa}}  \textit{məə=kaci}  \textit{ik-i+cja-\Highlight{sa=ccjɨ}}  \textit{jukkadɨ}  \textit{umoor-jur-tar-n=mun}  \textit{mae=ga=joo}  \textit{mae=ga} \textit{umoor-jur-tar-n=mun}  \textit{kinju}  \textit{atadan}\\                                                                                                                                                     
    2.\textsc{hon}.\textsc{sg}.\textsc{adn}Z  place=\textsc{all}  go-\textsc{inf}+want-\textsc{adj}=\textsc{qt}  always   say.HON-\textsc{umrk}-\textsc{pst}=\textsc{advrs}  before=\textsc{foc}=\textsc{cfm}1  before=FOC   say.HON-UMRK-P\textsc{st}=ADVRS  yesterday  suddenly\\
    \glt ‘(The present author) always used to say that (he) wants to go to your place before, but yesterday (he) suddenly (visited me).’ [Co: 110328\_00.txt]
\z

In \textsc{ref}{ex:10.65}, the predicate preceding \textit{ccjɨ} (\textsc{qt}) does not take the participle \textit{ar-n} (\textsc{st}V-\textsc{ptcp}). However, the deictic center of the pronominl \textit{naa} (2.\textsc{hon}.\textsc{sg}.\textsc{adn}Z) ‘your’ is the speaker \textsc{tm} (not the original speaker, since there was not US when the present author had spoke to TM about US). That is, the pronominal deixis expresses an indirect speech, but the verbal form in the complement slot expresses a direct speech in \REF{ex:10.65}.

  Furthermore, there are cases where \textit{ccjɨ} (\textsc{qt}) does not follow any predicate phrase as in (10-66 a-b).

\ea\label{ex:10.66}   After non-predicative NPs\\
  \ea  US: \glll kunəəda,  ude,  wattəə  hanasija  sjanbanga,   naa,  urɨcjɨ  jˀicjutɨ,\\
      \textit{kunəəda}  \textit{ude}  \textit{wattəə}  \textit{hanas-i=ja}  \textit{sɨr-tar-n=ban=ga}  \textit{naa}  [\textit{u-rɨ}]\textsubscript{NP}\textit{=\Highlight{ccjɨ}}  \textit{\Highlight{jˀ}-tur-tɨ}\\
      the.other.day  well  1\textsc{du}  talk-\textsc{inf}=\textsc{top}  do-\textsc{pst}-\textsc{ptcp}=\textsc{advrs}=\textsc{foc}  \textsc{fil}  \textsc{mes}-\textsc{nlz}=\textsc{qt}  say-\textsc{prog}-\textsc{seq}\\
      \glt       ‘We [i.e. US and the present author] talked the other day, but (I) have said, “That” [i.e. US can’t teach Yuwan for the present author]. And then ...’ [Co: 110328\_00.txt]

  \ex  \textsc{tm}:     \glll    waakjaga  {\textbar}gakkoo{\textbar}  sjuinnjajo  {\textbar}sjeesikoozjoo{\textbar}cjɨ    jˀicjɨ,  {\textbar}koozjoo{\textbar}gadɨ  tatɨtattujaa.\\
      \textit{waakja=ga}  \textit{gakkoo}  \textit{sɨr-tur-i=n=ja=joo}  [\textit{sjeesikoozjoo}]\textsubscript{NP}\textit{=\Highlight{ccjɨ}}  \textit{\Highlight{jˀ}-tɨ  koozjoo=gadɨ  tatɨr-tar-tu=jaa}\\
      1\textsc{pl}=\textsc{nom}  school  do{}-\textsc{prog}-\textsc{inf}=\textsc{dat}1=\textsc{top}=\textsc{cfm}1  silk.mill=\textsc{qt} say-\textsc{seq}  mill=\textsc{lmt}  build-\textsc{pst}-\textsc{csl}=\textsc{sol}\\
      \glt       ‘When we do [i.e. went to] school, (there was a building called) the silk mill, and (some people were so rich as to) build a (silk) mill.’ [Co: 111113\_01.txt]
    \z
\z

In (10-66 a), \textit{ccjɨ} (\textsc{qt}) follows the NP \textit{u-rɨ} ‘that,’ which is diffcult to reconstruct the original clause structure where the NP would be set. Similarly, the NP followed by \textit{ccjɨ} (QT) in (10-66 b), i.e. \textit{sjeesikoozjoo} ‘silk mill,’ is diffcult to reconstruct the original clause structure where it would be set. In fact, the structure “NP\textit{=ccjɨ} \textit{jˀ-tɨ} (NP=QT say-\textsc{seq})” is frequently used to express the meaning such as ‘there is something (or someone) called NP,’ which is used to introduce a referent that is thought (by the speaker) to be unfamiliar to the hearer.

  Before concluding this section, I want to mention that there are cases where the contraction between the preceding \textit{ccjɨ} (\textsc{qt}) and the following \textit{jˀ-} ‘say’ occurs as in (10-67 a-b). Strictly speaking, the following \textit{jˀ-} ‘say’ always takes the converbal affix \textit{{}-ba} (\textsc{csl}) in the contraction: \textit{ccjɨ} (QT) + \textit{jˀ-ba} (say-CSL) > /(c)cjuuba/.

\ea\label{ex:10.67}   Contraction of \textit{ccjɨ} (\textsc{qt}) and \textit{jˀ-ba} (say-\textsc{csl})\\
  \ea  %\textsc{tm}:
      \glll    naa  {\textbar}nisanci{\textbar}  sjəəroo,  mudui\Highlight{cjuuba}.\\
          \textit{naa}  \textit{nisanci}  \textit{sɨr-təəra=ja}  \textit{mudur-i=\Highlight{ccjɨ+jˀ-ba}}\\
      \textsc{fil}  two.or.three.days  do-after=\textsc{top}  return-\textsc{inf}=\textsc{qt}+say-\textsc{csl}\\
      \glt       ‘(The present author) said that (he) would return (to Tokyo) in two or three days, so (I am glad I was able to have you see him).’ [Co: 110328\_00.txt]

  \ex  %\textsc{tm}:
      \glll    {\textbar}sanzikkiro{\textbar}\Highlight{ccjuuba}  {\textbar}nangin{\textbar}?\\
      \textit{sanzikkiro=\Highlight{ccjɨ+jˀ-ba}}  \textit{nangin}\\
      thirty.kilogram=\textsc{qt}+say-\textsc{csl}  what.kin\\
    \glt       ‘How many \textit{kin} [i.e. a kind of measure of weight] is thirty kilograms?’ [lit. ‘Speaking of thirty kilograms, how many \textit{kin} (is it)?’] [Co: 111113\_02.txt]
    \z
\z

In (10-67 a), \textit{{}-ba} (\textsc{csl}) retains its causal meaning, but in (10-67 b), it lost the causal meaning, and the contracted expression /(c)cjuuba/ means ‘speaking of’ as a whole. Interestingly, there are examples, where the affix \textit{{}-ba} (CSL) seems to directly attach to the preceding \textit{ccjɨ} (\textsc{qt}), where the expression /(c)cjɨba/ means also ‘speaking of’ as in (10-68 a). Furthermore, there is an expression where \textit{{}-boo} (\textsc{cnd}) seems to directly attach to \textit{ccjɨ} (QT) and the expression /(c)cjɨboo/ also means ‘speaking of’ as in (10-68 b).

\ea\label{ex:10.68}
\ea \textit{ccjɨba} ‘speaking of’\\
  %\textsc{tm}:
      \glll    {\textbar}wasjeunsjuu{\textbar}ccjɨba  nama{\textbar}goro{\textbar}  huntoo  mukasitoo  cigəəbajaa.\\
    \textit{wasjeunsjuu=\Highlight{ccjɨba}}  \textit{nama-goro}  \textit{huntoo}  \textit{mukasi=tu=ja}  \textit{cigjaw-ba=jaa}\\
    k.o.orange=speaking.of  now-around  really  past=\textsc{com}=\textsc{top}  different-\textsc{csl}=\textsc{sol}\\
\glt     ‘Speaking of \textit{wasjeunsjuu}, (those growing up) these days are really different from (those) in the past, so (I feel the time has passed away).’  [Co: 101023\_01.txt]

  \ex \textit{ccjɨboo} ‘speaking of’\\
  %\textsc{tm}:
      \glll    buncjɨboo  {\textbar}tada{\textbar}  jaanɨntəkkwa  urɨ  janmun.\\
    \textit{bun=\Highlight{ccjɨboo}}  \textit{tada}  \textit{jaa+nintəə-kkwa}  \textit{u-rɨ}  \textit{jar-n=mun}\\
    bon.festival=speaking.of  only  house+people{}-\textsc{dim}  \textsc{mes}-\textsc{nlz}  \textsc{cop}-\textsc{ptcp}=\textsc{advrs}\\
    \glt     ‘Speaking of the bon festival, only the family is that [i.e. only the family member gathered].’  [Co: 111113\_01.txt]
    \z
\z

In modern Yuwan, each of these expressions is analyzed as a single morpheme such as \textit{ccjɨba} ‘speaking of’ and \textit{ccjɨboo} ‘speaking of.’

\subsubsection{To form the complement of the other language-oriented verbs}\label{sec:10.4.1.2}

The particle \textit{ccjɨ} (\textsc{qt}) can also embed any kind of utterance into the complement of language-oriented verbs other than \textit{jˀ-} ‘say,’ e.g., \textit{umuw-} ‘think’ or \textit{kak-} ‘write.’ The difference between the direct speech and the indirect speech discussed in \sectref{sec:10.4.1.1} also applies to these language-oriented verbs. I will present examples of \textit{umuw-} ‘think’ below.

\ea\label{ex:10.69}   To form the complement of \textit{umuw-} ‘think’\\
   After verbal predicate phrase\\
  \ea\relax  [= (10-52 b)]\\
    %\textsc{tm}:
      \glll    gan  sjɨ  natɨ,  (naa)  naa  mudutɨdaroccjɨ umututanwakejo.\\
      \textit{ga-n}  \textit{sɨr-tɨ}  \textit{nar-tɨ}  \textit{naa}  \textit{naa}  \textit{mudur-tɨ=\Highlight{daroo=ccjɨ}} \textit{\Highlight{umuw}-tur-tar-n=wake=joo}\\
      \textsc{mes}-\textsc{advz}  do-\textsc{seq}  \textsc{cop}-SEQ  already  already  return-SEQ=\textsc{supp}=\textsc{qt}  think-\textsc{prog}-\textsc{pst}-\textsc{ptcp}=\textsc{cfp}=\textsc{cfm}1\\
      \glt ‘Then [lit. Since (it) does like that], (I)’ve been thinking that (the present author) had probably already returned (to Tokyo).’  [Co: 110328\_00.txt]

  \ex\relax  [= (8-41)]\\
    %\textsc{tm}:
      \glll    wanga  kicjuncjɨ  umutɨdu,  urattəə  gan  sjan  aran  hanasi  sjaroogai?\\
      \textit{wan=ga}  \textit{kik-tur-\Highlight{n=ccjɨ}}  \textit{\Highlight{umuw}-tɨ=du  urattəə}   \textit{ga-n}  \textit{sɨr-tar-n}  \textit{ar-an}  \textit{hanasi}  \textit{sɨr-tar-oo=ga=i}\\
      1\textsc{sg}=\textsc{nom}  hear-\textsc{prog}-\textsc{ptcp}=\textsc{qt}  think-\textsc{seq}=\textsc{foc}  2.N\textsc{hon}.\textsc{du}  \textsc{mes}-\textsc{adn}Z  do-\textsc{pst}-PTCP  \textsc{cop}-\textsc{neg}  tale  do-P\textsc{st}-\textsc{supp}=\textsc{cfm}3=\textsc{plq}\\
      \glt       ‘Probably you told the unlikely tale like that since (you) thought that I was listening to (that), didn’t you?’ [Fo: 090307\_00.txt]

  \ex\relax [= (8-141 b)]\\
    %\textsc{tm}:
      \glll    unin{\textbar}goro{\textbar}kara  naacɨbaacjɨ  umuwannən,  jəito  hamɨcɨkɨtɨ narəəboo,  (mmm)  zjoozɨ  najutənmundoojaa.\\
      \textit{unin-goro=kara}  \textit{naacɨbaa=\Highlight{ccjɨ}}  \textit{\Highlight{umuw}-an-nən  jəito  hamɨcɨkɨr-tɨ}   \textit{naraw-boo}    \textit{zjoozɨ}  \textit{nar-jur-təər-n=mun=doo=jaa}\\
      that.time-around=\textsc{abl}  tone.deaf=\textsc{qt}  think-\textsc{neg}-\textsc{seq}  well  do.one’s.best-SEQ  learn-\textsc{cnd}    good.at  become-\textsc{umrk}-\textsc{rsl}-\textsc{ptcp}=\textsc{advrs}=\textsc{ass}=\textsc{sol}\\
      \glt       ‘If (I) didn’t think that (I was) tone-deaf and did my best to learn (the traditional songs) since those days, (I) would have been good at (them), but (I didn’t do that).’ [Co: 111113\_01.txt]
    \z
\z

In (10-69 a), \textit{ccjɨ} (\textsc{qt}) follows the clause-final particle \textit{daroo} (\textsc{supp}). That means the complement clause is reported in the direct-speech style. In (10-69 b), \textit{ccjɨ} (QT) follows the participle /kicjun/ \textit{kik-tur-n} (hear-\textsc{prog}-\textsc{ptcp}), which means the complement clause is reported in the indirect-speech style. In (10-69 c), \textit{ccjɨ} (QT) follows the nominal predicate phrase \textit{naacɨbaa} ‘a tone-deaf person,’ where we cannot formally distinguish the speech style, since the nominal predicate cannot take participle in the non-past tense and also in affirmative as discussed in\sectref{sec:10.4.1.1}.

\subsubsection{To form the complement of \textit{sɨr-} ‘do’}\label{sec:10.4.1.3}

\textit{ccjɨ} (\textsc{qt}) can embed the verb that ends with \textit{{}-oo} (\textsc{int}) into the complement of \textit{sɨr-} ‘do.’

\ea\label{ex:10.70}   To form the complement of \textit{sɨr-} ‘do’ [= (9-26)]\\
  %\textsc{tm}:
      \glll    ikjoccjɨ  sjun  turooja  aran?\\
    \textit{ik-\Highlight{oo=ccjɨ}}  \textit{\Highlight{sɨr}-tur-n  turoo=ja  ar-an}\\
    go-\textsc{int}=\textsc{qt}  do-\textsc{prog}-\textsc{ptcp}  scene=\textsc{top}  \textsc{cop}-\textsc{neg}\\
\glt     ‘(It is) a scene where (they) were about to go (somewhere), isn’t (it)?’  [Co: 120415\_00.txt]
\z

As mentioned in (9-23 c) in \sectref{sec:9.1.2.1}, the combination of \textit{{}-oo=ccjɨ sɨr-} (\textsc{int}=\textsc{qt} do) means ‘be about to.’

\subsubsection{To form a conditional adverbial clause}\label{sec:10.4.1.4}

\textit{ccjɨ} (\textsc{qt}) can make a conditional adverbial clause in the following combination: \textit{{}-tar-n=ccjɨ=n} (\textsc{pst}-\textsc{ptcp}=QT=even) ‘even if (someone) did ...’ This expression may have some relation with \textit{{}-tɨ=n} (\textsc{seq}=even) ‘even if’ in \sectref{sec:10.1.3}.

\ea\label{ex:10.71}   \textit{{}-tar-n=ccjɨ=n} (\textsc{pst}-\textsc{ptcp}=\textsc{qt}=even) ‘even if’\\
  \ea  %\textsc{tm}:
      \glll    naa,  {\textbar}mokujoobi{\textbar}ninkja  izjancjɨn, ..  siman  cˀjuga  wuranba.\\
      \textit{naa}  \textit{mokujoobi=n=nkja}  \textit{ik-\Highlight{tar-n=ccjɨ=n}}  \textit{sima=nu}  \textit{cˀju=ga}  \textit{wur-an-ba}\\
      \textsc{fil}  Thursday=\textsc{dat}1=\textsc{appr}  go-\textsc{pst}-\textsc{ptcp}=\textsc{qt}=even  community=\textsc{gen} person=\textsc{nom}  exist-\textsc{neg}-\textsc{csl}\\
      \glt       ‘Even if (I) went (to the day-care center), there are no people (from the same) community, so (I don’t speak in Yuwan there).’ [Co: 120415\_01.txt]

  \ex  %\textsc{tm}:
      \glll    naa,  gan  sjɨ  natəəroo,  {\textbar}nansai{\textbar}gadɨ  wutancjɨn,\\
      \textit{naa}  \textit{ga-n}  \textit{sɨr-tɨ}  \textit{nar-təəra=ja}  \textit{nansai=gadɨ}  \textit{wur-\Highlight{tar-n=ccjɨ=n}}\\
      \textsc{fil}  \textsc{mes}-\textsc{advz}  do-\textsc{seq}  become-after=\textsc{top}  how.old=\textsc{lmt}   exist-\textsc{pst}-\textsc{ptcp}=\textsc{qt}=even\\
      \glt       ‘After becoming like that [i.e. bedridden], even if (the person) lived very long, ...’ [Co: 120415\_01.txt]
    \z
\z

\subsubsection{To form a clause that has a few nominal properties}\label{sec:10.4.1.5}

The clause followed by \textit{ccjɨ} (\textsc{qt}) slightly behaves like the nominal since it can take the genitive case as in (10-72 a), or it can precede the copula verb as in (10-72 b).

\ea\label{ex:10.72} 
\ea \textit{ccjɨ} (\textsc{qt}) followed by \textit{nu} (\textsc{gen})\\\relax
  [Context: \textsc{tm} asked her daughter to bring the lunch at noon.]\\
  %\textsc{tm}:
      \glll    nama  {\textbar}zjuunizi{\textbar}  narancjɨnu  kutukai?\\
    \textit{nama}  \textit{zjuunizi}  \textit{nar-an=\Highlight{ccjɨ=nu}}  \textit{kutu=kai}\\
    yet  noon  become-\textsc{neg}=\textsc{qt}=\textsc{gen}  thing=\textsc{du}B\\
\glt     ‘Does (she) think that (it) is not noon yet?’  [Co: 120415\_01.txt]

  \ex \textit{ccjɨ} (\textsc{qt}) followed by the copula verb\\
  %\textsc{tm}:
      \glll    {\textbar}itoko{\textbar}cjɨ  jˀicjɨn,  wuran  mun  natɨ,  {\textbar}maa{\textbar}   wurancjəə  aranban,  tusinu  {\textbar}sa{\textbar}ga  nənkara,\\
    \textit{itoko=ccjɨ}  \textit{jˀ-tɨ=n}  \textit{wur-an}  \textit{mun}  \textit{nar-tɨ}  \textit{maa} \textit{wur-an=\Highlight{ccjɨ}=ja}  \textit{\Highlight{ar}-an=ban  tusi=nu  sa=ga  nə-an=kara}\\
    cousin=\textsc{qt}  say-\textsc{seq}=even  exist-\textsc{neg}  thing  become-SEQ  \textsc{fil} exist-NEG=QT=\textsc{top}  \textsc{cop}-NEG=\textsc{advrs}  age=\textsc{gen}  difference=\textsc{nom}  exist-NEG=\textsc{csl}\\
\glt     ‘Even if (they are) cousin (to me), (they) are not (in this community), well, (it) is too much (to say) that (they) are not (in this community), but there is (almost) no difference in age (between us), so ...’  [Co: 120415\_01.txt]
\z
\z

\subsubsection{To embed an onomatopoeia}\label{sec:10.4.1.6}

\textit{ccjɨ} (\textsc{qt}) can embed an onomatopoeia into the complement slot of the superordinate clause as in \textsc{ref}{ex:10.73}.

\ea\label{ex:10.73}   \textit{ccjɨ} (\textsc{qt}) to embed an onomatopoeia\\
  %\textsc{tm}:
      \glll    tuisuzjɨ  izjan  micjaija  isjoobiki  hucjɨ,  hjuucjɨ  abɨjuroogai?\\
    \textit{tuur-i+sug-tɨ}  \textit{ik-tar-n}  \textit{micjai=ja}  \textit{isjoobiki}  \textit{huk-tɨ}  \textit{\Highlight{hjuu=ccjɨ}}  \textit{abɨr-jur-oo=ga=i}\\
    pass-\textsc{inf}+pass-\textsc{seq}  go-\textsc{pst}-\textsc{ptcp}  three.person.\textsc{clf}=\textsc{top}  whistle  blow-SEQ  [sound effect]=\textsc{qt}  call-\textsc{umrk}-\textsc{supp}=\textsc{cfm}3=\textsc{plq}\\
    \glt     ‘The three (boys) who passed by whistled and called (another boy with a whistling sound like) “phweee.”’ [\textsc{pf}: 090827\_02.txt]
    \z

\subsubsection{Without the superordinate clause}\label{sec:10.4.1.7}

The clause followed by \textit{ccjɨ} (\textsc{qt}) can be used without the superordinate clause (at least in the phonetic level) as in (10-74 a-b).

\ea\label{ex:10.74}   \textit{ccjɨ} (\textsc{qt}) without the superordinate clause\\
  \ea  %\textsc{tm}:
      \glll    nama  (umooju)  umoojuncjɨdoo.\\
      \textit{nama}  \textit{umoor-jur}  \textit{umoor-jur-n=\Highlight{ccjɨ}=doo}\\
      still  exist.\textsc{hon}-\textsc{umrk}  exist.HON-UMRK-\textsc{ptcp}=\textsc{qt}=\textsc{ass}\\
      \glt       ‘(Someone said) that (he) is still alive.’ [Co: 120415\_00.txt]

  \ex\relax  [Context: Talking about \textsc{my}] = (6-24 a)\\
    %\textsc{tm}:
      \glll    attaaja  (un)  un  hutəənan   wutancjɨjaa.\\
      \textit{a-rɨ-taa=ja}  \textit{u-n}  \textit{u-n}  \textit{hutəə=nan} \textit{wur-tar-n=\Highlight{ccjɨ}=jaa}\\
      \textsc{dist}-\textsc{nlz}-\textsc{pl}=\textsc{top}  \textsc{mes}-\textsc{adn}Z  MES-\textsc{adnz}  vicinity=\textsc{loc}1 exist-\textsc{pst}-\textsc{ptcp}=\textsc{qt}=\textsc{sol}\\
      \glt       ‘(I heard) that she and her family were around there.’ [Co: 110328\_00.txt]
    \z
\z

In (10-74 a-b), the clauses followed by \textit{ccjɨ} (\textsc{qt}) are not embedded in any superordinate clause (in the phonetic level). In fact, the clause-final particle \textit{doo} (\textsc{ass}) directly follows \textit{ccjɨ} (QT) in (10-74 a). The superordinate clauses in these examples may be inferred from the context, and the heads of the superordinate clauses are thought to be \textit{jˀ-} ‘say,’ which is expressed by ‘(someone said)’ or ‘(I heard)’ in the free translation. It is important to note that \textit{ccjɨ=doo} (QT=ASS) and \textit{ccjɨ=jaa} (QT=\textsc{sol}) express that the speaker’s uncertainty over the information from the hearsay evidence.

  On the other hand, there is a case where the superordinate clause of (the clause followed by) \textit{ccjɨ} (\textsc{qt}) cannot be inferred from the context. I will show the examples below, where \textit{ccjɨ} (QT) is always followed by \textit{joo} (\textsc{cfm}1).

\ea\label{ex:10.75}   \textit{ccjɨ} (\textsc{qt}) followed by \textit{joo} (\textsc{cfm}1)\\
  \ea\relax [Context: The speaker explains the story of the Pear Film to the hearer.]\\
    %\textsc{tm}:
      \glll    tuutɨ  izjancjɨjoo.\\
      \textit{tuur-tɨ}  \textit{ik-tar-n=\Highlight{ccjɨ=joo}}\\
      pass-\textsc{seq}  go-\textsc{pst}-\textsc{ptcp}=\textsc{qt}=\textsc{cfm}1\\
      \glt       ‘(A young man who pulls a goat) passed away.’ [\textsc{pf}: 090305\_01.txt]

  \ex\relax [Context: \textsc{tm} describs US’s behavior to the present author in front of US.]\\
    %\textsc{tm}:
      \glll    {\textbar}ittoki{\textbar}n  joosjurancjɨjo.  kan  sjɨ   sjutɨ,  jukkadɨ  nunkuin  izjasiccjɨjo. hanasinkjoo  sɨrancjɨjo.\\
      \textit{ittoki=n}  \textit{joosjur-an=\Highlight{ccjɨ=joo}}  \textit{ka-n}  \textit{sɨr-tɨ}  \textit{sɨr-tur-tɨ}  \textit{jukkadɨ}  \textit{nuu-nkuin}  \textit{izjas-i=\Highlight{ccjɨ=joo}} \textit{hansi=nkja=ja}  \textit{sɨr-an=\Highlight{ccjɨ=joo}}\\
      for.a.moment=even  keep.still-\textsc{neg}=\textsc{qt}=\textsc{cfm}1  \textsc{prox}-\textsc{advz}  do-\textsc{seq}  do-\textsc{prog}-SEQ  continuously  what-\textsc{indfz}  serve-\textsc{inf}=QT=CFM1   conversation=\textsc{appr}=\textsc{top}  do-NEG=QT=CFM1\\
      \glt ‘(US) cannot keep still. Like this, (US) is continuously serving things. (US) does not do [i.e. enjoy] the conversation.’   [Co: 110328\_00.txt]
    \z
\z

In the above examples, the clauses followed by \textit{ccjɨ=joo} (\textsc{qt}=\textsc{cfm}1) do not report someone’s utterance in the past. Therefore, the head of the superordinate clause, if any, cannot be \textit{jˀ-} ‘say.’ Moreover, the head of the superordinate clasue, if any, cannot be \textit{umuw-} ‘think’ either. For example, the speaker describes the image in the film as soon as she watched it as in (10-75 a), and also describes the behavior of her friend (“US”) in front of her in (10-75 b). In these examples, the events described by the speaker are rather objective, and unlikely to be familiar with a verb that implies the speaker’s subjectivity, i.e. \textit{umuw-} ‘think.’ Thus, the clauses followed by \textit{ccjɨ=joo} (QT=CFM1) in (10-75 a-b) are thought to be independent from any superordinate clause. In other words, they are examples of insubordination (see \sectref{sec:11.2}).

  The difference between \textit{ccjɨ=doo} (\textsc{qt}=\textsc{ass}) marking the hearsay information and \textit{ccjɨ=joo} (QT=\textsc{cfm}) marking the objective (or non-hearsay) information is clarified in the following minimal pairs taken in the elicitaion.

\ea\label{ex:10.76}   \textit{ccjɨ=doo} (\textsc{qt}=\textsc{ass}) vs. \textit{ccjɨ=joo} (QT=\textsc{cfm}1)\\
  \begin{xlist}
  \exi{} First-person subject

  \ex  %\textsc{tm}:
      \glll    wanna  kamancjɨjoo.  \\
      \textit{wan=ja}  \textit{kam-an=\Highlight{ccjɨ=joo}}  \\
      1\textsc{sg}=\textsc{top}  eat-\textsc{neg}=\textsc{qt}=\textsc{cfm}1  \\
      \glt       ‘I won’t eat (it).’ [El: 101023]

  \ex  %\textsc{tm}:
      \glll    \textsuperscript{\#}wanna  kamancjɨdoo.\\
      \textit{wan=ja}  \textit{kam-an=\Highlight{ccjɨ=doo}}\\
      1\textsc{sg}=\textsc{top}  eat-\textsc{neg}=\textsc{qt}=\textsc{ass}\\
      \glt [El: 101023]

  \exi{} Third-person subject

  \ex  %\textsc{tm}:
      \glll     an  cˀjoo  kamancjɨjoo.\\
      \textit{a-n}  \textit{cˀju=ja}  \textit{kam-an=\Highlight{ccjɨ=joo}}\\
      \textsc{dist}-\textsc{adn}Z  person=\textsc{top}  eat-\textsc{neg}=\textsc{qt}=\textsc{cfm}1\\
      \glt        ‘That person does not eat (it).’ [El: 101023]

  \ex  %\textsc{tm}:
      \glll     an  cˀjoo  kamancjɨdoo.\\
      \textit{a-n}  \textit{cˀju=ja}  \textit{kam-an=\Highlight{ccjɨ=doo}}\\
      \textsc{dist}-\textsc{adn}Z  person=\textsc{top}  eat-\textsc{neg}=\textsc{qt}=\textsc{ass}\\
      \glt        ‘(Someone said) that that person does not eat (it).’ [El: 101023]
    \end{xlist}
\z

In (10-76 a, c), the speaker presents the information as objective facts. On the other hand, in (10-76 d), the speaker presents the information on the hearsay evindence. As mentioned before, \textit{ccjɨ=doo} (\textsc{qt}=\textsc{ass}) implies the speaker’s uncertainty over the information. Thus, the example in (10-76 b) cannot be acceptable, since it is unnatural that the speaker herself is unsure of whether she is willing to eat something or not.

\subsection{\textit{ka} (\textsc{du}B)}\label{sec:10.4.2}

\textit{ka} (\textsc{du}B) has two functions as in (10-77 a-b), which also apply to \textit{gajaaroo} (\textsc{dub}) in \sectref{sec:10.4.3}.

\ea\label{ex:10.77} Functions of \textit{ka} (\textsc{du}B)
 \ea Can embed a clause into the complement of \textit{sij-} ‘know’ or \textit{wa(k)ar-} ‘understand; know’;
 \ex Can derive the indefinite NP from the interrogative NP.
 \z
\z

If \textit{ka} (\textsc{du}B) attahces to the clause that includes the interrogative word, which expresses the information question, \textit{ka} (\textsc{dub}) functions as the marker of indirect question as in (10-78 a-b).

\ea\label{ex:10.78}   As a maker of indirect information question (or “Wh-question”)\\
  \ea\relax  [= (5-38 a)]\\
    %\textsc{tm}:
      \glll    wanna  {\textbar}bettarazukee{\textbar}ja  naa  ikjasaa  sjakka  wakarandoo.\\
      \textit{wan=ja}  \textit{bettarazuke=ja}  \textit{naa}  \textit{\Highlight{ikja-saa}}  \textit{sɨr-tar=\Highlight{ka}}  \textit{wakar-an=doo}\\
      1\textsc{sg}=\textsc{top}  k.o.pickle=TOP  \textsc{fil}  how-\textsc{advz}  do-\textsc{pst}=\textsc{du}B  know-\textsc{neg}=\textsc{ass}\\
      \glt       ‘I don’t know how much (I) did [i.e. made] the \textit{bettarazuke} [i.e. k.o. pickles].’ [Co: 101023\_01.txt]

  \ex %\textsc{tm}:
      \glll    nuucjɨ  jˀicjɨ  cˀjakka  wakaranmun.\\
      \textit{\Highlight{nuu}=ccjɨ}  \textit{jˀ-tɨ}  \textit{k-tar=\Highlight{ka}}  \textit{wakar-an=mun}\\
      what=\textsc{qt}  say-\textsc{seq}  come-\textsc{pst}=\textsc{du}B  know-\textsc{neg}=\textsc{advrs}\\
      \glt       ‘Though, (I) don’t know what (I) have said (about the contents of the Pear Film).’ [\textsc{pf}: 090222\_00.txt]
    \z
\z

Additionally, \textit{ka} (\textsc{du}B) can be used as the marker of the indirect polar question, where there is no interrogative word.

\ea\label{ex:10.79}   As a maker of indirect polar question (or “Yes-no question”)\\
  \ea  %\textsc{tm}:
      \glll    un  kawajəəka  sijanban,\\
      \textit{u-n}  \textit{kawajəə=\Highlight{ka}}  \textit{sij-an=ban}\\
      \textsc{mes}-\textsc{adn}Z  substitute=\textsc{du}B  know-\textsc{neg}=\textsc{advrs}\\
      \glt       ‘(I) don’t know whether (it is) a substitute (for a hat), but ...’ [\textsc{pf}: 090225\_00.txt]

  \ex  %\textsc{tm}:
      \glll    wanna  ikjukka  ikjanka  waarandoo.\\
      \textit{wan=ja}  \textit{ik-jur=\Highlight{ka}}  \textit{ik-an=\Highlight{ka}}  \textit{waar-an=doo}\\
      1\textsc{sg}=\textsc{top}  go-\textsc{umrk}=\textsc{du}B  go-\textsc{neg}=\textsc{dub}  know-NEG=\textsc{ass}\\
      \glt       ‘I don’t know whether (I) will go (there) or not.’ [El: 130812]
    \z
\z

The examples in (10-78 a-b) and (10-79 b) show that \textit{ka} (\textsc{du}B) directly attaches to the preceding verbal stem, which means it is an affix-like clitic (see \sectref{sec:4.2.2.2}).

  Secondly, \textit{ka} (\textsc{du}B) can follow an interrogative NP (i.e. an NP headed by an interrogative word), and it derives an indefinite NP as in (10-80 a-d) (see also \sectref{sec:5.3.2}).

\ea\label{ex:10.80}   As a maker to derive an indefinite NP from an interrogative NP\\
   \ea\relax [Context: \textsc{tm} said to \textsc{ms} that her son was always busy.] = (5-39 a)\\
    \textsc{tm}:
      \glll    {\textbar}dojoo{\textbar}.  {\textbar}nicijoo{\textbar}.  jazin  nuukanu  ai.\\
      \textit{dojoo}  \textit{nicijoo}  \textit{jazin}  \textit{\Highlight{nuu=ka}=nu}  \textit{ar-i}\\
      Saturday  Sunday  necessarily  what=\textsc{du}B=\textsc{nom}  exist-\textsc{npst}\\
      \glt       ‘Saturday. Sunday. There is always something.’ [Co: 120415\_01.txt]

  \ex\relax  [Context: \textsc{tm} explained to \textsc{my} why she had called her.] = (5-39 c)\\
    \textsc{tm}:
      \glll    uran  daacika  ikjarɨncjɨga, ...\\
      \textit{ura=n}  \textit{\Highlight{daa}=kaci=ka}  \textit{ik-arɨr-n=cjɨ=ga}\\
      2.N\textsc{hon}.\textsc{sg}=\textsc{dat}1  where=\textsc{all}=\textsc{du}B  go-P\textsc{ass}-\textsc{ptcp}=\textsc{qt}=\textsc{foc}\\
      \glt       ‘(I thought I) would suffer from your going somewhere, (so I called you.)’ [Co: 101020\_01.txt]

  \ex  \textsc{tm}:
      \glll    daananka  aroo.\\
      \textit{\Highlight{daa}=nan=\Highlight{ka}}  \textit{ar-oo.}\\
      where=\textsc{loc}1=\textsc{du}B  exist-\textsc{supp}\\
      \glt       ‘Probably, (a mallet) is somewhere.’ [Co: 120415\_00.txt]

  \ex   US: \glll taruutuka  oojunwakecjɨjo.\\
      \textit{\Highlight{ta-ru}=tu=\Highlight{ka}}  \textit{oow-jur-n=wake=ccjɨ=joo}\\
      who-\textsc{nlz}=\textsc{com}=\textsc{du}B  see-\textsc{umrk}-\textsc{ptcp}=\textsc{cfp}=\textsc{qt}=\textsc{cfm}1\\
      \glt       ‘(I) see someone (when I go shopping to the store in this neighborhood).’ [Co: 110328\_00.txt]
    \z
\z

The above examples show that \textit{ka} (\textsc{du}B) can intervene between the nominal and \textit{nu} (\textsc{nom}) as in (10-80 a), but it cannot in the case of \textit{kaci} (\textsc{all}), \textit{nan} (\textsc{loc}1) and \textit{tu} (\textsc{com}), and it follows them as in (10-80 b-d).

\subsection{\textit{gajaaroo} (\textsc{du}B)}\label{sec:10.4.3}

\textit{gajaaroo} (\textsc{du}B) has the same functions as \textit{ka} (\textsc{dub}) discussed in \sectref{sec:10.4.2}. \textit{gajaaroo} (DUB) is frequentely realized as /garoo/ (or /karoo/) as in (10-81 a, c-d).

\ea\label{ex:10.81}   As a maker of an indirect information question (or “Wh-question”)\\
  \ea\relax  [Context: Looking at a picture, \textsc{tm} remembered a man.] = (5-38 b)\\
    \textsc{tm}:
      \glll    daanan  wukkaroo,  wakaija  sɨranbajaa.\\
      \textit{\Highlight{daa}=nan}  \textit{wur=\Highlight{gajaaroo}}  \textit{wakar-i=ja}  \textit{sɨr-an-ba=jaa}\\
      where=\textsc{loc}1  exist=\textsc{du}B  understand-\textsc{inf}=\textsc{top}  do-\textsc{neg}-\textsc{csl}=\textsc{sol}\\
      \glt       ‘(I) don’t know where (he) is.’ [Co: 120415\_01.txt]

  \ex  US:  \glll un  kacjən  kabikkwaga  daakaci  ucjɨgajaaroo,\\
      \textit{u-n}  \textit{kak-təər-n}  \textit{kabi-kkwa=ga}  \textit{\Highlight{daa}=kaci}  \textit{uk-tɨ=\Highlight{gajaaroo}}\\
      \textsc{mes}-\textsc{adn}Z  write-\textsc{rsl}{}-\textsc{ptcp}  paper{}-\textsc{dim}=\textsc{nom}  where=\textsc{all}  put-\textsc{seq}=\textsc{du}B\\
      \glt       ‘(I don’t know) where (I) put the paper that (I) had written (my granddaughter’s name on).’ [Co: 110328\_00.txt]

  \ex  \textsc{tm}:
      \glll    {\textbar}josizoo{\textbar}ga  wuija  sjunban,  daanan wukkaroo  wakaija  sɨranbajaa.\\
      \textit{josizoo=ga}  \textit{wur-i=ja}  \textit{sɨr-jur-n=ban}  \textit{\Highlight{daa}=nan}  \textit{wur=\Highlight{gajaaroo}}  \textit{wakar-i=ja}  \textit{sɨr-an-ba=jaa}\\
      Yoshizo=\textsc{nom}  exist-\textsc{inf}=\textsc{top}  do-\textsc{umrk}-\textsc{ptcp}=\textsc{advrs}  where=\textsc{loc}1  exist=\textsc{du}B  know-INF=TOP  do-\textsc{neg}-\textsc{csl}=\textsc{sol}\\
      \glt       ‘There is Yoshizo [i.e. Yoshizo is still alive], but (I) don’t know where (he) lives, so ...’ [Co: 120415\_01.txt]

  \ex  \textsc{tm}:
      \glll    ɨcɨɨ  cɨrɨtɨ  izjɨgaroo  wakarancjɨdu.\\
      \textit{\Highlight{ɨcɨɨ}}  \textit{cɨrɨr-tɨ}  \textit{ik-tɨ=\Highlight{gajaaroo}}  \textit{wakar-an=ccjɨ=du}\\
      when  go.with-\textsc{seq}  go-SEQ=\textsc{du}B  know-\textsc{neg}=\textsc{qt}=\textsc{foc}\\
      \glt       ‘(She said) that (she) doesn’t know when (the person) went with (the other person).’ [Co: 120415\_01.txt]
    \z
\z

Additionally, \textit{gajaaroo} (\textsc{du}B) can be used as a marker of the indirect polar question, where there is no interrogative word.

\ea\label{ex:10.82}   As a maker of indirect polar question (or “Yes-no question”)\\
  %\textsc{tm}:
      \glll    wanna  ikjukkajaaroo  ikjangajaaroo  waarandoo.\\
    \textit{wan=ja}  \textit{ik-jur=\Highlight{gajaaroo}}  \textit{ik-an=\Highlight{gajaaroo}}  \textit{waar-an=doo}\\
    1\textsc{sg}=\textsc{top}  go-\textsc{umrk}=\textsc{du}B  go-\textsc{neg}=\textsc{dub}  know-NEG=\textsc{ass}\\
    \glt     ‘I don’t know whether (I) will go (there) or not.’ [El: 130812]
\z

The above examples show that \textit{gajaaroo} (\textsc{du}B) has the same function as \textit{ka} (\textsc{dub}), i.e., they can be used to mark the indirect question. If the embedded clause indicates the non-past tense, both \textit{gajaaroo} (DUB) and \textit{ka} (DUB) can follow directly the bound verbal stem as in (10-81 a, c), \textsc{ref}{ex:10.82}, and (10-79 b) in \sectref{sec:10.4.2}. That is, \textit{gajaaroo} (DUB) is an affix-like clitic as well as \textit{ka} (DUB) (see \sectref{sec:4.2.2.2}). However, there is a difference between them. On the one hand, if the embedded clause indicates the past tense, the verb takes \textit{{}-tɨ} (\textsc{seq}) before \textit{gajaaroo} (DUB) as in (10-81 b, d). On the other hand, in the same environment, the verb takes \textit{{}-tar} (\textsc{pst}) before \textit{ka} (DUB) as in (10-78 a-b) in \sectref{sec:10.4.2}.

  \textit{gajaaroo} (\textsc{du}B) can follow an interrogative NP, and can derive an indefinite NP as in (10-83 a-c) (see also \sectref{sec:5.3.2}).

\ea\label{ex:10.83}   As the maker to derive an indefinite NP from an interrogative NP\\
  \ea\relax  [Context: Looking at pictures of the shopping street in the village] = (5-40 b)\\
    %\textsc{tm}:
      \glll    nuucjɨgajaaroo  kacjəəttujaa.\\
      \textit{\Highlight{nuu}=ccjɨ=\Highlight{gajaaroo}}  \textit{kak-təər-tu=jaa}\\
      what=\textsc{qt}=\textsc{du}B  write-\textsc{rsl}-\textsc{csl}=\textsc{sol}\\
      \glt       ‘Something has been drawn (on the sign board of the store).’ [Co: 120415\_00.txt]

  \ex  %\textsc{tm}:
      \glll    daanangaroo  sjasinnan  {\textbar}nakaudo{\textbar}nu,  (an..)  ukɨnnantɨ  sangun  sjunturonkja,\\
      \textit{\Highlight{daa}=nan=\Highlight{gajaaroo}}  \textit{sjasin=nan}  \textit{nakaudo=nu}  \textit{a-n}  \textit{ukɨn=nantɨ}  \textit{sangun}  \textit{sɨr-tur-n=turoo=nkja}\\
      where=\textsc{loc}1=\textsc{du}B  picture=LOC1  matchmaker=\textsc{nom}  \textsc{dist}-\textsc{adn}Z  Uken=LOC2  betrothal.present  do-\textsc{prog}-\textsc{ptcp}=scene=\textsc{appr}\\
      \glt       ‘The scene where the matchmaker was doing [i.e. was having the couple exchange] the betrothal presents at the Uken community (appeared) somewhere in the picture.’ [Co: 120415\_01.txt]

  \ex  %\textsc{tm}:
      \glll    naa  ɨcɨn  madungajaaroo  naa  un  utankjan  {\textbar}zjenzjen{\textbar},\\
      \textit{naa}  \textit{\Highlight{ɨcɨ}=n}  \textit{madu=n=\Highlight{gajaaroo}}  \textit{naa}  \textit{u-n}  \textit{uta=nkja=n} \textit{zjenzjen}\\
      \textsc{fil}  when=\textsc{gen}  time=\textsc{dat}1=\textsc{du}B  yet  \textsc{mes}-\textsc{adn}Z  song=\textsc{appr}=also  at.all\\
      \glt       ‘At the time (when I don’t know) when (it began), (old people in the community began) not to sing (the song) at all anymore.’ [Co: 120415\_01.txt]
    \z
\z

In (10-83 a), \textit{nuu} ‘what’ and \textit{gajaaroo} (\textsc{du}B) means ‘something,’ where \textit{ccjɨ} (\textsc{qt}) intervenes between them and embeds them into the complement of \textit{kak-} ‘write’ (see also \sectref{sec:10.4.1.2}). In (10-83 b), \textit{daa} ‘where’ and \textit{gajaaroo} (\textsc{dub}) means ‘somewhere.’ In (10-83 c), it is ambiguous whether it is an example of the indefinite NP or that of the indirect question. In the latter interpretation, it is thought that the predicate of the superordinate clause, e.g., \textit{sij-an} (know-\textsc{neg}) ‘(I) don’t know,’ was omitted.

Furthermore, \textit{gajaaroo} (\textsc{du}B) can be used neither to express an indirect question nor to derive an indifinite NP. In that case, \textit{gajaaroo} (\textsc{dub}) expresses the speaker’s dubitation (or uncertainty) about (the referents of) the units they are attached to. This kind of function has not been found in \textit{ka} (DUB) so far.

\ea\label{ex:10.84}   To express the speaker’s dubitation\\
  \ea  %\textsc{tm}:
      \glll    kurəə  burɨncjɨgajaaroo  jutattujaa.\\
      \textit{ku-rɨ=ja}  \textit{burɨn=ccjɨ=\Highlight{gajaaroo}}  \textit{jˀ-tar-tu=jaa}\\
      \textsc{prox}-\textsc{nlz}=\textsc{top}  Buren=\textsc{qt}=\textsc{du}B  say-\textsc{pst}-\textsc{csl}=\textsc{sol}\\
      \glt       ‘(Someone) said that this (picture was) Buren, so (I think it is that of Buren).’ [Co: 120415\_01.txt]

  \ex  %\textsc{tm}:
      \glll    {\textbar}ken{\textbar}nantɨ  abɨnəə  {\textbar}iciban{\textbar}cjɨgajaaroodu  jutattu,\\
      \textit{ken=nantɨ}  \textit{abɨnəə}  \textit{iciban=ccjɨ=\Highlight{gajaaroo}=du}  \textit{jˀ-tar-tu}\\
      prefecture=\textsc{loc}2  nearly  the.most=\textsc{qt}=\textsc{du}B=\textsc{foc}  say-\textsc{pst}-\textsc{csl}\\
      \glt       ‘(Someone) said that (she was) nearly the (old)est in the (Kagoshima) Prefecture, so ...’ [Co: 120415\_01.txt]

  \ex  %\textsc{tm}:
      \glll    kurɨbəi,  ude,  naikwa  nootutɨgaroo,  an ...  {\textbar}sjuusjencjokugo{\textbar}ja,\\
      \textit{ku-rɨ=bəi}  \textit{ude}  \textit{naikwa}  \textit{noor-tur-tɨ=\Highlight{gajaaroo}}  \textit{a-n}  \textit{sjuusjencjokugo=ja}  \\
      \textsc{prox}-\textsc{nlz}=only  well  a.few  remain-\textsc{prog}-\textsc{seq}=\textsc{du}B  \textsc{dist}-\textsc{adn}Z  immediately.after.the.war=\textsc{top}        \\
      \glt       ‘Only this (building), a few (parts of it), remained, (I) suppose, immediately after that war, ...’ [Co: 120415\_00.txt]
    \z
\z

\subsection{\textit{nən} ‘such as’}\label{sec:10.4.4}

\textit{nən} ‘such as’ always embeds the preceding units into the complement of \textit{sɨr-} ‘do.’ The complement’s head, i.e. \textit{sɨr-} ‘do,’ usually takes \textit{{}-tɨ} (\textsc{seq}) when modifying a verb, or takes \textit{-tar-n} (\textsc{pst}-\textsc{ptcp}) when modifying a nominal.

  First, I will show the examples where the units followed by \textit{nən} ‘such as’ fill the complements of /sjɨ/ \textit{sɨr-tɨ} (do-\textsc{seq}), which in turn modify the verb in the superordinate clause.

\ea\label{ex:10.85}   \textit{nən} ‘such as’ + \textit{sɨr-tɨ} (do-\textsc{seq})\\
  \ea After a nominal [= (9-33)]\\
  %\textsc{tm}:
      \glll    muru  kjoodəənən  sjɨ,  sjɨ  moojutattujaa.\\
    \textit{muru}  \textit{kjoodəə=\Highlight{nən}}  \textit{\Highlight{sɨr-tɨ}}  \textit{sɨr-tɨ}  \textit{moor-jur-tar-tu=jaa}\\
    very  brother=such.as  do-\textsc{seq}  do-SEQ  \textsc{hon}-\textsc{umrk}-\textsc{pst}-\textsc{csl}=\textsc{sol}\\
\glt     ‘(They) used to keep company with each other like brothers.’  [Co: 120415\_01.txt]

  \ex After an infinitive + \textit{n} (\textsc{dat}1)\\
  %\textsc{tm}:
      \glll    nobuaritaaga  {\textbar}kjooikuiin{\textbar}nan  wuinnən  sjɨ  jappoo,  hɨmanu  anban,\\
    \textit{nobuari-taa=ga}  \textit{kjooikuiin=nan}  \textit{wur-i=n=\Highlight{nən}}  \textit{\Highlight{sɨr-tɨ}}  \textit{jar-boo}  \textit{hɨma=nu}  \textit{ar-n=ban}\\
    Nobuari-\textsc{pl}=\textsc{nom}  Board.of.Education=\textsc{loc}1  exist-\textsc{inf}=\textsc{dat}1=such.as  do-\textsc{seq}  \textsc{cop}-\textsc{cnd}  time=NOM  exist-\textsc{ptcp}=\textsc{advrs}\\
\glt     ‘If (it were) the time such as when Nobuari was in the Board of Education, (he) has (plenty of) time, but ...’  [Co: 120415\_01.txt]

  \ex After a participle\\
  %\textsc{tm}:
      \glll    mukasinu  huccjunu  jun  tuki ..   jutannən  sjɨ,\\
    \textit{mukasi=nu}  \textit{huccju=nu}  \textit{jˀ-jur-n}  \textit{tuki} \textit{jˀ-jur-tar-n=\Highlight{nən}}  \textit{\Highlight{sɨr-tɨ}}\\
    the.past=\textsc{gen}  old.people=\textsc{nom}  say-\textsc{umrk}-\textsc{ptcp}  time  say-UMRK-\textsc{pst}-PTCP=such.as  do-\textsc{seq}\\
\glt     ‘When the old people in the past used to say, just as (they) used to say, ...’  [Co: 120415\_01.txt]

  \ex After a participle (interrupted by \textit{ga})\\
  %\textsc{tm}:
      \glll    naa,  cukutun  cˀjunkjaga,  naa,  ura,  ɨcɨɨ  sizjɨn,  naa,  ɨrɨrarɨnganən  sjɨ,  (sɨcɨ)  sɨcɨcjɨ  jˀicjɨjo,\\                                                                                                                                                                                                      
    \textit{naa}  \textit{cukur-tur-n}  \textit{cˀju=nkja=ga}  \textit{naa}  \textit{ura}  \textit{ɨcɨɨ} \textit{sin-tɨ=n}  \textit{naa}  \textit{ɨrɨr-arɨr-n=\Highlight{ga=nən}}  \textit{\Highlight{sɨr-tɨ}}  \textit{sɨcɨ} \textit{sɨcɨ=ccjɨ}  \textit{jˀ-tɨ=joo}\\                                                                                                                                                                                                      
    \textsc{fi}l  make-\textsc{prog}-\textsc{ptcp}  person=\textsc{appr}=\textsc{nom}  \textsc{fil}  2.N\textsc{hon}.\textsc{sg}  when   die-\textsc{seq}=even  FIL  put.in-\textsc{cap}-PTCP=\textsc{ga}=such.as  do-SEQ  coffin  coffin=\textsc{qt}  say-SEQ=\textsc{cfm}1\\
    \glt ‘As the person who made (the coffin) can be put (there) whenever (the person) dies, (there is a thing) called \textit{sɨcɨ} [i.e. coffin], and ...’ [Co: 111113\_01.txt]
    \z
\z

/nən sjɨ/ \textit{nən} \textit{sɨr-tɨ} (such.as do-\textsc{seq}) follows a nominal as in (10-85 a), and follows a verb as in (10-85 b-d). In (10-85 c), \textit{nən} directly follows a participle, but in (10-85 d), it is interrupted by \textit{ga}. This particle has the same form with the focus particle \textit{ga}, but I am not sure whether it is \textit{ga} (\textsc{foc}) or not for now.

  Secondly, I will present the examples where the units followed by \textit{nən} ‘such as’ fill the complements of /sjan/ \textit{sɨr-tar-n} (do-\textsc{pst}-\textsc{ptcp}), which in turn modify the nominal in the superordinate clause.

\ea\label{ex:10.86}   \textit{nən} ‘such as’ + \textit{sɨr-tar-n} (do-\textsc{pst}-\textsc{ptcp})\\
  \ea After a nominal\\
  %\textsc{tm}:
      \glll    maganən  sjan  injawarabɨnu  cˀjɨ,\\
    \textit{maga=\Highlight{nən}}  \textit{\Highlight{sɨr-tar-n}}  \textit{inja+warabɨ=nu}  \textit{k-tɨ}\\
    grandchild=such.as  do-\textsc{pst}-\textsc{ptcp}  small+child=\textsc{nom}  come-\textsc{seq}\\
    \glt     ‘A small child such as a grandchild came, and ...’ [\textsc{pf}: 090225\_00.txt]

  \ex After a participle\\

  %\textsc{tm}:
      \glll    noogusukuja  naanai  pˀaacjɨ  aagai  cɨkɨtutannən  sjan  {\textbar}kanzi{\textbar}.\\
    \textit{noogusuku=ja}  \textit{naa+nai}  \textit{pˀaa=ccjɨ}  \textit{aagai}  \textit{cɨkɨr-tur-tar-n=\Highlight{nən}}  \textit{\Highlight{sɨr-tar-n}}  \textit{kanzi}\\
    Nogusuku=\textsc{top}  other+a.little  shining=\textsc{qt}  light turn.on-\textsc{prog}-\textsc{pst}-\textsc{ptcp}=such.as  do-P\textsc{st}-PTCP  atmosphere\\
\glt     ‘Nogusuku [i.e. the name of a place] has an atmosphere just as (someone) was turning on a shining light a little.’  [Co: 120415\_01.txt]

  \ex After a participle (interrupted by \textit{ga})\\

  %\textsc{tm}:
      \glll    {\textbar}kawa{\textbar}bunɨccjɨ  kan  sjɨ  an  {\textbar}hunakudari{\textbar}  sjunganən  sjan  {\textbar}kanzi{\textbar}sjɨ,  {\textbar}soko{\textbar}ja   mattawu  natɨ,\\
    \textit{kawa+hunɨ=ccjɨ}  \textit{ka-n}  \textit{sɨr-tɨ}  \textit{a-n}  \textit{hunakudari}  \textit{sɨr-jur-n=\Highlight{ga=nən}}  \textit{\Highlight{sɨr}-tar-n  kanzi=sjɨ  soko=ja}  \textit{mattawu}  \textit{nar-tɨ}\\
    river+boat=\textsc{qt}  \textsc{prox}-\textsc{advz}  do-\textsc{seq}  \textsc{dist}-\textsc{adn}Z  descending.by.the.boat do-\textsc{umrk}-\textsc{ptcp}=\textsc{ga}=such.as  do-\textsc{pst}-PTCP  atmosphere=\textsc{inst}  bottom=\textsc{top}  very.flat  \textsc{cop}-SEQ    \\
    \glt     ‘(Speaking of) \textit{kawabunɨ} [i.e. a river boat], (it) is similar to (the boat) by which (people) descend (a river) like this [lit. with an atmosphere where (people) descend (a river) like this], and the bottom is very flat, and ...’  [Co: 111113\_01.txt]
    \z
\z

/nən sjan/ \textit{nən} \textit{sɨr-tar-n} (such.as do-\textsc{pst}-\textsc{ptcp}) follows a nominal as in (10-86 a), and follows a verb as in (10-86 b-c). In (10-86 b), \textit{nən} directly follows a participle, but in (10-86 c), it is interrupted by \textit{ga} as well as in (10-85 d).

  In the text data, \textit{sɨr-} ‘do’ (as the head of the complement, following \textit{nən} ‘such as’) always takes \textit{{}-tɨ} (\textsc{seq}) as in \textsc{ref}{ex:10.85} or \textit{{}-tar-n} (\textsc{pst}-\textsc{ptcp}) as in \REF{ex:10.86}. However, it can take other inflections in elicitation as in (10-87 a-b).

\ea\label{ex:10.87}   
\ea \textit{nən} ‘such as’ + \textit{sɨr-tur-i} (do-\textsc{prog}-\textsc{npst})\\
  %\textsc{tm}:
      \glll    tarun  wuranga  nən  sjui.\\
    \textit{ta-ru=n}  \textit{wur-an=ga}  \textit{\Highlight{nən}}  \textit{\Highlight{sɨr-tur-i}}\\
    who-\textsc{nlz}=even  exist-\textsc{neg}=\textsc{ga}  such.as  do-\textsc{prog}-\textsc{npst}\\
    \glt     ‘(It) seems (that) there isn’t anyone.’ [El: 120914]

  \ex \textit{nən} ‘such as’ + \textit{sɨr-tur-tar} (do-\textsc{prog}-\textsc{pst})\\
  %\textsc{tm}:
      \glll    tarun  wuranga  nən  sjutattoo.\\
    \textit{ta-ru=n}  \textit{wur-an=ga}  \textit{\Highlight{nən}}  \textit{\Highlight{sɨr-tur-tar}=doo}\\
    who-\textsc{nlz}=even  exist-\textsc{neg}=\textsc{ga}  such.as  do-\textsc{prog}-\textsc{pst}=\textsc{ass}\\
    \glt     ‘(It) seemed (that) there wasn’t anyone.’ [El: 120914]
    \z
\z

  Before concluding this section, it should be mentioned that \textit{nən} ‘such as’ has the same form with the existential verb in negative, i.e. /nən/ \textit{nə-an} (exist-\textsc{neg}) ‘not exist’ (see \sectref{sec:8.3.2.3}) and the sequential convebal affix \textit{{}-nən} (\textsc{seq}) (see \sectref{sec:8.4.3.5}). For now, I could not say anything about the diachronic relation or the synchronic commonality among these morphemes.

\section{Utterance-final particles B}\label{sec:10.5}

Yuwan has the utterance-final particles B as in \tabref{tab:101}. The utterance-final particles B can be hosted by the utterance, but the units followed by the utterance-final particles B are not necessarily embedded into the superordinate clauses, which is different from the utterance-final particles A discussed in \sectref{sec:10.4}. The term “utterance” here is used to indicate an abstract unit that can include both of the phrase and the clause.

\begin{table}
\caption{Utterance-final particles B\label{tab:101}}
\begin{tabular}{ll}
\lsptoprule
Form & Meaning\\\midrule
\textit{joo} &  Confirmation\\
\textit{jaa} &  Solidality\\\lspbottomrule
\end{tabular}
\end{table}

\textit{joo} (\textsc{cfm}1) and \textit{jaa} (\textsc{sol}) can follow many of the other particles discussed in the preceding sections. Additionally, \textit{jaa} (\textsc{so}L) can follow \textit{joo} (CFM1).

\textit{jaa}  (\textsc{sol}) and \textit{joo} (\textsc{cfm}1) have the counterparts in the interjections (see \sectref{sec:4.3.7}). \textit{jaa} (\textsc{so}L) and \textit{joo} (CFM1) as the interjections can start an utterance only by themselves, which is also disscussed in the following sections. This means that the particle-like uses of \textit{jaa} (SOL) and \textit{joo} (CFM1) are continuous with their interjection-like uses. The interjection \textit{naa} (\textsc{fil}) also often loses its own pitch (although it can start an utterance). Thus, it may be appropriate that such \textit{naa} (\textsc{fi}L) be regarded as a particle. However, the unit followed by the clitic-like \textit{naa} (FIL) is always embedded in another superordinate clause. Thus, it may be appropriate to categorize it as the sentence-final particle A, although it needs further investigation.

First, I will present examples of \textit{joo} (\textsc{cfm}1) in \sectref{sec:10.5.1}. Then, I will present examples of \textit{jaa} (\textsc{sol}) in \sectref{sec:10.5.2}.

\subsection{\textit{joo} (\textsc{cfm}1)}\label{sec:10.5.1}

\textit{joo} (\textsc{cfm}1) is used to draw the hearer’s attention. \textit{joo} (CFM1) often becomes /jo/ as in (10-88 a-d, f). The units that can precede \textit{joo} (CFM1) are full of variety.

\ea\label{ex:10.88}   \textit{joo} (\textsc{cfm}1)\\
    \begin{xlist}
  \exi{} After predicates

  \ex  After the verbal predicate phrase whose final verbal form is a finite form [= (9-4 b)]\\
    %\textsc{tm}:
      \glll    nu-nkuin  atɨ  moojuijo.\\
      \textit{nuu-nkuin}  \textit{ar-tɨ}  \textit{moor-jur-i=\Highlight{joo}}\\
      what-\textsc{indfz}  exist-\textsc{seq}  \textsc{hon}-\textsc{umrk}-\textsc{npst}=\textsc{cfm}1\\
      \glt       ‘(At \textsc{ms}’s grandfather’s place,) they had everything.’ [Co: 120415\_01.txt]

  \ex After the verbal predicate phrase whose final verbal form is a converb\\
    %\textsc{tm}:
      \glll    mukasinu  sɨcɨzɨbatɨja,  naa,  kɨɨnu  muituppoojo,  un  sɨcɨzɨja,  naa,  nən  najuttɨjaa.\\
      \textit{mukasi=nu}  \textit{sɨcɨzɨ+hatɨɨ=ja}  \textit{naa}  \textit{kɨɨ=nu}  \textit{muij-tur-boo=\Highlight{joo}}     \textit{u-n}  \textit{sɨcɨzɨ=ja}  \textit{naa}  \textit{nə-an}  \textit{nar-jur-tɨ=jaa}\\
      the.past=\textsc{gen}  cycad+field=\textsc{top}  \textsc{fil}  tree=\textsc{nom}  grow-\textsc{prog}-\textsc{cnd}=\textsc{cfm}1  \textsc{mes}-\textsc{adn}Z  cycad=TOP  \textsc{fi}L  exist-\textsc{neg}  become-\textsc{umrk}-\textsc{seq}=\textsc{sol}\\
      \glt       ‘About the cycad field in the past, if other trees grew (around the cycad trees), the cycad trees became extinct.’ [Co: 111113\_02.txt]

   \ex  After the adjectival predicate phrase [= (9-25 b)]\\
    %\textsc{tm}:
      \glll    nuuga?  kurɨ  kurɨ.  kusarəə  sɨranba,  jiccjaijo.\\
      \textit{nuu=ga}  \textit{ku-rɨ}  \textit{ku-rɨ}  \textit{kusarɨr-∅=ja}  \textit{sɨr-an-ba}  \textit{jiccj-sa+ar-i=\Highlight{joo}}\\
      what=\textsc{foc}  \textsc{prox}-\textsc{nlz}  PROX-NLZ  rot-\textsc{inf}=\textsc{top}  do-\textsc{neg}-\textsc{csl}  no.problem-\textsc{adj}+\textsc{st}V-\textsc{npst}=\textsc{cfm}1\\
      \glt       ‘What? This (one), this (one). (It) will not rot, so (it) is no problem (for you to bring it back).’ [Co: 101023\_01.txt]

   \ex After the nominal predicate phrase\\
    %\textsc{tm}:
      \glll    jonesige  {\textbar}neesan{\textbar}.jo\\
      \textit{jonesige}  \textit{neesan=\Highlight{joo}}\\
      Yoneshige  elder.sister=\textsc{cfm}1\\
      \glt       ‘(She is) Yoneshige’s elder sister.’ [Co: 110328\_00.txt]

  \exi{} After argument NPs

  \ex  After the nominative NP [= (6-95 a)]\\
    %\textsc{tm}:
      \glll    jonesigetaa  cˀjantu  attaa  ziisantugajoo   {\textbar}itoko{\textbar}bəi  najuncjɨ.\\                                                                                             
      \textit{jonesige-taa}  \textit{cˀjan=tu}  \textit{a-rɨ-taa}  \textit{ziisan=tu=ga=\Highlight{joo}} \textit{itoko=bəi}  \textit{nar-jur-n=ccjɨ}\\                                                                                             
      Yoneshige-\textsc{pl}  father=\textsc{com}  \textsc{dist}-\textsc{nlz}-PL  grandfather=COM=\textsc{nom}=\textsc{cfm}1   cousin=only  become-\textsc{umrk}-\textsc{ptcp}=\textsc{qt}\\
      \glt ‘Yoneshige’s father and his [i.e the present speaker’s] grandfather are cousin, (I heard).’ [Co: 110328\_00.txt]

  \exi{} After an adverb

  \ex  %\textsc{tm}:
      \glll    asahuci,  asajo  izjɨ  cˀjɨn  njɨcjɨ  kinju  jˀicjanwakejo.\\                                                                                              
      \textit{asahuci}  \textit{asa=joo}  \textit{ik-tɨ}  \textit{k-tɨ=n}  \textit{nj-ɨ=ccjɨ} \textit{kinju}  \textit{jˀ-tar-n=wake=\Highlight{joo}}\\                                                                                              
      morning  morning=\textsc{cfm}1  go-\textsc{seq}  come-SEQ=ever  \textsc{exp}-\textsc{imp}=\textsc{qt}   yesterday  say-\textsc{pst}-\textsc{ptcp}=\textsc{cfp}=CFM1\\
      \glt ‘Yesterday morning, (I) said, “Try to go (to your place)!”’      [Co: 110328\_00.txt]
    \end{xlist}
\z

Additionally, \textit{joo} (\textsc{cfm}1) can follow the imperative, e.g., \textit{mukk-\Highlight{oo=joo}} (bring-\Highlight{\textsc{imp}=CFM1}) ‘Bring (it)!’ as in (10-31 a) in \sectref{sec:10.2.2}, the modifier NP, e.g., \textit{nama=\Highlight{nu=joo}} \textit{warabɨ=nkja} (now=\Highlight{\textsc{gen}=CFM1} child=\textsc{appr}) ‘the children in these days [lit. the children of now]’ as in \textsc{ref}{ex:10.7} in \sectref{sec:10.1.1.2}, or \textit{nusi=\Highlight{nu=joo}} \textit{jinga-nəə=nkja} (now=\Highlight{GEN=CFM1} man-parent=\textsc{app}R) ‘her father [lit. herself’s father]’ as in \REF{ex:10.64} in \sectref{sec:10.4.1.1}.

  If \textit{joo} (\textsc{cfm}1) follows \textit{ccjɨ} (\textsc{qt}), the clause followed by \textit{ccjɨ} (QT) can be used as the main clause expressing that it is of the objective (not hearsay) information (see \sectref{sec:10.4.1.7} for more details).

  Before concluding this section, I will present an example of an interjection that seems to have the same origin with \textit{joo} (\textsc{cfm}1).

\ea\label{ex:10.89}   \textit{joo} (\textsc{cfm}1) as an interjection\\\relax
  [Context: \textsc{tm} describs US’s behavior to the present author in front of US.]\\
  %\textsc{tm}:
      \glll    joo.  cˀjunu  məəci  cˀjəəran,  naa,  {\textbar}ittoki{\textbar}n  joosjurancjɨjo.\\
    \textit{\Highlight{joo}}  \textit{cˀju=nu}  \textit{məə=kaci}  \textit{k-təəra=n}  \textit{naa}  \textit{ittoki=n}   \textit{joosjur-an=ccjɨ=joo}\\
    \textsc{cfm}1  person=\textsc{gen}  front=\textsc{all}  come-after=even  \textsc{fil}  for.a.moment=even  keep.still-\textsc{neg}=\textsc{qt}=CFM1\\
\glt     ‘Hey. (US) cannot keep still, even after (she) came to a person’s place [i.e. even when (she) visit a friend (like this)].’  [Co: 110328\_00.txt]
\z

In \textsc{ref}{ex:10.89}, the speaker started her utterance with \textit{joo} (\textsc{cfm}1), which is used to attract the hearer’s [i.e. the present author’s] attention.

\subsection{\textit{jaa} (\textsc{sol})}\label{sec:10.5.2}

First, the basic characteristics of \textit{jaa} (\textsc{sol}) are presented in \sectref{sec:10.5.2.1}. Then, \textit{jaa} (\textsc{so}L) is compared with \textit{jəə} (\textsc{cfm}2) in \sectref{sec:10.5.2.2}, since they express a distinction that is similar to that of the first-person inclusive vs. exclusive found in the languages around the world (cf. \citealt{Payne1997}: 45).

\subsubsection{Basic characteristics of \textit{jaa} (\textsc{sol})}\label{sec:10.5.2.1}

\textit{jaa} (\textsc{sol}) is used to require the hearer’s empathy or to express the speaker’s empathy with the hearer. The units that can precede \textit{jaa} (\textsc{so}L) are full of variety. For example, \textit{jaa} (SOL) can follow the verbal predicate as in (10-9 a) in \sectref{sec:10.1.2.1} (the verb is a finite form) or (10-31 a) in \sectref{sec:10.2.2} (the verb is a participle with the conjunctive particle \textit{sjutɨ} (\textsc{seq})), the adjectival predicate as in (9-44 a) in \sectref{sec:9.2.1} (immediately after the adjective) or (10-62 d) in \sectref{sec:10.4.1.1} (after the stative verb), the nominal predicate as in (10-90 a) (immediately after the predicate NP) or (4-13 b) in \sectref{sec:4.1.3.3} (after the copula verb). Additionally, \textit{jaa} (SOL) can follow another particles, such as the conjunctive particle \textit{ban} (\textsc{advrs}) as in (10-90 b), the clause-final particle \textit{doo} (\textsc{ass}) as in (10-90 c) or \textit{kai} (\textsc{du}B) as in \textsc{ref}{ex:10.50} in \sectref{sec:10.3.6}, the utterance-final particle A \textit{ccjɨ} (\textsc{qt}) as in (10-74 b) in \sectref{sec:10.4.1.7}, or the utterance-final particle B \textit{joo} (\textsc{cfm}1) as in (10-90 d). There are many examples that include \textit{jaa} (SOL) in the text data, but I have not yet found the example where \textit{jaa} (SOL) follows any case particle.

\ea\label{ex:10.90}   \textit{jaa} (\textsc{sol})\\
  \ea After the nominal predicate (immediately after the predicate NP)\\\relax
  [Context: Looking at a picture; \textsc{ms}: ‘Hey, this is the public well, (isn’t it?)’]\\
  %\textsc{tm}:
      \glll    tuinkoojaa.\\
    \textit{tuinkoo=\Highlight{jaa}}\\
    public.well=\textsc{sol}\\
  \glt     ‘(Actually, it is) the public well.’ [Co: 120415\_00.txt]

  \ex After the conjunctive particle \textit{ban} (\textsc{advrs})\\
  %\textsc{tm}:
      \glll    namanu  munna  naikwoo  wakajunban.jaa.\\
    \textit{nama=nu}  \textit{mun=ja}  \textit{naikwa=ja}  \textit{wakar-jur-n=ban=\Highlight{jaa}}\\
    now=\textsc{gen}  thing=\textsc{top}  a.little=TOP  know-\textsc{umrk}-\textsc{ptcp}=\textsc{advrs}=\textsc{sol}\\
  \glt     ‘(I) know the things from these days a little, but (it is easier to remember the things from the old days).’ [Co: 120415\_01.txt]

  \ex After the clause-final particle \textit{doo} (\textsc{acc})\\
  %\textsc{tm}:
      \glll    waa  məənannja  attojaa.\\
    \textit{waa}  \textit{məə=nan=ja}  \textit{ar=doo=\Highlight{jaa}}\\
    1\textsc{sg}.\textsc{adn}Z  place=\textsc{loc}1=\textsc{top}  exist=\textsc{ass}=\textsc{sol}\\
    \glt     ‘I have (the model plate to make \textit{katakˀwasi} [a kind of sweets]).’ [lit. ‘(It) exists at my place.’]  [Co: 111113\_01.txt]

   \ex After the utterance-final particle B \textit{joo} (\textsc{cfm}1)\\
  %\textsc{tm}:
      \glll    arəə  siccjuijojaa?  gazimaruja.\\
    \textit{a-rɨ=ja}  \textit{sij-tur-i=joo=\Highlight{jaa}}  \textit{gazimaru=ja}\\
    \textsc{dist}-\textsc{nlz}=\textsc{top}  know-\textsc{prog}-\textsc{npst}=\textsc{cfm}1=\textsc{sol}  bayan.tree=TOP\\
 \glt     ‘(You) know that, (i.e.) the banyan tree (don’t you?)’  [Co: 110328\_00.txt]
 \z
\z

The long vowel of \textit{doo} (\textsc{ass}) sometimes becomes short before \textit{jaa} (\textsc{sol}) as in (10-90 c). The long vowel of \textit{joo} (\textsc{cfm}1) always becomes short before \textit{jaa} (\textsc{so}L) as in (10-90 d).

  \textit{jaa} (\textsc{sol}) has its counterpart in the interjection as in \textsc{ref}{ex:10.91}.

\ea\label{ex:10.91}   \textit{jaa} (\textsc{sol}) as an interjection\\\relax
  [Context: Taking of the old days; US: ‘(I) borrowed (the money to let my children go to high school) from many people.’]\\
  %\textsc{tm}:
      \glll    jaa.  huntoo  {\textbar}kookoo{\textbar}  izjasijajaa.\\
    \textit{\Highlight{jaa}}  \textit{huntoo}  \textit{kookoo}  \textit{izjas-i=ja=jaa}\\
    \textsc{sol}  really  high.shool  let.go-\textsc{inf}=\textsc{top}=\textsc{so}L\\
\glt     ‘Yeah. Really (it is hard) to let (one’s children) go to high school.’  [Co: 110328\_00.txt]
\z

In the conversation described in \textsc{ref}{ex:10.91}, the speaker started her utterance with \textit{jaa} (\textsc{sol}), which is used to express the speaker’s empathy to the hearer.

\subsubsection{Comparison between \textit{jaa} (\textsc{sol}) and \textit{jəə} (\textsc{cfm}2) following \textit{{}-oo} (\textsc{int})}\label{sec:10.5.2.2}

\textit{jaa} (\textsc{sol}) can co-occur with many of the particles, but cannot with \textit{jəə} (\textsc{cfm}2). Both \textit{jaa} (\textsc{so}L) and \textit{jəə} (CFM2) can follow the finite-form affix \textit{-oo} (\textsc{int}) as in (7-25 g) in \sectref{sec:7.7} and \textsc{ref}{ex:10.46} in \sectref{sec:10.3.4}, but their meanings are critically different from each other. Their difference can be summarized as in \REF{ex:10.92}.

\ea\label{ex:10.92}
Comparison between \textit{jaa} (\textsc{sol}) and \textit{jəə} (\textsc{cfm}2) following \textit{{}-oo} (\textsc{int})
  \ea \textit{{}-oo=jaa} (\textsc{int}=\textsc{sol}) necessarily includes the hearer into the action indicated by the verbal stem;
  \ex \textit{{}-oo=jəə} (\textsc{int}=\textsc{cfm}2) necessarily excludes the hearer from the action indicated by the verbasl stem.
  \z
\z

The above distinction between \textit{{}-oo=jaa} (\textsc{int}=\textsc{sol}) and \textit{{}-oo=jəə} (INT=\textsc{cfm}2) is similar to the distinction between the first-person inclusive and the first-person exclusive found in the languages around the world (cf. \citealt{Payne1997}: 45). I will show the minimal pairs that exemplify (10-92 a-b).

First, (10-92 a) is attested by (10-93 a-b).

\ea\label{ex:10.93}   \textit{{}-oo=jaa} (\textsc{int}=\textsc{sol})\\
  \ea\relax [Context: Inviting the hearer]\\
    %\textsc{tm}:
      \glll     mazin  ikjoojaa.\\
       \textit{mazin}  \textit{ik-\Highlight{oo=jaa}}\\
       together  go-\textsc{int}=\textsc{sol}\\
      \glt       ‘Let’s go together.’ [El: 090830]

  \ex  %\textsc{tm}:
      \glll    *wan  cˀjui  ikjoojaa.\\
       \textit{wan}  \textit{cˀjui}  \textit{ik-\Highlight{oo=jaa}}\\
       1\textsc{sg}  one.person.\textsc{clf}  go-\textsc{int}=\textsc{sol}\\
       \glt    [El: 090830]
    \z
\z

In (10-93 a), /ikjoojaa/ \textit{ik-oo=jaa} (go-\textsc{int}=\textsc{sol}) can be used to invite the hearer. However, it cannot be used with the numeral \textit{cˀjui} (one.person.\textsc{clf}) ‘one person,’ which implies ‘alone,’ as in (10-93 b). These examples show that the combination of \textit{{}-}\textit{oo} (INT) and \textit{jaa} (\textsc{so}L) necessarily includes the hearer.

Secondly, (10-92 b) is attested by (10-94 a-b).

\ea\label{ex:10.94}   \textit{{}-oo=jəə} (\textsc{int}=\textsc{cfm}2)\\
  \ea\relax  [Context: Inviting the hearer]\\
    %\textsc{tm}:
      \glll    *mazin  ikjoojəə.\\
       \textit{mazin}  \textit{ik-\Highlight{oo=jəə}}\\
       together  go-\textsc{int}=\textsc{cfm}2\\
       \glt\relax      [El: 090830]

  \ex  %\textsc{tm}:
      \glll    wan  cˀjui  ikjoojəə.\\
       \textit{wan}  \textit{cˀjui}  \textit{ik-\Highlight{oo=jəə}}\\
       1\textsc{sg}  one.person.\textsc{clf}  go-\textsc{int}=\textsc{cfm}2\\
       \glt\relax    [El: 090830]
    \z
\z

In (10-94 a), /ikjoojəə/ \textit{ik-oo=jəə} (go-\textsc{int}=\textsc{cfm}2) cannot be used to invite the hearer. However, it can be used with the numeral \textit{cˀjui} (one.person.\textsc{clf}) ‘one person,’ which implies ‘alone,’ as in (10-94 b). These examples show that the combination of \textit{{}-oo} (INT) and \textit{jəə} (CFM2) necessarily excludes the hearer.
