\addchap{\lsAcknowledgementTitle} 
This grammar is the revised version of my PhD dissertation, which is submitted to the University of Tokyo in 2014. During the sixteen years that I have been working on this grammar, many people have offered data, ideas, comments, and other help. It is a pleasure to record my thanks to them.

My profound thanks to the speakers in Yuwan: Miyoko Fuji, Hisako Ishihara, Shigezo Mae, Fumio Matsumoto, Tama Mizobe, Masako Motoda, Nobuari Motoda, Sawako Motoda, Etsue Murano, Humizu Naka, Umine Shinozaki, Satoko Taira, Mitsuko Toshioka, and Sumie (Mutsu) Yamaki. My deepest gratitude should be expressed to Sachi (Tsuneko) Motoda, who taught me the Yuwan dialect when I first visited the village and who has been teaching it to me since then. Her special sense of humor always encouraged me to continue my research. My sincere thanks should be offered to Mioya Sunao, who spent a lot of time for helping me to transcribe the natural discourse in Yuwan. Without his help I could not have had the reliable data which my grammar is based on. I am also very grateful to the people in Suko district and the members of the Board of Education in Uken village.

Valuable advice was given from many researchers: Hayato Aoi, Reiko Aso, Kevin Baetscher, Joel Bradshaw, Shinichiro Fukuda, Margaret Ransdell-Green, Martin Haspelmath, Yuka Hayashi, David J. Iannucci, Shigehisa Karimata, Shinjiro Kazama, Nobuko Kibe, Kento Nagatsugu, Chiharu Obata, Shinji Ogawa, Mark Rosa, Mika Sakai, Matt Shibatani, Hiromi Shigeno, Rihito Shirata, Pellard Thomas, Nana Toyama, Tasaku Tsunoda, Zendo Uwano, and John Whitman. Above all, I owe endless gratitude to Michinori Shimoji, who had a substantial influence on this grammar. When I first read his PhD dissertation (\textit{A grammar of Irabu, a Southern Ryukyuan language}) submitted to the Australian National University, I realized this one is what I wanted to accomplish in my course of research. Most of the ideas, i.e. the construction of chapters, the terminologies, and the analyses of linguistic phenomena, largely depend on his grammar, although there are some reconstruction and modification since the target languages are different from each other.

The Language Science Press offered me a precious opportunity to publish my work. I genuinely appreciate their policy to make their publication available at no charge. An anonymous reviewer gave me critical remarks which were very precious to me. I could not find any word appropriate to show my gratitude to XXXX, XXX, XXX, and XXX, who found a huge number of typos in my original draft and gave me beneficial comments. Felix Kopecky kindly converted my original draft written as document files into that of LaTeX files.

At different times the research reported here has been supported by Research Institute for Languages and Cultures of Asia and Africa during April 2008 -- May 2009, and by Japan Society for the Promotion of Science during April 2010 -- May 2011. These supports are gratefully acknowledged.

Finally, I am particulary thankful to my family.
