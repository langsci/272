\chapter{Inter-clausal phenomena}\label{chap:11}

This chapter describes several inter-clausal phenomena. In \sectref{sec:11:1}, we will discuss the subordinate clauses, which can modify another clause. There are four types in the subordinate clauses: adverbial clause (where the subordinate clause functions as an adverb) (see \sectref{sec:11:1.1}); adnominal clause (where the subordinate clause functions as an adnominal) (see \sectref{sec:11:1.2}); nominal clause (where the subordinate clause functions as a nominal) (see \sectref{sec:11:1.3}); and complement clause (where the subordinate clause fills the complement slot of the verbal predicate phrase) (see \sectref{sec:11:1.4}). Some of the subordinate clauses can be used without their superordinate clauses. The conventionalized omission of the superordinate clause is called “insubordination” \citep{Evans2007}, which will be discussed in \sectref{sec:11:2}. In \sectref{sec:11:3}, I will present the phenomena that are related with the focus markers, especially the phenomenon called “kakari-musubi” (i.e. ‘government-predication’) in Japanese and Ryukyuan linguistics.

\section{Subordinate clauses}\label{sec:11:1}

Yuwan has four types of subordinate clauses: adverbial clauses (see \sectref{sec:11:1.1}); adnominal clauses (see \sectref{sec:11:1.2}); nominal clauses (see \sectref{sec:11:1.3}); and complement clauses (see \sectref{sec:11:1.4}). The dependency of the subordinate clauses on the superordinate clause is different from one to another. Many of the subordinate clauses can take their own subjects different from those in the superordinate clauses. However, the adverbial clauses headed by the converbs \textit{{}-tai} (LST) and \textit{{}-jagacinaa} (SIM) and the nominal clauses headed by the infinitives (not accompanied with \textit{n} (DAT1)) cannot take their own subjects (see \sectref{sec:8:4.3} and \sectref{sec:8:4.4.2} for more details).

\subsection{Adverbial clause}\label{sec:11:1.1}

The adverbial clause is the subordinate clause that functions as an adverb. The adverbial clause precedes its superordinate clause in principle. The adverbial clause can be expressed in two ways. First, the adverbial clause can be expressed by the converbal affixes. For example, \textit{{}-ba} (CSL) following the verbal stem can express a causal meaning as in (11-1 a\todo{Come back to this}) (see \sectref{sec:08:4.3} for more details). Secondly, the adverbial clause can also be expressed by the conjunctive particles as in (11-1 b) (see \sectref{sec:10:2} for more details).

%1
\ea\label{ex:11:1}
    Adverbial clauses in Yuwan\\
    \ea\label{ex:11:1a}
    Using a converb [= (8-86 a)]\\
    \relax[Context: MY asked TM if TM had made the pickles; TM: ‘(I) don’t know. How (was it)?’]\\
    \glll   nɨɨzinnu appa, arandaroo.\\
            [nɨɨzin=nu ar-ba]\textsubscript{Adverbial clause} ar-an=daroo\\
            carrot\textsc{=nom} exist\textsc{-csl} \textsc{cop-neg=supp}\\
    \glt    ‘There are (pieces of) a carrot, so maybe (the pickles) are not (mine).’\hfill [Co: 101023\_01.txt]
    \ex\label{ex:11:1b}
    Using a conjunctive particle [= (4-20 b)]\\
    \glll   wanna honami-{\textbar}cjan{\textbar} naaja siccjunban, naakjaa jumɨnu naaja sijandoojaa.\\
            [wan=ja honami-cjan naa=ja sij-tur-n=ban]\textsubscript{Adverbial clause} naakjaa jumɨ=nu naa=ja sij-an=doo=jaa\\
            1SG\textsc{=top} Honami\textsc{-dim} name\textsc{=top} know\textsc{-prog-ptcp=advrs} \textsc{2pl.hon.adnz} daughter.in.law\textsc{=gen} name\textsc{=top} know\textsc{-neg=ass=sol}\\
    \glt    ‘I know Honami’s name, but don’t know the name of your daughter in law.’\hfill [Co: 110328\_00.txt]
    \z
\z

All of the converbal affixes and some of the conjunctive particles are restricted in their choice of tense markers. However, a few conjunctive particles, i.e. \textit{ban} (ADVRS), \textit{kara} (CSL) and \textit{mun} (ADVRS), are not restricted in their choice of tense markers.

It is common in Yuwan that the adverbial clauses (especially including \textit{{}-tɨ} (SEQ)) are used sequentially, which is called clause-chaining (cf. \citealt{Payne1997}: 321-325). In that case, the adverbial clauses do not seem to be embedded in the superordinate clauses as adverbs, and it is natural to translate the meanings of the relations among the clauses into ‘and then’ as in (11-2).

\tablefirsthead{}

\tabletail{}
\tablelasttail{}
\begin{tabularx}{\textwidth}{XXXXXXXXXXXX}
\lsptoprule
{ (11\nobreakdash-\stepcounter{Remark}{\theRemark})} & \multicolumn{11}{X}{{ Clause-chaining in Yuwan [= (8-102 b)]}}\\
& { TM:} & { idocjɨ} & \multicolumn{2}{X}{{ jˀicjɨ,}} & \multicolumn{3}{X}{{ (an)}} & { mata} & \multicolumn{2}{X}{{ (an)}} & { agan}\\
&  & { [\textit{ido=ccjɨ}} & \multicolumn{2}{X}{{ \textit{jˀ-tɨ}]\textsubscript{Adverbial clause}}} & \multicolumn{3}{X}{{\itshape a-n}} & {\itshape mata} & \multicolumn{2}{X}{{\itshape a-n}} & { [\textit{aga-n}}\\
&  & { oh=QT} & \multicolumn{2}{X}{{ say-SEQ}} & \multicolumn{3}{X}{{ DIST-ADNZ}} & { again} & \multicolumn{2}{X}{{ DIST-ADNZ}} & { DIST-ADVZ}\\
&  & \multicolumn{2}{X}{{ izjibatɨ}} & \multicolumn{3}{X}{{ izjɨ,}} & \multicolumn{3}{X}{{ amanan}} & \multicolumn{2}{X}{{ sawakotankja}}\\
&  & \multicolumn{2}{X}{{\itshape izir-i+bar-tɨ}} & \multicolumn{3}{X}{{ \textit{ik-tɨ}]\textsubscript{Adverbial clause}}} & \multicolumn{3}{X}{{ [\textit{a-ma=nan}}} & \multicolumn{2}{X}{{\itshape sawako-taa=nkja}}\\
&  & \multicolumn{2}{X}{{ go.out-INF+?-SEQ}} & \multicolumn{3}{X}{{ go-SEQ}} & \multicolumn{3}{X}{{ DIST-place=LOC1}} & \multicolumn{2}{X}{{ Sawako-PL=APPR}}\\
&  & \multicolumn{4}{X}{{ minakotankjaga}} & \multicolumn{6}{X}{{ wutattu,}}\\
&  & \multicolumn{4}{X}{{\itshape minako-taa=nkja=ga}} & \multicolumn{6}{X}{{ \textit{wur-tar-tu}]\textsubscript{Adverbial clause}}}\\
&  & \multicolumn{4}{X}{{ Minako-PL=APPR=NOM}} & \multicolumn{6}{X}{{ exist-PST-CSL}}\\
&  & \multicolumn{10}{X}{{ ‘Saying that “Oh!” (I) went out there again, and there were Sawako, Minako and their friends, so ...’}}\\
&  & \multicolumn{10}{X}{\raggedleft [Co: 101020\_01.txt]}\\
\lspbottomrule
\end{tabularx}
Interestingly, some clauses headed by converbs can be used without their superordinate clauses. The conventionalized omission of the superordinate clauses is called “insubordination” (see §\ref{bkm:Ref365191910} for more details).

\subsection{Adnominal clause}
\label{bkm:Ref306757474}\hypertarget{RefHeadingToc395697253}{}\label{bkm:Ref366707692}
The adnominal clause is the subordinate clause that functions as an adnominal. The adnominal clause always precedes its head nominal. The predicate of the adnominal clause is always filled by the participles that end with \textit{{}-n} (PTCP) as in (11-3 a) or \textit{{}-an} (NEG) as in (11-3 b) (see §\ref{bkm:Ref358583589} for more details), but not vice versa since the participle followed by the conjunctive particles function as the adverbial clauses as in (11-1 b) in §\ref{bkm:Ref367266587} (see also §\ref{bkm:Ref367401418}).

\tablefirsthead{}

\tabletail{}
\tablelasttail{}
\begin{tabularx}{\textwidth}{XXXXXXXXXXXXX}
\lsptoprule
{ (11\nobreakdash-\stepcounter{Remark}{\theRemark})} & \multicolumn{12}{X}{{ Adnominal clauses in Yuwan}}\\
& \multicolumn{12}{X}{{ a. Using the participial affix \textit{{}-n} (PTCP) [= (8-80 a)]}}\\
& { TM:} & \multicolumn{2}{X}{{ sakkiija}} & \multicolumn{2}{X}{{ (hinzjaa)}} & \multicolumn{2}{X}{{ xxx}} & \multicolumn{2}{X}{{ hinzjaaba}} & \multicolumn{2}{X}{{ succjun}} & \\
&  & \multicolumn{2}{X}{{\itshape sakkii=ja}} & \multicolumn{2}{X}{{\itshape hinzjaa}} & \multicolumn{2}{X}{} & \multicolumn{2}{X}{{ [\textit{hinzjaa=ba}}} & \multicolumn{2}{X}{{ \textit{sukk-tur-n}]\textsubscript{Adnominal clause}}} & \\
&  & \multicolumn{2}{X}{{ a.short.while.ago}} & \multicolumn{2}{X}{{ goat}} & \multicolumn{2}{X}{} & \multicolumn{2}{X}{{ goat=ACC}} & \multicolumn{2}{X}{{ pull-PROG-PTCP}} & \\
&  & \multicolumn{3}{X}{{ cˀjunu}} & \multicolumn{6}{X}{{ atooradu}} & { cˀjanmun.} & \\
&  & \multicolumn{3}{X}{{\itshape cˀju=nu}} & \multicolumn{6}{X}{{\itshape atu=kara=du}} & {\itshape k-tar-n=mun} & \\
&  & \multicolumn{3}{X}{{ person=NOM}} & \multicolumn{6}{X}{{ after=ABL=FOC}} & { come-PST-PTCP=ADVRS} & \\
&  & \multicolumn{10}{X}{{ ‘A short while ago, the person who was pulling a goat came afterward, but (this time he came beforehand).’}} & \\
&  & \multicolumn{10}{X}{\raggedleft [PF: 090827\_02.txt]} & \\
& \multicolumn{11}{X}{{ b. Using the participial affix \textit{{}-an} (NEG) [= (8-83 b)]}} & \\
& { TM:} & { kˀwaga} & \multicolumn{4}{X}{{ dɨkɨran}} & \multicolumn{2}{X}{{ cˀju}} & \multicolumn{3}{X}{{ natɨ,}} & \\
&  & { [\textit{kˀwa=ga}} & \multicolumn{4}{X}{{ \textit{dɨkɨr-an}]\textsubscript{Adnominal clause}}} & \multicolumn{2}{X}{{\itshape cˀju}} & \multicolumn{3}{X}{{\itshape nar-tɨ}} & \\
&  & { child=NOM} & \multicolumn{4}{X}{{ be.born-NEG}} & \multicolumn{2}{X}{{ person}} & \multicolumn{3}{X}{{ COP-SEQ}} & \\
&  & \multicolumn{10}{X}{{ ‘Since (the woman) was a person who cannot have a baby, ...’}} & \\
&  & \multicolumn{10}{X}{\raggedleft [Co: 120415\_00.txt]} & \\
\lspbottomrule
\end{tabularx}
If the constituent of a clause is focused by \textit{du} (FOC), the predicate-final verb may take the participle without the following head NP, which is called the focus construction (or “kakari-musubi”) (see §\ref{bkm:Ref365192124} for more details).

\subsection{Nominal clause}
\label{bkm:Ref365191598}\hypertarget{RefHeadingToc395697254}{}\label{bkm:Ref365541635}
The nominal clause is the subordinate clause that functions as a nominal. The nominal clause can be expressed in three ways. First, the nominal clause can be expressed by the compound. For example, \textit{mai} (OBL) is compounded with the preceding verbal stem: /ikimai/ \textit{ik-i+mai} (go-INF+OBL) ‘to have to go’ (see §\ref{bkm:Ref364279917} for more details) as in (11-4 a). Secondly, the nominal clause can be expressed by the infinitival affix \textit{{}-i}/\textit{{}-Ø} as in (11-4 b) (see §\ref{bkm:Ref364199344} for more details). Thirdly, the nominal clause can be expressed by the formal noun \textit{sɨ}, which can directly follow the bound verbal stem and forms a nominal clause as in (11-4 c) (see §\ref{bkm:Ref367143723} for more details).

\tablefirsthead{}

\tabletail{}
\tablelasttail{}
\begin{tabularx}{\textwidth}{XXXXXXXXXXXXXXX}
\lsptoprule
{ (11\nobreakdash-\stepcounter{Remark}{\theRemark})} & \multicolumn{14}{X}{{ Nominal clauses in Yuwan}}\\
& \multicolumn{14}{X}{{ a. Using a nominal compound [= (4-35 d)]}}\\
& { TM:} & \multicolumn{2}{X}{{ wanna}} & \multicolumn{5}{X}{{ urɨba}} & \multicolumn{6}{X}{{ kakimaidoo.}}\\
&  & \multicolumn{2}{X}{{\itshape wan=ja}} & \multicolumn{5}{X}{{ [\textit{u-rɨ=ba}}} & \multicolumn{6}{X}{{ \textit{kak-i+mai}]\textsubscript{Nominal clause}\textit{=doo}}}\\
&  & \multicolumn{2}{X}{{ 1SG=TOP}} & \multicolumn{5}{X}{{ MES-NLZ=ACC}} & \multicolumn{6}{X}{{ write-INF+OBL=ASS}}\\
&  & \multicolumn{13}{X}{‘I have to write it.’}\\
&  & \multicolumn{13}{X}{\raggedleft [El: 130816]}\\
& \multicolumn{14}{X}{{ b. Using an infinitive [= (8-113 a)]}}\\
& \multicolumn{14}{X}{[Context: Remembering the days when people send off the people who went to mainland Japan]}\\
& \multicolumn{2}{X}{TM:} & \multicolumn{4}{X}{umanan} & \multicolumn{3}{X}{sanbasinu} & \multicolumn{5}{X}{atɨ,}\\
& \multicolumn{2}{X}{} & \multicolumn{4}{X}{{\itshape u-ma=nan}} & \multicolumn{3}{X}{{\itshape sanbasi=nu}} & \multicolumn{5}{X}{{\itshape ar-tɨ}}\\
& \multicolumn{2}{X}{} & \multicolumn{4}{X}{MES-place=LOC1} & \multicolumn{3}{X}{pier=NOM} & \multicolumn{5}{X}{exist-SEQ}\\
& \multicolumn{2}{X}{} & \multicolumn{4}{X}{umantɨ} & \multicolumn{5}{X}{cɨkɨ} & \multicolumn{3}{X}{jatattu.}\\
& \multicolumn{2}{X}{} & \multicolumn{4}{X}{{ [\textit{u-ma=nantɨ}}} & \multicolumn{5}{X}{{ \textit{cɨkɨr-Ø}]\textsubscript{Nominal clause}}} & \multicolumn{3}{X}{{\itshape jar-tar-tu}}\\
& \multicolumn{2}{X}{} & \multicolumn{4}{X}{MES-place=LOC2} & \multicolumn{5}{X}{attach-INF} & \multicolumn{3}{X}{COP-PST-CSL}\\
& \multicolumn{2}{X}{} & \multicolumn{12}{X}{‘There is a pier there, and (the ship) came alongside there [lit. (the ship) was to dock there].’}\\
& \multicolumn{2}{X}{} & \multicolumn{12}{X}{\raggedleft [Co: 120415\_00.txt]}\\
& \multicolumn{13}{X}{{ c. Using the formal noun \textit{sɨ}}} & \\
& \multicolumn{13}{X}{{ [Context: Talking about the present author] = (6-13 a)}} & \\
& { TM:} & \multicolumn{3}{X}{{ an}} & \multicolumn{3}{X}{{ nɨsəə}} & \multicolumn{3}{X}{{ muccjɨ}} & \multicolumn{2}{X}{{ ikjusəə}} & { nun} & \\
&  & \multicolumn{3}{X}{{ [\textit{a-n}}} & \multicolumn{3}{X}{{\itshape nəɨsəə}} & \multicolumn{3}{X}{{\itshape mut-tɨ}} & \multicolumn{2}{X}{{ \textit{ik-jur=sɨ}]\textsubscript{Nominal clause} \textit{=ja}}} & {\itshape nuu=n} & \\
&  & \multicolumn{3}{X}{{ DIST-ADNZ}} & \multicolumn{3}{X}{{ young.man}} & \multicolumn{3}{X}{{ have-SEQ}} & \multicolumn{2}{X}{{ go-UMRK=FN=TOP}} & { what=any} & \\
&  & \multicolumn{4}{X}{{ nənba,}} & \multicolumn{8}{X}{{ jakkəə.}} & \\
&  & \multicolumn{4}{X}{{\itshape nə-an-ba}} & \multicolumn{8}{X}{{\itshape jakkəə}} & \\
&  & \multicolumn{4}{X}{{ exist-NEG-CSL}} & \multicolumn{8}{X}{{ trouble}} & \\
&  & \multicolumn{12}{X}{‘There is not anything [i.e. any food] the young man can take (for meals), so it’s pity.’} & \\
&  & \multicolumn{12}{X}{\raggedleft [Co: 101023\_01.txt]} & \\
\lspbottomrule
\end{tabularx}
All of the above strategies can make the nominal clause, but the degree of the nominal characteristic and the verbal characteristic (or “clause-hood”) is different from one another. Their differences are summrized in the following \tabref{tab:key:102}.

\begin{styleqwertya}
\label{bkm:Ref365475191}\hypertarget{RefHeadingToc395697367}{}\textmd{Table \stepcounter{Table}{\theTable}}\textmd{. Comparison among the clauses headed by} \textmd{\textit{mai}}\textmd{ (OBL),} \textmd{\textit{{}-i}}\textmd{/}\textmd{\textit{{}-Ø}}\textmd{ (INF), or} \textmd{\textit{sɨ}} (FN)
\end{styleqwertya}

\tablefirsthead{}

\tabletail{}
\tablelasttail{}
\begin{tabularx}{\textwidth}{XXXXX} &  & { \textit{mai} (OBL)} & { \textit{{}-i}/\textit{{}-Ø} (INF)} & { \textit{sɨ} (FN)}\\
\lsptoprule
\multicolumn{2}{X}{{ Nominal characteristics}} &  &  & \\
{ a.} & { May be follwed by the copula verbs} & { +} & { +} & { +}\\
{ b.} & { May be followed by case particles} & { {}-} & { (+)} & { +}\\
\multicolumn{2}{X}{{ Verbal characteristics (or “clause-hood”)}} &  &  & \\
{ c.} & { Retains the internal syntax} & { +} & { +} & { +}\\
{ d} & { May take the subject different from that of the superordinate clause} & { {}-} & { (+)} & { +}\\
\lspbottomrule
\end{tabularx}
Notes: (+) means that there are a few cases where \textit{{}-i}/\textit{{}-Ø} (INF) can satisfy the nominal/verbal characteristics.

About the nominal characteristics in \tabref{tab:key:102}, all of the nominal clauses headed by (the compound including) \textit{mai} (OBL), the infinitive, and \textit{sɨ} (FN) may be followed by the copula verbs. In this respect, they behave like nominals. However, the compound including \textit{mai} (OBL) cannot take any case particle. In other words, it cannot become an argument. Similarly, the infinitive cannot take any case particles with the exception of the nominative case \textit{ga} and the dative case 1 \textit{n} (see §\ref{bkm:Ref364199344} for more details). On the contrary, \textit{sɨ} (FN) has more freedom to take case particle than the others. Thus, the clause headed by \textit{sɨ} (FN) has more nominal characteristics than those headed by \textit{mai} (OBL) or \textit{{}-i}/\textit{{}-Ø} (INF). About the verbal characteristics in \tabref{tab:key:102}, all of the verbal stems that are followed by \textit{mai} (OBL), \textit{{}-i}/\textit{{}-Ø} (INF), and \textit{sɨ} (FN) may retain their internal syntax. In this respect, these words behave like verbs. However, the clause headed by (the compound including) \textit{mai} (OBL) cannot have its own subject different from the superordinate (i.e. modified) clause. The clause headed by the infinitive also cannot take its own subject with the exception of the case where the infitive takes \textit{n} (DAT1) as in (8-114) - (8-115) in \sectref{sec:key:8.4.4.2.} On the contrary, the clause headed by \textit{sɨ} (FN) can take its own subject different from the superordinate clause. Thus, the clause headed by \textit{sɨ} (FN) has more verbal characteristics (or “clause-hood”) than those headed by \textit{mai} (OBL) or \textit{{}-i}/\textit{{}-Ø} (INF). From another point of view, it is probable that the clause headed by \textit{sɨ} (FN) has the status sufficient to be called the nominal clause, but that the clauses headed by (the compound that includes) \textit{mai} (OBL) or the infinitives are better analyzed as the components of the complex predicate (with the copula verb in a single clause).

\subsection{Complement clause}
\label{bkm:Ref306757544}\hypertarget{RefHeadingToc395697255}{}
The complement clause in Yuwan is the subordinate clause that functions as a complement of the verbal predicate phrase (see §\ref{bkm:Ref357004771} about the complement slot). A complement clause ends with one of the utterance-final particles A, i.e. \textit{ccjɨ} (QT), \textit{ka} (DUB), \textit{gajaaroo} (DUB), and \textit{nən} ‘such as.’ I present an example of \textit{ccjɨ} in (11-5) (see \sectref{sec:key:10.4} for more details).

\tablefirsthead{}

\tabletail{}
\tablelasttail{}
\begin{tabularx}{\textwidth}{XXXXXX}
\lsptoprule
{ (11\nobreakdash-\stepcounter{Remark}{\theRemark})} & \multicolumn{5}{X}{{ Complement clause in Yuwan [= (10-63 c)]}}\\
& { TM:} & { isaburootaa,} & { tomokkotaaga} & { atai} & { jatancjɨ}\\
&  & { [\textit{isaburoo-taa}} & {\itshape tomokko-taa=ga} & {\itshape atai} & { \textit{jar-tar-n=ccjɨ}]\textsubscript{Complement clause}}\\
&  & { Isaburo-PL} & Tomohiko-PL=NOM & { 50.years.old} & { COP-PST-PTCP=QT}\\
&  & \multicolumn{4}{X}{{ jˀicjɨ,}}\\
&  & \multicolumn{4}{X}{{ \textit{jˀ{}-tɨ}}}\\
&  & \multicolumn{4}{X}{{ say-SEQ}}\\
&  & \multicolumn{4}{X}{{ ‘(People) said that Isaburo (and) Tomohiko were fifty years old, and ...’}}\\
&  & \multicolumn{4}{X}{\raggedleft [Co: 120415\_01.txt]}\\
\lspbottomrule
\end{tabularx}
Other examples of complement clauses were shown in (9-23 b-e) in \sectref{sec:key:9.1.2.1} and (9-39) in \sectref{sec:key:9.1.2.2.}

In fact, the clause followed by \textit{ccjɨ} (QT) is similar to the nominal clause (in §\ref{bkm:Ref365541635}), since it may be followed by the copula verb, may take the genitive case \textit{nu}, and can retain the internal syntax including its own subject (see §\ref{bkm:Ref367403220} for more details). However, I propose that the clause followed by \textit{ccjɨ} (QT) is different from the nominal clause since it does not take any argument case (i.e. the cases other than the genitive). In fact, the clause headed by (the compound including) \textit{mai} (OBL) does not take any argument case as well as the clause followed by \textit{ccjɨ} (QT). However, the former, i.e. the clause headed by \textit{mai} (OBL), only fills the predicate phrase of the superordinate clause, but the latter, i.e. the clause followed by \textit{ccjɨ} (QT), can (and frequently) fill the slot other than the head of the predicate phrase of the superordinate clasue. In other words, the clause followed by \textit{ccjɨ} (QT) fills the complement slot of the verbal predicate phrase. The components in the complement slot do not take any argument case since they are not the arguments of the clause (see §\ref{bkm:Ref357004771}). Thus, it is more appropriate to call the clause followed by \textit{ccjɨ} (QT) the “complement clause” (not the nominal clause).

\section{Insubordination}
\label{bkm:Ref365191910}\hypertarget{RefHeadingToc395697256}{}
Insubordination is defined by \citet[367]{Evans2007} as follows: “I will apply the term “insubordination” to \textit{the} \textit{conventionalized} \textit{main} \textit{clause} \textit{use} \textit{of} \textit{what,} \textit{on} \textit{prima} \textit{facie} \textit{grounds,} \textit{appear} \textit{to} \textit{be} \textit{formally} \textit{subordinate} \textit{clauses}” (italic in original). As \citet[367]{Evans2007} said, the insubordination is a phenomenon strongly related with the diachronic linguistic change. Therefore, it is probable that there is a case where the subordinate use is very rare and also the main-clause use dominats in the modern language. In fact, the affix \textit{{}-ɨba} (SUGS) in Yuwan is a good candidate for that (see §\ref{bkm:Ref367403924} for more details). In Yuwan, the omission of the main clause is very common, where the (meaning of the) omitted clause can be often restored by the context. However, there are a few cases where the restoring is difficult. In those cases, the predicates have gained some grammatical functions different from the functions in the original subordinate clauses. In the following sections, I will present four examples: \textit{-tɨ} (SEQ) in §\ref{bkm:Ref365546585}, \textit{{}-ba} (CSL) in §\ref{bkm:Ref365546592}, \textit{ccjɨ=joo} (QT=CFM1) in §\ref{bkm:Ref365990436}, and \textit{{}-an-boo} (NEG-CND) in §\ref{bkm:Ref366699636}.

\subsection{\textit{{}-tɨ} (SEQ) as insubordination}
\label{bkm:Ref365546585}\hypertarget{RefHeadingToc395697257}{}
Non-finite uses of the converbal affix \textit{-tɨ} (SEQ) are found in the adverbial clause expressing sequential meaning as in §\ref{bkm:Ref367394296} or in the auxiliary verb construction as in §\ref{bkm:Ref361002248}. However, there is a finite use of the converbal affix \textit{-tɨ} (SEQ), which expresses the past tense as in (11-6 a-c).

\tablefirsthead{}

\tabletail{}
\tablelasttail{}
\begin{tabularx}{\textwidth}{XXXXXXXXX}
\lsptoprule
{ (11\nobreakdash-\stepcounter{Remark}{\theRemark})} & \multicolumn{8}{X}{{ \textit{{}-tɨ} (SEQ) expressing the past tense as the insubordination}}\\
& { a.} & { TM:} & { naakjoo} & \multicolumn{4}{X}{{ injasainnja}} & { dantɨ}\\
&  &  & {\itshape naakja=ja} & \multicolumn{4}{X}{{\itshape inja-sa+ar-i=n=ja}} & {\itshape daa=nantɨ}\\
&  &  & { 2.HON.PL=TOP} & \multicolumn{4}{X}{{ small-ADJ+STV-INF=DAT1=TOP}} & { where=LOC2}\\
&  &  & { asɨbjutɨ?} & \multicolumn{4}{X}{} & \\
&  &  & { \textit{asɨb-jur-tɨ}} & \multicolumn{4}{X}{} & \\
&  &  & { play-UMRK-SEQ} & \multicolumn{4}{X}{} & \\
&  &  & \multicolumn{6}{X}{{ ‘Where did you used to play when (you) were in your childhood?’}}\\
&  &  & \multicolumn{6}{X}{\raggedleft [Co: 110328\_00.txt]}\\
& { b.} & { TM:} & \multicolumn{3}{X}{{ gazimarugɨɨnu}} & { sjantɨ} & \multicolumn{2}{X}{{ asɨbantɨ?}}\\
&  &  & \multicolumn{3}{X}{{\itshape gazimaru+kɨɨ=nu}} & {\itshape sja=nantɨ} & \multicolumn{2}{X}{{ \textit{asɨb-an-tɨ}}}\\
&  &  & \multicolumn{3}{X}{{ bayan.tree+tree=GEN}} & { under=LOC2} & \multicolumn{2}{X}{{ play-NEG-SEQ}}\\
&  &  & \multicolumn{6}{X}{{ ‘Didn’t you play under the banyan tree?’}}\\
&  &  & \multicolumn{6}{X}{\raggedleft [Co: 110328\_00.txt]}\\
& { c.} & { TM:} & \multicolumn{2}{X}{{ jadunkjoo}} & \multicolumn{4}{X}{{ akɨtɨdoo.}}\\
&  &  & \multicolumn{2}{X}{{\itshape jaduu=nkja=ja}} & \multicolumn{4}{X}{{ \textit{akɨr-tɨ=doo}}}\\
&  &  & \multicolumn{2}{X}{{ door=APPR=TOP}} & \multicolumn{4}{X}{{ open-SEQ=ASS}}\\
&  &  & \multicolumn{6}{X}{{ ‘(We) opened the doors (on New Year’s Eve in the old days).’}}\\
&  &  & \multicolumn{6}{X}{\raggedleft [Co: 111113\_02.txt]}\\
\lspbottomrule
\end{tabularx}
  In fact, the finite-form affix \textit{{}-tar} (PST) cannot appear in the interrogative clause (see also §\ref{bkm:Ref358674542}). In that case, \textit{{}-tɨ} (SEQ) is used to express the past tense as in (11-6 a-b). Therefore, the particle that expresses the polar question, e.g., \textit{na} (PLQ), cannot co-occur with \textit{{}-tar} (PST) as in (11-7 b), but can with \textit{{}-tɨ} (SEQ) as in (11-7 a).

\tablefirsthead{}

\tabletail{}
\tablelasttail{}
\begin{tabularx}{\textwidth}{XXXX}
\lsptoprule
{ (11\nobreakdash-\stepcounter{Remark}{\theRemark})} & \multicolumn{3}{X}{{ \textit{na} (PLQ) in the past tense}}\\
& { a.} & { TM:} & waatɨna?\\
&  &  & \textit{waar-tɨ=na}\\
&  &  & understand-SEQ=PLQ\\
&  &  & ‘(Did you) understand?’\\
&  &  & \raggedleft [El: 090830]\\
& { b.} & { TM:} & { *waatana?}\\
&  &  & {  \textit{waar-tar=na}}\\
&  &  & understand-PST=PLQ\\
&  &  & (Intended meaning) ‘(Did you) understand?’\\
&  &  & \raggedleft [El: 090830]\\
\lspbottomrule
\end{tabularx}
  It should be noted that \textit{{}-tar} (PST) can appear in the interrogative clasue if it is followed by \textit{{}-u} (PFC) as in (11-18 a-b) in §\ref{bkm:Ref366708117}, or if it is followed by \textit{{}-mɨ} (PLQ), although the combination of \textit{{}-tar-mɨ} (PST-PLQ) has not yet appeared in the text data (it only appears in elicitation). Additionally, if the alleged interrogative clause is used to express the speaker’s wondering to herself, \textit{{}-tar} (PST) can be used as in (11-8) (see also §\ref{bkm:Ref366686164}).

\tablefirsthead{}

\tabletail{}
\tablelasttail{}
\begin{tabularx}{\textwidth}{XXXX}
\lsptoprule
{ (11\nobreakdash-\stepcounter{Remark}{\theRemark})} & \multicolumn{3}{X}{{ \textit{nuu} ‘what’ co-occuring with \textit{{}-tar} (PST) because of \textit{kai} (DUB) [= (10-50)]}}\\
& \multicolumn{3}{X}{{ [Context: MS asked TM whether the place in the picture used to be called “Yubinhana.”]}}\\
& { TM:} & { nuucjɨga} & { jutakaijaa?}\\
&  & { \textit{nuu=ccjɨ=ga}} & { \textit{jˀ-jur-tar=kai=jaa}}\\
&  & { what=QT=FOC} & { call-UMRK-PST=DUB=SOL}\\
&  & \multicolumn{2}{X}{{ ‘(I) wonder what (people) used to call (the place).’}}\\
&  & \multicolumn{2}{X}{\raggedleft [Co: 120415\_00.txt]}\\
\lspbottomrule
\end{tabularx}
\subsection{\textit{{}-ba} (CSL) as the insubordination}
\label{bkm:Ref365546592}\hypertarget{RefHeadingToc395697258}{}
Non-finite uses of the converbal affix \textit{-ba} (CSL) are found in the adverbial clause expressing causal meaning as in §\ref{bkm:Ref359092748}. However, there is a finite use of the converbal affix \textit{-ba} (CSL), which expresses the speaker’s request to the hearer as in (11-9 a-c). In that case, \textit{{}-ba} (CSL) always appears in the AVC following the auxiliary verbs \textit{kurɨr-} (BEN) or \textit{taboor-} (BEN.HON).

\tablefirsthead{}

\tabletail{}
\tablelasttail{}
\begin{tabularx}{\textwidth}{XXXXXXXXXX}
\lsptoprule
{ (11\nobreakdash-\stepcounter{Remark}{\theRemark})} & \multicolumn{9}{X}{{ \textit{kurɨr-} (BEN) \textit{+-ba} (CSL)}}\\
& { a.} & { TM:} & \multicolumn{2}{X}{{ hanacjɨ}} & \multicolumn{3}{X}{{ kurɨppa.}} & \multicolumn{2}{X}{{ dooka.}}\\
&  &  & \multicolumn{2}{X}{{\itshape hanas-tɨ}} & \multicolumn{3}{X}{{ \textit{kurɨr-ba}}} & \multicolumn{2}{X}{{\itshape dooka}}\\
&  &  & \multicolumn{2}{X}{{ talk-SEQ}} & \multicolumn{3}{X}{{ BEN-CSL}} & \multicolumn{2}{X}{{ please}}\\
&  &  & \multicolumn{2}{X}{{ [Lex. verb}} & \multicolumn{3}{X}{{ Aux. verb]\textsubscript{AVC}}} & \multicolumn{2}{X}{}\\
&  &  & \multicolumn{7}{X}{{ ‘Please, talk (to me).’}}\\
&  &  & \multicolumn{7}{X}{\raggedleft [Co: 120415\_01.txt]}\\
& { b.} & { TM:} & { naa} & \multicolumn{3}{X}{{ hazimɨtɨ}} & \multicolumn{3}{X}{{ kurɨppajoo.}}\\
&  &  & {\itshape naa} & \multicolumn{3}{X}{{\itshape hazimɨr-tɨ}} & \multicolumn{3}{X}{{ \textit{kurɨr-ba=joo}}}\\
&  &  & { FIL} & \multicolumn{3}{X}{{ begin-SEQ}} & \multicolumn{3}{X}{{ BEN-CSL=CFM1}}\\
&  &  &  & \multicolumn{3}{X}{{ [Lex. verb}} & \multicolumn{3}{X}{{ Aux. verb]\textsubscript{AVC}}}\\
&  &  & \multicolumn{7}{X}{{ ‘(Please) begin (the training for the traditional dance for our community).’}}\\
&  &  & \multicolumn{7}{X}{\raggedleft [Co: 120415\_01.txt]}\\
& \multicolumn{9}{X}{{ \textit{taboor-} (BEN.HON) \textit{+-ba} (CSL)}}\\
& { c.} & { TM:} & \multicolumn{3}{X}{{ umoojaganaa,}} & \multicolumn{3}{X}{{ abɨtɨ}} & { tabooppajoo.}\\
&  &  & \multicolumn{3}{X}{{\itshape umoor-jaganaa}} & \multicolumn{3}{X}{{\itshape abɨr-tɨ}} & { \textit{taboor-ba=joo}}\\
&  &  & \multicolumn{3}{X}{{ come.HON-SIM}} & \multicolumn{3}{X}{{ call-SEQ}} & { BEN.HON-CSL=CFM1}\\
&  &  & \multicolumn{3}{X}{} & \multicolumn{3}{X}{{ [Lex. verb}} & { Aux. verb]\textsubscript{AVC}}\\
&  &  & \multicolumn{7}{X}{{ ‘Coming (here), call (the person for me please).’}}\\
&  &  & \multicolumn{7}{X}{\raggedleft [El: 120930]}\\
\lspbottomrule
\end{tabularx}
\subsection{\textit{ccjɨ=joo} (QT=CFM1) as the insubordination}
\label{bkm:Ref365990436}\hypertarget{RefHeadingToc395697259}{}
\textit{ccjɨ} (QT) embeds any utterance into the complement of the superordinate clause in principle. For example, an imperative clause is embedded into the complement of \textit{jˀ-} ‘say’ as in (11-10).

\tablefirsthead{}

\tabletail{}
\tablelasttail{}
\begin{tabularx}{\textwidth}{XXXXXX}
\lsptoprule
{ (11\nobreakdash-\stepcounter{Remark}{\theRemark})} & \multicolumn{5}{X}{{ \textit{ccjɨ} (QT) in the complement clause [= (8-148 g)]}}\\
& { TM:} & { kanɨcɨboja} & { urakja} & { tuikurawɨcjɨ} & { jˀicjɨ,}\\
&  & { [\textit{kanɨ+cɨbo=ja}} & {\itshape urakja} & { \textit{tur-i+kuraw-ɨ=ccjɨ}]\textsubscript{Complement clause}} & {\itshape jˀ-tɨ}\\
&  & { gold+pot=TOP} & { 2.NHON.PL} & { take-INF+DRG-IMP=QT} & { say-SEQ}\\
&  & \multicolumn{4}{X}{{ ‘(The man) said that, “You take (this) damn gold pot!” and ...’}}\\
&  & \multicolumn{4}{X}{\raggedleft [Fo: 090307\_00.txt]}\\
\lspbottomrule
\end{tabularx}
However, if it is followed by \textit{joo} (CFM1), it always expresses an objective (not hearsay) information without any superordinate clause as in (11-11).

\tablefirsthead{}

\tabletail{}
\tablelasttail{}
\begin{tabularx}{\textwidth}{XXXX}
\lsptoprule
{ (11\nobreakdash-\stepcounter{Remark}{\theRemark})} & \multicolumn{3}{X}{{ \textit{ccjɨ} (QT) in the insubordination [= (10-75 a)]}}\\
& \multicolumn{3}{X}{[Context: The speaker explains the story of the Pear Film to the hearer.]}\\
& TM: & tuutɨ & izjancjɨjoo.\\
&  & {\itshape tuur-tɨ} & { \textit{ik-tar-n=ccjɨ=joo}}\\
&  & pass-SEQ & go-PST-PTCP=QT=CFM1\\
&  & \multicolumn{2}{X}{‘(A young man who pulls a goat) passed by.’}\\
&  & \multicolumn{2}{X}{\raggedleft [PF: 090305\_01.txt]}\\
\lspbottomrule
\end{tabularx}
The more detail discussion was done in §\ref{bkm:Ref366668885}.

\subsection{\textit{{}-an-boo} (NEG-CND) as the pre-insubordination}
\label{bkm:Ref366699636}\hypertarget{RefHeadingToc395697260}{}
The converbal affix \textit{{}-boo} (CND) expresses the conditional meaning. Interestingly, the combination of \textit{{}-an-boo} (NEG-CND) in the adverbial clause and \textit{nar-an} (become-NEG) in the main clause expresses the obligative meaning as in (11-12), where the obligative meaning is expressed in the adverbial clause.

\tablefirsthead{}

\tabletail{}
\tablelasttail{}
\begin{tabularx}{\textwidth}{XXXXXX}
\lsptoprule
{ (11\nobreakdash-\stepcounter{Remark}{\theRemark})} & \multicolumn{5}{X}{{ Obligation expressed by \textit{{}-an-boo} (NEG-CND) plus \textit{nar-an} (become-NEG) [=(9-40)]}}\\
& TM: & waasan & ucjəə, & ganba & hatarakanboo,\\
&  & {\itshape waa-sa+ar-n} & {\itshape uci=ja} & {\itshape ganba} & { \textit{hatarak-an-boo}}\\
&  & young-ADJ+STV-PTCP & period=TOP & therefore & work-NEG-CND\\
&  & \multicolumn{4}{X}{naranbajaa.}\\
&  & \multicolumn{4}{X}{{ \textit{nar-an{}-ba=jaa}}}\\
&  & \multicolumn{4}{X}{become-NEG-CSL=SOL}\\
&  & \multicolumn{4}{X}{‘While (one) is young, (one) has to work.’}\\
&  & \multicolumn{4}{X}{\raggedleft [Co: 120415\_01.txt]}\\
\lspbottomrule
\end{tabularx}
The above collocation has an idiomatic meaning (i.e. obligation), and it is difficult to construct the meaning from the literal meaning of each morpheme. The idiomatic meaning is frequently expressed without the main clause, which is the “conventionalization of ellipsis” \citep[372-373]{Evans2007} as in (11-13 a-d).

\tablefirsthead{}

\tabletail{}
\tablelasttail{}
\begin{tabularx}{\textwidth}{XXXXXXXXXXXXXXXXXXX}
\lsptoprule
{ (11\nobreakdash-\stepcounter{Remark}{\theRemark})} & \multicolumn{18}{X}{{ Obligation expressed only by \textit{{}-an-boo} (NEG-CND)}}\\
& { a.} & \multicolumn{17}{X}{{ [= (8-122 b)]}}\\
&  & { TM:} & \multicolumn{3}{X}{{ nan}} & \multicolumn{9}{X}{{ umoorasanboocjɨ}} & \multicolumn{4}{X}{{ umutɨ,}}\\
&  &  & \multicolumn{3}{X}{{\itshape nan}} & \multicolumn{9}{X}{{ \textit{umoor-as-an-boo=ccjɨ}}} & \multicolumn{4}{X}{{\itshape umuw-tɨ}}\\
&  &  & \multicolumn{3}{X}{{ 2.HON.SG}} & \multicolumn{9}{X}{{ come.HON-CAUS-NEG-CND=QT}} & \multicolumn{4}{X}{{ think-SEQ}}\\
&  &  & \multicolumn{16}{X}{{ ‘(I) thought that (I) have to make you come, and ...’}}\\
&  &  & \multicolumn{16}{X}{\raggedleft [Co: 110328\_00.txt]}\\
& { b.} & \multicolumn{17}{X}{{ [= (10-33)]}}\\
&  & { TM:} & \multicolumn{2}{X}{{ jazin}} & \multicolumn{9}{X}{{ kjunmuncjɨ}} & \multicolumn{3}{X}{{ umutɨ}} & \multicolumn{2}{X}{{ kurɨranboo.}}\\
&  &  & \multicolumn{2}{X}{{\itshape jazin}} & \multicolumn{9}{X}{{\itshape k-jur-n=mun=ccjɨ}} & \multicolumn{3}{X}{{\itshape umuw-tɨ}} & \multicolumn{2}{X}{{ \textit{kurɨr-an-boo}}}\\
&  &  & \multicolumn{2}{X}{{ necessarily}} & \multicolumn{9}{X}{{ come-UMRK-PTCP=ADVRS=QT}} & \multicolumn{3}{X}{{ think-SEQ}} & \multicolumn{2}{X}{{ BEN-NEG-CND}}\\
&  &  & \multicolumn{16}{X}{{ ‘(You) have to think that necessarily (you) will come.’}}\\
&  &  & \multicolumn{16}{X}{\raggedleft [Co: 101023\_01.txt]}\\
& { c.} & \multicolumn{17}{X}{{ [= (4-57)]}}\\
&  & { TM:} & { ude,} & \multicolumn{3}{X}{{ naa,}} & \multicolumn{3}{X}{{ ganboo,}} & \multicolumn{3}{X}{{ urakjoo}} & \multicolumn{3}{X}{{ ude,}} & \multicolumn{2}{X}{{ ude,}} & { kamanboo,}\\
&  &  & {\itshape ude} & \multicolumn{3}{X}{{\itshape naa}} & \multicolumn{3}{X}{{\itshape ganboo}} & \multicolumn{3}{X}{{\itshape urakja=ja}} & \multicolumn{3}{X}{{\itshape ude}} & \multicolumn{2}{X}{{\itshape ude}} & { \textit{kam-an-boo}}\\
&  &  & { well} & \multicolumn{3}{X}{{ FIL}} & \multicolumn{3}{X}{{ if.so}} & \multicolumn{3}{X}{{ 2.NHON.SG=TOP}} & \multicolumn{3}{X}{{ well}} & \multicolumn{2}{X}{{ well}} & { eat-NEG-CND}\\
&  &  & \multicolumn{5}{X}{{ udeccjɨdu}} & \multicolumn{3}{X}{{ xxx}} & \multicolumn{8}{X}{{ jutattujaa.}}\\
&  &  & \multicolumn{5}{X}{{\itshape ude=ccjɨ=du}} & \multicolumn{3}{X}{{ N/A}} & \multicolumn{8}{X}{{\itshape jˀ-jur-tar-tu=jaa}}\\
&  &  & \multicolumn{5}{X}{{ well=QT=FOC}} & \multicolumn{3}{X}{{ N/A}} & \multicolumn{8}{X}{{ say-UMRK-PST-CSL=SOL}}\\
&  &  & \multicolumn{16}{X}{{ ‘(The old people) would say, ‘Well, now, then, you have to eat (more).’’}}\\
&  & \multicolumn{17}{X}{\raggedleft [Co: 120415\_01.txt]}\\
& { d.} & { TM:} & \multicolumn{6}{X}{{ uraba}} & \multicolumn{3}{X}{{ həəku}} & \multicolumn{7}{X}{{ tɨmɨranbooccjɨga.}}\\
&  &  & \multicolumn{6}{X}{{\itshape ura=ba}} & \multicolumn{3}{X}{{\itshape həə-ku}} & \multicolumn{7}{X}{{ \textit{tɨmɨr-an-boo=ccjɨ=ga}}}\\
&  &  & \multicolumn{6}{X}{{ 2.NHON.SG=ACC}} & \multicolumn{3}{X}{{ quick-ADVZ}} & \multicolumn{7}{X}{{ find-NEG-CND=QT=FOC}}\\
&  &  & \multicolumn{16}{X}{{ ‘(I think) that (I) have to find you quickly.’}}\\
&  &  & \multicolumn{16}{X}{\raggedleft [Co: 101023\_01.txt]}\\
\lspbottomrule
\end{tabularx}
In the above examples, \textit{{}-an-boo} (NEG-CND) expresses obligation without \textit{nar-an} (become-NEG). In other words, the subordiante clauses headed by (the verb that includes) \textit{{}-an-boo} (NEG-CND) has obtained the grammatical meaning of obligation.

\section{Focus construction (or “Kakari-musubi”)}
\label{bkm:Ref365192124}\hypertarget{RefHeadingToc395697261}{}
It is famous that there are a kind of focus constructions (i.e. constructions that include focus particles) that are traditoinally called \textit{kakari-musubi} (i.e. ‘government-predication’) in Japanese linguistics and Ryukyuan linguistics (cf. \citealt{Shimoji2008}: 565-570). The characteristics of the focus constructions in Yuwan can be summarized as follows.

(11\nobreakdash-\stepcounter{Remark}{\theRemark})   Focus construction (or “Kakari-musubi”) in Yuwan

a.   \textit{{}-n} (PTCP) is in the predicate of the main clause

> \textit{du} (FOC) is in the clause, but not vice versa;

b.   \textit{{}-u} (PFC) is in the predicate

> \textit{du} (FOC) or an interrogative word is in the clause, but not vice versa.

The argumentation for (11-14) is shown in the following sections. First, I will present examples of the focus construction of \textit{du} (FOC) in §\ref{bkm:Ref366708110}. Then, I will present examples of the focus construction of \textit{ga} (FOC) in §\ref{bkm:Ref366708117}.

\subsection{Focus construction of \textit{du} (FOC)}
\label{bkm:Ref366708110}\hypertarget{RefHeadingToc395697262}{}
In Yuwan, the participle that has \textit{{}-n} (PTCP) fills the predicate of the adnominal clause, and it cannot fill the predicate of the main claue in principle (see also §\ref{bkm:Ref366707692}). However, if the focus particle \textit{du} appears in the same clause, the participle can fill the predicate of the main claue as in (11-15 a-d).

\tablefirsthead{}

\tabletail{}
\tablelasttail{}
\begin{tabularx}{\textwidth}{XXXXXXXXXXX}
\lsptoprule
{ (11\nobreakdash-\stepcounter{Remark}{\theRemark})} & \multicolumn{10}{X}{{ \textit{du} (FOC) co-occuring with \textit{{}-n} (PTCP) in the main clause}}\\
& { a.} & \multicolumn{9}{X}{{ [= (6-108 a)]}}\\
&  & { TM:} & \multicolumn{2}{X}{{ nuunu}} & \multicolumn{4}{X}{{ nangikaicjɨdu}} & \multicolumn{2}{X}{{ umujun.}}\\
&  &  & \multicolumn{2}{X}{{\itshape nuu=nu}} & \multicolumn{4}{X}{{ \textit{nangi=kai=ccjɨ=du}}} & \multicolumn{2}{X}{{ \textit{umuw-jur-n}}}\\
&  &  & \multicolumn{2}{X}{{ what=GEN}} & \multicolumn{4}{X}{{ trouble=DUB=QT=FOC}} & \multicolumn{2}{X}{{ think-UMRK-PTCP}}\\
&  &  & \multicolumn{8}{X}{{ ‘(I) wonder what (kinds) of trouble (I took).’}

{ [i.e. ‘I didn’t want to take such trouble.’]}}\\
&  &  & \multicolumn{8}{X}{\raggedleft [Co: 120415\_01.txt]}\\
& { b.} & { TM:} & \multicolumn{3}{X}{{ kadɨdu,}} & \multicolumn{2}{X}{{ cikjaranu}} & \multicolumn{3}{X}{{ izijun.}}\\
&  &  & \multicolumn{3}{X}{{ \textit{kam-tɨ=du}}} & \multicolumn{2}{X}{{\itshape cikjara=nu}} & \multicolumn{3}{X}{{ \textit{izir-jur-n}}}\\
&  &  & \multicolumn{3}{X}{{ eat-SEQ=FOC}} & \multicolumn{2}{X}{{ power=NOM}} & \multicolumn{3}{X}{{ go.out-UMRK-PTCP}}\\
&  &  & \multicolumn{8}{X}{{ ‘(One) eat (food), and then the power goes out.’}

{ [i.e. ‘One can become powerful after eating a meal.’]}}\\
&  &  & \multicolumn{8}{X}{\raggedleft [Co: 120415\_01.txt]}\\
& { c.} & { TM:} & \multicolumn{7}{X}{{ dujasankutubəidu}} & { siccjun.}\\
&  &  & \multicolumn{7}{X}{{ \textit{duja-sa+ar-n=kutu=bəi=du}}} & { \textit{sij-tur-n}}\\
&  &  & \multicolumn{7}{X}{{ rich-ADJ+STV-PTCP=fact=only=FOC}} & { know-PROG-PTCP}\\
&  &  & \multicolumn{8}{X}{{ ‘(I) know only the fact that (your grandparents) were rich.’}}\\
&  & \multicolumn{9}{X}{\raggedleft [Co: 120415\_01.txt]}\\
& { d.} & \multicolumn{9}{X}{{ [Context: TM has been taught to chew her food well, but her stomach was not good until two or three years before.]}}\\
&  & { TM:} & { naa,} & \multicolumn{3}{X}{{ kunugurudu}} & \multicolumn{4}{X}{{ jiccjan.}}\\
&  &  & {\itshape naa} & \multicolumn{3}{X}{{ \textit{kunuguru=du}}} & \multicolumn{4}{X}{{ \textit{jiccj-sa+ar-n}}}\\
&  &  & { FIL} & \multicolumn{3}{X}{{ recently=FOC}} & \multicolumn{4}{X}{{ good-ADJ+STV-PTCP}}\\
&  &  & \multicolumn{8}{X}{{ ‘(My stomach) has been good recently.’}}\\
&  & \multicolumn{9}{X}{\raggedleft [Co: 120415\_01.txt]}\\
\lspbottomrule
\end{tabularx}
The above examples show that \textit{-n} (PTCP) can fill the predicate of the main clause if there is \textit{du} (FOC) in the same clause. However, its opposite is not necessarily true. For example, \textit{{}-u} (PFC) can also fill the predicate of the main clause if there is \textit{du} (FOC) in the same clause as in (11-16 a-b).

\tablefirsthead{}

\tabletail{}
\tablelasttail{}
\begin{tabularx}{\textwidth}{XXXXXXXXXXXX}
\lsptoprule
{ (11\nobreakdash-\stepcounter{Remark}{\theRemark})} & \multicolumn{11}{X}{{ \textit{du} (FOC) co-occuring with \textit{{}-u} (PFC) [= (8-77 a)]}}\\
& { a.} & { TM:} & \multicolumn{4}{X}{{ utuzjoobasanna}} & \multicolumn{2}{X}{{ un}} & \multicolumn{2}{X}{{ cˀjunu}} & { samisjentudu}\\
&  &  & \multicolumn{4}{X}{{\itshape utuzjo+obasan=ja}} & \multicolumn{2}{X}{{\itshape u-n}} & \multicolumn{2}{X}{{\itshape cˀju=nu}} & { \textit{samisjen=tu=du}}\\
&  &  & \multicolumn{4}{X}{{ Utujo+old.woman=TOP}} & \multicolumn{2}{X}{{ MES-ADNZ}} & \multicolumn{2}{X}{{ person=GEN}} & { samisen=COM=FOC}\\
&  &  & { utoo} & \multicolumn{2}{X}{{ (sii..)}} & \multicolumn{4}{X}{{ sɨrarɨɨru.}} & \multicolumn{2}{X}{}\\
&  &  & {\itshape uta=ja} & \multicolumn{2}{X}{{\itshape sɨr-i}} & \multicolumn{4}{X}{{ \textit{sɨr-arɨr-u}}} & \multicolumn{2}{X}{}\\
&  &  & { song=TOP} & \multicolumn{2}{X}{{ do-INF}} & \multicolumn{4}{X}{{ do-CAP-PFC}} & \multicolumn{2}{X}{}\\
&  &  & \multicolumn{9}{X}{{ ‘Utujo can sing a song [lit. do a song] just with that person’s samisen. (Otherwise, she cannot sing a song.)’}}\\
&  &  & \multicolumn{9}{X}{\raggedleft [Co: 120415\_00.txt]}\\
& { b.} & { TM:} & \multicolumn{2}{X}{{ tacuu{\textbar}toka{\textbar}ga}} & \multicolumn{3}{X}{{ juubadu,}} & \multicolumn{4}{X}{{ jˀarɨɨru.}}\\
&  &  & \multicolumn{2}{X}{{\itshape tacuu=toka=ga}} & \multicolumn{3}{X}{{ \textit{jˀ-ba=du}}} & \multicolumn{4}{X}{{ \textit{jˀ-arɨr-u}}}\\
&  &  & \multicolumn{2}{X}{{ Tatsu=APPR=NOM}} & \multicolumn{3}{X}{{ say-CSL=FOC}} & \multicolumn{4}{X}{{ say-CAP-PFC}}\\
&  &  & \multicolumn{9}{X}{{ ‘(People) can say (a piece of advice to her), since (it is) Tatsu (who) says (it). (Otherwise, no one can give any advice to her.)’}}\\
&  &  & \multicolumn{9}{X}{\raggedleft [Co: 101023\_01.txt]}\\
\lspbottomrule
\end{tabularx}
  Furtheremore, other inflectional affixes (or affix-like clitics) can co-occur with \textit{du} (FOC) in the same clause as in (11-17 a-g).

\tablefirsthead{}

\tabletail{}
\tablelasttail{}
\begin{tabularx}{\textwidth}{XXXXXXXXXXXXX}
\lsptoprule
{ (11\nobreakdash-\stepcounter{Remark}{\theRemark})} & \multicolumn{12}{X}{{ a. \textit{du} (FOC) co-occuring with \textit{{}-i} (NPST)}}\\
& \multicolumn{12}{X}{{ [Context: Mutsu went away saying that she would stop in an electric appliance store.]}}\\
& { TM:} & \multicolumn{2}{X}{{ muccuuja}} & \multicolumn{5}{X}{{ jaakacidu}} & \multicolumn{4}{X}{{ izjəijaa.}}\\
&  & \multicolumn{2}{X}{{\itshape muccuu=ja}} & \multicolumn{5}{X}{{ \textit{jaa=kaci=du}}} & \multicolumn{4}{X}{{ \textit{ik-təər-i=jaa}}}\\
&  & \multicolumn{2}{X}{{ Mutsu=TOP}} & \multicolumn{5}{X}{{ house=ALL=FOC}} & \multicolumn{4}{X}{{ go-RSL-NPST=SOL}}\\
&  & \multicolumn{11}{X}{{ ‘Mutsu has gone (back) home.’}}\\
&  & \multicolumn{11}{X}{\raggedleft [Co: 110328\_00.txt]}\\
& \multicolumn{12}{X}{{ b. \textit{du} (FOC) co-occuring with \textit{doo} (ASS)}}\\
& \multicolumn{12}{X}{{ [Context: TM said that there were no people who were able to make a wooden boat in Yuwan.]}}\\
& { TM:} & \multicolumn{3}{X}{{ kusinandu}} & \multicolumn{8}{X}{{ wutattoo.}}\\
&  & \multicolumn{3}{X}{{ \textit{kusi=nan=du}}} & \multicolumn{8}{X}{{ \textit{wur-tar=doo}}}\\
&  & \multicolumn{3}{X}{{ Kushi=LOC1=FOC}} & \multicolumn{8}{X}{{ exist-PST=ASS}}\\
&  & \multicolumn{11}{X}{{ ‘(People who can make a wooden boat) were in Kushi.’}}\\
&  & \multicolumn{11}{X}{\raggedleft [Co: 111113\_01.txt]}\\
& \multicolumn{12}{X}{{ c. \textit{du} (FOC) co-occuring with \textit{{}-tar} (PST) [= (8-134 a)]}}\\
& { TM:} & \multicolumn{2}{X}{{ kunugurudu}} & \multicolumn{6}{X}{{ kurəə}} & \multicolumn{2}{X}{{ mucjɨ\footnotemark{}}} & { kjuuta.}\\
&  & \multicolumn{2}{X}{{\itshape kunuguru=du}} & \multicolumn{6}{X}{{\itshape ku-rɨ=ja}} & \multicolumn{2}{X}{{\itshape mut-tɨ}} & {\itshape k-jur-ta}\\
&  & \multicolumn{2}{X}{{ recently=FOC}} & \multicolumn{6}{X}{{ PROX-NLZ=TOP}} & \multicolumn{2}{X}{{ have-SEQ}} & { come-UMRK-PST}\\
&  & \multicolumn{11}{X}{{ ‘(Satsue’s child) brought this (picture) recently.’}}\\
&  & \multicolumn{11}{X}{\raggedleft [Co: 120415\_00.txt]}\\
& \multicolumn{12}{X}{{ d. \textit{du} (FOC) co-occuring with \textit{{}-ba} (CSL) or \textit{-tɨ} (SEQ) [= (10-9 c)]}}\\
& { TM:} & \multicolumn{5}{X}{{ naa{\textbar}nihon{\textbar}bəidu}} & \multicolumn{4}{X}{{ appa,}} & \multicolumn{2}{X}{{ {\textbar}hacikiro{\textbar}naadu}}\\
&  & \multicolumn{5}{X}{{ \textit{naa+nihon=bəi=du}}} & \multicolumn{4}{X}{{ \textit{ar-ba}}} & \multicolumn{2}{X}{{ \textit{hacikiro+naa=du}}}\\
&  & \multicolumn{5}{X}{{ another+two.CLF=about=FOC}} & \multicolumn{4}{X}{{ exist-CSL}} & \multicolumn{2}{X}{{ eight.kilogram+each=FOC}}\\
&  & { kinmɨ} & \multicolumn{3}{X}{{ sjɨ,}} & \multicolumn{7}{X}{{ haatɨ,}}\\
&  & {\itshape kinmɨ} & \multicolumn{3}{X}{{ \textit{sɨr-tɨ}}} & \multicolumn{7}{X}{{ \textit{haar-tɨ}}}\\
&  & { measure} & \multicolumn{3}{X}{{ do-SEQ}} & \multicolumn{7}{X}{{ measure-SEQ}}\\
&  & \multicolumn{11}{X}{{ ‘There are other two white radishes, so (one) measures eight kilograms (of the materials) for each, and ...’}}\\
&  & \multicolumn{11}{X}{\raggedleft [Co: 101023\_01.txt]}\\
& \multicolumn{12}{X}{{ e. \textit{du} (FOC) co-occuring with \textit{{}-tu} (CSL)}}\\
& { TM:} & \multicolumn{3}{X}{{ kamɨccjɨdu}} & \multicolumn{8}{X}{{ jutattu.}}\\
&  & \multicolumn{3}{X}{{ \textit{kam-ɨ=ccjɨ=du}}} & \multicolumn{8}{X}{{ \textit{jˀ-tar-tu}}}\\
&  & \multicolumn{3}{X}{{ eat-IMP=QT=FOC}} & \multicolumn{8}{X}{{ say-PST-CSL}}\\
&  & \multicolumn{11}{X}{{ ‘(The people in the past) said (roughly to children), “Eat!”’}}\\
&  & \multicolumn{11}{X}{\raggedleft [Co: 120415\_01.txt]}\\
& \multicolumn{12}{X}{{ f. \textit{du} (FOC) co-occuring with \textit{{}-i} (INF)}}\\
& { TM:} & \multicolumn{6}{X}{{ iccjaijaacjɨdu}} & \multicolumn{5}{X}{{ umuii.}}\\
&  & \multicolumn{6}{X}{{ \textit{jiccj}\footnotemark{}\textit{{}-sa+ar-i=jaa=ccjɨ=du}}} & \multicolumn{5}{X}{{ \textit{umuw-i}}}\\
&  & \multicolumn{6}{X}{{ good-ADJ+STV-NPT=QT=FOC}} & \multicolumn{5}{X}{{ think-INF}}\\
&  & \multicolumn{11}{X}{{ ‘(I) think that (it) is good.’}}\\
&  & \multicolumn{11}{X}{\raggedleft [Co: 120415\_01.txt]}\\
\lspbottomrule
\end{tabularx}
\addtocounter{footnote}{-2}
\stepcounter{footnote}\footnotetext{ \textit{mut-tɨ} (have-SEQ) usually becomes /muccjɨ/ according to the rule in \sectref{sec:key:8.3.1.2.} However, it becomes /mucjɨ/ in this example.}
\stepcounter{footnote}\footnotetext{ \textit{jiccj-sa} (good-ADJ) usually becomes /jiccja/ [itt͡ɕɑ̟], but it becomes /iccja/ [ʔitt͡ɕɑ̟] in this example.}

The above examples show that \textit{du} (FOC) does not necessarily induce \textit{{}-n} (PTCP) or \textit{{}-u} (PFC) in the predicate in the same clause. \textit{du} (FOC) can occur not only in the main clause, but also in the adverbial clause as in (11-17 d). Furthermore, \textit{du} (FOC) can occur in the adnominal clause in the literal meaning (i.e. the clasue that modifies an NP in effect) as in (10-9 d) in \sectref{sec:key:10.1.2.1.}

\subsection{Focus construction of \textit{ga} (FOC)}
\label{bkm:Ref366708117}\hypertarget{RefHeadingToc395697263}{}
The finite-form affix \textit{{}-u} (PFC) only appears in the clauses that include \textit{du} (FOC) or in the interrogative clauses of information question (see also §\ref{bkm:Ref358844654}). The interrogative words are often followed by \textit{ga} (FOC) (see also §\ref{bkm:Ref347247904}). I will present examples of \textit{{}-u} (PFC) co-occuring with \textit{ga} (FOC) as in (11-18 a-d). The examples of \textit{-u} (PFC) co-occuring with \textit{du} (FOC) were already shown in (11-16) in §\ref{bkm:Ref366708110}.

\tablefirsthead{}

\tabletail{}
\tablelasttail{}
\begin{tabularx}{\textwidth}{XXXXXXXXXXXXX}
\lsptoprule
{ (11\nobreakdash-\stepcounter{Remark}{\theRemark})} & \multicolumn{12}{X}{{ \textit{ga} (FOC) co-occuring with \textit{{}-u} (PFC) and the interrogative word}}\\
& { a.} & \multicolumn{11}{X}{{ [Context: TM was surprised that US brought a lot of foods to TM’s house.] = (6-101 a)}}\\
&  & { TM:} & \multicolumn{5}{X}{{ nunkjabaga}} & { mata} & \multicolumn{3}{X}{{ muccjɨ}} & { moocjaru?}\\
&  &  & \multicolumn{5}{X}{{ \textit{nuu=nkja=ba=ga}}} & {\itshape mata} & \multicolumn{3}{X}{{\itshape mut-tɨ}} & { \textit{moor-tar-u}}\\
&  &  & \multicolumn{5}{X}{{ what=APPR=ACC=FOC}} & { again} & \multicolumn{3}{X}{{ have-SEQ}} & { HON-PST-PFC}\\
&  &  & \multicolumn{10}{X}{{ ‘What did (you) bring (here) again?’}}\\
&  &  & \multicolumn{10}{X}{\raggedleft [Co: 110328\_00.txt]}\\
& { b.} & { TM:} & { nuu} & \multicolumn{3}{X}{{ sjɨga,}} & \multicolumn{6}{X}{{ asɨbjutaru?}}\\
&  &  & {\itshape nuu} & \multicolumn{3}{X}{{ \textit{sɨr-tɨ=ga}}} & \multicolumn{6}{X}{{ \textit{asɨb-jur-tar-u}}}\\
&  &  & { what} & \multicolumn{3}{X}{{ do-SEQ=FOC}} & \multicolumn{6}{X}{{ play-UMRK-PST-PFC}}\\
&  &  & \multicolumn{10}{X}{{ ‘What did (you) do for play (in your childhood)?’ [lit. ‘Doing what, did (you) play?’]}}\\
&  &  & \multicolumn{10}{X}{\raggedleft [Co: 110328\_00.txt]}\\
& { c.} & { TM:} & \multicolumn{3}{X}{{ kurəə}} & \multicolumn{5}{X}{{ nuu{\textbar}sjooten{\textbar}cjɨga}} & \multicolumn{2}{X}{{ kacjəəru?}}\\
&  &  & \multicolumn{3}{X}{{\itshape ku-rɨ=ja}} & \multicolumn{5}{X}{{ \textit{nuu+sjooten=ccjɨ=ga}}} & \multicolumn{2}{X}{{ \textit{kak-təər-u}}}\\
&  &  & \multicolumn{3}{X}{{ PROX-NLZ=TOP}} & \multicolumn{5}{X}{{ what+shop=QT=FOC}} & \multicolumn{2}{X}{{ write-RSL-PFC}}\\
&  &  & \multicolumn{10}{X}{{ ‘What was written on this shop(’s signboard in the picture)?’}

{ [lit. ‘What shop have (people) written on this?’]}}\\
&  &  & \multicolumn{10}{X}{\raggedleft [Co: 120415\_00.txt]}\\
& { d.} & { TM:} & \multicolumn{2}{X}{{ nuucjɨga}} & \multicolumn{5}{X}{{ arɨboo}} & \multicolumn{3}{X}{{ juru?}}\\
&  &  & \multicolumn{2}{X}{{ \textit{nuu=ccjɨ=ga}}} & \multicolumn{5}{X}{{\itshape a-rɨ=ba=ja}} & \multicolumn{3}{X}{{ \textit{jˀ-jur-u}}}\\
&  &  & \multicolumn{2}{X}{{ what=QT=FOC}} & \multicolumn{5}{X}{{ DIST-NLZ=ACC=TOP}} & \multicolumn{3}{X}{{ say-UMRK-PFC}}\\
&  &  & \multicolumn{10}{X}{{ ‘What is that person called?’ [i.e. ‘What is his name?’]}}\\
&  &  & \multicolumn{10}{X}{\raggedleft [Co: 120415\_00.txt]}\\
\lspbottomrule
\end{tabularx}
In (11-18 a-d), \textit{{}-u} (PFC) co-occurs with \textit{ga} (FOC). However, the existense of \textit{ga} (FOC) does not induce that of \textit{{}-u} (PFC). For example, \textit{ga} (FOC) in the (alleged) interrogative clause can appear without \textit{-u} (PFC) if it is followed by \textit{kai} (DUB) as in (11-8) in §\ref{bkm:Ref365546585}. Moreover, \textit{ga} (FOC) can be used in the non-interrogative clauses, where \textit{ga} (FOC) does not take \textit{{}-u} (FOC) as in (11-19) (see §\ref{bkm:Ref367405185} for more details).

\tablefirsthead{}

\tabletail{}
\tablelasttail{}
\begin{tabularx}{\textwidth}{XXXXXXX}
\lsptoprule
{ (11\nobreakdash-\stepcounter{Remark}{\theRemark})} & \multicolumn{6}{X}{{ \textit{ga} (FOC) not co-occuring with \textit{-u} (PFC) [= (10-14 b)]}}\\
& { TM:} & { kunəədaga} & { waakja} & { dusinu,} & { asikendusinu,} & { wutɨ,}\\
&  & { \textit{kunəəda=ga}} & {\itshape waakja-a} & {\itshape dusi=nu} & {\itshape asiken+dusi=nu} & { \textit{wur-tɨ}}\\
&  & { the.other.day=FOC} & 1PL-ADNZ & { friend=NOM} & { Ashiken+frend=NOM} & { exist{}-SEQ}\\
&  & \multicolumn{5}{X}{{ ‘The other day, there is my friend, (i.e.) a friend in Ashiken, and ...’}}\\
&  & \multicolumn{5}{X}{\raggedleft [Co: 120415\_00.txt]}\\
\lspbottomrule
\end{tabularx}
In the above example, \textit{ga} (FOC) co-occurs with \textit{{}-tɨ} (SEQ).

