\chapter{Verbal morphology}\label{chap:8}

The verbal morphology of Yuwan is agglutinative; it begins with a root, which is followed by an affix (or affixes) (see \sectref{sec:key:8.1}). There is no number (or gender) agreement between arguments and verbs in Yuwan. Inflectional morphology of Yuwan is not straightforward; a certain gruop of inflectional affixes cannot directly follow the verbal root, but always take a group of derivational affixes (see \sectref{sec:key:8.1}). The verbal morphology of Yuwan is rich in morphophonological alternation (see \sectref{sec:key:8.2}). The clausal types, i.e. main clause, adnominal (or relative) clause, nominal clause, and adverbilal clause, can be expressed by the word-final inflectional affix. For example, a clause ending with \textit{{}-ɨ} (IMP) is a main clause, but a clause ending with \textit{-n} (PTCP) (and without any focus on another constituent in the same clause) is an adnominal clause (see \sectref{sec:key:8.4}). Regarding tense, aspect, and modality, each of them can be expressed by verbal affixes, although they can be expressed by other morphosyntactic means. Tense affixes have the opposition of non-past vs past. Aspectual affixes express progressive, resulative, non-progressive, or habitual (see \sectref{sec:key:8.5.1.4} - \sectref{sec:key:8.5.1.6}). Modality is grammaticalized as a restricted set of mood affixes, e.g. the suppositional affix \textit{{}-oo}. However, it typically surfaces in the tense affixes; the tense marker \textit{-tar} (PST) (in the finite-form use) expresses the speaker’s confidence in the factuality of the event (see \sectref{sec:key:8.4.1.1}).

\section{The structure of the verb}
\label{bkm:Ref303739828}
The verb has the structure as in (8-1), which begins with a root and ends with an inflectional affix. Roughly speaking, the initial root and the final inflectional affix are obligatory, and the medial affixes are all optional; more details are explained later. In the following displays, the braces mean that the affixes in the same vertical column cannot appear simultaneously; for example, \textit{{}-tur} (PROG) and \textit{{}-jur} (UMRK) cannot appear simultaneously.

(8\nobreakdash-1)  Structure of the verb

  Root  \textit{{}-as  {}-arɨr} %%[Warning: Draw object ignored]
\textit{{}-tuk  {}-arɨr  {}-tur  {}-jawur} %%[Warning: Draw object ignored]
\textit{{}-an  {}-təər  {}-tar  {}-} Inflectional affix

    CAUS  PASS  PRPR  CAP  PROG  POL  NEG  RSL  PST  

          \textit{{}-jur} 

          UMRK    

There are some restrictions concerning their combinations. The impossible combinations are summarized below, where “impossible combinations” means that the combinations have not appeared in my texts, or that the present author cannot find proper contexts for the questions in elicitation.

(8\nobreakdash-2)  Impossible combinations

  a.  *\textit{{}-arɨr} (PASS)   +  \textit{{}-arɨr} (CAP) 

  b.  *\textit{{}-arɨr} (PASS)   +  \textit{{}-jur} (UMRK)

  c.  *\textit{{}-tuk} (PRPR)  +  \textit{{}-tur} (PROG)

  d.  *\textit{{}-tuk} (PRPR)  +  \textit{{}-tar} (PST)

  e.  *\textit{{}-jawur} (POL)  +  \textit{{}-təər} (RSL)

The possibility of combinations described above is about the one composed of two derivational affixes. The combination composed of more than two derivational affixes is not so common in the text corpus, and to find proper contexts to investigate such a combination is so difficult that their possibility is not clear so far.

  In the top of this section, I said the word-final inflectional affix in a verb is obligatory but that the preceding affixes are optional; however, the morphology of Yuwan is a little more complicated. The word-final inflectional affixes in Yuwan can be categorized into two distinct groups, one of which cannot directly follow the verbal root, and also cannot follow \textit{{}-as} (CAUS) or \textit{{}-tuk} (PRPR), and obligatorily needs a certain affix as in (8-3 b) to precede.

(8\nobreakdash-3)  Inflectional affixes

  a.  Group I: Can directly follow the verbal root

    Finite-form affixes  : \textit{{}-oo} (INT), \textit{{}-ɨ} (IMP), \textit{{}-na} (PROH), \textit{{}-ɨba} (SUGS), \textit{{}-azɨi} (NEG.PLQ), \textit{{}-tar} (PST)

    Participial affix  : \textit{{}-an} (NEG)

    Converbal affixes  : \textit{{}-ba} (CSL), \textit{{}-boo} (CND), \textit{{}-tɨ} (SEQ), \textit{{}-təəra} ‘after’, \textit{{}-tai} (LST),

\textit{{}-jagacinaa} (SIM), \textit{{}-gadɨ} ‘until’

    Infinitival affix  : \textit{{}-i}/\textit{{}-Ø} (INF)

  b.  Group II: Cannot directly follow the verbal root

    Finite-form affixes  : \textit{{}-i} (NPST), \textit{{}-oo} (SUPP), \textit{{}-mɨ} (PLQ), \textit{{}-sa} (POL), \textit{{}-sɨga} (POL), \textit{{}-u} (PFC)

    Participial affix  : \textit{{}-n} (PTCP)

    Converbal affixes  : \textit{{}-tu} (CSL), \textit{{}-too} (CSL), \textit{{}-nən} (SEQ)

On the one hand, Group-I affixes can directly follow the verbal root; on the other hand, Group-II affixes cannot, but need another affix to precede. The minimal combinations with the above two types of inflectional affixes are shown below.

(8\nobreakdash-4)  Minimal combinations

  a.  Group I                

    Root  {}-  Affix        e.g.  /turoo/  \textit{tur-oo}  (take-INT)  ‘will take’

  b.  Group II                

    Root  {}-  Affix  {}-  Affix    e.g.  /tujui/  \textit{tu-jur-i}  (take-UMRK-NPST)  ‘take’

The non-past affixe \textit{{}-i} in Group-II affixes cannot follow the verbal root directly: */tui/ \textit{tur-i} (take-NPST) is not permitted. The affixes required by Group-II affixes are shown below, where non-relevant affixes are deleted by double lines.

(8\nobreakdash-5)  Affixes needed by Group-II affixes

  Root  \textit{{}-as  {}-arɨr} %%[Warning: Draw object ignored]
\textit{{}-tuk  {}-arɨr  {}-tur  {}-jawur} %%[Warning: Draw object ignored]
\textit{{}-an  {}-təər  {}-tar  {}-} Inflectional affixes

    CAUS  PASS  PRPR  CAP  PROG  POL  NEG  RSL  PST  (Group II)

          \textit{{}-jur} 

          UMRK    

The above arrangement shows that if the word-final affix belongs to the Group-II affixes in (8-3 b), one of the following affixes must precede them: \textit{{}-arɨr} (PASS), \textit{-arɨr} (CAP), \textit{-tur} (PROG), \textit{{}-jawur} (POL), \textit{{}-jur} (UMRK), \textit{{}-an} (NEG), \textit{{}-təər} (RSL), or \textit{{}-tar} (PST). However, three kinds of verbal roots, i.e. the existential verbal root, the copula verbal root, and the stative verbal root, can take Group-II affixes directly (see \sectref{sec:key:8.3.5}). It should be noted that there are some restrictions on the combinations between these affixes in (8-5) and Group II inflectional affixes. For example, there is no combination made of \textit{{}-an} (NEG) plus \textit{{}-i} (NPST). The possible combinations between derivational affixes and inflectional affixes will be shown in \sectref{sec:key:8.4.}

There are two special affixes: \textit{{}-an} (NEG) and \textit{-tar} (PST). In (8-1), they are in non-word-final positions. They can, however, stand in a word-final position without any inflectional affix. For example, /turan/ \textit{tur-an} (take-NEG) ‘don’t take,’ and /tuta/ \textit{tur-tar} (take-PST) ‘took.’ In other words, I propose that \textit{{}-an} (NEG) and \textit{-tar} (PST) can behave similarly with the inflectional affixes in (8-3), which is shown in (8-6). They are underlined below.

(8\nobreakdash-6)  a.  Ending with \textit{{}-an} (NEG)

    Root  \textit{{}-as  {}-arɨr  {}-tuk  {}-arɨr  {}-tur  {}-jawur  {}-an}   

      CAUS  PASS  PRPR  CAP  PROG  POL  NEG    

  b.  Ending with \textit{{}-tar} (PST)

    Root  \textit{{}-as  {}-arɨr} %%[Warning: Draw object ignored]
\textit{{}-tuk}  \textit{{}-arɨr  {}-tur  {}-jawur} %%[Warning: Draw object ignored]
\textit{{}-an  {}-təər  {}-tar}

      CAUS  PASS  PRPR  CAP  PROG  POL  NEG  RSL  PST

            \textit{{}-jur} 

            UMRK  

\textit{{}-an} (NEG) and \textit{{}-tar} (PST) in word-final positions can be regarded as Group-I affixes since they can directly follow verbal roots. It should be noted that these affixes “can” finish a verb. Therefore, they are free to finish the verbal string, and can continue it. For example, \textit{{}-an} (NEG) can be followed by \textit{{}-ba} (CSL), or \textit{{}-tar} (PST) can be followed by \textit{{}-oo} (SUPP): /turanba/ \textit{tur-an-ba} (take-NEG-CSL) ‘because (someone) does not take’ and /tutaroo/ \textit{tur-tar-oo} (take-PST-SUPP) ‘may have taken.’ In fact, the above analysis in (8-6) suggests that there are no zero inflectional affixes that follow \textit{{}-an} (NEG) or \textit{{}-tar} (PST). In other words, we do not accept the analysis that presupposes zero inflectional affixes as in (8-7), where “...” means that there are several more candidates of inflectional affixes.

(8\nobreakdash-7)  Analysis not to be accepted

    Derivational affixes  Inflectional affixes

  Root  \textit{{}-as  {}-arɨr} %%[Warning: Draw object ignored]
\textit{{}-tuk  {}-arɨr  {}-tur  {}-jawur} %%[Warning: Draw object ignored]
\textit{{}-an} %%[Warning: Draw object ignored]
\textit{{}-təər} %%[Warning: Draw object ignored]
\textit{{}-tar  {}-Ø} (ASS)

    CAUS  PASS  PRPR  CAP  PROG  POL  NEG  RSL  PST  \textit{{}-oo} (SUPP)

          \textit{{}-jur    {}-i}/\textit{{}-Ø  {}-n}/\textit{{}-Ø} (PTCP)

          UMRK    NPST  \textit{{}-sɨga} (POL)

                ...

The above table shows that the tense contrast is expressed in the penultimate slot of the verb: \textit{{}-tar} (PST) vs. \textit{{}-i}/\textit{{}-Ø} (NPST). Additionally, new zero affixes are postulated in the final slot of the verb, i.e. \textit{{}-Ø} (ASS) and \textit{{}-Ø} (PTCP). In this analysis, the final and penultimate slots would be inflectional. We do not take this zero-affix analysis, because of the following two reasons. First, the analysis postulates the zero affix \textit{{}-Ø} (ASS), which does not have any non-zero form. This kind of zero morpheme is less convincing than another zero morpheme that has a non-zero form, e.g, \textit{{}-i}/\textit{{}-Ø} (NPST) or \textit{{}-n}/\textit{{}-Ø} (PTCP) (cf. \citealt{Haas1974}: 49). Second, if we accept this analysis, there appears a case where we have to recognize a distinction between non-visible zero affixes, i.e. \textit{{}-Ø} (ASS) and \textit{{}-Ø} (PTCP) as in (8-8 a-b).

(8\nobreakdash-8)  Negative polarity

  a.  Main clause

    wanna  amanu  ziija  jumarandoo.

    \textit{wan=ja}  \textit{a-ma=nu}  \textit{zii=ja}  \textit{jum-ar-an-Ø-Ø=doo}

    1SG-TOP  DIST-place=GEN  character=TOP  read-CAP-NEG-NPST-ASS=ASS

    ‘I cannot read the Chinese character there.’

    [El: 130821]

  b.  Adnominal clause

    uraga  jumaran  ziija  dɨruu?

    \textit{ura=ga}  \textit{jum-ar-an-Ø-Ø}  \textit{zii=ja}  \textit{dɨ-ru}

    2.NHON.SG  read-CAP-NEG-NPST-PTCP  character=TOP  which-NLZ

    ‘Which is the Chinese character that you cannot read?’

    [El: 130821]

  Affirmative polarity

  c.  Main clause

    wanna  amanu  ziigadəə  jumarɨttoo.

    \textit{wan=ja}  \textit{a-ma=nu}  \textit{zii=gadɨ=ja}  \textit{jum-arɨr-Ø-Ø=doo}

    1SG-TOP  DIST-place=GEN  character=LMT=TOP  read-CAP-NPST-ASS=ASS

    ‘I can read the Chinese character there.’

    [El: 130821]

  d.  Adnominal clause

    uraga  jumarɨn  ziija  dɨruu?

    \textit{ura=ga}  \textit{jum-arɨr-Ø-n}  \textit{zii=ja}  \textit{dɨ-ru}

    2.NHON.SG  read-CAP-NPST-PTCP  character=TOP  which-NLZ

    ‘Which is the Chinese character that you can read?’

    [El: 130821]

The examples (8-8 a, c) express the verbal forms in the predicates of the main clauses (in negative and affirmative polarity). The examples (8-8 b, d) express the verbal forms in the predicates of the adnominal clauses (in negative and affirmative polarity). The verbal forms in (8-8 a-b) are the same /jumaran/, and their differences are expressed only by the underlying two different zero morphemes, i.e. \textit{{}-Ø} (ASS) in (8-8 a) and \textit{{}-Ø} (PTCP) in (8-8 b). Such a nonvisible opposition is called “distinction of indiscernibles” \citep[36]{Haas1974}, and it was said that “within a set of paradigmatic contrasts distinction of indiscernibles is inadmissible” \citep[83]{McGregor2003}. In fact, we can avoid this “distinction of indiscernibles” by postulating \textit{{}-n} (PTCP) in (8-8 b). In that case, the verb form /jumaran/ is analyzed as \textit{jum-ar-an-Ø-n} (read-CAP-NEG -NPST-PTCP). However, this analysis needs another morphophonological rule, where \textit{{}-an} (NEG) becomes /-a/ before \textit{{}-n} (PTCP). This rule is irregular, since the ordinary measure to avoid /n.n/ sequence in Yuwan is a vowel insertion (see \sectref{sec:key:2.4.3}). Therefore, we do not take the zero-morpheme analysis as in (8-7), and admit special kinds of affixes that can both close and continue the verbal stems, i.e. \textit{{}-an} (NEG) and \textit{{}-tar} (PST). The word-final use of \textit{{}-tar} (PST) will be discussed in \sectref{sec:key:8.4.1.1.} The word-final use of \textit{{}-an} (NEG) will be discussed in \sectref{sec:key:8.4.2.2.} The non-word-final use of these affixes will be discussed in \sectref{sec:key:8.5.1.9.}

  All of the above verbal affixes are summarized as in \tabref{tab:key:55} using the inflectional criteria as in (8-9).

(8\nobreakdash-9)  Inflectional criteria

  A.  Appears only in the word-final position;

  B.  Can finish a word without another preceding affix;

  C.  Relevant to syntactic finiteness.

In (8-9), A and C have some relations with the features of inflection recognized in the languages of the world \citep[90]{Haspelmath2010}.

\begin{styleBeschriftung}
\textmd{\tabref{tab:key:55}}\textmd{. Inflectional affixes and derivational affixes of verbs}
\end{styleBeschriftung}

  A  B  C  Examples

Inflectional affixes        

  Group I  +  +  +  \textit{{}-oo} (INT), \textit{{}-ɨ} (IMP), \textit{{}-na} (PROH), \textit{{}-ɨba} (SUGS), \textit{{}-azɨi} (NEG.PLQ),

\textit{{}-ba} (CSL), \textit{{}-boo} (CND), \textit{{}-tɨ} (SEQ), \textit{{}-təəra} ‘after’, \textit{{}-tai} (LST),

\textit{{}-jagacinaa} (SIM), \textit{{}-gadɨ} ‘until’

  Group II  +  {}-  +  \textit{{}-i} (NPST), \textit{{}-oo} (SUPP), \textit{{}-mɨ} (PLQ), \textit{{}-sa} (POL), \textit{{}-sɨga} (POL),

\textit{{}-u} (PFC), \textit{{}-n} (PTCP), \textit{{}-tu} (CSL), \textit{{}-too} (CSL), \textit{{}-nən} (SEQ)

  (Group I)  {}-  +  +  \textit{{}-an} (NEG), \textit{-tar} (PST), \textit{{}-i}/\textit{{}-Ø} (INF)

Derivational affixes  {}-  {}-  +  \textit{{}-arɨr} (PASS), \textit{-arɨr} (CAP), \textit{-tur} (PROG), \textit{{}-təər} (RSL), \textit{{}-jawur} (POL), \textit{{}-jur} (UMRK)

  {}-  {}-  {}-  \textit{{}-as} (CAUS), \textit{{}-tuk} (PRPR)

Note: The infinitival affixes \textit{{}-i}/\textit{{}-Ø} can appear in the word-internal position of compounds (see \sectref{sec:key:4.2.3.1}). Therefore, they cannot fulfill the criterion A in (8-9).

Group-I \& Group-II affixes appear only in the word-final position (8-9 A) with the exception of \textit{{}-an} (NEG), \textit{{}-tar} (PST), and \textit{{}-i}/\textit{{}-Ø} (INF). Only Group-I affixes and \textit{{}-an} (NEG) and \textit{-tar} (PST) can finish a verb without another preceding affix (8-9 B). As mentioned in the beginning of this chapter, the verbal form in the predicate determines the clausal type. In other words, all of the Group-I affixes, Group-II affixes, \textit{{}-an} (NEG), and \textit{-tar} (PST) are relevant to syntactic finiteness. Additionally, the affixes in the fourth row of \tabref{tab:key:55}, i.e. \textit{{}-arɨr} (PASS), \textit{-arɨr} (CAP), \textit{-tur} (PROG), \textit{{}-təər} (RSL), \textit{{}-jawur} (POL), and \textit{{}-jur} (UMRK) (also with \textit{{}-an} (NEG) and \textit{{}-tar} (PST)) are necessarily required by Group-II affixes. Thus, those affixes are also relevant to syntactic finiteness. We will call the affixes which satisfy two or more criteria of (8-9) “inflectional affixes,” and the other remained affixes “derivational affixes” in the verbal morphology. It should be noted that the productivity among the above verbal affixes is not so much different from one another. For example, the derivational affix \textit{{}-jur} (UMRK) can follow no less verbal roots than the inflectional affix \textit{{}-ɨ} (IMP) can. Therefore, the term “derivational” does not imply less productivity, at least for verbal affixes, in this grammar.

  Additionally, it should be mentioned that certain clitics are very similar to Group II inflectional affixes, i.e. the affix-like clitics (see \sectref{sec:key:4.2.2.2}): \textit{sɨ} (FN), \textit{doo} (ASS), \textit{ka} (DUB), \textit{kai} (DUB), \textit{kamo} (POS), \textit{ga} (CFM3), and \textit{gajaaroo} (DUB). These clitics fill the final slot of the verb, which is usually filled by inflectional affixes as in (8-1), and the clitics cannot follow a verbal root directly (except for \textit{kai} (DUB)), and need one of the affixes in (8-5) in order for them to follow a verbal stem.

  In the following sections, the morphophonology of verbs will be discussed in \sectref{sec:key:8.2.} The special types of verbal stems that have some morphological, syntactical, and semantical characteristics will be discussed in \sectref{sec:key:8.3.} The verbal inflectional morphology will be discussed in \sectref{sec:key:8.4} The verbal derivational morphology will be discussed in \sectref{sec:key:8.5.}

\section{Morphophonology of verbs}
\label{bkm:Ref356245430}\subsection{Rules for verbal roots and affixes}

In this section, we examine the morphophonological rules needed in order to correctly produce the output verbal forms. A complete list of the possible combinations of roots, derivational affixes, and inflectional affixes are shown in appendix. Morphophonology of infinitives will be discussed in another section (see \sectref{sec:key:8.4.4.1}). Additionally, the morphophonological rule of \textit{{}-tar} (PST) and \textit{{}-mɨ} (PLQ) will be discussed in each section (see \sectref{sec:key:8.4.1.1} and \sectref{sec:key:8.4.1.4}).

  Verbal affixes can be grouped into four (morphophonological) types, chiefly distinguished by their initial morphophonemes. In \tabref{tab:key:56}, the four types disregard the differences between derivational affixes and inflectinal affixes, or the syntax-related differences among inflectional affixes (i.e. finite-form affixes or converbal affixes).

\begin{styleBeschriftung}
\textmd{\tabref{tab:key:56}}\textmd{. Four types of verbal affixes (or clitics)}
\end{styleBeschriftung}

Types  Main characteristics  All examples

A.  vowel-initial  \textit{{}-an} (NEG), \textit{{}-arɨr} (PASS), \textit{{}-as} (CAUS), \textit{-azɨi} (NEG.PLQ), \textit{{}-ɨ} (IMP),

\textit{{}-ɨba} (SUGS), \textit{-oo}(INT), \textit{-oo} (SUPP)

B.  \textit{t}{}-initial  \textit{{}-tar} (PST), \textit{{}-tuk} (PRPR), \textit{{}-tur} (PROG), \textit{{}-təər} (RSL), \textit{{}-tɨ} (SEQ), \textit{{}-tai} (LST),

\textit{{}-təəra} ‘after’

C.  deletion of the prededing non-nasal resonants  \textit{{}-jawur} (POL), \textit{{}-jaa} ‘person,’ \textit{{}-jur} (UMRK), \textit{{}-jagacinaa} (SIM), \textit{{}-mɨ} (PLQ), \textit{{}-n} (PTCP), \textit{sɨ} (FN)

D.  assimilation;

vowel insertion  \textit{{}-ba} (CSL), \textit{{}-boo} (CND), \textit{{}-gadɨ} ‘until,’ \textit{{}-na} (PROH), \textit{{}-sa} (POL), \textit{{}-sɨga} (POL), \textit{{}-too} (CSL), \textit{{}-tu} (CSL), \textit{doo} (ASS) , \textit{ka} (DUB), \textit{kai} (DUB), \textit{kamo} (POS), \textit{ga} (CFM3), \textit{gajaaroo} (DUB)

Each type of affix needs a different set of (morpho)phonological rules to output the correct surface forms (see \sectref{sec:key:8.2.1.1} - \sectref{sec:key:8.2.1.4}).

The verbal stems are distinguished into 17 types, determined by their final morphophonemes (except for the irregular types). The types of verbal stems are shown below with a few examples.

\begin{styleBeschriftung}
\textmd{\tabref{tab:key:57}}\textmd{. 17 types of verbal stems}
\end{styleBeschriftung}

No.  Stem-final morphophonemes  Examples

1.  V\textsubscript{non-back}r  \textit{hingir-} ‘escape,’ \textit{abɨr-} ‘call,’ \textit{kəər-} ‘exchange’

2.  V\textsubscript{back}r, V\textsubscript{back}w  \textit{tur-} ‘take,’ \textit{umuw-} ‘think,’ \textit{nuuw-} ‘sew,’ \textit{kˀuur-}/\textit{kˀuuw-} ‘close,’ \textit{nugoor-} ‘don’t do,’ \textit{koor-}/\textit{koow-}/\textit{kawur-} ‘buy,’ \textit{wa}(\textit{k})\textit{ar-} ‘understand’

3.  pp  \textit{app-} ‘play’

4.  b  \textit{narab-} ‘line up,’ \textit{asɨb-} ‘paly’

5.  Vm  \textit{jum-} ‘read,’ \textit{kam-} ‘eat,’ \textit{num-} ‘drink’

6.  nm  \textit{tanm-} ‘ask,’ \textit{cɨnm-} ‘wrap’

7.  V\textsubscript{non-}\textit{\textsubscript{i} }k  \textit{kak-} ‘write,’ \textit{maruk-} ‘bandle’

8.  V\textsubscript{non-}\textit{\textsubscript{i} }kk  \textit{sukk-} ‘draw,’ \textit{mukk-} ‘bring’

9.  Vs  \textit{us-} ‘push,’ \textit{kˀjoos-} ‘break’

10.  ss  \textit{kuss-} ‘kill’

11.  t  \textit{ut-} ‘hit,’ \textit{mat-} ‘wait,’ \textit{kat-} ‘win’

12.  \$C(G)  \textit{jˀ-} ‘say,’\footnote{The word-initial glottalization of \textit{jˀ-} ‘say’ is frequently weakened to become /j/.} \textit{mj-} ‘see’

13.  ij  \textit{kij-} ‘cut,’ \textit{kij-} ‘put on (clothes),’ \textit{kˀubij-} ‘tie,’ \textit{hasij-} ‘run’

14.  V\textsubscript{non-}\textit{\textsubscript{i} }g  \textit{tug-} ‘whet,’ \textit{hag-} ‘peel’

15.  ik  \textit{kik-} ‘hear,’ \textit{sik-} ‘spread’

16.  i(n)g  \textit{uig-} ‘swim,’ \textit{ming-} ‘grasp’

17.  in  \textit{sin-} ‘die,’ \textit{ikin-} ‘live’

Notes:

(a) “V\textsubscript{non-back}” indicates the non-back vowels //i, ɨ, ə//, “V\textsubscript{back}” indicates the back vowels //u, o, a//, “V\textsubscript{non-}\textit{\textsubscript{i}}” indicates vowels excluding //i//, and “\$” represents a word boundary;

(b) The verbal roots ending with //ir// are \textit{hingir-} ‘escape,’ \textit{izir-} ‘go out,’ and \textit{ubuir-} ‘memorize.’ \textit{izir-} ‘go out’ may be pronounced as \textit{izjɨr}, although the former is preferred over the latter. These roots do not go through the \textit{j}{}-insertion rule that is described in \sectref{sec:key:8.2.1.1}, which may imply that historically the final //i// of these verbal stems is different from that of the other verbal stems (e.g. \textit{kik-} ‘hear’ or \textit{sin-} ‘die’);

\begin{styleFootnote}
(c) \textit{kˀuur-} ‘close’ may alternate with \textit{kˀuuw-}, and \textit{koor-} ‘buy’ may alternate with \textit{koow-} or \textit{kawur-}. In addition, \textit{oor-} ‘meet’ may alternate with \textit{oow-}. However, \textit{nugoor-} ‘don’t do’ does not have any other underlying form.
\end{styleFootnote}

Each type of verbal stem undergoes a different application of morphophonological rules according to the four types of verbal affixes (or clitics) in \tabref{tab:key:56}. The examples in \tabref{tab:key:58} illustrate the different results caused by the applications of different morphophonological rules. The morpheme boundaries at the surface form level are shown in some of the following examples.

\begin{styleBeschriftung}
\textmd{\tabref{tab:key:58}}\textmd{. Different applications of rules to verbal stems and affixes showing their surface forms}
\end{styleBeschriftung}

    Affix types

    A. vowel-initial  B. \textit{t}{}-initial  C. deletion  D. others

No.  Stems’ final  e. g.  {}-an  {}-ta  {}-jur  {}-na

1.  V\textsubscript{non-back}r    {}-an  Ø-ta  Ø-jur  C\textit{\textsubscript{i} }{}-na

2.  V\textsubscript{back}r, V\textsubscript{back}w    {}-an  Ø-ta  Ø-jur  C\textit{\textsubscript{i} }{}-na

3.  pp    {}-an  C\textit{\textsubscript{i} }Ø-ta  {}-jur  {}-una

4.  b    {}-an  Ø-da  {}-jur  {}-una

5.  Vm    {}-an  Ø-da  {}-jur  {}-na

6.  nm    {}-an  Ø-da  {}-jur  {}-una

7.  V\textsubscript{non-}\textit{\textsubscript{i}} k    {}-an  Ø-cja  {}-jur  {}-una

8.  V\textsubscript{non-}\textit{\textsubscript{i}} kk    {}-an  C\textit{\textsubscript{i} }Ø-cja  {}-jur  {}-una

9.  Vs    {}-an  Ø-cja  {}-jur  {}-ɨna

10.  ss    {}-an  C\textit{\textsubscript{i} }Ø-cja  {}-jur  {}-ɨna

11.  t    {}-an  C\textit{\textsubscript{i} }{}-cja  c-jur  c-ɨna

12.  \$C(G)    {}-an  {}-icja  (Ø)-jur  {}-uuna

13.  ij    {}-an  {}-cja  {}-jur  C\textit{\textsubscript{i} }{}-na

14.  V\textsubscript{non-}\textit{\textsubscript{i} }g    {}-an  Ø-zja  {}-jur  {}-una

15.  ik    {}-jan  Ø-cja  {}-jur  {}-una

16.  i(n)g    {}-jan  Ø-zja  {}-jur  {}-una

17.  in    {}-jan  Ø-zja  {}-jur  {}-na

Note:

(a) “Ø” indicates the deletion of a morphophoneme before the morpheme boundary;

(b) “C\textit{\textsubscript{i}}” indicates the consonant before the morpheme boundary is assimilated to the following consonant;

(c) /c/ before the morpheme boundary means the original //t// alternates with /c/.

The above table shows that each stem has a different set of outputs. Thus, I propose that there are 17 types of verbal stems (from the morphophonological perspective).

There are, however, some verbal stems that do not conform to the regular (morpho)phonological rules. For example, these stems include the light verb \textit{sɨr-} ‘do,’ the deictic motion verbs \textit{ik-} ‘go,’ \textit{k-} ‘come,’ and \textit{tɨkk-} ‘bring,’ the honorific verbs \textit{umoor-} (move.HON), \textit{misjoor-} (eat.HON), \textit{moor-} (HON), \textit{taboor-} (give.HON), and \textit{moosɨr-} (die.HON), the verbal roots ending with //aw// (such as \textit{hijaw-} ‘pick up,’ \textit{waraw-} ‘laugh,’ and \textit{juraw-} ‘gather’), and others such as \textit{sij-} ‘know,’ \textit{jurukub-} ‘happy,’ and \textit{hənk-} ‘enter.’ The subdivision of these verbal stems is shown below (for their actual surface forms, see appendix).

\begin{styleBeschriftung}
\textmd{\tabref{tab:key:59}. Irregular type verbal stems}
\end{styleBeschriftung}

    Affix types

Irregular stems    A. vowel-initial  B. t-initial  C. deletion  D. others

a.  \textit{sɨr-} ‘do’    {}-  IR  IR  {}-

b.  \textit{k-} ‘come’    IR  IR  {}-  IR

c.  \textit{ik-} ‘go’    {}-  IR  {}-  {}-

d.  \textit{umoor-} (move.HON)    {}-  IR  {}-  {}-

e.  \textit{hijaw-} ‘pick up’    IR  {}-  IR  IR

f.  \textit{sij-} ‘know’    {}-  IR  {}-  {}-

g.  \textit{jurukub-} ‘happy’    {}-  {}-  {}-  IR

h.  \textit{hənk-} ‘enter’    IR  IR  {}-  {}-

(IR: irregular process, “-”: regular process)

The deictic motion verb \textit{tɨkk-} ‘bring’ behaves in the same way as \textit{k-} ‘come.’ One may think that \textit{tɨkk-} ‘bring’ is a compound composed of \textit{tur-} ‘take’ + \textit{k-} ‘come.’ However, the first vowel is not /u/ but /ɨ/, and \textit{tur-} ‘take’ should become /tui/ \textit{tur+i} (take-INF) when it fills the preceding stem of a compound (see \sectref{sec:key:4.2.3.1}). Thus, we do not regard \textit{tɨkk-} ‘bring’ as a compound. All the honorific verbs behave in the same way as \textit{umoor-} (move.HON); however, only \textit{moosɨr-} (die.HON) behaves in the same way as \textit{sɨr-} ‘do.’

The following four subsections (\sectref{sec:key:8.2.1.1}-\sectref{sec:key:8.2.1.4}) discuss the relevant morphophonological rules needed for each type of verbal affixes (with the relevant phonological rules). Additionally, a special attention should be paid to the passive affix and the capable affix, which will be discussed in \sectref{sec:key:8.2.1.5.}

\subsubsection{Type A: rule for vowel-initial verbal affixes}

Verbal affixes that begin with a vowel need a rule to explain the following difference.

(8\nobreakdash-10)  a.  \textit{kak-}  ‘write’  +  \textit{{}-an} (NEG)  >  /kak-an/

  b.  \textit{kik-}  ‘hear’        >  /kik-jan/

The example in (8-10 a) presents a simple combination of \textit{kak-} ‘write’ + \textit{{}-an} (NEG) > /kakan/, but the example in (8-10 b) needs \textit{j}{}-insertion between the morphemes such as \textit{kik-} ‘hear’ + \textit{{}-an} (NEG) > /kikjan/.

There are nine verbal affixes that cause \textit{j}{}-insertion: \textit{{}-an} (NEG), \textit{{}-arɨr} (PASS), \textit{{}-arɨr} (CAP), \textit{{}-as} (CAUS), \textit{{}-azɨi} (NEG.PLQ), \textit{{}-ɨ} (IMP), \textit{{}-ɨba} (SUGS), \textit{{}-oo}(INT), and \textit{{}-oo} (SUPP). These affixes will be called “vowel-initial affixes” (or “Type-A affixes”). It should be mentioned, however, that there is an affix that begins with a vowel, but does not cause \textit{j}{}-insertion, i.e. \textit{{}-i} (INF) discussed in \sectref{sec:key:8.4.4.1.} If the following conditions are met, /j/ is inserted before vowel-initial affixes: (a) the verbal stem has //i// in the word-final syllable, and (b) the verbal stem does not end with //j\footnote{Stem-final //j// prohibits the \textit{j}{}-insertion because it would make the /jj/ sequence, which never appears in Yuwan.}// or //r// (for the explanation of the restriction of //r//, see note (b) of the \tabref{tab:key:57}). These conditions can be schematized as in (8-11), where “A-affix” means the Type-A (i.e. vowel-initial) affixes. In the following schemata, morphological units are surrounded by square brackets, which are attached by their morphological information at the lower-right side. Supplemental information is also provided in square brackets under the rule schema.

(8\nobreakdash-11)  Ø  >  j  /  [   iC]\textsubscript{stem}  [ \_   ]\textsubscript{A-affix}

          [C is not //j, r//]  

The rule application and the output forms are shown in \tabref{tab:key:60}. In the following tables, the hyphen “-” in the cells means non-application of the rules.

\begin{styleBeschriftung}
\textmd{\tabref{tab:key:60}}\textmd{. Verbal stems +} \textmd{\textit{{}-an}}\textmd{ (NEG)}
\end{styleBeschriftung}

Stem No.  1. V\textsubscript{non-back}r  2. V\textsubscript{back}r, V\textsubscript{back}w

e.g.  \textit{hingir-}  \textit{abɨr-}  \textit{kəər-}  \textit{ˀkuur-}  \textit{nugoor-}  \textit{koow-}

  ‘escape’  ‘call’  ‘exchange’  ‘close’  ‘don’t do’  ‘buy’

(Input)  hingir-an  abɨr-an  kəər-an  ˀkuur-an  nugoor-an  koow-an

Insertion  {}-  {}-  {}-  {}-  {}-  {}-

(Output)  hingir-an  abɨr-an  kəər-an  ˀkuur-an  nugoor-an  koow-an

Stem No.  2. V\textsubscript{back}r  3. pp  4. b  5. Vm  6. nm  7. V\textsubscript{non-}\textit{\textsubscript{i} }k

e.g.  \textit{tur-}  \textit{app-}  \textit{narab-}  \textit{jum-}  \textit{tanm-}  \textit{kak-}

  ‘take’  ‘play’  ‘line up’  ‘read’  ‘ask’  ‘write’

(Input)  tur-an  app-an  narab-an  jum-an  tanm-an  kak-an

Insertion  {}-  {}-  {}-  {}-  {}-  {}-

(Output)  tur-an  app-an  narab-an  jum-an  tanm-an  kak-an

Stem No.  8. V\textsubscript{non-}\textit{\textsubscript{i} }kk  9. Vs  10. ss  11. t  12. \$C(G)

e.g.  \textit{sukk-}  \textit{us-}  \textit{kuss-}  \textit{ut-}  \textit{jˀ-}  \textit{mj-}

  ‘pull’  ‘push’  ‘kill’  ‘hit’  ‘say’  ‘see’

(Input)  sukk-an  us-an  kuss-an  ut-an  jˀ-an  mj-an

Insertion  {}-  {}-  {}-  {}-  {}-  {}-

(Output)  sukk-an  us-an  kuss-an  ut-an  jˀ-an  mj-an

Stem No.  13. ij  14. V\textsubscript{non-}\textit{\textsubscript{i}} g  15. ik  16. i(n)g    17. in

e.g.  \textit{kij-}  \textit{tug-}  \textit{kik-}  \textit{uig-}  \textit{ming-}  \textit{sin-}

  ‘cut’  ‘whet’  ‘hear’  ‘swim’  ‘grab’  ‘die’

(Input)  kij-an  tug-an  kik-an  uig-an  ming-an  sin-an

Insertion  {}-  {}-  kik-jan  uig-jan  ming-jan  sin-jan

(Output)  kij-an  tug-an  kik-jan  uig-jan  ming-jan  sin-jan

The affix \textit{{}-ɨba} (SUGS) tends to become /ba/ after the verbal stems No. 5 and 17, e.g. \textit{jum-} ‘read’ + \textit{{}-ɨba} (SUGS) > /jumba/ (rather than /jumjɨba/) and \textit{sin-} ‘die’ + \textit{{}-ɨba} (SUGS) > /sinba/ (rather than /sinjɨba/). In addition, the combination of \textit{uig-} ‘swim’ and \textit{{}-ɨba} (SUGS) always becomes /uig-iba/ (not /uig-jɨba/).

\tabref{tab:key:60} shows that the verbal stems No. 15-17, which satisfy the conditions of the rule application discussed above, induce \textit{j}{}-insertion. In order to achieve simplicity with the above combination, we choose these output phonemes of the verbal stems as their underlying morphophonemes.

\subsubsection{Type B: rules for \textit{t}{}-initial verbal affixes}
\label{bkm:Ref347175824}
The rules for affixes that begin with //t// are required in order to explain the differences as follows.

(8\nobreakdash-12)  a.  \textit{abɨr-}  ‘call’  +  \textit{{}-tɨ} (SEQ)  >  /abɨ-tɨ/

  b.  \textit{jum-}  ‘read’        >  /ju-dɨ/

  c.  \textit{kak-}  ‘write’        >  /ka-cjɨ/

  d.  \textit{sin-}  ‘die’        >  /si-zjɨ/

The first example shows a relatively simple combination of \textit{abɨr-} ‘call’ + \textit{{}-tɨ} (SEQ) > /abɨtɨ/, but the other three examples need voicing \textit{{}-tɨ} > /dɨ/, affrication \textit{{}-tɨ} > /cjɨ/, or both \textit{{}-tɨ} > /zjɨ/.

There are seven verbal affixes that cause the above alternations: \textit{{}-tar} (PST), \textit{{}-tuk} (PRPR), \textit{{}-tur} (PROG), \textit{{}-təər} (RSL), \textit{-tɨ} (SEQ), \textit{{}-tai} (LST), and \textit{{}-təəra} ‘after.’ These affixes are called “\textit{t}{}-initial affixes” (or “Type-B affixes”) because they all begin with //t//. It should be mentioned, however, that there are two affixes that begin with //t// but do not conform to the following rules, i.e. \textit{{}-tu} (CSL) and \textit{{}-too} (CSL) discussed in \sectref{sec:key:8.4.3.1.} If there is a combination of a verbal stem and a \textit{t}{}-initial affix, the five rules below are applied in the following order: \REF{ex:key:1} if the stem only contains consonants, //i// is inserted after the stem; \REF{ex:key:2} if the stem has the vowel //i// in its final syllable (and the final consonant is not //r//) or if the stem-final morphophoneme is //t, s, k, g//, the initial //t// of the \textit{t}{}-initial verbal affix becomes //cj//; \REF{ex:key:3} if the stem ends with //b, g, m, n//, the initial consonant of the \textit{t}{}-initial verbal affix is voiced; \REF{ex:key:4} the final consonant (except for //t//) of the stem is deleted; \REF{ex:key:5} if the stem ends with a non-nasal consonant, it is assimilated with the following consonant. In the following schema, “B-affix” refers to the above Type-B (i.e. \textit{t}{}-initial) verbal affixes.

(8\nobreakdash-13)  1.  Insertion

    Ø  >  i  /  [C(G)]\textsubscript{stem}  \_  [   ]\textsubscript{B-affix}

  2.  Affrication (palatalization)

    t  >  cj  /  [   VC]\textsubscript{stem}  [ \_  ]\textsubscript{B-affix}

            [V is //i// and C is not /r/]

or [C is //t, s, k, g//]    

  3.  Voicing

    C(G)  >  C(G)  /  [   C(G)]\textsubscript{stem}  [ \_  ]\textsubscript{B-affix}

    [-v]    [+v]    [C is //b, g, m, n//]    

  4.  Deletion

    C  >  Ø  /  [   \_ ]\textsubscript{stem}  [   ]\textsubscript{B-affix}

    [C is not //t//]      

  5.  Assimilation

    C  >  C\textit{\textsubscript{i}}  /  [   \_ ]\textsubscript{stem}  [C\textit{\textsubscript{i}}  ]\textsubscript{B-affix}

    [C is not nasal]      

\begin{styleBeschriftung}
\textmd{\tabref{tab:key:61}. Verbal stems +} \textmd{\textit{{}-tɨ}}\textmd{ (SEQ)}
\end{styleBeschriftung}

Stem No.  1. V\textsubscript{non-back}r  2. V\textsubscript{back}r, V\textsubscript{back}w

  e.g.  \textit{hingir-}  \textit{abɨr-}  \textit{kəər-}  \textit{ˀkuur-}  \textit{nugoor-}  \textit{koow-}

    ‘escape’  ‘call’  ‘exchange’  ‘close’  ‘don’t do’  ‘buy’

  (Input)  hingir-tɨ  abɨr-tɨ  kəər-tɨ  ˀkuur-tɨ  nugoor-tɨ  koow-tɨ

1.  Insertion  {}-  {}-  {}-  {}-  {}-  {}-

2.  Affrication  {}-  {}-  {}-  {}-  {}-  {}-

3.  Voicing  {}-  {}-  {}-  {}-  {}-  {}-

4.  Deletion  hingi-tɨ  abɨ-tɨ  kəə-tɨ  ˀkuu-tɨ  nugoo-tɨ  koo-tɨ

5.  Assimilation  {}-  {}-  {}-  {}-  {}-  {}-

  (Output)  hingi-tɨ  abɨ-tɨ  kəə-tɨ  ˀkuu-tɨ  nugoo-tɨ  koo-tɨ

Stem No.  2. V\textsubscript{back}r  3. pp  4. b  5. Vm  6. nm  7. V\textsubscript{non-}\textit{\textsubscript{i} }k

  e.g.  \textit{tur-}  \textit{app-}  \textit{narab-}  \textit{jum-}  \textit{tanm-}  \textit{kak-}

    ‘take’  ‘play’  ‘line up’  ‘read’  ‘ask’  ‘write’

  (Input)  tur-tɨ  app-tɨ  naab-tɨ  jum-tɨ  tanm-tɨ  kak-tɨ

1.  Insertion  {}-  {}-  {}-  {}-  {}-  {}-

2.  Affrication  {}-  {}-  {}-  {}-  {}-  kak-cjɨ

3.  Voicing  {}-  {}-  narab-dɨ  jum-dɨ  tanm-dɨ  {}-

4.  Deletion  tu-tɨ  ap-tɨ  nara-dɨ  ju-dɨ  tan-dɨ  ka-cjɨ

5.  Assimilation  {}-  at-tɨ  {}-  {}-  {}-  {}-

  (Output)  tu-tɨ  at-tɨ  nara-dɨ  ju-dɨ  tan-dɨ  ka-cjɨ

  Stem No.  8. V\textsubscript{non-}\textit{\textsubscript{i} }kk  9. Vs  10. ss  11. t  12. \$C(G)

  e.g.  \textit{sukk-}  \textit{us-}  \textit{kuss-}  \textit{ut-}  \textit{jˀ-}  \textit{mj-}

    ‘pull’  ‘push’  ‘kill’  ‘hit’  ‘say’  ‘see’

  (Input)  sukk-tɨ  us-tɨ  kuss-tɨ  ut-tɨ  jˀ-tɨ  mj-tɨ

1.  Insertion  {}-  {}-  {}-  {}-  jˀi-tɨ  mji-tɨ

2.  Affrication  sukk-cjɨ  us-cjɨ  kuss-cjɨ  ut-cjɨ  jˀi-cjɨ  mji-cjɨ

3.  Voicing  {}-  {}-  {}-  {}-  {}-  {}-

4.  Deletion  suk-cjɨ  u-cjɨ  kus-cjɨ  {}-  {}-  {}-

5.  Assimilation  suc-cjɨ  {}-  kuc-cjɨ  uc-cjɨ  {}-  {}-

  (Output)  suc-cjɨ  u-cjɨ  kuc-cjɨ  uc-cjɨ  jˀi-cjɨ  mji-cjɨ

  Stem No.  13. ij  14. V\textsubscript{non-}\textit{\textsubscript{i}} g  15. ik  16. i(n)g    17. in

  e.g.  \textit{kij-}  \textit{tug-}  \textit{kik-}  \textit{uig-}  \textit{ming-}  \textit{sin-}

    ‘cut’  ‘whet’  ‘hear’  ‘swim’  ‘grab’  ‘die’

  (Input)  kij-tɨ  tug-tɨ  kik-tɨ  uig-tɨ  ming-tɨ  sin-tɨ

1.  Insertion  {}-  {}-  {}-  {}-  {}-  {}-

2.  Affrication  kij-cjɨ  tug-cjɨ  kik-cjɨ  uig-cjɨ  ming-cjɨ  sin-cjɨ

3.  Voicing  {}-  tug-zjɨ  {}-  uig-zjɨ  ming-zjɨ  sin-zjɨ

4.  Deletion  ki-cjɨ  tu-zjɨ  ki-cjɨ  ui-zjɨ  min-zjɨ  si-zjɨ

5.  Assimilation  {}-  {}-  {}-  {}-  {}-  {}-

  (Output)  ki-cjɨ  tu-zjɨ  ki-cjɨ  ui-zjɨ  min-zjɨ  si-zjɨ

It should be noted that the above rules do not apply to the negative affix \textit{{}-an} (NEG). All of the “\textit{t}{}-initial affixes” can follow \textit{{}-an} (NEG) without any morphophonological change, e.g., \textit{{}-an-tɨ} (NEG-SEQ) becomes /-an-tɨ/ (not /-a-dɨ/) as in (8-104) in \sectref{sec:key:8.4.3.5.}

\subsubsection{Type C: rules for affixes (and clitics) deleting non-nasal resonants}
\label{bkm:Ref347177215}
There are affixes and clitics that delete the preceding non-nasal resonants: \textit{{}-jawur} (POL), \textit{{}-jaa} ‘person,’ \textit{{}-jur} (UMRK), \textit{{}-jagacinaa} (SIM), \textit{{}-mɨ} (PLQ), \textit{{}-n} (PTCP), \textit{jaa} (SOL), and \textit{sɨ} (FN), which are called “Type-C affixes (or clitics).” In the following schema, “C-affix/clitic” refers to these affixes and clitics.

(8\nobreakdash-14)  Deletion

  C (or G)  >  Ø  /  [   \_ ]\textsubscript{stem}  [   ]\textsubscript{C-affix/clitic}

  [C is non-nasal resonant]    

\begin{styleBeschriftung}
\textmd{\tabref{tab:key:62}. Verbal stems +} \textmd{\textit{{}-jur}}\textmd{ (UMRK)}
\end{styleBeschriftung}

Stem No.  1. V\textsubscript{non-back}r  2. V\textsubscript{back}r, V\textsubscript{back}w

e.g.  \textit{hingir-}  \textit{abɨr-}  \textit{kəər-}  \textit{ˀkuur-}  \textit{nugoor-}  \textit{koow-}

  ‘escape’  ‘call’  ‘exchange’  ‘close’  ‘don’t do’  ‘buy’

(Input)  hingir-jur  abɨr-jur  kəər-jur  ˀkuur-jur  nugoor-jur  koow-jur

Deletion  hingi-jur  abɨ-jur  kəə-jur  ˀkuu-jur  nugoo-jur  koo-jur

(Output)  hingi-jur  abɨ-jur  kəə-jur  ˀkuu-jur  nugoo-jur  koo-jur

Stem No.  2. V\textsubscript{back}r  3. pp  4. b  5. Vm  6. nm  7. V\textsubscript{non-}\textit{\textsubscript{i} }k

e.g.  \textit{tur-}  \textit{app-}  \textit{narab-}  \textit{jum-}  \textit{tanm-}  \textit{kak-}

  ‘take’  ‘play’  ‘line up’  ‘read’  ‘ask’  ‘write’

(Input)  tur-jur  app-jur  narab-jur  jum-jur  tanm-jur  kak-jur

Deletion  tu-jur  {}-  {}-  {}-  {}-  {}-

(Output)  tu-jur  app-jur  narab-jur  jum-jur  tanm-jur  kak-jur

Stem No.  8. V\textsubscript{non-}\textit{\textsubscript{i} }kk  9. Vs  10. ss  11. t  12. \$C(G)

e.g.  \textit{sukk-}  \textit{us-}  \textit{kuss-}  \textit{ut-}  \textit{jˀ-}  \textit{mj-}

  ‘pull’  ‘push’  ‘kill’  ‘hit’  ‘say’  ‘see’

(Input)  sukk-jur  us-jur  kuss-jur  ut-jur  jˀ-jur  mj-jur

Deletion  {}-  {}-  {}-  {}-  Ø-jur/jˀ-ur\footnote{As an exception, there is a rare case where the stem-final //jˀ// is not deleted in order to retain the original root form, and the affix-initial //j// is deleted instead.}  m-jur

(Output)  sukk-jur  us-jur  kuss-jur  uc{}-jur  Ø-jur/jˀ-ur  m-jur

Stem No.  13. ij  14. V\textsubscript{non-}\textit{\textsubscript{i}} g  15. ik  16. i(n)g    17. in

e.g.  \textit{kij-}  \textit{tug-}  \textit{kik-}  \textit{uig-}  \textit{ming-}  \textit{sin-}

  ‘cut’  ‘whet’  ‘hear’  ‘swim’  ‘grab’  ‘die’

(Input)  kij-jur  tug-jur  kik-jur  uig-jur  ming-jur  sin-jur

Deletion  ki-jur  {}-  {}-  {}-  {}-  {}-

(Output)  ki-jur  tug-jur  kik-jur  uig-jur  ming-jur  sin-jur

Note: In the example of the stem No. 11, //t// becomes /c/ before //j// because of the phonological rule in \sectref{sec:key:2.4.2.}

Only the affix \textit{{}-jagacinaa} (SIM) requires an additional rule, i.e., it becomes /jaagacinaa/ after a verbal root containing only consonant(s).

(8\nobreakdash-15)  Lengthening

  \textit{{}-jagacinaa} (SIM)  >  {}-jaagacinaa  /  [C(G)]\textsubscript{stem}  \_

\begin{styleBeschriftung}
\textmd{\tabref{tab:key:63}. Verbal stems +} \textmd{\textit{{}-jagacinaa}}\textmd{ (SIM)}
\end{styleBeschriftung}

Stem No.  12. Only C(G)    cf.  5. Vm

e.g.  \textit{jˀ-}  \textit{mj-}      \textit{jum-}

  ‘say’  ‘see’      ‘read’

(Input)  jˀ-jagacinaa  mj-jagacinaa      jum-jagacinaa

Deletion  jˀ-agacinaa\footnote{Stem-final //jˀ// is not deleted in order to retain the original root form; instead, the affix-initial //j// is deleted.}  m-jagacinaa      {}-

Lengthening  jˀ-aagacinaa  m-jaagacinaa      {}-

(Output)  jˀ-aagacinaa  m-jaagacinaa      jum-jagacinaa

\subsubsection{Type D: rules for the other verbal affixes (or clitics)}
\label{bkm:Ref347177096}
It is necessary to derive rules for the other verbal affixes in order to explain the differences as follows.

(8\nobreakdash-16)  a.  \textit{jum-}  ‘read’  +  \textit{{}-na} (PROH)  >  /jum-na/

  b.  \textit{abɨr-}  ‘call’        >  /abɨn-na/

  c.  \textit{kak-}  ‘write’        >  /kak-una/

  d.  \textit{us-}  ‘push’        >  /us-ɨna/

The first example shows a simple combination of \textit{jum-} ‘read’ + \textit{{}-na} (PROH) > /jumna/, but the next three require either nasal assimilation or vowel-insertion at the morpheme boundary. The verbal affixes that require these rules include \textit{{}-na} (PROH), \textit{{}-ba} (CSL), \textit{{}-boo} (CND), \textit{{}-gadɨ} ‘until,’ \textit{{}-sa} (POL), \textit{{}-sɨga} (POL), \textit{{}-tu} (CSL), and \textit{{}-too} (CSL). In addition, some “affix-like clitics” (see \sectref{sec:key:4.2.2.2}) are subject to the same rules, i.e. \textit{doo} (ASS), \textit{ka} (DUB), \textit{kai} (DUB), \textit{kamo} (POS), \textit{ga} (CFM3), and \textit{gajaaroo} (DUB). They are called “Type-D affixes (or clitics).” If a verbal stem is combined with these affixes (or clitics), six rules should be applied in the following order. Please note that if two rules have the same number, such as \REF{ex:key:3a} and \REF{ex:key:3b}, their order is free. The rules are: \REF{ex:key:1} if the final morphophoneme of the verbal stem is //t//, it becomes //c//; \REF{ex:key:2} if the final morphophoneme of the verbal stem is a consonant after a syllable boundary, //u// is inserted before the affix; \REF{ex:key:3a} if the final morphophoneme of the verbal stem is //w, j, r// (non-nasal resonants), it is assimilated to the following consonant; \REF{ex:key:3b} if the final morphophoneme of the verbal stem is not resonant and the following affix begins with consonant (i.e. there is no inserted vowel), //u// is inserted before the affix; \REF{ex:key:4a} if the stem originally contains only consonants, the inserted vowel of following syllable is lengthened; \REF{ex:key:4b} if the final morphophoneme of the stem is //c, s//, the following //u// becomes /ɨ/. In the following schema, “D-affix (or clitic)” refers to the verbal affixes and clitics discussed above. It should be noted that if \textit{kai} (DUB) or \textit{kamo} (POS) follows \textit{{}-tar} (PST), these rules do not apply and they simply delete the //r// of \textit{{}-tar} (PST) (see \sectref{sec:key:8.4.1.1}).

(8\nobreakdash-17)  1.  Affrication

    t  >  c  /  [    \_ ]\textsubscript{stem}  [  ]\textsubscript{D-affix (or clitic)}

  2  Insertion

    Ø  >  u  /  \#C]\textsubscript{stem}  [ \_C  ]\textsubscript{D-affix (or clitic)}

  3a.  Assimilation

    C  >  C\textit{\textsubscript{i}}  /  [    \_ ]\textsubscript{stem}  [C\textit{\textsubscript{i}}  ]\textsubscript{D-affix (or clitic)}

    [C is //w, j, r//]      

  3b.  Insertion

    Ø  >  u  /  [     C]\textsubscript{stem}  [ \_C  ]\textsubscript{D-affix (or clitic)}

        [C is not //m, n, w, j, r//]  

  4a.  Lengthening\footnote{The stems preceding type D affixes seem to behave as if they were phonological words since they become bimoraic like many of the phonological words in Yuwan (cf. §\ref{bkm:Ref381399409}).}

    Ø  >  V\textit{\textsubscript{i}}  /  [C(G)]\textsubscript{stem}  [V\textit{\textsubscript{i}} \_   ]\textsubscript{D-affix (or clitic)}

  4b.  Centralizing

    u  >  ɨ  /  [    C]\textsubscript{stem}  [ \_  ]\textsubscript{D-affix (or clitic)}

            [C is //c, s//]    

\begin{styleBeschriftung}
\textmd{\tabref{tab:key:64}. Verbal stems +} \textmd{\textit{{}-na}}\textmd{ (PROH)}
\end{styleBeschriftung}

Stem No.  1. V\textsubscript{non-back}r  2. V\textsubscript{back}r, V\textsubscript{back}w

  e.g.  \textit{hingir-}  \textit{abɨr-}  \textit{kəər-}  \textit{ˀkuur-}  \textit{nugoor-}  \textit{koow-}

    ‘escape’  ‘call’  ‘exchange’  ‘close’  ‘don’t do’  ‘buy’

  (Input)  hingir-na  abɨr-na  kəər-na  ˀkuur-na  nugoor-na  koow-na

1.  Affrication  {}-  {}-  {}-  {}-  {}-  {}-

2.  Insertion  {}-  {}-  {}-  {}-  {}-  {}-

3a.  Assimilation  hingin-na  abɨn-na  kəən-na  ˀkuun-na  nugoon-na  koon-na

3b.  Insertion  {}-  {}-  {}-  {}-  {}-  {}-

4a.  Lengthening  {}-  {}-  {}-  {}-  {}-  {}-

4b.  Centralizing  {}-  {}-  {}-  {}-  {}-  {}-

  (Output)  hingin-na  abɨn-na  kəən-na  ˀkuun-na  nugoon-na  koon-na

Stem No.  2. V\textsubscript{back}r  3. pp  4. b  5. Vm  6. nm  7. V\textsubscript{non-}\textit{\textsubscript{i} }k

  e.g.  \textit{tur-}  \textit{app-}  \textit{narab-}  \textit{jum-}  \textit{tanm-}  \textit{kak-}

    ‘take’  ‘play’  ‘line up’  ‘read’  ‘ask’  ‘write’

  (Input)  tur-na  app-na  narab-na  jum-na  tanm-na  kak-na

1.  Affrication  {}-  {}-  {}-  {}-  {}-  {}-

2.  Insertion  {}-  app-una  {}-  {}-  tanm-una  {}-

3a.  Assimilation  tun-na  {}-  {}-  {}-  {}-  {}-

3b.  Insertion  {}-  {}-  narab-una  {}-  {}-  kak-una

4a.  Lengthening  {}-  {}-  {}-  {}-  {}-  {}-

4b.  Centralizing  {}-  {}-  {}-  {}-  {}-  {}-

  (Output)  tun-na  app-una  narab-una  jum-na  tanm-una  kak-una

  Stem No.  8. V\textsubscript{non-}\textit{\textsubscript{i} }kk  9. Vs  10. ss  11. t  12. \$C(G)

  e.g.  \textit{sukk-}  \textit{us-}  \textit{kuss-}  \textit{ut-}  \textit{jˀ-}  \textit{mj-}

    ‘pull’  ‘push’  ‘kill’  ‘hit’  ‘say’  ‘see’

  (Input)  sukk-na  us-na  kuss-na  ut-na  jˀ-na  mj-na

1.  Affrication  {}-  {}-  {}-  uc-na  {}-  {}-

2.  Insertion  sukk-una  {}-  kuss-una  {}-  jˀ-una  mj-una

3a.  Assimilation  {}-  {}-  {}-  {}-  {}-  {}-

3b.  Insertion  {}-  us-una  {}-  uc-una  {}-  {}-

4a.  Lengthening  {}-  {}-  {}-  {}-  jˀ-uuna  mj-uuna

4b.  Centralizing  {}-  us-ɨna  kuss-ɨna  uc-ɨna  {}-  {}-

  (Output)  sukk-una  us-ɨna  kuss-ɨna  uc-ɨna  jˀ-uuna  mj-uuna

  Stem No.  13. ij  14. V\textsubscript{non-}\textit{\textsubscript{i}} g  15. ik  16. i(n)g    17. in

  e.g.  \textit{kij-}  \textit{tug-}  \textit{kik-}  \textit{uig-}  \textit{ming-}  \textit{sin-}

    ‘cut’  ‘whet’  ‘hear’  ‘swim’  ‘grab’  ‘die’

  (Input)  kij-na  tug-na  kik-na  uig-na  ming-na  sin-na

1.  Affrication  {}-  {}-  {}-  {}-  {}-  {}-

2.  Insertion  {}-  {}-  {}-  {}-  ming-una  {}-

3a.  Assimilation  kin-na  {}-  {}-  {}-  {}-  {}-

3b.  Insertion  {}-  tug-una  kik-una  uig-una  {}-  {}-

4a.  Lengthening  {}-  {}-  {}-  {}-  {}-  {}-

4b.  Centralizing  {}-  {}-  {}-  {}-  {}-  {}-

  (Output)  kin-na  tug-una  kik-una  uig-una  ming-una  sin-na

\subsubsection{Passive and capable affixes alternation}

The passive affix (see \sectref{sec:key:8.5.1.2}) and the capable affix (see \sectref{sec:key:8.5.1.3}) have many similar allomorphs. Their output forms are determined by the following affixes. For a more economical analysis, I postulate three underlying forms for the passive and capable affixes respectively: \textit{{}-arɨr}, \textit{{}-arɨɨr}, and \textit{-ar}.

Both of the forms \textit{{}-arɨr} and \textit{{}-arɨɨr} conform to the (morpho)phonological rules already presented in the previous sections. However, the form \textit{{}-ar} needs special attention, because the means taken to avoid syllable-final /r/ are different from the other rules. The final //r// of \textit{{}-ar} is relatively “strong,” as it were. The //r// is not deleted but retained in all cases, which is contrary to the rules in \sectref{sec:key:8.2.1.2} and \sectref{sec:key:8.2.1.3}, where //r// before Type-B affixes or Type-C affixes must be deleted.

(8\nobreakdash-18)  Rule for \textit{{}-ar} (PASS/CAP)

  a.  Assimilation:  \textit{{}-ar} (PASS/CAP)  >  {}-at  /  \_  [  ]\textsubscript{B-affix}  

  b.  Deletion:  \textit{{}-jagacinaa} (SIM)  >  {}-agacinaa  /  \textit{{}-ar} (PASS)  \_

(8\nobreakdash-19)  Examples

  a.  Assimilation (to the following morphophoneme)

      \textit{tur-}  ‘take’  +  \textit{{}-ar} (PASS)  +  \textit{{}-tar} (PST)

    >  tur-      {}-at      {}-ta  

  b.  Deletion (of the following morphophoneme)

      \textit{oos-}  ‘scold’  +  \textit{{}-ar} (PASS)  +  \textit{{}-jagacinaa} (SIM)

    >  oos-      {}-ar      {}-agacinaa  

These rules show that the //r// of \textit{{}-ar} (PASS) does not drop but rather assimilates with the following //t// as in (8-19 a). In addition, the //r// of \textit{{}-ar} (PASS) does not drop but instead deletes the following //j// of \textit{{}-jagacinaa} (SIM) as in (8-19 b).

\begin{styleBeschriftung}
\textmd{\tabref{tab:key:65}. Combinations of the passive and capable affixes and other affixes showing their surface forms}
\end{styleBeschriftung}

Preceding

passive/capable affixes  Following

affixes (or clitics)    Preceding

passive/capable affixes  Following

affixes (or clitics)

\textit{{}-arɨr  {}-arɨɨr  {}-ar} Type A    \textit{{}-arɨr  {}-arɨɨr  {}-ar} Type C

    ar\textsubscript{ P/C}  \textit{{}-an} (NEG)    arɨ\textsubscript{ P}\textstyleFootnoteSymbol{} \footnote{\citet[70]{Niinaga2010} stated that \textit{{}-jaa} ‘person’ chooses the form \textit{{}-ar} as in /utaraa/ \textit{ut-ar-jaa} (hit-PASS-person). However, a later research shows that the form is not permitted, and instead the form /utarɨjaa/ \textit{ut-arɨr-jaa} (hit-PASS-person), which chooses \textit{{}-arɨr}, was permitted by the same speaker TM.}      \textit{{}-jaa} ‘person’

    ar\textsubscript{ C}  \textit{{}-azɨi} (NEG.PLQ)    arɨ\textsubscript{ P/C}      \textit{{}-joor} (POL)

    ar \textsubscript{P}  \textit{{}-ɨ} (IMP)        ar\textsubscript{ P}  \textit{{}-jagacinaa} (SIM)

arɨr\textsubscript{ C}      \textit{{}-ɨba} (SUGS)      arɨɨ\textsubscript{ P/C}    \textit{sɨ}  (FN)

    ar\textsubscript{ P}  \textit{{}-oo} (INT)      arɨɨ\textsubscript{ P/C}    \textit{{}-mɨ} (PLQ)

arɨr\textsubscript{ C}  arɨɨr\textsubscript{ C}    \textit{{}-oo} (SUPP)    arɨ\textsubscript{ P/C}      \textit{{}-n} (PTCP)

  arɨɨr\textsubscript{ C}    \textit{{}-u} (PFC)    \textit{{}-arɨr  {}-arɨɨr  {}-ar} Type D

arɨ\textsubscript{ P/C}      \textit{{}-i} (NPST)    arɨp\textsubscript{ P/C}      \textit{{}-ba} (CSL)

\textit{{}-arɨr  {}-arɨɨr  {}-ar} Type B    arɨp\textsubscript{ P/C}      \textit{{}-boo} (CND)

arɨ\textsubscript{ C}  arɨɨ\textsubscript{ P/C}\footnote{In the text data, \textit{{}-arɨɨr} (PASS/CAP) is used only in the combination of /-arɨɨ-tat-tu/ \textit{{}-arɨɨr-tar-tu} (PASS/CAP-PST-CSL).}  at\textsubscript{ P/C}  \textit{{}-tar} (PST)    arɨt\textsubscript{ P/C}      \textit{doo}  (ASS)

    at\textsubscript{ P}  \textit{{}-tuk} (PRPR)    arɨk\textsubscript{ P/C}      \textit{kai}  (DUB)

    at\textsubscript{ P/C}  \textit{{}-tur} (PROG)    arɨs\textsubscript{ P/C}      \textit{{}-sa}/\textit{{}-sɨga}  (POL)

  arɨɨ\textsubscript{ C}  at\textsubscript{ P}  \textit{{}-təər} (RSL)            

    at\textsubscript{ P/C}  \textit{{}-tɨ} (SEQ)            

    at\textsubscript{ P/C}  \textit{{}-tai} (LST)            

Notes:

(a) The lower right symbols on the surface (i.e. non-italic) forms express whether the form is the passive affix (“\textsubscript{P}”), the capable affix (“\textsubscript{C}”), or both (“\textsubscript{P/C}”);

(b) The passive affix cannot precede \textit{{}-oo} (SUPP). The assumed meaning is expressed by the combination of \textit{{}-arɨr} (PASS) + \textit{{}-Ø} (INF) + \textit{daroo} (SUPP), e.g. /acjaa wanga utarɨdaro/ \textit{acja} \textit{wan=ga} \textit{ut-arɨr-Ø=daroo} (tomorrow 1SG=NOM hit-PASS-INF=SUPP) ‘Probably, I will be hit tomorrow’;

(c) The politeness affix has two forms \textit{{}-jawur} and -\textit{joor}, and the passive and capable affixes prefer the latter form, e.g. \textit{ut-} ‘hit’ + \textit{{}-arɨr} (PASS) + \textit{{}-joor} (POL) + \textit{doo} (ASS) > /ut-arɨ-joot=too/ ‘(I) will be hit (by you).’

\subsection{Some notes on the interpretation of the verbal paradigm}
\subsubsection{\textit{r}{}-final stems}

There are two kinds of \textit{r}{}-final stems in Yuwan (stem No. 1-2 in \tabref{tab:key:57} in \sectref{sec:key:8.2.1}). It is worth noting that stem No. 1 (whose final morphophonemes are a non-back vowel plus //r//) does not require /i/ insertion to produce infinitives, but stem No. 2 (whose final morphophonemes are a back vowel plus //r// or //w//) do require this insertion, similar to other consonant-final stems. The combination of a verbal stem plus the infinitival affix is called infinitive (see \sectref{sec:key:8.4.4} for more details).

\begin{styleBeschriftung}
\textmd{\tabref{tab:key:66}}\textmd{. Infinitives of the verbal stems No. 1, 2, and 7}
\end{styleBeschriftung}

Stem No.  1    2    7

Ex.  \textit{abɨr-} ‘call’    \textit{tur-} ‘take’    \textit{kak-} ‘write’

Infinitives (in surface forms)  abɨ      tui\footnote{Phonological rule (see §\ref{bkm:Ref381399452}): tur + i > tui}      kaki  

Infinitives (in underlying forms)  \textit{abɨr-Ø}  (call-INF)    \textit{tur-i}  (take-INF)    \textit{kak-i}  (write-INF)

Considering \tabref{tab:key:66}, one might think that the stem-final //r// of stem No. 1 (e.g. \textit{abɨr-} ‘call’) is not part of the preceding stem but rather part of the following affix as in (8-20).

(8\nobreakdash-20)  Current analysis:  \textit{abir-}  ‘call’  +  \textit{{}-an}  (NEG)

  Possible analysis:  \textit{abɨ-}  ‘call’  +  \textit{{}-ran}  (NEG)

In that case, we would be able to explain the phenemenon in \tabref{tab:key:66} more simply. The consonant-final verbal stems, e.g. \textit{tur-} ‘take’ and \textit{kak-} ‘write,’ would require \textit{{}-i} (INF), but the vowel-final verbal stems, e.g. \textit{abɨ-} ‘call,’ would require \textit{{}-Ø} (INF). However, we will not adopt this analysis for the reasons discussed below.

\begin{styleBeschriftung}
\textmd{\tabref{tab:key:67}. Combinations of verbal roots and Type-A affixes and Type-D affixes}
\end{styleBeschriftung}

Stem No.  1    2    7

Ex.  \textit{abɨr-} ‘call’    \textit{tur-} ‘take’    \textit{kak-} ‘write’

Followed by Type-A affixes  abɨr    an (NEG)    tur    an (NEG)    kak    an (NEG)

      ɨ (IMP)        ɨ (IMP)        ɨ (IMP)

Followed by Type-D affixes  abɨn    na (PROH)    tun    na (PROH)    kak  u  na (PROH)

  abɨb    ba (CSL)    tub    ba (CSL)    kak  u  ba (CSL)

If we propose the final //r// of stem No. 1 (e.g. \textit{abɨr-} ‘call’) does not belong to the root but to the following affix, we would then have to interpret the root-final /n/ or /b/ before Type-D affixes (e.g. \textit{{}-na} (PROH) or \textit{{}-ba} (CSL)) as affix-initial consonants, such as \textit{{}-nna} (PROH) or \textit{{}-bba} (CSL). This analysis, however, is not applicable since these forms could not appear after other verbal stems, such as \textit{kak-} ‘write’ + \textit{{}-na} (PROH) > /kak-una/ (*/kak-unna/), or \textit{kak-} ‘write’ + \textit{{}-ba} (CSL) > /kak-uba/ (*/kak-ubba/ nor */kak-uppa/). Thus, it is more appropriate to propose that the //r// belongs not to the following affixes but to the preceding stems.

\subsubsection{Not setting up “base types”}

Some of the previous research on Northern Ryukyuan languages proposed an analysis of the verbal stems, which is different from that adopted by the present author. They propose that the initial (morpho)phonemes of the verbal derivational affixes are treated as the final (morpho)phonemes of the verbal roots; for example, Uchima et al. (1976: 74ff.) for Yuwan (Amami), and Nishioka \& Nakahara (2000: 37, 55) for Shuri (Okinawa). The example below is taken from \citet{UchimaEtAl1976}’s analysis, where the term “base” is used to refer to what I call a verbal root (the phonological representations and glosses are adjusted by the present author).

\begin{styleBeschriftung}
\textmd{\tabref{tab:key:68}}\textmd{. Analysis of the verb in \citet{UchimaEtAl1976}}
\end{styleBeschriftung}

Base types    Stem-derivational affix  Ending

  E.g. ‘write’    

Basic  kak    oo (INT), ɨ (IMP), etc.

Renyou  kakj  {}-u\textsubscript{1} (UMRK)  i (NPST), ru (PFC), etc.

Onbin (‘euphony’)  kacj  {}-ɨ/-i (SEQ), -eera, -əə, -a, -u\textsubscript{2} (PROG)  i (NPST), n (PTCP), etc.

Notes:

(a) \citet[78]{UchimaEtAl1976} propose that the “real base” is /kak/ and the other forms, i.e. /kakj/ and /kacj/, are its variants depending on the morphological environments;

(b) \citet[91-92]{UchimaEtAl1976} argue that the sequential converbal forms (“SEQ” in \tabref{tab:key:68}), which are labeled \textit{Setsuzoku-kei} ‘conjunctive form’ in their terms, can be /ɨ/ or /i/. However, the speaker TM, who is the main consultant for the present research, says it should be /ɨ/ in all cases. Although, it sometimes sounds like /i/ after alveolar affricates or fricatives.

The above table shows that \citet{UchimaEtAl1976} distinguishes three “base types,” although, I do not make such a distinction (see Chapter 8). I found three disadvantages in proposing the base types: (a) the redundancy in the explanation of the semantic differences between verbs; (b) the emergence of unnecessary homophonic affixes; (c) the inability to explain a sequence of \textit{t}{}-initial affixes.

First, if we allow the above segmentation as in \tabref{tab:key:68}, the difference between /kak-ɨ/ (write-IMP) and /kacj-ɨ/ (write-SEQ) would be explained by the difference in base (i.e. Basic vs. Onbin) and also by the difference in affix (i.e. /ɨ/ (IMP) vs. /ɨ/ (SEQ)). On the other hand, if we assume only one base (i.e. root) \textit{kak-} ‘write,’ and regard the alleged base-final (morpho)phonemes /cj/ as the initial (morpho)phonemes of the following affix such as /cjɨ/ (SEQ), then the above difference can be more succinctly explained by the difference in affix, i.e. /ɨ/ (IMP) vs. /cjɨ/ (SEQ).

\begin{styleBeschriftung}
\textmd{\tabref{tab:key:69}. Comparison of analyses by \citet{UchimaEtAl1976} and the present author (in surface forms)}
\end{styleBeschriftung}

  Gloss  write-IMP    Gloss  write-SEQ

\citet{UchimaEtAl1976}  e.g.  kak-ɨ    e.g.  kacj-ɨ

The present author  e.g.  kak-ɨ    e.g.  ka-cjɨ

Note: In the present author’s analysis, the deletion of the root-final morphophoneme //k// in \textit{kak-} ‘write’ is explained by a morphophonological rule (see \sectref{sec:key:8.2.1.2}).

  Furthermore, the analysis proposed by \citet{UchimaEtAl1976} creates unnecessary homophonic morphemes such as \textit{{}-ɨ} (IMP) vs. \textit{{}-ɨ} (SEQ), and \textit{{}-u}\textsubscript{1} (UMRK) vs. \textit{{}-u}\textsubscript{2} (PROG). On the other hand, our analysis does not fall into this trap, e.g. \textit{{}-ɨ} (IMP) vs. \textit{{}-tɨ} (SEQ), and \textit{{}-jur} (UMRK) vs. \textit{{}-tur} (PROG).

  Finally, the “base type” analysis cannot explain a sequence of \textit{t}{}-initial affixes (for more discussion on \textit{t}{}-initial affixes, see \sectref{sec:key:8.2.1.2}). For example, a combination such as \textit{nar-} ‘become’ + \textit{-tur} (PROG) + \textit{{}-tɨ} (SEQ) > /na-tu-tɨ/\footnote{Morphophonological rules (see §\ref{bkm:Ref347175824}): nar + tur + tɨ > natutɨ.} (become-PROG-SEQ) exists in Yuwan. If we adopt the “base type” analysis, the first two morphemes would be analyzed as /nat-u/ (become-PROG), but we are unable to explain the final morpheme, i.e. /tɨ/ (SEQ), because \citet[91-92]{UchimaEtAl1976} considers the affix to be /ɨ/ (SEQ). In other words, their analysis would result in the ill-formed utterance */nat-u-ɨ/.

\begin{styleBeschriftung}
\textmd{\tabref{tab:key:70}. Comparison of analyses by \citet{UchimaEtAl1976} and the present author (in surface forms)}
\end{styleBeschriftung}

  Output forms expected by each analysis  Gloss

\citet{UchimaEtAl1976}  *nat-u-ɨ  (become-PROG-SEQ)

The present author  na-tu-tɨ  (become-PROG-SEQ)

\citet{UchimaEtAl1976} cannot predict the correct form /-tɨ/ (SEQ) because they have misunderstood the initial phoneme of /-tɨ/ (SEQ) (and also other \textit{t}{}-initial affixes) as a part of a root (not of an affix). Therefore, the affix cannot begin with //t// in their analysis.

  In conclusion, in order to achieve an economical, clear, and exhaustive analysis, we avoid setting up “base types” as previous researchers have done.

\section{Stem types}

The stem types classified by morphophonological criteria were all shown in \tabref{tab:key:57} in \sectref{sec:key:8.2.1.} In this section, we will consider some stems which have unique semantic-syntactical and/or morphosyntctic characteristics.

  First, Yuwan has semantically and syntactically interesting stems, i.e. honorific verbal stems. The honorific verbal stems can express the speaker’s respect for the subject of the predicate (see Chapter 3). The details of the honorific verbs will be discussed in \sectref{sec:key:8.3.1.}

  Second, we will look at the differences between three kinds of verbal stems: the existential verbs, the copula verbs, and the stative verbs. These verbal stems have a few alternate morphemes. Let us see the following table, where the variation of affirmative copula forms is a little simplified.

\begin{styleBeschriftung}
\textmd{\tabref{tab:key:71}}\textmd{. Existential verb vs. copula verb vs. stative verb (simplified)}
\end{styleBeschriftung}

Polarity  Affirmative  Negative

Core NPs  Animate  Inanimate  Animate  Inanimate

Existential verbs  \textit{wur-}  \textit{ar-}  \textit{wur-}  \textit{nə-}

Copula verbs  \textit{jar-}  \textit{ar-}

Stative verbs  \textit{ar-}  \textit{nə-}

\textit{wur-} is always an existential verb, and \textit{jar-} is always a copula verb. The form /ar-/, however, can be a morpheme of all of the three verbal stems. Similarly, the form /nə-/ may be a morpheme of either the existential verb or the stative verb. The details of \tabref{tab:key:71} will be shown in the follwoing subsections: the existential verbs (see \sectref{sec:key:8.3.2}), the copula verbs (see \sectref{sec:key:8.3.3}), and the stative verbs (see \sectref{sec:key:8.3.4}). The morphosyntactic similarities among these three verbs will be discussed in \sectref{sec:key:8.3.5.}

\subsection{Honorific verbs}

As mentioned in Chapter 3, honorific verbs express the speaker’s respect for the subject of the predicate. Generally, the respect is dedicated to the people older than the speaker. There are, however, some cases where the people younger than the speaker receive the speaker’s respect; in that case, there is another factor that induces such respect, e.g. the academic prestige as in (8-22 a-b) and (8-23) in \sectref{sec:key:8.3.1.1.}

There are two types of honorific verbs. One of them can fill the predicate slot of a clause by itself, i.e. lexical honorific verbs. The other cannot fill the predicate slot only by itself, i.e. auxiliary honorific verbs, and it needs a lexical verb to precede it, which is called the auxiliary verb construction (see \sectref{sec:key:9.1.1}).

(8\nobreakdash-21)  Two types of honorific verbs

  a.  Lexical honorific verb

    [Context: TM thanks to US, who is older than TM.]

    TM:  nanga  umoocjattu,  {\textbar}cjoodo{\textbar}  jiccja  ata.

      \textit{nan=ga}  \textit{umoor{}-tar-tu  cjoodo  jiccj-sa  ar-tar}

      2.HON.SG=NOM  [come.HON-PST]  just  good-ADJ  STV-PST

        [Lex. verb]\textsubscript{VP}      

      ‘You came, so (it) was very good.’

      [Co: 110328\_00.txt]

  b.  Auxiliary honorific verb

    [Context: TM explained to US that the present author had wanted to see her.]

    TM:  nanga  hanacjɨ  moojun  mun

      \textit{nan=ga}  \textit{hanas-tɨ}  \textit{moor{}-jur-n  mun}

      2.HON.SG=NOM  [speak-SEQ  HON-UMRK-PTCP]  thing

        [Lex. verb  Aux. verb]\textsubscript{VP}  

      kikicjasancjɨ  jˀicjɨ,

      \textit{kik-i-cja-sa+ar-n=ccjɨ}  \textit{jˀ-tɨ}

      hear-INF+want-ADJ+STV-PTCP=QT  say-SEQ

      ‘(The present author) said that (he) wanted to hear what you said.’

      [Co: 110328\_00.txt]

In (8-21 a), \textit{umoor-} (come.HON) is a lexical honorific verb, and it expresses the speaker’s respect for the subject \textit{nan} (2.HON.SG) ‘you.’ In (8-21 b), \textit{moor-} (HON) is an auxiliary honorific verb, that follows the lexical verb \textit{hanas-} ‘speak,’ and \textit{moor-} (HON) expresses the speaker’s respect for the subject \textit{nan} (2.HON.SG) ‘you.’

  In the following subsections, I will discuss the lexical honorific verb (see \sectref{sec:key:8.3.1.1}) and the auxiliary honorific verb (see \sectref{sec:key:8.3.1.2}).

\subsubsection{Lexical honorific verb}

Yuwan has the follwing four lexical honorific verbs.

\begin{styleBeschriftung}
\textmd{\tabref{tab:key:72}}\textmd{. Lexical honorific verbs}
\end{styleBeschriftung}

Lexical honorific verbs  Relevant non-honorific verbs

\textit{umoor-} (exist/go/come/say.HON)  \textit{wur-} ‘exist’, \textit{ik-} ‘go’, \textit{k-} ‘come’, \textit{jˀ-} ‘say’

\textit{imoor-} (exist/go/come.HON)  \textit{wur-} ‘exist’, \textit{ik-} ‘go’, \textit{k-} ‘come’

\textit{misjoor-} (eat.HON)  \textit{kam-} ‘eat’

\textit{moosɨr-} (die.HON)  \textit{sin-} ‘die’

The speaker TM said that \textit{umoor-} is more traditonal than \textit{imoor-}. Actually, \textit{umoor-} is used more often than \textit{{}-imoor} in my texts. The example of \textit{umoor-} meaning ‘come’ was already shown in (8-21 a). I will present other examples where \textit{umoor-} means ‘go,’ ‘exist,’ or ‘say.’

(8\nobreakdash-22)  Lexical honorific verb \textit{umoor-}

  a.  Meaning ‘go’

    [Context: US thought that the present author went to the house of TM, who is \textit{cɨnəə} ‘Tsune’ in the following example.]

    US:  cɨnəə  məə  xxx  saki  umoocjɨdarocjɨ  umutɨga,

      \textit{cɨnəə}  \textit{məə}    \textit{saki}  \textit{umoor{}-tɨ=daroo=ccjɨ  umuw-tɨ=ga}

      Tsune  front    first  go.HON-CSN=SUPP=QT  think-SEQ=FOC

      ‘(I) thought that (he) probably went to Tsune’s place, and ...’

      [Co: 110328\_00.txt]

  b.  Meaning ‘exist’

    [Context: Talking about the present author]

    US:  jonesigetaaga  wutan  jaanan  

      \textit{jonesige-taa=ga}  \textit{wur-tar-n}  \textit{jaa=nan}  

      Yoneshige-PL=NOM  exist-PST-PTCP  house=LOC1  

      umoojunwake?

      \textit{umoor{}-jur-n=wake}

      exist.HON-UMRK-PTCP=CFP

      ‘Is (he) in the house where Yoneshige and his family lived?’

      [Co: 110328\_00.txt]

  c.  Meaning ‘say’

    [Context: Talking about an incantation old people chanted when they felt the earthquakes]

    TM:  naakja\footnote{The regular process must be \textit{naakja-a} (2.HON.PL-ADNZ) > /naakjaa/, but it becomes /naakja/ in this example.}  anmataa  zisinnu  tuki,  zisinnu  sɨboo\footnote{The regular process must be \textit{sɨr-boo} (do-CND) > /sɨbboo/ (or /sɨppoo/), but it becomes /sɨboo/ in this example.},

      \textit{naakja-a}  \textit{anmaa-taa}  \textit{zisin=nu}  \textit{tuki}  \textit{zisin=nu}  \textit{sɨr-boo}

      2PL-ADNZ  mother-PL  earthquake=GEN  time  earthquake=NOM  do-CND

      kjon  cɨkɨ  kjon  cɨkɨcjəə  umoorantɨ?

      \textit{kjoo=n}  \textit{cɨk-ɨ}  \textit{kjoo=n}  \textit{cɨk-ɨ=ccjɨ=ja}  \textit{umoor{}-an-tɨ}

      Kyoto=DAT1  attach-IMP  Kyoto=DAT1  attach-IMP=TOP  say.HON-NEG-SEQ

      ‘Did your mother say, “Send (it) to Kyoto! Send (it) to Kyoto!” [lit. “Attach to Kyoto! Attach to Kyoto!”], when (they) feel earthquakes, (at) the time of earthquakes?’

      [Co: 110328\_00.txt]

In (8-22 a), \textit{umoor-} expresses the speaker US’s respect for the subject, although it did not overtly appear in the clause. The subject indicates the present author, who was younger than US, but the academic prestige of the university seems to have made her use honorific verbs. In (8-22 b), \textit{umoor-} expresses the speaker US’s respect for the (not appearing) subject, i.e. the present author. In (8-22 c), the speaker TM expresses the respect for /naakja anmataa/ ‘your mother,’ i.e. US’s mother.

Next, I will present an example of \textit{misjoor-} (eat.HON).

(8\nobreakdash-23)  Lexical honorific verb \textit{misjoor-} (eat.HON)

  [Context: Talking about the present author]

  US:  misjoorankai?

    \textit{misjoor{}-an=kai}

    eat.HON-NEG=DUB

    ‘Does (he) eat (the snacks US brought)?’

    [Co: 110328\_00.txt]

In (8-23), \textit{misjoor-} (eat.HON) expresses the speaker’s respect for the (not appearing) subject, i.e. the present author.

  Finally, I will present an example is of \textit{moosɨr-} (die.HON).

(8\nobreakdash-24)  Lexical honorific verb \textit{moosɨr-} (die.HON)

  [Context: Talking about TM’s friend who is older than her]

  TM:  kunəəda  tacuuga  moosjarooga.

    \textit{kunəəda}  \textit{tacuu=ga}  \textit{moosɨr{}-tar-oo=ga}

    the.other.day  Tatsu=NOM  die.HON-PST-SUPP=CFM3

    ‘(You) probably (know that) the other day, Tatsu passed away.’

    [Co: 120415\_00.txt]

In (8-24), \textit{moosɨr-} (die.HON) expresses the speaker’s respect for the subject, i.e. \textit{tacuu} ‘Tatsu,’ who was older than the speaker. If you want to express a more respect than that expressed by \textit{moosɨr-} (die.HON), you may use the light verb construction where the complement slot is filled by \textit{umoor-an} (exist.HON-NEG) and the light verb is \textit{nar-} ‘become’ as in (9-39 a) in \sectref{sec:key:9.1.2.2.}

  The speaker TM said that there is a lexical honorific verb that shows the speaker’s respect for the recepient (not the subject): \textit{huur-} (give.back.HON) ‘give (something) back.’ However, this honorific verb has never appeared in my texts. The same form can be used in my texts to mean ‘send (somebody) off,’ but it does not express the speaker’s respect to anyone. In other words, it is not a honorific verb.

\subsubsection{Auxiliary honorific verb}

There are two auxiliary honorific verbs in Yuwan.

\begin{styleBeschriftung}
\textmd{\tabref{tab:key:73}}\textmd{ Auxiliary honorific verbs}
\end{styleBeschriftung}

Auxiliary honorific verbs  Relevant non-honorific verbs

\textit{moor-} (HON)  N/A

\textit{taboor-} (BEN.HON)  \textit{kurɨr-} (BEN)

\textit{umoor-} (come.HON)  \textit{k-} ‘come’

The auxiliary honorific verbs in \tabref{tab:key:73} need to be preceded by a lexical verb, and the lexical verb always takes \textit{{}-tɨ} (SEQ) (see \sectref{sec:key:9.1.1} for more details). \textit{moor-} (HON) is used just to add an honorific meaning to the preceding verb. In other words, \textit{moor-} (HON) is an auxiliary honorific verb that is semantically unmarked. On the contrary, \textit{taboor-} (BEN.HON) and \textit{umoor-} (come.HON) add other meanings besides the honorific meaning. First, I will present examples of \textit{moor-} (HOM).

\textbf{(8\nobreakdash-25)}  Auxiliary honorific verb \textit{moor-} (HON)

  \textbf{a.}  [Context: Speaking to US]

    TM:  gazjumaru  sicjɨ  moojuijojaa.

      \textit{gazjumaru}  \textit{sij-tɨ}  \textit{moor{}-jur-i=joo=jaa}

      banyan.tree  [know-SEQ  HON-UMRK]=CFM1=SOL

        [Lex. verb  Aux. verb]\textsubscript{VP}

      ‘(You) would know the banyan tree, wouldn’t you?’

      [Co: 110328\_00.txt]

  b.  [Context: Speaking to US, whose family used to deal in fish] = (6-99 b)

    TM:  naakjaga  sjɨ  moojuinnja,  simanu

      \textit{naa-kja=ga}  \textit{sɨr-tɨ}  \textit{moor{}-jur-i=n=ja  sima=nu}

      2.HON-PL=NOM  [do-SEQ  HON-UMRK-INF]=DAT1=TOP  island=GEN

        [Lex. verb  Aux. verb]\textsubscript{VP}  

      jˀudarooga?

      \textit{jˀu=daroo=ga}

      fish=SUPP=CFM3

      ‘When you dealt in (fishes), (I) suppose (they are) fishes from the community [i.e. fish caught around the community].’

      [Co: 110328\_00.txt]

In (8-25 a), \textit{moor-} (HON) expresses the speaker’s respect for the subject of the predicate, i.e. the hearer US. In (8-25 b), \textit{moor-} (HON) expresses the speaker’s respect for the subject of the predicate, i.e. US’s family.

  The next example is \textit{taboor-} (BEN.HON). \textit{taboor-} (BEN.HON) adds not only a honorific meaning to the preceding verb, but also expresses that the event expressed by the preceding verb is to the speaker’s benefit.

\textbf{(8\nobreakdash-26)}  Auxiliary honorific verb \textit{taboor-} (BEN.HON)

  TM:  {\textbar}sinsjei{\textbar},  an  kˀwa  abɨtɨ  taboorɨ.

    \textit{sinsjei}  \textit{a-n}  \textit{kˀwa}  \textit{abɨr-tɨ}  \textit{taboor{}-ɨ}

    teacher  DIST-ADNZ  child  [call-SEQ  BEN.HON-IMP]

          [Lex. verb  Aux. verb]\textsubscript{VP}

    ‘Teacher, would (you) please call that child (for me)?’

    [El: 130820]

In (8-26), \textit{taboor-} (BEN.HON) expresses the speaker’s respect for the subject of the predicate, i.e. \textit{sinsjei} ‘teacher.’ Additionally, \textit{taboor-} (BEN.HON) expresses that the action indicated by the preceding lexical verb \textit{abɨr-} ‘call’ is beneficial to the speaker (see \sectref{sec:key:9.1.1.3} for more details).

  Finally, the auxiliary verb \textit{umoor-} (come.HON) is shown below.

\textbf{(8\nobreakdash-27)}  Auxiliary honorific verb \textit{umoor-} (come.HON)

  [Context: Talking about the present author]

  US:  urɨn  tazɨnɨtɨ  umoocjattu,

    \textit{u-rɨ=n}  \textit{tazɨnɨr-tɨ}  \textit{umoor{}-tar-tu}

    MES-NLZ=also  [ask-SEQ  come.HON-PST-CSL]

      [Lex. verb  Aux. verb]\textsubscript{VP}

    ‘(He) came and ask (me) of that, so ...’

    [Co: 110328\_00.txt]

In (8-27), \textit{umoor-} (come.HON) expresses the speaker’s respect for the subject of the predicate, i.e. the present author. The verbal form /umoor-/ can also be used as a lexical honorific verb as in \tabref{tab:key:72}, and the lexical verb \textit{umoor-} can mean several meanings such as ‘exist (honorific),’ or ‘go (honorific).’ Therefore, the honorific auxiliary verb \textit{umoor-} may also mean those meanings. So far, however, I have found only the meaning of ‘come (honorific)’ as in (8-27) in my texts.

\subsection{Existential verb}

Semantically, the existential verbs in Yuwan express the existence of a core argument. The “core argument” here usually indicates the subject of a clause, but sometimes it does not, which is discussed in \sectref{sec:key:8.3.2.4.} Syntactically, the existential verbs fill the predecate phrase of a clause, and makes a verbal predicate phrase (see \sectref{sec:key:9.1} about the verbal predicate phrase). Yuwan has three existential verbs \textit{wur-}, \textit{ar-}, and \textit{nə-}, which correlate with the animacy (in a narrow sense) of the core arguments, which is summarized in the following table. A kind of possession can be expressed by the existential verbs, which will be discussed in \sectref{sec:key:8.3.2.4.}

\begin{styleBeschriftung}
\textmd{\tabref{tab:key:74}}\textmd{. Existential verbs (not in AVC)}
\end{styleBeschriftung}

Core NPs  Animate  Inanimate

Polarity  Affirmative / Negative  Affirmative  Negative

Existential verbs  \textit{wur-}  \textit{ar-}  \textit{nə-}

If the core argument is animate, \textit{wur-} ‘exist’ is used. If the core argument is inanimate, \textit{ar-} ‘exist’ or \textit{nə-} ‘exist’ is used. \textit{wur-} ‘exist’ can take negative affixes, but \textit{ar-} ‘exsit’ cannot. \textit{nə-} ‘exist’ always takes one of the negative affixes directly. The negative affixes are \textit{{}-an} (NEG) or \textit{{}-azɨi} (NEG.PLQ), which go through reduction or assimilation with \textit{nə-} ‘exist’ such as /nə-n/ (exist-NEG) or /nə-əzɨi/ (exist-NEG.PLQ). I present examples of \tabref{tab:key:74} in turn below: \textit{wur-} ‘exist’ in \sectref{sec:key:8.3.2.1}, \textit{ar-} ‘exist’ in 8.3.2.2, and \textit{nə-} ‘exist’ in \sectref{sec:key:8.3.2.3.}

\subsubsection{\textit{wur-} ‘exist’}

If the core argument of the clause indicates an animate referent, \textit{wur-} ‘exist’ is chosen as the existential verb (see \sectref{sec:key:8.3.2.4} about the core arguments of existential verbs). In (8-28 a-b), the core arguments are animate, i.e. \textit{anma-taa} ‘(such a person like my) mother’ and \textit{mukasi=nu} \textit{cˀju} ‘old people.’ Thus, \textit{wur-} ‘exist’ is used.

\textbf{(8\nobreakdash-28)}  Core argument is animate

  \textbf{a.}  Affirmative polarity

    TM:  anmataaga  wuppoojaa.

      \textit{anmaa-taa=ga}  \textit{wur{}-boo=jaa}

      mother-PL=NOM  exist-CND=SOL

      ‘If there were (my) mother.’

      [Co: 110328\_00.txt]

  \textbf{b.}  Negative polarity

    TM:  mukasinu  cˀjunkjoo  wuranbajaa.

      \textit{mukasi=nu}  \textit{cˀju=nkja=ja}  \textit{wur{}-an-ba=jaa}

      past=GEN  person=APPR=TOP  exist-NEG-CSL=SOL

      ‘There are no old people.’

      [Co: 101023\_01.txt]

Yuwan has several phenomena which is concerned with the animacy in a broad sense (see \sectref{sec:key:6.4}). The existential verbs, however, are chosen by the animacy in a narrow sense. Therefore, even if the referent is not a human but still is an animate referent, \textit{wur-} ‘exist’ (not \textit{ar-}) is chosen.

\textbf{(8\nobreakdash-29)}  Non-human animate subject

  [Context: Talking about silkworms that were in the silk-reeling factory in the community]

  TM:  namanu  cjoodo  an ...  kˀurusan

    \textit{nama=nu}  \textit{cjoodo}  \textit{a-n}  \textit{kˀuru-sa+ar-n}

    now=GEN  just  DIST-ADNZ  black-ADJ+STV-PTCP

    cjoocjonu,  (mmm)  arɨnu  wuncjɨjo.

    \textit{cjoocjo=nu}    \textit{a-rɨ=nu}  \textit{wur{}-n=ccjɨ=joo}

    butterfly=NOM    DIST-NLZ=NOM  exist-PTCP=QT=CFM1

    ‘(In those days) there were (moths of silkworms) just (like) that black butterfly (in these days), and that [i.e. the moths] actually existed.’

    [Co: 111113\_01.txt]

In (8-29), the core argument, which is also the subject, indicates a non-human animate referent, i.e. a butterfly, and still \textit{wur-} ‘exist’ is chosen. Similarly, the lexical honorific verb \textit{umoor-} (exist.HON), which is a honorific counterpart of \textit{wur-} ‘exist,’ can be used only when the core argument is animate as in (8-21 a) in \sectref{sec:key:8.3.1.}

\subsubsection{\textit{ar-} ‘exist’}

If the core argument of the clause indicates an inanimate referent and the predicate is in affirmative, \textit{ar-} ‘exist’ is chosen as the existential verb (see \sectref{sec:key:8.3.2.4} about the core arguments of existential verbs).

\textbf{(8\nobreakdash-30)}  Core argument is inanimate (affirmative polarity)

  TM:  hanankjanu  aijaa.

    \textit{hana=nkja=nu}  \textit{ar{}-i=jaa}

    flower=APPR=NOM  exist-NPST=SOL

    ‘There are flowers (in this picture).’

    [Co: 111113\_01.txt]

In (8-30), the core argument, which is also the subject, is an inanimate referent, i.e. \textit{hana} ‘flower,’ and also the clause is in affirmative. Thus, \textit{ar-} ‘exist’ is used. In principle, \textit{ar-} ‘exist’ conforms to the deletion of the final //r// before \textit{t}{}-initial affixes as in (8-31 a) (see \sectref{sec:key:8.2.1.2} for more details). However, it is sometimes not deleted, but assimilates to the following //t// as in (8-31 b).

\textbf{(8\nobreakdash-31)}  a.  TM:  dandannu  atɨjaa.

      \textit{dandan=nu}  \textit{ar-tɨ=jaa}

      step=NOM  exist-SEQ=SOL

      ‘There were steps (at the place in the picture).’

      [Co: 120415\_00.txt]

  b.  TM:  un  kabəə  nama  attɨjaa,  wanna.

      \textit{u-n}  \textit{kabi=ja}  \textit{nama}  \textit{ar-tɨ=jaa}  \textit{wan=ja}

      MES-ADNZ  paper=TOP  still  exist-SEQ=SOL  1SG=TOP

      ‘I still have the paper.’ [lit. ‘As for me, there were still papers.’]

      [Co: 110328\_00.txt]

So far, the assimilation of the root final //r// of \textit{ar-} ‘exist’ occurs only in the combination of \textit{ar-tɨ=jaa} (exist-SEQ=SOL), although it is not obligatory as in (8-31 a).

  Basically, \textit{ar-} ‘exist’ is used only in affirmative. However, there are two cases where \textit{ar-} ‘exist’ is used in negative. First, if the existential verb takes the politeness affix \textit{{}-jawur}, \textit{ar-} ‘exist’ is always used, no matter which polarity the predicate is in.

\textbf{(8\nobreakdash-32)}  \textit{ar-} ‘exist’ + \textit{{}-jawur} (POL)

  TM:  nun  ajawurandoo.

    \textit{nuu=n}  \textit{ar-jawur-an=doo}

    what=any  exist-POL-NEG=ASS

    ‘There is not anything.’

    [El: 1201xx]

In (8-32), the existential verb is in negative taking \textit{{}-an} (NEG), but the existentail verb is \textit{ar-} ‘exist’ (not \textit{nə-}).

  Secondly, if the existential verb fills the lexical verb slot in the auxiliary verb construction (see \sectref{sec:key:9.1.1}), it is always \textit{ar-} ‘exist,’ no matter which polarity the predicate is in.

\textbf{(8\nobreakdash-33)}  \textit{ar-} ‘exist’ in AVC [= (8-35 d)]

  TM:  an  sinsjeija  kanɨja  atɨ  moorancjɨdoo.

    \textit{a-n}  \textit{sinsjei=ja}  \textit{kanɨ=ja}  \textit{ar-tɨ}  \textit{moor-an=ccjɨ=doo}

    DIST-ADNZ  teacher=TOP  money=TOP  [exist-SEQ  HON-NEG]=QT=ASS

          [Lex. verb  Aux. verb]\textsubscript{VP}

    ‘That teacher does not have any money.’

    [El: 120924]

In (8-33), the VP that contains an existential verb is in negative, but the existentail verb is \textit{ar-} ‘exist’ (not \textit{nə-}).

\subsubsection{\textit{nə-} ‘exist’}

If the core argument of the clause indicates an inanimate referent and the predicate is in negative, \textit{nə-} ‘exist’ is chosen as the existential verb (with the exception of a few cases discussed in \sectref{sec:key:8.3.2.2}) (see \sectref{sec:key:8.3.2.4} about the core arguments of existential verbs).

\textbf{(8\nobreakdash-34)}  Core argument is inanimate (negative polarity)

  \textit{{}-an} (NEG)

  a.  [Context: TM told that she cannot move her tongue very well.]

    TM:  han  nənba.

      \textit{haa=n}  \textit{nə-an{}-ba}

      teeth=also  exist-NEG-CSL

      ‘Also, I don’t have any teeth.’

      [Co: 110328\_00.txt]

  b.  TM:  umanannja  nənnən,

      \textit{u-ma=nan=ja}  \textit{nə-an{}-nən}

      MES-place=LOC1=TOP  exist-NEG-SEQ

      ‘(The storehouse) did not exist there, and ...’

      [Co: 120415\_00.txt]

  \textit{{}-azɨi} (NEG.PLQ)

  c.  [Context: TM and MS were looking for a pounder.]

    TM:  nəəzɨi?  umanannja?

      \textit{nə-azɨi}  \textit{u-ma=nan=ja}

      exist-NEG.PLQ  MES-place=LOC1=TOP

      ‘Isn’t (it there)? At the place?’

      [Co: 120415\_00.txt]

Strictly speaking, \textit{nə-} ‘exist’ is obligatorily chosen when it is directly followed by the negative affixes. Therefore, if the negative affixes cannot directly follow the existential verbal stems, \textit{nə-} ‘exist’ cannot be chosen, and instead \textit{ar-} ‘exist’ is chosen as in (8-32) and (8-33) in \sectref{sec:key:8.3.2.2.}

\subsubsection{Core argument of the existential verbs}

The choice of existential verbs is determined by the core arguments in the clauses, and the core arguments do not necessarily indicate the subjects of the clauses. I present examples below, where the existential verbs are used to mean possessional meaning. Roughly speaking, the construction literally meaning ‘About X, there is Y’ means ‘X has Y.’ Besides, \textit{umoor-} (exist.HON) in the following examples is a honorific lexical verb, whose non-honorific counterpart is \textit{wur-} ‘exist.’ Therefore, the core argument of \textit{umoor-} (exist.HON) must indicate an animate referent. In the following examples, the core arguments and existential verbs are underlined.

\textbf{(8\nobreakdash-35)}  a.  \textit{umoor-} (core argument is animate)

    TM:  an  sinsjeija  jɨɨija  umoorancjɨdoo.

      \textit{a-n}  \textit{sinsjei=ja}  \textit{jɨɨi=ja}  \textit{umoor{}-an=ccjɨ=doo}

      [DIST-ADNZ  teacher]=TOP  brother=TOP  [exist.HON-NEG]=QT=ASS

      [Subject]      [Honorific verb]

      ‘That teacher does not have a brother.’

      [El: 120924]

  b.  \textsuperscript{\#}\textit{umoor-} (core argument is animate)

    TM:  \textsuperscript{\#}an  warabɨja  jɨɨija  umoorancjɨdoo

      \textit{a-n}  \textit{warabɨ=ja}  \textit{jɨɨi=ja}  \textit{umoor{}-an=ccjɨ=doo}

      [DIST-ADNZ  child]=TOP  brother=TOP  [exist.HON-NEG]=QT=ASS

      [Subject]      [Honorific verb]

      (Intended meaning) ‘That child does not have any money.’

      [El: 140227]

  c.  *\textit{umoor-} (core argument is inanimate)

    TM:  *an  sinsjeija  kanɨja  umoorancjɨdoo

      \textit{a-n}  \textit{sinsjei=ja}  \textit{kanɨ=ja}  \textit{umoor{}-an=ccjɨ=doo}

      [DIST-ADNZ  teacher]=TOP  money=TOP  [exist.HON-NEG]=QT=ASS

      [Subject]      [Honorific verb]

      (Intended meaning) ‘That teacher does not have any money.’

      [El: 120924]

  d.  \textit{ar-} (core argument is inanimate)

    TM:  an  sinsjeija  kanɨja  atɨ  moorancjɨdoo.

      \textit{a-n}  \textit{sinsjei=ja}  \textit{kanɨ=ja}  \textit{ar{}-tɨ  moor-an=ccjɨ=doo}

      [DIST-ADNZ  teacher]=TOP  money=TOP  exist-SEQ  [HON-NEG]=QT=ASS

      [Subject]        [Honorific verb]

      ‘That teacher does not have any money.’

      [El: 120924]

In (8-35 a), the subject of the clause is \textit{sinsjei} ‘teacher,’ which is clear from the unacceptability of (8-35 b). The difference between (8-35 a) and (8-35 b) is only on the subjects of the clauses (see also Chapter 3). On the contrary, the difference between (8-35 a) and (8-35c) is only on the core arguments immediately preceding the predicates, i.e. \textit{jɨɨi} ‘brother’ and \textit{kanɨ} ‘money.’ As mentioned before, the core argument of \textit{umoor-} (exist.HON) must indicate an animate referent. Thus, (8-35 c) is ungrammatical since the core argument, i.e. \textit{kanɨ} ‘money,’ is inanimate. If we replace \textit{umoor-} (exist.HON) in (8-35 c) with \textit{ar-tɨ} \textit{moor-} (exist-SEQ HON), which is a honorific expression of \textit{ar-} ‘exist’ (see \sectref{sec:key:8.3.2.2}), as in (8-35 d), the sentence can be grammatical, since \textit{ar-} ‘exist’ may take an inanimate core argument. These examples show that the core argument of the existential verbs is sometimes different from the subject.

\subsection{Copula verbs}

Syntactically, the copula verb in Yuwan fills the predecate phrase together with an NP, and makes a nominal predicate (see \sectref{sec:key:9.3} for more details). Yuwan has four copula verbs, i.e. \textit{jar-}, \textit{zjar-}, \textit{nar-} and \textit{ar-}, and they correlate with the polarity of the predicates in principle.

  \textit{jar-}, \textit{zjar-}, and \textit{nar-} appear only in affirmative, and \textit{ar-} appears basically in negative. Syntactically, the copula verbs always follow an NP, but there is a case where \textit{ar-} (COP) can appear only by itslef (see \sectref{sec:key:8.3.3.3} for more details). Basically, the NP followed by \textit{ar-} (COP) in the predicate phrase takes \textit{ja} (TOP) in the main clause. However, there are some cases where the NP preceding \textit{ar-} (COP) takes the nominative case in a subordinate clause (see \sectref{sec:key:9.3.3.1} for more details).

  If the copula does not take any negative affix, one of the copula verbs, i.e. \textit{jar-}, \textit{zjar-}, or \textit{nar-} is chosen. Among them, \textit{jar-} (COP) is most productive, i.e., it can be followed by many kinds of verbal affixes. Interestingly, the copula verbs can take particular inflectional affixes directly, and the distinction between Group-I affixes and Group-II affixes in \sectref{sec:key:8.1} is neutralized here. I will present the verbal affixes that can directly follow the copula roots in \tabref{tab:key:75}. “+” indicates the copula roots can be followed by the right-most verbal affixes.

\begin{styleBeschriftung}
\textmd{\tabref{tab:key:75}}\textmd{. The possible combinations of the copula roots and verbal affixes}
\end{styleBeschriftung}

Copula roots  Verbal affixes

\textit{jar-}  \textit{ar-}  \textit{nar-}  \textit{zjar-}  Finite-form affixes

+        \textit{{}-tar} (PST)

+        \textit{{}-oo} (SUPP)

  +      \textit{{}-u} (PFC)

  +      \textit{{}-azɨi} (NEG.PLQ)

\textit{jar-}  \textit{ar-}  \textit{nar-}  \textit{zjar-}  Participial affixes

+      +  \textit{{}-n} (PTCP)

  +      \textit{{}-an} (NEG)

\textit{jar-}  \textit{ar-}  \textit{nar-}  \textit{zjar-}  Converbal affixes

+  +  +    \textit{{}-tɨ} (SEQ)

+        \textit{{}-tai} (LST)

+        \textit{{}-ba} (CSL)

+        \textit{{}-boo} (CND)

+      +  \textit{{}-sa} (POL)

+      +  \textit{{}-sɨga} (POL)

\textit{jar-}  \textit{ar-}  \textit{nar-}  \textit{zjar-}  Derivational affix

+        \textit{{}-təər} (RSL)

The above table shows the following facts: (a) \textit{jar-} (COP) can precede every verbal affix in \tabref{tab:key:75}, with the exception of the negative affixes, i.e. \textit{-an} (NEG) and \textit{{}-azɨi} (NEG.PLQ), and \textit{{}-u} (PFC); (b) the negative affixes always take \textit{ar-} (COP); (c) \textit{nar-} takes only \textit{{}-tɨ} (SEQ). In \tabref{tab:key:75}, the environments where \textit{zjar-} (COP) appears are very restricted. However, it does not mean that \textit{zjar-} (CIP) is hardly used in Yuwan. In fact, \textit{zjar-} (COP) often appears in other environments, where the nominal predicate is followed by the particles \textit{jaa} (SOL) or \textit{ga} (CFM3), or without any affix nor particle (see \sectref{sec:key:8.3.3.2}).

The following subsections will discuss each copula verbal root: \textit{jar-} (COP) in \sectref{sec:key:8.3.3.1}, \textit{zjar-} (COP) in \sectref{sec:key:8.3.3.2}, and \textit{ar-} (COP) in \sectref{sec:key:8.3.3.3.} The three copula verbal roots \textit{nar-} (COP), \textit{jar-} (COP), and \textit{ar-} (COP) can take \textit{{}-tɨ} (SEQ), and the differences among them are discussed in \sectref{sec:key:8.3.3.4.} Additionally, \textit{zjar-} (COP) can take the same affixes as \textit{jar-} (COP), the detail of which will be discussed in \sectref{sec:key:8.3.3.5.}

\subsubsection{\textit{jar-} (COP)}

All of the combinations of \textit{jar-} (COP) and verbal inflectional affixes are shown below, with the exception of the cases discussed in \sectref{sec:key:8.3.3.4} and \sectref{sec:key:8.3.3.5.}

\textbf{(8\nobreakdash-36)}  a.  \textit{{}-tar} (PST)

    [Context: Speaking about acquaintances of TM and MS; TM: ‘Muha is as old as those people,and...’]

    TM:  muru  dusi  jata.

      \textit{muru}  \textit{dusi}  \textit{jar-tar}

      very  friend  COP-PST

      ‘(They) were very (good) friends.’

      [Co: 120415\_00.txt]

  b.  \textit{{}-oo} (SUPP)

    TM:  ukka  cugəə,  mata,  (maga,)  maga  jaroo.

      \textit{u-rɨ=ga}  \textit{cugi=ja}  \textit{mata}  \textit{maga}  \textit{maga}  \textit{jar-oo}

      MES-NLZ=GEN  next=TOP  again  grandchild  grandchild  COP-SUPP

      ‘About the next (scene) after that, again, probably (it is) a grandchild.’

      [PF: 090827\_02.txt]

  c.  \textit{{}-tai} (LST)

    TM:  gan  sjɨ  jatai,

      \textit{ga-n}  \textit{sɨr-tɨ}  \textit{jar-tai}

      MES-ADVZ  do-SEQ  COP-LST

      ‘(It) is like that, and ...’

      [El: 120921]

  d.  \textit{{}-ba} (CSL)

    TM:  tawuja  tawu  jappa.

      \textit{tawu=ja}  \textit{tawu}  \textit{jar-ba}

      plain=TOP  plain  COP-CSL

      ‘The plain is (actually) plain, so ...’

      [PF: 090222\_00.txt]

  e.  \textit{{}-boo} (CND)

    [Context: TM remembered a story that her acquaintance told in the speech contest spoken in the dialects in Amami before.]

    TM:  urɨ  jappoo,  cjoo  ukkarajo.

      \textit{u-rɨ}  \textit{jar-boo}  \textit{cjoo}  \textit{u-rɨ=kara=joo}

      MES-NLZ  COP-CND  just  MES-NLZ=ABL=CFM1

      ‘If (it) is that [i.e. If I tell the story remembering his talk], (it begins) just from that (scene).’

      [Fo: 090307\_00.txt]

Additionally, \textit{jar-} (COP) can take the derivational affix \textit{{}-təər} (RSL). The combination \textit{jar-} (COP) and \textit{{}-təər} (RSL) can take either \textit{{}-i} (NPST) or \textit{{}-tu} (CSL) as in (8-37).

\textbf{(8\nobreakdash-37)}  \textit{{}-təər} (RSL)

  \textbf{a.}  TM:  an  gazimarunu  appoo,  naa,  huntoo,  naa,

      \textit{a-n}  \textit{gazimaru=nu}  \textit{ar-boo}  \textit{naa}  \textit{huntoo}  \textit{naa}

      DIST-ADNZ  banyan.tree=NOM  exist-CND  FIL  real  FIL

      urɨkusa,  naa,  {\textbar}nippon.ici{\textbar}  jatəijoo.

      \textit{u-rɨ=kusa}  \textit{naa}  \textit{nippon+ici}  \textit{jar-təər-i=joo}

      MES-NLZ=just  FIL  Japan+one  COP-RSL-NPST=CFM1

      ‘If that banyan tree existed, it would be number one in Japan.’

      [Co: 111113\_02.txt]

  \textbf{b.}  TM:  uziitu  waakjaa  anmaatu, ...  mukasiutankja

      \textit{uzii=tu}  \textit{waakja-a}  \textit{anmaa=tu}  \textit{mukasi+uta=nkja}

      grandfather=COM  1PL-ADNZ  mother=COM  past+song=APPR

      nunkuin  zjoozɨ  jatəttujaa.

      \textit{nuu=n=kui=n}  \textit{zjoozɨ}  \textit{jar-təər-tu=jaa}

      what=any=INDF=any  good.at  COP-RSL-CSL=SOL

      ‘(MS’s) grandfather and my mother were good at everything.’

      [Co: 111113\_02.txt]

  The other combinations made from \textit{jar-} (COP) with other affixes are shown in \sectref{sec:key:8.3.3.4} and \sectref{sec:key:8.3.3.5.}

\subsubsection{\textit{zjar-} (COP)}

\textit{zjar-} (COP) may appear when the nominal predicate is followed by nothing as in (8-38 a). On the other hand, \textit{zjar-} (COP) always appears when the nominal predicate is followed by \textit{jaa} (SOL) or \textit{ga} (CFM3) in the non-past tense and in affirmative as in (8-38 b-c) (see \sectref{sec:key:4.1.3.3} for more details).

(8\nobreakdash-38)  a. Followed by nothing

  TM:  kurɨ  jamatuhuui  zja.

    \textit{ku-rɨ}  \textit{jamatu+huu-i}  \textit{zjar}

    PROX-NLZ  mainland.Japan+see.off-INF  COP

    ‘This is (the scene of) seeing off (the people who go to) mainland Japan.’

    [Co: 111113\_01.txt]

  b. Followed by \textit{jaa} (SOL)

  TM:  kurəə  (eee)  sjenzjen  ucɨsjən  mun  zjajaa.

    \textit{ku-rɨ=ja}    \textit{sjenzjen}  \textit{ucɨs-təər-n}  \textit{mun}  \textit{zjar=jaa}

    PROX-NLZ=TOP    before.war  take-RSL-PTCP  thing  COP=SOL

    ‘This is the thing [i.e. the picture] taken before the war.’

    [Co: 111113\_02.txt]

  c. Followed by \textit{ga} (CFM3) [= (6-19 a)]

  TM:  umanuhazɨ  zjaga.

    \textit{u-ma=nu=hazɨ}  \textit{zjar=ga}

    MES-place=GEN=certatinty  COP=CFM3

    ‘(The place you are speaking of) must be there.’

    [Co: 111113\_01.txt]

These examples show that if \textit{zjar-} (COP) is followed by particles, it does not take any affix. In other words, \textit{zjar-} (COP) behaves like a particle by itsef (not like a verb taking an inflectional affix). Actually, the stem-final //r// of \textit{zjar-} (COP) appears only when it is followed by \textit{{}-sa} (POL) (or \textit{{}-sɨga} (POL)) as in (8-45 b) in \sectref{sec:key:8.3.3.5}, where the assimilation from //r// to /s/ occurs. The stem-final //r// had been deduced from the following two facts: \REF{ex:key:1} other copula verbs, especially, \textit{jar-} (COP) and \textit{ar-} (COP), have the stem-final //r//, which appears even in the surface forms, e.g. /jaroo/ \textit{jar-oo} (COP-SUPP) as in (8-36 b) in \sectref{sec:key:8.3.3.1} or /aran/ \textit{ar-an} (COP-NEG) as in (8-39 a) in \sectref{sec:key:8.3.3.3}; \REF{ex:key:2} the most productive verbal stem-final morphophoneme is //r// in Yuwan. In fact, \textit{zjar-} (COP) seems to be in the process of grammaticalization to become a particle. Interestingly, the younger generation (in their sixties in 2013) use the same copula form \textit{zjar-} in any case in the non-past tense, e.g. /zjappoo/ \textit{zjab-boo} (COP-CND) (not /jappoo/ as in the older generation).

\subsubsection{\textit{ar-} (COP)}

\textit{ar-} (COP) usually takes one of the negative affixes, i.e. \textit{{}-an} (NEG) or \textit{{}-azɨi} (NEG.PLQ) as in (8-39 a-c), with the exception of the cases where \textit{ar-} (COP) takes \textit{{}-u=i} (PFC=PLQ) as in (8-39 d) or \textit{{}-tɨ} (SEQ) in AVC (see \sectref{sec:key:8.3.3.4}).

(8\nobreakdash-39)  \textit{{}-an} (NEG)

  a.  TM:  kurəə  (an ..)  kazumataaja  aranna?

      \textit{ku-rɨ=ja}  \textit{a-n}  \textit{kazuma-taa=ja}  \textit{ar-an=na}

      PROX-NLZ=TOP  DIST-ADNZ  Kazuma-PL=TOP  COP-NEG=PLQ

      ‘Isn’t this [i.e. the scene in the picture] (about) Kazuma and his friends?’

      [Co: 120415\_00.txt]

  b.  TM:  jakubaja  arannən,  xxx  {\textbar}kendoo{\textbar}daroo.

      \textit{jakuba=ja}  \textit{ar-an{}-nən    kendoo=daroo}

      village.office=TOP  COP-NEG-SEQ    prefectural.road=SUPP

      ‘(It) is not the village office, but (it is) the prefectural road.’

      [Co: 120415\_00.txt]

  \textit{{}-azɨi} (NEG.PLQ)

  c.  TM:  kurəə  hakaja  arazɨi?

      \textit{ku-rɨ=ja}  \textit{haka=ja}  \textit{ar-azɨi}

      PROX-NLZ=TOP  tomb=TOP  COP-NEG.PLQ

      ‘Isn’t this a tomb?’

      [Co: 120415\_01.txt]

  \textit{{}-u=i} (PFC=PLQ)

  d.  TM:  arəə  akiradu  arui?

      \textit{a-rɨ=ja}  \textit{akira=du}  \textit{ar-u=i}

      DiST-NLZ=TOP  Akira=FOC  COP-PFC=PLQ

      ‘Is that person Akira?’

      [El: 130822]

  In principle, the copula verbs need a preceding NP in order to fill in the nominal predicate phrases (see \sectref{sec:key:9.3}). However, the copula form \textit{ar-an} (COP-NEG) can be uttered only by itself as in (8-40).

(8\nobreakdash-40)  Independent use of \textit{ar-an} (COP-NEG)

  [Context: Conversation between MY and TM]

  MY:  miicɨdu  cigajurooga?

    \textit{miicɨ=du}  \textit{cigaw-jur-oo=ga}

    three.thing=FOC  different-UMRK-SUPP=CFM3

    ‘Probably, (you) are three years younger (than she)?’

  TM:  aran.

    \textit{ar-an}

    COP-NEG

    ‘No.’

    [Co: 110328\_00.txt]

In (8-40), MY asked TM if TM was three years younger than US, and TM answered negatively. This example shows that \textit{ar-an} (COP-NEG) can be used only by itself as a negative reply to a polar question.

  Furthermore, \textit{ar-an} (COP-NEG) can relativize its subject without any predicative NP as in (8-41).

(8\nobreakdash-41)  TM:  wanga  kicjuncjɨ  umutɨdu,  urattəə

    \textit{wan=ga}  \textit{kik-tur-n=ccjɨ}  \textit{umuw-tɨ=du}  \textit{urattəə}

    1SG=NOM  hear-PROG-PTCP=QT  think-SEQ=FOC  2.NHON.DU

    gan  sjan  aran  hanasi  sjaroogai?

    \textit{ga-n}  \textit{sɨr-tar-n}  \{[\textit{ar-an}]\textsubscript{Adnominal clause}  \textit{hanasi}\}\textsubscript{NP}  \textit{sɨr-tar-oo=ga=i}

    MES-ADNZ  do-PST-PTCP  COP-NEG  tale  do-PST-SUPP=CFM4=PLQ

    ‘Probably you told the unlikely tale like that since (you) thought that I was listening to (that), didn’t you?’

    [Fo: 090307\_00.txt]

In (8-41), the head of the NP, i.e. \textit{hanasi} ‘tale,’ is modified by the adnominal clause that is only filled by a copula verb \textit{ar-an} (COP-NEG), which means ‘unlikely’ in this example. The literal translation of the NP is ‘a tale not being,’ where the so-called “copula complement” cannot be recovered. In other words, \textit{ar-an} (COP-NEG) in this example means ‘unlikely’ only by itself. The preceding words, i.e. /gan sjan/ \textit{ga-n} \textit{sɨr-tar-n} (MES-ADNZ do-PST-PTCP) ‘like that,’ are not the copula complement of \textit{ar-an} (COP-NEG); in fact, they form another adnominal clasue that modifies the following NP.

\subsubsection{\textit{{}-tɨ} (SEQ) with \textit{nar-} (COP), \textit{ar-} (COP), and \textit{jar-} (COP)}

It should be noted that \textit{{}-tɨ} (SEQ) can be preceded by three types of copula roots, i.e. \textit{nar-} (COP), \textit{ar-} (COP), and \textit{jar-} (COP).

First, \textit{nar-} (COP) plus \textit{{}-tɨ} (SEQ) expresses the reason.\footnote{This remark owes to the grammar sketch of Kamikatetsu (Nothern Ryukyuan) (\citealt{ShirataEtAl2011}: 146).}

\textbf{(8\nobreakdash-42)}  \textit{nar-} (COP) + \textit{{}-tɨ} (SEQ)

  \textbf{a.}  TM:  naacɨbaa  natɨ,  ucizjasiga  dɨkɨranba.

      \textit{naacɨbaa}  \textit{nar-tɨ}  \textit{ut-i+izjas-i=ga}  \textit{dɨkɨr-an-ba}

      tone.deafness  COP-SEQ  hit-INF+put.out-INF=NOM  able.to.do-NEG-CSL

      ‘(I) am tone deaf, so (I) am not able to start hitting (the hand drums in singing and dancing with the traditonal songs).’

      [Co: 111113\_01.txt]

  \textbf{b.}  [= (4-13 c)]

    TM:  jusɨga  sɨki  natɨjoo,

      \textit{jusɨr-Ø=ga}  \textit{sɨki}  \textit{nar-tɨ=joo}

      teach-INF=NOM  fond  COP-SEQ=CFM1

      ‘(My mother) was fond of teaching, so (everyone came to learn the traditional songs from my mother).’

      [Co: 111113\_02.txt]

In (8-42 a), \textit{naacɨbaa} ‘a tone deaf’ and \textit{nar-} (COP) express that the speaker is a member of the people who are tone deaf, and with \textit{{}-tɨ} (SEQ) they express the reason for the speaker’s incapability of hitting drums in singing. In (8-42 b), \textit{sɨkɨ} ‘fond’ and \textit{nar-} (COP) express that the speaker’s mother was fond of teaching, and with \textit{{}-tɨ} (SEQ) they express the reason why everyone came to her place.

  Second, although \textit{ar-} (COP) is used with negative affixes in principle (see \sectref{sec:key:8.3.3.2}), there is a case where \textit{ar-} (COP) appears in another environment, i.e. the auxiliary verb constructoin (see also \sectref{sec:key:9.1.1}).

(8\nobreakdash-43)  \textit{ar-} (COP) + \textit{{}-tɨ} (SEQ) in AVC

  a.  US:  {\textbar}niizimasanto  otoosan{\textbar}taaga  {\textbar}kjoodai{\textbar}  atɨ  moojukkai?

      \textit{niizima-san=to}  \textit{otoosan-taa=ga}  \textit{kjoodai}  \textit{ar-tɨ}  \textit{moor-jur=kai}

      Niijima{}-HON=COM  father{}-PL=NOM  brother  [COP-SEQ  HON-UMRK]=DUB

            [Lex. verb  Aux. verb]\textsubscript{VP}

      ‘Are Mr. Niijima and (the author’s) father brothers?’

      [Co: 110328\_00.txt]

  b.  TM:  an  cˀjoo  sinsjei  atɨ  moojunnja?

      \textit{a-n}  \textit{cˀju=ja}  \textit{sinsjei}  \textit{ar-tɨ}  \textit{moor-jur-i=na}

      DIST-ADNZ  person=TOP  teacher  [COP-SEQ  HON-UMRK-NPST]=PLQ

            [Lex. verb  Aux. verb]\textsubscript{VP}

      ‘Is that person a teacher?’

      [El: 130820]

The above examples show that the copula \textit{ar-} (COP) is always followed by \textit{{}-tɨ} (SEQ) when it fills the lexical verb slot in the AVC.

  Finally, \textit{jar-} (SEQ) is also followed by \textit{{}-tɨ} (SEQ). In the non-sentence-final position, \textit{jar-} (COP) plus \textit{{}-tɨ} (SEQ) is always followed by \textit{n} ‘even’ as in (8-44 a) showing the meaning such as ‘even if’ (see also \sectref{sec:key:10.1.3}). In the sentence-final position, \textit{jar-} (COP) plus \textit{{}-tɨ} (SEQ) expresses both of the past tense and the lack of perceived certainty as in (8-44 b-c) (see also \sectref{sec:key:11.2.1} about insubordination).

(8\nobreakdash-44)  \textit{jar-} (COP) + \textit{{}-tɨ} (SEQ)

  Non-sentence-final position

  \textbf{a.}  TM:  {\textbar}reitou{\textbar}nansəəka  ucjukuboo,  ɨcɨɨgadɨ  jatɨn,

      \textit{reitou=nan=səəka}  \textit{uk-tuk-boo}  \textit{ɨcɨɨ=gadɨ}  \textit{jar-tɨ=n}

      freezer=LOC1=just  put-PFV-CND  when=LMT  COP-SEQ=even

      ucjukarɨi.

      \textit{uk-tuk-arɨr-i}

      put-PRPR-CAP-NPST

      ‘If (you) put (the pickles) in the freezer, you can keep (them) no matter how long (the period of preservation) was.’

      [Co: 101023\_01.txt]

  Sentence-final position

  \textbf{b.}  TM:  tukunusimacˀju  jatɨkai?

      \textit{tukunusima+cˀju}  \textit{jar-tɨ=kai}

      Tokunoshima+person  COP-SEQ=DUB

      ‘Is (that person) from Tokunoshima island?’

      [Co: 120415\_01.txt]

  \textbf{c.}  TM:  an  cˀjoo  taru  jatɨga?

      \textit{a-n}  \textit{cˀju=ja}  \textit{ta-ru}  \textit{jar-tɨ=ga}

      DIST-ADNZ  person=TOP  who-NLZ  COP-SEQ=FOC

      ‘Who was that person?’

      [El: 110327]

\subsubsection{Environments where both of \textit{zjar-} (COP) and \textit{jar-} (COP) are used}

Both of \textit{zjar-} (COP) and \textit{jar-} (COP) may take \textit{{}-sa} (POL) and \textit{{}-sɨga} (POL). So far, I have not found any difference between them. I present examples of \textit{{}-sa} (POL).

\textbf{(8\nobreakdash-45)}  \textit{{}-sa} (POL)

  \textbf{a.}  TM:  an  cˀjoo  akira  jassa.

      \textit{a-n}  \textit{cˀju=ja}  \textit{akira}  \textit{jar-sa}

      DIST-ADNZ  person=TOP  Akira  COP-POL

      ‘That person is Akira.’

      [El: 120921]

  \textbf{b.}  TM:  an  cˀjoo  akira  zjassa.

      \textit{a-n}  \textit{cˀju=ja}  \textit{akira}  \textit{zjar-sa}

      DIST-ADNZ  person=TOP  Akira  COP-POL

      ‘That person is Akira.’

      [El: 120921]

Both of \textit{jar-} (COP) and \textit{zjar-} (COP) can take the participial affix \textit{{}-n} (PTCP), but the environments where they appear are different from each other. Before \textit{mun} (ADVRS), \textit{jar-n} (COP-PTCP) is chosen, and before \textit{kara} (CSL), \textit{zja-n} (COP-PTCP) is chosen as in the following examples.

\textbf{(8\nobreakdash-46)}  \textbf{a.}  TM:  takenna  cjoo  tabukuruccjɨ  an  bun  janmun.

      \textit{taken=ja}  \textit{cjoo}  \textit{tabukuru=ccjɨ}  \textit{a-n}  \textit{bun}  \textit{jar-n=mun}

      Taken=TOP  just  rice.field=QT  DIST-ADNZ  share  COP-PTCP=ADVRS

      ‘(Speaking of) rice fields, Taken has [lit. is] just such a share.’

      [Co: 111113\_02.txt]

  \textbf{b.}  TM:  ujankjaga  izjasi  zjankara,  nusinkjoo

      \textit{uja=nkja=ga}  \textit{izjas-i}  \textit{zjar-n=kara}  \textit{nusi=nkja=ja}

      parent=APPR=NOM  put.out-INF  COP-PTCP=CSL  RFL=APPR=TOP

      sijanbajaa.

      \textit{sij-an-ba=jaa}

      know-NEG-CSL=SOL

      ‘Parents pay (the tuition fee), so (pupils) themselves do not know (the amount).’

      [Co: 120415\_00.txt]

The speaker TM said that the expression of the latter, i.e. /zjankara/ \textit{zjar-n=kara} (COP-PTCP=CSL) can be replaced by /natɨ/ \textit{nar-tɨ} (COP-SEQ) in \sectref{sec:key:8.3.3.4.} The copular participles are restricted in the cases where conjunctive particles follow them as in (8-46 a-b). There is no case where nominal predicates fill the modifier slot of an NP in the non-past tense and the affirmative polarity (see \sectref{sec:key:9.4.1} for more details).

\subsection{Stative verbs}

Syntactically, the stative verb in Yuwan fills the predecate phrase together with an adjective, and makes an adjectival predicate phrase (see \sectref{sec:key:9.2} for more details). Yuwan has two stative verbs, i.e. \textit{ar-} and \textit{nə-}. The former, i.e. \textit{ar-} (STV), appears in affirmative with the exception of the cases of AVC. The latter, i.e. \textit{nə-} (STV), appears only in negative.

\subsubsection{\textit{ar-} (STV)}

If the polarity of the predicate is affirmative, \textit{ar-} (STV) may appear after the adjective inflected with \textit{{}-sa} (ADJ).

(8\nobreakdash-47)  Affirmative polarity

  a.  US:  cjaa.  uninna  zjanasa  atattujaa.

      \textit{cjaa}  \textit{unin=ja}  \textit{zjana-sa}  \textit{ar{}-tar-tu=jaa}

      that.is.right  that.time=TOP  many-ADJ  STV-PST-CSL=SOL

      ‘That’s right. At that time there were many (students) [lit. (the students) were many].’

      [Co: 110328\_00.txt]

  b.  TM:  urəə  jiccja  aroogai?

      \textit{u-rɨ=ja}  \textit{jiccj-sa}  \textit{ar{}-oo=ga=i}

      MES-NLZ=TOP  good-ADJ  STV-SUPP=CFM3=PLQ

      ‘That is good (, isn’t it)?’

      [El: 130820]

In (8-47 a), the stative verb \textit{ar-} makes an adjectival predicate together with the preceding adjective \textit{zjana-sa} (many-ADJ). In (8-47 b), the stative verb \textit{ar-} makes an adjectival predicate together with the preceding adjective /jiccja/ \textit{jiccj-sa} (good-ADJ).

  The stative verb \textit{ar-} undergoes contraction with the preceding adjectival inflectional affix \textit{{}-sa} when the stative verb takes \textit{{}-i} (NPST) or \textit{{}-n} (PTCP). For example, \textit{jiccj-sa} (good-ADJ) + \textit{ar-i} (STV-NPST) > /jiccjai/ (not */jiccjaai/) ‘good’ (see \sectref{sec:key:9.2.2.2} for more details).

As menitoned above, \textit{ar-} (STV) basically appears in affirmative. However, there is a case where \textit{ar-} (STV) can appear in negative. If the stative verb fills the lexical verb slot in the auxiliary verb construction (see \sectref{sec:key:9.1.1}), the stative verb is always \textit{ar-} (STV) (not \textit{nə-}).

(8\nobreakdash-48)  \textit{ar-} (STV) in AVC

  TM:  an  cˀjoo  dujasoo  atɨ  mooran.jaa.

    \textit{a-n}  \textit{cˀju=ja}  \textit{duja-soo}  \textit{ar{}-tɨ  moor-an=jaa}

    DIST-ADNZ  person=TOP  rich-ADJ  [STV-SEQ  HON-NEG]=SOL

          [Lex. verb  Aux. verb]\textsubscript{VP}

    ‘That person is not rich, you know.’

    [El: 130820]

In the auxiliary verb constructin where the auxiliary verb is the honorific verb \textit{moor-} (HON), the stative verb is always \textit{ar-}, even though the predicate is in negative as in (8-48).

\subsubsection{\textit{nə-} (STV)}

If the stative verb is followed by one of the negative affixes, i.e. \textit{{}-an} (NEG) or \textit{{}-azɨi} (NEG.PLQ), the stative verb is always \textit{nə-}. They go through reduction or assimilation like /nə-n/ \textit{nə-an} (STV-NEG) or /nə-əzɨi/ \textit{nə-azɨi} (STV-NEG.PLQ). The adjective that precedes \textit{nə-} (STV) always inflects with \textit{{}-soo} (ADJ).

(8\nobreakdash-49)  Negative polarity

  a.  \textit{{}-an} (NEG)

    [Context: Talking about the wooden beams of MS’s house; MS: ‘(The wooden beams of my

house) haven’t become so black as those (of your house), you know.’ ] = (4-11 b)

    TM:  kˀurusoo  nəndarooga.

      \textit{kˀuru-soo}  \textit{nə-an=daroo=ga}

      black-ADJ  STV-NEG=SUPP=CFM3

      ‘(Those) are not black, right?’

      [Co: 111113\_01.txt]

  b.  \textit{nə-} (STV) + \textit{{}-azɨi} (NEG.PLQ)

    TM:  an  kasoo  kˀurusoo  nəəzɨi?

      \textit{a-n}  \textit{kasa=ja}  \textit{kˀuru-soo}  \textit{nə-azɨi}

      DIST-ADNZ  hat=TOP  black-ADJ  STV-NEG.PLQ

      ‘Isn’t that hat black?’

      [El: 111118]

\subsection{Comparison among the existential verbs, copula verbs, and stative verbs (“ECS verbs”)}

In the above sections, we have discussed the differences among the thee verbal stems, i.e. the existential verb, the copula verb, and the stative verb (henceforth, “ECS verbs”). The existential verb is sensitive to the animacy of the core argument, but the others are not. Moreover, the copula verb is likely to use \textit{ar-} in negative. In contrast, the stative verb is likely to use \textit{ar-} in affirmative (see also \tabref{tab:key:71}).

Moreover, they fill different kinds of predicate phrases. The existential verb fills the verbal predicate phrase, the copula verb fills the nominal predicate phrase, and the stative verb fills the adjectival predicate phrase (see Chapter 9 for more details). Thus, these ECS verbs are different from one another. There are, however, a few similarities among them: (A) they can directly precede Group-II affixes; (B) they choose the form /ar-/ in AVC.

  First, in (8-3 b) in \sectref{sec:key:8.1}, we have discussed a certain group of inflectional affixes, i.e. Group-II affixes, which cannot directly follow any verbal root. However, ECS verbs can directly precede Group-II affixes. For example, \textit{{}-i} (NPST) and \textit{{}-oo} (SUPP) are members of Group-II affixes, but they can follow the existential verbs directly.

\textbf{(8\nobreakdash-50)}  Existential verbs + Group-II affixes

  \textbf{a.}  \textit{wur-} ‘exist (animate)’ + \textit{-i} (NPST)

    [Context: Talking about an acquaintance; US: ‘Has she passed away?’]

    TM:  aran.  namoo  umanan  wui.

      \textit{ar-an}  \textit{nama=ja}  \textit{u-ma=nan}  \textit{wur-i}

      COP-NEG  now=TOP  MES-place=LOC1  exist-NPST

      ‘No. (She) is there now.’

      [Co: 110328\_00.txt]

  \textbf{b.}  \textit{ar-} ‘exist (inanimate)’ + \textit{-oo} (SUPP)

    TM:  an,  namanu  {\textbar}jakkjoku{\textbar}nu  aroogai?

      \textit{a-n}  \textit{nama=nu}  \textit{jakkjoku=nu}  \textit{ar-oo=ga=i}

      DIST-ADNZ  now=GEN  pharmacy=NOM  exist-SUPP=CFM3=PLQ

      ‘That (pharmacy), (i.e.) the pharmacy (that exists there) now probably (still) exists, right?’

      [Co: 111113\_01.txt]

In (8-50 a), \textit{wur-} ‘exist’ directly precedes \textit{{}-i} (NPST). In (8-50 b), \textit{ar-} ‘exist’ directly precedes \textit{{}-oo} (SUPP). It should be noted that \textit{{}-oo} (SUPP) has the same form with \textit{{}-oo} (INT). They can usually be distinguished by their morphological environments, since the former belongs to Group-II affixes, and the latter belongs to Group-I affixes, and Group-I affixes can follow verbal roots directly. However, the existential verb \textit{wur-} ‘exist’ can take Group-II affixes directly. Thus, we cannot distinguish them by their morphological environments. The following examples show this case.

\textbf{(8\nobreakdash-51)}  \textbf{a.}  \textit{wur-} ‘exist’ + \textit{{}-oo} (SUPP)

    [Context: Talking about TM’s daughter in law]

    TM:  jaanan  wuroojo.

      \textit{jaa=nan}  \textit{wur-oo=joo}

      house=LOC1  exist-SUPP=CFM1

      ‘(She) may be in the house.’

      [Co: 120415\_01.txt]

  \textbf{b.}  \textit{wur-} ‘exist’ + \textit{{}-oo} (INT)

    TM:  wanna  kumanan  {\textbar}ittoki{\textbar}  wuroojəə.

      \textit{wan=ja}  \textit{ku-ma=nan}  \textit{ittoki}  \textit{wur-oo=jəə}

      1SG=TOP  PROX-place=LOC1  for.a.while  exist-INT=CFM2

      ‘I will be here for a while.’

      [El: 120919]

In (8-51 a-b), we can distinguish \textit{{}-oo} (SUPP) from \textit{{}-oo} (INT) only by the contexts. In contrast with \textit{wur-} ‘exist,’ another existential verb \textit{ar-} ‘exist’ cannot take animate subjets. Thus, it is difficult for \textit{ar{}-} ‘exist’ to take \textit{{}-oo} (INT), since \textit{{}-oo} (INT) expresses the subject’s intention (see \sectref{sec:key:8.4.1.2}). The example where the copula verb takes the Group II affix \textit{{}-oo} (SUPP) was shown in (8-36 b) in \sectref{sec:key:8.3.3.1.} An example where the stative verb takes \textit{{}-oo} (SUPP) was shown in (8-47 b) in \sectref{sec:key:8.3.4.1.}

  Secondly, ECS verbs choose the form /ar-/ among their variant morphemes when they fill the lexical verb slot in the auxiliary verb construction (“AVC”), although there is the exception \textit{wur-} ‘exist.’ This behavior can be summarized as in \tabref{tab:key:76}.

\begin{styleBeschriftung}
\textmd{\tabref{tab:key:76}}\textmd{. ECS verbs in the lexical verb slot in AVC}
\end{styleBeschriftung}

Core NPs  Animate  Inanimate

Existential verbs  \textit{wur-}  \textit{ar-}

Copula verbs  \textit{ar-}

Stative verbs  \textit{ar-}

Compare \tabref{tab:key:76} with \tabref{tab:key:71}. We can notice that the form /ar-/ dominates over the other forms. The example of the existentail verb in AVC was shown in (8-33) in \sectref{sec:key:8.3.2.2.} The example of the copula verb in AVC was shown in (8-43) in \sectref{sec:key:8.3.3.4.} The example of the stative verb in AVC was shown in (8-48) in \sectref{sec:key:8.3.4.1.}

\section{Inflectional morphology}

We have discussed the criteria of verbal inflectional affixes in (8-9) in \sectref{sec:key:8.1.} Verbal inflectional affixes can be classified in three ways. By the morphophonological criteria, the verbal affixes can be separated into four groups (Type-A to Type-D affixes) as in \tabref{tab:key:56} in \sectref{sec:key:8.2.1.} By the morphological criteria, the verbal inflectional affixes can be separated into two groups (Group-I and Group-II affixes) as in (8-3) in \sectref{sec:key:8.1.} In this section, the verbal inflectional affixes will be separated into four groups: the finite-form affix, the participial affix, the converbal affix, and the infinitival affix. The verb forms that take these affixes will be called finite forms, participles, converbs, and infinitives respectively. These groups will be called “inflectional categories” in this grammar.

  The inflectional categories are determined by two types of criteria. The main criterion is syntactic, and the secondary criterion is morphosyntactic. First, we can divide the inflectional categories according to their “external syntax” \citep{Haspelmath1996}, i.e. their behavior in a phrase or their behavior toward the main clause. If a verb form can behave like an adnominal in an NP, it is called participle. If a verb form can behave like an adverb (without any particle) toward the predicate of the main clause, it is called a converb \citep{Haspelmath1995}. If a verb form can behave like a nominal toward the predicate of the main clause, it is called an infinitive. The remaining verbal forms are called “finite forms” in this grammar. These verbal forms can fill the predicate slot of a clause (see also \sectref{sec:key:4.1.1} about the clause structure in Yuwan). In other words, they behave as the verb in their “internal syntax” \citep{Haspelmath1996} in respect of retaining, if partly, the original argument structures. That is the reason why they are categorized as verbs.

\begin{styleBeschriftung}
\textmd{\tabref{tab:key:77}}\textmd{. Inflectional categories (with the main criteria)}
\end{styleBeschriftung}

Inflectional categories  External syntax

Finite form  N/A

Participle  Adnominal

Converb  Adverb

Infinitive  Nominal

The degree of retention of the internal syntax, or “clausehood,” is not the same among the above inflectional categories. All of the finite forms and participles can have their own subjects. Many of the converbs can have their own subjects, but \textit{{}-tai} (LST) and \textit{{}-jagacinaa} (SIM) cannot, and their subjects always coincide with those of the main clauses. Similarly, the infinitives cannot take their own subjects when they fill the predicate slot of the main clause, or fill the complement slot of the light verb construction (see \sectref{sec:key:8.4.4.2}). Regarding arguments other than subjects, all of the verbs in the above inflectional categories can take their own ones.

  Secondly, the subsidiary criteria for the inflectional categories are morphosyntactic ones, which are composed of the morphological defectiveness and syntactic autonomy of the verbal form. These criteria have something to do with the term “finiteness” (cf. \citealt{Nikolaeva2007}: 1). However, there is not a clear-cut boundary between “finite” and “non-finite” in Yuwan. For example, converbs, which would be “non-finite forms,” can terminate a sentence (i.e. “insubordination” in \sectref{sec:key:11.2}). Furthermore, the participle usually modifies the head nominal in an NP, but it can also terminate a sentence in a focus construction (see “Kakari-musubi” in \sectref{sec:key:11.3}), and can head an adverbial clause with some conjunctive particles (see \sectref{sec:key:10.2}). Therefore, we do not propose “finite” vs. “non-finite” distinction in this grammar, and we will use the following criteria only for the distinction of the four inflectional categories. The selective criteria are as follows: (A) the word form can include the past affix \textit{{}-tar}; (B) the word form can include the negative affix \textit{{}-an}; (C) the verbal form can only fill the predicate of a main clause.

\begin{styleBeschriftung}
\textmd{\tabref{tab:key:78}}\textmd{. Inflectional categories (with the subsidiary criteria)}
\end{styleBeschriftung}

Inflectional categories  Can include \textit{{}-tar} (PST)  Can include \textit{{}-an} (NEG)  Can only fill the predicate of a main clause

Finite form  + / -  + / -  +

Participle  + / -  + / -  {}-

Converb  {}- / (+)  + / (-)  {}-

Infinitive  {}-  {}-  {}-

Note:

  “+” means that all of the affixes satisfy the criterion;

“+ / (-)” means that almost all of the affixes satisfy the criterion, but that a few affixes do not;

“+ / -” means that some affixes safisfy the criterion, but that the other affixes do not;

“- / (+)” means that almost all of the affixes do not satisfy the criterion, but that a few affixes do;

“-” means that no affixes satisfy the criterion.

Considering the difficulty to determine the “finiteness” by the subsidiary criteria in \tabref{tab:key:78}, we will give the priority to the criteria of the external syntax shown in \tabref{tab:key:77}.

\begin{styleBeschriftung}
\textmd{\tabref{tab:key:79}. Inflectional categories and affixes}
\end{styleBeschriftung}

Inflectional categories  All examples

Finite-form affixes  \textit{{}-oo} (INT), \textit{{}-oo} (SUPP), \textit{{}-ɨ} (IMP), \textit{{}-na} (PROH), \textit{{}-ɨba} (SUGS), \textit{{}-azɨi} (NEG.PLQ),

\textit{{}-i} (NPST), \textit{{}-mɨ} (PLQ), \textit{{}-u} (PFC), \textit{{}-sa} (POL), \textit{{}-sɨga} (POL), \textit{{}-tar} (PST)

Participial affixes  \textit{{}-n} (PTCP), \textit{{}-an} (NEG)

Converbal affixes  \textit{{}-ba} (CSL), \textit{{}-tu} (CSL), \textit{{}-too} (CSL), \textit{{}-boo} (CND), \textit{{}-tai} (LST), \textit{{}-gadɨ} ‘until’, \textit{{}-jagacinaa} (SIM), \textit{{}-təəra} ‘after’, \textit{{}-tɨ} (SEQ), \textit{{}-nən} (SEQ)

Infinitival affixes  \textit{{}-i}/\textit{{}-Ø} (INF)

As mentioned in \sectref{sec:key:8.1}, \textit{{}-an} (NEG) and \textit{{}-tar} (PST) do not necessarily close a word; in other words, they can be in either word-final position or non-word-final position. If they fill the non-word-final position, they are not concerned with the discussion here. However, if they fill the word-final position, the verb forms need to be classified into one of the above inflectional categories.

First, the verb form ending with \textit{{}-an} (NEG) cannot include \textit{{}-tar} (PST) within itself (but the verb form ending with \textit{{}-tar} can include \textit{{}-an}, see \sectref{sec:key:8.1}) and can fill not only the predicate of a main clause but also that of an adnominal clause. Thus, \textit{{}-an} (NEG) cannot be classified into the finite forms by the subsidiary criteria in \tabref{tab:key:78}. I will propose that the verb form ending with \textit{-an} (NEG) is a participle, and that the \textit{{}-an} (NEG) itself is a participial affix in the word-final environment.

Secondly, the verb form ending with \textit{{}-tar} (PST) can include itself. It can also include \textit{{}-an} (NEG), and can only fill the predicate of a main clause. Thus, we can regard the verb form ending with \textit{{}-tar} (PST) as a finite form, and also can regard \textit{{}-tar} (PST) as a finite-form affix in the word-final environment.

In the following sections, I will present examples of each inflectional category: the finite form (see \sectref{sec:key:8.4.1}), the participle (see \sectref{sec:key:8.4.2}), the converb (see \sectref{sec:key:8.4.3}), and the infinitive (see \sectref{sec:key:8.4.4}). Additionally, the possible combination of the inflectional affixes and the derivational (and non-word-final inflectional) affixes will be shown together in those sections. The lists composed of 17 types of verbal stems (see \sectref{sec:key:8.2.1}) and the inflectional affixes (excluding the Group-II affixes) are shown in appendix.

\subsection{Finite form}

The finite form is a verbal form that ends with the finite-form affixes in (8-52). The finite forms can fill only the predicate slot of a main clause. The finite-form affixes can be separated further by their functions.

(8\nobreakdash-52)  Finite-form affixes

  a.  Tense

    \textit{{}-i} (NPST) and \textit{{}-tar} (PST)

  b.  Mood

    \textit{{}-oo} (INT) and \textit{-oo} (SUPP)

  c.  Politeness

    \textit{{}-sa} (POL) and \textit{{}-sɨga} (POL)

  d.  Speech act (Question)

    \textit{{}-mɨ} (PLQ) and \textit{{}-azɨi} (NEG.PLQ)

  e.  Speech act (Command)

    \textit{{}-ɨ} (IMP), \textit{{}-na} (PROH), and \textit{{}-ɨba} (SUGS)

  f.  Information structure

    \textit{{}-u} (PFC)

As mentioned in \sectref{sec:key:8.1}, the finite-form affixes can be separated into two groups, i.e. Group-I affixes or Group-II affixes. Therefore, the finite-form affixes that belong to Group-II affixes, i.e. \textit{{}-i} (NPST), \textit{{}-oo} (SUPP), \textit{{}-mɨ} (PLQ), \textit{{}-sa} (POL), \textit{{}-sɨga} (POL), and \textit{{}-u} (PFC), cannot directly follow the verbal roots (with the exception of ECS verbs discussed in \sectref{sec:key:8.3.5}). A complete lists of the possible combinations of 17 types of verbal stems (see \sectref{sec:key:8.2.1}) and the finite-form affixes will be shown in appendix.

In the following subsections, I will present the contrasts shown in (8-52) in turn.

\subsubsection{Tense: \textit{{}-i} (NPST) and \textit{{}-tar} (PST)}

The finite-form affixes \textit{{}-i} (NPST) and \textit{{}-tar} (PST) can express the tense opposition: non-past vs. past. First, I will present the verbal morphemes that can directly precede \textit{{}-i} (NPST). The affixes deleted by double lines cannot directly precede \textit{{}-i} (NPST).

(8\nobreakdash-53)  Verbal morphemes that can directly precede \textit{{}-i} (NPST) (Finite-form affix; Group II)

  Root  \textit{{}-as  {}-arɨr} %%[Warning: Draw object ignored]
\textit{{}-tuk  {}-arɨr  {}-tur  {}-jawur} %%[Warning: Draw object ignored]
\textit{{}-an  {}-təər  {}-tar  {}-i} (NPST)

    CAUS  PASS  PRPR  CAP  PROG  POL  NEG  RSL  PST  

          \textit{{}-jur} 

          UMRK    

The finite-form affix \textit{{}-i} (NPST) belongs to Group-II affixes (see \sectref{sec:key:8.1}). Thus, it cannot directly follow any verbal root and always takes one of the affixes in (8-53) to close the word. I will present an example in (8-54).

(8\nobreakdash-54)  \textit{{}-i} (NPST)

  [Context: TM and US were talking about the present author.]

  TM:  {\textbar}hoogen{\textbar}nu  attakəə  wakajui.

    \textit{hoogen=nu}  \textit{attakəə}  \textit{wakar-jur-i}

    dialect=NOM  everything  understand-UMRK-NPST

    ‘(He) understands everything (about our) dialect.’

    [Co: 110328\_00.txt]

  On the contrary, \textit{{}-tar} (PST) can directly follow any verbal root as in (8-55). I will present the verbal morphemes that can directly precede \textit{{}-tar} (PST) in (8-55).

(8\nobreakdash-55)  Verbal morphemes that can directly precede \textit{{}-tar} (PST) (Finite-form affix; Group I)

  Root  \textit{{}-as  {}-arɨr} %%[Warning: Draw object ignored]
\textit{{}-tuk  {}-arɨr  {}-tur  {}-jawur} %%[Warning: Draw object ignored]
\textit{{}-an  {}-təər  {}-tar}

    CAUS  PASS  PRPR  CAP  PROG  POL  NEG  RSL  PST

          \textit{{}-jur} 

          UMRK  

I will present an example of \textit{{}-tar} (PST) in (8-56).

(8\nobreakdash-56)  \textit{{}-tar} (PST)

  TM:  nobuarija  mjicjɨ  cˀjancjɨ  jˀicja.

    \textit{nobuari=ja}  \textit{mj-tɨ}  \textit{k-tar-n=ccjɨ}  \textit{jˀ-tar}

    Nobuari=TOP  see-SEQ  come-PST-PTCP=QT  say-PST

    ‘Nobuari said that (he) visited (the person).’

    [Co: 120415\_01.txt]

The above example shows that \textit{{}-tar} (PST) directly follows the verbal root \textit{jˀ-} ‘say.’

In principle, the affix-final //r// or \textit{{}-tar} (PST) assimilates to the initial consonant of the Type-D affixes (or clitics) (see \sectref{sec:key:8.2.1.4}). However, \textit{{}-tar} (PST) becomes /ta/ (not /tak/) only before \textit{kai} (DUB) or \textit{kamo} (POS).

(8\nobreakdash-57)  a. \textit{{}-tar} (PST) before \textit{kai} (DUB)

  TM:  cukutəə  wutakai?

    \textit{cukur-tɨ=ja}  \textit{wur-tar=kai}

    make-SEQ=TOP  PROG-PST=DUB

    ‘Was (anyone) making (cocoons)?’

    [Co: 111113\_01.txt]

  b. \textit{{}-tar} (PST) before \textit{kamo} (POS)

  TM:  takencˀjunkjoo  kˀuwasisan  cˀjoo

    \textit{taken+cˀju=nkja=ja}  \textit{kˀuwasi-sa+ar-n}  \textit{cˀju=ja}

    Taken+person=APPR=TOP  know.very.well-ADJ+STV-PTCP  person=TOP

    wurantakamodoojaa.

    \textit{wur-an-tar=kamo=doo=jaa}

    exist-NEG-PST=POS=ASS=SOL

    ‘(It is) possible (that) there is no person who knows (about that) very well among the people in Taken.’

    [Co: 111113\_01.txt]

  It should be mentioned that \textit{{}-tar} (PST) in the finite-form use cannot appear in the interrogative clause. In that case, \textit{{}-tɨ} (SEQ) is used to express the past tense (see \sectref{sec:key:11.2.1} for more details). It should be noted that a clause that includes \textit{{}-tar} (PST) and \textit{kai} (DUB) is permitted as in (8-57 a), since \textit{kai} (DUB) expresses wondering to oneself, which is a peripheral type of the question (i.e. question to oneself) (see also \sectref{sec:key:10.3.6}). In other words, \textit{{}-tar} (PST) expresses the speaker’s confidence in the factuality of the event, no matter how weak it is.

\subsubsection{Mood: \textit{{}-oo} (INT) and \textit{{}-oo} (SUPP)}

The finite-form affixes \textit{-oo} (INT) and \textit{{}-oo} (SUPP) express the mood. First, I will present the verbal morphemes that can directly predede \textit{{}-oo} (INT). The affixes deleted by double lines cannot directly precede the word-final affix.

(8\nobreakdash-58)  Verbal morphemes that can directly precede \textit{-oo} (INT) (Finite-form affix; Group I)

  Root  \textit{{}-as  {}-arɨr} %%[Warning: Draw object ignored]
\textit{{}-tuk  {}-arɨr  {}-tur  {}-jawur} %%[Warning: Draw object ignored]
\textit{{}-an  {}-təər  {}-tar  {}-oo} (INT)

    CAUS  PASS  PRPR  CAP  PROG  POL  NEG  RSL  PST  

          \textit{{}-jur} 

          UMRK    

As mentioned before, \textit{{}-oo} (INT) belongs to Group-I affixes, and it can directly follow the verbal roots as in (8-59 a). It may also follow another verbal affix as in (8-59 b-c).

(8\nobreakdash-59)  \textit{{}-oo} (INT)

  a.  US:  wanna  ikjoojəə.

      \textit{wan=ja}  \textit{ik-oo=jəə}

      1SG=TOP  go-INT=CFM2

      ‘I will go.’

      [Co: 110328\_00.txt]

  b.  TM:  {\textbar}onigiri{\textbar}  sjɨ,  mutasoojəə.

      \textit{onigiri}  \textit{sɨr-tɨ}  \textit{mut-as-oo=jəə}

      rice.ball  do-SEQ  have-CAUS-INT=CFM2

      ‘(I) will make a rice ball, and get (the present author) to have (it).’

      [Co: 101023\_01.txt]

  c.  TM:  kimucjagɨsanu,  wanga  kawajəə  utaroo.

      \textit{kimucjagɨ-sa=nu}  \textit{wan=ga}  \textit{kawajəə}  \textit{ut-ar-oo}

      feel.pity-ADJ=CSL  1SG=NOM  substitute  hit-PASS-INT

      ‘Since (I) feel pity (for you), I will be hit in place (of you).’

      [El: 130820]

The example (8-59 c) contains the passive affix \textit{{}-ar}, and the verb as a whole expresses the intention of the subject (not the agent). In other words, \textit{{}-oo} (INT) expresses the subject’s (not the agent’s) intention. The subject of the finite-form verb composed of \textit{{}-oo} (INT) is always the speaker.

  Secondly, \textit{{}-oo} (SUPP) belongs to Group-II affixes. Thus, it cannot follow any verbal root directly.

(8\nobreakdash-60)  Verbal morphemes that can directly precede \textit{-oo} (SUPP) (Finite-form affix; Group II)

  Root  \textit{{}-as  {}-arɨr} %%[Warning: Draw object ignored]
\textit{{}-tuk  {}-arɨr  {}-tur  {}-jawur} %%[Warning: Draw object ignored]
\textit{{}-an  {}-təər  {}-tar  {}-oo} (SUPP)

    CAUS  PASS  PRPR  CAP  PROG  POL  NEG  RSL  PST  

          \textit{{}-jur} 

          UMRK    

I will present examples of \textit{{}-oo} (SUPP) in (8-61 a-b).

(8\nobreakdash-61)  \textit{{}-oo} (SUPP)

  a.  TM:  namanu,  usi  sjurooga?

      \textit{nama=nu}  \textit{usi}  \textit{sɨr-jur-oo=ga}

      now=GEN  cow  do-UMRK-SUPP=CFM3

      ‘Now (someone) raises cows, doesn’t he?’

      [Co: 111113\_01.txt]

  b.  TM:  nanga  jˀujaa  sjutarooga?

      \textit{nan=ga}  \textit{jˀu+jaa}  \textit{sɨr-jur-tar-oo=ga}

      2.HON.SG=NOM  fish+house  do-UMRK-PST-SUPP=CFM3

      ‘You used to run a fish shop, didn’t you?’

      [Co: 110328\_00.txt]

  Apparently, \textit{-oo} (INT) and \textit{{}-oo} (SUPP) have the same form. Therefore, there are a few cases, where it is difficult to draw a distinction between the two affixes by their morphological environments, e.g. after “ECS verbs” (see \sectref{sec:key:8.3.5}) or after the derivational affix \textit{{}-tur} (PROG) as in (8-62).

(8\nobreakdash-62)  After \textit{{}-tur} (PROG)

  a.  \textit{{}-oo} (INT)

    TM:  wanna  amanantɨ  juduroo.

      \textit{wan=ja}  \textit{a-ma=nantɨ}  \textit{jum-tur-oo}

      1SG=TOP  DIST-place=LOC2  read-PROG-INT

      ‘I will be reading (the book) there.’

      [El: 130820]

  b.  \textit{{}-oo} (SUPP)

    TM:  akiroo  amanantɨ  juduroo.

      \textit{akira=ja}  \textit{a-ma=nantɨ}  \textit{jum-tur-oo}

      Akira=TOP  DIST-place=LOC2  read-PROG-SUPP

      ‘Probably, Akira is reading (the book) there.’

      [El: 130820]

In these examples, we can distinguish \textit{{}-oo} (INT) from \textit{{}-oo} (SUPP) only by the contexts (e.g. the subjects of the clauses).

\subsubsection{Politeness: \textit{{}-sa} (POL) and \textit{{}-sɨga} (POL)}

The finite-form affixes \textit{{}-sa} (POL) and \textit{{}-sɨga} (POL) are used to express politeness to the hearer. They belong to Group-II affixes, so they cannot directly follow any verbal root. The verbal affixes that can directly precede \textit{{}-sa} (POL) and \textit{{}-sɨga} (POL) are almost the same, but only \textit{{}-an} (NEG) cannot precede \textit{{}-sa} (POL) as in (8-63 a). The affixes deleted by double lines cannot directly precede the word-final affix.

(8\nobreakdash-63)  a. Verbal morphemes that can directly precede \textit{{}-sa} (POL) (Finite-form affix; Group II)

  Root  \textit{{}-as  {}-arɨr} %%[Warning: Draw object ignored]
\textit{{}-tuk  {}-arɨr  {}-tur  {}-jawur} %%[Warning: Draw object ignored]
\textit{{}-an  {}-təər  {}-tar  {}-sa} (POL)

    CAUS  PASS  PRPR  CAP  PROG  POL  NEG  RSL  PST  

          \textit{{}-jur} 

          UMRK    

  b. Verbal morphemes that can directly precede \textit{{}-sɨga} (POL) (Finite-form affix; Group II)

  Root  \textit{{}-as  {}-arɨr} %%[Warning: Draw object ignored]
\textit{{}-tuk  {}-arɨr  {}-tur  {}-jawur} %%[Warning: Draw object ignored]
\textit{{}-an  {}-təər  {}-tar  {}-sɨga} (POL)

    CAUS  PASS  PRPR  CAP  PROG  POL  NEG  RSL  PST  

          \textit{{}-jur} 

          UMRK    

As mentioned in \sectref{sec:key:1.4.2}, the old people rarely use the derivational politeness affix \textit{{}-jawur}. On the contrary, they use the inflectional politeness affix \textit{{}-sa} or \textit{{}-sɨga} as in (8-64 a-c).

(8\nobreakdash-64)  \textit{{}-sa} (POL)

  a.  [Context: TM asks MS to make a topic of conversation; TM: ‘Please make a topic.’]

    TM:  həntooja  sjussa.

      \textit{həntoo=ja}  \textit{sɨr-jur-sa}

      reply=TOP  do-UMRK-POL

      ‘(I) will reply (to you).’

      [Co: 120415\_01.txt]

  \textit{{}-sɨga} (POL)

  b.  TM:  sjemenbukuruja,  (arɨ,)  sazikkiroccjɨ  jutassɨga.

      \textit{sjemen+hukuru=ja}  \textit{a-rɨ}  \textit{sazikkiro=ccjɨ}  \textit{jˀ-jur-tar-sɨga}

      cement+bag=TOP  DIST-NLZ  thirty.kilogram=QT  say-UMRK-PST-POL

      ‘(People) used to say that a cement bag (weighs) thirty kilograms.’

      [Co: 111113\_02.txt]

  c.  TM:  uraa  naa  anmai  jansɨga.

      \textit{ura-a}  \textit{naa}  \textit{anmai}  \textit{jˀ-an-sɨga}

      2.NHON.SG-ADNZ  name  very.much  say-NEG-POL

      ‘(The person) does not say your name (as) many times (as before).’

      [Co: 120415\_01.txt]

\textit{{}-sa} (POL) and \textit{{}-sɨga} (POL) are functionally very similar to each other. However, there seems to be a difference that only \textit{{}-sɨga} (POL) follows \textit{{}-tar} (PST) such as (8-6 b). There are 27 examples of \textit{{}-sɨga} (POL) and eight examples of \textit{{}-sa} (POL) in my texts, and there are eight examples where \textit{{}-sɨga} (POL) follows \textit{{}-tar} (PST) but no example where \textit{{}-sa} (POL) follows \textit{-tar} (PST) (although \textit{-sa} (POL) can follow \textit{{}-tar} (PST) in elicitation).

\subsubsection{Speech act (Question): \textit{{}-mɨ} (PLQ) and \textit{{}-azɨi} (NEG.PLQ)}

The finite-form affixes \textit{{}-mɨ} (PLQ) and \textit{{}-azɨi} (NEG.PLQ) express the polar question (i.e. “yes-no question”).   First, \textit{{}-mɨ} (PLQ) belongs to the Group-II affixes, so it cannot directly follow any verbal root. The affixes deleted by double lines cannot directly precede the word-final affix.

(8\nobreakdash-65)  Verbal morphemes that can directly precede \textit{-mɨ} (PLQ) (Finite-form affix; Group II)

  Root  \textit{{}-as  {}-arɨr} %%[Warning: Draw object ignored]
\textit{{}-tuk  {}-arɨr  {}-tur  {}-jawur} %%[Warning: Draw object ignored]
\textit{{}-an  {}-təər  {}-tar  {}-mɨ} (PLQ)

    CAUS  PASS  PRPR  CAP  PROG  POL  NEG  RSL  PST  

          \textit{{}-jur} 

          UMRK    

(8\nobreakdash-66)  \textit{{}-mɨ} (PLQ)

  a. Affirmative polarity

  TM:  waakjaa  jantɨ ..  kamjumɨ?

    \textit{waakja-a}  \textit{jaa=nantɨ}  \textit{kam{}-jur-mɨ}

    1PL-ADNZ  house=LOC1  eat-UMRK-PLQ

    ‘Do (you) eat in my house?’

    [Co: 120415\_01.txt]

  b. Negative polarity

  TM:  uroo  kakamɨ?

    \textit{ura=ja}  \textit{kak-an-mɨ}

    2.NHON.SG=TOP  write-NEG-PLQ

    ‘Don’t you write (it)?’

    [El: 121012]

\textit{{}-mɨ} (PLQ) can be used both in affirmative and negative. It should be noted that \textit{{}-an} (NEG) necessarily becomes /a/ when it precedes \textit{{}-mɨ} (PLQ) as in (8-66 b), i.e. \textit{{}-an-mɨ} (NEG-PLQ) > /a-mɨ/.

Secondly, the other quesition finite-form affix \textit{{}-azɨi} (NEG.PLQ) cannot be used in affirmative. In other words, \textit{{}-azɨi} (NEG.PLQ) always expresses the negative polarity, and it cannot be preceded by \textit{{}-an} (NEG).

(8\nobreakdash-67)  Verbal morphemes that can directly precede \textit{-azɨi} (NEG.PLQ) (Finite-form affix; Group I)

  Root  \textit{{}-as  {}-arɨr} %%[Warning: Draw object ignored]
\textit{{}-tuk  {}-arɨr  {}-tur  {}-jawur} %%[Warning: Draw object ignored]
\textit{{}-an  {}-təər  {}-tar  {}-azɨi} (NEG.PLQ)

    CAUS  PASS  PRPR  CAP  PROG  POL  NEG  RSL  PST  

          \textit{{}-jur} 

          UMRK    

I will present examples of \textit{{}-azɨi} (NEG.PLQ) in (8-68).

(8\nobreakdash-68)  \textit{{}-azɨi} (NEG.PLQ)

  a.  TM:  nəəzɨi?

      \textit{nə-azɨi}

      exist-NEG.PLQ

      ‘Aren’t (they [i.e. the lamps]) there?’

      [Co: 120415\_00.txt]

  b.  TM:  cɨcjurazɨi?

      \textit{cɨk-tur-azɨi}

      attach-PROG-NEG.PLQ

      ‘Isn’t (the outdoor lamp) set (there yet)?’

      [Co: 120415\_00.txt]

  c.  TM:  turazɨi?

      \textit{tur-azɨi}

      take-NEG.PLQ

      ‘Don’t (you) take (it)?’

      [El: 110917]

\textit{{}-azɨi} (NEG.PLQ) in (8-68 a-c) express the polar question in negative.

\subsubsection{Speech act (Command): \textit{{}-ɨ} (IMP), \textit{{}-na} (PROH), and \textit{{}-ɨba} (SUGS)}

The finite-form affixes \textit{{}-ɨ} (IMP) and \textit{{}-na} (PROH) express command in a narrow sense, and \textit{{}-ɨba} (SUGS) expresses suggestion. The same affixes can precede these finite-form affixes as in (8-69). The affixes deleted by double lines cannot directly precede the word-final affix.

(8\nobreakdash-69)  Verbal morphemes that can directly precede \textit{-ɨ} (IMP), \textit{{}-na} (PROH), or \textit{{}-ɨba} (SUGS)

(Finite-form affixes; Group I)

  Root  \textit{{}-as  {}-arɨr} %%[Warning: Draw object ignored]
\textit{{}-tuk  {}-arɨr  {}-tur  {}-jawur} %%[Warning: Draw object ignored]
\textit{{}-an  {}-təər  {}-tar} %%[Warning: Draw object ignored]
  %%[Warning: Draw object ignored]
\textit{{}-ɨ} (IMP)

    CAUS  PASS  PRPR  CAP  PROG  POL  NEG  RSL  PST    \textit{{}-na} (PROH)

          \textit{{}-jur      {}-ɨba} (SUGS)

          UMRK      

These three finite-form affixes cannot be preceded by the negative affix \textit{{}-an}, which means that the polarity of them cannot be changed by \textit{{}-an} (NEG). Thus, the finite-form affix that can express the affirmative command is only \textit{{}-ɨ} (IMP), and the finite-form affix that can express the negative command (i.e. prohibition) is only \textit{{}-na} (PROH).

  The examples of \textit{{}-ɨ} (IMP) are shown below.

(8\nobreakdash-70)  \textit{{}-ɨ} (IMP)

  a.  TM:  kucjəəci  ɨrɨrɨ!

      \textit{kuci=kaci}  \textit{ɨrɨr-ɨ}

      mouth=ALL  put.in-IMP

      ‘Put (it) in (your) mouth!’

      [El: 121010]

  b.  TM:  jəito  kamɨjoocjɨdu  jutattujaa.

      \textit{jəito}  \textit{kam-ɨ=joo=ccjɨ=du}  \textit{jˀ-jur-tar-tu=jaa}

      much  eat-IMP=CFM1=QT=FOC  say-UMRK-PST-CSL=SOL

      ‘(Old people) used to say that, “Eat very much!”’

      [Co: 120415\_01.txt]

It should be noted that the verbal roots \textit{k-} ‘come’ and \textit{mukk-} ‘bring’ take another morpheme, i.e. \textit{{}-oo} (IMP), to express command as in (8-71 a-b).

(8\nobreakdash-71)  \textit{{}-oo} (IMP)

  a.  TM:  arɨ ..  koo,  koocjɨ,

      \textit{a-rɨ}  \textit{k-oo}  \textit{k-oo=ccjɨ}

      DIST-NLZ  come-IMP  come-IMP=QT

      ‘That person (said) that, “Come, come!”’

      [Co: 120415\_01.txt]

  b.  TM:  mukkoojocjɨ  jˀicjanmun,

      \textit{mukk-oo=joo=ccjɨ}  \textit{jˀ-tar-n=mun}

      bring-IMP=CFM1=QT  say-PST-PTCP=ADVRS

      ‘(I) said that, “Bring (the tape)!” However, ...’

      [Co: 120415\_01.txt]

\textit{{}-oo} (IMP) in (8-71 a-b) has the same form with \textit{-oo} (INT) discussed in \sectref{sec:key:8.4.1.2.}

  The examples of \textit{{}-na} (PROH) are shown below.

(8\nobreakdash-72)  \textit{{}-na} (PROH)

  a.  TM:  umannja  jˀuunajoo.

      \textit{u-ma=nan=ja}  \textit{jˀ-na=joo}

      MES-place=LOC1=TOP  sit-PROH=CFM1

      ‘Don’t sit there!’

      [El: 120921]

  b.  TM:  urɨ  tɨɨ  kɨɨnnajoocjɨ.

      \textit{u-rɨ}  \textit{tɨɨ}  \textit{kɨɨr-na=joo=ccjɨ}

      MES-NLZ  hand  put.on-PROH=CFM1=QT

      ‘(My husband said), “Don’t touch it!”’

      [Co: 120415\_01.txt]

  The finite-form \textit{{}-ɨba} (SUGS) expresses suggestion, which is a kind of command in a broad sense, but the imperativeness of \textit{{}-ɨba} (SUGS) is much weaker than that of \textit{{}-ɨ} (IMP).

(8\nobreakdash-73)  \textit{{}-ɨba} (SUGS)

  TM:  kuci  muzikijɨba.

    \textit{kuci}  \textit{muzikij-ɨba}

    mouth  twist-SUGS

    ‘How about twisting (the child’s) mouth (since he is a naughty boy).’

    [El: 120521]

In fact, there are a few examples where the same form /-ɨba/ is used adverbially (or converbally) as in (8-74).

(8\nobreakdash-74)  Converbal use of /-ɨba/

  a.  TM:  ura  tanmɨba,  jiccja  ata.

      \textit{ura}  \textit{tanm-ɨba}  \textit{jiccj{}-sa  ar-tar}

      2.NHON.SG  ask-CND  good-ADJ  STV-PST

      ‘If only (I) had asked you (to help teaching the dialect to the present author).’

[lit. ‘If (I) asked you, (it) was good.’]

      [Co: 111113\_02.txt]

  b.  TM:  tubɨba,  jiccja  asɨgana.

      \textit{tub-ɨba}  \textit{jiccj{}-sa  ar-sɨga=na}

      jump.into-CND  good-ADJ  STV-POL=CFM3

      ‘How about jumping into (the sea)?’ [lit. ‘If you jump into (the sea), (it) is good.’]

      [El: 110914]

If /-ɨba/ is used converbally, it always expresses a conditional meaning and is followed by the adjective \textit{jiccj-} ‘good’ as in (8-74 a-b). It is probable that the meaning of suggestion as in (8-73) is derived (or grammaticalized) from the uses such as (8-74 b), which is an example of the insubordination (see \sectref{sec:key:11.2}). In modern Yuwan, the conditional meaning as in (8-74 a) is usually expressed by another affix, i.e. \textit{{}-boo} (CND) as in (8-90 c). The uses such as (8-74 a-b) are rare in Yuwan. Thus, I propose that the affix /-ɨba/ is mainly used as suppositional finite-form affix in modern Yuwan as in (8-73).

\subsubsection{Information sturcture: \textit{{}-u} (PFC)}

The finite-form affix \textit{{}-u} (PFC) is always preceded by an affix that ends with //r//. The affixes deleted by double lines cannot directly precede \textit{{}-u} (PFC).

(8\nobreakdash-75)  Verbal morphemes that can directly precede \textit{{}-u} (PFC) (Finite-form affix; Group II)

  Root  \textit{{}-as  {}-arɨr} %%[Warning: Draw object ignored]
\textit{{}-tuk  {}-arɨr  {}-tur  {}-jawur} %%[Warning: Draw object ignored]
\textit{{}-an  {}-təər  {}-tar  {}-u} (PFC)

    CAUS  PASS  PRPR  CAP  PROG  POL  NEG  RSL  PST  

          \textit{{}-jur} 

          UMRK    

The finite-form affix \textit{{}-u} (PFC) is often used in information questions (so called “wh-questions”) as in (8-76 a-c) or polar questoins (so called “yes-no questions”) as in (8-76 d). \textit{{}-u} (PFC) in the polar question is always followed by the clause-final particle \textit{i} (PLQ), and also there is always \textit{du} (FOC) in the same clause.

(8\nobreakdash-76)  \textit{{}-u} (PFC)

  Information question

  a.  [Context: TM asked MS where the present author went.] (=5-34 a)

    TM:  nɨsəə  mata  daaciga  izjaru?

      \textit{nɨsəə}  \textit{mata}  \textit{daa=kaci=ga}  \textit{ik-tar-u}

      young.man  again  where=ALL=FOC  go-PST-PFC

      ‘Where did the young man go again?’

      [Co: 120415\_01.txt]

  b.  TM:  (kun,)  kun  cˀjoo  (ido..)  taa..

      \textit{ku-n}  \textit{ku-n}  \textit{cˀju=ja}  \textit{ido}  \textit{ta-a}

      PROX-ADNZ  PROX-ADNZ  person=TOP  oh  who-ADNZ

      maga  jataru?      

      \textit{maga}  \textit{jar-tar-u}      

      grandchild  COP-PST-PFC      

      ‘Whose grandchild is this person?’

      [Co: 120415\_00.txt]

  c.  [Context: TM was surprised that US brought a lot of foods to TM’s house.] = (6-101 a)

    TM:  nunkjabaga  mata  muccjɨ  moocjaru?

      \textit{nuu=nkja=ba=ga}  \textit{mata}  \textit{mut-tɨ}  \textit{moor-tar-u}

      what=APPR=ACC=FOC  again  have-SEQ  HON-PST-PFC

      ‘What did (you) bring (here) again?’

      [Co: 110328\_00.txt]

  Polar question

  d.  TM:  kurəə  {\textbar}maiku{\textbar}du  muccjurui?

      \textit{ku-rɨ=ja}  \textit{maiku=du}  \textit{mut-tur-u=i}

      PROX-NLZ=TOP  microphone=FOC  hold-PROG-PFC=PLQ

      kun  cˀjoo.  

      \textit{ku-n}  \textit{cˀju=ja}  

      PROX-ADNZ  person=TOP  

      ‘About this (picture), is this person holding a microphone?’

      [Co: 111113\_02.txt]

In elicitation, it is easy to have the speaker utter the verbal form ending with \textit{{}-u} (PFC) in the question sentence, but it is difficult in the declarative sentence. However, I have found two examples in my texts so far, where the speaker uses the finite form ending with \textit{{}-u} (PFC) in the declarative sentence as in (8-77 a-b).

(8\nobreakdash-77)  Declarative

  a.  TM:  utuzjoobasanna  un  cˀjunu  samisjentudu

      \textit{utuzjo+obasan=ja}  \textit{u-n}  \textit{cˀju=nu}  \textit{samisjen=tu=du}

      Utujo+old.woman=TOP  MES-ADNZ  person=GEN  samisen=COM=FOC

      utoo  (sii..)  sɨrarɨɨru.  

      \textit{uta=ja}  \textit{sɨr-i}  \textit{sɨr-arɨr-u}  

      song=TOP  do-INF  do-CAP-PFC  

      ‘Utujo can sing a song [lit. do a song] just with that person’s samisen. (Otherwise, she cannot sing a song.)’

      [Co: 120415\_00.txt]

  b.  TM:  tacuu{\textbar}toka{\textbar}ga  juubadu,  jˀarɨɨru.

      \textit{tacuu=toka=ga}  \textit{jˀ-ba=du}  \textit{jˀ-arɨr-u}

      Tatsu=APPR=NOM  say-CSL=FOC  say-CAP-PFC

      ‘(People) can say (a piece of advice to her), since (it is) Tatsu (who) says (it). (Otherwise, no one cannot say a piece of advice to her.)’

      [Co: 101023\_01.txt]

In the above examples of the declarative sentence, \textit{{}-u} (PFC) is preceded by \textit{-arɨr} (CAP). Additionally, there is an example, where \textit{-u} (PFC) is not preceded by \textit{{}-arɨr} (CAP) in spite of being in the declarative sentence as in (8-78), although this example is from a proverb.

(8\nobreakdash-78)  Declarative (in a proverb)

  TM:  tuunu  ujubəə  məəkacidu  magajuru.

    \textit{tuu=nu}  \textit{ujubɨ=ja}  \textit{məə=kaci=du}  \textit{magar-jur-u}

    ten=GEN  finger=TOP  front=ALL=FOC  bend-UMRK-PFC

    usijoocjəə  magarandoo.

    \textit{usiju=kaci=ja}  \textit{magar-an=doo}

    back=ALL=TOP  bend-NEG=ASS

    ‘Ten fingers (on hands) bend just forward. (They) do not bend backward.’ [i.e. ‘The members of a family should be close to each other like fingers.’]

    [El: 110328]

There is a possibility that the uses of the finite-verb ending with \textit{{}-u} (PFC) in the declarative sentences in (8-77 a-b) and (8-78) have the same characteristic. That is, these sentences seem to express that the predicate can be valid only with the focused constituents, and that anything other than the focused constituents cannot make the predicate valid. For example, in (8-77 a), the focused constituent \textit{u-n} \textit{cˀju=nu} \textit{samisjen=tu=du} (MES-ADNZ person=GEN samisen=COM=FOC) ‘just with that person’s samisen’ make the predicate ‘can sing a song’ valid, and it implies that if the woman was not ‘with that person’s samisen,’ she cannot sing a song. Similar arguments may be applied in (8-77 b) and (8-78).

  In all of the above examples, there are foci in the sentences. The foci were on the interrogative words as in (8-76 a-c), or marked by \textit{ga} (FOC) as in (8-76 a, c) or \textit{du} (FOC) as in (8-76 d), (8-77 a-b), and (8-78). Thus, \textit{{}-u} (PFC) expresses that it forms a predicate of the focus construction (see \sectref{sec:key:11.3} for more details about the focus construction).

\subsection{Participle (verbal adnominal)}

The participle is a verbal form that ends with the participial affixes, i.e. \textit{{}-n} (PTCP) or \textit{{}-an} (NEG).

\subsubsection{\textit{{}-n} (PTCP)}

The participial affix \textit{{}-n} (PTCP) belongs to Group-II affixes (see \sectref{sec:key:8.1}), i.e., cannot directly follow the verbal roots, and takes one of the affixes in (8-79). The affixes deleted by double lines cannot directly precede \textit{{}-n} (PTCP).

(8\nobreakdash-79)  Verbal morphemes that can directly precede \textit{{}-n} (PTCP) (Participial affix; Group II)

  Root  \textit{{}-as  {}-arɨr} %%[Warning: Draw object ignored]
\textit{{}-tuk  {}-arɨr  {}-tur  {}-jawur} %%[Warning: Draw object ignored]
\textit{{}-an  {}-təər  {}-tar  {}-n} (PTCP)

    CAUS  PASS  PRPR  CAP  PROG  POL  NEG  RSL  PST  

          \textit{{}-jur} 

          UMRK    

The verbal form ending with \textit{{}-n} (PTCP) usually fills the predicate slot of an adnominal clause as in (8-80 a-b), but it can fill that of a main clause as in (8-80 c) or an adverbial clause as in (8-80 d).

(8\nobreakdash-80)  \textit{{}-n} (PTCP)

  Adnominal clause

  a.  TM:  sakkiija  (hinzjaa)  xxx  hinzjaaba  succjun

      \textit{sakkii=ja}  \textit{hinzjaa}    [\textit{hinzjaa=ba}  \textit{sukk-tur-n}]\textsubscript{Adnominal clause}

      a.short.while.ago  goat    goat=ACC  pull-PROG-PTCP

      cˀjunu  atooradu  cˀjanmun.

      \textit{cˀju=nu}  \textit{atu=kara=du}  \textit{k-tar-n=mun}

      person=NOM  after=ABL=FOC  come-PST-PTCP=ADVRS

      ‘A short while ago, the person who was pulling a goat came afterward, but (this time he came beforehand).’

      [PF: 090827\_02.txt]

  b.  TM:  naa  hanasjun  tanɨga  nənbajaa.

      \textit{naa}  [\textit{hanas-jur-n}]\textsubscript{Adnominal clause}  \textit{tanɨ=ga}  \textit{nə-an-ba=jaa}

      any.more  talk-UMRK-PTCP  seed=NOM  exist-NEG-CSL=SOL

      ‘There is no seed to talk (about).’

      [Co: 120415\_01.txt]

  Main clause

  c.  TM:  an  saeetu  ujuribəidu  kjun.

      \textit{a-n}  \textit{saee=tu}  \textit{ujuri=bəi=du}  \textit{k-jur-n}

      DIST-ADNZ  Sae=COM  Uyuri=only=FOC  come-UMRK-PTCP

      ‘Those (people, i.e.) Sae and Uyuri come (to the meeting for old people).’

      [Co: 120415\_01.txt]

  Adverbial clause

  d.  TM:  wanna  honami{\textbar}cjan{\textbar}  naaja  siccjunban,

      [\textit{wan=ja}  \textit{honami-cjan}  \textit{naa=ja}  \textit{sij-tur-n=ban}]\textsubscript{Adverbial clause}

      1SG=TOP  Honami-DIM  name=TOP  know-PROG-PTCP=ADVRS

      naakjaa  jumɨnu  naaja  sijandoojaa.

      \textit{naakja-a}  \textit{jumɨ=nu}  \textit{naa=ja}  \textit{sij-an=doo=jaa}

      2.HON.PL-ADNZ  daughter.in.law=GEN  name=TOP  know-NEG=ASS=SOL

      ‘I know the name of Honami, but do not know your daughter in law’s name.’

      [Co: 110328\_00.txt]

In (8-80 a), the participle /succjun/ \textit{sukk-tur-n} (pull-PROG-PTCP) fills the predicate of the adnominal clause, which modifies \textit{cˀju} ‘person.’ Similarly, in (8-80 b), the participle /hanasjun/ \textit{hanas-jur-n} (talk-UMRK-PTCP) fills the predicate of the adnominal clause, which modifies \textit{tanɨ} ‘topic.’ In (8-80 c), the participle /kjun/ \textit{k-jur-n} (come-UMRK-PTCP) fills the predicate of the main clause. When the participle terminates a sentence, there is always the focus marker \textit{du} in the sentence (see aslo \sectref{sec:key:11.3}). In fact, the sentence terminated by the participle that ends with \textit{{}-n} (PTCP) is not permitted by the speaker in elicitation. However, it appears in the texts several times. This interrelationship between \textit{du} (FOC) and \textit{{}-n} (PTCP) is similar to that of the focused constituents and \textit{-u} (PFC) in \sectref{sec:key:8.4.1.6.} These phenomena are called \textit{kakari-musubi} (i.e. ‘government-predication’) in Japanese linguistics, and their details will be discussed in \sectref{sec:key:11.3.} In (8-80 d), the participle /siccjun/ \textit{sij-tur-n} (know-PROG-PTCP) is followed by the conjunctive particle \textit{ban} (ADVRS), and fills the predicate of the adverbial clause. It should be noted that there is a saying as in (8-81), where the function of the participle is not very clear.

(8\nobreakdash-81)  Saying

  TM:  kamjun  cikjaradu  attoo.

    \textit{kam-jur-n}  \textit{cikjara=du}  \textit{ar=doo}

    eat-UMRK-PTCP  power=FOC  exist=ASS

    ‘If (you) eat (much), (you will have) power.’

    [Co: 120415\_01.txt]

In (8-81), the participle /kamjun/ \textit{kam-jur-n} (eat-UMRK-PTCP) functions like a converb meaning ‘if (you) eat (much).’ There is no other expression like (8-81) in Yuwan, so this saying seems to be a fossilized expression.

\subsubsection{\textit{{}-an} (NEG)}

The participial affix \textit{{}-an} (NEG) can directly follow the verbal roots (see \sectref{sec:key:8.1} for more details).

(8\nobreakdash-82)  Verbal morphemes that can directly precede \textit{{}-an} (NEG) (Participial affix; Group I)

  Root  \textit{{}-as  {}-arɨr  {}-tuk  {}-arɨr  {}-tur  {}-jawur  {}-an}

    CAUS  PASS  PRPR  CAP  PROG  POL  NEG

In contrast with \textit{{}-n} (PTCP), the participle composed of \textit{{}-an} (NEG) usually fills the predicate slot of a main clause as in (8-83 a), but it can fill that of an adnominal clause as in (8-83 b) or an adverbial clause as in (8-83 c-d).

(8\nobreakdash-83)  \textit{{}-an} (NEG)

  Main clause

  a.  TM:  kun  {\textbar}sjensjee{\textbar}ja  sijandoo.

      \textit{ku-n}  \textit{sjensjee=ja}  \textit{sij-an=doo}

      PROX-ADNZ  teacher=TOP  know-NEG=ASS

      ‘(I) don’t know this teacher (in the picture).’

      [Co: 120415\_00.txt]

  Adnominal clause

  b.  TM:  kˀwaga  dɨkɨran  cˀju  natɨ,

      [\textit{kˀwa=ga}  \textit{dɨkɨr-an}]\textsubscript{Adnominal clause}  \textit{cˀju}  \textit{nar-tɨ}

      child=NOM  be.born-NEG  person  COP-SEQ

      ‘Since (the woman) was a person who cannot have a baby, ...’

      [Co: 120415\_00.txt]

  Adverbial clauses

  c.  TM:  urɨnkjaba  jˀicjutɨga,  warəəcjɨjo,

      \textit{u-rɨ=nkja=ba}  \textit{j-tur-tɨ=ga}  \textit{waraw-ɨ=ccjɨ=joo}

      MES-NLZ=APPR=ACC  say-PROG-SEQ=FOC  laugh-INF=QT=CFM1

      {\textbar}nankai{\textbar}n, ...  {\textbar}hakkiri{\textbar}  jˀiikijansjutɨ.

      \textit{nankai=n}  [\textit{hakkiri}  \textit{jˀ-i+kij-an=sjutɨ}]\textsubscript{Adverbial clause}

      what.time=even  clearly  say-INF+CAP-NEG=since

      ‘(I) laughed saying those things many times, ... since (I) couldn’t pronounce (them) clearly.’

      [Co: 110328\_00.txt]

  d.  TM:  un  kawajəəka  sijanban,  (nasinu  miicɨ,)

      [\textit{u-n}  \textit{kawajəə=ka}  \textit{sij-an=ban}]\textsubscript{Adverbila clause}  \textit{nasi=nu}  \textit{miicɨ}

      MES-ADNZ  instead=DUB  know-NEG=ADVRS  pear=GEN  three.thing

      {\textbar}sanninzure{\textbar}  jatattu,  nasinu  miicɨ  muratɨ,

      \textit{sanninzure}  \textit{jar-tar-tu}  \textit{nasi=nu}  \textit{miicɨ}  \textit{muraw-tɨ}

      three.people  COP-PST-CSL  pear=GEN  three.thing  receive-SEQ

      ‘(I) don’t know whether (the boys gave the pears) in return (for) the (help), but (the boys) received three pears, since there were three, and ...’

      [PF: 090225\_00.txt]

In (8-83 a), the participle \textit{sij-an} (know-NEG) fills the predicate of the main clause, where the clause-final particle \textit{doo} (ASS) follows it. In (8-83 b), the participle \textit{dɨkɨr-an} (be.born-NEG) fills the predicate of the adnominal clause, which modifies \textit{cˀju} ‘person.’ In (8-83 c), the participle /jˀiikijan/ \textit{jˀ-i+kij-an} (say-INF+CAP-NEG) is followed by the conjunctive particle \textit{sjutɨ} ‘since,’ and fills the predicate of the adverbial clause. Similarly in (8-83 d), the participle \textit{sij-an} (know-NEG) is followed by the conjunctive particle \textit{ban} (ADVRS), and fills the predicate of the adverbial clause. It should be noted that \textit{{}-an} (NEG) can also fill the non-word-final position (see \sectref{sec:key:8.1}). In that case, the \textit{{}-an} (NEG) does not paradigmatically contrast with \textit{{}-n} (PTCP); in fact, they can co-occur (see \sectref{sec:key:8.5.1.9} for more details).

\subsection{Converb (verbal adverb)}

A converb is a verbal form that ends with a converbal affix in (8-84). Converbs cannot include the past tense affix \textit{{}-tar} (with the exceptions of \textit{{}-tu} (CSL) and \textit{{}-too} (CSL)). Converbs can fill the verbal predicate slot of an adverbial clause and also a main clause. The converbal affixes can be separated by their functions.

(8\nobreakdash-84)  Converbal affixes

  a.  Causal

    \textit{{}-ba} (CSL), \textit{{}-tu} (CSL), and \textit{{}-too} (CSL)

  b.  Conditional

    \textit{{}-boo} (CND)

  c.  Listing

    \textit{{}-tai} (LST)

  d.  Temporal relation

    \textit{{}-gadɨ} ‘until,’ \textit{{}-jagacinaa} (SIM), and \textit{{}-təəra} ‘after’

  e.  Sequential

    \textit{{}-tɨ} (SEQ)

As mentioned in \sectref{sec:key:8.1}, the converbal affixes can be separated into two groups, i.e. Group-I affixes or Group-II affixes. The converbal affixes \textit{{}-tu} (CSL) and \textit{{}-too} (CSL) belong to Group-II affixes, and cannot directly follow any verbal root. It should be mentioned that \textit{{}-tu} (CSL) and \textit{{}-too} (CSL) always follow the past tense affix \textit{{}-tar}, although \textit{{}-tu} (CSL) can also follow \textit{{}-təər} (RSL). A complete list of the possible combinations of 17 types of verbal stems (see \sectref{sec:key:8.2.1}) and the converbal affixes will be shown in appendix. Many of the converbs in (8-84) can take their own subject different from that of the main (or superordinate) clause, although the two convebs \textit{{}-tai} (LST) and \textit{{}-jagacinaa} (SIM) cannot. According to the criteria introduced by \citet[98-99]{Nedjalkov1995}, who did a typological overview of the converbs, almost all of the converbs in Yuwan can be grouped into “conjunctional converbs,” which has “(t)he function of the predicate of a subordinate clause” and “can have its own subject (i.e. subject different from the subject of the superordinate verb)” (ibid: 99). However, \textit{{}-tɨ} (SEQ) may be categorized as “coordinative converbs,” which has “(t)he function of a secondary or coordinate predicate” and “is similar to the function of the English conjunction \textit{and} (sometimes \textit{but}) or to asyndetic coordination” (ibid: 98). Furtheremore, \textit{{}-tai} (LST) may be categorized as “converbs proper,” which has “(t)he function of an adverbial in a simple sentence” (ibid: 98) (see also \sectref{sec:key:9.1.2.1} on the LVC composed of \textit{{}-tai} (LST) and \textit{sir-} ‘do’), although there is a case where \textit{{}-tai} (LST) seems to head a clause as in (8-93 a) in \sectref{sec:key:8.4.3.3.} \textit{{}-jagacinaa} (SIM) does not seem to fit any one of the criteria perfectly.

  In principle, the converbs behave like the adverb in their “external syntax” (see \sectref{sec:key:8.4}). However, \textit{{}-təəra} ‘after’ and \textit{{}-tɨ} (SEQ) sometimes behave like the nominal (see \sectref{sec:key:8.4.3.4} and \sectref{sec:key:9.3.2.2}). It is probable that these affixes will be classified into another new inflectional category in an alternative analysis.

In the following subsections, I will present the contrasts shown in (8-84) in turn.

\subsubsection{Causal: \textit{{}-ba} (CSL), \textit{{}-tu} (CSL), and \textit{{}-too} (CSL)}

The converbal affixes \textit{{}-ba} (CSL), \textit{{}-tu} (CSL), and \textit{{}-too} (CSL) fill the predicate of adverbial clauses, which express the cause for the event of the superordinate clause. They are very similar in function to each other, but morphologically the former, i.e. \textit{{}-ba} (CSL), and the latters, i.e. \textit{{}-tu} (CSL) and \textit{{}-too} (CSL), are nearly in complemantary distribution. On the one hand, \textit{{}-ba} (CSL) belongs to Group-I affixes. Thus, it can directly follow a verbal root. Additionally, it can follow all of the derivational affixes and the inflectional affix \textit{{}-an} (NEG), but cannot follow \textit{{}-tar} (PST) as in (8-85 a). On the other hand, \textit{{}-tu} (CSL) and \textit{{}-too} (CSL) almost always follow \textit{{}-tar} (PST), and rarely \textit{{}-tu} (CSL) follows \textit{{}-təər} (RSL) as in (8-85 b-c). Both \textit{-tu} (CSL) and \textit{{}-too} (CSL) begin with //t//, but they do not conform to the morphophonological rules for Type-B affixes discussed in \sectref{sec:key:8.2.1.2.} Rather, they conform to the rules for Type-D affixesin \sectref{sec:key:8.2.1.4.}

(8\nobreakdash-85)  a. Verbal morphemes that can directly precede \textit{{}-ba} (CSL) (Converbal affix; Group I)

  Root  \textit{{}-as  {}-arɨr} %%[Warning: Draw object ignored]
\textit{{}-tuk  {}-arɨr  {}-tur  {}-jawur} %%[Warning: Draw object ignored]
\textit{{}-an  {}-təər  {}-tar  {}-ba} (CSL)

    CAUS  PASS  PRPR  CAP  PROG  POL  NEG  RSL  PST  

          \textit{{}-jur} 

          UMRK    

  b. Verbal morphemes that can directly precede \textit{{}-tu} (CSL) (Converbal affix; Group II)

  Root  \textit{{}-as  {}-arɨr} %%[Warning: Draw object ignored]
\textit{{}-tuk  {}-arɨr  {}-tur  {}-jawur} %%[Warning: Draw object ignored]
\textit{{}-an  {}-təər  {}-tar  {}-tu} (CSL)

    CAUS  PASS  PRPR  CAP  PROG  POL  NEG  RSL  PST  

          \textit{{}-jur} 

          UMRK    

  c. Verbal morphemes that can directly precede \textit{{}-too} (CSL) (Converbal affix; Group II)

  Root  \textit{{}-as  {}-arɨr} %%[Warning: Draw object ignored]
\textit{{}-tuk  {}-arɨr  {}-tur  {}-jawur} %%[Warning: Draw object ignored]
\textit{{}-an  {}-təər  {}-tar  {}-too} (CSL)

    CAUS  PASS  PRPR  CAP  PROG  POL  NEG  RSL  PST  

          \textit{{}-jur} 

          UMRK    

The affixes deleted by double lines indicate that they cannot directly precede the word-final affix. The combinations in (8-85) show that \textit{{}-ba} (CSL) is used only in the non-past tense, but that \textit{{}-tu} (CSL) and \textit{{}-too} (CSL) are used almost only in the past tense. In fact, the combination of \textit{{}-təər} (RSL) and \textit{{}-tu} (CSL) is very rare in my texts. This means that the contrast of \textit{{}-ba} (CSL) vs. \textit{{}-tu}/\textit{{}-too} (CSL) is made by the tense opposition. In fact, \textit{-too} (CSL) is always preceded by \textit{{}-tar} (PST). Thus, one may think that \textit{{}-tar} (PST) and \textit{{}-too} (CSL) form a single portmanteau morpheme, i.e. \textit{{}-tattoo} (PST.CSL). I do not propose this analysis simply because of the covenience of showing the complementary distributions among the affixes in (8-85 a-c).

  First, I will present examples of \textit{{}-ba} (CSL).

(8\nobreakdash-86)  \textit{{}-ba} (CSL)

  a.  [Context: MY asked TM if TM had made the pickles; TM: ‘(I) don’t know. How (was it)?’]

    TM:  nɨɨzinnu  appa,  arandaroo.

      \textit{nɨɨzin=nu}  \textit{ar-ba}  \textit{ar-an=daroo}

      carrot=NOM  exist-CSL  COP-NEG=SUPP

      ‘There are (pieces of) a carrot, so maybe (the pickles) are not (mine).’

      [Co: 101023\_01.txt]

  b.  TM:  umanan  mata  nagɨcɨkɨtəəppa,

      \textit{u-ma=nan}  \textit{mata}  \textit{nagɨr-Ø+cɨkɨr-təər-ba}

      MES-place=LOC1  again  throw-INF+attach-RSL-CSL

      urɨ  tɨɨ  kɨɨnnajoocjɨ.

      \textit{u-rɨ}  \textit{tɨɨ}  \textit{kɨɨr-na=joo=ccjɨ}

      MES-NLZ  hand  hang-PROH=CFM1=QT

      ‘(My husband said) that, “(The person) have thrown (some sweets) again (at our house), so don’t touch it.”’

      [Co: 120415\_01.txt]

The above examples show that \textit{{}-ba} (CSL) has causal meaning. Interestingly, if \textit{{}-ba} (CSL) follows the auxiliary verbs \textit{kurɨr-} (BEN) or \textit{taboor-} (BEN.HON) without the superordinate clauses, it means the “request” for the hearer (see \sectref{sec:key:11.2.2} for more details).

  Secondly, I will present examples of \textit{{}-tu} (CSL). It should be noted that \textit{-an} (NEG) cannot “directly” precede \textit{{}-tu} (CSL), but it can “indirectly” precede it with \textit{{}-tar} (PST) as in (8-87 c).

(8\nobreakdash-87)  \textit{{}-tu} (CSL)

  a.  TM:  boosi  utucjəətattu,  urɨ  jaraccjɨ,

      \textit{boosi}  \textit{utus-təər-tar-tu}  \textit{u-rɨ}  \textit{jaras-tɨ}

      hat  drop-RSL-PST-CSL  MES-NLZ  give-SEQ

      ‘(The boy) have dropped (his) hat, so (the three boys picked it up and) handed it (to him), and ...’

      [PF: 090305\_01.txt]

  b.  [= (5-39 b)]

    TM:  nuucjɨgajaaroo  kacjəəttujaa.

      \textit{nuu=ccjɨ=gajaaroo}  \textit{kak-təər-tu=jaa}

      what=QT=DUB  write-RSL-CSL=SOL

      ‘Something has been drawn (on the sign board of the store).’

      [Co: 120415\_00.txt]

  c.  TM:  uci(ga)zjasiga  siikijantattu,  waakjaa

      \textit{ut-i+izjas-i=ga}  \textit{sɨr-i+kij-an-tar-tu}  \textit{waakja-a}

      hit-INF+put.out-INF=NOM  do-INF+CAP-NEG-PST-CSL  1PL-ADNZ

      anmaaja  gan  sjɨ  uta  jusɨrooccjɨ,

      \textit{anmaa=ja}  \textit{ga-n}  \textit{sɨr-tɨ}  \textit{uta}  \textit{jusɨr-oo=ccjɨ}

      mother=TOP  MED-ADVZ  do-SEQ  song  teach-INT=QT

      ‘(I) couldn’t start hitting (the hand drums in singing), so my mother (tried) to teach (me) the (traditional) songs like this, and ...’

      [Co: 111113\_01.txt]

\textit{{}-tu} (CSL) is sometims realized as /tuu/ as in (9-20 c) in \sectref{sec:key:9.1.1.4.}

Not only the morphological environmetns, but also the meanings of \textit{{}-tu} (CSL) and \textit{{}-too} (CSL) are very similar to each other. However, there seems to be the tendency that the causal relationships between the adverbial clause and the superordinate clause bound by \textit{-too} (CSL) are more arbitrary than those by \textit{{}-tu} (CSL). In other words, the causal relationships bound by \textit{{}-too} (CSL) seem to be naturally translated into ‘and then’ in English as in (8-88 a-c).

(8\nobreakdash-88)  \textit{{}-too} (CSL)

  a.  TM:  miicɨ  nasi  kurɨtattoo,  un  micjaija

      \textit{miicɨ}  \textit{nasi}  \textit{kurɨr-tar-too}  \textit{u-n}  \textit{micjai=ja}

      three.things  pear  give-PST-CND  MES-ADNZ  three.person=TOP

      jurukudɨ,  kan  sjɨ  hucjutɨ,

      \textit{jurukub-tɨ}  \textit{ka-n}  \textit{sɨr-tɨ}  \textit{huk-tur-tɨ}

      be.pleased-SEQ  PROX-ADVZ  do-SEQ  wipe-PROG-SEQ

      ‘(The boy) gave (them) pears, and then those three (boys) were pleased, and were wiping (the pears) like this, and ...’

      [PF: 090827\_02.txt]

  b.  TM:  urəə  mata  taruga  jatakai?

      \textit{u-rɨ=ja}  \textit{mata}  \textit{ta-ru-Ø=ga}  \textit{jar-tar=kai}

      MES-NLZ=TOP  again  who-NLZ-SG=NOM  COP-PST=DUB

      cˀjutattoo,  (urɨ,)  mukasinu  {\textbar}zjuukunu

      \textit{k-tur-tar-too}  \textit{u-rɨ}  \textit{mukasi=nu}  \textit{zjuuku=nu}

      come-PROG-PST-CSL  MES-NLZ  past=GEN  ten.nine=GEN

      haru{\textbar}ja  kurɨdu  utajutattujaacjɨ  jˀicjɨ,

      \textit{haru=ja}  \textit{ku-rɨ=du}  \textit{utaw-jur-tar-tu=jaa=ccjɨ}  \textit{jˀ-tɨ}

      spring=TOP  PROX-NLZ=FOC  sing-UMRK-PST-CSL=SOL=QT  say-SEQ

      ‘And who was that person (who had brought the pamphlet of songs)? (Anyway, a person) was coming (here), and then (the person) said that, “(We) sang the old song \textit{The} \textit{spring} \textit{in} \textit{nineteen} \textit{years} \textit{old} with this (pamphlet), so (it is very familiar to us).”’

  c.  TM:  kˀwan  dɨkɨrantattoo,  nusjəə  jaakara

      \textit{kˀwa=n}  \textit{dɨkɨr-an-tar-too}  \textit{nusi=ja}  \textit{jaa=kara}

      child=even  be.born-NEG-PST-CSL  RFL=TOP  house=ABL

      izibatɨ  izjanwake.

      \textit{izibar-tɨ}  \textit{ik-tar-n=wake}

      go.out-SEQ  go-PST-PTCP=CFP

      ‘(The person) cannot have a baby, and then (the person) went out the house.’

      [Co: 120415\_00.txt]

It should be noted again that \textit{-an} (NEG) cannot “directly” precede \textit{{}-too} (CSL), but it can “indirectly” precede it with \textit{{}-tar} (PST) as in (8-88 c).

\subsubsection{Conditional: \textit{{}-boo} (CND)}

The converbal affix \textit{{}-boo} (CND) fills the predicates of adverbial clauses that express the condition that can realize the event of the superordinate clause. \textit{{}-boo} (CND) belongs to Group-I affixes. Thus, it can directly follow a verbal root. Additionally, it can follow all of the derivational affixes and the inflectional affix \textit{{}-an} (NEG), but cannot follow \textit{{}-tar} (PST) as in (8-89).

(8\nobreakdash-89)  Verbal morphemes that can directly precede \textit{{}-boo} (CND) (Converbal affix; Group I)

  Root  \textit{{}-as  {}-arɨr} %%[Warning: Draw object ignored]
\textit{{}-tuk  {}-arɨr  {}-tur  {}-jawur} %%[Warning: Draw object ignored]
\textit{{}-an  {}-təər  {}-tar  {}-boo} (CND)

    CAUS  PASS  PRPR  CAP  PROG  POL  NEG  RSL  PST  

          \textit{{}-jur} 

          UMRK    

\textit{{}-boo} (CND) cannot follow \textit{{}-tar} (PST). However, \textit{-boo} (CND) can be used to express the situation that occured in the past as in (8-90 c).

(8\nobreakdash-90)  \textit{{}-boo} (CND)

  a.  TM:  kuci  hɨɨsanma  akɨppoo,  {\textbar}ireba{\textbar}nu

      \textit{kuci}  \textit{hɨɨ-sanma}  \textit{akɨr-boo}  \textit{ireba=nu}

      mouth  wide-ADVZ  open-CND  artificial.tooth=NOM

      utɨjunkara,

      \textit{utɨr-jur-n=kara}

      drop-UMRK-PTCP=CSL

      ‘If (I) open the mouth wide, the artificial teeth will fall out, so ...’

      [Co: 110328\_00.tx]

  b.  [Context: TM said that the hearer MY was better than her, since MY could walk around only with a stick.]

    TM:  wanna  arɨ  usanboo,  aikikijanba.

      \textit{wan=ja}  \textit{a-rɨ}  \textit{us-an-boo}  \textit{aik-i+kij-an-ba}

      1SG=TOP  PROX-NLZ  push-NEG-CND  walk-INF+CAP-NEG-CSL

      ‘If I don’t push that [i.e. handcart], (I) cannot walk (around) (so I think you are better than me).’

      [Co: 110328\_00.txt]

  c.  TM:  {\textbar}kjonen{\textbar}bəikara  mioja{\textbar}kun{\textbar}  siccjuppoo,  jiccja

      \textit{kjonen=bəi=kara}  \textit{mioja-kun}  \textit{sij-tur-boo}  \textit{jiccj-sa}

      last.year=around=ABL  Mioya-N/A  do-PROG-CND  good-ADJ

      atənmundoojaa.

      \textit{ar-təər-n=mun=doo=jaa}

      STV-RSL-PTCP=ADVRS=ASS=SOL

      ‘If (I) had known Mioya since around the last year, (it) would have been good (but unfortunately I haven’t known him that long).’

      [Co: 111113\_02.txt]

  d.  TM:  naa  naratuppoo,  {\textbar}gomennasai{\textbar}cjɨnkjoo

      \textit{naa}  \textit{naraw-tur-boo}  \textit{gomennasai=ccjɨ=nkja=ja}

      already  get.accustomed-PROG-CND  I.am.sorry=QT=APPR=TOP

      jˀiimicjəə  sijan.

      \textit{jˀ-i+mici=ja}  \textit{sij-an}

      say-INF+way=TOP  know-NEG

      ‘(I) have already got accustomed to (the present author), and then (I) didn’t remember to say, “I’m sorry” (when I forgot to serve the tea when he visited here).’

      [Co: 110328\_00.txt]

  e.  TM:  tˀaija  amanan  taccjuppoo,

      \textit{tˀai=ja}  \textit{a-ma=nan}  \textit{tat-tur-boo}

      two.person=TOP  DIST-place=LOC  stand-PROG-CND

      un  cˀjuiga  muccjattoo,

      \textit{u-n}  \textit{cˀjui=ga}  \textit{mukk-tar-too}

      MES-ADNZ  one.person=NOM  bring-PST-CSL

      ‘Two (of the three boys) were standing there, and then the one (of them who remained) brought (pears), and then ...’

      [PF: 090827\_02.txt]

In the first three examples (8-90 a-c), \textit{{}-boo} (CND) expresses the conditional meaning such as ‘if’ in English. However, in the last two examples (8-90 d-e), \textit{{}-boo} (CND) expresses the meaning such as ‘and then’ in English, which is similar to the meaning expressed by \textit{{}-too} (CSL) in \sectref{sec:key:8.4.3.1.} Interestingly, the combination of \textit{{}-an} (NEG) plus \textit{{}-boo} (CND) has come to be used without a main clause, where the combination means an obligatory meaning such as ‘has to’ (see \sectref{sec:key:11.2.4} for more details).

  Before concluding this section, I want to present an affix, i.e. \textit{{}-tarabacjɨ}, which expresses a concessive meaning such as ‘even if’ in English. This affix has not appeared in my texts, but it was found in elicitation.

(8\nobreakdash-91)  \textit{{}-tarabacjɨ} ‘even if’

  a.  TM:  gan  sjɨ  sjarabacjɨ,  nugoorasandoo.

      \textit{ga-n}  \textit{sɨr-tɨ}  \textit{sɨr-tarabacjɨ}  \textit{nugoor-as-an=doo}

      MES-ADVZ  do-SEQ  do-even.if  escape-CAUS-NEG=ASS

      ‘Even if (you) do that, (I) won’t let you escape.’

      [El: 120924]

  b.  TM:  uraga  ikjasaa  nacjarabacjɨ,  nugoorasandoo.

      \textit{ura=ga}  \textit{ikja-saa}  \textit{nak-tarabacjɨ}  \textit{nugoor-as-an=doo}

      2.NHON.SG=NOM  how-ADVZ  cry-even.if  escape-CAUS-NEG=ASS

      ‘No matter how much you cry, (I) won’t let you escape.’

      [El: 120924]

Interestingly, the verb form ending with \textit{{}-tarabacjɨ} deprives the questional meaning of the interrogative word \textit{ikja-saa} (how-ADVZ) ‘how much.’ \textit{{}-tarabacjɨ} ‘even if’ may be divided into \textit{{}-tar} (PST) plus \textit{{}-abacjɨ} ‘even if,’ since it is common for the past-tense morpheme to be used in the counterfactual proposition such as the subjunctive mood in English. We need to clarify the details of this affix in future research.

\subsubsection{Listing: \textit{{}-tai} (LST)}

The converbal affix \textit{{}-tai} (LST) means that there are several events, and that the speaker indicates one (or a few) of the events using it. The following affixes can precede \textit{{}-tai} (LST). The affixes deleted by double lines cannot directly precede \textit{{}-tai} (LST).

(8\nobreakdash-92)  Verbal morphemes that can directly precede \textit{{}-tai} (LST) (Converbal affix; Group I)

  Root  \textit{{}-as  {}-arɨr} %%[Warning: Draw object ignored]
\textit{{}-tuk  {}-arɨr  {}-tur  {}-jawur} %%[Warning: Draw object ignored]
\textit{{}-an  {}-təər  {}-tar  {}-tai} (LST)

    CAUS  PASS  PRPR  CAP  PROG  POL  NEG  RSL  PST  

          \textit{{}-jur} 

          UMRK    

I will present examples of \textit{{}-tai} (LST).

(8\nobreakdash-93)  \textit{{}-tai} (LST)

  a.  TM:  nunkuin  jusɨtɨ  kurɨtai,  urɨ  sjɨ  kurɨtan

      \textit{nuu-nkuin}  \textit{jusɨr-tɨ}  \textit{kurɨr-tai}  \textit{u-rɨ}  \textit{sɨr-tɨ}  \textit{kurɨr-tar-n}

      what-INDFZ  teach-SEQ  BEN-LST  MES-NLZ  do-SEQ  BEN-PST-PTCP

      cˀjunu  kutoo  (umui,  ɨɨ)  wasɨrɨrannən,  urɨ

      \textit{cˀju=nu}  \textit{kutu=ja}  \textit{umuw-i}    \textit{wasɨrɨr-annən}  \textit{u-rɨ}

      person=GEN  event=TOP  think-INF    forget-NEG.SEQ  MES-NLZ

      sjunban,

      \textit{sɨr-jur-n=ban}

      do-UMRK-PTCP=ADVRS

      ‘About a person who taught (me) everything and did it [i.e. the help] (for me), (I) don’t forget (the person), and do it [i.e. remember], but ...’

      [Co: 120415\_01.txt]

  b.  TM:  uba\footnote{The regular morphophonological alternation is \textit{u-rɨ=ba} (MES-NLZ=ACC) > /uppa/, but it sounds like /uba/ here.}  (mm)  uziija  jukkadɨ  nubutai

      \textit{u-rɨ=ba}    \textit{uzii=ja}  \textit{jukkadɨ}  \textit{nubur-tai}

      MES-NLZ=ACC    old.man=TOP  continuously  climb-LST

      urɨtai  sjutɨ,  nasi  mutuii.

      \textit{urɨr-tai}  \textit{sɨr-tur-tɨ}  \textit{nasi}  \textit{mur-tur-i}

      decend-LST  do-PROG-SEQ  pear  pick.up-PROG-INF

      ‘The old man kept climbing and descending it [i.e. the ladder], and was picking up the pears.’

      [PF: 090827\_02.txt]

In (8-93 a), the VP /jusɨtɨ kurɨtai/ \textit{jusɨr-tɨ} \textit{kurɨr-tai} (teach-SEQ BEN-LST) ‘teaching (everything to me), and ...’ fills the the head of an adverbial clause, and the superordinate clause again functions as an adnominal clause, which modifies \textit{cˀju} ‘person.’ In (8-93 b), the converbs /nubutai/ \textit{nubur-tai} (climb-LST) ‘climbming, and ...’ and /urɨtai/ \textit{urɨr-tai} (decend-LST) ‘descending, and ...’ fill the complement slot of the light verb construction (see also \sectref{sec:key:9.1.2} for the light verb construction).

\subsubsection{Temporal relation: \textit{{}-gadɨ} ‘until,’ \textit{{}-jagacinaa} (SIM), and \textit{{}-təəra} ‘after’}

The converbal affixes \textit{{}-gadɨ} ‘until,’ \textit{{}-jagacinaa} (SIM), and \textit{{}-təəra} ‘after’ can express temporal relationships between the events expressed by the adverbial clauses and those of the superordinate clauses. First, \textit{{}-gadɨ} ‘until’ indicates the time until which the event of the modified clause continues. It can directly follow these verbal morphemes in (8-94). The affixes deleted by double lines cannot directly precede the word-final affix.

(8\nobreakdash-94)  Verbal morphemes that can directly precede \textit{{}-gadɨ} ‘until’ (Converbal affix; Group I)

  Root  \textit{{}-as  {}-arɨr} %%[Warning: Draw object ignored]
\textit{{}-tuk  {}-arɨr  {}-tur  {}-jawur} %%[Warning: Draw object ignored]
\textit{{}-an  {}-təər  {}-tar  {}-gadɨ} ‘until’

    CAUS  PASS  PRPR  CAP  PROG  POL  NEG  RSL  PST  

          \textit{{}-jur} 

          UMRK    

It is probable that \textit{{}-gadɨ} ‘until’ is cognate with the limiter particle \textit{gadɨ} (LMT). However, \textit{{}-gadɨ} ‘until’ can directly attach to the verbal root. On the other hand, any particle cannot follow the verbal root directly (except for \textit{kai} (DUB)). Thus, I propose that \textit{{}-gadɨ} ‘until’ is a morpheme different from \textit{gadɨ} (LMT) in modern Yuwan. Examples of \textit{{}-gadɨ} ‘until’ are shown below.

(8\nobreakdash-95)  \textit{{}-gadɨ} ‘until’

  a.  TM:  naakja  kˀuugadɨ,  wutarooga?

      \textit{naakja}  \textit{k-gadɨ}  \textit{wur-tar-oo=ga}

      2.HON.PL  come-until  exist-PST-SUPP=CFM3

      ‘(I) suppose (that) until you came (here), (the person) had been (there, hadn’t he)?’

      [Co: 110328\_00.txt]

  b.  TM:  waakjoo  {\textbar}socugjoo{\textbar}  sɨkkadɨ  kuzɨɨ  hakandoojaa.

      \textit{waakja=ja}  \textit{socugjoo}  \textit{sɨr-gadɨ}  \textit{kuzɨ}  \textit{hak-an=doo=jaa}

      1PL=TOP  graduation  do-until  shoe  put.on-NEG=ASS=SOL

      ‘I hadn’t put on shoes until (I) graduated (from elementary school).’

      [Co: 110328\_00.txt]

Interestingly, \textit{{}-gadɨ} expresses a meaning different from ‘until’ if it is followed by the particle \textit{n} ‘even,’ i.e. \textit{{}-gadɨ=n} ‘by the time.’

(8\nobreakdash-96)  \textit{{}-gadɨ} ‘until’ + \textit{n} ‘even’

  MS:  ikugadɨnnja  kinunkja  kəətukɨjoo.

    \textit{ik-gadɨ=n=ja}  \textit{kin=nkja}  \textit{kəər-tuk-ɨ=joo}

    go-until=even=TOP  clothes=APPR  change-PRPR-IMP=CFM1

    ‘By the time (you) go (out), change (your) clothes (to the formal ones), right?’

    [El: 120926]

  Secondly, \textit{{}-jagacinaa} (SIM) indicates the time during which the event of the modified clause occurs. It can directly follow only the verbal root, or two derivational affixes \textit{{}-as} (CAUS) and \textit{{}-arɨr} (PASS) as in (8-97).

(8\nobreakdash-97)  Verbal morphemes that can directly precede \textit{{}-jagacinaa} (SIM) (Converbal affix; Group I)

  Root  \textit{{}-as  {}-arɨr} %%[Warning: Draw object ignored]
\textit{{}-tuk  {}-arɨr  {}-tur  {}-jawur} %%[Warning: Draw object ignored]
\textit{{}-an  {}-təər  {}-tar  {}-jagacinaa} (SIM)

    CAUS  PASS  PRPR  CAP  PROG  POL  NEG  RSL  PST  

          \textit{{}-jur} 

          UMRK    

Morphophonologically, the //ci// of \textit{{}-jagacinaa} (SIM) may be omitted. For example, \textit{ik-jagacinaa} (go-SIM) can be realized either as /ikjagacinaa/ or /ikjaganaa/. Additionally, there is another form that express the same meaning with \textit{{}-jagacinaa} (SIM), i.e. \textit{{}-ganaa} (SIM). \textit{{}-ganaa} (SIM) always needs to be preceded by \textit{{}-i}/\textit{{}-Ø} (INF), e.g. \textit{ik-i-ganaa} (go-INF-SIM). Among them, \textit{{}-jagacinaa} (SIM) is most productive. Therefore, I will present only examples of \textit{{}-jagacinaa} (SIM) below.

(8\nobreakdash-98)  \textit{{}-jagacinaa} (SIM)

  a.  TM:  kusa  musijagacinan,  jukkadɨ  uta.

      \textit{kusa}  \textit{musij-jagacinaa=n}  \textit{jukkadɨ}  \textit{uta}

      grass  pull-SIM=even  always  song

      ‘Even while (my mother) was pulling weeds, (she was) always (singing) a song.’

      [Co: 111113\_01.txt]

  b.  TM:  ikjasjɨga  sjuruccjɨ,  nattəənkja  hanasjagacinaa,

      \textit{ikja-sjɨ=ga}  \textit{sɨr-jur-u=ccjɨ}  \textit{naa-ttəə=nkja}  \textit{hanas-jagacinaa}

      how-ADVZ=FOC  do-UMRK-PFC=QT  2.HON-DU=APPR  talk-SIM

      kutusjəə  sjoogacɨja  urɨ  jappa,  un

      \textit{kutusi=ja}  \textit{sjoogacɨ=ja}  \textit{u-rɨ}  \textit{jar-ba}  \textit{u-n}

      this.year=TOP  New.Year’s.Day=TOP  MES-NLZ  COP-CSL  MES-ADNZ

      sjoogacɨ  nusjəə  usikkwa  kawuroojaacjɨ  jˀicjɨ

      \textit{sjoogacɨ}  \textit{nusi=ja}  \textit{usi-kkwa}  \textit{kawur-oo=jaa=ccjɨ}  \textit{jˀ-tɨ}

      New.Year’s.Day  REF=TOP  cow-DIM  buy-INT=SOL=QT  say-SEQ

      ‘The couple was saying, “What should (we) do?” and (said) that, “About the New Year’s Day in the next year [lit. this year], (the fact) is that [i.e. they don’t have a child]. Thus, let’s buy a cow by ourselves (on) the New Year’s Day.”’

      [Fo: 090307\_00.tx]

  Thirdly, \textit{{}-təəra} ‘after’ indicates the time after which the event of the modified clause occurs. It can directly follow only the verbal root, or two derivational affixes \textit{{}-as} (CAUS) and \textit{{}-arɨr} (PASS) as in (8-99).

(8\nobreakdash-99)  Verbal morphemes that can directly precede \textit{{}-təəra} ‘after’

  Root  \textit{{}-as  {}-arɨr} %%[Warning: Draw object ignored]
\textit{{}-tuk  {}-arɨr  {}-tur  {}-jawur} %%[Warning: Draw object ignored]
\textit{{}-an  {}-təər  {}-tar  {}-təəra} ‘after’

    CAUS  PASS  PRPR  CAP  PROG  POL  NEG  RSL  PST  

          \textit{{}-jur} 

          UMRK    

I will present examples of \textit{{}-təəra} ‘after.’

(8\nobreakdash-100)  \textit{{}-təəra} ‘after’

  a.  [= (6-117 d)]

    TM:  naakjaga  {\textbar}socugjoo{\textbar}  sjəəraga  waakjoo  {\textbar}gakkoo{\textbar}kai?

      \textit{naakja=ga}  \textit{socugjoo}  \textit{sɨr-təəra=ga}  \textit{waakja=ja}  \textit{gakkoo=kai}

      2.HON.PL=NOM  graduation  do-after=FOC  1PL=TOP  school=DUB

      ‘(Is it) after you had graduated (from the elementary school, when) I (began to go to) school?’

      [Co: 110328\_00.txt]

  b.  TM:  uninkara  hɨɨtəəraga,  uraa  məəci  {\textbar}denwa{\textbar}ba

      \textit{unin=kara}  \textit{hɨɨr-təəra=ga}  \textit{ura-a}  \textit{məə=kaci}  \textit{denwa=ba}

      that.time=ABL  wake.up-after=FOC  2.NHON.SG  front=ALL  phone=ACC

      sjəəraga,  bocuubocu  cɨra  aratɨ,

      \textit{sɨr-təəra=ga}  \textit{bocu+bocu}  \textit{cɨra}  \textit{araw-tɨ}

      do-after=FOC  RED+slowly  face  wash-SEQ

      ‘After waking up at that time, (and) after calling you, (I) washed my face, and ...’

      [Co: 101020\_01.txt]

  c.  TM:  juwannintəə  (xxx)  nkjoo  {\textbar}zjuusannici{\textbar}n  hii

      \textit{juwan+nintəə}    \textit{=nkja=ja}  \textit{zjuusannici=n}  \textit{hii}

      Yuwan+people    =APPR=TOP  ten.three.day=GEN  day

      hakaba  izjɨ  cˀjəəra,  ujahuzimacɨiccjɨ  jˀicjɨ,

      \textit{haka=ba}  \textit{ik-tɨ}  \textit{k-təəra}  \textit{ujahuzi+macɨr-i=ccjɨ}  \textit{jˀ-tɨ}

      tomb=ACC  go-SEQ  come-after  ancestor+celebrate-INF=QT  say-SEQ

      ujahuzinu  (mm)  sɨnsoomutukaci  miinna  acɨmɨtɨ,

      \textit{ujahuzi=nu}    \textit{sɨnsoomutu=kaci}  \textit{minna}  \textit{acɨmɨr-tɨ}

      ancestor=GEN    head.family=ALL  everybody  gather-SEQ

      ‘After going to and coming back from the tomb at the thirteenth day (of every month), the people of Yuwan, (who) called (the event) “the celebration of the ancestors,” gathered all of the relatives at the head family’s house.’

      [Co: 111113\_01.txt]

  d.  TM:  jakɨtəəranu  atuga  wakaran.

      \textit{jakɨr-təəra=nu}  \textit{atu=ga}  \textit{wakar-an}

      burn-after=GEN  after=NOM  understand-NEG

      ‘(I) don’t know (what happened) after (the houses) burned (because of the air raid in the World War II).’

      [Co: 120415\_01.txt]

  e.  [Context: TM was remembering the days when the present author came for the first time.]

    TM:  naa,  mutoo  cˀjəəranu  sɨgoo  koo  zja,  un  zja,

      \textit{naa}  \textit{mutu=ja}  \textit{k-təəra=nu}  \textit{sɨgu=ja}  \textit{koo}  \textit{zjar}  \textit{un}  \textit{zjar}

      FIL  first=TOP  come-after=GEN  soon=TOP  river  COP  sea  COP

      jama  zjaccjɨ  gan  sjan  munbəidu

      \textit{jama}  \textit{zjar=ccjɨ}  \textit{ga-n}  \textit{sɨr-tar-n}  \textit{mun=bəi=du}

      mountain  COP=QT  MES-ADVZ  do-PST-PTCP  thing=only=FOC

      tazɨnɨjutattujaa.

      \textit{tazɨnɨr-jur-tar-tu=jaa}

      ‘At first, immediately after (the present author) came (to TM’s place), (he) used to ask only these kinds of things (like) the river, the sea, and the mountain.’

      [Co: 111113\_02.txt]

  f.  TM:  kurɨ  josidanu  {\textbar}nikai{\textbar}nkjanu  dɨkɨtəəra

      \textit{ku-rɨ}  \textit{josida=nu}  \textit{nikai=nkja=nu}  \textit{dɨkɨr-təəra}

      PROX-NLZ  Yoshida=GEN  second.floor=APPR=NOM  be.built-after

      jappa.

      \textit{jar-ba}

      COP-CSL

      ‘This [i.e. the date when the outdoor lamp was set] is after Yoshida’s second floor was built.’

      [Co: 120415\_00.txt]

In (8-100 a-c), the clauses that include the verb forms composed of \textit{{}-teera} ‘after’ adverbially modify the following clauses. In (8-100 d-e), however, the clauses that include the verb forms composed of \textit{{}-teera} ‘after’ fill the modifier slot of an NP. In fact, they are followed by \textit{nu} (GEN). In (8-100 f), the clause that includes the verb form composed of \textit{{}-təəra} ‘after’ fills the NP slot of the nominal predicate phrase with a copula verb.

\subsubsection{Sequential: \textit{{}-tɨ} (SEQ) and \textit{{}-nən} (SEQ)}

The converbal affix \textit{{}-tɨ} (SEQ) and \textit{{}-nən} (SEQ) can express the sequential relationship between the events. In addition, the verbal form composed of \textit{{}-tɨ} (SEQ) is obligatorily used to fill the non-final verbal slot in AVC (see \sectref{sec:key:9.1.1} for more details). In (8-101 a-b), the affixes deleted by double lines cannot directly precede the word-final affix.

(8\nobreakdash-101)  a. Verbal morphemes that can directly precede \textit{{}-tɨ} (SEQ) (Converbal affix; Group I)

  Root  \textit{{}-as  {}-arɨr} %%[Warning: Draw object ignored]
\textit{{}-tuk  {}-arɨr  {}-tur  {}-jawur} %%[Warning: Draw object ignored]
\textit{{}-an  {}-təər  {}-tar  {}-tɨ} (SEQ)

    CAUS  PASS  PRPR  CAP  PROG  POL  NEG  RSL  PST  

          \textit{{}-jur} 

          UMRK    

  b. Verbal morphemes that can directly precede \textit{{}-nən} (SEQ) (Converbal affix; Group II)

  Root  \textit{{}-as  {}-arɨr} %%[Warning: Draw object ignored]
\textit{{}-tuk  {}-arɨr  {}-tur  {}-jawur} %%[Warning: Draw object ignored]
\textit{{}-an  {}-təər  {}-tar  {}-nən} (SEQ)

    CAUS  PASS  PRPR  CAP  PROG  POL  NEG  RSL  PST  

          \textit{{}-jur} 

          UMRK    

\textit{{}-tɨ} (SEQ) can directly follow the verbal root. Basically, it is used in affirmative as in (8-102 a-b). On the contrary, \textit{{}-nən} (SEQ) is always preceded by \textit{{}-an} (NEG), i.e., always used in negative as in (8-102 c-d).

(8\nobreakdash-102)  \textit{{}-tɨ} (SEQ)

  a.  TM:  cjuuto  ikinnja  {\textbar}zitensja{\textbar}  hankəəracjɨ,

      \textit{cjuuto}  \textit{ik-i=n=ja}  \textit{zitensja}  \textit{hankəər-as-tɨ}

      middle  go-INF=DAT1=TOP  bicycle  tumble-CAUS-SEQ

      kˀugəəracjɨ,  baramukasjanwake.

      \textit{kˀugəər-as-tɨ}  \textit{baramukasɨr-tar-n=wake}

      tumble-CAUS-SEQ  scatter-PST-PTCP=CFP

      ‘When (the boy) went halfway, (he) tumbled off the bicycle (that he was riding on), and scattered (the pears).’

      [PF: 090222\_00.txt]

  b.  TM:  idocjɨ  jˀicjɨ,  (an)  mata  (an)  agan

      \textit{ido=ccjɨ}  \textit{jˀ-tɨ}  \textit{a-n}  \textit{mata}  \textit{a-n}  \textit{aga-n}

      oh=QT  say-SEQ  DIST-ADNZ  again  DIST-ADNZ  DIST-ADVZ

      izjibatɨ  izjɨ,  amanan  sawakotankja

      \textit{izir-i+bar-tɨ}  \textit{ik-tɨ}  \textit{a-ma=nan}  \textit{sawako-taa=nkja}

      go.out-INF+?-SEQ  go-SEQ  DIST-place=LOC1  Sawako-PL=APPR

      minakotankjaga  wutattu,

      \textit{minako-taa=nkja=ga}  \textit{wur-tar-tu}

      Minako-PL=APPR=NOM  exist-PST-CSL

      ‘Saying that “Oh!” (I) went out there again, and there were Sawako, Minako and their friends, so ...’

      [Co: 101020\_01.txt]

  \textit{{}-nən} (SEQ)

  c.  TM:  jazin  {\textbar}hucuugo{\textbar}ja  cɨkawannən,  {\textbar}hoogen{\textbar}bəidujaa.

      \textit{jazin}  \textit{hucuugo=ja}  \textit{cɨkaw-an-nən}  \textit{hoogen=bəi=du=jaa}

      necessarily  standard.Japanese=TOP  use-NEG-SEQ  dialect=only=FOC=SOL

      nunkuin  wakappa.

      \textit{nuu-nkuin}  \textit{wakar-ba}

      what-INDFZ  understand-CSL

      ‘Necessarily not using the standard Japanese, (it is OK) only with (our) dialect. Since (the present author) can understand anything.’

      [Co: 110328\_00.txt]

  d.  TM:  {\textbar}sjoogakusjei{\textbar}nu  {\textbar}sjeito{\textbar}  cɨrɨtɨ,  {\textbar}hito ...  kurabu{\textbar}gadəə

      \textit{sjoogakusjei=nu}  \textit{sjeito}  \textit{cɨrɨr-tɨ}  \textit{hito+}  \textit{kurabu=gadɨ=ja}

      primary.schoolchild=GEN  pupil  accompany-SEQ  one  club=LMT=TOP

      arannən,  minna  cɨrɨtɨjo,

      \textit{ar-an-nən}  \textit{minna}  \textit{cɨrɨr-tɨ=joo}

      COP-NEG-SEQ  everybody  accompany-SEQ=CFM1

      ‘(A teacher) came with the primary school children, and (they) are not enough (to be able to form) a club, and (the teacher) came (to my mother’s house) with all (these children), and ...’

      [Co: 110328\_00.txt]

In principle, \textit{{}-tɨ} (SEQ) links clauses sequentially, which can usually be translated into ‘and.’ However, the combination of \textit{{}-tɨ} plus \textit{n} ‘even’ can mean ‘even if ...’ as in (8-103) (see \sectref{sec:key:10.1.3} for more details).

(8\nobreakdash-103)  TM:  abɨtɨn,  kikjanba.  jˀicjɨn,  kikjanba.

    \textit{abɨr-tɨ=n}  \textit{kik-an-ba}  \textit{jˀ-tɨ=n}  \textit{kik-an-ba}

    call-SEQ=even  hear-NEG-CSL  say-SEQ=even  hear-NEG-CSL

    ‘Even if (I) call (her), (she) doesn’t hear. Even if (I) says (something to her), (she) doesn’t hear, so (I don’t visit her these days).’

    [Co: 120415\_01.txt]

In principle, \textit{{}-tɨ} (SEQ) is used in the affirmative polality as in (8-102 a-b) and (8-103). However, \textit{{}-tɨ} (SEQ) can be used in negative in the following cases. (A) \textit{{}-tɨ} (SEQ) is followed by \textit{n} ‘even’ and means a conditional meaning such as ‘(there is no problem) even if not, ...’ (B) \textit{{}-tɨ} (SEQ) is used in insubordination.

First, I will present examples of (A).

(8\nobreakdash-104)  \textit{{}-an-tɨ=n} (NEG-SEQ=even) to mean ‘(there is no problem) even if not ...’

  a.  TM:  naa,  mutunu  kutunkjagadəə  sijantɨn,

      \textit{naa}  \textit{mutu=nu}  \textit{kutu=nkja=gadɨ=ja}  \textit{sij-an-tɨ=n}

      FIL  origin=GEN  event=APPR=LMT=TOP  know-NEG-SEQ=even

      jiccjaccjɨdu  juuba.

      \textit{jiccj-sa=ccjɨ=du}  \textit{jˀ} \textit{-ba}

      no.problem-ADJ=QT=FOC  say-CSL

      ‘(Younger people) say that, “(There) is no problem, even if (we) don’t know about the old events.”’

      [Co: 111113\_02.txt]

  b.  TM:  naa,  huccjunkjoo  minna  urəə

      \textit{naa}  \textit{huccju=nkja=ja}  \textit{minna}  \textit{u-rɨ=ja}

      FIL  old.people=APPR=TOP  everybody  MES-NLZ=TOP

      mjantɨn,  sicjutattujaa.  {\textbar}jonban{\textbar}gadɨ.

      \textit{mj-an-tɨ=n}  \textit{sij-tur-tar-tu=jaa}  \textit{jonban=gadɨ}

      see-NEG-SEQ=even  know-PROG-PST-CSL=SOL  number.four=LMT

      ‘Even if (they) didn’t see that [i.e. a pamphlet of songs], all of the old people knew [i.e. had memorized] (the songs from No. 1) to No. 4.’

      [Co: 120415\_01.txt]

Generally, the adjectival root \textit{jiccj-} can be translated as ‘good’ in English. After the combinations \textit{{}-an-tɨ=n} (NEG-SEQ=even), however, it is more appropriate to translate \textit{jiccj-} as ‘no problem’ as in (8-104 a). In fact, there is a case where \textit{jiccj-} can be translated as ‘no problem’ without following \textit{{}-an-tɨ=n} (NEG-SEQ=even) as in (9-45 d) in \sectref{sec:key:9.2.2.1.}

  Secondly, the verbal form \textit{{}-an-tɨ} (NEG-SEQ) can be used in the case of insubordination, i.e. the use of non-finite form in the main clause (see \sectref{sec:key:11.2}). In the interrogative clause, the finite-form affix \textit{{}-tar} (PST) cannot be used, and instead \textit{{}-tɨ} (SEQ) can be used to indicate the past tense, where \textit{{}-an} (NEG) can precede \textit{{}-tɨ} (SEQ) as in (8-105).

(8\nobreakdash-105)  \textit{{}-an-tɨ} (NEG-SEQ) in the insubordination

  TM:  naakjoo  ukka  sjantɨ  asɨbantɨ?

    \textit{naakja=ja}  \textit{u-rɨ=ga}  \textit{sja=nantɨ}  \textit{asɨb-an-tɨ}

    2.HON.SG=TOP  MES-NLZ=GEN  under=LOC1  play-NEG-SEQ

    ‘Didn’t you play under that [i.e. a big bayan tree]?’

    [Co: 110328\_00.txt]

The above example expresses the negative question in the past tense using \textit{{}-an-tɨ} (NEG-SEQ).

  There are examples where the converb \textit{{}-tɨ} (SEQ) behaves similarly with the nominal, which will be discussed in \sectref{sec:key:9.3.2.2.}

\subsection{Infinitive (verbal noun)}

An infinitive is a verbal form that ends with the infinitival affixes, i.e. \textit{{}-i} (INF) or \textit{{}-Ø} (INF). Infinitive cannot include the past tense affix \textit{{}-tar} and the negative affix \textit{{}-an} (NEG). The clause headed by an infinitive functions as a nominal, i.e. a nominal clause (see also \sectref{sec:key:11.1.3}). The morphophonology and the morhposyntax of the infinitives are fairly complicated. The morphophonology of the infinitives will be discussed in \sectref{sec:key:8.4.4.1.} The morphosyntax of the infinitives will be discussed in \sectref{sec:key:8.4.4.2.}

\subsubsection{Morphophonology of the infinitives}

First of all, the two types of forms of infinitives, i.e. simple forms and lengthened forms, are shown below.

\begin{styleBeschriftung}
\textmd{\tabref{tab:key:80}. Infinitives (simple forms and lengthened forms)}
\end{styleBeschriftung}

Stem No.  1. V\textsubscript{non-back}r  2. V\textsubscript{back}r, V\textsubscript{back}w\footnote{Phonological rule (see \sectref{sec:key:2.4.1}): w/r + i > i}

ex.  \textit{hingir-}  \textit{abɨr-}  \textit{kəər-}  \textit{ˀkuur-}  \textit{nugoor-}  \textit{koow-}\footnote{Phonological rule (see \sectref{sec:key:2.4.5}): kooi > koi}

  ‘escape’  ‘call’  ‘exchange’  ‘close’  ‘don’t do’  ‘buy’

Simple  hingi  abɨ  kəə  ˀkuu-i  nugoo-i  koo-i / ko-i

Lengthened  hingii  abɨɨ  kəə  ˀkuu-ii  nugoo-ii  koo-ii

Stem No.  2. V\textsubscript{back}r  3. pp  4. b  5. Vm  6. nm  7. V\textsubscript{non-}\textit{\textsubscript{i} }k

ex.  \textit{tur-}\footnote{Phonological rule (see \sectref{sec:key:2.4.1}): tur + i(i) > tui(i)}  \textit{app-}  \textit{narab-}  \textit{jum-}  \textit{tanm-}  \textit{kak-}

  ‘take’  ‘play’  ‘line up’  ‘read’  ‘ask’  ‘write’

Simple  tu-i  app-i  narab-i  jum / jum-i  tanm-i  kak-i

Lengthened  tu-ii  app-ii  narab-ii  jum / jum-ii  tanm-ii  kak-ii

Stem No.  8. V\textsubscript{non-}\textit{\textsubscript{i} }kk  9. Vs  10. ss  11. t  12. Only C(G)

ex.  \textit{sukk-}  \textit{us-}  \textit{kuss-}  \textit{ut-}  \textit{jˀ-}  \textit{mj-}

  ‘pull’  ‘push’  ‘kill’  ‘hit’  ‘say’  ‘see’

Simple  sukk-i  us-i  kuss-i  uc-i\footnote{Phonological rule (see \sectref{sec:key:2.4.2}): ut + i(i) > uci(i)}  jˀ-ii  m-ii

Lengthened  sukk-ii  us-ii  kuss-ii  uc-ii  jˀ-ii  m-ii

Stem No.  13. ij  14. V\textsubscript{non-}\textit{\textsubscript{i}} g  15. ik  16. i(n)g    17. in

ex.  \textit{kij-}  \textit{tug-}  \textit{kik-}  \textit{uig-}  \textit{ming-}  \textit{sin-}

  ‘cut’  ‘whet’  ‘hear’  ‘swim’  ‘grab’  ‘die’

Simple  ki-i  tug-i  kik-i  uig-i  ming-i  sin / sin-i

Lengthened  ki-i  tug-ii  kik-ii  uig-ii  ming-ii  N/A

The above table shows that the infinitives in Yuwan have two types of surface forms, i.e. the simple forms and the lengthened forms. Many of the simple forms have the single vowel /i/ at their final position, and many of the lengthened forms have the vowel sequence /ii/ at their final position. The lengthened forms can be used if the infinitive is a clause-final free form (not a clitic). Otherwise, the simple forms are used.

  First, we will discuss the simple forms. The morphophonological rules for the simple infinitival forms are summarized as in (8-106).

(8\nobreakdash-106)  The rules for the simple infinitival form;

  1.  The verbal stem No. 1 always takes \textit{{}-Ø} (INF);

  2.  If both (A) the verbal stem belongs to 5, or 17, and (B) there is no possibility to form /C.C./, then the verbal stem takes \textit{{}-Ø} (INF);

  3.  Otherwise, the verbal stems take \textit{-i} (INF);

  4.  //r// before \textit{{}-Ø} (INF) and //j// before \textit{{}-i} (INF) are deleted;

  5.  If the infinitive has only one mora in itself, its final vowel is lengthened.

This rule in (8-106 “4”) is required to explain the following behavior: \textit{kij-i} (cut-INF) + \textit{ja} (TOP) > /ki-i=ja/ (not */ki-jəə/), where the topic marker is never fused with the preceding morphophoneme (see also \sectref{sec:key:10.1.1.1}).

I will present examples of simple infinitival forms below. In the following tables, \textit{{}-Ø} (INF) is expressed even in the surface forms, and the infinitives are underlined.

\begin{styleBeschriftung}
\textmd{\tabref{tab:key:81}}\textmd{. Simple forms with}\textmd{ \textit{mai}}\textmd{ (OBL)}
\end{styleBeschriftung}

Stem No.  1  5  12  13  17  The others

Infinitival affix  \textit{{}-Ø  {}-Ø  {}-i  {}-i  {}-Ø  {}-i}

ex.  \textit{abɨr-} ‘call’  \textit{jum-} ‘read’  \textit{mj-} ‘see’  \textit{kij-} ‘cut’  \textit{sin-}‘die’  \textit{kak-} ‘write’

(Input)  abɨr-Ø+mai  jum-Ø+mai  mj-i+mai  kij-i+mai  sin-Ø+mai  kak-i+mai

Deletion of //r// or //j//  abɨ-Ø+mai  {}-  m-i+mai  ki-i+mai  {}-  {}-

Lengthening  {}-  {}-  m-ii+mai  {}-  {}-  {}-

(Output)  abɨ-Ø+mai  jum-Ø+mai  m-ii+mai  ki-i+mai  sin-Ø+mai  kak-i+mai

\textit{mai} (OBL) in \tabref{tab:key:81} does not have a possibility to form a /C.C./ (not /C.C/) syllable structure. However, \textit{n} ‘also’ in \tabref{tab:key:82} has the possibility to form a /C.C./ syllable structure with \textit{jum-} (the verbal stem No. 5) and \textit{sin-} ‘die’ (the verbal stem No. 17). Therefore, these verbal stems take \textit{-i} (INF) as in \tabref{tab:key:82} (not \textit{{}-Ø} (INF) as in \tabref{tab:key:81}).

\begin{styleBeschriftung}
\textmd{\tabref{tab:key:82}}\textmd{. Simple forms with} \textmd{\textit{n} }\textmd{‘also’}
\end{styleBeschriftung}

Stem No.  1  5  12  13  17  The others

Infinitival affix  \textit{{}-Ø  {}-i  {}-i  {}-i  {}-i  {}-i}

ex.  \textit{abɨr-} ‘call’  \textit{jum-} ‘read’  \textit{mj-} ‘see’  \textit{kij-} ‘cut’  \textit{sin-}‘die’  \textit{kak-} ‘write’

(Input)  abɨr-Ø=n  jum-i=n  mj-i=n  kij-i=n  sin-i=n  kak-i=n

Deletion of //r// or //j//  abɨ-Ø=n  {}-  m-i=n  ki-i=n  {}-  {}-

Lengthening  {}-  {}-  m-ii=n  {}-  {}-  {}-

(Output)  abɨ-Ø=n  jum-i=n  m-i=n\footnote{Phonological rule (\sectref{sec:key:2.4.5}): ii + n > in}  k-i=n\footnote{Phonological rule (\sectref{sec:key:2.4.5}): ii + n > in}  sin-i=n  kak-i=n

\tabref{tab:key:82} is different from \tabref{tab:key:81} in that the verbal stems No. 5 and 17 take \textit{{}-i} (INF) in order to avoid */jum.n./ \textit{jum=n} (read=also) or */sin.n./ \textit{sin=n} (die=also).

  Next, we will discuss the lengthened forms. The rules for the lengthened infinitival forms are summarized as in (8-107).

(8\nobreakdash-107)  The rules for the lengthened infinitival form;

  1.  The verbal stem No. 1 takes \textit{{}-Ø} (INF) and the other stems take \textit{{}-i} (INF);

  2.  //r// before \textit{{}-Ø} (INF) and //j// before \textit{{}-i} (INF) are deleted;

  3.  If the infinitive has only one vowel at its final syllable, the vowel is lengthened.

I will present the lengthened infinitival forms in \tabref{tab:key:83}.

\begin{styleBeschriftung}
\textmd{\tabref{tab:key:83}}\textmd{. Lengthened forms}
\end{styleBeschriftung}

Stem No.  1  5  12  13  The others

Infinitival affix  \textit{{}-Ø  {}-i  {}-i  {}-i  {}-i}

ex.  \textit{abɨr-} ‘call’  \textit{jum-} ‘read’  \textit{mj-} ‘see’  \textit{kij-} ‘cut’  \textit{kak-} ‘write’

(Input)  abɨr-Ø  jum-i  mj-i  kij-i  kak-i

Deletion of //r// or //j//  abɨ-Ø  {}-  m-i  ki-i  {}-

Lengthening  abɨɨ-Ø  jum-ii  m-ii  {}-  kak-ii

(Output)  abɨɨ-Ø  jum-ii  m-ii  ki-i  kak-ii

It was difficult to find the appropriate questions to let the speaker say the lengthened form of the verbal stem No. 17. Thus, \tabref{tab:key:83} excludes the example of No. 17. 

As mentioned before, the lengthened forms are frequently used if the infinitive is a free form (not a clitic) that fills the clause-final position as in (8-108 a-b). If the infinitive is followed by another free form, the infinitive does not become a lengthened form, but it becomes a simple form as in (8-108 c).

(8\nobreakdash-108)  Lengthened form and simple form

  a.  Followed by \textit{doo} (ASS)

    TM:  minnasjɨ  abɨɨdoo.

      \textit{minna=sjɨ}  \textit{abɨ-Ø=doo}

      everybody=INST  call-INF=ASS

      ‘(We) call (him) together.’

      [El: 130814]

  \textbf{b.}  Followed by nothing

    TM:  namaara  abɨɨ?  

      \textit{nama=kara}  \textit{abɨ{}-Ø} 

      now=ABL  call-INF  

      ‘Do (you) call (her) now?’

      [El: 110917]

  \textbf{c.}  Followed by \textit{jar-} (COP)

    TM:  minnasjɨ  abɨ  jataroo.

      \textit{minna=sjɨ}  \textit{abɨ-Ø}  \textit{jar-tar-oo}

      everybody=INST  call-INF  COP-PST-SUPP

      ‘Probably (they) called (him) together.’

      [El: 130814]

In (8-108 a-b), the infinitive \textit{abɨ-Ø} (call-INF) is a clause-final free form. Thus, it takes the lengthened form /abɨɨ/. In (8-108 c), the infinitive \textit{abɨ-Ø} (call-INF) is not the clause-final free form, but the copular verb /jataroo/ \textit{jar-tar-oo} (COP-PST-SUPP) is the clause-final free form. Therefore, the infinitive takes the simple form (not the lengthened form), i.e. /abɨ/. Usually, the infinitive takes the lengthened form if it is a clause-final free form as in (8-108 a-b). In fact, there is a case where the infinitive that is a clause-final free form does not take the lengthened form as in (8-114 a) in \sectref{sec:key:8.4.4.2.}

  In addition, \textit{doo} (ASS) permits the verbal stem No. 5 (ending with //Vm//) to become not only the lengthened form, e.g. /jum-ii=doo/ (read-INF=ASS), but also the simple form, e.g. /jum-Ø=doo/ (read-INF=ASS), even in the clause-final position. This alternation is not permitted before \textit{na} (PLQ), e.g. */jum-Ø=na/ (read-INF=PLQ), where the verbal stem No. 5 always takes the lengthened form, e.g. /jum-ii=na/ (read-INF=PLQ) ‘Does (someone) read?’ It is probable that this restriction avoids the confusion between \textit{na} (PLQ) and \textit{{}-na} (PROH), since the latter can form /jum-na/ (read-PROH) ‘Don’t read!’

  Before concluding this section, it should be mentioned that the difference between the simple form and the lengthened form of infinitives may indicates an intonational unit. In other words, an infinitive would be lengthened if it is in the final position of the intonational unit. In that case, the clause-final particles, e.g. \textit{doo} (ASS), seem to attach to the intonational unit. This analysis is in need of further resarch.

\subsubsection{Morphosyntax of the infinitives}

In this section, we will discuss the morphology and syntax of the infinitives. We will begin with the morphology. The verbal morphemes that can directly precede the infinitival affix \textit{{}-i}/\textit{{}-Ø} are shown in (8-109).

(8\nobreakdash-109)  Verbal morphemes that can directly precede \textit{{}-i}/\textit{{}-Ø} (INF) (Infinitival affix; Group I)

  Root  \textit{{}-as  {}-arɨr  {}-tuk  {}-arɨr  {}-tur  {}-jawur  {}-an  {}-təər  {}-tar  {}-i}/\textit{{}-Ø} (INF)

    CAUS  PASS  PRPR  CAP  PROG  POL  NEG  RSL  PST  

The above example shows that the verbal root can also directly precede \textit{{}-ɨ}/\textit{{}-Ø} (INF). The affixes that can directly precede the infinitival affix, i.e. \textit{{}-as} (CAUS), \textit{{}-arɨr} (PASS), \textit{{}-tuk} (PRPR), \textit{{}-arɨr} (CAP), and \textit{{}-tur} (PROG), belong to derivational affixes (see \sectref{sec:key:8.1}).

  The infinitives can appear only by themselves, or appear in the compounding. The infinitive that appears in th non-final position in the comopound takes the simple form discussd in \sectref{sec:key:8.4.4.1.} The examples of compounding were already presented in \sectref{sec:key:4.2.3.1} and \sectref{sec:key:4.2.3.2.} We will discuss the infinitives that fill the word-final position below.

  Syntactically, the infinitives in the word-final position can appear in the following syntactic environments in the clause.

(8\nobreakdash-110)  The infinitives in the word-final postion can appear

  a. In the complement slot of the light verb \textit{sɨr-} ‘do’;

  b. As the core argument of the nominal predicate;

  c. In the predicate slot in the main clause;

  d. Before \textit{n} (DAT1) meaning ‘when.’

The lengthened form may appear only in the case of (8-110 c). The infinitives of (8-110 a-c) cannot take their own subjects. In other words, in those cases, the subjects of infinitives always coincide with those of the main clauses. The stative verb \textit{ar-} can be followed by \textit{{}-i} (INF) in the conditions of (8-110 a, d) as in the examples (8-111 c) and (8-115 f). However, the copula verb cannot take the infinitival affix.

With regard to (8-110 a), the infinitive can appear in the complement slot of the VP, where the lexical verb is always \textit{sɨr-} ‘do’ as in (8-111 a-c). The infinitives take simple forms in this environment.

(8\nobreakdash-111)  In the complement slot of the light verb \textit{sɨr-} ‘do’

  a.  TM:  zjenzjen  munun  janbajoo,  kikin  sɨran.

      \textit{zjenzjen}  \textit{mun=n}  \textit{jˀ-an-ba=joo}  \textit{kik-i=n}  \textit{sɨr{}-an}

      at.all  thing=also  say-NEG-CSL=CFM1  [ask-INF=even]  [do-NEG]

            \{[Complement]  [LV]\}\textsubscript{VP}

      ‘(He) does not say anything, so (I) do not ask him (either).’

      [Co: 120415\_01.txt]

  \textbf{b.}  TM:  wanun  tanmidu  sjan.  {\textbar}oiwai{\textbar}kkwa

      \textit{wan=n}  \textit{tanm-i=du}  \textit{sɨr{}-tar-n  oiwai-kkwa}

      1SG=also  [ask-INF=FOC]  [do-PST-PTCP]  monetary.gift-DIM

        \{[Complement]  [LV]\}\textsubscript{VP}  

      ‘I also asked (them). (To prepare) the monetary gift (on behalf of TM).’

      [Co: 110328\_00.txt]

  \textbf{c.}  TM:  makanəicjasoo  aija  sjunban,

      \textit{makanaw-i+cja-soo}  \textit{ar-i=ja}  \textit{sɨr{}-jur-n=ban}

      [give.a.feast-ING+want-ADJ  STV-INF=TOP]  [do-UMRK-PTCP]=ADVRS

      \{[Complement]    [LV]\}\textsubscript{VP}

      ‘(I) want to give a feast (to the present author), but ...’

      [Co: 101023\_01.txt]

The above examples show that the infinitives fill the complement slots of the VPs composed of the light verb \textit{sɨr-} ‘do.’ These structures are called the light verb construction, and details will be disscussed in \sectref{sec:key:9.1.2.}

  With regard to (8-110 b), the infinitive can become the core argument of the nominal predicate as in (8-112 a-c) (see \sectref{sec:key:9.3} for more details on nominal predicate). The infinitives take simple forms in this environment.

(8\nobreakdash-112)  As the core argument of the nominal predicate

  a.  TM:  waakjaa  anmaaja  gan  sjɨ  uta  jusɨrooccjɨ,

      \textit{waakja-a}  \textit{anmaa=ja}  \textit{ga-n}  \textit{sɨr-tɨ}  \textit{uta}  \textit{jusɨr-oo=ccjɨ}

      1PL-ADNZ  mother=TOP  MES-ADVZ  do-SEQ  song  teach-INT=QT

      jusɨga  sɨki  jatanmundoo.

      \textit{jusɨr-Ø=ga}  \textit{sɨki}  \textit{jar-tar-n=mun=doo}

      [teach-INF]=NOM  [favorite  COP-PST-PTCP]=ADVRS=ASS

      [Core argument]  [Nominal predicate]

      ‘My mother (thought) that (she) tried to teach (me) the (traditional) songs in this way, and (she) liked teaching [lit. About her, teaching was a favorite (thing)].’

      [Co: 111113\_01.txt]

  b.  TM:  heisjeikaci  kawaija  {\textbar}sjoowanannen{\textbar}gadɨ?

      \textit{heisjei=kaci}  \textit{kawar-i=ja}  \textit{sjoowa+nan+nen=gadɨ}

      [Heisei=ALL  change-INF]=TOP  [Showa+what+year]=LMT

      [Core argument]  [Nominal predicate]

      ‘When did Showa [Japanese era, 1926-1989] change to Heisei [Japanese era, 1989 to present]?’ [lit. ‘The change into Heisei is until what year of Showa?’]

      [Co: 110328\_00.txt]

  c.  TM:  cˀjun  simacˀjutu  hanasiga

      \textit{cˀju=nu}  \textit{sima+cˀju=tu}  \textit{hanas-i=ga}

      [person=GEN  community+person=COM  talk-INF]=NOM

      [Core argument]  

      sɨkiccjɨjo.

      \textit{sɨki=ccjɨ=joo}

      [favorite]=QT=CFM1

      [Nominal predicate]

      ‘(The person) likes talking with a person from another community.’ [lit. ‘(About the person) talking with a person of (another) person’s community is favorite.’]

      [Co: 120415\_01.txt]

It should be noted that the infinitive /kawai/ \textit{kawar-i} (change-INF) ‘changing’ in (8-112 b) retains its own argument \textit{heisjei=kaci} (Heisei=ALL) ‘to Heisei.’ Similarly, the infinitive /hanasi/ \textit{hanas-i} (talk-INF) ‘talking’ in (8-112 c) retains its own argument \textit{cˀju=nu} \textit{sima+cˀju=tu} (person=GEN community+person=COM) ‘with a person from another community.’

  With regard to (8-110 c), the infinitive can be used in the predicate slot in the main clause. The infinitives take either simple forms or lengthened forms in this environment (see \sectref{sec:key:8.4.4.1} for more details). The infinitive in the predicate slot may be followed by a copula verb as in (8-113 a-c). That is, it forms a nominal predicate phrase.

(8\nobreakdash-113)  In the predicate slot in the main clause

  a.  [Context: Remembering the days when people sent off the people who went to mainland Japan]

    TM:  umanan  sanbasinu  atɨ,

      \textit{u-ma=nan}  \textit{sanbasi=nu}  \textit{ar-tɨ}

      MES-place=LOC1  pier=NOM  exist-SEQ

      umantɨ  cɨkɨ  jatattu.

      \textit{u-ma=nantɨ}  \textit{cɨkɨr-Ø}  \textit{jar-tar-tu}

      [MES-place=LOC2  attach-INF  COP-PST-CSL]

      [Nominal predicate]

      ‘There is a pier there, and (the ship) came alongside there [lit. (the ship) was to dock there].’

      [Co: 120415\_00.txt]

  b.  TM:  {\textbar}heitai{\textbar}kaci  xxx  turarɨ  jappoo,  nusjee

      \textit{heitai=kaci}    \textit{tur-arɨr-Ø}  \textit{jar-boo}  \textit{nusi=ja}

      [soldier=ALL    take-PASS-INF  COP-CND]  RFL=TOP

      [Nominal predicate]  

      {\textbar}konoehei{\textbar}ccjɨ  jˀicjɨ,  

      \textit{konoe+hei=ccjɨ}  \textit{jˀ-tɨ}  

      imperial.guard+soldier  say-SEQ  

      ‘(He said) that, “if (I) am called up to the military [lit. if (I) am taken to the military], (I) myself (will be) an imperal guard,” and ...’

      [Co: 120415\_00.txt]

  c.  TM:  ukkaci  makikum  jatattujaa.

      \textit{u-rɨ=kaci}  \textit{mak-i+kum{}-Ø  jar-tar-tu=jaa}

      [MES-NLZ=ALL  roll-INF+into-INF  COP-PST-CSL=SOL]

      [Nominal predicate]

      ‘(The old-type audio recorder) rolled up (the tape of one side) into that [i.e. the other side] (during the recording).’

      [Co: 120415\_01.txt]

  d.  TM:  an  junɨzooanjootaaga  simautaba

      \textit{a-n}  \textit{junɨzoo+anjoo-taa=ga}  \textit{sima+uta=ba}

      DIST-ADNZ  Yonezo+older.brother-PL=NOM  [community+song=ACC

      [Subject]  [Nominal predicate]

      {\textbar}hozon{\textbar}  siicjɨ  jˀicjɨ,

      \textit{hozon}  \textit{sɨr-i=ccjɨ}  \textit{jˀ-tɨ}

      preservation  do-INF]=QT  say-SEQ

      ‘Those (people,) Yonezo and his family said that (they would) do the preservation of the (traditional) songs (of) the community.’

      [Co: 111113\_01.txt]

In (8-113 a-d), the infinitives fill the predicate slot as nominals, which is clear from the copula verbs following them, although there is no copula in (8-113 d). The infinitives in (8-113 a-d) retain their “internal syntax” \citep{Haspelmath1996} such as \textit{u-ma=nantɨ} (MES-place=LOC2) in (8-113 a), \textit{heitai=kaci} (soldier=ALL) in (8-113 b), /ukkaci/ \textit{u-rɨ=kaci} (MES-NLZ=ALL) in (8-113 c), and \textit{sima+uta=ba} (community+song=ACC) in (8-113 d). However, infinitives in these environments cannot have its own subject, which is attested by the following examples.

(8\nobreakdash-114)  a. \textit{mizjuu} ‘ditch’ being the subject of the nominal predicate [= (6-120 b)]

  TM:  kun  {\textbar}ike{\textbar}karanu  mizjuuga  agan  iki.

    \textit{ku-n}  \textit{ike=kara=nu}  \textit{mizjuu=ga}  \textit{aga-n}  \textit{ik-i}

    [PROX-ADNZ  pond=ABL=GEN  ditch]=NOM  DIST-ADVZ  [go-INF]

    [Subject]    [Nominal predicate]

    ‘The ditch from this pond extends there.’ [lit. ‘The ditch from this ponds (is) to go there.’]

    [Co: 120415\_00.txt]

  b. \textit{mizjuu} ‘ditch’ being the subject of the verbal predicate

  TM:  mizjuunu  atattoo.      

    \textit{mizjuu=nu}  \textit{ar-tar=doo}      

    ditch=NOM  exist-PST=ASS      

    ‘There was a ditch.’

    [Co: 120415\_00.txt]

The nominative particle has two forms \textit{ga} and \textit{nu}. The former \textit{ga} (NOM) is used when the preceding nominal belongs to the higher position in the animacy hierarchy, and the latter \textit{nu} (NOM) is used when the preceding nominal belongs to the lower position in the animacy hierarchy (see \sectref{sec:key:6.4} for more details). Therefore, \textit{mizjuu} ‘ditch’ normally takes \textit{nu} (NOM) as in (8-114 b), since it indicates an inanimate referent, which is in the lowest position in the animacy hierarchy. However, if the predicate is a filled by an NP, i.e. a nominal predicate, the subject always takes \textit{ga} (NOM) irrespective the animacy of the preceding nominal (see \sectref{sec:key:6.4.3.2} for more details). Thus, \textit{mizjuu} ‘ditch’ in (8-114 a) takes the nominative particle \textit{ga} (not \textit{nu}). This means the infinitive \textit{ik-i} (go-INF) ‘going’ fills the predicate slot of the main clause, and the subject is \textit{mizjuu} ‘ditch.’ In other words, \textit{mizjuu} ‘ditch’ and \textit{ik-i} (go-INF) does not form a single (nominal) clause. Otherwise, the alleged nominal clause as a whole would fill the predicate of the main clause, where the subject of the \textit{ik-} ‘go’ has to take the nominative particle \textit{nu}, since the internal syntax of the alleged nominal clause does not require \textit{mizjuu} ‘ditch’ to take \textit{ga} (NOM). Considering the above examples, we can conclude that the infinitive as the nominal predicate in the main clause (or complement clause) is a verbal form that can retain its arguments with the exception of the subject. The infinitive followed by \textit{n} (DAT1), however, is not the case since it can retain the subject’s nominative \textit{nu} as in \textit{a-n} \textit{cˀju=nkja=nu} (DIST-ADNZ person=APPR=NOM) in (8-115 b) below.

  With regard to (8-110 d), if the infinitive is followed by \textit{n} (DAT1), it can indicate a certain temporal point as in (8-115 a-f). The infinitives take simple forms in this environment.

(8\nobreakdash-115)  Before \textit{n} (DAT1) indicating a temporal point

  a.  TM:  usatoobasanga  wuinnja  muru  iccja

      \textit{usato+obasan=ga}  \textit{wur-i=n=ja}  \textit{muru}  \textit{iccj-a}

      Usato+old.woman=NOM  exist-INF=DAT1=TOP  very  good-ADJ

      atanmuncjɨjo.

      \textit{ar-tar-n=mun=ccjɨ=joo}

      STV-PST-PTCP=ADVRS=QT=CFM1

      ‘When Usato was (with us) [i.e. was alive and healthy] it was very good.’

      [Co: 110328\_00.txt]

  b.  TM:  an  cˀjunkjanu  {\textbar}koocjoosjensjei{\textbar}

      \textit{a-n}  \textit{cˀju=nkja=nu}  \textit{koocjoo+sjensjei}

      DIST-PTCP  person=APPR=NOM  principal+teacher

      sjuinga,  amurontɨ  singa,

      \textit{sɨr-tur-i=n=ga}  \textit{amuro=nantɨ}  \textit{sɨr-i=n=ga}

      do-PROG-INF=DAT1=NOM  Amuro=LOC1  do-INF=DAT1=NOM

      amuronu  kˀwasainu  sjan  tukidarooga.

      \textit{amuro=nu}  \textit{kˀwasai=nu}  \textit{sɨr-tar-n}  \textit{tuki=daroo=ga}

      Amuro=NOM  fire=NOM  do-PST-PTCP  time=SUPP=CFM3

      ‘Probably, the time (when) that person was doing the principal (of the elementry school), the time (when he) did (it) at Amuro, is the time when Amuro caught fire.’

      [Co: 110328\_00.txt]

  c.  [Context: Speaking to US, whose family used to deal in fish] = (6-99 b)

    TM:  naakjaga  sjɨ  mooinnja,  simanu

      \textit{naa-kja=ga}  \textit{sɨr-tɨ}  \textit{moor-i=n=ja}  \textit{sima=nu}

      2.HON-PL=NOM  do-SEQ  HON-INF=DAT1=TOP  island=GEN

      jˀudarooga?

      \textit{jˀu=daroo=ga}

      fish=SUPP=CFM3

      ‘When you dealt in (fish), (they were) probably fish from the community [i.e. fish taken around the community].’

      [Co: 110328\_00.txt]

  d.  [= (6-56)]

    TM:  amanan  wuinkara,  naa  naikwa  kawatɨ,

      \textit{a-ma=nan}  \textit{wur-i=n=kara}  \textit{naa}  \textit{naikwa}  \textit{kawar-tɨ}

      DIST-place=LOC1  exist-INF=DAT1=ABL  already  a.little  strange-SEQ

      ‘(The person) was already strange since [lit. from when] (the person) was there, and ...’

      [Co: 120415\_01.txt]

  e.  TM:  uraga  amaaci  ikinnja,

      \textit{ura=ga}  \textit{a-ma=kaci}  \textit{ik-i=n=ja}

      2.NHON.SG=NOM  DIST-place=ALL  go-INF=DAT1=TOP

      wanna  kumaaci  ikjoojəə.

      \textit{wan=ja}  \textit{ku-ma=kaci}  \textit{ik-oo=jəə}

      1SG=TOP  PROX-place=ALL  go-INT=CFM2

      ‘When you go to that way, I will go to this way.’

      [El: 130814]

  f.  TM:  waasainkara  {\textbar}sjoku{\textbar}ja  nəncjɨjo.

      \textit{waa-sa+ar-i=n=kara}  \textit{sjoku=ja}  \textit{nə-an=ccjɨ=joo}

      young-ADJ+STV-INF=DAT1=ABL  appetite=TOP  exist-NEG=QT=CFM1

      ‘(I) do not eat much since (I) am young.’

[lit. ‘There is not appetite from when (I) am young.’]

      [Co: 120415\_01.txt]

In (8-115), the infinitival affix \textit{{}-i} directly follows the verbal roots, e.g. \textit{sɨr-} ‘do’ in (8-115 b) or \textit{ik-} ‘go’ in (8-115 e). In addition, \textit{{}-i} (INF) can follow the derivational affix \textit{{}-tur} (PROG) as in (8-115 b). On the one hand, an infinitive may be followed by \textit{n=kara} (DAT1=ABL) as in (8-115 d, f). On the one hand, a common noun cannot be followed by \textit{n=kara} (DAT1=ABL), e.g. *\textit{tuki=n=kara} (time=DAT1=ABL). These facts may imply that the \textit{n} (DAT1) after infinitives has been reanalyzed as a temporal marker with the infinitival affixes such as \textit{-}(\textit{i})\textit{n} ‘when.’

In all of the above examples, the predicate filled by the infinitive did not appear sequentially. However, there is an example where the clause-final infinitives are used sequentially (or in a clause chaining) as in (8-116). 

(8\nobreakdash-116)  Infinitives in a clause chaining

  [Context: After telling a short story, TM remembered the secret of good health told by the original story teller.]

  TM:  naa,  urɨga,  jˀiigajo,  hɨru  kamii,  gakkjuu

    \textit{naa}  \textit{u-rɨ=ga}  \textit{jˀ-i=ga=joo}  \textit{hɨru}  \textit{kam-i}  \textit{gakkjuu}

    FIL  MES-NLZ=NOM  say-INF=NOM=CFM1  garlic  eat-INF  shallot

    kamii,  {\textbar}zjagaimo{\textbar}  kamii,  hansɨ  kamii,  koosjaa  kamii,

    \textit{kam-i}  \textit{zjagaimo}  \textit{kam-i}  \textit{hansɨ}  \textit{kam-i}  \textit{koosjaa}  \textit{kam-i}

    eat-INF  potato  eat-INF  sweet.potato  eat-INF  yam  eat-INF

    unuu  kamiicjɨnkja  umujuncjɨjo.

    \textit{unuu}  \textit{kam-i=cjɨ=nkja}  \textit{umuw-jur-n=ccjɨ=joo}

    taro  eat-INF=QT=APPR  think-UMRK-PTCP=QT=CFM1

    ‘That (person) said that (he) thought that eating garlic, shallot, potato, sweet potato, yam, and taro (is good for his health).’

    [Fo: 090307\_00.txt]

The above example shows that clause-final infinitives may be used in clause chaining. However, this kind of sequential use of infinitives has been found only in (8-116) in my texts.

Before concluding this section, I want to mention two affixes that have the same form and can appear in the predicate slot of the main clause, i.e. \textit{{}-i} (INF) and \textit{{}-i} (NPST). As discussed in \sectref{sec:key:8.1}, the non-past affix \textit{{}-i} (Group-II affix) cannot directly follow any verbal root, e.g. *\textit{jum-i} (read-NPST). However, the same form \textit{jum-i} (read-INF) can appear in the sentence-final position. So far, we have regarded this as the infinitival affix \textit{{}-i} (not the non-past affix \textit{{}-i}). This analysis is supported by the following facts that the non-past affx \textit{{}-i} assimilates to the questional particle \textit{na} as in (8-117 a) (see \sectref{sec:key:10.3.2} for more details), but the infinitival affix \textit{{}-i} does not as in (8-117 b).

(8\nobreakdash-117)  a.  \textit{{}-i} (NPST)

    TM:  namaara  hon  jumjunnja?

      \textit{nama=kara}  \textit{hon}  \textit{jum-jur-i=na}

      now=ABL  book  read-UMRK-NPST=PLQ

      ‘Do you read a book from now?’

      [El: 130814]

  b.  \textit{{}-i} (INF)

    TM:  namaara  hon  jumiina?

      \textit{nama=kara}  \textit{hon}  \textit{jum-i=na}

      now=ABL  book  read-INF=PLQ

      ‘Do you read a book from now?’

      [El: 110914]

In (8-117 a), \textit{na} (PLQ) is palatalized by \textit{{}-i} (NPST) and also \textit{{}-i} (NPST) is nasalized by \textit{na} (PLQ): //-i=na// > (palatalization) > /-i=nja/ > (nasalization) > /-n=nja/. If the \textit{{}-i} in (8-117 b) is the non-past affix \textit{{}-i}, the same rules have to be applied, and the results would be a form like /jumunnja/: //jum-i=na// > (palatalization) > /jum-i=nja/ > (nasalization) > /jum-n=nja/ > (vowel insertion) > /jum-un=nja/ (about the alleged vowel insertion, see \sectref{sec:key:2.4.3}). However, \textit{{}-i} (INF) is lengthened before \textit{na} (PLQ) forming /iina/ (see \sectref{sec:key:8.4.4.1} for more details about the lengthened infinitive). Thus, we assume that \textit{{}-i} (INF) in (8-117 b) is different from \textit{{}-i} (NPST).

\subsection{Affix that seems to be across word classes}

The participial affix \textit{{}-n} and the adnominalizer \textit{{}-n} have the same form as in (8-118 a-b).

(8\nobreakdash-118)  a. The participial affix \textit{{}-n}

  TM:   hinzjaa  succjun  nɨsəənu  tuutai,

    [\textit{hinzjaa}  \textit{sukk-tur-n}]\textsubscript{Adnominal clause}  \textit{nɨsəə=nu}  \textit{tuur-tai}

    goat  pull-PROG-PTCP  young.man=NOM  pass-LST

    ‘A young man who was pulling a goat passed (there), and ...’

  [PF: 090305\_01.txt]

  b. The adnominalizer \textit{{}-n}

  [Context: TM and MY were asked to talk alone, so they felt difficulty to find a topic to talk of.]

  TM:   kjuuja  an  nɨsəənu  mjanba,  jakkəə.

    \textit{kjuu=ja}  [\textit{a-n}]\textsubscript{Adnominal (word)}  \textit{nɨsəə=nu}  \textit{mj-an-ba}  \textit{jakkəə}

    today=TOP  DSIT-ADNZ  young.man=NOM  see-NEG-CSL  trouble

    ‘Today that young man [i.e. the present author] does not see (us), so (we are in) trouble.’

  [Co: 101023\_01.txt]

Both of the affixes have the adnominal function. In (8\nobreakdash-118a), /succjun/ \textit{sukk-tur-n} (pull-PROG-PTCP) ‘pulling’ (and its object \textit{hinzjaa} ‘goat’ in the same clause) modifies the following nominal \textit{nɨsəə} ‘young man.’ In (8\nobreakdash-118b), \textit{a-n} (DIST-ADNZ) ‘that (one)’ also modifies the following nominal \textit{nɨsəə} ‘young man.’ Thus, one might think these two affixes are the same single affix. However, I do not take the analysis, because of the difference of the root classes that precede \textit{{}-n} (PTCP) and \textit{-n} (ADNZ).

The root \textit{sukk-} ‘pull’ can take an aspectual affix \textit{{}-tur} (PROG) as in (8\nobreakdash-118a) and a temporal affix \textit{{}-tar} (PST) such as /succja/ \textit{sukk-tar} (pull-PST). On the contrary, \textit{a-} (DIST) cannot take those affixes such as */atun/ \textit{a-tur-n} (DIST-PROG-PTCP) or */ata/ \textit{a-tar} (DIST-PST). Thus, the former root \textit{sukk-} ‘pull’ is morphologically different from tha latter root \textit{a-} (DIST). Furthermore, \textit{a-} (DIST) contrasts with \textit{ku-} (PROX) and \textit{u-} (MES) in deictic function (see \sectref{sec:key:5.2.1}). In this grammar, the root class such as \textit{sukk-} ‘pull’ is called the verbal root (see \sectref{sec:key:8.1}), and the root class such as \textit{a-} (DIST) is called the demonstrative root (see \sectref{sec:key:5.2}). Moreover, the root such as \textit{sukk-} ‘pull’ can take its own core (or peripheral) argument, e.g. \textit{hinzjaa} ‘goat’ as in (8\nobreakdash-118a). On the contrary, \textit{a-} (DIST) cannot take any argument. Thus, \textit{sukk-} ‘pull’ is also syntactically different from \textit{a-} (DIST). A word that includes a verbal root and that can take its own argument may be called the verb. A word that includes a demonstrative root may be called the demonstrative. Therefore, /succjun/ \textit{sukk-tur-n} (pull-PROG-PTCP) ‘pulling’ as in (8\nobreakdash-118 a) is a verb, and \textit{a-n} (DIST-ADNZ) ‘that (one)’ as in (8\nobreakdash-118 b) is a demonstrative.

  In conclusion, \textit{{}-n} (PTCP) in (8\nobreakdash-118 a) appears in the verb, but \textit{-n} (ADNZ) in (8\nobreakdash-118 b) does not appear in the verb. Thus, the former may be called the verbal affix, but the latter may not. That is, we do not regard them as the same affix (at least synchronically). The verbal affixes such as \textit{{}-n} (PTCP) are kinds of “word-class-changing” inflections (cf. \citealt{Haspelmath1996}).

\section{Derivational morphology}

In this section, I will present the derivational affixes (see \sectref{sec:key:8.5.1}) and the verbal compounding (see \sectref{sec:key:8.5.2}).

\subsection{Derivational affixes}

There are eight verbal derivational affixes in Yuwan: \textit{{}-as} (CAUS), \textit{{}-arɨr} (PASS), \textit{{}-tuk} (PRPR), \textit{-arɨr} (CAP), \textit{{}-tur} (PROG), \textit{{}-jawur} (POL), \textit{-jur} (UMRK) and \textit{{}-təər} (RSL). Additionally, two inflectional affixes can appear in the non-word-final position like derivational affixes, i.e. \textit{{}-an} (NEG) and \textit{{}-tar} (PST). The possible (and impossible) combinations of them were already shown in (8-1) and (8-2) in \sectref{sec:key:8.1.} It is worth noting that \textit{-tur} (PROG) and \textit{{}-təər} (RSL) originated from the auxiliary verb construction (“AVC”): \textit{{}-tur} (PROG) < \textit{*-tɨ} \textit{*wur-} (SEQ PROG); \textit{{}-təər} (RSL) < \textit{*-tɨ} \textit{*ar-} (SEQ RSL) (see \sectref{sec:key:9.1.1.1} for more details). It is probable that \textit{{}-tuk} (PRPR) originated from the AVC composed of *\textit{{}-tɨ} (SEQ) and *\textit{uk-} (PRPR) (< *\textit{uk-} ‘put’). However, there is no use of the \textit{uk-} ‘put’ as the auxiliary verb in modern Yuwan.

  The derivational affixes can be classified into the following categories.

\begin{styleBeschriftung}
\textmd{\tabref{tab:key:84}}\textmd{. Derivational affixes in Yuwan}
\end{styleBeschriftung}

Category  Form  Meaning

Valency-changing  \textit{{}-as} Causative

  \textit{{}-arɨr} Passive

  \textit{{}-arɨr} Capability

Aspect  \textit{{}-jur} Unmarked

  \textit{{}-tur} Progressive

  \textit{{}-təər} Resultative

Modality  \textit{{}-tuk} Preparative

  \textit{{}-jawur} Politeness

In the following subsections, I will present examples of the derivational affixes in \tabref{tab:key:84} in turn.

\subsubsection{\textit{{}-as} (CAUS)} 

\textit{{}-as} (CAUS) makes the agent (or experiencer) of the action indicated by the verbal root become the causee, which is marked by \textit{ba} (ACC) or \textit{n} (DAT1) in principle. The causee of the intransitive verb is likely to be marked by \textit{ba} (ACC), and that of the transitive verb is usually marked by \textit{n} (DAT1), but the latter may also be marked by \textit{kaci} (ALL). Additionally, \textit{{}-as} (CAUS) can introduce the causer, which is marked by the nominative case \textit{ga} (or \textit{nu}).

  First, I will present the example of the intransitive verb \textit{jam-} ‘have a pain.’

\textbf{(8\nobreakdash-119)}  Intransitive verbal root: \textit{jam-} ‘have a pain’

  a. Without \textit{-as} (CAUS)

  [Context: A boy fell off a bicycle on which a basketful of pears had been loaded .]

  TM:  jinganu  kˀwoo  nasi  (un)  baramacjattu,  naa,

    \textit{jinga=nu}  \textit{kˀwa=ja}  \textit{nasi}  \textit{u-n}  \textit{baramak-tar-tu}  \textit{naa}

    male=GEN  child=TOP  pear  MES-ADNZ  scatter-PST-CSL  FIL

    jukkadɨ  kan  sjɨ  sjutɨ,

    \textit{jukkadɨ}  \textit{ka-n}  \textit{sɨr-tɨ}  \textit{sɨr-jur-tɨ}

    continuously  PROX-ADVZ  do-SEQ  do-UMRK-SEQ

    jamjuncjɨ  jˀicjutɨ,

    \textit{jam{}-jur-n=ccjɨ  jˀ-tur-tɨ}

    have.a.pain-UMRK-PTCP=QT  say-PROG-SEQ

    ‘The boy scattered the pears, and was saying (he) was continuously in pain doing like this, and ...’

    [PF: 090827\_02.txt]

  b. With \textit{-as} (CAUS) [= (6-68)]

  [Context: Complaining about an acquaintance’s slander]

  TM:  wanga  kucisjɨ  nusiboo  jamacjuncjɨ.

    \textit{wan=ga}  \textit{kuci=sjɨ}  \textit{nusi=ba=ja}  \textit{jam-as{}-tur-n=ccjɨ}

    1SG=NOM  mouth=INST  RFL=ACC=TOP  have.a.pain-CAUS-PROG-PTCP=QT

    ‘(The person\textit{\textsubscript{} }said) that I was making the person ill using (my) mouth, and ...’

    [Co: 120415\_01.txt]

In (8-119 a), the experiencer (i.e. \textit{jinga=nu} \textit{kˀwa} ‘boy’) of the intransitive verb \textit{jam-} ‘have a pain’ is the subject of the clause. Thus, it does not take \textit{ba} (ACC). However, if \textit{jam-} ‘have a pain’ takes the causative affix \textit{{}-as}, the experiencer (i.e. \textit{nusi} (RFL), which is a participant different from the speaker TM) takes \textit{ba} (ACC) as a causee, and the causer (i.e. \textit{wan} ‘I,’ which is the speaker TM) takes \textit{ga} (NOM) as in (8-119 b).

  Secondly, I will present examples of the transitive verb \textit{koow-} ‘buy.’

\textbf{(8\nobreakdash-120)}  Transitive verbal root: \textit{koow-} ‘buy’

  a. Without \textit{-as} (CAUS)

  TM:  akiraga  {\textbar}hon{\textbar}  koojui

    \textit{akira=ga}  \textit{hon}  \textit{koow{}-jur-i}

    Akira=NOM  book  buy-UMRK-NPST

    ‘Akira buys a book.’

    [El: 111118]

  b. With \textit{-as} (CAUS)

  TM:  wanga  akiran  {\textbar}hon{\textbar}  koowasoojəə.

    \textit{wan=ga}  \textit{akira=n}  \textit{hon}  \textit{koow-as{}-oo=jəə}

    1SG=NOM  Akira=DAT1  book  buy-CAUS-INT=CFM2

    ‘I will have Akira buy a book.’

    [El: 111118]

In fact, there is no example where all of the causee, causer, and object of a transitive verb appear in the text data. That is not uncommon cross-linguistically \citep[79]{Dryer2007}. Thus, the example in (8-120 a) was taken in elicitation. In (8-120 a), the agent (i.e. \textit{akira} ‘Akira’) of the transitive verb \textit{koow-} ‘buy’ is the subject of the clause, and marked by \textit{ga} (NOM). However, if \textit{koow-} ‘buy’ takes the causative affix \textit{{}-as}, the agent (i.e. \textit{akira} ‘Akira’) takes \textit{ba} (ACC) as a causee, and the causer (i.e. \textit{wan} ‘I’) takes \textit{ga} (NOM) as in (8-120 b). Interestingly, the causee of the transitive verb may be marked by \textit{kaci} (ALL) as in (8-121), where the transitive verb is \textit{kak-} ‘write.’

\textbf{(8\nobreakdash-121)}  [= (6-82 b)]

  TM:  arɨn/arɨkaci/*arɨnkatɨ  kakasoojəə.

    \textit{a-rɨ=n}/\textit{a-rɨ=kaci}/\textit{a-rɨ=nkatɨ}  \textit{kak-as{}-oo=jəə}

    DIST-NLZ=DAT1/DIST-NLZ=ALL/DIST-NLZ=DAT2  write-CAUS-INT=CFM2

    ‘(I) will make that person write (it).’

    [El: 130820]

  As mentioned in \sectref{sec:key:6.3.2.2}, \textit{ba} (ACC) may be omitted. Thus, the causee of the transitive verbs may be marked by nothing as in (8-122 a-b).

(8\nobreakdash-122)  Causee of the transitive verbs being not marked

  a. Causee is an inanimate referent

  TM:  cjuuto  ikinnja  {\textbar}zitensja{\textbar}  hankəəracjɨ,

    \textit{cjuuto}  \textit{ik-i=n=ja}  \textit{zitensja}  \textit{hankəər-as{}-tɨ}

    middle  go-INF=DAT1=TOP  bicycle  tumble-CAUS-SEQ

    ‘When (the boy) went halfway, (he) tumbled off the bicycle (that he was riding on), and ...’

    [PF: 090222\_00.txt]

  b. Causee is a personal pronoun

  TM:  nan  umoorasanboocjɨ  umutɨ,

    \textit{nan}  \textit{umoor-as{}-an-boo=ccjɨ  umuw-tɨ}

    2.HON.SG  come.HON-CAUS-NEG-CND=QT  think-SEQ

    ‘(I) thought that (I) have to make you come, and ...’

    [Co: 110328\_00.txt]

In (8-122 a), the causee (i.e. \textit{zitensja} ‘bicycle’) of the verbal stem \textit{hankəər-as} (tumble-CAUS) ‘to have (something or someone) tumble’ does not take any case particle. Similarly, in (8-122 b), the causee (i.e. \textit{nan} ‘you’) of the verbal stem \textit{umoor-as} (come.HON-CAUS) ‘to have (someone) come’ does not take any case particle. Interestingly, when the head nominal is the personal pronoun, the alternation between \textit{ba} (ACC) and nothing is rarely found in the non-causative clauses (see \sectref{sec:key:6.3.2.2}). However, in the causative-clause as in (8-122 b), \textit{ba} (ACC) may alternate with nothing.

  The light verb \textit{sɨr-} ‘do’ has a causative counterpart, i.e. \textit{sɨmɨr-} (do.CAUS), which is composed of a single root, and it cannot be divided into more than one morpheme such as *\textit{sɨr-mɨr-} (do-CAUS), since one cannot say, e.g. */jummɨroo/ \textit{jum-mɨr-oo} (read-CAUS-INT) in any context.

(8\nobreakdash-123)  \textit{sɨmɨr-} (do.CAUS)

  a.  TM:  kurəə  kunuguru  (sadaega  sɨ)

      \textit{ku-rɨ=ja}  \textit{kunuguru}  \textit{sadae=ga}  \textit{sɨmɨr}

      PROX-NLZ=TOP  these.days  Sadae=NOM  do.CAUS

      sadaega  sɨmɨtəətɨ  zja.

      \textit{sadae=ga}  \textit{sɨmɨr{}-təər-tɨ  zjar}

      Sadae=NOM  do.CAUS-RSL-SEQ  COP

      ‘This one [i.e. a picture] is (what) Sadae has made (my son) do [i.e. enlarge the picture] these dasys.’

      [Co: 120415\_00.txt]

  b.  TM:  kurəə  akiran  sɨmɨroojəə.

      \textit{ku-rɨ=ja}  \textit{akira=n}  \textit{sɨmɨr{}-oo=jəə}

      PROX-NLZ=TOP  Akira=DAT1  do.CAUS-INT=CFM2

      ‘(I) will make Akira do this.’

      [El: 111116]

In (8-123 a), the causee (i.e. ‘my son’) is not expressed, and the causer (i.e. \textit{sadae} ‘Sadae’) is marked by \textit{ga} (NOM). In (8-123 b), the causee (i.e. \textit{akira} ‘Akira’) is marked by \textit{n} (DAT1), and the causer (i.e. ‘I’) is not expressed. It should be mentioned that \textit{sɨr-} ‘do’ can take \textit{-as} (CAUS) as in (8-124), although it does not appear in the text data.

(8\nobreakdash-124)  \textit{sɨr-} ‘do’ + \textit{{}-as} (CAUS)

  TM:  atoora  akiran  sɨrasoojəə.

    \textit{atu=kara}  \textit{akira=n}  \textit{sɨr-as{}-oo=jəə}

    after=ABL  Akira=DAT1  do-CAUS-INT=CFM2

    ‘(I) will make Akira do (it) later.’

    [El: 111116]

Furthermore, the lexical causative verb \textit{sɨmɨr-} (do.CAUS) can take the causative affix \textit{{}-as} (CAUS) redundantly. However, the combination of \textit{sɨmɨr-} (do.CAUS) and \textit{{}-as} (CAUS) introduces only one participant (not two participants) in the event of the causal chain as in (8-125 a-b).

(8\nobreakdash-125)  \textit{sɨmɨr-} (do.CAUS) + \textit{{}-as} (CAUS)

  a.  TM:  {\textbar}daibu{\textbar}  an  cˀjunkjannja  {\textbar}daibu  kuroo{\textbar}

      \textit{daibu}  \textit{a-n}  \textit{cˀju=nkja=n=ja}  \textit{daibu}  \textit{kuroo}

      many  DIST-ADNZ  person=APPR=DAT1=TOP  many  hardship

      sɨmɨrasatta.

      \textit{sɨmɨr-as-ar-ta}

      do.CAUS-CAUS-PASS-PST

      ‘(I) was made to undergo many hardships by that person.’

      [Co: 120415\_01.txt]

  b.  TM:  atoora  akiran  sɨmɨrasoojəə.

      \textit{atu=kara}  \textit{akira=n}  \textit{sɨmɨr-as-oo=jəə}

      after=ABL  Akira=DAT1  do.CAUS-CAUS-INT=CFM2

      ‘(I) will make Akira do (it) later.’

      [El: 111116]

In (8-125 a), the event expressed by the predicate includes only two participants, i.e. the causee (i.e. ‘I’), which is not expressed in the clause, and the causer (i.e. \textit{a-n} \textit{cˀju=nkja} ‘that person’). Similarly, in (8-125 b), the event expressed by the predicate \textit{sɨmɨr-as} (do.CAUS-CAUS) includes only two participants, i.e. the causee (i.e. \textit{akira} ‘Akira’) and the causer (i.e. ‘I’), although the causer is not overtly expressed in the clause. In other words, (8-125 b) has the same meaning with (8-124). The examples in (8-125 a-b) show that the double causative markings (both lexically and affixally) do not double the causal event itself. In other words, they do not mean ‘A causes B to make C do (something),’ but only mean ‘A causes B to do (something).’

\subsubsection{\textit{{}-ar(ɨr)} (PASS)}

\textit{{}-ar(ɨr)} (PASS) changes the syntactic valency of the transitive verb as in (8-126 a-b). The morphophonological alternation of \textit{{}-ar(ɨr)} (PASS) was discussed in \sectref{sec:key:8.2.1.5.} On the one hand, in (8-126 a), the non-passive verbal stem, i.e. \textit{sjug-i+agɨr-} (hit-INF+severely) ‘to hit severely,’ marks the agent with \textit{ga} (NOM) and the patient with \textit{ba} (ACC). On the other hand, in (8-126 b), the passive verbal stem, i.e. \textit{sjug-i+agɨr-ar} (hit-INF+ severely-PASS) ‘to be hit severely,’ marks the agent with \textit{n} (DAT1) and the patient with \textit{ga} (NOM). In fact, the agent in the passive clause can be marked only by \textit{n} (DAT1) (see also (6-82 a) in \sectref{sec:key:6.3.3.1}).

(8\nobreakdash-126)  a.  Without \textit{{}-ar(ɨr)} (PASS)

    TM:  akiraba  zjuuga  sjugjagɨtuddoo.

      \textit{akira=ba}  \textit{zjuu=ga}  \textit{sjug-i+agɨr{}-tur=doo}

      Akira=ACC  father=NOM  hit-INF+severely-PROG=ASS

      Patient  Agent  

      ‘(His) father is hitting Akira severely.’

      [El: 111116]

  b.  With \textit{{}-ar(ɨr)} (PASS)

    TM:  akiraga  zjun  sjugjagɨrattuddoo.

      \textit{akira=ga}  \textit{zjuu=n}  \textit{sjug-i+agɨr-ar{}-tur=doo}

      Akira=NOM  father=DAT1  hit-INF+severely-PASS-PROG=ASS

      Patient  Agent  

      ‘Akira is being hit severely by (his) father.’

      [El: 111116]

  The above example changes the case alignment of the arguments, but do not introduce another participant in the event expressed by the verbal root. However, there are examples that use \textit{{}-ar(ɨr)} (PASS) to introduce another participant as in (8-127 b).

(8\nobreakdash-127)  Malefactive use of \textit{{}-ar(ɨr)} (PASS) with the intransitive verb

  a.  Without \textit{{}-ar(ɨr)} (PASS)

    TM:  wanga  agan  ikjussaccjɨ

      \textit{wan=ga}  \textit{aga-n}  \textit{ik{}-jur-sa=ccjɨ}

      1SG=NOM  DIST-ADVZ  go-UMRK-POL=QT

      ‘(I said to the present author) that, “I will go there.”’

      [Co: 110328\_00.txt]

  b.  With \textit{{}-ar(ɨr)} (PASS)

    [Context: TM explained to MY why she had called her.] = (5-38 c)

    TM:  uran  daacika  ikjarɨncjɨga, ...

      \textit{ura=n}  \textit{daa=kaci=ka}  \textit{ik-ar(ɨr){}-n=ccjɨ=ga}

      2.NHON.SG=DAT1  where=ALL=DUB  go-PASS-PTCP=QT=FOC

      ‘(I thought) that (I) would suffer from your going somewhere, (so I called you.)’

      [Co: 101020\_01.txt]

In (8-127 a), the intransitive verb \textit{ik-} ‘go’ has a single participant (i.e. ‘I’). In (8-127 b), the same “intranstive” verb \textit{ik-} ‘go’ takes the “passive” affix \textit{{}-ar(ɨr)}. Here, besides the agent of \textit{ik-} ‘go’ (i.e. \textit{ura} ‘you’), another participant was introduced into the event, i.e. ‘I,’ although it is not expressed overtly in the clause. The participant introduced by \textit{{}-ar(ɨr)} (PASS) is always suffering from the action indicated by the verbal stem preceding it. This kind of use of the passive affix is called “malefactive” in Irabu Ryukyuan \citep[493-498]{Shimoji2008}.

\subsubsection{\textit{{}-ar(ɨr)} (CAP)}
\label{bkm:Ref366851018}
\textit{{}-ar(ɨr)} (CAP) expresses that the subject of the clause is capable to do the action indicated by the preceding verbal stem. The morphophonological behavior of \textit{{}-ar(ɨr)} (CAP) is similar to \textit{{}-ar(ɨr)} (PASS), but there are a few differences between them (see \sectref{sec:key:8.2.1.5} for more details). \textit{{}-ar(ɨr)} (CAP) can attach to the intransitive verb as well as the malfactive use of \textit{{}-ar(ɨr)} (PASS) as in (8-128).

(8\nobreakdash-128)  With \textit{{}-ar(ɨr)} (CAP)

  TM:  waasan  cˀjunu  məəci  ikjaranbajaa.

    \textit{waa-sa+ar-n}  \textit{cˀju=nu}  \textit{məə=kaci}  \textit{ik-ar{}-an-ba=jaa}

    young-ADJ+STV-PTCP  person=GEN  place=ALL  go-CAP-NEG-CSL=SOL

    ‘(I) cannot go to the young people’s place.’

    [Co: 120415\_01.txt]

Compare (8-128) with (8-127 a-b). In (8-128), \textit{{}-ar} (CAP) attaches to \textit{ik-} ‘go,’ but it does not introduce another participant, which is different form the malfactive use of \textit{{}-ar(ɨr)} (PASS) (see \sectref{sec:key:8.5.1.2}).

  Moreover, there is another difference between \textit{{}-ar(ɨr)} (CAP) and \textit{{}-ar(ɨr)} (PASS). The former follows \textit{{}-tuk} (PRPR) as in (8-129 a), but the latter precedes it as in (8-129 b), although the combination of \textit{{}-ar(ɨr)} (PASS) and \textit{{}-tuk} (PRPR) is only found in elicitation.

(8\nobreakdash-129)  a. \textit{{}-ar(ɨr)} (CAP) follows \textit{{}-tuk} (PRPR) [= (8-44 a)]

  TM:  {\textbar}reitou{\textbar}nansəəka  ucjukuboo,  ɨcɨɨgadɨ  jatɨn,

    \textit{reitou=nan=səəka}  \textit{uk-tuk-boo}  \textit{ɨcɨɨ=gadɨ}  \textit{jar-tɨ=n}

    freezer=LOC1=just  put-PFV-CND  when=LMT  COP-SEQ=even

    ucjukarɨi.

    \textit{uk-tuk-ar(ɨr){}-i}

    put-PRPR-CAP-NPST

    ‘If (you) put (the pickles) in the freezer, you can keep (them) no matter how long (the period of preservation) was.’

    [Co: 101023\_01.txt]

  b. \textit{{}-ar(ɨr)} (PASS) precedes \textit{{}-tuk} (PRPR)

  TM:  oosattukɨ!

    \textit{oos-ar-tuk{}-ɨ}

    scold-PASS-PRPR-IMP

    ‘Be scolded (to be mature)!’

    [El: 100221]

  \textit{{}-ar(ɨr)} (CAP) can change the syntactic valency. In (8-130 a), the subject of /kacja/ \textit{kak-tar} (write-PST) ‘wrote’ is marked by the nominative \textit{ga}, which may be replaced by \textit{n} ‘also’ as in (8-130 b). If the verb takes \textit{{}-ar(ɨr)} (CAP), the subject may be marked by the dative particle \textit{n} (DAT1) as in (8-130 c), where \textit{n} (DAT1) is not replaced, but followed by \textit{n} ‘also.’

(8\nobreakdash-130)  Without \textit{{}-ar} (CAP)

  a.  TM:  wanga  kacjattoo.

      \textit{wan=ga}  \textit{kak-tar=doo}

      1SG=NOM  write-PST=ASS

      ‘I wrote (it).’

      [El: 140227]

  b.  TM:  wanun  kacjattoo.

      \textit{wan=n}  \textit{kak-tar=doo}

      1SG=also  write-PST=ASS

      ‘I also wrote (it).’

      [El: 140227]

  With \textit{{}-ar(ɨr)} (CAP)

  c.  TM:  wannin  kakattattoo.

      \textit{wan=n=n}  \textit{kak-ar-tar=doo}

      1SG=DAT1=also  write-CAP-PST=ASS

      ‘I was also able to write (it).’

      [El: 140227]

  Before concluding this subsection, it should be mentioned that there are few rare cases where the double marking of \textit{{}-ar} (CAP) occurs. The affix \textit{{}-ar} (CAP) is always reduplicated when the verbal root ends with //aw// and is in the non-past tense with \textit{-an} (NEG): /hijoo-r-ar-an/ \textit{hijaw-ar-ar-an} (pick.up-CAP-CAP-NEG) ‘cannot pick up,’ /waroo-r-ar-an/ \textit{waraw-ar-ar-an} (laugh-CAP- CAP-NEG) ‘cannot laugh,’ and /juroo-r-ar-an/ \textit{juraw-ar-ar-an} (gather-CAP-CAP-NEG) ‘cannot gather’ (see also the appendix).

\subsubsection{\textit{{}-jur} (UMRK)}

\textit{{}-jur} (UMRK) has multiple functions and it’s prototypical function is difficult to determine. In principle, it has the characteristics as in (8-131); see also (8-1) and (8-2) in \sectref{sec:key:8.1.}

(8\nobreakdash-131)  Morphologically, \textit{{}-jur} (UMRK)

  a.  Cannot co-occur with \textit{{}-arɨr} (PASS)\footnote{From the description in §\ref{bkm:Ref303739828}, one may think of the combination of \textit{{}-arɨr-tuk-jur} (PASS-PRPR-UMRK). However, the combination of \textit{{}-arɨr} (PASS) and \textit{{}-tuk} (PRPR) is rare (see §\ref{bkm:Ref366851018}), and the combination more than two derivational affixes is also rare (see §\ref{bkm:Ref303739828}). Thus, we may postulate that \textit{{}-jur} (UMRK) cannot co-occur with (or at least rarely co-occurs with) \textit{{}-arɨr} (PASS).} or \textit{{}-arɨr} (CAP);

  b.  Cannot co-occur with \textit{{}-an} (NEG);

  c.  Cannot co-occur with \textit{{}-tur} (PROG);

  d.  Cannot co-occur with \textit{{}-jawur} (POL).

I will discuss each of these functions in turn.

With regard to (8-131 a), \textit{{}-jur} (UMRK) necessarily indicates the active voice. In Yuwan, there are three affixes that have the valency-changing function: \textit{-as} (CAUS), \textit{{}-arɨr} (PASS), and \textit{{}-arɨr} (CAP). Thus, its incapability of co-occurence with \textit{{}-arɨr} (PASS) and \textit{{}-arɨr} (CAP) greatly reduces the possibility of the change of valency.

With regard to (8-131 b), \textit{{}-jur} (UMRK) cannot co-occur with the negative affixes, i.e. \textit{{}-an} (NEG) as in (8-1) in \sectref{sec:key:8.1} or \textit{{}-azɨi} (NEG.PLQ) as in (8-67) in \sectref{sec:key:8.4.1.4.} Yuwan does not have another method to express the negative polarity. Thus, the existence of \textit{{}-jur} (UMRK) necessarily indicates the affirmative polarity.

  With regard to (8-131 c), \textit{{}-jur} (UMRK) necessarily indicates non-progressive aspect. In Yuwan, there are three affixes (except for \textit{{}-jur}) that have aspectual meaning: \textit{{}-tuk} (PRPR), \textit{{}-tur} (PROG), and \textit{{}-təər} (RSL). Among them, \textit{{}-tuk} (PRPR) and \textit{{}-təər} (RSL) can co-occur with \textit{{}-jur} (UMRK). The combination of \textit{{}-jur} (UMRK) and \textit{{}-tuk} (PRPR) will be discussed in \sectref{sec:key:8.5.1.7.} The combination of \textit{{}-jur} (UMRK) and \textit{{}-təər} (RSL) requires a special attention and it will be discussed in later in this section.

With regard to (8-131 d), \textit{{}-jur} (UMRK) necessarily indicates the non-polite style, although it does not necessarily mean the rudeness in a general sense, since \textit{{}-jur} (UMRK) can co-occur with the honorific expression (see \sectref{sec:key:8.3.1} for more details).

Additionally, \textit{{}-jur} (UMRK) belongs to the Group-II affixes, which are required by some inflectional affixes such as \textit{{}-i} (NPST) or \textit{{}-mɨ} (PLQ), since those inflectional affixes cannot directly follow the verbal root (see (8-3 b) in \sectref{sec:key:8.1} for more details).

  Considering the above facts, i.e. the active voice, the affirmative polarity, the non-progressive aspect, the non-politeness, and the necessity to some inflections, I propose that \textit{{}-jur} has some “unmarked” characteristics and abbreviate them as “UMRK” in this grammar. I will show the examples of \textit{{}-jur} (UMRK) below.

(8\nobreakdash-132)  \textit{{}-jur} (UMRK)

  a. With \textit{{}-i} (NPST) [= (8-54)]

  [Context: TM and US were talking about the present author.]

  TM:  {\textbar}hoogen{\textbar}nu  attakəə  wakajui.

    \textit{hoogen=nu}  \textit{attakəə}  \textit{wakar-jur-i}

    dialect=NOM  everything  understand-UMRK-NPST

    ‘(He) understands everything (about our) dialect.’

    [Co: 110328\_00.txt]

  b. With \textit{{}-mɨ} (PLQ) [= (8-66 a)]

  TM:  waakjaa  jantɨ ..  kamjumɨ?

    \textit{waakja-a}  \textit{jaa=nantɨ}  \textit{kam-jur-mɨ}

    1PL-ADNZ  house=LOC1  eat-UMRK-PLQ

    ‘Do (you) eat in my house?’

    [Co: 120415\_01.txt]

  In addition, \textit{{}-jur} (UMRK) can express habitual aspect if it precedes \textit{{}-tar} (PST), \textit{{}-tɨ} (SEQ), or \textit{{}-təər} (RSL) as shown in (8-133 a-g).

(8\nobreakdash-133)  \textit{{}-jur} (UMRK) expressing habitual aspect

  With \textit{{}-tar} (PST)

  a.  TM:  naakjaa  jaakacjəə  {\textbar}nenzjuu{\textbar}  ikjutanban,

      \textit{naakja-a}  \textit{jaa=kaci=ja}  \textit{nenzjuu}  \textit{ik-jur-tar{}-n=ban}

      2.HON.SG-ADNZ  house=ALL=TOP  always  go-UMRK-PST-PTCP=ADVRS

      ‘(I) always used to go to your house, but ...’

      [Co: 110328\_00.txt]

  b.  TM:  injasainnja,  minoetankjatu

      \textit{inja-sa+ar-i=n=ja}  \textit{minoe-taa=nkja=tu}

      young-ADJ+STV-INF=DAT1=TOP  Minoe-PL=APPR=COM

      asɨbjutancjɨ.

      \textit{asɨb-jur-tar{}-n=ccjɨ}

      play-UMRK-PST-PTCP=QT

      ‘(I heard MY said) that (MY) used to play with Minoe in her childhood.’

      [Co: 110328\_00.txt]

  c.  TM:  {\textbar}kanarazu{\textbar}  amantɨ  utoosjutattoo.

      \textit{kanarazu}  \textit{a-ma=nantɨ}  \textit{utaw-as-jur-tar=doo}

      necessarily  DIST-place=LOC1  sing-CAUS-UMRK-PST=ASS

      ‘(Peopole) used to necessarily have (the participants) sing (the song) there.’

      [Co: 110328\_00.txt]

  d.  TM:  gan  sjan  mununkja  sicjun\footnote{\textit{sij-} ‘know’ and \textit{{}-tur} (PROG) usually becomes /siccju(r)/ (see appendix), but it becomes /sicju(r)/ in this example.}

      \textit{ga-n}  \textit{sɨr-tar-n}  \textit{mun=nkja}  \textit{sij-tur-n}

      MES-ADVZ  do-PST-PTCP  thing=APPR  know-PROG-PTCP

      cˀjunu  wuranbaccjɨ  jˀicjutɨga,  {\textbar}nenzjuu{\textbar}

      \textit{cˀju=nu}  \textit{wur-an-ba=ccjɨ}  \textit{jˀ-tur-tɨ=ga}  \textit{nenzjuu}

      person=NOM  exist-NEG-CSL=QT  say-PROG-SEQ=FOC  always

      jutanmun,  ura  tanmɨba,  jiccja  ata.

      \textit{jˀ-jur-tar{}-n=mun  ura  tanm-ɨba  jiccj-sa  ar-tar}

      say-UMRK-PST-PTCP=ADVRS  2.NHON.SG  ask-CND  good-ADJ  STV-PST

      ‘(I) always used to say that, “There is no one who knows things like that [i.e. the dialect]” but if (I) asked you, (it) would have been good.’

      [Co: 111113\_02.txt]

  With \textit{{}-tɨ} (SEQ)

  e.  TM:  ɨcɨn  waakjoo  ikjutɨ,  urɨ  sjutassɨga.

      \textit{ɨcɨɨ=n}  \textit{waakja=ja}  \textit{ik-jur-tɨ}  \textit{u-rɨ}  \textit{sɨr-jur-tar-sɨga}

      when=any  1PL=TOP  go-UMRK-SEQ  MES-NLS  do-UMRK-PST-POL

      ‘I always used to go (to the class of kimono-making), and used to do it.’

      [Co: 120415\_01.txt]

  f.  [Context: Looking at a picture taken in the old days, where some people wore European clothes (not Japanese clothes)]

    TM:  kan  sjan  urɨnkjoo  {\textbar}nannengoro{\textbar}kara

      \textit{ka-n}  \textit{sɨr-tar-n}  \textit{u-rɨ=nkja=ja}  \textit{nannengoro=kara}

      PROX-ADVZ  do-PST-PTCP  MES-NLZ=APPR=TOP  when=ABL

      kijutɨ?

      \textit{kij-jur-tɨ}

      wear-UMRK-SEQ

      ‘Since when (people) got accustomed to wear that like this [i.e. European clothes]?’

      [Co: 111113\_01.txt]

  With \textit{{}-təər} (RSL)

  g.  TM:  urɨn  sjɨ, ..  nunkuin  sjɨ

      \textit{u-rɨ=n}  \textit{sɨr-tɨ,}  \textit{nuu-nkuin}  \textit{sɨr-tɨ}

      MES-NLZ=also  do-SEQ  what-INDFZ  do-SEQ

      moojutənwakejoo.

      \textit{moor-jur-təər{}-n=wake=joo}

      HON-UMRK-RSL-PTCP=CFP=CFM1

      ‘(The person) did it too, and used to do (everything, and we can still see the results ).’

      [Co: 120415\_01.txt]

The above examples show that the combinations of \textit{{}-jur} (UMRK) with \textit{{}-tar} (PST), \textit{{}-tɨ} (SEQ), or \textit{{}-təər} (RSL) can express habitual meaning. The habitual meaning of the clauses are also expressed by the co-occuring temporal words, i.e. \textit{nenzjuu} ‘always’ as in (8-133 a) and /ɨcɨn/ \textit{ɨcɨɨ=n} (when=any) ‘always’ as in (8-133 e).

  In fact, there are a few examples where the combination of \textit{{}-jur-tar} (UMRK-PST) does not express habitual meaning as in (8-134 a-b).

(8\nobreakdash-134)  \textit{{}-jur-tar} not expressing habitual aspect

  a.  TM:  kunugurudu  kurəə  mucjɨ  kjuuta.

      \textit{kunuguru=du}  \textit{ku-rɨ=ja}  \textit{mut-tɨ}  \textit{k-jur-ta}

      recently=FOC  PROX-NLZ=TOP  have-SEQ  come-UMRK-PST

      ‘(Satsue’s child) brought this (picture) recently.’

      [Co: 120415\_00.txt]

  b.  [Context: The following three examples are from the conversation between TM and US.]

    TM:  ikjasjɨ  sjɨ  ikjutakai,  amerikaacinkjoo?

      \textit{ikja-sjɨ}  \textit{sɨr-tɨ}  \textit{ik-jur-tar=kai}  \textit{amerika=kaci=nkja=ja}

      how-ADVZ  do-SEQ  go-UMRK-PST=DUB  America=ALL=APPR=TOP

      amerikaacjəə,  ikjasjɨ  sjɨ  watajutakai?

      \textit{amerika=kaci=ja}  \textit{ikja-sjɨ}  \textit{sɨr-tɨ}  \textit{watar-jur-tar=kai}

      America=ALL=TOP  how-ADVZ  do-SEQ  cross.over-UMRK-PST=DUB

      ‘How did (the Uncle America) go to America? How did (he) cross over to America?’

  c.  US:  nuujo?

      \textit{nuu=joo}

      what=CFM1

      ‘What?’

  d.  TM:  amerikaacinkjoo  ikjasjɨ  sjɨ  izjakai,

      \textit{amerika=kaci=nkja=ja}  \textit{ikja-sjɨ}  \textit{sɨr-tɨ}  \textit{ik-tar=kai}

      America=ALL=APPR=TOP  how-ADVZ  do-SEQ  go-PST=DUB

      un  ameeziija?

      \textit{u-n}  \textit{ameezii=ja}

      MES-ADNZ  Uncle.America

      ‘How did the Uncle America [i.e. a nickname] go to America?’

      [Co: 110328\_00.txt]

In (8-134 a), the event expressed by the clause (i.e. Satsue’s child’s bringing the picture) took place only once. Thus, \textit{{}-jur} (UMRK) in this example cannot express habitual aspect. Similarly, the event in (8-134 b-d) (i.e. the Uncle America’s crossing over to the US) took place only once. TM’s utterance in (8-134 b) is almost the same with that in (8-134 d), where \textit{{}-jur-tar} (UMRK-PST) in (8-134 b) is replaced by \textit{{}-tar} (PST). The details of the function of \textit{{}-jur} (UMRK) in (8-134 a-b) is not very clear for the present author for now, and a finer investigation is required in the future.

\subsubsection{\textit{{}-tur} (PROG)}

\textit{{}-tur} (PROG) is originated from the AVC \textit{{}-tɨ} (SEQ) plusl \textit{wur-} (PROG) (see \tabref{tab:key:93} in \sectref{sec:key:9.1.1.1} for more details). \textit{{}-tur} (PROG) can express progressive aspect. That is, \textit{{}-tur} (PROG) expresses continuing to do the action indicated by the verbal stem as in (8-135 a), or keeping up the state caused by the action indicated by the verbal stem as in (8-135 b-c).

(8\nobreakdash-135)  \textit{{}-tur} (PROG) expressing progressive aspect

  [Context: The very beginning of the monologue. TM: ‘(I will) start from the scene (where a man) picks up the pears. There is a pear-tree, (i.e.) a big tree, ...’] = (6-136)

  a.  TM:  unnəntɨ  uziiga  cˀjui  joonasi

      \textit{u-n=nəntɨ}  \textit{uzii=ga}  \textit{cˀjui}  \textit{joonasi}

      MES-ADNZ=LOC2  old.man=NOM  one.CLF.person  pear

      mutunwake.

      \textit{mur-tur{}-n=wake}

      pick.up-PROG-PTCP=CFP

      ‘There, an old man is picking up pears.’

      [PF: 090225\_00.txt]

  b.  [= (6-134 a)]

    TM:  {\textbar}ittoki{\textbar}  motojamaga  misje  katuta.

      \textit{ittoki}  \textit{motojama=ga}  \textit{misje}  \textit{kar-tur{}-tar}

      for.a.while  Motoyama=NOM  shop  borrow-PROG-PST

      ‘For a while, Motoyama was renting the shop.’

      [Co: 120415\_00.txt]

  c.  [= (6-62 a)]

    TM:  kɨɨnu  sjanannja  kagonu  tˀaacɨ  ucjutɨ,

      \textit{kɨɨ=nu}  \textit{sja=nan=ja}  \textit{kago=nu}  \textit{tˀaacɨ}  \textit{uk-tur{}-tɨ}

      tree=GEN  below=LOC1=TOP  basket=GEN  two.CLF.thing  put-PROG-SEQ

      ‘Under the tree, (the old man) put two baskets, and ...’

      [PF: 090222\_00.txt]

In (8-135 a), the old man continued to pick up the pears. In (8-135 b), Motoyama rented a shop and kept the contract for a while. In (8-135 c), the old man put baskets down and left them there.

  Interestingly, \textit{{}-tur} (PROG) can follow the existential verb \textit{wur-} ‘exist (animate).’ In that case, the verbal stem expresses a punctual state of being there as in (8-136 a-b).

(8\nobreakdash-136)  \textit{{}-tur} (PROG) following \textit{wur-} ‘exist’

  a.  [Context: TM is talking about the meeting for old people held once a month in Yuwan.]

    TM:  taruka  tˀaibəi  wututɨ,  kan

      \textit{ta-ru=ka}  \textit{tˀai=bəi}  \textit{wur-tur{}-tɨ  ka-n}

      who-NLZ=DUB  two.CLF.person=about  exist-PROG-SEQ  PROX-ADVZ

      sjan  hanasinkja  sɨrarɨppoo,  jiccjanban,

      \textit{sɨr-tar-n}  \textit{hanasi=nkja}  \textit{sɨr-arɨr-boo}  \textit{jiccj-sa+ar-n=ban}

      do-PST-PTCP  conversation=APPR  do-CAP-CND  good-ADJ+STV-PTCP=ADVRS

      ‘(It) will be good if some two (or three) people (including me) are being (there) and can make conversation like this, but ...’

      [Co: 120415\_01.txt]

  b.  TM:  waakja  umanan  wututɨn,  məə

      \textit{waakja}  \textit{u-ma=nan}  \textit{wur-tur{}-tɨ=n  məə}

      1PL  MES-place=LOC1  exist-PROG-SEQ=even  front

      tuutɨn,  munna  jan  kˀwa  jatattu.

      \textit{tuur-tɨ=n}  \textit{mun=ja}  \textit{jˀ-an}  \textit{kˀwa}  \textit{jar-tar-tu}

      pass-SEQ=even  thing=TOP  say-NEG  child  COP-PST-CSL

      ‘(The child) was a child who did not say anything even if I was being there, even if (the child) passed right in front (of me).’

      [Co: 120415\_01.txt]

In the above examples, the combination of \textit{wur-} ‘exist’ and \textit{{}-tur} (PROG) expresses the temporary state of being at these places. This phenomenon is similar to “the Progress” form of \textit{live} or \textit{stand} in English discussed in \citet{Comrie1976}, since it is said that \textit{be} \textit{living} (or \textit{be} \textit{standing}) “refers to a more temporary state” (ibid.: 37).

  In fact, \textit{{}-tur} (PROG) does not necessarily express habitual meaning. However, it can be used in the context where the clauses have habitual meaning as in (8-137 a-b).

(8\nobreakdash-137)  \textit{{}-tur} (PROG) used in the contexts that have the habitual meaning

  a.  In the non-past tense [= (7-22 c)]

    TM:  waakjoo  ɨcɨnkuin  waratuncjɨjo.

      \textit{waakja=ja}  \textit{ɨcɨɨ-nkuin}  \textit{waraw-tur{}-n=ccjɨ=joo}

      1PL=TOP  when-INDFZ  laugh-PROG-PTCP=QT=CFM1

      ‘I am always laughing (remembering the old days).’

      [Co: 120415\_00.txt]

  b.  In the past tense [= (5-31)]

    [Context: Talking with US about how they played in the past]

    TM:  nuu  sjutɨga,  asɨdutakai?

      \textit{nuu}  \textit{sɨr-jur-tɨ=ga}  \textit{asɨb-tur{}-tar=kai}

      what  do-UMRK-SEQ=FOC  play-PROG-PST=DUB

      ‘What did (we) do (when we) were playing (around here)?’

[lit. ‘Doing what, were (we) playing?’]

      [Co: 110328\_00.txt]

In the above examples, the acts indicated by the verbal stems are (or were) being carried out habitually.

\subsubsection{\textit{{}-təər} (RSL)}

\textit{{}-təər} (RSL) is originated from the AVC \textit{{}-tɨ} (SEQ) plusl \textit{ar-} (RSL) (see \tabref{tab:key:93} in \sectref{sec:key:9.1.1.1} for more details). \textit{{}-təər} (RSL) has a function that is similar to the “perfect of result” that means that “a present state is reffered to as being the result of some past situation” \citep[56]{Comrie1976}. This aspect is called “resultative” in this grammar. \textit{{}-təər} (RSL) can appear in any kind of predicate phrase as in (8-138 a-d).

(8\nobreakdash-138)  \textit{{}-təər} (RSL) expressing resultative

  In the verbal predicates

  a.  [= (6-132 a)]

    TM:  un  kˀwaga  umanan  {\textbar}boosi{\textbar}  utucjəətattu,

      \textit{u-n}  \textit{kˀwa=ga}  \textit{u-ma=nan}  \textit{boosi}  \textit{utus-təər{}-tar-tu}

      MES-ADNZ  child=NOM  MES-place=LOC1  hat  drop-RSL-PST-CSL

      ‘That boy had left [lit. dropped] (his) hat there, so ...’

      [PF: 090222\_00.txt]

  b.  TM:  zjennjukianjooga  {\textbar}heitai{\textbar}kaci  izjɨ,  (mm ..)  mɨɨ

      \textit{zjennjuki+anjoo=ga}  \textit{heitai=kaci}  \textit{ik-tɨ}    \textit{mɨɨ}

      Zenyuki+brother=NOM  soldier=ALL  go-SEQ    eye

      sɨrattəətɨ,        

      \textit{sɨr-ar-təər{}-tɨ}       

      do-PASS-RSL-SEQ        

      ‘Zenyuki went to the military, and injured [lit. had been done] (his) eyes, and ...’

      [Co: 120415\_00.txt]

  In the adjectival predicates

  c.  [Context: When the present author asked TM of the meaning of /kˀumɨtta/, TM said to MY.]

    TM:  urakjaga,  mukasi  jappoo,  kˀumɨtta

      \textit{urakja=ga}  \textit{mukasi}  \textit{jar-boo}  \textit{kˀumɨtt-sa}

      2.NHON.SG=NOM  the.past  COP-CND  scrupulous-ADJ

      atəətɨjaa.

      \textit{ar-təər-tɨ=jaa}

      STV-RSL-SEQ=SOL

      ‘If (it) is in the past, you (must have been regarded as) /kˀumɨtta/ [i.e. scrupulous].’

      [El: 120914]

  In the nominal predicates

  d.  US:  haccjanna  ikɨgaci  jatəi?

      \textit{haccjan=ja}  \textit{ikɨgaci}  \textit{jar-təər{}-i}

      Hachan=TOP  Ikegachi  COP-RSL-NPST

      ‘Was Hachan (from) Ikegachi?’

      [Co: 110328\_00.txt]

In (8-138 a), a boy dropped a hat, and the hat remained there (until another boy picked it up). In (8-138 b), Zenyuki injured his eyes, and the injury lasted thereafter. In (8-138 c), \textit{{}-təər} (RSL) shows that the situation expressed by the clause is assumed in a possible world (other than the present real world). This kind of function of \textit{{}-təər} (RSL) will be discussed later. In (8-138 d), the place where Hachan was born [i.e. Ikegachi] cannot be changed from the past to the present. Therefore, \textit{-təər} (RSL) is used in these examples.

  As mentioned in \sectref{sec:key:8.4.3}, most of the converbal affixes, e.g. \textit{{}-ba} (CSL), cannot co-occur with \textit{{}-tar} (PST). In that case, \textit{{}-təər} (RSL) expresses the past tense on behalf of \textit{{}-tar} (PST) as in (8-139 a-c).

(8\nobreakdash-139)  \textit{{}-təər} (RSL) expressing the past tense before \textit{{}-ba} (CSL)

  a.  [Context: TM was wondering when the picture had been taken. In the picture, the men wore European clothes and the women wore Japanese clothes; TM: ‘When I was a child, there were no European clothes.’]

    TM:  jingankjan  kindu  kicjutəəppajaa.

      \textit{jinga=nkja=n}  \textit{kin=du}  \textit{kij-tur-təər-ba=jaa}

      man=APPR=also  kimono=FOC  put.on-PROG-RSL-CSL=SOL

      ‘Men (in my childhood) were also wearing kimono [i.e. Japanese clothes], so (probably this picture was taken around the end of World War II).’

      [Co: 111113\_01.txt]

  b.  TM:  daaciga\footnote{It is probable that this /ga/ is not \textit{gajaaroo} (DUB), but \textit{ga} (FOC). In that case, this example would express question; that is, \textit{daa} ‘where’ is not “indefinitised.”}  cukuracjɨ  kii  jataroojaa.

      \textit{daa=kaci=gajaaroo}  \textit{cukur-as-tɨ}  \textit{k-i}  \textit{jar-tar-oo=jaa}

      where=ALL=DUB  make-CAUS-SEQ  come-INF  COP-PST-SUPP=SOL

      juwancˀjoo  cukujun  cˀjoo  wurantəəppa.

      \textit{juwan+cˀju=ja}  \textit{cukur-jur-n}  \textit{cˀju=ja}  \textit{wur-an-təər-ba}

      Yuwan+person=TOP  make-UMRK-PTCP  person=TOP  exist-NEG-RSL-CSL

      ‘Probably (they) had (someone) make (the riverboats) somewhere. Since there were no people in Yuwan who make (the riverboats).’

      [Co: 111113\_01.txt]

  c.  [Context: Remembering a bayan tree that was famous since it was very big]

    TM:  juwancˀjoo  gan  sjan  {\textbar}sjumi{\textbar}ga

      \textit{juwan+cˀju=ja}  \textit{ga-n}  \textit{sɨr-tar-n}  \textit{sjumi=ga}

      Yuwan+person=TOP  MES-ADVZ  do-PST-PTCP  hobby=NOM

      nəntəəppajaa.

      \textit{nə-an-təər-ba=jaa}

      exist-NEG-RSL-CSL=SOL

      ‘The people in Yuwan did not have a hobby like that [i.e. taking pictures], so (there is no picture of the famous banyan tree).’

      [Co: 111113\_02.txt]

In (8-139 a-c), \textit{-təər} (RSL) preceding \textit{{}-ba} (CSL) expresses the past tense. Especially, it is clear from (8-139 a), where the speaker compared the European clothes in the picture with the Japanese clothes in the past [i.e. in her childhood]. If one wants to express the resultative meaning in the same environment, one can reduplicate \textit{-təər} (RSL) as in (8-140).

(8\nobreakdash-140)  Double marking of \textit{{}-təər} (RSL) expressing the resultative and the past tense before \textit{{}-ba} (CSL)

  [Context: TM tried to remember the day when MS’s grandfather died.]

  TM:  attaaja  mˀarɨtətəəppajaa.

    \textit{a-rɨ-taa=ja}  \textit{mˀarɨr-təər-təər-ba=jaa}

    DIST-NLZ-PL=TOP  be.born-RSL-RSL-CSL=SOL

    ‘Those people had already been born (at the time when MS’s grandfather died), so ...’

    [Co: 120415\_01.txt]

In (8-140), the first \textit{{}-təər} (RSL) expresses the resultative aspect, and the second \textit{{}-təər} (RSL) expresses the past tense preceding \textit{-ba} (CSL). The double marking of \textit{{}-təər} (RSL) is the only exception for the generalization in (8-1) in \sectref{sec:key:8.1.}

  Finally, I will present the examples where \textit{{}-təər} (RSL) is used in the clauses that express counter-factual situation as in (8-141 a-c). 

(8\nobreakdash-141)  \textit{{}-təər} (RSL) used in the contexts that express counter-factual situation

  a.  TM:  kan  sjanturoonan  {\textbar}nannen{\textbar}cjɨ  kacjukuboo,

      \textit{ka-n}  \textit{sɨr-tar-n=turoo=nan}  \textit{nannen=ccjɨ}  \textit{kak-tuk-boo}

      PROX-ADVZ  do-PST-PTCP=place=LOC1  what.year=QT  write-PRPR-CND

      jiccja  atənban.jaa.  

      \textit{jiccj-sa}  \textit{ar-təər{}-n=ban=jaa} 

      good-ADJ  STV-RSL-PTCP=ADVRS=SOL  

      ‘If (someone) put the date (when the picture was taken) around here, (it) would be good (for us), but (there is no date).’

      [Co: 120415\_01.txt]

  b.  TM:  unin{\textbar}goro{\textbar}kara  naacɨbaacjɨ  umuwannən,  jəito  hamɨcɨkɨtɨ

      \textit{unin-goro=kara}  \textit{naacɨbaa=ccjɨ}  \textit{umuw-an-nən}  \textit{jəito}  \textit{hamɨcɨkɨr-tɨ}

      that.time-around=ABL  tone.deaf=QT  think-NEG-SEQ  well  do.one’s.best-SEQ

      narəəboo,  (mmm)  zjoozɨ  najutənmundoojaa.

      \textit{naraw-boo}    \textit{zjoozɨ}  \textit{nar-jur-təər{}-n=mun=doo=jaa}

      learn-CND    good.at  become-UMRK-RSL-PTCP=ADVRS=ASS=SOL

      ‘If (I) didn’t think that (I was) tone-deaf and did my best to learn (the traditional songs) since those days, (I) would have been good at (them), but (I didn’t do that).’

      [Co: 111113\_01.txt]

  c.  [Context: TM regretted that she couldn’t think of MS as a supporter to teach the dialect to the present author. Then, TM said the following utterance to the present author.]

    TM:  {\textbar}benkjoo{\textbar}  najutənmundoo.

      \textit{benkjoo}  \textit{nar-jur-təər{}-n=mun=doo}

      study  become-UMRK-RSL-PTCP=ADVRS=ASS

      ‘(If you had asked him, it) must have become good study (for you), but (it did not become so).’

      [Co: 111113\_02.txt]

All of the above examples have the conditional adverbial clauses (i.e. protasis), overtly in (8-141 a-b) and covertly in (8-141 c), and these adverbial clauses express counter-factual situations. Thus, the superordinate clauses that express their conclusions (i.e. apodosis) also express counter-factual situations, where \textit{{}-təər} (RSL) is used. The use of \textit{{}-təər} (RSL) as in (8-141 b) provides a clear contrast to \textit{{}-tar} (PST) as in (8-133 d) in \sectref{sec:key:8.5.1.4.} In (8-141 b), \textit{nar-jur-təər-n=mun} (become-UMRK-RSL-PTCP=ADVRS) ‘would have become (good at singing), but ...’ expresses a counter-factural situation. On the contrary, in (8-133 d), \textit{jˀ-jur-tar-n=mun} (say-UMRK-PST-PTCP=ADVRS) ‘used to say (a phrase), but ...’ expresses the real fact.

\subsubsection{\textit{{}-tuk} (PRPR)}

\textit{{}-tuk} (PRPR) expresses that one does the act (indicated by the verbal stem) in preparation for the future. I will tentatively call this function as “preparative (PRPR)” in this grammar. Interestingly, \textit{{}-tuk} (PRPR) cannot co-occur with \textit{{}-tar} (PST). Thus, it is probable that this affix belongs to the irrealis modality. I will present examples of \textit{{}-tuk} (PRPR) below.

(8\nobreakdash-142)  a.  [= (8-44 a)]

    TM:  {\textbar}reitou{\textbar}nansəəka  ucjukuboo,  ɨcɨɨgadɨ  jatɨn,

      \textit{reitou=nan=səəka}  \textit{uk-tuk-boo}  \textit{ɨcɨɨ=gadɨ}  \textit{jar-tɨ=n}

      freezer=LOC1=just  put-PFV-CND  when=LMT  COP-SEQ=even

      ucjukarɨi.

      \textit{uk-tuk{}-arɨr-i}

      put-PRPR-CAP-NPST

      ‘If (you) put (the pickles) in the freezer (in preparation for future), you can keep (them) no matter how long (the period of preservation) was.’

      [Co: 101023\_01.txt]

  b.  [= (8-141 a)]

    TM:  kan  sjanturoonan  {\textbar}nannen{\textbar}cjɨ  kacjukuboo,

      \textit{ka-n}  \textit{sɨr-tar-n=turoo=nan}  \textit{nannen=ccjɨ}  \textit{kak-tuk{}-boo}

      PROX-ADVZ  do-PST-PTCP=place=LOC1  what.year=QT  write-PRPR-CND

      jiccja  atənban.jaa.  

      \textit{jiccj-sa}  \textit{ar-təər-n=ban=jaa}  

      good-ADJ  STV-RSL-PTCP=ADVRS=SOL  

      ‘If (someone) put the date (when the picture was taken) around here (in preparation for future), (it) would be good (for us), but (there is no date).’

      [Co: 120415\_01.txt]

  \textbf{c.}  [Context: There was a person who threw a pack of sweets against the door of TM’s house.]

    TM:  urɨ  tɨɨ  kɨɨnnajoocjɨ, ...  ucjukɨjoocjɨ

      \textit{u-rɨ}  \textit{tɨɨ}  \textit{kɨɨr-na=joo=ccjɨ}  \textit{uk-tuk{}-ɨ=joo=ccjɨ}

      MES-NLZ  hand  hang-PROH=CFM1=QT  put-PRPR-IMP=CFM1=QT

      jˀicjɨ,      

      \textit{jˀ-tɨ}      

      say-SEQ      

      ‘(My husband) said that, “Don’t touch (it). Put (it still there in preparation for future).” And then ...’

      [Co: 120415\_01.txt]

In (8-142 a), to put the pickles in the freezer is required to preserve them. In (8-142 b), to write the date in the picture is required to prepar for someone to know in future the correct date when the picture was taken. In (8-142 c), to put the pack untouched is required for the person (who threw it) to notice that the pack is still there. In (8-142 a-b), the clasues express counter-factual (or imaginary) events. In (8-142 c), the clause that includes \textit{-tuk} (PRPR) expresses command. That is, in all of the above examples, \textit{{}-tuk} (PRPR) is used in irrealis mood.

\subsubsection{\textit{{}-jawur} (POL)}

\textit{{}-jawur} (POL) expresses the hearer-oriented politeness. \textit{{}-jawur} (POL) sometimes alternates with \textit{{}-joor}. In fact, TM and MY seldom use this politeness affix even if they speak with person who is older than them. In that case, they are likely to use the honorific verbs (see \sectref{sec:key:8.3.1}). However, MS, who is quite younger than other consultants, frequently uses the politeness affix. I will present examples of \textit{{}-jawur} (POL) below, although they were used only in elicitation.

(8\nobreakdash-143)  \textit{{}-jawur} (POL)

  \textbf{a.}  TM:  wanga  jumjawuroojəə.

      \textit{wan=ga}  \textit{jum-jawur{}-oo=jəə}

      1SG=NOM  read-POL-INT=CFM2

      ‘I will read (it).’

      [El: 110827]

  b.  TM:  wanga  dooka  utarɨjawussa.

      \textit{wan=ga}  \textit{dooka}  \textit{ut-arɨr-jawur{}-sa}

      1SG=NOM  please  hit-PASS-POL-POL

      ‘I will be hit (to play a role in the comedy), please.’

      [El: 121010]

  Additionally, there is another politeness affix, i.e. -(\textit{i})\textit{nsjoor}. However, it is not used productively in modern Yuwan, and it appeared only twice in the text corpus where the speaker imitated the phrase which she had heard when she was young as in (8-144).

(8\nobreakdash-144)  \textit{{}-}(\textit{i})\textit{nsjoor} (POL)

  TM:  {\textbar}sjooju,  sjekiju{\textbar}  konsjoorɨccjɨ.

    \textit{sjooju}  \textit{sjekiju}  \textit{koow-nsjoor{}-ɨ=ccjɨ}

    soy.sauce  oil  buy-POL-IMP=QT

    ‘(I heared that people say), “Buy the soy sauce or the oil!”’

    [Co: 110328\_00.txt]

\subsubsection{\textit{{}-an} (NEG) and \textit{{}-tar} (PST) in the non-word-final position}

\textit{{}-an} (NEG) and \textit{{}-tar} (PST) can fill the word-final position: \textit{{}-an} (NEG) as a participial affix (see \sectref{sec:key:8.4.2}), and \textit{{}-tar} (PST) as a finite-form affix (see \sectref{sec:key:8.4.1.1}). However, they can also fill the non-word-final position in the verb as in (8-145), where \textit{{}-an} (NEG) and \textit{{}-tar} (PST) is neither a participial affix nor a finite-form affix any more.

(8\nobreakdash-145)  \textit{{}-an} (NEG) and \textit{-tar} (PST) in the non-word-final position

  TM:  uihutəənu  (mm)  {\textbar}jaker{\textbar}antan  turoodu  an.

    \textit{ui+hutəə=nu}    \textit{jaker-an-tar{}-n  turoo=du  ar-n}

    upper.place+around=GEN    burn-NEG-PST-PTCP  place=FOC  exist-PTCP

    ‘(Old houses) exist just (in) the places which did not burn (by the air raid in the World War II) around the upper place (of the mountain).’

    [Co: 111113\_01.txt]

\subsection{Compounding}
\subsubsection{Basic structure}

There are several verbs composed of more than one verbal stem. The sequential verbal stems is called the verbal compound. Usually, the verbal compound is composed of only two verbal stems. The final stem in the compounds can take any kind of verbal affixes, but the non-final stem can take only \textit{{}-i}/\textit{{}-Ø} (INF), which is a kind of “nominalizer” affix (see \sectref{sec:key:8.4.4} for more details). The verbal compounds can be divided into two types depending on the strength of the unity of the stems. One type of the verbal compounds has a relatively strong unity between the stems. I have found the following three verbal compounds of this type.

\begin{styleBeschriftung}
\textmd{\tabref{tab:key:85}}\textmd{. Verbal compounds (strong unity)}
\end{styleBeschriftung}

Initial stem    Non-initial stem    Compound

\textit{us-}  ‘push’  +  \textit{{}-i} (INF)  +  \textit{jaas-}  ‘give’  >  /usijaas-/  ‘push forward’

\textit{nagɨr-}  ‘throw’  +  \textit{{}-Ø} (INF)  +  \textit{cɨkɨr-}  ‘attach’  >  /nagɨcɨkɨr-/  ‘throw at’

\textit{izir-}  ‘go out’  +  \textit{{}-Ø} (INF)  +  \textit{bar-}  N/A  >  /izibar-/  ‘go out’

All of the verbal stems in \tabref{tab:key:85}, i.e. \textit{us-} ‘push,’ \textit{jaas-} ‘give,’ \textit{nagɨr-} ‘throw,’ \textit{cɨkɨr-} ‘make,’ and \textit{izir-} ‘go out,’ can be used even by themselves, although \textit{bar-} of /izibar-/ ‘go out’ cannot appear only by itself. In other words, the \textit{bar-} is a so-called cranberry morpheme. \textit{izir-} ‘go out’ and \textit{izir-Ø+bar-} ‘go out’ seem to have the same meaning. In my texts, however, the former \textit{izir-} ‘go’ is almost always used only by itself, and the latter \textit{izir-Ø+bar-} ‘go out’ is used only to fill the lexical verb slot in the auxiliary verb construnction as in (8-146 c). I will present examples of the compounds in \tabref{tab:key:85} below.

(8\nobreakdash-146)  Verbal compounds (strong unity)

  a.  /usijaas-/ ‘push forward’

    TM:  usijaasɨ!

      \textit{us-i+jaas{}-ɨ}

      push-INF+give-IMP

      ‘Push (it) forward!’

      [El: 110330]

  b.  /nagɨcɨkɨr-/ ‘throw at’ [=(8-86 b)]

    TM:  umanan  mata  nagɨcɨkɨtəəppa,

      \textit{u-ma=nan}  \textit{mata}  \textit{nagɨr-Ø+cɨkɨr{}-təər-ba}

      MED-place=LOC1  again  throw-INF+attach-RSL-CSL

      ‘(The person) have thrown (some sweets) again (at our house), so ...’

      [Co: 120415\_01.txt]

  c.  /izibar-/ ‘go out’

    TM:  agan  izibatɨ  izjɨ,

      \textit{aga-n}  \textit{izi-Ø+bar{}-tɨ  ik-tɨ}

      DIST-ADVZ  go.out-INF+?-SEQ  go-SEQ

      ‘(I) went out (of my house into) there, and ...’

      [Co: 101020\_01.txt]

  Next, the other type of the verbal compounds has a relatively weak unity between the stems, where either the initial stem or the non-initial stem expresses a grammatical (rather than lexical) meaning. First, I will present an example where the initial stem expresses a grammatical meaning.

\begin{styleBeschriftung}
\textmd{\tabref{tab:key:86}. Verbal stem that expresses a grammatical meaning in the initial stem of a compound}
\end{styleBeschriftung}

Form    Meaning only by itself    Meaning in the initial stem in a compound

\textit{ut-}    ‘hit’    Emphasis

(8\nobreakdash-147)  Verbal compounds (weak unity; initial stem expresses a grammatical meaning)

  a.  \textit{ut-} (EMP)

    TM:  ucitoocja,  {\textbar}amerikazin{\textbar}gadɨ.

      \textit{ut{}-i+toos-tar  amerikazin=gadɨ}

      EMP-INF+lay.down-PST  Amerika.person=LMT

      ‘(They) knocked out the American (soldiers stationed in Yuwan).’

      [Co: 120415\_00.txt]

  b.  \textit{ut-} (EMP)

    TM:  saisai  ucikˀurawɨ!

      \textit{sai+sai}  \textit{ut{}-i+kˀuraw-ɨ}

      RED+quickly  EMP-INF+eat.DRG-IMP

      ‘Eat (the meal) quickly!’

      [El: 130821]

A morpheme that can express a grammatical meaning in filling in the initial slot in the compound is only \textit{ut-}. It lexically means ‘hit,’ but it means some emphatic meaning when it precedes another verbal stem in the compound as in (8-147 a-b).

  Secondly, I will present verbal stems that can express grammatical meanings when they fill in the non-initial slot in the compound.

\begin{styleBeschriftung}
\textmd{\tabref{tab:key:87}}\textmd{. Verbal stems that express grammatical meanings in the non-initial stems in} \textmd{compounds}
\end{styleBeschriftung}

Form    Meaning only by itself    Meaning in the non-initail stem in a compound

\textit{kij-}    ‘cut’    Capability

\textit{agɨr-}    ‘raise’    ‘elaborately’

\textit{hatɨr-}    N/A    ‘thoroughly’

\textit{kˀuraw-}    (eat.DRG)    Derogative

\textit{kum-}    N/A    ‘into’

\textit{jukkjaar-}\footnote{The final consonant //r// of the underlying form \textit{jukkaar-} ‘begin’ is only included based on the supposition of the present author, since I could not elicit the speaker to utter the example where it is followed by a vowel-initial affix. There is another form /jukkjaajui/ \textit{jukkjaa}(\textit{r})\textit{{}-jur-i} (begin-UMRK-NPST) ‘begins to do.’ Thus, I attach //r// to the stem, which is the most productive morphophoneme in the verbal stem-final positions.}    N/A    Ingressive

Among the verbal stems in \tabref{tab:key:87}, \textit{kij-} (CAP) is the most productive one (see also \sectref{sec:key:8.5.2.2}). \textit{hatɨr-}, \textit{kum-}, and \textit{jukkjaar-} cannot be used only by themselves, i.e., they always follow another verbal stem as in (8-148 e-f, i-k). I will present below examples of compounds where the verbal stems in \tabref{tab:key:87} follow other verbal stems.

(8\nobreakdash-148)  Verbal compounds (weak unity; non-initial stems express grammatical meanings)

  \textit{kij-} (CAP)

  a.  TM:  naa{\textbar}ittoki{\textbar}du  siikijuijo.

      \textit{naa+ittoki=du}  \textit{sɨr-i+kir{}-jur-i=joo}

      other+moment=FOC  do-INF+CAP-UMRK-NPST=CFM1

      ‘(She) can do [i.e. can sing and dance the traditoinal music] for a while.’

      [Co: 120415\_01.txt]

  b.  TM:  wˀaacjɨnkjoo  jˀiikijantanmun.

      \textit{wˀaa=ccjɨ=nkja=ja}  \textit{jˀ-i+kij{}-an-tar-n=mun}

      pig=QT=APPR=TOP  say-INF+CAP-NEG-PST-PTCP=ADVRS

      ‘(A teacher who came to Yuwan before) was not able to say \textit{wˀaa} [i.e. ‘pig’] (in the correct pronuciation in Yuwan).’

      [Co: 110328\_00.txt]

  \textit{agɨr-} ‘elaborately’

  c.  [Context: Telling a person to scour all the metal goods in the kitchen]

    TM:  attakəə  tugjagɨrɨjoo!

      \textit{attakəə}  \textit{tug-i+agɨr{}-ɨ=joo}

      everything  whet-INF+elaborately-IMP=CFM1

      ‘Scour out all (of the metal goods) completely!’

      [El: 121006]

  d.  TM:  un  mamɨnkjoo  kjuraasanma  sjugjagɨrɨjoo!

      \textit{u-n}  \textit{mamɨ=nkja=ja}  \textit{kjura-sanma}  \textit{sjug-i+agɨr{}-ɨ=joo}

      MES-ADNZ  bean=APPR=TOP  beautiful{}-ADVZ  hit{}-INF+elabolately{}-IMP=CFM1

      ‘Smash the beans beautifully [i.e. elaborately]!’

      [El: 130821]

  \textit{hatɨr-} ‘thoroughly’

  e.  [Context: Talking about a man who came from mainland Japan to buy cycad leaves for

business.] = (4-25 b)

    TM:  kiihatɨppoo,  sirɨtuppajaa.

      \textit{kij-Ø+hatɨr{}-boo  sirɨr-tur-ba=jaa}

      cut-INF+thoroughly-CND  easy.to.understand-PROG-CSL=SOL

      ‘If (he) cut all the cycad leaves, you may know (what would happen then).’

      [Co: 111113\_01.txt]

  f.  TM:  attakəə  jumhatɨrɨjoo.

      \textit{attakəə}  \textit{jum-Ø+hatɨr{}-ɨ=joo}

      everything  read-INF+thoroughly-IMP=CFM1

      ‘Read thoroughly all of (the pages)!’

      [El: 121006]

  \textit{kˀuraw-} (DRG)

  g.  TM:  kanɨcɨboja  urakja  tuikˀurawɨcjɨ  jˀicjɨ,

      \textit{kanɨ+cɨbo=ja}  \textit{urakja}  \textit{tur-i+kˀuraw{}-ɨ=ccjɨ  jˀ-tɨ}

      gold+pot=TOP  2.NHON.PL  take-INF+DRG-IMP=QT  say-SEQ

      ‘(The man) said that, “You take (this) damn gold pot!” and ...’

      [Fo: 090307\_00.txt]

  h.  TM:  agaraa  munnu  wuikˀuratɨ,  sɨrarantajaa.

      \textit{aga-raa}  \textit{mun=nu}  \textit{wur-i+kˀuraw{}-tɨ  sɨr-arɨr-an-tar=jaa}

      DIST-DRG  person=NOM  exist-INF+DRG-SEQ  do-CAP-NEG-PST=SOL

      ‘That awful person was (there), and (we) could not do (any conversation).’

      [El: 111104]

  \textit{kum-} ‘into’

  i.  [= (8-113 c)]

    TM:  ukkaci  makikum  jatattujaa.

      \textit{u-rɨ=kaci}  \textit{mak-i+kum{}-Ø  jar-tar-tu=jaa}

      MES-NLZ=ALL  roll-INF+into-INF  COP-PST-CSL=SOL

      ‘(The old-type audio recorder) rolled up (the tape of a side) into that [i.e. the other side] (during the recoding).’

      [Co: 120415\_01.txt]

  j.  TM:  wuduikumɨ!

      \textit{wudur-i+kum{}-ɨ}

      jump-INF+into-IMP

      ‘Jump into (there)!’

      [El: 110914]

  \textit{jukkjaar-} (INGR)

  k.  [= (7-3 d)]

    TM:  kan  sjɨ  jankjanu  dɨkɨɨjukkjaija

      \textit{ka-n}  \textit{sɨr-tɨ}  \textit{jaa=nkja=nu}  \textit{dɨkɨr-Ø+jukkjaar{}-i=ja}

      PROX-ADVZ  do-SEQ  house=APPR=NOM  be.made-INF+INGR-INF=TOP

      {\textbar}nan+nengoro{\textbar}karakai?

      \textit{nan+nen-goro=kara=kai}

      what+year-about=ABL=DUB

      ‘When did the houses begin to be made like this?’

      [Co: 110328\_00.txt]

It should be noted that the stem-boundary of the verbal compounds in (8-148 c-d) behaves differently from that of the nominal compounds, e.g. /hidesianjoo/ \textit{hidesi+anjoo} (Hideshi+older.brother) ‘Hideshi.’ Their difference is presented in \tabref{tab:key:88}, where the syllable boundaries in the surface forms of the compounds are indicated by periods.

\begin{styleBeschriftung}
\textmd{\tabref{tab:key:88}}\textmd{. Morphophonological difference of //i// + //a// in a nominal compound and a verbal compound}
\end{styleBeschriftung}

  Preceding stem    Following stem    Compound

Nominal compound  \textit{hidesi}  ‘Hideshi’  +  \textit{anjoo}  ‘older brother’  >  /hi.de.si.a.njoo/

[çide̞ɕiɑ̞nʲo̞ː]

Verbal comopund  \textit{kakjoos-i}  (mix-INF)  +  \textit{agɨr-}  ‘elaborately’  >  /ka.kjoo.sja.gɨr/

[kɑ̞kʲo̞ːɕɑ̞gɨɾɨɾ]

The above table shows that in the nominal compound the stem-final //i// and the stem-initial //a// retain their forms such as /i.a/. In the verbal compound, however, they are fused into /ja/.

\subsubsection{Remarks on \textit{kij-} (CAP)}

\textit{kij-} (CAP) introduced in \sectref{sec:key:8.5.2.1} needs two more explanations. First, there is a case where the semantic scope of \textit{kij-} (CAP) goes beyond the compound. I will present examples below, where the compounds are underlined.

(8\nobreakdash-149)  \textit{kij-} (CAP) with AVC

  a.  TM:  kacjɨ  moikijunnja?

      \textit{kak-tɨ}  \textit{moor-i+kij{}-jur-i=na}

      \{[write-SEQ]  [HON-INF]\}+CAP-UMRK-NPST=PLQ

      \{[Lexical verb]  [Auxiliary verb]\}\textsubscript{VP}

      ‘Would (you) be able to write (it)?’

      [El: 120924]

  b.  TM:  hɨɨtɨ  moikijanna?

      \textit{hɨɨr-tɨ}  \textit{moor-i+kij{}-an=na}

      \{[get.up-SEQ  HON-INF]\}+CAP-NEG=PLQ

      \{[Lexical verb]  [Auxiliary verb]\}\textsubscript{VP}

      ‘Wouldn’t (you) be able to get up?’

      [El: 120929]

It will be discussed in \sectref{sec:key:9.1.1} that Yuwan has the auxiliary verb construction (AVC) in the verbal phrase (VP), and the AVC is composed of a preceding lexical verb and a following auxiliary verb. For example, /kacjɨ/ \textit{kak-tɨ} (write-SEQ) in (8-149 a) is a lexical verb, and it forms an AVC with the following auxiliary verb \textit{moor-} (HON). Similarly, /hɨɨtɨ/ \textit{hɨɨr-tɨ} (get.up-SEQ) in (8-149 b) is a lexical verb, and it also forms an AVC with \textit{moor-} (HON). In (8-149 a-b), \textit{kij-} (CAP) forms a compound. Morphologically, the compound only includes the auxiliary verbal stem, because there is a word boundary between the lexical verb and the auxiliary verb. Semantically, however, the scope of \textit{kij-} (CAP) includes the whole AVC, i.e. both of the lexical verb and the auxiliary verb. This can be diagrammed as in the following table.

\begin{styleBeschriftung}
\textmd{\tabref{tab:key:89}. The difference of morphological unity and semantic scope of} \textmd{\textit{kij-}}\textmd{ (CAP) (part 1)}
\end{styleBeschriftung}

  Lexical verb  Auxiliary verb\textit{+kij-}

Morphological unity    <<<<<<<<<<<<

Semantic scope  <<<<<<<<<  <<<<<<<<<<<<

The above table shows that \textit{kij-} (CAP) morphologically forms a compound only with the auxiliary verbal stem. However, its semantic scope also includes the preceding lexical verb. In other words, \textit{kij-} (CAP) seems to attach to the preceding VP as a whole, which may be diagrammed as follows.

\begin{styleBeschriftung}
\textmd{\tabref{tab:key:90}. The difference of morphological unity and semantic scope of} \textmd{\textit{kij-}}\textmd{ (CAP) (part 2)}
\end{styleBeschriftung}

E.g.  \textit{kak-tɨ} (write-SEQ)  \textit{moor-i+kij-} (HON-INF+CAP)

Morphologically  [Lexical verb]\textsubscript{word}  [Auxiliary verb+\textit{kij-}]\textsubscript{Compound}

Semantically  \{Lexical verb  Auxiliary verb\}\textsubscript{VP}+ \textit{kij-}

The semantic scope of the verbal affixes that attach to the auxiliary verb always include both of the lexical verb and the auxiliary verb. In that meaning, \textit{kij-} (CAP) has the same characteristic with the verbal affixes. For example, if \textit{{}-an} (NEG) attaches to the auxiliary verb, its semantic scope necessarily includes the preceding lexical verb as in (9-15) in \sectref{sec:key:9.1.1.3}, where \textit{-an} (NEG) negates \textit{umuw-} ‘think’ as well as \textit{kurɨr-} (BEN).

  Secondly, both of the verbal root \textit{kij-} (CAP) and the verbal affix \textit{{}-arɨr} (CAP) (see \sectref{sec:key:8.5.1.3}) can express capability. However, the range of capablity they can express is different as in \tabref{tab:key:91}.

\begin{styleBeschriftung}
\textmd{\tabref{tab:key:91}}\textmd{. The range of capability that} \textmd{\textit{kij-}}\textmd{ (CAP) and} \textmd{\textit{{}-arɨr}}\textmd{ (CAP) express}
\end{styleBeschriftung}

  \textit{kij-} (CAP)  \textit{{}-arɨr} (CAP)

Capability construed (by the speaker) as depending on one’s ability  +  +

Capability construed (by the speaker) as depending on the surroundings  {}-  +

First, if the speaker construes that the capability of the action indicated by the verbal stem depends on the agent’s ability, one can use both \textit{kij-} (CAP) and \textit{{}-arɨr} (CAP) as in (8-150 a-b).

(8\nobreakdash-150)  Capability construed (by the speaker) as depending on one’s ability

  a. \textit{kij-} (CAP)

  TM:  sijansjutɨ,  cukuikijanta.

    \textit{sij-an=sjutɨ}  \textit{cukur-i+kij{}-an-tar}

    know-NEG=SEQ  make-INF+CAP-NEG-PST

    ‘(I) don’t know (how to make the dish), and could not make (it).’

    [El: 101119]

  b. \textit{{}-arɨr} (CAP)

  TM:  sijansjutɨ,  cukuraranta.

    \textit{sij-an=sjutɨ}  \textit{cukur-ar{}-an-tar}

    know-NEG=SEQ  make-CAP-NEG-PST

    ‘(I) don’t know (how to make the dish), and could not make (it).’

    [El: 101119]

In both of the examples in (8-150 a-b), the speaker does not know how to make the dish. Thus, the capability in (8-150 a-b) is construed by the speaker as depending on the speaker’s ability, where both of \textit{kij-} (CAP) and \textit{{}-arɨr} (CAP) can be used.

Secondly, if the speaker construes the capability of the action indicated by the verbal stem depends on the surroundings (not the agent’s ability), one cannot use \textit{kij-} (CAP), and can only use \textit{{}-arɨr} (CAP) as in (8-151 a-b).

(8\nobreakdash-151)  Capability construed (by the speaker) as depending on the surroundings

  a. \textit{kij-} (CAP)

  TM:  *himanu  nənsjutɨ,  cukuikijanta.

    \textit{hima=nu}  \textit{nə-an=sjutɨ}  \textit{cukur-i+kij{}-an-tar}

    time=NOM  exist-NEG=SEQ  make-INF+CAP-NEG-PST

    [Intended meaning]‘(I) have no time (to spare), and could not make (it).’

    [El: 101119]

  b. \textit{{}-arɨr} (CAP)

  TM:  himanu  nənsjutɨ,  cukuraranta.

    \textit{hima=nu}  \textit{nə-an=sjutɨ}  \textit{cukur-ar{}-an-tar}

    time=NOM  exist-NEG=SEQ  make-CAP-NEG-PST

    ‘(I) have no time (to spare), and could not make (it).’

    [El: 101119]

In both of the examples in (8-151 a-b), the speaker does not have enough time to spare. Thus, the capability in (8-151 a-b) is construed by the speaker as depending on the surroundings (not the speaker’s ability), where \textit{kij-} (CAP) cannot be used, and only \textit{{}-arɨr} (CAP) can be used.
