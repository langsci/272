\chapter{Predicate phrases}\label{chap:9}

The basic clause of Yuwan is made of an argument (or arguments) and a predicate phrase (see \sectref{sec:4.1.1}). Yuwan has three types of predicate phrases as in (9-1), where the contents enclosed within parentheses may not appear in some environments.

\ea   Three types of predicate phrases \label{ex:9.1}
  \begin{tabular}{@{}l@{ }lll@{}}
  a. & Verbal predicate phrase      &  (Complement)         &  VP\footnotemark[1]\\
  b. & Adjectival predicate phrase  &  A\footnotemark[2]    & (STV\footnotemark[3])\\
  c. & Nominal predicate phrase     &  NP                   & (COP\footnotemark[4])\\
  \end{tabular}
\z
\footnotetext[1]{“VP” indicates the verbal phrase.}
\footnotetext[2]{“A” indicates the adjective.}
\footnotetext[3]{“STV” indicates a stative verb.}
\footnotetext[4]{“COP” indicates a copular verb.}

The verbal predicate phrase is composed of a verbal phrase (VP) and a complement. The VP is always necessary, and it is composed of an obligatory lexical verb and an optional auxiliary verb (see \sectref{sec:9.1.1}). The complement is required when the lexical verb is a light verb (see \sectref{sec:9.1.2}). The adjectival predicate phrase is composed of an obligatory adjectival word, which may be followed by a VP whose lexical verb is the stative verb (see \sectref{sec:9.2}). The nominal predicate phrase is composed of an obligatory NP, which may be followed by a VP whose lexical verb is the copular verb (see \sectref{sec:9.3}). For the people who are interested in the argumentation for the structural analyses presented in (9-1), it is recommended to see \sectref{sec:9.4}.

\section{Verbal predicate phrase}\label{sec:9.1}

The verbal predicate phrase has the following structure.

\ea   Structure of the verbal predicate phrase \label{ex:9.2}\\\relax
  [(Complement) \hspace{2\tabcolsep} VP]\textsubscript{Verbal predicate phrase}
\z

The verbal phrase (VP) is composed of an obligatory lexical verb and an optional auxiliary verb, which will be discussed in \sectref{sec:9.1.1}. Furthermore, the complement is necessary when the lexical verb is a light verb. This will be discussed in \sectref{sec:9.1.2}. The complement is required by the verb (in the VP), but it is not the argument of the verb. Thus, the component in the complement slot does not take any case particle (except for the case in (6-42 e) in \sectref{sec:6.3.2.1}). It should be mentioned that the “verbal predicate phrase” is different from the “verbal phrase (VP),” and that both of the descriptive ideas do not include any NP argument within them (cf. \citealt{Andrews2007}: 135). Arguments in Yuwan frequently undergo ellipsis if they are inferable from the context. This non-obligatory characteristic of arguments is the reason why they are not included in the VP or the verbal predicate phrase.

\subsection{Verbal phrase and the auxiliary verb construction}\label{sec:9.1.1}

The verbal phrase (VP) is made of an obligatory lexical verb and an optional auxiliary verb. The VP structures are diagramed below. “(Lexical or Auxiliary verb\textsubscript{ 0...n})” means that a number of lexical verbs or auxiliary verbs may fill the slot.

\ea   Structures of the VP \label{ex:9.3}
  \ea Minimal VP\\
  \begin{tabular}{@{}ll@{}}
  Syntax: & [Lexical verb]\textsubscript{VP}\\
  Morphology: & Unrestricted\\
  \end{tabular}
  \ex Non-minimal VP (= Auxiliary verb construction)\\
  \begin{tabular}{@{}llll@{}}
  Syntax: & [Lexical verb & (Lexical or Auxiliary verb\textsubscript{0...n}) & Auxiliary verb]\textsubscript{VP}\\
  Morphology: & SEQ & SEQ & Unrestricted
  \end{tabular}
  \z
\z

The minimal VP is only composed of a lexical verb. The lexical verb in the minimal VP can take all of the inflections, i.e., it is morphologically unrestricted as in (9-3 a). A VP may be composed of more than a verb. In that case, a lexical verb stands in the initial place, and an auxiliary verb stands in the final place. Between them, a number of lexical verbs or auxiliary verbs may intervene, though it is rare. This structure of non-minimal VP is called the auxiliary verb construction (AVC). Interestingly, the non-\textit{final} verbs in the AVC can take only an inflection, i.e. \textit{{}-tɨ} (SEQ), and only the final auxiliary verb can take all of the inflections as in (9-3 b). In other words, the coincidence of the lexical meaning and the morphological freedom (i.e. the “semantic head” and the “inflectional head” in \citealt{Anderson2006}: 22-23) in the minimal VP is separated into two different verbs in AVC, which is not uncommon in the languages in the world (\citealt{Lehmann1995}: 33-34, \citealt{Anderson2006}: 24). The examples of the minimal VP and the non-minimal VP (i.e. AVC) are shown below.

\ea  \label{ex:9.4}

\exi{} Minimal VP 

\ea %TM:
 \gllll  nuukanu  ai.\\
      \textit{nuu=ka=nu}  \textit{\Highlight{ar-i}}\\
      what=DUB=NOM  [exist-NPST]\\
        [Lex. V]\textsubscript{VP}\\
      \glt       ‘There is something.’ [Co: 120415\_01.txt]

\exi{}  Auxiliary verb construction (= Non-mimial VP)

\ex \label{ex:.b} %TM:
    \gllll  nu-nkuin  atɨ  moojuijo.\\
      \textit{nuu-nkuin}  \textit{\Highlight{ar-tɨ}}  \textit{\Highlight{moor-jur-i}=joo}\\
      what-INDFZ  [exist-SEQ  HON-UMRK-NPST]=CFM1\\
        [Lex. V  Aux. V]\textsubscript{VP}\\
      \glt       ‘There is anything (at the place of the grandfather of MS).’ [Co: 120415\_01.txt]

\ex \label{ex:.c} %TM:
    \gllll  nannja  kacjɨ  moocjɨn  njan?\\
      \textit{nan=ja}  \textit{\Highlight{kak-tɨ}}  \textit{\Highlight{moor-tɨ}=n}  \textit{\Highlight{nj-an}}\\
      2.HON.SG=TOP  [write-SEQ  HON-SEQ=even  EXP-NEG]\\
        [Lex. V  Aux. V  Aux. V]\textsubscript{VP}\\
      \glt       ‘Have you never written (it before)?’ [El: 120929]
    \z
\z

In (9-4 a), the VP is only composed of a lexical verb /ai/ \textit{ar-i} (exist-NPST). In (9-4 b), /atɨ/ \textit{ar-tɨ} (exist-SEQ) and /moojui/ \textit{moor-jur-i} (HON-UMRK-NPST) forms a single VP, where the auxilary verb adds some honorific meaning to the preceding lexical verb. In (9-4 c), the VP is composed of a sequence of three verbs. As mentioned above, the non-final verbs in AVC necessarily take the inflection \textit{{}-tɨ} (SEQ) such as /atɨ/ \textit{ar-tɨ} (exist-SEQ) in (9-4 b) and /kacjɨ/ \textit{kak-tɨ} (write-SEQ) and /moocjɨ/ \textit{moor-tɨ} (HON-SEQ) in (9-4 c).

  The AVC is a mono-clausal structure that minimally consists of a lexical verb and an auxiliary verb, the latter expressing grammatical function (cf. \citealt{Anderson2006}: 7). In fact, the verbal form of the non-final position in the AVC has the same form with the verbal form in the adverbial clause. That is, both of them take \textit{{}-tɨ} (SEQ). However, the \textit{{}-tɨ} (SEQ) in AVC does not form a clausal boundary, but it is included in a mono-clause. The mono-clausality of AVC is exemplified by the semantic scope of the negation. I will present the relevant examples below.

\ea   Difference of the semantic scope of negation \label{ex:9.5}

  \ea\label{ex:9.5a} Mono-clausal AVC\\\relax
  [Context: Akira wanted something of Yuto’s, but Yuto did not want to give it to him. Therefore, Yuto asked Hayato to deny Akira’s wish, but Hayato did not do it for Yuto. In that case, TM thought that Yuto can utter the following sentence.]

%TM:
 \gllll  kurɨrancjəə  jˀicjɨ  kurɨranta.\\
    \textit{kurɨr-an=ccjɨ=ja}  \textit{jˀ-tɨ}  \textit{kurɨr-\Highlight{an}-tar}\\
    [give-NEG=QT=TOP  say-SEQ  BEN-NEG-PST]\\
    [Complement  Lex. verb  Aux. verb]\textsubscript{VP (in a clause)}\\
    \glt     ‘(Hayato) did not say for me that, “(Yuto) don’t give (it to you).”’ [El: 130821]

  \ex\label{ex:9.5b} Clause chaining\\\relax
  [Context: Yuto asked Hayato to give Hayato’s precious thing to him. However, Hayato denied the Yuto’s wish. In that case, TM thought that Yuto can utter the following sentence.]

%TM:
 \gllll  kurɨrancjɨ  jˀicjɨ,  kurɨrantattoo.\\
    \textit{kurɨr-an=ccjɨ}  \textit{jˀ-tɨ}  \textit{kurɨr-\Highlight{an}-tar=doo}\\
    [give-NEG=QT  say-SEQ]  [give-NEG-PST=ASS]\\
    [Complement  Lex. verb]\textsubscript{VP (in a clause)}  [Lex. verb]\textsubscript{VP (in a clause)}\\
    \glt     ‘(He) said, “(I) don’t give (it),” and didn’t give (it to me).’ [El: 130821]
\z
\z

In (9-5 a), the verbal form /jˀicjɨ/ \textit{jˀ-tɨ} (say-SEQ) forms a mono-clausal VP with the following auxiliary verb, i.e. \textit{kurɨr-} (BEN), since the semantic scope of negation of the following verb includes the whole VP. In this example, \textit{jˀ-} ‘say’ is also negated by the \textit{{}-an} (NEG) of \textit{kurɨr-an-tar} (BEN-NEG-PST). In (9-5 b), however, the semantic scope of negation of the following verb does not include the preceding verb. That is, the \textit{{}-an} (NEG) of \textit{kurɨr-an-tar} (give-NEG-PST) does not negate the preceding \textit{jˀ-} ‘say.’ Thus, we can regard that the verbal forms /jˀicjɨ/ \textit{jˀ-tɨ} (say-SEQ) and \textit{kurɨr-an-tar} (give-NEG-PST) in (9-5 b) are not in the same clause. In fact, the above syntactic difference is also reflected in the semantic difference of the verbal form /kurɨr-/. In (9-5 a), it functions as an auxiliary verb \textit{kurɨr-} (BEN), but in (9-5 b) it functions as a lexical verb \textit{kurɨr-} ‘give.’ Additionally, the suprasegmental behavior in (9-5 a-b) is different. In (9-5 a), \textit{jˀ-tɨ} \textit{kurɨr-an-tar} (say-SEQ BEN-NEG-PST) forms a single prosodical unit, but in (9-5 b), \textit{jˀ-tɨ} (say-SEQ) and \textit{kurɨr-an-tar} (give-NEG-PST) does not. Moreover, there is a pause between \textit{jˀ-tɨ} (say-SEQ) and \textit{kurɨr-an-tar} (give-NEG-PST) in (9-5 b), but there is no pause between \textit{jˀ-tɨ} (say-SEQ) and \textit{kurɨr-an-tar} (BEN-NEG-PST) in (9-5 a).

Another difference between a mono-clausal AVC and a clause chaining is that the latter allows another word to intervene between the clauses.

\ea   The possibility of the insertion of another word \label{ex:9.6}

  \ea \label{ex:9.6a}Mono-clausal AVC\\\relax
  [Context: The same context with (9-5 a)]\\
%TM:
 \glll  *kurɨrancjəə  jˀicjɨ  akiran  kurɨranta.\\
    \textit{kurɨr-an=ccjɨ=ja}  \textit{jˀ-tɨ}  \textit{\Highlight{akira=n}}  \textit{kurɨr-an-tar}\\
    give-NEG=QT=TOP  say-SEQ  Akira=DAT1  BEN-NEG-PST\\
    \glt     (Intended meaning) ‘(Hayato) did not say \Highlight{to Akira} for me that, “(Yuto) doesn’t give (it).”’ [El: 130821]

  \ex\label{ex:9.6b}\relax [Context: The same context with (9-5 b)]

%TM:
 \glll  kurɨrancjɨ  jˀicjɨ,  wannin  kurɨranta.\\
    \textit{kurɨr-an=ccjɨ}  \textit{jˀ-tɨ}  \textit{\Highlight{wan=n=n}}  \textit{kurɨr-an-tar}\\
    give-NEG=QT  say-SEQ  1SG=DAT1=even  give-NEG-PST\\
    \glt     ‘(Hayato) said, “(I) don’t give (it),” and didn’t give (it) \Highlight{to me}.’ [El: 130821]
    \z
\z

In (9-6 a), the NP \textit{akira=n} (Akira=DAT1) ‘to Akira’ cannot be inserted between the lexical verb and the auxiliary verb. On the contrary, in (9-6 b), the NP \textit{wan=n} (1SG=DAT1) ‘to me’ can be inserted between two clauses.

Yuwan has the following auxiliary verbs as in \tabref{tab:92}, many of which can also be used as lexical verbs. In other words, many of the verbs in the following table are in the diachronic change of grammaticalization (cf. \citealt{Lehmann1995}: 37).

\begin{table}
\caption{\label{tab:92}Auxiliary verbs in Yuwan}
\begin{tabularx}{\textwidth}{l@{ }QllQ}
\lsptoprule
& Category & Forms  & \multicolumn{2}{c}{Meaning}\\\cmidrule(lr){4-5}
&          &        &as auxiliary verbs  &as lexical verbs\\\midrule
1. &  Aspect & \textit{wur-} & PROG & ‘exist (animate)’\\
   &         & \textit{ar-}/\textit{nə-} & RSL &  ‘exist (inanimate)’\\
   &         & \textit{nj-}\footnote{The auxiliary verb \textit{nj-} (EXP) has the same form with the verb of another dialect of Amami, i.e. \textit{nj-} ‘see,’ in Ura (Nothern Amami) (Dr. Hiromi Shigeno, 2013, p.c.)} & EXP &  N/A\\
   &         & \textit{mj-} & ‘try to’ & ‘see’\\
2. &  Honorific & \textit{moor-}\footnote{One may think that the cognate of \textit{moor-} (HON) is \textit{umoor-} (exist/go/come/speak.HON). However, there is no initial glottalization on \textit{moor-} (HON). On the contrary, the words that are supposed to have had the sequence of a vowel and a nasal in the word-initial positions are thought to have lost their initial vowels with glottalization of the following nasals, e.g. *\textit{uma} > \textit{mˀa} ‘horse’ or *\textit{inoci} > \textit{nˀjuci} ‘life’ (see also \sectref{sec:2.3.2.3}).} & HON & N/A\\
3. &  Valency-changing & \textit{kurɨr-} & BEN & ‘give’\\
   &                   & \textit{muraw-} & BEN & ‘receive’\\
   & Valency-changing + Honorific & \textit{taboor-} & BEN.HON & N/A\\
4. & Spatial deixis & \textit{ik-} &  ‘go’  &  ‘go’\\
   &                & \textit{k-}  & ‘come’ &   ‘come’\\
   & Spatial deixis + Honorific & \textit{umoor-} & go/come.HON & go\slash come\slash exist\slash speak.HON\\
\lspbottomrule
\end{tabularx}
\end{table}

\tabref{tab:92} shows that the auxiliary verbs in Yuwan can be grouped into four categories, i.e. aspect, honorific, valency-changing, and spatial deixis. In principle, the aspectual auxiliaries can follow other types of auxiliary verbs as in (9-4 c). Additionally, the valency-changing auxiliaries can follow the spatial deictic auxliary verbs as in (9-21) in \sectref{sec:9.1.1.4}. The examples of the each auxiliary verb in \tabref{tab:92} will be discussed in the following subsections.

\subsubsection{Aspectual auxiliary verbs: \textit{wur-} (PROG), \textit{ar-}/\textit{nə-} (RSL), \textit{nj-} (EXP), and \textit{mj-} ‘try to’}\label{sec:9.1.1.1}

Yuwan has four aspectual auxiliary verbs: \textit{wur-} (PROG), \textit{ar-}/\textit{nə-} (RSL), \textit{nj-} (EXP), and \textit{mj-} ‘try to.’ First, we will discuss \textit{wur-}, which expresses the aspect of progressive, and \textit{ar-}/\textit{nə-}, which express the aspect of resultative (see \sectref{sec:8.5.1.5} - \sectref{sec:8.5.1.6} for their aspectual meanings). The auxiliary verbs that express the resultative aspect, i.e. \textit{ar-} and \textit{nə-}, are in the complementary distribution. \textit{nə-} (RSL) is always chosen immediately before the negative affixes, e.g. \textit{{}-an} (NEG). Otherwise, \textit{ar-} (RSL) is selected.

\ea   \textit{wur-} (PROG) \label{ex:9.7}
  \ea \label{ex:9.7a} [= (8-57 a)]\\
% TM:
   \gllll     cukutəə  wutakai?\\
      \textit{cukur-tɨ=ja}  \textit{\Highlight{wur}-tar=kai}\\
      make-SEQ=TOP  PROG-PST=DUB\\
      Lex. verb  Aux. verb\\
      \glt       ‘Was (anyone) making (cocoons)?’ [Co: 111113\_01.txt]

\ex \label{ex:9.7b} %TM:
    \gllll  mˀarɨtəə  wuijo.\\
      \textit{mˀarɨr-tɨ=ja}  \textit{\Highlight{wur}-i=joo}\\
      be.born-SEQ=TOP  PROG-NPST=CFM1\\
      Lex. verb  Aux. verb\\
      \glt       ‘(MY) was already born (at that time).’ [Co: 110328\_00.txt]

\ex \textit{ar-} (RSL)\label{ex:9.7c}\\
% TM:
    \gllll     gan  sjan  mun  utəə  aroojaa.\\
      \textit{ga-n}  \textit{sɨr-tar-n}  \textit{mun}  \textit{uw-tɨ=ja}  \textit{\Highlight{ar}-oo=jaa}\\
      MES-ADVZ  do-PST-PTCP  thing  plant-SEQ=TOP  RSL-SUPP=SOL\\
            Lex. verb  Aux. verb\\
      \glt       ‘Such a thing [i.e. a pear tree] has been planted (there), probably.’ [PF: 090222\_00.txt]
  
\ex \textit{nə-} (RSL) \label{ex:9.7d}\\
%    TM: 
    \gllll    {\textbar}nendai{\textbar}  kacjəə  nən?\\
      \textit{nendai}  \textit{kak-tɨ=ja}  \textit{\Highlight{nə}-an}\\
      date  write-SEQ=TOP  RSL-NEG\\
        Lex. verb  Aux. verb\\
      \glt       ‘Wasn’t the date (when the picture was taken) written (on it)?’ [Co: 111113\_01.txt]
    \z
\z

In (9-7 a-d), all of the lexical verbs are followed by the topic particle \textit{ja}. Additionally, other limiter particles (see \sectref{sec:10.1}), e.g. \textit{n} ‘even,’ \textit{bəi} ‘only,’ or \textit{du} (FOC), can appear between the lexical verb and the auxiliary verb. Interestingly, the nominative case \textit{ga/nu} can appear between the lexical verb and the auxiliary verb only when the auxiliary verb is \textit{nə-} (RSL) as in (9-8 a-c).

\ea   Lexical verb + \textit{ga}/\textit{nu} (NOM) + \textit{nə-} (RSL) \label{ex:9.8}
\ea %TM:
 \gllll  kacjɨga  nənbajaa.\\
      \textit{kak-tɨ=\Highlight{ga}}  \textit{\Highlight{nə}-an-ba=jaa}\\
      write-SEQ=NOM  RSL-NEG-CSL=SOL\\
      Lex. verb  Aux. verb\\
      \glt       ‘(The date when the picture was taken) was not written, so (we don’t know it).’ [Co: 120415\_00.txt]

\ex \label{ex:9.8b} %TM:
    \gllll  injasainkara  noogjoonkjaga ..  (ɨɨ)                   sjɨga  nənsjutɨga,\\
      \textit{inja-as+ar-i=n=kara}  \textit{noogjoo=nkja=ga}            \textit{sɨr-tɨ=\Highlight{ga}}  \textit{\Highlight{nə}-an=sjutɨ=ga}\\
      small-ADJ+STV-INF=DAT1=ABL  agriculture=APPR=NOM                  do-SEQ=NOM  RSL-NEG=SEQ=FOC\\
                                                                        Lex. verb  Aux. verb\\
      \glt       ‘Since (she) was young, (she) has never worked in the fields, and ...’ [Co: 120415\_01.txt]

\ex \label{ex:9.8c} %TM: 
     \gllll     zjenzjen  jinkjoodənkjanu  cɨkiai  sjɨnu                                           nənboo,\\
      \textit{zjenzjen}  \textit{jin+kjoodəə=nkja=nu}  \textit{cɨkiai}  \textit{sɨr-tɨ=\Highlight{nu}}          \textit{\Highlight{nə}-an-boo}\\
      very.much  [same+brother=APPR=GEN  acquaintance]  do-SEQ=NOM                                  RSL-NEG-CND\\
        [Complement]    Lex. verb                                                                   Aux. verb\\
      \glt       ‘If (people) have not made the acquaintance like brothers (of the) same (parents), ...’ [Co: 120415\_01.txt]
     \z
\z

The nominative case appears when \textit{nə-} (RSL) takes \textit{-ba} (CSL), \textit{{}-n=sjutɨ} (PTCPSEQ), or \textit{{}-boo} (CND) as in (9-8 a-c). This phenomenon seems to have some relationship with the occurence of the nominative case in the nominal predicate of the subordinate clause (see \sectref{sec:9.3.3.1}), since in both cases the occurence of \textit{ja} (TOP) is avoided within the predicate phrases, and instead the nominative case appears in the place where \textit{ja} (TOP) is expected. We have not yet found the reason for the choice between \textit{ga} (NOM) as in (9-8 a-b) and \textit{nu} (NOM) as in (9-8 c), but it seems that \textit{ga} (NOM) is somewhat preferred over \textit{nu} (NOM) in the texts. This fact seems to have some relationship with the preference of \textit{ga} (NOM) to \textit{nu} (NOM) before \textit{nə-} ‘exist’ (see \sectref{sec:6.4.3.5}).

In the modern Yuwan, I have seldom found the AVC of \textit{wur-} (PROG) and \textit{ar-} (RSL) without any intervening particle.\footnote{There is only an example where \textit{ar-} (RSL) is not preceded by any particle, and is not fused with the preceding lexical verb. /sjemenunkjoo ucjɨ aijaa/ \textit{sjemen=nkja=ja} \textit{\Highlight{ut-tɨ}} \textit{\Highlight{ar-i}=jaa} (cement=APPR=TOP \Highlight{pour-SEQ} \Highlight{RSL-NPST}=SOL) ‘Cement has been poured (there)’ [Co: 120415\_00.txt].} Instead, I found the affixes with the similar meanings, i.e. \textit{{}-tur} (PROG) and \textit{{}-təər} (RSL). Probably, \textit{{}-tur} (PROG) was made of *\textit{{}-tɨ} (SEQ) plus *\textit{wur-} (PROG), and \textit{{}-təər} (RSL) was made of *\textit{{}-tɨ} (SEQ) plus *\textit{ar-} (RSL), which is shown in \tabref{tab:93}.

\begin{table}
\caption{\label{tab:93}Grammaticalization of \textit{wur-} (PROG) and \textit{ar-} (\textbf{RSL})}
\begin{tabular}{lclcl}
\lsptoprule
\multicolumn{3}{c}{Supposed previous synchrony}  & &  \multicolumn{1}{c}{Modern synchrony}\\\cmidrule(lr){1-3}\cmidrule(lr){5-5}
Lexical verb  & &   Auxiliary verb & &  Stem + Affix\\\midrule
Stem + \textit{{}-\Highlight{tɨ}} (SEQ) & + & \textit{\Highlight{wur}{}-} (PROG) &  >  & Stem + \textit{{}-\Highlight{tur}} (PROG)\\
Stem + \textit{{}-\Highlight{tɨ}} (SEQ) & + & \textit{\Highlight{ar}{}-} (RSL)   &  >  & Stem + \textit{{}-\Highlight{təər}} (RSL)\\
\lspbottomrule
\end{tabular}
\end{table}

In other words, \textit{wur-} (PROG) and \textit{ar-} (RSL) show much progress in the grammaticalization channels in the cases of \textit{{}-tur} (PROG) and \textit{{}-təər} (RSL) (cf. \citealt{Lehmann1995}: 37). Interestingly, \textit{nə-} (RSL) is always preceded by some particle, and there is no example where \textit{-tɨ} (SEQ) appears to be fused with \textit{nə-} (RSL). This seems to have some relationship with the fact that there is always a particle, i.e. \textit{ja} (TOP), before the negated copula verb (see (9-54) in \sectref{sec:9.3.1}). I will present examples of \textit{{}-tur} (PROG) and \textit{{}-təər} (RSL) below.

\ea   Grammaticalized auxiliary verbs \label{ex:9.9}

\exi{} \textit{{}-tur} (PROG)
\ea %TM:
 \glll  kunugurugadɨ  (kun ..)  unnantɨ    cukututanmundoojaa.\\
      \textit{kunuguru=gadɨ}  \textit{ku-n}  \textit{u-n=nantɨ}       \textit{cukur-\Highlight{tur}-tar-n=mun=doo=jaa}\\
      recently=LMT  PROX-ADNZ  MES-ADNZ=LCO2    make-PROG-PST-PTCP=ADVRS=ASS=SOL\\
      \glt       ‘(They) used to do dyeing until recently there.’ [Co: 111113\_01.txt]

  \ex\relax [Context: TM is talking about the meeting for old people held once a month in Yuwan.] = (8-136 a)\\
    %TM:
     \glll      taruka  tˀaibəi  wututɨ,  kan   sjan  hanasinkja  sɨrarɨppoo,  jiccjanban,\\
      \textit{ta-ru=ka}  \textit{tˀai=bəi}  \textit{\Highlight{wur-tur}-tɨ}  \textit{ka-n} \textit{sɨr-tar-n}  \textit{hanasi=nkja}  \textit{sɨr-arɨr-boo}  \textit{jiccj-sa+ar-n=ban}\\
      who-NLZ=DUB  two.CLF.person=about  exist-PROG-SEQ  PROX-ADVZ   do-PST-PTCP  conversation=APPR  do-CAP-CND  good-ADJ+STV-PTCP=ADVRS   \\
      \glt       ‘(It) will be good if some two (or three) people are being (there) and can make conversation like this, but ...’ [Co: 120415\_01.txt]

\exi{} \textit{{}-təər} (RSL)

\ex \label{ex:9.9c} %TM:
     \glll     kurəə  nuucjɨga  kacjəəru?\\
      \textit{ku-rɨ=ja}  \textit{nuu=ccjɨ=ga}  \textit{kak-\Highlight{təər}-u}\\
      PROX-NLZ=TOP  what=QT=FOC  write-RSL-PFC\\
      \glt       ‘What is written (on) this?’ [Co: 120415\_00.txt]

\ex \label{ex:9.9d} % TM: 
      \glll  umaga  atəkkamojaa.\\
      \textit{u-ma=ga}  \textit{\Highlight{ar-təər}=kamo=jaa}\\
      MES-place=FOC  exist-RSL=POS=SOL\\
      \glt       ‘(The chamber of commerce) may have been there.’ [lit. ‘(At) that place, (the chamber of commerce) may have existed.’] [Co: 120415\_00.txt]

\ex \label{9.9e} %  TM:
      \glll ziisanna  mata  {\textbar}iciban  monosiri{\textbar}  jatəəppa,  waakjaa  anmaaja  utaja  (mm)  uraa  ziisan  məəradu  naratancjɨ  jutattujaa.\\
      \textit{ziisan=ja}  \textit{mata}  \textit{iciban}  \textit{monosiri}  \textit{\Highlight{jar-təər}-ba}  \textit{waakja-a}  \textit{anmaa=ja}  \textit{uta=ja}    \textit{ura-a}   \textit{ziisan} \textit{məə=kara=du}  \textit{naraw-tar-n=ccjɨ}  \textit{jˀ-tar-tu=jaa}\\
      grandfather=TOP  again  most  well.informed.person  COP-RSL-CSL  1PL-ADNZ  mother=TOP  song=TOP    2.NHON.SG-ADNZ   grandfather  front=ABL=FOC  learn-PST-PTCP=QT  say-PST-CSL=SOL\\      
      \glt  ‘(Your) grandfather was the most well-informed person, so my mother said that (she) learned (the traditional) songs from your grandfather.’ [Co: 120415\_01.txt]
    \z
\z

The details of the aspectual meanings of the above auxiliary verbs, i.e. \textit{wur-} (RPOG) and \textit{ar-}/\textit{nə-} (RSL), and their grammaticalized affixes has been discussed in \sectref{sec:8.5.1.5} - \sectref{sec:8.5.1.6}. Interestingly, the grammaticalized affixes \textit{{}-tur} (PROG) and \textit{{}-təər} (RSL) can follow their original lexical counterparts, i.e. \textit{wur-} ‘exist (animate)’ and \textit{ar-} ‘exist (inanimate)’ as in (9-9 b, d). On the contrary, combinations such as the lexical verb \textit{wur-} ‘exist (animate)’ followed by the auxiliary verb \textit{wur-} (PROG), or the lexical verb \textit{ar-} ‘exist (inanimate)’ followed by the auxiliary verb \textit{ar-} (RSL) in the AVCs have not yet been found in the text corpus, and it is difficult to make a question that will bring about forms such as these in elicitation. Thus, the existence of the combinatins as in (9-9 b, d) expresses that the affixes, i.e. \textit{{}-tur} (PROG) and \textit{{}-təər} (RSL), have come to be used in new contexts, and it is a proof of grammaticalization (cf. Heine\& \citealt{Kuteva2002}: 2). Furthermore, there is a combination of \textit{jar-} (COP) and \textit{{}-təər} (RSL) as in (9-9 e), which has never been realized in the form of the AVC, i.e, there is no combination such as \textit{jar-tɨ} (COP-SEQ) plus \textit{ar-} (RSL). This fact also supports the analysis that \textit{{}-təər} (RSL) is an independent affix in the modern Yuwan, and that it is not derived from the “synchronic” fusion of \textit{{}-tɨ} (SEQ) and \textit{{}-ar} (RSL). Considering the behavior of \textit{{}-təər} (RSL) as such, and the irregular reduction and assimilation of morphophonemes between the lexical verb and the auxiliary verb as in \tabref{tab:93}, it is appropriate to regard \textit{{}-tur} (PROG) and \textit{{}-təər} (RSL) as members of the verbal affixes in modern Yuwan (see \chapref{chap:8}).

Secondly, we will discuss another auxiliary verb \textit{nj-} (EXP), which expresses the aspect of the experiential perfect. If \textit{nj-} (EXP) is followed by \textit{{}-i} (NPST) or \textit{{}-an} (NEG), it means that the event has occured at least once or has never occured in the past leading up to the present (cf. \citealt{Comrie1976}: 58-59) as in (9-10 a-c). If \textit{nj-} (EXP) is followed by \textit{{}-ɨ} (IMP) or \textit{{}-oo} (INT), it means that the event will be experienced by the agent at least once during the recent future. In that case, it is appropriate to translate \textit{nj-} (EXP) into ‘try to’ as in (9-10 d-e). Interestingly, \textit{nj-} (EXP) cannot be followed by \textit{{}-na} (PROH), which is the negative counterpart of \textit{{}-ɨ} (IMP).

\ea   \textit{nj-} (EXP) \label{ex:9.10}

\ea %TM:
 \gllll  asɨdɨn  njan.jaa.\\
      \textit{asɨb-tɨ=n}  \textit{\Highlight{nj}-an=jaa}\\
      play-SEQ=ever  EXP-NEG=SOL\\
      Lex. verb  Aux. verb\\
      \glt       ‘(We) have never played (together), (have we?)’ [Co: 110328\_00.txt]

\ex \label{ex:9.10b} %TM: 
      \gllll    nudɨn  njui?\\
      \textit{num-tɨ=n}  \textit{\Highlight{nj}-jur-i}\\
      drink-SEQ=ever  EXP-UMRK-NPST\\
      Lex. verb  Aux. verb\\
      \glt       ‘Have (you) ever drunk (it)?’ [El: 120926]

\ex \label{ex:9.10c} % TM: 
     \gllll an  tacɨgəə  cˀjukəəin  toorɨtɨn  njan.\\
      \textit{a-n}  \textit{tacɨgɨ=ja}  \textit{cˀjukəəi=n}  \textit{toorɨr-tɨ=n}  \textit{\Highlight{nj}-an}\\
      DIST-ADNZ  prop=TOP  one.CLF.time=even  fall-SEQ=ever  EXP-NEG\\
            Lex. verb  Aux. verb\\
      \glt       ‘That prop has never fallen even once.’ [El: 130816]

\ex \label{ex:9.10d} %  TM: 
   \gllll   ude,  kun  nɨkan  kadɨn  njɨ!\\
      \textit{ude}  \textit{ku-n}  \textit{nɨkan}  \textit{kam-tɨ=n}  \textit{\Highlight{nj}-ɨ}\\
      well  PROX-ADNZ  mikan  eat-SEQ=ever  EXP-IMP\\
            Lex. verb  Aux. verb\\
      \glt       ‘Well, try to eat this \textit{mikan}!’ [Co: 101023\_01.txt]

\ex \label{ex:9.10e} % TM: 
   \gllll naa  məəci  cˀjɨn  njoojəəcjɨ    jˀicjattu,\\
      \textit{naa-a}  \textit{məə=kaci}  \textit{k-tɨ=n}  \textit{\Highlight{nj}-oo=jəə=ccjɨ}   \textit{jˀ-tar-tu}\\
      2.HON.SG=ADNZ  front=ALL  come-SEQ=ever  EXP-INT=CFM2=QT  say-PST-CSL\\
          Lex. verb  Aux. verb\\          
      \glt ‘(The person) said, “(I) will try to come to your place,” so ...’    [Co: 120415\_00.txt]
    \z
\z

In (9-10 a-e), \textit{nj-} (EXP) is necessarily preceded by \textit{n} ‘ever.’ In fact, \textit{nj-} (EXP) is always preceded by \textit{n} ‘ever’ in my texts. In other words, there seems to be no morpheme boundary between \textit{n} ‘ever’ and \textit{nj-} (EXP). I do not, however, regard them as a single morpheme such as \textit{nnj-} (EXP), since there is an example as in (9-11).

\ea    \label{ex:9.11}
\ea\label{ex:9.11a} % TM:  
     \gllll kicjɨn  mjicjɨn  njanmun.  ...  ukka  ujankjanu,  ude,\\
      \textit{kik-tɨ=\Highlight{n}}  \textit{mj-tɨ=\Highlight{n}}  \textit{\Highlight{nj}-an=mun}    \textit{u-rɨ=ga}    \textit{uja=nkja=nu}  \textit{ude}\\
      hear-SEQ=ever  see-SEQ=ever  EXP-NEG=ADVRS    MES-NLZ=NOM  parent=APPR=NOM  well\\
      {Lex. verb}  {Lex. verb}  {Aux. verb}\\
    \glt      ‘(I) have never heard of or seen (him). That person’s parent was, ...’

\ex\label{ex:9.11b} %MS:  
      \glll jaa.\\
      \textit{jaa}\\
      FIL\\
    \glt  ‘Yeah.’  

\ex \label{ex:9.11c} %TM:
     \gllll     kicjɨn  mjicjɨn  ...  \\
      \textit{kik-tɨ=\Highlight{n}}  \textit{mj-tɨ=\Highlight{n}}    \\
      hear-SEQ=ever  see-SEQ=ever    \\
      {Lex. verb}  {Lex. verb}    \\
      \glt       ‘(I have never) heard of or seen ...’   [Co: 120415\_01.txt]
    \z
\z

The above example is a sequence of a conversation. In (9-11 a, c), \textit{n} ‘ever’ attaches to the initial lexical verb (not only to the lexical verb immediately before \textit{nj-} (EXP)), i.e. \textit{kik-tɨ=\Highlight{n}} \textit{mj-tɨ=n} (hear-SEQ=ever see-SEQ=ever). Additionally, the initial sentence of (9-11 a) is partially repeated in (9-11 c), where the utterance-final \textit{n} ‘ever’ attaches to the lexical verb without \textit{nj-} (EXP), i.e. \textit{mj-tɨ=\Highlight{n}} (see-SEQ=ever). Thus, I propose that \textit{n} ‘ever’ can be divided from the auxiliary verb \textit{nj-} (EXP), although their unity is very strong.

  Finally, I will present examples of \textit{mj-} ‘try to.’

\ea   \textit{mj-} ‘try to’ \label{ex:9.12}
\ea \label{ex:9.12a}%TM:
 \gllll  attaatun  hanacjɨ  mjicjɨn  njanban,\\
      \textit{a-rɨ-taa=tu=n}  \textit{hanas-tɨ}  \textit{\Highlight{mj}-tɨ=n}  \textit{nj-an=ban}\\
      DIST-NLZ-PL=COM=also  talk-SEQ  try.to-SEQ=ever  EXP-NEG=ADVRS\\
        {Lex. verb}  {Aux. verb}  {Aux. verb}\\
      \glt       ‘(I) have never tried to talk with that person, but ...’ [Co: 120415\_01.txt]

\ex \label{ex:9.12b} % TM:  
     \gllll  cˀjɨ  mjoojəə.\\
      \textit{k-tɨ}  \textit{\Highlight{mj}-oo=jəə}\\
      come-SEQ  try.to-INT=CFM2\\
      {Lex. verb}  {Aux. verb}\\
      \glt       ‘(I) will try to come (here).’ [El: 120929]
     \z
\z

The meaning of \textit{mj-} ‘try to’ is partially similar to \textit{nj-} (EXP); compare (9-12 a-b) to (9-10 d-e). \textit{mj-} ‘try to’ does not need to be preceded by \textit{n} ‘any,’ which is different from \textit{nj-} (EXP).

  Many of the aspectual AVs are in a diachronic change of grammticalization. \textit{wur-} (PROG) and \textit{ar-}/\textit{nə-} (RSL) have their lexical counterparts, i.e. \textit{wur-} ‘exist (animate)’ and \textit{ar-}/\textit{nə-} ‘exist (inanimate)’ (see \sectref{sec:8.3.2} for more details about these existential verbs). The lexical counterpart of \textit{mj-} ‘try to’ is \textit{mj-} ‘see’ as in (6-122 a-b) in \sectref{sec:6.4.3.3}. There is no lexical counterpart of \textit{nj-} (EXP) (see note (a) of \tabref{tab:92}).

\subsubsection{Honorific auxiliary verb: \textit{moor-} (HON)}\label{sec:9.1.1.2}

The auxiliary verb \textit{moor-} expresses the speaker’s respect for the subject of the predicate (see also chapter 3 about the grammatical relations). Other honorific AVs, i.e. \textit{taboor-} (BEN.HON) and \textit{umoor-} (come.HON), are discussed in \sectref{sec:9.1.1.3} and \sectref{sec:9.1.1.4} respectively. I will present an example of \textit{moor-} (HON).

\ea   \textit{moor-} (HON) \label{ex:9.13}\\
%TM:
 \gllll  minna  {\textbar}gakkoo{\textbar}  izjacjɨ  moocjəppajaa.\\
    \textit{minna}  \textit{gakkoo}  \textit{izj-as-tɨ}  \textit{\Highlight{moor}-təər-ba=jaa}\\
    everybody  school  go.out-CAUS-SEQ  HON-RSL-CSL=SOL\\
        Lex. verb  Aux. verb\\
    \glt     ‘(Your great-grandparents) had all of (their chidren) go out [i.e. graduate from] the school.’ [Co: 120415\_01.txt]
\z

In (9-13), the lexical verb takes \textit{{}-tɨ} (SEQ) before the auxiliary verb \textit{moor-} (HON). The honorific AVC expresses the speaker’s respect for the subject of the clause, i.e. for the hearer’s great-grandparents. For more details about the auxiliary honorific verbs, see \sectref{sec:8.3.1.2}.

\subsubsection{Valency-changing auxiliary verbs: \textit{kurɨr-} (BEN), \textit{muraw-} (BEN), and \textit{taboor-} (HON.BEN)}\label{sec:9.1.1.3}

The auxiliary verbs \textit{kurɨr-} (BEN), \textit{muraw-} (BEN), and \textit{taboor-} (HON.BEN) increase the semantic valency of the predicates. Additionally, only \textit{muraw-} can change the syntactic valency. The semantic valency relates to the number of participant semantically required by the predicate of a clause. The syntactic valency relates to the morphosyntactic means (especially, case markers) to express the participants. I borrow those of \citet[169-173]{Payne1997} regarding the terms of the semantic valency and syntactic valency.

Semantically, these valency-changing auxiliary verbs add a beneficiary as a participant of the event indicated by the clause. In many cases, the added beneficiary is the speaker, but it can be a referent to whom the speaker “empathize” with (cf. \citealt{Kuno1987}: 206). The differences among these valency-changing auxiliary verbs are determined by the correspondence between the subject and the referent that causes or receives the benefaction. In other words, if the VP’s subject is the benefactor, \textit{kurɨr-} (BEN) or \textit{taboor-} (BEN.HON) is used. If the VP’s subject is the beneficiary, \textit{muraw-} (BEN) is used. These are summarized below.

\ea   Principle of the use of the valency-changing auxiliary verbs \label{ex:9.14}
  \ea Subject = Benefactor\\
    \textit{kurɨr-} (BEN) or \textit{taboor-} (BEN.HON)
  \ex Subject = Beneficiary\\
    \textit{muraw-} (BEN)
    \z
\z

    
  First, I will present the example of \textit{kurɨr-} (BEN).

\ea   \textit{kurɨr-} (BEN): the subject is the benefactor\label{ex:9.15}\\
%TM:
 \gllll  uran  jazin  kjunmuncjɨ  dooka  umutɨ  kurɨranboo.\\
    \textit{\Highlight{ura}=n}  \textit{jazin}  \textit{k-jur-n=mun=ccjɨ}  \textit{dooka}  \textit{umuw-tɨ}  \textit{\Highlight{kurɨr-an-boo}}\\
    2.NHON.SG=also  necessarily  come-UMRK-PTCP=ADVRS=QT  please  think-SEQ  BEN-NEG-CND\\
    Subject/Benefactor                                             Lex. verb  Aux. verb\\
    \glt     ‘If you don’t think that (you) will necessarily come (here for me, I will run into a problem).’ [Co: 101023\_01.txt]
\z

In (9-15), the subject of the VP /umutɨ kurɨranboo/ \textit{umuw-tɨ} \textit{kurɨr-an-boo} (think-SEQ BEN-NEG-CND) ‘if (you) don’t think (of it for me)’ is \textit{ura} (2.NHON.SG) ‘you,’ who is the subject the clause and also the benefactor of the event. The beneficiary is the speaker TM.

  Secondly, the auxiliary verb \textit{taboor-} (BEN.HON) is the honorific counterpart of \textit{kurɨr-} (BEN). Thus, it can also be used when the benefactor of the event is the subject of the clause.

\ea   \textit{taboor-} (BEN.HON): the subject is the benefactor [= (8-26)]\label{ex:9.16}\\
%TM:
 \gllll  {\textbar}sinsjei{\textbar},  an  kˀwa  abɨtɨ  taboorɨ.\\
    \textit{\Highlight{sinsjei}}  \textit{a-n}  \textit{kˀwa}  \textit{abɨr-tɨ}  \textit{\Highlight{taboor}-ɨ}\\
    teacher  DIST-ADNZ  child  call-SEQ  BEN.HON-IMP\\
    Subject/Benefactor      Lex. verb  Aux. verb\\
    \glt     ‘Teacher, would (you) please call that child (for me)?’ [El: 130820]
\z

In (9-16), the subject of the VP /abɨtɨ taboorɨ/ \textit{abɨr-tɨ} \textit{taboor-ɨ} (call-SEQ BEN.HON-IMP) ‘Would (you) please call (that child)?’ is \textit{sinsjei} ‘teacher,’ who is the subject the clause and also the benefactor of the event. The beneficiary is the speaker TM. Additionally, \textit{taboor-} (BEN.HON) expresses the speaker’s respect for the subject of the clause, i.e. \textit{sinsjei} ‘teacher.’

Finally, I will present examples of \textit{muraw-} (BEN).

\ea   \textit{muraw-} (BEN): the subject is the beneficiary\label{ex:9.17}\\
\gllll  US:  umantɨ  ɨrɨtɨ  muratanbanga,  {\textbar}moo  zenzen{\textbar}  ooran.    \\
    \textit{u-ma=nantɨ}  \textit{ɨrɨr-tɨ}  \textit{\Highlight{muraw}-tar-n=ban=ga  moo}   \textit{zenzen}  \textit{oor-an}    \\
    MES-place=LOC2  put.in-SEQ  BEN-PST-PTCP=ADVRS=FOC  FIL   much  fit-NEG    \\
      Lex. verb  Aux. verb     \\
    \glt     ‘(I) had (the dentist) put in (the artificial teeth), but (it) does not fit (me) very much.’ [Co: 110328\_00.txt]
\z

In (9-17), the subject of the VP /ɨrɨtɨ muratan/ \textit{ɨrɨr-tɨ} \textit{muraw-tar-n} (put.in-SEQ BEN-PST-PTCP) ‘having had (the dentist) put in (the artificial teeth)’ is the speaker, and she is also the beneficiary of the event, although she is not overtly expressed in (9-17). An example that is more understandable is shown below, where two sentences are compared. The first example is a minimal VP that does not include \textit{muraw-} (BEN). The second example is an AVC, where the lexical verb in the first example, i.e. \textit{kak-} ‘write,’ is followed by \textit{muraw-} (BEN).

\ea   Valency changing of \textit{muraw-} (BEN) \label{ex:9.18}
  \ea Non-derived sentence (Minimal VP)\label{ex:9.18a}\\
% TM: 
\glll     an  cˀjuga  kakjui.\\
      \textit{a-n}  \textit{cˀju=\Highlight{ga}}  \textit{kak-jur-i}\\
      DIST-ADNZ  person=NOM  write-UMRK-NPST\\
      \glt       ‘That person will write (it).’ [El: 130822]

  \ex  Derived sentence (AVC)\label{ex:9.18b}\\
%     TM:  
      \gllll wanna  an  cˀjun  kacjɨ   murawoojəə.\\
      \textit{\Highlight{wan}=ja}  \textit{a-n}  \textit{cˀju=\Highlight{n}}  \textit{kak-tɨ}      \textit{\Highlight{muraw}-oo=jəə}\\
      1SG=TOP  DIST-ADNZ  person=DAT1  write-SEQ   BEN-INT=CFM2\\
      Subject/Beneficiary Benefactor  Lex. verb  Aux. verb \\      
      \glt ‘I will have that person write (it for me).’       [El: 130822]
    \z
\z

In (9-18 a), the participant of the event is only one, i.e. /an cˀju/ ‘that person.’ In (9-18 b), another participant, i.e. \textit{wan} (1SG), is added to the event of (9-18 a). The added participant is the subject of the clause and also the beneficiary of the event. Furthermore, \textit{muraw-} (BEN) changes the syntactic valency of the predicate. That is, it changes the coding of the case particle. In (9-18 a), the agent of \textit{kak-} ‘write’ is marked by \textit{ga} (NOM), but in (9-18 b), the agent of \textit{kak-} ‘write,’ who is also the benefactor of the event, is marked by \textit{n} (DAT1).

  Before concluding this section, I will present the lexical counterparts of the above valency-changing auxiliary verbs.

\ea   Lexical counterparts of the valency-changing AVs \label{ex:9.19}
\ea \textit{kurɨr-} ‘give’\\
% TM: 
   \gllll miicɨ  kurɨtattoo,  un  micjaija  jurukudɨ, kan  sjɨ  hucjutɨ,  kadɨ,  ikii.\\
    \textit{miicɨ}  \textit{\Highlight{kurɨr}-tattoo  u-n  micjai=ja  jurukub-tɨ}  \textit{ka-n}  \textit{sɨr-tɨ}  \textit{huk-tur-tɨ}  \textit{kam-tɨ}  \textit{ik-i}\\
    three.CLF  give-PST.CSL  MES-ADNZ  three.CLF=TOP  be.pleased-SEQ  PROX-ADVZ  do-SEQ  wipe-PROG-SEQ  eat-SEQ  go-INF\\
      Lex. Verb    \\
    \glt  ‘When (the boy) gave three (pears to the three boys), the three (boys) were pleased, and were wiping (the pears) like this, and ate (them), and went (away).’   [PF: 090827\_02.txt]

\ex \textit{muraw-} ‘receive’\\
%TM:
 \gllll  nasinu  miicɨ  muratɨ,\\
    \textit{nasi=nu}  \textit{miicɨ}  \textit{\Highlight{muraw}-tɨ}\\
    pear=GEN  three.thing  receive-SEQ\\
        Lex. Verb\\
    \glt     ‘(They) received three pears, and ...’ [PF: 090225\_00.txt]
  \z
\z

In (9-19 a-b), both of the lexical verbs, i.e. \textit{kurɨr-} ‘give’ and \textit{muraw-} ‘receive,’ express the locomotion of concrete things, i.e. ‘pears.’ On the contrary, the examples of the valency-changing auxiliary verbs as in (9-15) or (9-17) do not express such locomotion of things. Thus, the so-called “semantic bleaching” (\citealt{HopperTraugott2003}: 94) has happened in these auxiliary verbs. Interestingly, \textit{taboor-} (BEN.HON) does not have its lexical counterpart. That is, it is not used to fill the lexical verb slot. If we want to mean ‘give’ with the honorific meaning, we may use an AVC where the lexical verb slot is filled by \textit{kurɨr-} ‘give’ and the auxiliary verb slot is filled by \textit{taboor-} (BEN.HON), e.g. /kurɨtɨ taboorɨ/ \textit{kurɨr-tɨ} \textit{taboor-ɨ} (give-SEQ BEN.HON-IMP) ‘Would you please give (it for me)?’

\subsubsection{Spatial deictic auxiliary verbs: \textit{ik-} ‘go,’ \textit{k-} ‘come,’ and \textit{umoor-} (go/come.HON)}\label{sec:9.1.1.4}

Yuwan has three spatial deictic auxiliary verbs: \textit{ik-} ‘go,’ \textit{k-} ‘come,’ and \textit{umoor-} (go/come.HON). The example of \textit{umoor-} (come.HON) was already shown in (8-27) in \sectref{sec:8.3.1.2}. I will present examples of \textit{ik-} ‘go’ and \textit{k-} ‘come.’

\ea  
\exi{}\textit{ik-} ‘go’ \label{ex:9.20}
\ea %TM:
   \gllll   kun  {\textbar}nimocu{\textbar}  muccjɨ  ikii.\\
      \textit{ku-n}  \textit{nimocu}  \textit{mut-tɨ}  \textit{\Highlight{ik}-i}\\
      PROX-ADNZ  load  have-SEQ  go-INF\\
          Lex. verb  Aux. verb\\
      \glt       ‘(They) take this load.’ [lit. ‘(They) have this load and go.’] [Co: 120415\_00.txt]
\ex 
% TM:
\gllll   uroo  {\textbar}okazu{\textbar}ja  ...  muccjɨ  ikjan?\\
      \textit{ura=ja}  \textit{okazu=ja}    \textit{mut-tɨ}  \textit{\Highlight{ik}-an}\\
      2.NHON.SG=TOP  side.dish=TOP    have-SEQ  go-NEG\\
            Lex. verb  Aux. verb\\
      \glt       ‘Don’t you take the side dish?’ [lit. ‘Don’t you have the side dish and go?’] [Co: 120415\_01.txt]

\exi{} \textit{k-} ‘come’

\ex %TM: 
\gllll  TM:  naa,  cjuutokara  mata  wunagunu  kˀwanu  {\textbar}zitensja{\textbar} nutɨ  cˀjattuu,\\
      \textit{naa}  \textit{cjuuto=kara}  \textit{mata}  \textit{wunagu=nu}  \textit{kˀwa=nu}  \textit{zitensja}  \textit{nur-tɨ}  \textit{\Highlight{k}-tar-tu}\\
      FIL  middle=ABL  again  woman=GEN  child=NOM  bicycle    ride-SEQ  come-PST-CSL\\
                                                             Lex. verb  Aux. verb\\
      \glt       ‘(At) the middle (of the film), a girl came riding a bicycle, and then ...’ [PF: 090305\_01.txt]

\ex\relax [Context: An old man found gold under the ground, but he did not bring it home, so his wife was surprised to hear that.] = (6-55 c)\\
% TM:
    \gllll gan  jiccjan  mun  həəku  tutɨ  konboo,  cˀjun  tɨmɨrarɨɨdoocjɨ  jˀicjanmun,\\
      \textit{ga-n}  \textit{jiccj-sa+ar-n}  \textit{mun}  \textit{həə-ku}  \textit{tur-tɨ} \textit{\Highlight{k}-on-boo  cˀju=n  tɨmɨr-arɨr=doo  jˀ-tar-n=mun}\\
      MES-ADVZ  good-ADF+STV-PTCP  thing  early-ADVZ  take-SEQ  come-NEG-CND  person=DAT1  find-PASS.INF=ASS  say-PST-PTCP=ADVRS\\
              Lex. verb                                          Aux. verb \\              
      \glt ‘(The wife) said, “If you don’t bring such a good thing, (it) will be found by another person,” but ...’   [Fo: 090307\_00.txt]
    \z
\z

In (9-20 a-d), all of the \textit{ik-} ‘go’ and \textit{k-} ‘come’ fill the auxiliary verb slot. In fact, \textit{ik-} ‘go’ and \textit{k-} ‘come’ can fill the lexical verb slot, and their auxiliary uses do not show any morphophonemic reduction or semantic change. However, they can really fill the auxiliary verb slot. For example, in (9-20 b, d), the semantic scope of negation of \textit{{}-an}/\textit{{}-on} (NEG) includes the preceding lexical verbs (not only the auxiliary verbs), which means they are mono-clausal. In other words, \textit{ik-} ‘go’ and \textit{k-} ‘come’ are filling the auxiliary verb slots in the mono-clausal VPs.

Before concluding this section, I will present an example of the combination of two auxiliary verbs.

\ea   \textit{ik-} ‘go’ + \textit{kurɨr-} (BEN) \label{ex:9.21}\\
%TM:
 \gllll  muccjɨ  izjɨ  kurɨppa.\\
    \textit{mut-tɨ}  \textit{\Highlight{ik}-tɨ  \Highlight{kurɨr}-ba}\\
    have-SEQ  go-SEQ  BEN-CSL\\
    Lex. verb  Aux. verb  Aux. verb\\
    \glt     ‘Please take (the lunch boxes).’ [lit. ‘Please have (the lunch boxes) and go (for me).’] [Co: 120415\_01.txt]
\z

The above example shows that the spatial deictic auxiliary verb can precede the valency-changing auxiliary verb.

\subsection{Light verb construction}\label{sec:9.1.2}

The light verb construction (LVC) is composed of the light verb and its complement (plus an optional auxiliary verb) as in the following model.

\ea   Light verb construction (LVC) \label{ex:9.22}\\
  \{Complement \hspace{\tabcolsep} [Light verb \hspace{\tabcolsep} (Auxiliary verb)]\textsubscript{VP}\}\textsubscript{Verbal predicate phrase}
\z

The LVC minimally consists of the light verb and its complement. Additionally, since the light verb fills the lexical verb slot of an VP, it may be followed by an auxiliary verb forming an auxiliary verb construction within the VP.

Yuwan has two kinds of light verbs, which are all semantically “light” and need thier complements. First, there is the light verb \textit{sɨr-} ‘do,’ whose complement slot may be filled by NPs, verbs, adjectives, and adverbs (see \sectref{sec:9.1.2.1} for more details). The second light verb is \textit{nar-} ‘become,’ whose complement slot is filled by NPs, adverbs, the participle that ends with \textit{{}-an} (NEG), or the converbs that end with \textit{{}-an-ba} (NEG-CSL) or \textit{{}-an-boo} (NEG-CND) (see \sectref{sec:9.1.2.2} for more details).

\subsubsection{\textit{sɨr-} ‘do’}\label{sec:9.1.2.1}

The verb \textit{sɨr-} ‘do’ is semantically so “light” that it usually needs its complement to fill the predicate slot of a clause, unless it takes its own argument as in /denwaba sjui/ \textit{denwa=ba} \textit{sɨr-jur-i} (telephone=ACC do-UMRK-NPST) ‘call [lit. do the telephone].’ In fact, there is an example of \textit{sɨr-} ‘do’ without any component as in (9-37) in \sectref{sec:9.1.2.2}, although it occured in elicitation.

The complement slot of \textit{sɨr-} ‘do’ can be filled by the following components.

\ea  Complements of \textit{sɨr-} ‘do’ may be filled by, \label{ex:9.23}
  \ea  common nouns;
  \ex  infinitives;
  \ex  the finite form \textit{{}-oo} (INT) followed by \textit{ccjɨ} (QT);
  \ex  the converb \textit{{}-tai} (LST);
  \ex  the compound including \textit{madəə} ‘fail to’;
  \ex  demonstrative adverbs;
  \ex  adverbs derived from adjectival stems;
  \ex  adjectives;
  \ex  the units followed by \textit{nən} ‘such as.’
  \z
\z

With regard to (9-23 a), I will present examples where commoun nouns fill the complement slot of \textit{sɨr-} ‘do.’

\ea   Complements filled by common nouns \label{ex:9.24}
\ea\relax [Context: Speaking with MY about the present author]\\
% TM:  
 \gllll    {\textbar}benkjoo{\textbar}  sjun  cˀjunkjaccjɨboo,  gan  sjɨ  sjutɨ,  {\textbar}benkjoo{\textbar}  sii  jappajaa.\\
      \textit{\Highlight{benkjoo}}  \textit{\Highlight{sɨr}-jur-n  cˀju=nkja=ccjɨboo  ga-n}   \textit{sɨr-tɨ}  \textit{sɨr-jur-tɨ}  \textit{\Highlight{benkjoo}}  \textit{\Highlight{sɨr}-i  jar-ba=jaa}\\
      study  do-UMRK-PTCP  person=APPR=speaking.of  MES-ADVZ  do-SEQ  do-UMRK-SEQ  study  do-INF  COP-CSL=SOL\\
      Complement  LV    Complement                            LV    Complement  LV  \\
      \glt       ‘Speaking of a person who does studies, (the one) does studying like that, you know.’ [Co: 101023\_01.txt]

\ex %TM:
\gllll   {\textbar}kokkei{\textbar}  sjutɨ,  waroocja.\\
      \textit{\Highlight{kokkei}}  \textit{\Highlight{sɨr}{}-tur-tɨ  waraw-as-tar}\\
      funny  do-PROG-SEQ  laugh-CAUS-PST\\
      Complement  LV  \\
      \glt       ‘(He) did funny things, and made (people) laugh.’ [Co: 120415\_00.txt]

\ex\relax  [= (8-61 a)]\\
%TM:
    \gllll  namanu  usi  sjurooga?\\
      \textit{nama=nu}  \textit{\Highlight{usi}}  \textit{\Highlight{sɨr}-jur-oo=ga}\\
      now=GEN  cow  do-UMRK-SUPP=CFM3\\
        Complement  LV\\
      \glt       ‘Now (someone) raises cows, doesn’t he?’ [Co: 111113\_01.txt]

\ex\relax [= (6-65 b)]\\
%TM:
    \gllll  uroo  jaanantɨ  nusisjɨ  hanməə  sjɨ,  kamii?\\
      \textit{ura=ja}  \textit{jaa=nantɨ}  \textit{nusi=sjɨ}  \textit{\Highlight{hanməə}}  \textit{\Highlight{sɨr}-tɨ  kam-i}\\
      2.NHON.SG=TOP  house=LOC2  RFL=INST  cooking  do-SEQ  eat-INF\\
            Complement  LV  \\
      \glt       ‘You do cooking by yourself, and eat (the meal) at home?’ [Co: 120415\_01.txt]
    \z
\z

In (9-24 a-d), the common nouns \textit{benkjoo} ‘study,’ \textit{kokkei} ‘funny (action),’ \textit{usi} ‘cow,’ and \textit{hanməə} ‘cooking’ fill the complement slots of each example.

With regard to (9-23 b), the examples where the infinitive fill the complement slot of \textit{sɨr-} ‘do’ are shown (see \sectref{sec:8.4.4} for more details on the infinitive).

\ea   Complements filled by the infinitive \label{ex:9.25}
\ea \label{ex:9.25a}%TM:
 \gllll  hainu  tubəə  sjunban,  janakɨsaccjɨn  nuucjɨn  umuwanbajaa.  mukasjəə.\\
      \textit{hai=nu}  \textit{\Highlight{tub-i}=ja}  \textit{\Highlight{sɨr}-jur-n=ban  janakɨ-sa=ccjɨ=n}    \textit{nuu=ccjɨ=n}  \textit{umuw-an-ba=jaa}  \textit{mukasi=ja}\\
      ash=NOM  fly-INF=TOP  do-UMRK-PTCP=ADVRS  dirty-ADJ=QT=even  what=QT=even  think-NEG-CSL=SOL  past=TOP\\
        Complement  LV      \\
    \glt       ‘In the old days, the ash (of the cooking stove) was flying, but (I) didn’t think of it as dirty.’ [Co: 111113\_02.txt]

\ex \label{ex:9.25b} %TM:
    \gllll  nuuga?  kurɨ  kurɨ.  kusarəə  sɨranba,   jiccjaijo.\\
      \textit{nuu=ga}  \textit{ku-rɨ}  \textit{ku-rɨ}  \textit{\Highlight{kusarɨr-∅}=ja}  \textit{\Highlight{sɨr}-an-ba}   \textit{jiccj-sa+ar-i=joo}\\
      what=FOC  PROX-NLZ  PROX-NLZ  rot-INF=TOP  do-NEG-CSL   no.problem-ADJ+STV-NPST=CFM1\\
            Complement  LV\\
    \glt       ‘What? This (one), this (one). (It) will not rot, so (it) is no problem (for you to bring it back).’ [Co: 101023\_01.txt]

\ex\label{ex:9.25c}\relax [= (6-49)]\\
%TM:
    \gllll  aikiga  siikijanba.\\
      \textit{\Highlight{aik-i}=ga}  \textit{\Highlight{sɨr}-i+kij-an-ba}\\
      walk-INF=NOM  do-INF+CAP-NEG-CSL\\
      Complement  LV\\
    \glt       ‘(I) cannot walk [lit. do walking], so (I cannot bring the pickles from my house).’ [Co: 120415\_01.txt]

\ex \label{ex:9.25d} %TM:
    \gllll  waakjoo  iziga  sɨransjutɨ,\\
      \textit{waakja=ja}  \textit{\Highlight{izir-i}=ga}  \textit{\Highlight{sɨr}-an=sjutɨ}\\
      1PL=TOP  go.out-INF=NOM  do-NEGSEQ\\
        Complement  LV\\
    \glt       ‘(Since I was afraid of the American soldiers) I could not go out, and ...’ [Co: 120415\_00.txt]
    \z
\z

In (9-25 a-b), the infinitives fill the complement slots of \textit{sɨr-} ‘do.’ In these LVCs, the lexical meanings of the verbs are topicalized by \textit{ja} (TOP). In (9-25 c-d), the infinitives take the nominative case \textit{ga}. Interestingly, both of the sentences in (9-25 c-d) mean (or imply) the incapability of the speaker, i.e. ‘cannot walk’ in (9-25 c) or ‘could not go out’ in (9-25 d), which is the same phenomenon discussed in \sectref{sec:6.4.3.3} about the nominative case.

With regard to (9-23 c), the complement slot of \textit{sɨr-} ‘do’ can be filled by the finite form including \textit{{}-oo} (INT) followed by \textit{ccjɨ} (QT). The combination means ‘be about to’ as in (9-26).

\ea   Complements filled by \textit{{}-oo=ccjɨ} (INT=QT) \label{ex:9.26}
%TM:
 \gllll  ikjoccjɨ  sjun  turooja  aran?\\
    \textit{ik-\Highlight{oo=ccjɨ}}  \textit{\Highlight{sɨr}-tur-n  turoo=ja  ar-an}\\
    go-INT=QT  do-PROG-PTCP  scene=TOP  COP-NEG\\
    Complement  LV    \\
    \glt     ‘(It is) a scene where (they) were about to go (somewhere), isn’t (it)?’ [Co: 120415\_00.txt]
\z

  With regard to (9-23 d), the complement slot of \textit{sɨr-} ‘do’ can be filled by the converb that includes \textit{{}-tai} (LST).

\ea   Complements filled by the converbs that include \textit{{}-tai} (LST) \label{ex:9.27}

  \ea \label{ex:9.27a}\relax [= (8-93 b)]\\
%TM:
    \gllll  uba  (mm)  uziija  jukkadɨ  nubutai  urɨtai  sjutɨ,  nasi  mutuii.\\
      \textit{u-rɨ=ba}    \textit{uzii=ja}  \textit{jukkadɨ}  \textit{nubur-\Highlight{tai}} \textit{urɨr-\Highlight{tai}}  \textit{\Highlight{sɨr}-tur-tɨ  nasi  mur-tur-i}\\
      MES-NLZ=ACC    old.man=TOP  continuously  climb-LST  decend-LST  do-PROG-SEQ  pear  pick.up-PROG-INF\\
              Complement                                   Complement  LV    \\
      \glt       ‘The old man kept climbing and decending it [i.e. the ladder], and was picking up the pears.’ [PF: 090827\_02.txt]

\ex \label{ex:9.27b} %TM:
    \gllll  mata  ..  uma  tˀakəi  izjai  cˀjai  sjattu,\\
      \textit{mata}    \textit{u-ma}  \textit{tˀakəi}  \textit{ik-\Highlight{tai}}  \textit{k-\Highlight{tai}}  \textit{\Highlight{sɨr}-tar-tu}\\
      again    MES-place  two.time  go-LST  come-LST  do-PST-CSL\\
              Complement  Complement  LV\\
      \glt       ‘Again, (the boys) went there and came (back) twice, and then ...’ [PF: 090225\_00.txt]
    \z
\z

In (9-27 a-b), the converbs composed of \textit{{}-tai} (LST) fill the complement slots of \textit{sɨr-} ‘do.’ Interestingly, \textit{{}-tai} (LST) is often used in a sequence as in (9-27 a-b), although there is a case where it is used only once as in \sectref{sec:8.4.3.3}. In these examples, the converb composed of \textit{{}-tai} (LST) does not seem to head its own adverbial clause. Rather, the converb composed of \textit{{}-tai} (LST) seems to function as a simple adverb (cf. “converbs proper” in \citealt{Nedjalkov1995}: 98). There is another converb that fills the complement slot of \textit{sɨr-} ‘do’ as in (9-28).

\ea   Complements filled by the converbs that include \textit{{}-ganaa} (SIM) \label{ex:9.28}
%TM:
 \glll  waakjoo,  naa,  sɨtɨganaa  sɨrattuppoo.\\
    \textit{waakja=ja}  \textit{naa}  \textit{\Highlight{sɨtɨr-∅-ganaa}}  \textit{\Highlight{sɨr}-ar-tur-boo}\\
    1PL=TOP  FIL  throw.away-INF-SIM  do-PASS-PROG-CND\\
    \glt     ‘I was being thrown away [i.e. was left to myself] (in those days).’ [Co: 101023\_01.txt]
\z

In fact, the use of the \textit{{}-ganaa} (SIM) in the LVC is found only in the cases where \textit{{}-ganaa} (SIM) takes \textit{sɨtɨr-} ‘throw away.’ In other words, \textit{{}-ganaa} (SIM) is not as productive as \textit{{}-tai} (LST) when used as complements of \textit{sɨr-} ‘do.’ I propose that the combination of \textit{sɨtɨr-∅-ganaa} (throw.away-INF-SIM) and \textit{sɨr-} ‘do’ is a kind of collocation.

With regard to (9-23 e), the compound that includes \textit{madəə} ‘fail to’ can fill the complement slot of \textit{sɨr-} ‘do.’

\ea   Complement filled by the compound that includes \textit{madəə} ‘fail to’ \label{ex:9.29}
%TM:
 \gllll  amakaci  ikjoocjɨ  umututanmun,   ikimadəə  sja.  \\
    \textit{a-ma=kaci}  \textit{ik-oo=cjɨ}  \textit{umuw-tur-tar-n=mun}    \textit{\Highlight{ik-i+madəə}}  \textit{\Highlight{sɨr}-tar} \\
    DIST-place=ALL  go-INT=QT  think-PROG-PST-PTCP=ADVRS  go-INF+fail.to  do-PST  \\
                                                           Complement  LV  \\
  \glt     ‘(I) thought to go there, but couldn’t go.’ [El: 121001]
\z

With regard to (9-23 f), the examples where demonstrative adverbs fill the complement slot of \textit{sɨr-} ‘do’ are shown.

\ea   Complements filled by demonstrative adverbs \label{ex:9.30}
\ea %TM:
 \gllll  kan  sjɨ  hɨɨsai?\\
      \textit{\Highlight{ka-n}}  \textit{\Highlight{sɨr}-tɨ  hɨɨ-sa+ar-i}\\
      PROX-ADVZ  do-SEQ  big-ADJ+STV-NPST\\
      Complement  LV  \\
    \glt       ‘Is (it) big like this?’ [Co: 120415\_00.txt]

  \ex %TM:
  \glll kan  sjan  munna  juwanbəidu   atanmun.\\
      \textit{\Highlight{ka-n}}  \textit{\Highlight{sɨr}-tar-n  mun=ja  juwan=bəi=du} \textit{ar-tar-n=mun}\\
      PROX-ADVZ  do-PST-PTCP  thing=TOP  Yuwan=only=FOC exist-PST-PTCP=ADVRS\\
      Complement  LV\\      
      \glt ‘Things like this were only in Yuwan.’   [Co: 111113\_02.txt]
    \z
\z

In (9-30 a-b), the demonstrative adverb \textit{ka-n} (PROX-ADVZ) ‘like this’ fill the complement slots of \textit{sɨr-} ‘do.’ In fact, the LVC composed of the demonstrative adverb and \textit{sɨr-} ‘do’ has come to function as a single adverb as in (9-30 a) or a single adnominal as in (9-30 b) (see \sectref{sec:5.2.2} for more details).

With regard to (9-23 g), I will show the examples where the complement slots of \textit{sɨr-} ‘do’ are filled by the adverbs derived from adjectival stems.

\ea   Complements filled by the adverbs derived from adjectival stems \label{ex:9.31}
\ea %TM:
 \gllll  injainjaatu  sjui.\\
      \textit{\Highlight{inja+inja-tu}}  \textit{\Highlight{sɨr}-jur-i}\\
      RED+small-ADVZ  do-UMRK-NPST\\
      Complement  LV\\
      \glt       ‘(It) is small.’ [lit. ‘(It) does small.’] [El: 111116]

\ex \label{ex:9.31b} %TM:
    \gllll  waawaatu  sjun  tukin  turanba.\\
      \textit{\Highlight{waa+waa-tu}}  \textit{\Highlight{sɨr}-tur-n  tuki=n  tur-an-ba}\\
      RED+young-ADVZ  do-PROG-PTCP  time=DAT1  take-NEG-CSL\\
      Complement  LV    \\
      \glt  ‘(You) should take (the vegetables) while (they) are green.’ [lit. ‘If (you) don’t take (the vegetables) while (they) are doing young, (they will become bad soon).] [El: 111116]
    \z
\z

With regard to (9-23 h), the complement slot of \textit{sɨr-} ‘do’ can be filled by the adjectives.

\ea   Complements filled by the adjectives \label{ex:9.32}
\ea %TM:
 \gllll  cikjasa  sjutənhazɨjaa\\
      \textit{\Highlight{cikja-sa}}  \textit{\Highlight{sɨr}-jur-təər-n=hazɨ=jaa}\\
      close-ADJ  do-UMRK-RSL-PTCP=certainty=SOL\\
      Complement  LV\\
      \glt       ‘(They) must have been close [i.e. familiar] (to each other).’ [Co: 120415\_01.txt]

\ex \label{ex:9.32b} %TM:
    \gllll  nusinkjabəi  dujasa  (sɨ)  sɨppoo,  urɨ  janban,\\
      \textit{nusi=nkja=bəi}  \textit{\Highlight{duja-sa}}  \textit{sɨr-}  \textit{\Highlight{sɨr}{}-boo  u-rɨ  jar-n=ban}\\
      RFL=APPR=only  rich-ADJ  do  do-CND  MES-NLZ  COP-PTCP=ADVRS\\
        Complement    LV    \\
      \glt       ‘If (people) are rich only themselves, (it) is that [i.e. not good], but ...’ [Co: 120415\_01.txt]

\ex \label{ex:9.32c} %TM:
    \gllll  wanga  uigicjasa  sjɨ?\\
      \textit{wan=ga}  \textit{\Highlight{uig-i+cja-sa}}  \textit{\Highlight{sɨr}-tɨ}\\
      1SG=NOM  swim-INF+want-ADJ  do-SEQ\\
        Complement  LV\\
      \glt       ‘Did I seem to want to swim?’ [El: 110914]
    \z
\z

In (9-32 a-b), the (non-derived) adjectives fill the complement slots of \textit{sɨr-} ‘do.’ In (9-32 c), the complement slot is filled by the adjective derived from a verbal stem, i.e. \textit{uig-i+cja-sa} (swim-INF+want-ADJ) ‘want to swim’ (see also \sectref{sec:4.3.8.2}). If the complement of \textit{sɨr-} ‘do’ is filled by \textit{cja-sa} (want-ADJ), the LVC means that the subject \Highlight{seems to want to} do the action indicated by the verbal stem as in (9-32 c). These formes that take \textit{{}-sa} (ADJ) are adjectives, but they are used adverbially in these examples (see also \sectref{sec:4.3.4} on the adverbial use of adjectives).

With regard to (9-23 i), the complement slot of \textit{sɨr-} ‘do’ can be filled by the units followed by \textit{nən} ‘such as’ (see \sectref{sec:10.4.4} for more details).

\ea   \label{ex:9.33}
% TM: 
\glll  muru  kjoodəənən  sjɨ,  sjɨ  moojutattujaa.\\
    \textit{muru}  \textit{kjoodəə=\Highlight{nən}}  \textit{\Highlight{sɨr}-tɨ  sɨr-tɨ  moor-jur-tar-tu=jaa}\\
    very  brother=such.as  do-SEQ  do-SEQ  HON-UMRK-PST-CSL=SOL\\
    \glt     ‘(They) used to keep company with each other like they were brothers.’ [Co: 120415\_01.txt]
\z

It may be possible that the first /sjɨ/ is not the converb \textit{sɨr-tɨ} (do-SEQ) but the instrumental case \textit{sjɨ}. In that case, /kjoodəə=nən=sjɨ/ (brother=such.as=INST) would be in the complement slot of the second /sjɨ/ (do.SEQ).

Before concluding this section, I will present the combinations of the LVC and the AVC.

\ea   \label{ex:9.34}
\ea \textit{sɨr-} ‘do’ fills the lexical verb slot of an AVC \label{ex:9.34a}\\
%TM:
 \gllll  kakəə  sjɨ  mooranta.\\
    \textit{\Highlight{kak}-i=ja  \Highlight{sɨr}-tɨ  \Highlight{moor}-an-tar}\\
    write-INF=TOP  do-SEQ  HON-NEG-PST\\
    \{Complement  [LV/Lex. Verb  Aux. Verb]\textsubscript{AVC}\}\textsubscript{LVC}\\
    \glt     ‘(The person) did not write (it).’ [El: 121010]

\ex AVC fills the complement slot of LVC\label{ex:9.34b}\\
%TM:
 \gllll  kacjɨ  mooija  sɨranta.\\
    \textit{\Highlight{kak}-tɨ  \Highlight{moor}-i=ja  \Highlight{sɨr}-an-tar}\\
    write-SEQ  HON-INF=TOP  do-NEG-PST\\
    \{[Lex. Verb  Aux. Verb]\textsubscript{Complement}  LV\}\textsubscript{LVC}\\
    \glt     ‘(The person) did not write (it).’ [El: 121010]
    \z
\z

In (9-34 a-b), they use the same set of the verbal roots, i.e. \textit{kak-} ‘write,’ \textit{sɨr-} ‘do,’ and \textit{moor-} (HON). In (9-34 a), \textit{kak-} ‘write’ becomes the infinitive filling the complement slot, and the light verb \textit{sɨr-} ‘do’ fills the lexical verb slot, which is followed by the auxiliary verb \textit{moor}{}- (HON). In (9-34 b), \textit{kak-} ‘write’ and \textit{moor-} (HON) forms an AVC, and it fills the complement slot of the light verb \textit{sɨr-} ‘do.’ There seems to be little semantic difference between them. In the texts, however, the latter combination, where AVC fills the complement slot of LVC, is preferred as in (9-35 a-b).

\ea   AVCs fill the complement slots of LVCs \label{ex:9.35}
\ea \label{ex:9.35a}%TM:
 \gllll  naa,  hinzjaaba  succjun  cˀjoo  hinzjaa   succjɨ  ikibəidu  sjattoo.\\
      \textit{naa}  \textit{hinzjaa=ba}  \textit{sukk-tur-n}  \textit{cˀju=ja}  \textit{hinzjaa}  \textit{\Highlight{sukk-tɨ}}  \textit{\Highlight{ik-i}=bəi=du}  \textit{\Highlight{sɨr}-tar=doo}\\
      FIL  goat=ACC  pull-PROG-PTCP  person=TOP  goat pull-SEQ  go-INF=only=FOC  do-PST=ASS\\
      [Lex. Verb  Aux. Verb]\textsubscript{AVC (=Complement)}  LV\\
      \glt       ‘The person who was pulling the goat (actually) pulled the goat and went (out).’ [PF: 090827\_02.txt]

\ex \label{ex:9.35b}% TM: 
\gllll kurəə  {\textbar}reizooko{\textbar}nandu  ɨrɨtəə  aija   sjutanban,\\
      \textit{ku-rɨ=ja}  \textit{reizooko=nan=du}  \textit{\Highlight{ɨrɨr-tɨ}=ja}  \textit{\Highlight{ar-i}=ja} \textit{\Highlight{sɨr}-tur-tar-n=ban}\\
      PROX-NLZ=TOP  fridge=LOC1=FOC  put.in{}-SEQ=TOP  RSL-INF=TOP  do-PROG-PST-PTCP=ADVRS\\
          [Lex. Verb  Aux. Verb]\textsubscript{ AVC (=Complement)}  LV      \\
      \glt       ‘Although this has been put in the fridge, ...’ [Co: 101023\_01.txt]
    \z
\z

In (9-35 a), the AVC composed of the lexical verb \textit{sukk-} ‘pull’ and the auxiliary verb \textit{ik-} ‘go’ fills the complement slot. The AVC is nominalized by \textit{{}-i} (INF) and modifies \textit{sɨr-} ‘do.’ In (9-35 b), the AVC composed of the lexical verb \textit{ɨrɨr-} ‘put in’ and the auxiliary verb \textit{ar-} (RSL) fills the complement slot. The AVC is also nominalized by \textit{{}-i} (INF) and modifies \textit{sɨr-} ‘do.’

\subsubsection{\textit{nar-} ‘become’}\label{sec:9.1.2.2}

The light verb \textit{nar-} ‘become’ usually means a change of state, and the result of change is expressed in the complement slot. The complement slot is filled by an NP, an adverb, or a participle that ends with \textit{{}-an} (NEG). First, I will present the exmaples where NPs fill the complement slots of \textit{nar-} ‘become.’

\ea   Complements filled by NPs \label{ex:9.36}
\ea %TM:
 \gllll  naa  huccju  natəəroo,  jiccjancjɨ,                                       xxx  cjɨ  umujui.  \\
      \textit{naa}  \textit{\Highlight{huccju}}  \textit{\Highlight{nar}-təəra=ja  jiccj-sa+ar-n=ccjɨ}             \textit{=ccjɨ}  \textit{umuw-jur-i}  \\
      FIL  old.person  become-after=TOP  not.mind-ADJ+STV-PTCP=QT                            QT  think-UMRK-NPST  \\
        Complement  LV  \\
      \glt       ‘(I) think that after (I) became old (I) didn’t mind.’ [Co: 120415\_01.txt]

\ex \label{ex:9.36b} %TM:
    \gllll  ujankjatu  akka  ziisantaatuga    {\textbar}itoko{\textbar}  najuncjɨ.\\
      \textit{uja=nkja=tu}  \textit{a-rɨ=ga}  \textit{ziisan-taa=tu=ga}   \textit{\Highlight{itoko}}  \textit{\Highlight{nar}-jur-n=ccjɨ}\\
      parent=APPR=COM  DIST-NLZ=GEN  grandfather-PL=COM=NOM  cousin  correspond-UMRK-PTCP=QT\\
                                                             Complement  LV\\
      \glt       ‘(She said) that (her) parents and that person’s grandfather are cousins.’ [Co: 110328\_00.txt]

\ex \label{ex:9.36c} %TM:
    \gllll  amankjo  hamadu  natutattujaa.\\
      \textit{a-ma=nkja=ja}  \textit{\Highlight{hama=du}}  \textit{\Highlight{nar}-tur-tar-tu=jaa}\\
      DIST-place=APPR=TOP  beach=FOC  become-PROG-PST-CSL=SOL\\
        Complement  LV\\
      \glt       ‘That place was a beach (in those days).’ [Co: 120415\_00.txt]

\ex \label{ex:9.36d} %TM:
    \gllll  {\textbar}zjuunizi{\textbar}  natəəra,  mukkoocjɨkai?\\
      \textit{\Highlight{zjuunizi}}  \textit{\Highlight{nar}-təəra  mukk-oo=ccjɨ=kai}\\
      twelve.o’clock  become-after  bring-INT=QT=DUB\\
      Complement  LV  \\
      \glt       ‘(Does she think) that (she will) bring (the lunch) after (it) becomes twelve o’clock?’ [Co: 120415\_01.txt]
    \z
\z

In these examples, the complement slots of the light verb \textit{nar-} ‘become’ are filled by NPs, i.e. \textit{huccju} ‘old person,’ \textit{itoko} ‘cousin,’ \textit{hama} ‘beach,’ and \textit{zjuunizi} ‘twelve o’clock.’ The complement NP is sometimes followed by \textit{du} (FOC) as in (9-36 c). Sometimes, \textit{nar-} has a meaning similar to the copula (or “proper inclusion”) \citep[114]{Payne1997} if the complement is a term to express the relation of relatives, e.g. \textit{itoko} ‘cousin’ as in (9-36 b). Additionally, there is a case where \textit{nar-} can mean a temporary state when it takes \textit{{}-tur} (PROG) as in (9-36 c) (see aslo (8-136) in \sectref{sec:8.6.1.5}). Thus, one may think that \textit{nar-} ‘become’ in (9-36 a-d) fills the copula verb slot in the nominal predicate phrase. However, I do not accept this analysis, since there is a syntactic difference between \textit{nar-} ‘become’ and the copula verb \textit{ar-}.

\ea   Difference between \textit{nar-} ‘become’ and \textit{ar-} (COP) \label{ex:9.37}

 \exi{} Verbal predicate phrase (LVC of \textit{nar-} ‘become’)

\ea %TM:
 \gllll  *wanna  sinsjeiga/nu  naranba,  sɨrandoo.\\
       \textit{wan=ja}  \textit{sinsjei=\Highlight{ga/nu}}  \textit{nar-an-ba}  \textit{sɨr-an=doo}\\
       1SG=TOP  teacher=NOM  become-NEG-CSL  do-NEG=ASS\\
        [Complement  LV]\textsubscript{VP}  \\
      \glt       [Intended meaning] ‘I will not become a teacher, so (I) won’t do (the hard studying).’ [El: 130822]

\ex \label{ex:9.37b} %TM:
    \gllll  wanna  sinsjeija  naranba,  sɨrandoo.\\
       \textit{wan=ja}  \textit{sinsjei=\Highlight{ja}}  \textit{nar-an-ba}  \textit{sɨr-an=doo}\\
       1SG=TOP  teacher=TOP  become-NEG-CSL  do-NEG=ASS\\
        [Complement  LV]\textsubscript{VP}  \\
      \glt        ‘I will not become a teacher, so (I) won’t do (the hard studying).’ [El: 130822]

\exi{}  Nominal predicate phrase

\ex \label{ex:9.37c} %TM:
    \gllll  wanna  sinsjeiga  aranba,  sijandoo\\
       \textit{wan=ja}  \textit{sinsjei=\Highlight{ga}}  \textit{ar-an-ba}  \textit{sij-an=doo}\\
       1SG=TOP  teacher=NOM  COP-NEG-CSL  know-NEG=ASS\\
        [NP  Copula verb]\textsubscript{Nominal predicate}  \\
      \glt        ‘I am not a teacher, so (I) don’t know (it).’ [El: 140227]
    \z
\z

The NP in the predicate (not the subject NP) of the subordinate clause in negative takes the nominative case as in (9-37 c) (see \sectref{sec:9.3.3.1} for more details). On the contrary, the NP that precedes \textit{nar-} ‘become’ cannot take the nominative case in the same environment as in (9-37 a). In that case, the NP takes the topic marker \textit{ja} as in (9-37 b). Thus, I propose that \textit{nar-} ‘become’ is different from the copula verb in Yuwan.

  Next, I will present the exmaples where adverbs fill the complement slots of \textit{nar-} ‘become.’

\ea   Complements filled by adverbs \label{ex:9.38}
\ea \label{ex:9.38a}%TM:
 \gllll  jiciku  natancjɨjo.\\
      \textit{\Highlight{jiciku}}  \textit{\Highlight{nar}-tar-n=ccjɨ=joo}\\
      well  become-PST-PTCP=QT=CFM1\\
      Complement  LV\\
      \glt       ‘(You) became well.’ [Co: 110328\_00.txt]

\ex \label{ex:9.38b} %TM:
    \gllll  kˀuruguruutu  natajaa.  \\
      \textit{\Highlight{kˀuru+kˀuru-tu}}  \textit{\Highlight{nar}-tar=jaa} \\
      RED+black-ADVZ  become-PST=SOL  \\
      Complement  LV  \\
      \glt       ‘(You) became black [i.e. tanned].’ [El: 111116]
    \z
\z

In (9-38 a-b), the adverbs in the complement slots, i.e. \textit{jiciku} ‘well’ and /kuruguruutu/ \textit{kˀuru+kˀuru-tu} (RED+black-ADVZ), mean the result of changes.

Finally, the complement slot of \textit{nar-} ‘become’ may be filled by the participle that ends with \textit{{}-an} (NEG). These LVCs express that someone (or something) has come into a state not to do (or not to be) a certain thing as in (9-39 a-d).

\ea   Complements filled by the participle that ends with \textit{{}-an} (NEG) \label{ex:9.39}
 \ea\relax [Context: Rembering a person who kindly copied music tapes for everyone]\\
%TM:
    \gllll  arɨ  siicjagɨsan  cˀjunkjaga   cˀjuin  umooran  natattujaa.\\
      \textit{a-rɨ}  \textit{sɨr-i-cjagɨ-sa+ar-n}  \textit{cˀju=nkja=ga} \textit{cˀjui=n}  \textit{\Highlight{umoor-an}}  \textit{\Highlight{nar}-tar-tu=jaa}\\
      DIST-NLZ  do-INF-seem-ADJ+STV-PTCP  person=APPR=NOM  one.person=even  exist.HON-NEG  become-PST-CSL=SOL\\
                                                             Complement  LV\\
      \glt       ‘There are no people who are likely to do that [i.e. recording].’ [lit. ‘People who are likely to do that became not to exist.’] [Co: 120415\_01.txt]

  \ex\relax  [Context: Looking at the scene of funeral]\\
%     TM:  
    \gllll {\textbar}saikin{\textbar}doojaa.  {\textbar}moo{\textbar}  (kurɨ,)  kurɨnu  nən  najun  {\textbar}koro{\textbar}doojaa.\\
      \textit{saikin=doo=jaa}  \textit{moo}  \textit{ku-rɨ}  \textit{ku-rɨ=nu}  \textit{\Highlight{nə-an}}  \textit{\Highlight{nar}-jur-n  koro=doo=jaa}\\
      recent=ASS=SOL  already  PROX-NLZ  PROX-NLZ=NOM  exist-NEG  become-UMRK-PTCP  time=ASS=SOL\\
          {}             {}        {}          {}          Complement  LV \\
      \glt ‘(The scene) is the recent one. (It) is the time when this [i.e. a style of funeral] ceased to be done [lit. becomes not to exist].’   [Co: 111113\_01.txt]

\ex \label{ex:9.39c} %TM:
    \gllll  ujahuzinkjanu  wuran  natəəroo,  (ujan)  ...  huccjunkjanu  wuran  nappoo,\\
      \textit{ujahuzi=nkja=\Highlight{nu}}  \textit{\Highlight{wur-an}}  \textit{\Highlight{nar}-təəra=ja  uja=n} \textit{huccju=nkja=\Highlight{nu}}  \textit{\Highlight{wur-an}}  \textit{\Highlight{nar}-boo}\\
      ancestor=APPR=NOM  exist-NEG  become-after=TOP  parent=also  old.person=APPR=NOM  exist-NEG  become-CND\\
        Complement  LV                                              Complement  LV\\
      \glt       ‘When there are no longer ancestors, (and) if there are no old people, ...’ [lit. ‘After ancestors become not to exist, (and) if old people become not to exist, ...’ [Co: 120415\_01.txt]

\ex \label{ex:9.39d} %TM:
    \gllll  naa,  {\textbar}cue{\textbar}  cɨkan  natattu.\\
      \textit{naa}  \textit{cue}  \textit{\Highlight{cɨk-an}}  \textit{\Highlight{nar}-tar-tu}\\
      FIL  stick  carry-NEG  become-PST-CSL\\
          Complement  LV\\
      \glt       ‘(You) walk without a stick (these days).’ [lit. ‘(You) became not to carry a stick.’] [Co: 110328\_00.txt]
    \z
\z

In (9-39 c), the subjects have the nominative case \textit{nu} (not \textit{ga}), which is another reason why I do not think that \textit{nar-} ‘become’ is different from the copula verb in the nominal predicate. If it was a copula in the nominal predicate, the subject must take the nominative case \textit{ga} (not \textit{nu}) (see \sectref{sec:6.4.3.2} for more details).

Before concluding this section, I will present examples where \textit{nar-} ‘become’ seems to be used without its complement as in (9-40 a-b).

\ea \label{ex:9.40}

\ea \textit{nar-} ‘become’ with the converb that ends with \textit{{}-an-ba} (NEG-CSL)\\
%TM:
 \glll  jazin  kurɨsjɨ  kajuwanba,  narandarooga.\\
    \textit{jazin}  \textit{ku-rɨ=sjɨ}  \textit{kajuw-an-ba}  \textit{nar-an=daroo=ga}\\
    necessarily  PROX-NLZ=INST  go.often-NEG-CSL  become-NEG=SUPP=CFM3\\
    \glt     ‘(We) had to go often (to the hospital) by this [i.e. a ship].’ [Co: 111113\_02.txt]

\ex\textit{nar-} ‘become’ with the converb that ends with \textit{{}-an-boo} (NEG-CND)\\
%TM:
 \glll  waasan  ucjəə,  ganba,  hatarakanboo, naranbajaa.\\
    \textit{waa-sa+ar-n}  \textit{uci=ja}  \textit{ganba}  \textit{hatarak-an-boo}  \textit{nar-an-ba=jaa}\\
    young-ADJ+STV-PTCP  during=TOP  therefore  work-NEG-CND  become-NEG-CSL=SOL\\
  \glt     ‘While (one) is young, (one) has to work.’ [Co: 120415\_01.txt]
  \z
\z

Different from the preceding examples, \textit{nar-} in (9-40 a-b) do not seem to express the change of state. Rather it expresses the meaning of obligation with the preceding adverbial clause that is headed by converbs including \textit{{}-an-ba} (NEG-CSL) or \textit{{}-an-boo} (NEG-CND) (see also \sectref{sec:11.2.4} for more details).

\section{Adjectival predicate phrase}\label{sec:9.2}

The adjectival predicate phrase has the following structure.

\ea   Structure of the adjectival predicate phrase \label{ex:9.41}\\\relax
  [A \hspace{\tabcolsep} (STV)]\textsubscript{Adjectival predicate phrase}
\z

An adjectival predicate phrase always include an adjective (“A”). An adjective always takes the adjectival inflectional affixes \textit{{}-sa} or \textit{{}-soo} (see also \sectref{sec:4.3.4}), and the adjective cannot take affixes that can express time or aspect. The information about tense or aspect may be expressed by the stative verbs \textit{ar-} or \textit{nə-} (“STV”) that follow the adjective (see \sectref{sec:8.3.4}). Basically, \textit{ar-} (STV) co-occurs with an adjective that ends with \textit{{}-sa} (ADJ), and \textit{nə-} (STV) co-occurs with an adjective that ends with \textit{{}-soo} (ADJ). In AVC or LVC, \textit{ar-} (STV) can also co-occur with \textit{{}-soo} (ADJ) (see \sectref{sec:9.2.2.3}).

  In the following sections, I will present examples where the adjectives alone (without the stative verbs) fill the predicate phrase (see \sectref{sec:9.2.1}). Next, I will present examples where the adjectives and the stative verbs \textit{ar-} together fill the predicate phrase (see \sectref{sec:9.2.2}). Finally, I will present examples where the adjectives and the stative verbs \textit{nə-} together fill the predicate phrase (see \sectref{sec:9.2.3}).

\ea   Three possible combinations in the adjectival predicate phrase\label{ex:9.42}
\ea Without stative verbs\\\relax
    [Adjectival root + \textit{{}-sa}/\textit{{}-soo} (ADJ)]\textsubscript{Adjective}  (see \sectref{sec:9.2.1})
\ex  With \textit{ar-} (STV)\\\relax
    [Adjectival root + \textit{{}-sa}/\textit{{}-soo} (ADJ)]\textsubscript{Adjective}  + \textit{ar-} (STV)  (see \sectref{sec:9.2.2})
\ex With \textit{nə-} (STV)\\\relax
    [Adjectival root + \textit{{}-soo} (ADJ)]\textsubscript{Adjective}  + \textit{nə-} (STV)  (see \sectref{sec:9.2.3})
    \z
\z

The form in (9-42 a) is always used in affirmative, and the form in (9-42 b) is basically used in affirmative too (with the exception of AVC). The form in (9-42 c) is always used in negative.

\subsection{Adjectives alone in the predicate phrase}\label{sec:9.2.1}

An adjective that takes \textit{{}-sa} (ADJ) or \textit{{}-soo} (ADJ) can fill the predicate phrase without a stative verb, where the polarity is always affirmative. In this case, \textit{{}-sa} (ADJ) is more productive than \textit{{}-soo} (ADJ) as in the following examples.

\ea   \label{ex:9.43}
\exi{} Adjectives ending with \textit{{}-sa} (ADJ) 
\ea %TM:
 \glll  kjuu  sinbunnan  nutuppaga  utumarasja.\\
      \textit{kjuu}  \textit{sinbun=nan}  \textit{nur-tur-ba=ga}  \textit{\Highlight{utumarasj-sa}}\\
      today  newspaper=LOC1  appear-PROG-CSL=FOC  feel.strange-ADJ\\
      \glt       ‘Since (the person) appeared in the newspaper today, (I) feel strange.’ [Co: 120415\_01.txt]

  \ex\relax [Context: Looking at a picture taken in the old days]\\
%     TM:  
\glll nozomutaa  namanu  an  wunagunu  kˀwan   nissja.\\
      \textit{nozomu-taa}  \textit{nama=nu}  \textit{a-n}  \textit{wunagu=nu}  \textit{kˀwa=n} \textit{\Highlight{nissj-sa}}\\
      Nozomu-PL  now=GEN  DIST-ADNZ  woman=GEN  child=DAT1  similar-ADJ\\
      \glt    ‘Nozomu is similar to the girl [i.e. Nozomu’s daughter] (who lives) now.’  [Co: 111113\_02.txt]

 \ex\relax [= (4-50 a)]\\
%TM:
    \glll  agɨɨ,  nacɨkasja.\\
      \textit{agɨ}  \textit{\Highlight{nacɨkasj-sa}}\\
      oh  familiar-ADJ\\
      \glt       ‘(I) miss them (on the picture).’ [Co: 120415\_00.txt]

 \ex\relax  [= (8-104 a)]\\
%TM:
    \glll  naa,  mutunu  kutunkjagadəə  sijantɨn,  jiccjaccjɨdu  juuba.\\
      \textit{naa}  \textit{mutu=nu}  \textit{kutu=nkja=gadɨ=ja}  \textit{sij-an-tɨ=n}     \textit{\Highlight{jiccj-sa}=ccjɨ=du}  \textit{jˀ} \textit{-ba}\\
      FIL  origin=GEN  event=APPR=LMT=TOP  know-NEG-SEQ=even  no.problem-ADJ=QT=FOC  say-CSL\\
      \glt       ‘(Younger people) say that, “(There) is no problem, even if (we) don’t know about the old events.”’ [Co: 111113\_02.txt]

\ex\relax [Context: Remembering a silk mill that was used to be in Yuwan]\\
%TM:
    \glll  urɨnu,  warabɨ  sjuinnja,  mɨzɨrasjacjɨ  miigjaa  ikuboo,\\
      \textit{u-rɨ=nu}  \textit{warabɨ}  \textit{sɨr-tur-i=n=ja}  \textit{\Highlight{mɨzɨrasj-sa}=ccjɨ}     \textit{mj-i+gja}  \textit{ik-boo}\\
      MES-NLZ=NOM  child  do-PROG-INF=DAT1=TOP  rare-ADJ=QT   see-INF+PURP  go-CND      \\
      \glt       ‘When (I) was a child [lit. was doing a child], (I thought) that (it was) rare, and went to see (the way of silk reeling), and then ...’ [Co: 111113\_01.txt]
\ex % TM:
\glll      cɨkɨmununkjoo,  gan  utussja,  naa,  ippai,  naa,  cɨkɨjutanban,\\
      \textit{cɨkɨmun=nkja=ja}  \textit{ga-n}  \textit{\Highlight{utussj-sa}}  \textit{naa}  \textit{ippai}  \textit{naa}  \textit{cɨkɨr-jur-tar-n=ban}\\
      pickle=APPR=TOP  MES-ADVZ  fearful-ADJ  FIL  much  FIL    pickle-UMRK-PST-PTCP=ADVRS\\
      \glt       ‘About pickles, oh my god, (I) used to pickle (them) very much, but ...’ [Co: 101023\_01.txt]

\exi{} Adjectives ending with \textit{{}-soo} (ADJ)
\ex % TM: 
\glll   kˀwasinu  hɨɨsoo.\\
      \textit{kˀwasi=nu}  \textit{\Highlight{hɨɨ-soo}}\\
      snack=NOM  big-ADJ\\
      \glt       ‘The snack (is) big.’ [El: 120914]

\ex\relax [= (4-50 b)]\\
%TM:
    \glll  agɨɨ!  wuganduusoo.\\
      \textit{agɨ}  \textit{\Highlight{wuganduu-soo}}\\
      oh  not.see.for.a.long.time-ADJ\\
      \glt       ‘Oh! (I) haven’t seen (you) for a long time.’ [El: 120912]
    \z
\z

In (9-43 a-c), the adjectives terminate the sentences. In (9-43 d-e), the adjectives terminate the clauses that express the direct reported speech with the quotative marker \textit{ccjɨ}. The example in (9-43 f) express an interesting use of the adjectival predicate phrase. The combination of \textit{ga-n} (MES-ADVZ) and \textit{utussj-sa} (fearful-ADJ) functions as a kind of interjection as a whole, which is tentatively translated into ‘oh my god’ in this example.

  Furthermore, adjectives that end with \textit{-sa} (ADJ) without a stative verb, may be followed by the stentence-final particle \textit{jaa} (SOL), the conjunctive particle \textit{nu} (CSL), or the limitter particle \textit{gadɨ} (LMT) as in (9-44).

\ea   With \textit{jaa} (SOL) \label{ex:9.44}
\ea %TM:
 \glll  takesitu  nissjajaa.\\
      \textit{takesi=tu}  \textit{\Highlight{nissj-sa=jaa}}\\
      Takeshi=COM  similar-ADJ=SOL\\
      \glt       ‘(He) is similar to Takeshi, (don’t you think ?)’ [Co: 120415\_00.txt]

\ex \label{ex:9.44b} %TM:
    \glll  {\textbar}iro{\textbar}nu  kjurasajaa.\\
      \textit{iro=nu}  \textit{\Highlight{kjura-sa=jaa}}\\
      color=NOM  beautiful-ADJ=SOL\\
      \glt       ‘The color is beautiful, (don’t you think?)’ [Co: 120415\_00.txt]

\exi{}  With \textit{nu} (CSL)

\ex \label{ex:9.44c} %TM:
    \glll  waakjoo  utussjanu,  aicjɨn  njanta.\\
      \textit{waakja=ja}  \textit{\Highlight{utussj-sa=nu}}  \textit{aik-tɨ=n}  \textit{nj-an-tar}\\
      1PL=TOP  fearful-ADJ=CSL  walk-SEQ=ever  EXP-NEG-PST\\
      \glt       ‘I was fearful (of the American soldiers), so I did not walk (around).’ [Co: 111113\_01.txt]

\ex \label{ex:9.44d} % TM: 
 \glll  {\textbar}suiziba{\textbar}nkjaga  kjurasanu,  (umoo)  umoo  isigakinu  cɨmattutattujaa.\\
      \textit{suiziba=nkja=ga}  \textit{\Highlight{kjura-sa=nu}}  \textit{u-ma=ja}  \textit{u-ma=ja}  \textit{isigaki=nu}  \textit{cɨm-ar-tur-tar-tu=jaa}\\
      kitchen=APPR=NOM  beautiful-ADJ=CSL  MES-place=TOP   MES-place=TOP  stone.fence=NOM  pile-PASS-PROG-PST-CSL=SOL      \\
      \glt       ‘The kitchen is beautiful, and the stone (for the) fence had been piled there.’ [Co: 120415\_01.txt]

\ex\relax [Context: Talking about the fireplace that was set in the speaker’s old house]\\
%TM:
    \glll  hujunkjoo  jiccjanu.\\
      \textit{huju=nkja=ja}  \textit{\Highlight{jiccj-sa=nu}}\\
      clothes=APPR=TOP  good-ADJ=CSL\\
      \glt       ‘(The fireplace was) good in winter.’ [Co: 111113\_02.txt]

\ex %TM:
 \glll  agaraa  munna  kisjoonu  cjussanu.\\
      \textit{aga-raa}  \textit{mun=ja}  \textit{kisjoo=nu}  \textit{\Highlight{cjuss-sa=nu}}\\
      DIST-DRG  person=TOP  temper=NOM  strong-ADJ=CSL\\
      \glt       ‘That awful person (was) stubborn.’ [lit. ‘About that awful person the temper is strong.’] [Co: 120415\_01.txt]

\exi{}  With \textit{gadɨ} (LMT)

\ex\relax [Context: Talking about a butterfly that is similar to the moth] = (5-28 a)\\
%TM:
    \glll  arɨga  nissjagadɨ.  ganbəi  sjɨ  kucjəə  tugaracjɨ,\\
      \textit{a-rɨ=ga}  \textit{\Highlight{nissj-sa=gadɨ}}  \textit{ga-n=bəi}  \textit{sɨr-tɨ}   \textit{kuci=ja}  \textit{tugaras-tɨ}\\
      DIST-NLZ=NOM  similar-ADJ=LMT  MES-ADVZ=about  do-SEQ  mouth=TOP  pout-SEQ      \\
      \glt       ‘That one is very similar (to the moth). (The size is) about this, and it pouted, and ...’ [Co: 111113\_01.txt]
    \z
\z

In (9-44 a-b), \textit{jaa} (SOL) is used to request the hearer’s agreement about the speaker’s assertion. The conjunctive particle \textit{nu} (CSL) expresses causal meaning as in (9-44 c). It sometimes expresses a meaning such as ‘and (then)’ as in (9-44 d). In (9-44 g), \textit{gadɨ} (LMT) seems to express a little emphasis on the adjective (see chapter 10 about the functions of each particle).

\subsection{Adjective and the stative verb \textit{ar-} in the predicate phrase}\label{sec:9.2.2}

The stative verb \textit{ar-} basically follows an adjective that ends with \textit{{}-sa} (ADJ), where the polarity is always affirmative. However, if \textit{ar-} (STV) fills the lexical verb slot of an AVC in negative, it can follow an adjective that ends with \textit{{}-soo} (ADJ).

The stative verb \textit{ar-} is required when the predicate wants to express one of the functions indicated by verbal inflectional affixes, e.g. \textit{{}-tɨ} (SEQ), \textit{{}-ba} (CSL), or \textit{-i} (NPST), or some particles, e.g. \textit{na} (PLQ) or \textit{doo} (ASS) (see also \sectref{sec:9.4.1}). In some conditions, the stative verb \textit{ar-} is contracted with the preceding adjectives, where the combination of \textit{{}-sa} (ADJ) and \textit{ar-} (STV) becomes /sar/ (not /saar/). This contraction occurs when \textit{ar-} (STV) takes \textit{{}-i} (NPST) or \textit{{}-n} (PTCP).

  In the following subsections, I will present examples where the contraction between the adjectives and \textit{ar-} (STV) does not occur in \sectref{sec:9.2.2.1}. Next, I will present examples where the contraction occurs in \sectref{sec:9.2.2.2}. Lastly, I will present examples where adjectival predicate phrases occur in AVC or LVC in \sectref{sec:9.2.2.3}.

\subsubsection{Non-contracted forms}\label{sec:9.2.2.1}
\label{bkm:Ref361739107}
An adjective that ends with \textit{{}-sa} (ADJ) is followed by \textit{ar-} (STV) when the predicate wants to express the functions indicated by verbal inflectional affixes (with the exception where the stative verb takes the negative affixes, which will be discussed in \sectref{sec:9.2.3}).

\ea   The combinations of the adjectives and \textit{ar-} (STV) \label{ex:9.45}
\exi{}  \textit{ar-} (STV) with \textit{{}-tɨ} (SEQ)
\ea %TM:
 \glll  waakjaa  cˀjantaaja  kurɨga  nagasa  atɨ,\\
      \textit{waakja-a}  \textit{cˀjan-taa=ja}  \textit{ku-rɨ=ga}  \textit{\Highlight{naga-sa}}  \textit{\Highlight{ar-tɨ}}\\
      1PL-ADNZ  father-PL=TOP  PROX-NLZ=NOM  long-ADJ  STV-SEQ\\
      \glt       ‘My father was tall, and ...’ [lit. ‘About my father, this [i.e. height] was very tall, and ...’] [Co: 111113\_01.txt]

\ex \label{ex:9.45b} %TM:
    \glll  naa,  kurɨga  taasa  atɨ,\\
      \textit{naa}  \textit{ku-rɨ=ga}  \textit{\Highlight{taa-sa}}  \textit{\Highlight{ar-tɨ}}\\
      FIL  PROX-NLZ=NOM  tall-ADJ  STV-SEQ\\
      \glt       ‘My father was tall, and ...’ [lit. ‘About my father, this [i.e. height] was very tall, and ...’] [Co: 111113\_01.txt]

\exi{} \textit{ar-} (STV) with \textit{{}-ba} (CSL)

\ex \label{ex:9.45c} %TM:
    \glll  arijojukkumoo  hɨɨsa  appajaa.\\
      \textit{arijo=jukkuma=ja}  \textit{\Highlight{hɨɨ-sa}}  \textit{\Highlight{ar-ba}=jaa}\\
      Ario=CMP=TOP  big-ADJ  STV-CSL=SOL\\
      \glt       ‘(The wild boar) is bigger than Ario, so (it must be a big one).’ [Co: 120415\_01.txt]

\ex \label{ex:9.45d} % TM: 
 \glll  aran.  {\textbar}mou{\textbar},  wanna  jiccja  appa.\\
      \textit{ar-an}  \textit{mou}  \textit{wan=ja}  \textit{\Highlight{jiccj-sa}}  \textit{\Highlight{ar-ba}}\\
      COP-NEG  FIL  1SG=TOP  no.problem-ADJ  STV-CSL\\
      \glt       ‘No. I’m OK.’ [lit. ‘No. About me, (there is) no problem (about the quantity of the meal), so (I don’t need more).’] [Co: 120415\_01.txt]

\exi{} \textit{ar-} (STV) with \textit{{}-u} (PFC)

\ex \label{ex:9.45e} % TM: 
 \glll  tattankjaa  kˀumɨttagamarasja  aru?\\
      \textit{ta-ru-taa=nkja}  \textit{\Highlight{kˀumɨtta+kamarasj-sa}}  \textit{\Highlight{ar-u}}\\
      who-NLZ-PL=APPR  attentive+fussy-ADJ  STV-PFC\\
      \glt       ‘Who is fussy?’ [El: 120914]

\exi{}  \textit{ar-} (STV) with \textit{{}-tar} (PST)
\ex
% TM: 
\glll   nobuariga  mm  kɨga  sjun  tukininkjoo  huntoo  kuwasa  ata.\\
      \textit{nobuari=ga}    \textit{kɨga}  \textit{sɨr-tur-n}  \textit{tuki=n=nkja=ja}  \textit{huntoo}  \textit{\Highlight{kuwa-sa}}  \textit{\Highlight{ar-tar}}\\
      Nobuari=NOM    injury  do-PROG-PTCP  time=DAT1=APPR=TOP  really  hard-ADJ  STV-PST\\
      \glt       ‘When Nobuari was suffering injuries, (it) was really hard (for me).’ [Co: 111113\_02.txt]

\exi{}  \textit{ar-} (STV) with \textit{{}-oo} (SUPP)
% TM:
\ex
\glll     nacɨkasja  aroga.\footnotemark\\
      \textit{nacɨkasj-sa}  \textit{ar-oo=ga}\\
      familiar-ADJ  STV-SUPP=CFM3\\
      \glt       ‘(The song) is familiar (to you, isn’t it?)’ [Co: 110328\_00.txt]
    \z
\z\footnotetext{It is rare but \textit{{}-oo} (SUPP) becomes /o/ before \textit{ga} (CFM3) in this example.}

In the above examples, the adjectives that end with \textit{{}-sa} (ADJ) are followed by the stative verb \textit{ar-}, which takes several inflectional affixes.

\subsubsection{Contracted forms}\label{sec:9.2.2.2}

If \textit{ar-} (STV) takes \textit{{}-i} (NPST) or \textit{{}-n} (PTCP), the \textit{ar-} (STV) is contracted with the preceding adjectives, e.g. \textit{{}-sa} (ADJ) + \textit{ar-} (STV) > /sar/ (not /saar/).\footnote{\citet[71]{Niinaga2010} described that \textit{{}-oo} (SUPP) also makes the contraction. However, a further investigation proved that it is not correct as in (9-45 g) in §\ref{bkm:Ref361739107}.} I will present examples below, where the original word boundary between the adjective and the stative verb is expressed by “+” in the underlying level.

\ea   \textit{ar-} (STV) with \textit{{}-i} (NPST) \label{ex:9.46}
  \ea\relax [= (7-25 b)]\\
%TM:
    \glll  {\textbar}cjoodo  mikan{\textbar}nu  (kun)  kun  huukkwanu   tˀɨɨ  kamboo,  xxx  jiccjai.\\
      \textit{cjoodo}  \textit{mikan=nu}  \textit{ku-n}  \textit{ku-n}  \textit{huu-kkwa=nu}   \textit{tˀɨɨ}  \textit{kam-boo}    \textit{\Highlight{jiccj-sa+ar-i}}\\
      just  mikan=GEN  PROX-ADNZ  PROX-ADNZ  piece-DIM=GEN  one.thing  eat-CND    good-ADJ+STV-NPST\\
      \glt       ‘If (I) eat just a piece of this mikan, (it) is good [i.e. sufficient] (for me).’ [Co: 101023\_01.txt]

\ex \label{ex:9.46b} %TM:
    \glll  kan  sjɨnkja  hanasinkja  zjoozɨnu cˀjunkjoo  jiccjaijoo.\\
      \textit{ka-n}  \textit{sɨr-tɨ=nkja}  \textit{hanasi=nkja}  \textit{zjoozɨ=nu}  \textit{cˀju=nkja=ja}  \textit{\Highlight{jiccj-sa+ar-i}=joo}\\
      PROX-ADVZ  do-SEQ=APPR  talking=APPR  good=GEN   person=APPR=TOP  good-ADJ+STV-NPST=CFM1\\
      \glt       ‘The people who are good at talking like this are good.’ [Co: 120415\_01.txt]

\ex \label{ex:9.46c} %TM:
    \glll  {\textbar}cjotto{\textbar}  sɨppoo,  (kazɨ  hikija)  ..  kazɨ  hikijassai.\\
      \textit{cjotto}  \textit{sɨr-boo}  \textit{kazɨ}  \textit{hik-i-jass}    \textit{kazɨ} \textit{\Highlight{hik-i-jass-sa+ar-i}}\\
      a.little  do-CND  cold  catch-INF-easy    cold  catch-INF-easy-ADJ+STV-NPST\\
      \glt       ‘(I) tend to catch a cold (with) a little (walking around).’ [Co: 120415\_01.txt]

\ex \label{ex:9.46d} % TM: 
 \glll  {\textbar}iciban{\textbar}  dujasai.\\
      \textit{iciban}  \textit{\Highlight{duja-sa+ar-i}}\\
      most  rich-ADJ+STV-NPST\\
      \glt       ‘(He) is the richest.’ [Co: 111113\_01.txt]

\ex \label{ex:9.46e} %TM: 
\glll {\textbar}dɨɨsaabisu{\textbar}nkjoo  jasumjun  tukiga  huusai.\\
      \textit{dɨɨsaabisu=nkja=ja}  \textit{jasum-jur-n}  \textit{tuki=ga}  \textit{\Highlight{huu-sa+ar-i}}\\
      day.care=APPR=TOP  not.attend-UMRK-PTCP  time=NOM   many-ADJ+STV-NPST\\      
      \glt ‘(I) often don’t go to the daycare center.’ [lit. ‘The time when (I) do not attend the daycare (center) is many.’]    [Co: 120415\_01.txt]

\exi{} \textit{ar-} (STV) with \textit{{}-n} (PTCP)
\ex\label{ex:9.46f} %TM:
  \glll  jaa,  nacɨkasjan  nɨntəəbəi  zja.\\
      \textit{jaa}  \textit{\Highlight{nacɨkasj-sa+ar-n}}  \textit{nɨntəə=bəi}  \textit{zjar}\\
      FIL  familiar-ADJ+STV-PTCP  people=only  COP\\
      \glt       ‘(They) are all familiar (to me).’ [lit. ‘(They) are people who are all familiar (to me).’] [Co: 120415\_00.txt]
\ex\label{ex:9.46g}% TM:
\glll  waasan  tuzituunu  wutɨ,\\
      \textit{\Highlight{waa-sa+ar-n}}  \textit{tuzituu=nu}  \textit{wur-tɨ}\\
      young-ADJ+STV-PTCP  couple=NOM  exist-SEQ\\
      \glt       ‘There is a young couple.’ [Fo: 090307\_00.txt]

\ex\relax [Context: Talking about Wase-unshū, i.e. a kind of orange; TM: ‘(We usually) eat (the oranges) around September.’]\\
%TM:
    \glll  nama  haanu  awusan  ucin,  tutɨ,  kam  jappa.\\
      \textit{nama}  \textit{haa=nu}  \textit{\Highlight{awu-sa+ar-n}}  \textit{uci=n}  \textit{tur-tɨ}  \textit{kam-∅}  \textit{jar-ba}\\
      still  leaf=NOM  green-ADJ+STV-PTCP  during=DAT1  take-SEQ  eat-INF  COP-CSL\\
      \glt       ‘(We) took (the oranges) while the leaves were still green, and eat (them).’ [Co: 101023\_01.txt]
\ex % TM:  
\glll an,  hɨɨsan  noogin  muccjɨ,\\
      \textit{a-n}  \textit{\Highlight{hɨɨ-sa+ar-n}}  \textit{noogi=n}  \textit{mukk-tɨ}\\
      DIST-ADNZ  big-ADJ+STV-PTCP  saw=also  bring-SEQ\\
      \glt       ‘Bringing that big saw, (they went to the mountain to cut a tree for the coffin).’ [Co: 111113\_01.txt]
    \z
\z

In the above examples, the adjectives and the stative verb are contracted. This morphophonological phenomenon indicates that they are in the same phonological unit. Thus, I used the plus sign “+” to indicate their unity, although the sign is normally used to inidicate the boundary between the stems in the compounds in this grammar (cf. \sectref{sec:4.2.3}).

\subsubsection{AVC or LVC with the adjectival predicate phrase}\label{sec:9.2.2.3}

The stative verb \textit{ar-} fills the initial slot of the VP. Therefore, it may be followed by the auxiliary verb as in (9-47 a-b). “APP” in the following examples indicate the “adjectival predicate phrase.”

\ea   AVC in the adjectival predicate phrase \label{ex:9.47}
\ea\relax [= (8-48)]\\
%TM:
    \gllll  an  cˀjoo  dujasoo  atɨ  mooran.jaa.\\
      \textit{a-n}  \textit{cˀju=ja}  \textit{duja-soo}  \textit{ar-tɨ}  \textit{moor-an=jaa}\\
      DIST-ADNZ  person=TOP  \{rich-ADJ  [STV-SEQ  HON-NEG]\}=SOL\\
          \{A  [Lex. verb  Aux. verb]\textsubscript{AVC}\}\textsubscript{APP}\\
      \glt       ‘That person is not rich, you know.’ [El: 130820]

\ex \label{ex:9.47b} %TM:
    \gllll  urakjoo  ziisantaaga  dujasa  atɨ   moocjɨ,\\
      \textit{urakja=ja}  \textit{ziisan-taa=ga}  \textit{duja-sa}  \textit{ar-tɨ}  \textit{moor-tɨ}\\
      2.NHON.PL=TOP  grandfather-PL=NOM  \{rich-ADJ  [STV-SEQ  HON-SEQ]\}\\
          \{A  [Lex. Verb/STV   Aug. Verb]\textsubscript{AVC}\}\textsubscript{APP}\\
      \glt       ‘You have a rich grandfather, and ...’ [lit. ‘About you, the grandfather was rich, and ...’] [Co: 120415\_01.txt]
    \z
\z

In (9-47 a), the adjective takes \textit{{}-soo} (ADJ) since the predicate is in negative. In (9-47 b), the adjective takes \textit{{}-sa} (ADJ) since the predicate is in affirmative. In both of the examples, the stative verb is \textit{ar-} (STV), which fills the lexical verb slot in AVC with the auxiliary verb \textit{moor-} (HON).

There is also an example where the adjectival predicate phrase fills the complement slot of an LVC as in (9-48).

\ea   Adjectival predicate phrase in the complement slot of an LVC [= (8-111 c)] \label{ex:9.48}
%TM:
 \gllll  makanəicjasoo  aija  sjunban,\\
    \textit{makanaw-i+cja-soo}  \textit{ar-i=ja}  \textit{sɨr-jur-n=ban}\\
    \{[give.a.feast-ING+want-ADJ  STV-INF=TOP]  [do-UMRK-PTCP]\}=ADVRS\\
    \{[Complement]    [LV]\}\textsubscript{LVC}\\
    \glt     ‘(I) want to give a feast (to the present author), but ...’ [Co: 101023\_01.txt]
\z

The adjective in the complement slot of LVC always takes \textit{{}-soo} (ADJ).

\subsection{Adjective and the stative verb \textit{nə-} in the predicate phrase}\label{sec:9.2.3}

The stative verb \textit{nə-} (STV), which always takes a negative affix, always follows an adjective that ends with \textit{{}-soo} (ADJ) as in (9-49 a-c).

\ea   The combinations of the adjectives and \textit{nə-} (STV) \label{ex:9.49}
  \ea\relax [Context: Talking about the wooden beams of MS’s house; MS: ‘(The wooden beams of my house) haven’t become as black as those (of your house), you know.’] = (4-11 b)
%TM:
    \glll  kˀurusoo  nəndarooga.\\
      \textit{\Highlight{kˀuru-soo}}  \textit{\Highlight{nə-an}=daroo=ga}\\
      black-ADJ  STV-NEG=SUPP=CFM3\\
      \glt       ‘(Those) are not black, right?’ [Co: 111113\_01.txt]

\ex\relax [= (4-50 d)]
%TM:
    \glll  juwasoo  nən?\\
      \textit{\Highlight{juwa-soo}}  \textit{\Highlight{nə-an}}\\
      hungry-ADJ  STV-NEG\\
      \glt       ‘Aren’t (you) hungry?’ [El: 120926]

\ex\relax [= (8-49 b)]
%TM:
    \glll  an  kasoo  kˀurusoo  nəəzɨi?\\
      \textit{a-n}  \textit{kasa=ja}  \textit{\Highlight{kˀuru-soo}}  \textit{\Highlight{nə-azɨi}}\\
      DIST-ADNZ  hat=TOP  black-ADJ  STV-NEG.PLQ\\
      \glt       ‘Isn’t that hat black?’ [El: 111118]
    \z
\z

In the above examples, the adjectives that end with \textit{{}-soo} (ADJ) are followed by the stative verb \textit{nə-}, which takes negative affixes such as \textit{{}-an} (NEG) as in (9-49 a-b) or \textit{{}-azɨi} (NEG.PLQ) as in (9-49 c).

If an adjective is followed by \textit{nə-} (STV), it can also take \textit{{}-k}(\textit{k})\textit{oo} (ADJ) as in (9-50 a-b), but such cases are very rare.

\ea   \textit{{}-k}(\textit{k})\textit{oo} (ADJ) + \textit{nə-} (STV) \label{ex:9.50}
\ea %TM:
 \glll  naa  ikicjakoo  nən.\\
      \textit{naa}  \textit{ik-i+cja-\Highlight{koo}}  \textit{\Highlight{nə}-an}\\
      yet  go-INF+want-ADJ  STV-NEG\\
      \glt       ‘(I) don’t want to go yet.’ [Co: 120415\_01.txt]

\ex \label{ex:9.50b} %TM:
    \glll  hankəəcjakkoo  nənmun,  hankəəmai  zjajaa.\\
      \textit{hankəər-∅+cja-\Highlight{kkoo}}  \textit{\Highlight{nə}{}-an=mun  hankəə-∅+mai  zjar=jaa}\\
      tumble-INF+want-ADJ  STV-NEG=ADVRS  tumble-INF+OBL  COP=SOL\\
      \glt       ‘(I) don’t want to tumble, but will have to tumble (for the play).’ [El: 110917]
    \z
\z

\section{Nominal predicate phrase}\label{sec:9.3}

The nominal predecate phrase has the following structure.

\ea   Structure of the nominal predicate phrase \label{ex:9.51}
  [NP \hspace{\tabcolsep} (COP)]\textsubscript{Nominal predicate phrase}
\z

A nominal predicate phrase is filled by an NP. The NP can be followed by a copular verb (“COP”), i.e. \textit{jar-}, \textit{ar-}, \textit{nar-}, or \textit{zjar-} (see \sectref{sec:8.3.3}). In addition, the head of the nominal predicate phrase may be filled by an adnominal clause, or an adverbial clause that takes \textit{{}-tɨ} (SEQ). In the above structure, the head of the nominal predicate phrase is regarded as the NP (not as the copula verb), which will be discussed in \sectref{sec:9.4.3} in detail. A copular verb fills the initial lexical verb slot in the VP. Therefore, it may be followed by an auxiliary verb (see (8-43) in \sectref{sec:8.3.3.4}). In principle, the copula verb always follows an NP in the predicate. However, the copula form \textit{ar-an} (COP-NEG) ‘No’ can be uttered only by itself as a negative reply to a polar question (see (8-40) in \sectref{sec:8.3.3.3}).

In the following sections, I will present the ordinary examples of the nominal predicate phrases in \sectref{sec:9.3.1}. Next, in \sectref{sec:9.3.2}, I will present examples where the head of the nominal predicate phrase may be filled by two types of subordinate clauses, i.e. the adnominal clause or the adverbial clause whose head verb ends with \textit{{}-tɨ} (SEQ). Finally, in \sectref{sec:9.3.3}, I will present examples where the predicate phrases are filled by the extended NPs, which are NPs that take case particles (see also chapter 6 for the NP).

\subsection{Basic structure}\label{sec:9.3.1}

The main points of the nominal predicate phrase were already shown in \sectref{sec:4.1.3.3}. I will pick up some of them again and add another piece of information in this section. First, the nominal predicate can be filled by only an NP (not followed by the copula verb) as in (9-52).

\ea   Predicate filled by only an NP \label{ex:9.52}
%TM:
 \gllll  kurəə  jukimasa.\\
    \textit{ku-rɨ=ja}  \textit{\Highlight{jukimasa}}\\
    PROX-NLZ=TOP  Yukimasa\\
    Subject  [NP]\textsubscript{Nomimal predicate phrase}\\
    \glt     ‘This one is Yukimasa.’ [Co: 120415\_00.txt]
\z

In (9-52), the nominal predicate phrase is filled only by the NP \textit{jukimasa} ‘Yukimasa.’ Additionally, the nominal predicate phrase can be filled by an NP and a copula verb as in (9-53).

\ea   Predicate filled by an NP and a copula verb \label{ex:9.53}
%TM:
 \gllll  zjenbuga  asɨbizjaa  jatattujaa.\\
    \textit{zjenbu=ga}  \textit{\Highlight{asɨb-i+zjaa}}  \textit{\Highlight{jar}-tar-tu=jaa}\\
    all=NOM  play-INF+place  COP-PST-CSL=SOL\\
    Subject  [NP  Copula verb]\textsubscript{Nomimal predicate phrase}\\
    \glt     ‘All (of the places) were playgrounds [lit. place to play].’ [Co: 110328\_00.txt]
\z

In (9-53), the nominal predicate phrase is filled by the NP \textit{asɨb-i+zjaa} ‘playground’ and the copula verb \textit{jar-}. In affirmative, the NPs in the predicate phrase do not take any particle in the main clauses. However, if the predicate in the main clause is in negative, the NP (in the predicate phrase) always takes the topic particle \textit{ja}, and the following copula verb is always \textit{ar-} (COP) as in (9-54) (except for the cases in \sectref{sec:9.3.3.1}). In (9-54), the copula verb \textit{ar-an} (COP-NEG) is in negative, and the preceding NP (in the predicate phrase) \textit{jasuu} ‘Yasu (personal name)’ takes the topic particle \textit{ja}.

\ea   Nominal predicate phrase in negative (in the main clause) \label{ex:9.54}
%TM:
 \gllll  kurəə  jasuuja  aran?\\
    \textit{ku-rɨ=ja}  \textit{\Highlight{jasuu=ja}}  \textit{\Highlight{ar-an}}\\
    PROX-NLZ=TOP  Yasu=TOP  COP-NEG\\
    Subject  [NP  Copula verb]\textsubscript{Nomimal predicate phrase}\\
    \glt     ‘Is this person Yasu?’ [Co: 120415\_00.txt]
\z

Furthermore, an NP (in the predicate phrase) always takes the focus particle \textit{ga} when the NP is filled by an interrogative nominal as in (9-55 a-d) (see also \sectref{sec:10.1.2.2}).

\ea   Interrogative nominals in the predicate phrase  \label{ex:9.55}
\ea %TM:
 \gllll  urəə  mata  taruga  jatakai?\\
      \textit{u-rɨ=ja}  \textit{mata}  \textit{\Highlight{ta-ru=ga}}  \textit{\Highlight{jar}-tar=kai}\\
      MES-NLZ=TOP  again  who-NLZ=FOC  COP-PST=DUB\\
      Subject    [NP  Copula verb]\textsubscript{Nomimal predicate phrase}\\
      \glt       ‘(I wonder) who was that person (that had brought this pamphlet of songs)?’ [Co: 120415\_01.txt]

\ex \label{ex:9.55b} \gllll   US:  gazimaruu  ..  daaga  jataru?\\
      \textit{gazimaru}    \textit{\Highlight{daa=ga}}  \textit{\Highlight{jar}-tar-u}\\
      banyan.tree    where=FOC  COP-PST-PFC\\
      Subject    [NP  Copula verb]\textsubscript{Nomimal predicate phrase}\\
      \glt       ‘Where was the banyan tree?’ [Co: 110328\_00.txt]

\ex \label{ex:9.55c} %TM:
    \gllll  arəə  nuuga  jataru?\\
      \textit{a-rɨ=ja}  \textit{\Highlight{nuu=ga}}  \textit{\Highlight{jar}-tar-u}\\
      DIST-NLZ=TOP  what=FOC  COP-PST-PFC\\
      Subject  [NP  Copula verb]\textsubscript{Nomimal predicate phrase}\\
      \glt       ‘What was that box?’ [El: 130822]

\ex \label{ex:9.55d} %TM:  
    \glll uraga  jˀicjasəə  dɨruga    jataru?\\
      \textit{ura=ga}  \textit{jˀ-tar=sɨ=ja}  \textit{\Highlight{dɨ-ru=ga}} \textit{\Highlight{jar}-tar-u}\\
      2.NHON.SG=NOM  say-PST=FN=TOP  which-NLZ=FOC  COP-PST-PFC\\                                                   
          [NP                                       Copula verb]\textsubscript{Nomimal predicate phrase}\\
      \glt ‘Which is the one that you said.’    [El: 130822]
    \z
\z

In the above examples, the interrogative nominals, i.e. \textit{ta-ru} ‘who,’ \textit{daa} ‘where,’ \textit{nuu} ‘what,’ and \textit{dɨ-ru} ‘which,’ take \textit{ga} (FOC) in the predicate phrases.

It was pointed out that the nominal predicates in the languages around the world is used to indicate equation, e.g., \textit{He} \textit{is} \textit{my} \textit{father}, and proper inclusion, e.g., \textit{He} \textit{is} \textit{a} \textit{teacher} \citep[114]{Payne1997}. The nominal predicate in Yuwan also has both of these functions. For example, (9-52) is an example of equation, and (9-53) is an example of proper inclusion. In any case, the referents indicated by the subjects are the same with those indicated by the predicate NPs in those examples. However, there is a case where the referent of the subject does not coincide with the referent of the NP in the nominal predicate as in (9-56), where the relation between the subject and the nominal predicate has to be supplemented pragmatically.

\ea   Pragmatic relation \label{ex:9.56}\\
%TM:
 \gllll  urakjoo  naa  gakkoo  jatarooga.\\
    \textit{\Highlight{urakja}=ja}  \textit{naa}  \textit{\Highlight{gakkoo}}  \textit{jar-tar-oo=ga}\\
    2.NHON.PL=TOP  already  school  COP-SPT-SUPP=CFM3\\
    Subject    [NP  Copula verb]\textsubscript{Nomimal predicate phrase}\\
    \glt     ‘Probably, you had already begun to go to school.’ [lit. ‘Probably, you were already school.’] [Co: 120415\_00.txt]
\z

In (9-56), the subject \textit{urakja} ‘you’ and the NP in the nominal predicate \textit{gakkoo} ‘school’ do not indicate the same referent. In fact, there is a relation between them that can be supplemented by the pragmatic information. This kind of use of the nominal predicate is famous in Japanese linguistics as “\textit{unagi-bun}” (‘The “eel” construction’) (cf. \citealt{Okutsu1978}).

\subsection{Subordinate clause in the nominal predicate phrase}\label{sec:9.3.2}

There are examples where the head of the nominal predicate phrase is “directly” filled by a certain kind of subordinate clause. The subordinate clause is not filling in an NP, since it cannot be modified by an adnominal word nor become the argument of a clause. The reason why the subordinate clause is thought to fill the nominal predicate phrase (in spite of not filling in an NP) is that the subordinate clause can be followed by the copula verb. There are two kinds of subordinate clause that can fill in the nominal predicate phrase, i.e. adnominal clauses (see \sectref{sec:9.3.2.1}) and adverbial clauses (see \sectref{sec:9.3.2.2}).

\subsubsection{Adnominal clause in the nominal predicate phrase}\label{sec:9.3.2.1}

The adnominal clause can fill the head slot of the nominal predicate phrase by itself. In that case, the adnominal clause is always followed by the negative copula verb, i.e. \textit{ar-an} (COP-NEG), as in (9-57 a-g) (see also \sectref{sec:8.3.3} about the copula verb).

\ea   
\exi{} Adnominal clause including \textit{-n} (PTCP) in the nominal predicate phrase \label{ex:9.57}
\ea %TM:
 \glll  urakjabəiga  un  xxx ..  atu   cɨzjun  aran?\\
      [\textit{urakja=bəi=ga}  \textit{u-n}    \textit{atu}  \textit{cɨg-tur-n}]\textsubscript{Adnominal clause}  \textit{ar-an}\\
      2.NHON.PL=only=NOM  MES-ADNZ    after  succeed-PROG-PTCP  COP-NEG\\
      \glt       ‘Only you have inherited [i.e. your grandfather’s virtue], haven’t you [lit. aren’t you]?’ [Co: 120415\_01.txt]
  \ex\relax  [Context: Speaking of the outdoor lamps which was set in the past]
%TM:
    \glll  namanin  an  aran?\\
      [\textit{nama=n=n}  \textit{ar-n}]\textsubscript{Adnominal clause}  \textit{ar-an}\\
      now=DAT1=also  exist-PTCP  COP-NEG\\
      \glt       ‘There are (outdoor lamps) even now, aren’t there?’ [Co: 120415\_00.txt]
\ex \label{ex:9.57c} %TM:
    \glll  {\textbar}teinenmade{\textbar}  wutan  aran?\\
      [\textit{teinen=made}  \textit{wur-tar-n}]\textsubscript{Adnominal clause}\textbf{  }\textit{ar-an}\\
      retirement.age=LMT  exist-PST-PTCP  COP-NEG\\
      \glt       ‘(He) was (at work) until the retirement age, wasn’t (he)?’ [Co: 110328\_00.txt]

\ex \label{ex:9.57d} % TM: 
 \glll  {\textbar}tosjogakari{\textbar}  jatan  aran?\\
      [\textit{tasjogakari}  \textit{jar-tar-n}]\textsubscript{Adnominal clause}  \textit{ar-an}\\
      librarian  COP-PST-PTCP  COP-NEG\\
      \glt       ‘(Your father) was a librarian, wasn’t he?’ [Co: 120415\_01.txt]

\ex \label{ex:9.57e} % TM: 
 \glll  {\textbar}iciban{\textbar}  dujasa  atan  aran?\\
      [\textit{iciban}  \textit{duja-sa}  \textit{ar-tar-n}]\textsubscript{Adnominal clause}  \textit{ar-an}\\
      most  rich-ADJ  STV-PST-PTCP  COP-NEG\\
      \glt       ‘(Your grandfather) was the most rich, isn’t (he)?’ [Co: 120415\_01.txt]

\exi{}  Adnominal clause including \textit{-an} (NEG) in the nominal predicate phrase
\ex\relax [Context: Speaking of people who were friends before]\\
%TM:
    \glll  jurawan  aran?\\
      [\textit{juraw-an}]\textsubscript{Adnominal clause}  \textit{ar-an}\\
      get.together-NEG  COP-NEG\\
      \glt       ‘(They) don’t get together (now), do (they) [lit. arn’t (they)]?’ [Co: 120415\_01.txt]

\ex  %TM:  
     \glll namanu  cˀjunkjoo  gan  sjan  {\textbar}kansin{\textbar}na  mutan  aran?\\
      [\textit{nama=nu}  \textit{cˀju=nkja=ja}  \textit{ga-n}  \textit{sɨr-tar-n} \textit{kansin=ja}  \textit{mut-an}]\textsubscript{Adnominal clause}  \textit{ar-an}\\
      now=GEN  person=APPR=TOP  MES-ADVZ  do-PST-PTCP  interest=TOP  have-NEG  COP-NEG\\      
      \glt ‘The people in these days don’t have such a kind of interest, do (they) [lit. aren’t (they)]?’     [Co: 120415\_01.txt]
    \z
\z

In (9-57 a-e), the heads of the nominal predicates are filled by the adnominal clauses that include \textit{-n} (PTCP), i.e. \textit{cɨg-tur-n} (succeed-PROG-PTCP), \textit{ar-n} (exist-PTCP), \textit{wur-tar-n} (exist-PST-PTCP), \textit{jar-tar-n} (COP-PST-PTCP) and \textit{ar-tar-n} (STV-PST-PTCP). In (9-57 f-g), the heads of the nominal predicates are filled by the adnominal clauses that include \textit{-an} (NEG), i.e. \textit{juraw-an} (get.together-NEG) and \textit{mut-an} (have-NEG). These adnominal clauses are followed by the copula verb \textit{ar-an} (COP-NEG) with questional intonation, and have a kind of meaning similar to the tag question in English. In these examples, the copula verb \textit{ar-an} (COP-NEG) does not seem to fill the predicate phrase of the main clause; rather, it seems to behave as a particle, and the preceding adnominal clause seems to become the main clause. In the ordinary construction, the NP that precedes the negative copula verb \textit{ar-an} (COP-NEG) takes either the topic marker \textit{ja} (see (9-54) in \sectref{sec:9.3.1}) or the nominative case (see \sectref{sec:9.3.3.1}). In the examples in (9-57 a-g), however, the adnominal clauses in the predicate phrase do not take any particle, and they are directly followed by the copula verb. It is probable that these examples express the so-called “Mermaid construction (MMC),” which “is in the main confined to Asia, and that it is generally found in SOV languages” \citep[25]{Tsunoda2013}. The prototype of MMC has the following construction “[Clause] Noun Copula” \citep[16]{Tsunoda2013}. In short, the “Clause” seems to behave like the main clause, and the “Noun” and/or the “Copula” seems to behave a grammarical component, e.g. expressing a modal meaning (see \citealt{Tsunoda2013} for more details). The examples in (9-57 a-g) are similar to the MMC, since the adnominal clauses do not behave like the component of the nominal predicate phrase. Rather, they behave like the main clause by themselves, and the following copula verbs express a kind of supposition with the questional intonation. The “main-clausehood” of the adnoninal clause in the MMC in Yuwan is shown by the following examples.

\ea   Honorific AVC in MMC \label{ex:9.58}
\ea In affirmative\\
%TM:
 \gllll  an  sinsjeija  kacjɨ  moojun  aran?\\
    \textit{a-n}  \textit{sinsjei=ja}  [\textit{kak-tɨ}  \textit{moor-jur-n}]\textsubscript{Adnominal clause}  \textit{ar-an}\\
    DIST-ADNZ  teacher=TOP  write-SEQ  HON-UMRK-PTCP  COP-NEG\\
        Lex. verb  Aux. verb\\
    \glt     ‘That teacher would write (the Chinese character), wouldn’t (he) [lit. isn’t (he)]?’ [El: 130823]

\ex  In negative\\
%TM:
 \gllll  an  sinsjeija  kacjɨ  mooran  aran?\\
    [\textit{a-n}  \textit{sinsjei=ja}  \textit{kak-tɨ}  \textit{moor-an}]\textsubscript{Adnominal clause}  \textit{ar-an}\\
    DIST-ADNZ  teacher=TOP  write-SEQ  HON-NEG  COP-NEG\\
        Lex. verb  Aux. verb  \\
    \glt     ‘That teacher would not write (the Chinese character), would (he) [lit. isn’t (he)]?’ [El: 130821]
    \z
\z

The above examples show that the honorific AVCs appear in the predicates of the adnominal clauses (not those in the main clause, i.e. the copula verb). In fact, the speaker did not allow the copula verbs to take the honorific AVC in the above contexts. That is, the following sentence is not grammatical: */an sinsjei ja kakjun atɨ mooran?/ \textit{a-n} \textit{sinsjei=ja} \textit{kak-jur-n} \textit{\Highlight{ar-tɨ}} \textit{\Highlight{moor}-an} (DIST-ADNZ teacher=TOP write-UMRK-PTCP \Highlight{COP-SEQ HON}-NEG) [Intended meaning] ‘That teacher would write (the Chinese character), wouldn’t (he)?’ It is probable that the copula verbs in the MMC in Yuwan have come to lose the qualification to fill the predicate slot of the main clause, and that the predicate in the adnominal clause have come to gain the qualification. It should be mentioned that the MMCs in Yuwan do not coincide with the prototype of MMC since they lack the slot of “Noun”, and the adnominal clauses directly precede the copula verb. The examples which also lack the “Noun” are found in Early Middle Japanese (A.D. 800-1200) \citep[203-205]{Miyachi2013}.

Yuwan has a structure where an infinitive fills the head of the nominal predicate phrase. In the structure, the subject does not belong to the infinitive, but to the copula verb (see (8-114) in \sectref{sec:8.4.4.2}). On the contrary, the subjects of the constructions in (9-57 a-g) do not belong to the copula verb, but is included in the adnominal clause, which is attested by the following example.

\ea    \label{ex:9.59}
%TM:
 \glll  naa  maganu  kamjun  aran?\\
    [\textit{naa}  \textit{maga=\Highlight{nu}}  \textit{kam-jur-n}]\textsubscript{Adnominal clause}  \textit{ar-an}\\
    2.HON.SG.ADNZ  grandchild=NOM  eat-UMRK-PTCP  COP-NEG\\
    \glt     ‘Your grandson will eat (it), won’t [lit. isn’t] he?’ [El: 130816]
\z

In (9-59), the subject, i.e. \textit{naa} \textit{maga} ‘your grandchild,’ is marked by the nominative case \textit{nu}. If the subject is that of the copula verb, it cannot take \textit{nu} (NOM), and it has to take \textit{ga} (NOM) (see \sectref{sec:6.4.3.2} for more details). Therefore, the subject NP is included in the adnominal clause, whose head is \textit{kam-} ‘eat.’

There is an example where the quotational particle \textit{ccjɨ} intervene the adnominal clause and the copula verb \textit{ar-an} (COP-NEG) as in (9-60). 

\ea\relax   [Context: Remembering the bankruptcy of a shop in the past] = (4-31 a) \label{ex:9.60}
%TM:
 \glll  {\textbar}sjeiri{\textbar}  siimai  jatancjɨ  aran?\\
    [\textit{sjeiri}  \textit{sɨr-i+mai}  \textit{jar-tar-n}]\textsubscript{Adnominal clause}\textit{=\Highlight{ccjɨ}}  \textit{ar-an}\\
    disposal  do-INF+OBL  COP-PST-PTCP=QT  COP-NEG\\
    \glt     ‘(The people who had invested their money in the shop) had to dispose (of the goods), hadn’t they [lit. aren’t they]?’ [Co: 120415\_01.txt]
\z

  All of the above examples expressed questions. There are examples where the same construction does not express questions. They did not occur frequently in my texts, though.

\ea   In the declarative clauses \label{ex:9.61}
\ea %TM:
\glll   wurancjəə  aranban,\\
      [\textit{wur-an}]\textsubscript{Adnominal clause}\textit{=ccjɨ=ja}  \textit{ar-an=ban}\\
      exist-NEG=QT=TOP  COP-NEG=ADVRS\\
      \glt       ‘(It) isn’t that there isn’t (any cousin of mine), but ...’ [Co: 120415\_01.txt]

\ex\relax [Context: Replying the question such as “You don’t like the drink, do you?”]\\
%TM:
    \glll  numanna  arandoo.\\
      [\textit{num-an}]\textsubscript{Adnominal clause}\textit{=ja}  \textit{ar-an=doo}\\
      drink-NEG =TOP  COP-NEG=ASS\\
      \glt       ‘(It) isn’t (that I) don’t drink (it).’ [El: 120917]
    \z
\z

In (9-61 a-b), the copula verb \textit{ar-an} (COP-NEG) denies the proposition of the adnominal clauses as a whole. In the declarative clauses, I have not yet found examples where the head of the adnominal clause is filled by the participle that ends with \textit{{}-n} (PTCP).

\subsubsection{Adverbial clause whose head verb ends with \textit{{}-tɨ} (SEQ) in the nominal predicate phrase}\label{sec:9.3.2.2}

The adverbial clause whose head verb ends with \textit{{}-tɨ} (SEQ) can fill the head slot of the nominal predicate phrase. In that case, we can use any variant of the copula verbs, i.e. \textit{jar-}, \textit{ar-}, \textit{zjar-}, or \textit{nar-} as in (9-62 a-f).

\ea   Complements filled by the converb that ends with \textit{{}-tɨ} (SEQ) \label{ex:9.62}
 \exi Converb followed by \textit{jar-} (COP)
\ea %TM:
  \glll   attu  wattəə  jatɨn,  wutɨ   jatɨn.joo ..\\
      \textit{a-rɨ=tu}  \textit{wattəə}  \textit{jar-tɨ=n}  [\textit{wur-tɨ}]\textsubscript{Adverbial clause}    \textit{jar-tɨ=n=joo}\\
      DIST-NLZ=COM  1DU  COP-SEQ=even  exist-SEQ    COP-SEQ=even=CFM1\\
      \glt       ‘Even if there were two of us, (even if we) were (together) ...’ [Co: 120415\_01.txt]
\ex %TM:
    \glll  {\textbar}kˀuusjuu{\textbar}sjɨ  jakɨtɨ  jappajaa.\\
      [\textit{kˀuusjuu=sjɨ}  \textit{jakɨr-tɨ}]\textsubscript{Adverbial clause}  \textit{jar-ba=jaa}\\
      air.raid=INST  burn-SEQ  COP-CSL=SOL\\
      \glt       ‘The air raid (in World War II) burned (the banyan tree), so (there isn’t any tree).’ [Co: 111113\_02.txt]

\ex %US:
      \glll ɨɨ,  ɨɨ,  ɨɨ,  gan  sjɨ  gan   sjɨ  jata.\\
      \textit{ɨɨ}  \textit{ɨɨ}  \textit{ɨɨ}  [\textit{ga-n}  \textit{sɨr-tɨ}]\textsubscript{Adverbial clause}  [\textit{ga-n}  \textit{sɨr-tɨ}]\textsubscript{Adverbial clause}  \textit{jar-tar}\\
      yes  yes  yes  MES-ADVZ  do-SEQ  MES-ADVZ       do-SEQ  COP-PST\\
      \glt   ‘Yes, yes, yes. That (is right). That’s (right).’   [Co: 110328\_00.txt]

\exi{} Converb followed by \textit{ar-} (COP)
\ex %TM:
    \glll  namiotankja  dɨruka  xxx  wutəə    arankai?\\
      [\textit{namio-taa=nkja}  \textit{dɨ-ru=ka}    \textit{wur-tɨ}]\textsubscript{Adverbial clause}\textit{=ja}    \textit{ar-an=kai}\\
      Namio-PL=APPR  which-NLZ=DUB    exist-SEQ=TOP   COP-NEG=DUB\\
      \glt       ‘There were Namio and his friends somewhere (in the pictures), weren’t (they)?’ [Co: 120415\_00.txt]

\exi{} Converb followed by \textit{zjar-} (COP)

\ex \relax [= (8-123 a)]%TM:\\
    \glll  kurəə  kunuguru  (sadaega  sɨ)   sadaega  sɨmɨtəətɨ  zja.\\
      [\textit{ku-rɨ=ja}  \textit{kunuguru}  \textit{sadae=ga}  \textit{sɨmɨr}   \textit{sadae=ga}  \textit{sɨmɨr-təər-tɨ}]\textsubscript{Adverbial clause}  \textit{zjar}\\
      PROX-NLZ=TOP  these.days  Sadae=NOM  do.CAUS   Sadae=NOM  do.CAUS-RSL-SEQ  COP\\      
      \glt ‘This one [i.e. a picture] is (what) Sadae has made (my son) do [i.e. enlarge] these days.’   [Co: 120415\_00.txt]

\exi{}  Converb followed by \textit{nar-} (COP)

\ex %TM: 
     \glll  gan  sjɨ  natɨ,  simabanasinkjoo  sɨraran.\\
      [\textit{ga-n}  \textit{sɨr-tɨ}]\textsubscript{Adverbial clause}  \textit{nar-tɨ}  \textit{sima+hanasi=nkja=ja}  \textit{sɨr-ar-an}\\
      MES-ADVZ  do-SEQ  COP-SEQ  community+story=APPR=TOP   do-CAP-NEG\\
      \glt       ‘Therefore, (I) cannot do [i.e. tell] a story about (our) community.’ [Co: 120415\_01.txt]
    \z
\z

The above examples show that if the head of the nominal predicate pharase is filled by the adverbial clause that ends with \textit{{}-tɨ} (SEQ), there is no constraint on the variants of the copula verbs, which is largely different from the case of the adnominal clause in \sectref{sec:9.3.2.1}, which can take only \textit{ar-} (COP). In fact, the adverbial clause that precedes \textit{nar-} (COP) is only /gan sjɨ/ \textit{ga-n} \textit{sɨr-tɨ} (MES-ADVZ do-SEQ) ‘like this’ in almost all of the examples in my corpus, and the combination of \textit{ga-n} \textit{sɨr-tɨ} (MES-ADVZ do-SEQ) and \textit{nar-tɨ} (COP-SEQ) functions like a conjunction meaninig ‘therefore’ as a whole as in (9-62 f). Interestingly, the function of the adverbial clause composed of \textit{{}-tɨ} (SEQ) and the copula verb \textit{ar-an} (COP-NEG) is very similar to that of the adnomina clause \textit{{}-tar-n} (PST-PTCP) and the copula verb \textit{ar-an} (COP-NEG). For example, the converb \textit{wur-tɨ} (exist-SEQ) in (9-62 d) fills the head slot of the adverbial clause, which fills in turn the nominal predicate phrase with \textit{ar-an} (COP-NEG), where the converbal affix \textit{{}-tɨ} (SEQ) expresses the past tense (see also \sectref{sec:11.2.1}). Therefore, the meaning of /wutəə aran/ \textit{wur-tɨ=ja} \textit{ar-an} (exist-SEQ=TOP COP-NEG) in (9-62 d) is very similar to /wutan aran/ \textit{wur-tar-n} \textit{ar-an} (exist-PST-PTCP COP-NEG) of (9-57 c) in \sectref{sec:9.3.2.1}, where the past tense affix \textit{-tar} is used.

Yuwan has a structure where an infinitive fills the head of the nominal predicate phrase. In the structure, the subject does not belong to the infinitive, but to the copula verb (see (8-114) in \sectref{sec:8.4.4.2}). On the contrary, the subjects of the constructions as in (9-62 a-f) do not belong to the copula verb, but is included in the adverbial clause, which is attested by the following example.

\ea  \label{ex:9.63}
%TM:
 \glll  naa  maganu  kadəə  aranna?\\
    [\textit{naa}  \textit{maga=\Highlight{nu}}  \textit{kam-tɨ}]\textsubscript{Adverbial clause}\textit{=ja}  \textit{ar-an=na}\\
    2.HON.SG.ADNZ  grandchild=NOM  eat-UMRK-PTCP  COP-NEG=PLQ\\
    \glt     ‘Your grandson ate (it), didn’t (he)? [lit. aren’t (he)?]’ [El: 130820]
\z

In (9-63), the subject, i.e. \textit{naa} \textit{maga} ‘your grandchild,’ is marked by the nominative case \textit{nu}. If the subject is that of the copula verb, it cannot take \textit{nu} (NOM), and it has to take \textit{ga} (NOM) (see \sectref{sec:6.4.3.2} for more details). Therefore, the subject NP is included in the adverbial clause, whose head is \textit{kam-} ‘eat.’ This is similar to (9-59) in \sectref{sec:9.3.2.1}.

Considering the above examples, the converb \textit{{}-tɨ} (SEQ) seems to have some nominal property, since it can be followed by a copula verb as in (9-62 a-f). Additionally, there are other examples where the converb \textit{{}-tɨ} (SEQ) behaves like the nominal. For example, the converb \textit{{}-tɨ} (SEQ) can take the nominative case in a certain AVC (see (6-48) in \sectref{sec:6.3.2.1} and (9-8) in \sectref{sec:9.1.1.1}). Moreover, the converbal affix \textit{{}-əəra} ‘after’ can be thought to originate from *\textit{{}-tɨ=kara} (SEQ=ABL) considering the morphophonological rule in \sectref{sec:6.3.1.2}. In fact, the converbal affix \textit{{}-əəra} ‘after’ can take the genitive case \textit{nu} as in (8-100 d) in \sectref{sec:8.4.3.4}.

\subsection{Extended NP in the predicate phrase}\label{sec:9.3.3}

The extended NP is the NP that is followed by case particles (see chapter 6). A nominal predicate phrase is usually filled by an NP not followed by any case particle as in (9-52) - (9-54). However, there are two cases where the predicate may be filled by an NP followed by a case particle (i.e. an extended NP). They are discussed in \sectref{sec:9.3.3.1} and \sectref{sec:9.3.3.2} respectively.

\subsubsection{Nominative case in the subordinate clause in negative}\label{sec:9.3.3.1}

The NP in the predicate takes \textit{ja} (TOP) when the following copula is in negative in the main clause as in (9-54). However, if the predicate NP is in the subordinate clause and also in negative, it may take the nominative case \textit{ga} or \textit{nu} as in (9-64 a-e).

\ea   Nominative case in the nominal predicate phrases \label{ex:9.64}
\ea \relax [= (5-9 b)]
%TM:
    \gllll  uraga  tumainu  aran  tukin,\\
      \textit{ura=ga}  \textit{\Highlight{tumai=nu}}  \textit{\Highlight{ar-an}}  \textit{tuki=n}\\
      2.NHON.SG=NOM  night.duty=NOM  COP-NEG  time=DAT1\\
      Subject  [NP  Copula verb]\textsubscript{Nomimal predicate phrase}  \\
      \glt       ‘When you are not on night duty, ...’ [Co: 111113\_02.txt]

\ex %TM: 
     \gllll waakjaga  (mm)  arɨnu  aranboo, naacɨbanu  aranboo,\\
      \textit{waakja=ga}    \textit{\Highlight{a-rɨ=nu}}  \textit{\Highlight{ar-an-boo}}  \textit{\Highlight{naacɨba=nu}}  \textit{\Highlight{ar-an-boo}}\\                                                                  
      1PL=NOM                  DIST-NLZ=NOM  COP-NEG-CND              tone.deaf=NOM  COP-NEG-CND\\
      Subject    [NP  Copula verb]\textsubscript{Nomimal predicate phrase} [NP  Copula verb]\textsubscript{Nomimal predicate phrase}\\
      \glt  ‘If I am not that, (that is to say) if (I) am not tone deaf, ...’   [Co: 111113\_01.txt]
\ex \label{ex:.c} %TM:
    \gllll  namanən  sjɨ,  (ee)  ..  urɨga  aranba,\\
      \textit{nama=nən}  \textit{sɨr-tɨ}      \textit{\Highlight{u-rɨ=ga}}  \textit{\Highlight{ar-an-ba}}\\
      now=LOC1  do-SEQ      MES-NLZ=NOM  COP-NEG-CSL\\
              [NP  Copula verb]\textsubscript{Nomimal predicate phrase}\\
      \glt       ‘(The compulsory education) wasn’t [i.e. wasn’t conducted for nine years] like (it is) these days, so ...’ [Co: 120415\_00.txt]

\ex  %TM: 
     \glll mata  {\textbar}honnin{\textbar}nu  kjuranɨsəənu   aranboo,  ikjaran.\\
      \textit{mata}  \textit{honnin=nu}  \textit{\Highlight{kjura+nɨsəə=nu}}  \textit{\Highlight{ar-an-boo}}  \textit{ik-ar-an}\\                                                                 
      moreover  oneself=NOM  beautiful+young.man=NOM              COP-NEG-CND  go-CAP-NEG\\                                                                 
        Subject  [NP                                              Copula verb]\textsubscript{Nomimal predicate phrase}\\
    \glt ‘Moreoever, if the (person) himself is not a beautiful young man, (he) cannot go (to) [i.e. become] (an Imperial Guard).’   [Co: 120415\_00.txt]

\ex %TM:
\gllll   haroozinkjaga  aranbajaa.\\
      \textit{\Highlight{haroozi=nkja=ga}}  \textit{\Highlight{ar-an-ba}=jaa}\\
      relative=APPR=NOM  COP-NEG-CSL=SOL\\
      [NP  Copula verb]\textsubscript{Nomimal predicate phrase}\\
      \glt       ‘(They) are not relatives, so (one of them did not attend the marriage ceremony).’ [Co: 120415\_01.txt]
    \z
\z

In the above examples, the NPs in the predicate phrases take the nominative case \textit{ga} or \textit{nu}. All of the copula verbs in (9-64 a-e) take the negative affix \textit{{}-an} (NEG), and all of the predicates are in the subordinate clauses. (9-64 a) is in the adnominal clause whose external head is \textit{tuki} ‘time,’ and (9-64 b-e) are in the adverbial clauses. If the copula verbs do not take negative affixes, the NP in the predicate does not take the nominative cases as in (8-36 e) in \sectref{sec:8.3.3.1}. The selection of the nominative particles, i.e. \textit{ga} or \textit{nu}, depends on the relation between the head nominal in the NP and the animacy hierarchy (see \sectref{sec:6.4} for more details). However, it is irregular that the predicate NPs in (9-64 c) and (9-64 e), i.e. \textit{u-rɨ} ‘that (educational system)’ and \textit{haroozi} ‘relative,’ take \textit{ga} (not \textit{nu}), since inanimate referents or the human common nouns cannot take \textit{ga} in principle.

  The same phenomenon may occur in the adjectival predicate, although it has not occurred in the text corpus (i.e., it occurred only in elicitation).

\ea   Nominative case in the adjectival predicate phrase \label{ex:9.65}
%TM:
 \gllll  utussjanu  nənba,  micjɨn  njɨ!\\
    \textit{\Highlight{utussj-sa=nu}}  \textit{\Highlight{nə-an-ba}}  \textit{mj-tɨ=n}  \textit{nj-ɨ}\\
    frightening-ADJ=NOM  STV-NEG-CSL  see-SEQ=ever  EXP-IMP\\
    [Adjective  Stative verb]\textsubscript{Adjectival predicate phrase}    \\
    \glt     ‘(It) is not frightening, so try to see (it)!’ [El: 130822]
\z

In fact, the speaker utters naturally a sentence where /utussjanu/ \textit{utusssj-sa=nu} (frightening-ADJ=NOM) in (9-65) is replaced by /utussjoo/ \textit{utussj-soo} (frightening-ADJ).

\subsubsection{Cleft-like (or pseudo-cleft-like) construction}\label{sec:9.3.3.2}

Other than the examples discussed above, there are a few examples where extended NPs fill the predicate phrases as in (9-66 a-b).

\ea   Extended NP in the predicate phrases \label{ex:9.66}
\ea %TM:
 \gllll  kurɨ  kumantɨ  zjajaa.\\
      \textit{ku-rɨ}  \textit{\Highlight{ku-ma=nantɨ}}  \textit{zjar=jaa}\\
      PROX-NLZ  PROX-place=LOC1  COP=SOL\\
        [Extended NP  Copula verb]\textsubscript{Nomimal predicate phrase}\\
      \glt       ‘(The place where) this [i.e. the sumo wrestling] (was held) is at this place.’ [Co: 120415\_00.txt]
\ex %TM: 
  \glll kan  sjɨ  jaanu  dɨkəə    {\textbar}nannengoro{\textbar}karakai?\\
      \textit{ka-n}  \textit{sɨr-tɨ}  \textit{jaa=nu}  \textit{dɨkɨr-∅=ja} \textit{\Highlight{nannen-goro=kara}=kai}\\                                                                           
      PROX-ADVZ  do-SEQ  house=GEN  be.built-INF=TOP                       what.year-about=ABL=DUB\\                                                                           
                                                                           [Extended NP]\textsubscript{Nomimal predicate phrase}\\
      \glt ‘Since when did the houses like these (begin to) be built?’ [lit. ‘From about what year (was) the houses’ being built like these.’]   [Co: 110328\_00.txt]
    \z
\z

Probably, the extended NPs in (9-66 a-b) are arguments that are focused and derived from the “original” sentences where the extended NPs fill the ordinary slots, i.e. argument slots, in the clauses. These constructions seem to have some relationship with the “clefts” or “pseudo-clefts” in the languages around the world (cf. \citealt{Payne1997}: 278-281), and more elaborate research remains to be done.

\section{Argumentations for the suggested differences among the predicate phrases}\label{sec:9.4}

The structural differences (or analyses) among the three types of predicate phrases have so far examined in the previous sections. However, one may think that a type of the predicate phrases may be analyzed as another type of them. For example, one may ask if the adjectival predicate, e.g. /arəə sijusa/ \textit{a-rɨ=ja} \textit{siju-sa} (DIST-NLZ=TOP white-ADJ) ‘That is white.’ is really different from the nominal predicate, e.g. /arəə kasa/ \textit{a-rɨ=ja} \textit{kasa} (DIST-NLZ=TOP hat) ‘That is a hat.’

In this section, I will present the arguments for the suggested analyses that the three types of the predicate phrases are different from one another. The differences between the adjectival predicate and the nominal predicate are discussed in \sectref{sec:9.4.1}. The differences between the adjectival predicate and the verbal predicate are discussed in \sectref{sec:9.4.2}. The differences between the nominal predicate and the verbal predicate are discussed in \sectref{sec:9.4.3}.

\subsection{The differences between the adjectival predicate and the nominal predicate}\label{sec:9.4.1}

There are four differences between the adjectival predicate and the nominal predicate as in the following table.

\begin{table}
\caption{\label{tab:94}Morphosyntactic differences between the adjectival predicate and the nominal predicate}
\begin{tabularx}{\textwidth}{Qcc}
\lsptoprule
& Adjectival predicate  & Nominal predicate\\\midrule
Can appear in the adnominal clause in the non-past tense & + & \textminus\\
Can be followed by \textit{nu} (CSL) & + & \textminus\\
The head can directly take \textit{na} (PLQ), \textit{kai} (DUB), or \textit{doo} (ASS) & \textminus &  +\\
Take different verbal forms in the predicate phrase & \textit{ar-}/\textit{nə-} & \textit{jar-}/\textit{zjar-}/\textit{nar-}/\textit{ar-}\\
\lspbottomrule
\end{tabularx}
\end{table}

  Firstly, the adjectival predicate can appear in the adnominal clause in the non-past tense as in (9-67 a), but the nominal predicate cannot as in (9-67 b).

\ea  
\exi{} Adnominal clause in the non-past tense \label{ex:9.67}
\ea Adjectival predicate\\
%TM:
 \glll  kjurasan  nɨsəə\\
    [\textit{kjura-sa+ar-n}]\textsubscript{Adnominal clause} \textit{nɨsəə}\\
    beautiful-ADJ+STV-PTCP  young.man\\
    \glt     ‘a young man who is beautiful’ [El: 130822]

\ex Nominal predicate\\
%TM:
 \glll  *{\textbar}sinsjei{\textbar}  jan/zjan  nɨsəə\\
    [\textit{sinsjei}  \textit{jar-n}/\textit{zjar-n}]\textsubscript{Adnominal clause}  \textit{nɨsəə}\\
    teacher  COP-PTCP/COP-PTCP  young.man\\
    \glt     [Intended meaning] ‘a person who is a teacher’ [El: 130822]

\exi{} Adnominal clause in the past tense

\ex Adjectival predicate\\
%TM:
 \glll  kjurasa  atan  nɨsəə\\
    [\textit{kjura-sa}  \textit{ar-tar-n}]\textsubscript{Adnominal clause}  \textit{nɨsəə}\\
    beautiful-ADJ  STV-PST-PTCP  young.man\\
    \glt     ‘a young man who was beautiful’ [El: 130822]

\ex Nominal predicate\\
%TM:
 \glll  {\textbar}sinsjei{\textbar}  jatan  nɨsəə\\
    [\textit{sinsjei}  \textit{jar-tar-n}]\textsubscript{Adnominal clause}  \textit{nɨsəə}\\
    teacher  COP-PST-PTCP  young.man\\
    \glt      ‘a young man who was a teacher’ [El: 130822]
    \z
\z

The above examples show that the stative verbal root \textit{ar-} can take both \textit{{}-n} (PTCP) as in (9-67 a) and \textit{{}-tar-n} (PST-PTCP) as in (9-67 c). On the contrary, the copula verbal root \textit{jar-} (or \textit{zjar-}) cannot (directly) take \textit{{}-n} (PTCP) as in (9-67 b), although it can take \textit{{}-tar-n} (PST-PTCP) as in (9-67 d). In other words, the subject of the nominal predicate in the non-past tence in affirmative cannot be relativised.

Secondly, the adjectival predicate can take the conjunctive particle \textit{nu} (CSL) as in (9-68 a), but the nominal predicate cannot as in (9-68 b).

\ea   \label{ex:9.68}
\ea Adjectival predicate + \textit{nu} (CSL) [= (9-44 c)]\\
%TM:
 \glll  waakjoo  utussjanu,  aicjɨn  njanta.\\
    \textit{waakja=ja}  \textit{\Highlight{utussj-sa=nu}}  \textit{aik-tɨ=n}  \textit{nj-an-tar}\\
    1PL=TOP  fearful-ADJ=CSL  walk-SEQ=ever  EXP-NEG-PST\\
    \glt     ‘I was fearful (of the American soldiers), so I did not walk (around).’ [Co: 111113\_01.txt]

\ex Nominal predicate + \textit{nu} (CSL)\\
%TM:
 \glll  *arəə  warabɨnu,  waarandaro.\\
    \textit{a-rɨ=ja}  \textit{\Highlight{warabɨ=nu}}  \textit{waar-an=daro}\\
    DIST-NLZ=TOP  child=CSL  understand-NEG=SUPP\\
    \glt      [Intended meaning] ‘That one is a child, so (he) maybe does not understand (it).’ [El: 130822]
    \z
\z

In fact, the conjunctive particle \textit{nu} (CSL) has the same form with the nominative case particle \textit{nu} (NOM). However, the nominative particle \textit{nu} cannot express the causal meaning as in (9-68 b). Thus, \textit{nu} (NOM) is different from \textit{nu} (CSL), and the latter cannot attach to the nominal predicate.

Thirdly, the head NP in the nominal predicate can be directly followed by a few clause-final particles, i.e. \textit{na} (PLQ), \textit{kai} (DUB), or \textit{doo} (ASS) as in (9-69 a). On the contrary, the head adjective in the adjectival predicate cannot as in (9-69 b).

\ea \label{ex:9.69}
\exi{} Nominal predicate 
\ea %TM:
 \gllll  arəə  kasana?\\
      \textit{a-rɨ=ja}  \textit{kasa=na}\\
      DIST-NLZ=TOP  hat=PLQ\\
      Subject  Predicate\\
      \glt       ‘Is that a hat?’ [El: 130822]

\exi{} Adjectival predicate
\ex \label{ex:9.69b} %TM:
    \gllll  *arəə  sijusana?\\
      \textit{a-rɨ=ja}  \textit{siju-sa=na}\\
      DIST-NLZ=TOP  white-ADJ=PLQ\\
      Subject  Predicate\\
      \glt       [Intended meaning] ‘Is that white?’ [El: 130822]

\ex \label{ex:9.69c} %TM:
    \gllll  arəə  sijusannja?\\
      \textit{a-rɨ=ja}  \textit{siju-sa+ar-i=na}\\
      DIST-NLZ=TOP  white-ADJ+STV-NPST=PLQ\\
      Subject  Predicate\\
      \glt        ‘Is that white?’ [El: 130822]
    \z
\z

In (9-69 a), the NP in the predicate, i.e. \textit{kasa} ‘hat,’ can be directly followed by the question particle \textit{na} (PLQ). In (9-69 b), however, the adjective in the predicate, i.e. \textit{siju-sa} (white-ADJ), cannot directly take \textit{na} (PLQ). If the adjective is followed by the stative verb \textit{ar-}, the predicate can take \textit{na} (PLQ) as in (9-69 c).

Finally, there is a morphological difference between the verbal forms that appear in the predicate phrase, i.e. the stative verb and the copula verb. The stative verbs \textit{ar-}/\textit{nə-} are used in the adjectival predicate (see \sectref{sec:8.3.4}), and the copula verbs \textit{jar-}/\textit{zjar-}/\textit{nar-}/\textit{ar-} are used in the nominal predicate (see \sectref{sec:8.3.3}).

\subsection{The differences between the adjectival predicate and the verbal predicate}\label{sec:9.4.2}

The stative verbs in the adjectival predicate and the existential verbs in the verbal predicate have the same forms, i.e. /ar-/ and /nə-/ (see \sectref{sec:8.3.2} and \sectref{sec:8.3.4}). However, there are two differences between the adjectival predicate and the verbal predicate as in \tabref{tab:95}.

\begin{table}
\caption{\label{tab:95}Morphosyntactic differences between the adjectival predicate and the verbal predicate}
\begin{tabular}{lcc}
\lsptoprule
&  Adjectival predicate  & Verbal predicate\\\midrule
Contraction between /ar-/ and the preceding morpheme occurs & + & \textminus\\
The word preceding /ar-/ or /nə-/ can take the nominative case & \textminus &  +\\
\lspbottomrule
\end{tabular}
\end{table}

First, the adjective that inflects with \textit{{}-sa} (ADJ) is contracted with the following stative verb \textit{ar-}, if the \textit{ar-} (STV) takes \textit{{}-i} (NPST) or \textit{-n} (PTCP) (see \sectref{sec:9.2.2.2} for more details). The example taking \textit{{}-i} (NPST) is shown in (9-70 a), where the place of contraction is expressed by “+” in the underlying level.

\ea   \label{ex:9.70}
\ea Adjectival predicate [= (9-46 d)]\\
%TM:
    \glll  {\textbar}iciban{\textbar}  dujas\Highlight{a}i.\\
      \textit{iciban}  \textit{duja-s\Highlight{a}+\Highlight{a}r-i}\\
      most  rich-ADJ+STV-NPST\\
      \glt       ‘(He) is the richest.’ [Co: 111113\_01.txt]
\ex Verbal predicate\\
%TM:
    \glll  un  {\textbar}teepu{\textbar}ja  nam\Highlight{a}  \Highlight{a}i?\\
      \textit{u-n}  \textit{teepu=ja}  \textit{nam\Highlight{a}}  \textit{\Highlight{a}r-i}\\
      MES-ADNZ  cassette.tape=TOP  yet  exist-NPST\\
      \glt       ‘Is the cassette tape there [i.e. ready] yet?’ [Co: 120415\_01.txt]
    \z
\z

On the one hand, in (9-70 a), the adjective \textit{duja-sa} (rich-ADJ) and \textit{ar-i} (STV-NPST) induces contraction, and one of the vowel in \textit{{}-sa+ar-} (ADJ+STV) is deleted. On the other hand, in (9-70 b), the existential verb \textit{ar-i} (exist-NPST) does not induce contraction with the preceding morpheme \textit{nama} ‘yet,’ i.e., they do not become */namai/ \textit{nama+ar-i} (yet+exist-NPST).

Secondly, the adjective that precedes a stative verb cannot take the nominative case as in (9-71 a), but the argument NP that precedes existential verbs can take the nominative case as in (9-71 b).

\ea    \label{ex:9.71}
 \ea Adjectival predicate\\
%TM:
    \glll  huntoo  kuwasa  ata.\\
      \textit{huntoo}  \textit{\Highlight{kuwa-sa}}  \textit{\Highlight{ar-tar}}\\
      really  hard-ADJ  STV-PST\\
      \glt       ‘(It) was really hard (for me).’ [Co: 111113\_02.txt]
 \ex Verbal predicate\\
%TM:
    \glll  kˀuranu  ata.\\
      \textit{\Highlight{kˀura=nu}}  \textit{\Highlight{ar-tar}}\\
      storehouse=NOM  exist-PST\\
      \glt       ‘There was a storehouse.’ [Co: 120415\_00.txt]
    \z
\z

In (9-71 a), the adjective \textit{kuwa-sa} (hard-ADJ) does not take any case particle, which means that we cannot analyze the stative verb \textit{ar-} as the existential verb \textit{ar-}, and that the adjective \textit{kuwa-sa} (hard-ADJ) cannot be analyzed as the argument NP of \textit{ar-} ‘exist.’ On the contrary, \textit{kˀura} ‘storehouse’ in (9-71 b) is the argument NP of the existential verb \textit{ar-}. Thus, it takes the nominative case.

\subsection{The differences between the nominal predicate and the verbal predicate}\label{sec:9.4.3}

The head of the nominal predicate is the NP in the predicate (not the following copula verb). On the contrary, the head of the verbal predicate is the VP in the predicate (not its argument NP). This difference is attested by the focus construction, where the focus marker \textit{du} is used (see also \sectref{sec:11.3.1}). If we put the focus on the nominal predicate, it is the NP (not the copula verb) in the predicate which is focused as in (9-72 a). If we put the focus on the verbal predicate, it is the verb in the predicate (not the argument NP) which is focused as in (9-72 b).

\ea   \label{ex:9.72}
\ea Nominal predicate [= (8-39 d)]\\
%TM:
 \gllll  arəə  akiradu  arui?\\
    \textit{a-rɨ=ja}  \textit{\Highlight{akira=du}}  \textit{ar-u=i}\\
    DiST-NLZ=TOP  Akira=FOC  COP-PFC=PLQ\\
      [NP  Copula verb]\textsubscript{Nominal predicate phrase}\\
    \glt     ‘Is that person Akira?’ [El: 130822]

\ex Verbal predicate\\
%TM:
 \gllll  an  cˀjoo  uran  tanmidu  sjurui?\\
    \textit{a-n}  \textit{cˀju=ja}  \textit{ura=n}  \textit{\Highlight{tanm-i=du}}  \textit{sɨr-jur-u=i}\\
    DIST{}-ADNZ  person=TOP  2.NHON.SG=DAT1  ask-INF=FOC  do-UMRK{}-PFC=PLQ\\
      [Complement  VP]\textsubscript{Verbal predicate phrase}\\
    \glt     ‘Does that person ask you (about it)?’ [El: 130822]
    \z
\z

In (9-72 a), the NP (not the copula verb) in the predicate is focused by \textit{du} (FOC). In (9-72 b), the verb \textit{tanm-} ‘ask’ is focused by \textit{du} (FOC), where the focused component fills the complement slot becoming an infinitive, and the head of VP is filled by the light verb \textit{sɨr-} ‘do.’ The latter means cannot be taken by the nominal predicate. Thus, the copula verb \textit{ar-} cannot be followed by \textit{du} (FOC) such as *\textit{ar-i=du} (COP-INF=FOC).

Before concluding this section, I will also present the example where the adjectival predicate is focused by \textit{du} (FOC).

\ea   Adjectival predicate \label{ex:9.73}
%TM:
 \gllll  urəə  kuwasadu  arui?\\
    \textit{u-rɨ=ja}  \textit{\Highlight{kuwa-sa=du}}  \textit{ar-u=i}\\
    MES-NLZ=TOP  hard-ADJ=FOC  STV-PFC=PLQ\\
      \{[Adjective]  [Stative verb]\}\textsubscript{Adjectival predicate phrase}\\
    \glt     ‘Is that (rice cake) hard?’ [El: 130822]
\z

Similarly, the focus marker \textit{du} follows the adjective in the predicate, which indicates that the head of the adjectival predicate phrase is the adjective (not the stative verb).
