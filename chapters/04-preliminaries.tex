\chapter{Descriptive preliminaries}\hypertarget{RefHeadingToc395696990}{}
In this chapter, the basic components in morphosyntax will be addressed. The clause structure and the phrase structure, especially the nominal phrase (NP) and the differences among three types of predicate phrases, will be discussed in §\ref{bkm:Ref381194330}. In §\ref{bkm:Ref381195916}, basic morphological units, i.e. free forms, clitics, and affixes, and combinations of stems, i.e. compounding and reduplication, will be addressed. Finally, the word classes and the criteria to distinguish them will be discussed in §\ref{bkm:Ref306382551}.

\section{Clause structure and phrase structure}
\label{bkm:Ref381194330}\hypertarget{RefHeadingToc395696991}{}
Clause structure is discussed in §\ref{bkm:Ref306660546}, and phrase structures are discussed in §\ref{bkm:Ref306571490} and §\ref{bkm:Ref306571554}.

\subsection{Clause structure}
\label{bkm:Ref306660546}\hypertarget{RefHeadingToc395696992}{}
The canonical word order is SV and APV. Yuwan has a nominative-accusative case marking system. Canonically, S/A arguments are marked by \textit{ga/nu} (NOM), and P argument is marked by \textit{ba} (ACC). Argument NPs that are inferable from the context can be left unstated.

\ea \label{ex:4.1}
\ea Intransitive clause\hfill\relax[Co: 120415\_01.txt]\\
    \relax [Context: Remembering almost twenty years ago; TM: ‘When I was seventy years old, ...’]\\
    {TM:} \glll    hacukosanga  wutɨ,\\
                   [\textit{hacuko-san=ga}]\textsubscript{Argument}  [\textit{wur-tɨ}]\textsubscript{Predicate}\\
                   Hatsuko-HON=NOM  exist-SEQ\\
    \glt\hspaceThis{TM:} ‘There was Ms. Hatsuko, and ...’
\ex Transitive clause\hfill\relax[Co: 101020\_01.txt]\\
   {TM:} \glll  {hirooga}  {{kangɨɨba}}  {kicjɨ,}\\
                {[\textit{hiroo=ga}]\textsubscript{Argument}}  {[\textit{kangɨɨ=ba}]\textsubscript{Argument}}  {[\textit{kij-tɨ}]\textsubscript{Predicate}}\\
                {Hiro=NOM}  {{hedge=ACC}}  {cut-SEQ}\\
\glt\hspaceThis{TM:} ‘Hiro cut the hedge, and ...’
\z
\z
Each argument slot is filled by a nominal phrase (see \sectref{sec:key:4.1.2}). The predicate slot is filled by a verbal, nominal, or adjectival predicate phrase (see \sectref{sec:key:4.1.3}).

It should be noted that the choice between \textit{ga} (NOM) and \textit{nu} (NOM) depends on the lexical meaning (or “animacy hierarchy” in a broad sense) of the head nominal. In other words, the choice between \textit{ga} (NOM) and \textit{nu} (NOM) is not influenced by the meaning of the verbs (e.g., whether the verb is volitional or not). For example, the subject (i.e., /waakjaa anmatankja/ ‘my mother’) of the volitional verb (i.e., /izjɨ cˀjan/ ‘had been’ [lit. ‘go and come back’]) takes \textit{ga} (NOM) as in (6-103 c), as well as the subject (i.e., /tacuu/ ‘Tatsu’) of the non-volitional verb (i.e., /moosjaroo/ ‘passed away’) takes \textit{ga} (NOM) as in (8-24). Similarly, the subject (i.e., /nɨsəə/ ‘young man’) of the volitional verb (i.e., /tuutai/ ‘passed’) takes \textit{nu} (NOM) as in (8-118 a), as well as the subject (i.e., /ireba/ ‘artificial tooth’) of the non-volitional verb (i.e., /utɨjun/ ‘fall’) takes \textit{nu} (NOM) as in (8-90 a). The details about the choice between \textit{ga} (NOM) and \textit{nu} (NOM) will be discussed in \sectref{sec:key:6.4.}

\subsection{Nominal phrase (NP)}
\hypertarget{RefHeadingToc395696993}{}\label{bkm:Ref306571490}
Yuwan has the following nominal phrase (NP) structure.

\ea\relax [(Modifier) Head]\textsubscript{NP}\textit{\textsubscript{} }(=Case) \z

The head slot is obligatory, while the modifier slot is optional in principle (with the exception of the formal noun which will be discussed in §\ref{bkm:Ref367137128}). The head slot is filled by a nominal. A case particle follows the NP. However, there are many situations where case particles do not appear. The nominative case particle does not occur if the NP is followed by \textit{ja} (TOP), \textit{du} (FOC), or \textit{n} ‘also’ (see also \sectref{sec:key:10.1}). Likewise, the genitive case particle does not occur if the head is filled by an address noun (see §\ref{bkm:Ref306485857}), and the accusative case may be omitted after an inanimate nominal (see §\ref{bkm:Ref350963937}). Thus, we propose the core of an NP is the head nominal and not the case particle. An NP that contains a case particle is called an “extended NP” \citep[167]{Shimoji2008}. In this grammar, the label “NP” refers to either the NP (in a narrow sense) or the extended NP.

Syntactically, an NP can function either as a clausal dependent (argument), a clausal head (nominal predicate), or a phrasal modifier (NP in genitive function).

\ea\label{ex:4.2} 
\ea Argument NP\hfill\relax[PF: 090222\_00.txt]\\
   {TM:}\\
   \glll {{jinganu}}  {{{\textbar}hasigo{\textbar}}}  {{kɨɨtɨ,}}  {{nasiba}} {tˀɨɨ}  {{tˀɨɨ}}  {{mutunwakejo.}}\\
     {{[\textit{jinga=nu}]\textsubscript{Argument NP}}}  {{[\textit{hasigo}]\textsubscript{Argument NP}}}  {{\itshape kɨɨr-tɨ}}  {{[\textit{nasi=ba}]\textsubscript{Argument NP}}} {\itshape tˀɨɨ}  {{\itshape tˀɨɨ}}  {{\itshape mur-tur-n=wake=joo}}\\
     {{man=NOM}}  {{ladder}}  {{put-SEQ}}  {{pear=ACC}} {one.CLF}  {{one.CLF}}  {{pick.up-PROG-PTCP=CFP=CFM1}}\\
    \glt ‘A man put a ladder (against a tree) and was picking up pears one by one.’
\ex Nominal predicate\hfill\relax[Co: 120415\_00.txt]\\
   {TM:}\\
     \glll {kun}  {cˀjoo}  {tarukai?}\\
     {[\textit{ku-n}}  {\textit{cˀju}]\textsubscript{Argument NP}\textit{=ja}}  {[\textit{ta-ru}]\textsubscript{Nominal predicate}\textit{=kai}}\\
     {PROX-ADNZ}  {person=TOP}  {who-NLZ=DUB}\\
     \glt ‘Who is this person?’
\ex\label{ex:4.2c} Phrasal modifier\hfill\relax[Co: 110328\_00.txt]\\
   {TM:} \\
     {naakjaa}  {jumɨnu}  {naaja}  {sijandoojaa.}\\
     {\{[\textit{naakjaa}}  {\textit{jumɨ=nu}]\textsubscript{Phrasal modifier}}  {\textit{naa}\}\textsubscript{Argument NP}\textit{=ja}}  {\itshape sij-an=doo=jaa}\\
     {2PL.HON.ADNZ}  {daughter.in.law=GEN}  {name=TOP}  {know-NEG=ASS=SOL}\\
     \glt ‘(I) don’t know the name of your daughter in law.’

\z
\z
In \REF{ex:4.2c}, the NP \textit{naakjaa} \textit{jumɨ} ‘your daughter in law’ is composed of the modifier \textit{naakjaa} (2PL.HON.ADNZ) and the head \textit{jumɨ} ‘daughter in law.’ It functions as a phrasal modifier of the superordinate NP, which is indicated by curly brackets.

The modifier slot of an NP can be filled by an adnominal, adnominal clause, and NP with the genitive case, although address nouns do not take the genitive case. Address nouns are juxtaposed to fill the modifier slot of an NP (see §\ref{bkm:Ref367137362} for more details).


\ea\label{ex:4.3} 
\ea Adnominals\hfill\relax[Co: 110328\_00.txt]\\
   {TM:} \gll \ob\textit{naakjaa}\cb\textsubscript{Modifier}   \ob\textit{jumɨ}\cb\textsubscript{Head}\\
               2PL.HON.ADNZ  daughter.in.law\\
 \glt\hspaceThis{TM:} ‘your daughter in law’
\ex Adnominal clauses\hfill\relax[PF: 090222\_00.txt]\\
   {TM:} \glll {hinzjaa}   succjun  {jinga}\\
     {[\textit{hinzjaa}}   \textit{sukk-tur-n}]\textsubscript{Modifier}  [\textit{jinga}]\textsubscript{Head}\\
     {goat}   pull-PROG-PTCP  {man}\\
      \glt\hspaceThis{TM:} ‘the man who is pulling a goat’
\ex NP with genitive case\hfill\relax[Co: 110328\_00.txt]\\
   {TM:} \gll  [\textit{jumɨ=nu}]\textsubscript{Modifier}   [\textit{naa}]\textsubscript{Head}\\
      daughter.in.law=GEN   name\\
    \glt \glt\hspaceThis{TM:} ‘daughter in law’s name’
\ex Juxtaposition\hfill\relax[Co: 120415\_00.txt]\\
   {TM:}  \gll [\textit{tˀoomu+nii}]\textsubscript{Modifier}   [\textit{baasan}]\textsubscript{Head}\\
      Tsutomu+elder.brother  grandmother\\
      \glt\hspaceThis{TM:} ‘Tsutomu’s grandmother’
\z
\z


\subsection{Predicate phrase}
\label{bkm:Ref306571554}\hypertarget{RefHeadingToc395696994}{}
A predicate phrase appears clause-finally, and there are three subtypes of predicate phrase in Yuwan: verbal predicates, adjectival predicates, and nominal predicates.

\ea Three subtypes of predicate phrase\\
\ea 
\begin{tabbing}
Adjectival predicate phrase\hspace{1em}\= (Complement)\hspace{1em}\= (COP)\kill
Verbal predicate phrase     \> (Complement) \>  VP
\end{tabbing}
\ex 
\begin{tabbing}
Adjectival predicate phrase\hspace{1em}\= (Complement)\hspace{1em}\= (COP)\kill
Adjectival predicate phrase\>A\>(STV)
\end{tabbing}
\ex 
\begin{tabbing}
Adjectival predicate phrase\hspace{1em}\= (Complement)\hspace{1em}\= (COP)\kill
Nominal predicate phrase    \>  NP          \> (COP)
\end{tabbing}
\z

Notes:\\
\begin{description}[font=\normalfont]
\item[“VP”]  indicates the verbal phrase;
\item[“A”]   indicates the adjective;
\item[“STV”] indicates a stative verb;
\item[“COP”] indicates a copular verb.
\end{description}
\z


The verbal predicate is discussed in §\ref{bkm:Ref365019761}. The adjectival predicate is discussed in §\ref{bkm:Ref347240074}. The nominal predicate is discussed in §\ref{bkm:Ref347240067}. For more details, see \chapref{chap:9}.

\subsubsection{Verbal predicate}
\label{bkm:Ref365019761}\hypertarget{RefHeadingToc395696995}{}
A verbal predicate phrase is composed of a verbal phrase (VP) and optionally a complement as schematized in \REF{ex:4.5} (see §\ref{bkm:Ref357004771} for more details).

\ea\label{ex:4.5}
The structure of the verbal predicate phrase\\
\relax[(Complement) \hspace{1em} VP]\textsubscript{Verbal predicate phrase}
\z

A VP is composed minimally of a lexical verb as in \REF{ex:4.6}.

\ea\label{ex:4.6} Minimal VP\hfill\relax[Co: 120415\_01.txt]\\
TM: \glll {\itshape kam-ɨ!}\\
    eat-IMP\\
    {Lex.\ verb}\\
\glt\hspaceThis{TM:} ‘Eat (it)!’
\z

The VP may be composed of a lexical verb and an auxiliary verb as in \REF{ex:4.7}, which is called the auxiliary verb construction (AVC) (see §\ref{bkm:Ref361002248}).

\ea\label{ex:4.7} Auxiliary verb construction\hfill\relax[Co: 120415\_00.txt]\\
 {TM:} \gllll cˀjɨ  kurɨran?\\
   {\itshape k-tɨ}  {\itshape kurɨr-an}\\
   {come-SEQ}  {BEN-NEG}\\
   {Lex. verb}  {Aux. verb}\\
\glt\hspaceThis{TM:} ‘Will you come (to my son’s place)?’
\z


The light verbs \textit{sɨr-} ‘do’ and \textit{nar-} ‘become’ obligatorily take complements. This structure is called the light verb construction (see §\ref{bkm:Ref353886187}).

\ea\label{ex:4.8} Light verb construction\\
\ea \textit{sɨr-} ‘do’\hfill\relax[Co: 111113\_02.txt]\\
TM: \gllll   jˀiija   sɨranban,\\
   {\itshape jˀ-i=ja}   \textit{\Emph{sɨr}-an=ban}\\
    say-INF=TOP   do-NEG=ADVRS\\
    Complement   LV\\
\glt\hspaceThis{TM:} ‘(They) wouldn’t say (so), but ...’
\ex \textit{nar-} ‘become’\hfill\relax[Co: 120415\_01.txt]\\
TM: \gllll {joo,}   huccju {nappoojoo,}  {adoorɨtɨjo,}\\
   {\itshape joo}  {\itshape huccju}  {\textit{\Emph{nar}-boo=joo}}  {\itshape adoorɨr-tɨ=joo}\\
   {FIL}   old.person  {become-CND=CFM1}  {trip.over-SEQ=CFM1}\\
      {} Complement LV {}\\
\glt\hspaceThis{TM:} \parbox{\linewidth - \widthof{TM: }}{‘Well, if (people) become old, (they) trip over their own feet, and ...’}
\z
\z

\subsubsection{Adjectival predicate}\label{bkm:Ref347240074}\hypertarget{RefHeadingToc395696996}{}
An adjectival predicate phrase is composed of an adjective and optionally a stative verb as schematized in \REF{ex:4.9} (see §\ref{bkm:Ref357004964} for more details).

\ea\label{ex:4.9} Structure of the adjectival predicate phrase\\
\relax [A \hspace{1em} (STV)]\textsubscript{Adjectival predicate phrase}\\
\z

\noindent The minimal adjectival predicate phrase is illustrated in \REF{ex:4.10a}, where the head slot is filled by the adjectival word (see \sectref{sec:key:4.3.4} for more details).

\ea\label{ex:4.10}
\ea\label{ex:4.10a} \textit{-sa} (ADJ)\hfill\relax[Co: 120415\_01.txt]\\
\relax[Context: Looking at a fried vegetable]\\
 {TM:} \glll  {agɨ!}  {hɨɨsa.}\\
   {\itshape agɨ}  {[\textit{hɨɨ-sa}]\textsubscript{Adjectival Predicate}}\\
   {oh}  {big-ADJ}\\
   \glt\hspaceThis{TM:} ‘Oh! (It is) big.’
 \ex \textit{{}-soo} (ADJ)\hfill\relax[El: 130823]\\
 {TM:} \glll  {agɨɨ!}  {kjurasoo.}\\
   {\itshape agɨ}  {[\textit{kjura-soo}]\textsubscript{Adjectival} \textsubscript{Predicate}}\\
   {oh}  {beautiful-ADJ}\\
   \glt\hspaceThis{TM:} ‘Oh! (It is) beautiful.’\\
\z
\z

There are two stative verbs \textit{ar-} and \textit{nə-}. In many cases, \textit{ar-} (STV) co-occurs with the adjective whose inflection is \textit{{}-sa} (ADJ) as in \REF{ex:4.11a} (see §\ref{bkm:Ref358545085} for more details). \textit{nə-} (STV) co-occurs with the adjective whose inflection is \textit{{}-soo} (ADJ) as in \REF{ex:4.11b} (see §\ref{bkm:Ref367137632} for more details).

\ea 
\ea\label{ex:4.11a} \textit{{}-sa} (ADJ) with \textit{ar-} (STV)\hfill\relax[Co: 120415\_00.txt]\\
 \relax[Context: Remembering her childhood]\\
 {TM:} \glll {{asikencˀjuga}}  {huusa}  {ata.}\\
   {{\itshape asiken+cˀju=ga}}  {[\textit{huu-\Emph{sa}}}  {\textit{ar{}-\Emph{tar}}]\textsubscript{Adjectival Predicate}}\\
   {{Ashiken+person=FOC}}  {many-ADJ}  {STV-PST}\\
   \glt\hspaceThis{TM:} ‘There were many people from Ashiken.’
\ex \label{ex:4.11b} \textit{{}-soo} (ADJ) with \textit{nə-} (STV)\hfill\relax[Co: 111113\_01.txt]\\
 \relax[Context: Talking about the wooden beams of MS’s house; MS: ‘(The wooden beams of my house) haven’t become as black as those (of your house), you know.’]\\
 {{TM:}} \glll {kˀurusoo}  {nəndarooga.}\\
        \textit{kˀuru-\Emph{soo}}  \textit{\Emph{nə}-an=daroo=ga}\\
        {black-ADJ}  {STV-NEG=SUPP=CFM3}\\
 \glt\hspaceThis{TM:} ‘(Those) are not black, right?’
\z
\z
 
 \subsubsection{Nominal predicate}
\label{bkm:Ref347240067}\hypertarget{RefHeadingToc395696997}{}
A nominal predicate phrase is composed of a nomina phrase (NP) and optionally a copula verb (COP) as schematized in \REF{ex:4.12} (see §\ref{bkm:Ref357004964} for more details).

\ea\label{ex:4.12}Structure of the nominal predicate phrase\\
\relax [NP \hspace{1em}(COP)]\textsubscript{Nominal predicate phrase}\\
\z

\noindent The fact that the copula verb is optional indicates that the head of the nominal predicate is the NP (not the copula) as will be discussed below.

Yuwan has four copula verbs: \textit{jar-}, \textit{zjar-}, \textit{nar-} and \textit{ar-} (see \sectref{sec:key:8.4.3} for more details). The first three (\textit{jar-}, \textit{zjar-}, and \textit{nar-}) are used in affirmative, and the last one (\textit{ar-}) is used in negative with the exception of the AVC (see §\ref{bkm:Ref361945627}) and the focus construction (see §\ref{bkm:Ref361841066}). NPs are followed by the topic particle \textit{ja} when the copula verb is \textit{ar-} in negative (for other cases, see \sectref{sec:key:9.3.2.1}). I present the copula verbs, which are \EmphName in the following exmaples.

\ea\label{ex:4.13}
\ea \label{ex:4.13a} {\itshape jar-}\hfill\relax[Co: 110328\_00.txt]\\
   \relax[Context: Speaking of an acquaintance of both US and TM]\\
   {US:} \gllll  {{haccjanna}}  {{ikɨgaci}}  {{jatəi?}}\\
     {\itshape haccjan=ja}  {\itshape ikɨgaci}  {\textit{\Emph{jar}-təər-i}}\\
     {Hachan=TOP}  {Ikegachi}  {COP-RSL-NPST}\\
     {}  {[NP}  {Copular verb]\textsubscript{Nominal predicate}}\\
     \glt\hspaceThis{US:} ‘Hachan was (from) Ikegachi?’
     
\ex {\itshape zjar-}\hfill\relax[Co: 111113\_01.txt]\\
   \relax[Context: Seeing a photo of the Bon festival]\\
   {TM:} \gllll {katakˀwasi}  {zjajaa.}\\
     {\itshape kata+kˀwasi}  \textit{\Emph{zjar}=jaa}\\
     {model+snack}  {COP=SOL}\\
     {[NP}  {Copular verb]\textsubscript{Nominal predicate}}\\
     \glt\hspaceThis{TM:} ‘(That) is Katagasi, you know.’
\ex {\itshape nar-}\hfill\relax[Co: 111113\_02.txt]\\
   TM: \gllll {jusɨga}  {sɨki}  {natɨjoo,}\\
     {\itshape jusɨr-∅=ga}  {{\itshape sɨki}}  {{\textit{\Emph{nar}-tɨ=joo}}}\\
     {teach-INF=NOM}  {fond}  {COP-SEQ=CFM1}\\
     {}  {[NP}  {{Copular verb]\textsubscript{Nominal predicate}}}\\
     \glt\hspaceThis{TM:} \parbox{\linewidth - \widthof{TM: }}{‘(My mother) was fond of teaching, so (everyone came to learn the traditional songs from my mother).’}
     
\ex {\itshape ar-}\hfill\relax[Co: 120415\_00.txt]\\
   \relax[Context: Seeing a photo taken in celebration of setting up the first outdoor lamps on the shopping street in the village]\\
   {TM:} \gllll {un}  {{tukinnu}}  {juwəəja}  {aran?}\\
     {\itshape un}  {{\itshape tuki=n=nu}}  {\itshape juwəə=ja}  {\textit{\Emph{ar}-an}}\\
     {[that}  {{time=DAT1=GEN}}  {celebration]=TOP}  {COP-NEG}\\
     {\{[NP]}  {Copular verb\}\textsubscript{Nominal predicate}}\\
   \glt\hspaceThis{TM:} ‘Is (the photo about) the celebration at that time?’
   \z
\z

There are some cases where the copula verbs are free to occur in the nominal predicates as in \REF{ex:4.14}.

\ea\label{ex:4.14} Copular verb is free to appear\hfill\relax[Co: 120415\_00.txt]\\
\relax[Context: Seeing an album]\\
 {TM:} \glll {urəə}  {denzirosan.}\\
   {\itshape urɨ=ja}  {\itshape denziro-san}\\
   {that=TOP}  {[Denziro-POL]\textsubscript{Nominal predicate}}\\
\glt\hspaceThis{TM:} ‘That is Denziro.’
\z
\noindent However, the copula verbs must occur unless the nominal predicate fulfills all of the following conditions.

\ea\label{ex:4.15} The copula verbs must occur unless the nominal predicate fulfills all of the following conditions:
\ea In the non-past tense;
\ex In affirmative;
\ex Not taking verbal affixes or conjunctive particles;
\ex Predicate not being focused by \textit{du} (FOC).
\z
\z

\noindent For example, the nominal predicate takes the aspectual affix \textit{{}-təər} (RSL) in \REF{ex:4.13a}. Thus, it takes the copula verb \textit{jar-}. On the other hand, the nominal predicate in \REF{ex:4.14} fulfills all of the conditions in \REF{ex:4.15}. Thus, it is free to take a copula verb. It should be noted that the nominal predicate that fulfills all of the conditions in \REF{ex:4.15} “is free” to take copula verbs. In other words, such a nominal predicate “may” take a copula verb as in \REF{ex:4.16}.

\ea\label{ex:4.16} Copular verb may appear \hfill\relax[Co: 120415\_00.txt]\\
\relax[Context: Seeing an album]\\
 {TM:} \gllll {doosje}  {noogusuku}  {zja.}\\
   {\itshape doosje}  {\itshape noogusuku}  {\itshape zjar}\\
   {maybe}  {[Nogusuku}  {COP]}\\
     {[NP}  {Copular verb]\textsubscript{Nominal predicate}}\\
 \glt\hspaceThis{TM:} ‘(It) may be Nogusuku.’
\z

In addition, \textit{zjar-} (COP) always appears when the nominal predicate fulfills the conditions in \REF{ex:4.15}, and also is followed by \textit{jaa} (SOL) or \textit{ga} (CFM3).
 

\ea\label{ex:4.17} Followed by \textit{jaa} (SOL)\hfill\relax[Co: 111113\_02.txt]\\
 {TM:} \glll {an}  {ikɨn}  {məə}  {zjajaa.}\\
   {\itshape a-n}  {\itshape ikɨ=n}  {\itshape məə}  {\textit{zjar=jaa}}\\
   {DIST-ADNZ}  {pond=GEN}  {front}  {COP=SOL}\\
\glt\hspaceThis{TM:} ‘(This picture) is the front of that pond.’
\z

On the contrary, if a nominal predicate fulfills all of the conditions in (4-15) and (4-18), the copula verbs never appear as in (4-19 a-b).

\ea\label{ex:4.18}Additional condition:\\
Nominal predicate is followed by \textit{doo} (ASS), \textit{daroo} (SUPP), \textit{ga} (CFM3), \textit{kai} (DUB), \textit{joo} (CFM1), \textit{jaa} (SOL), or \textit{na} (PLQ).
\z

\noindent The following example shows that the clause-final particle \textit{doo} (ASS) directly attaches to the NP in the predicate.

\ea\label{ex:4.19} Copula verb cannot appear
\ea[]{[Context: Remembering the utterance of an acquaintance] [Co: 120415\_00.txt]\\\label{ex:4.19a}
   {TM:} \gllll akiradoo\\
     \textit{akira=doo}\\
     [Akira]=ASS\\
     [NP]\textsubscript{Nominal predicate}\\
   \glt\hspaceThis{TM:} ‘(This is) Akira.’}
\ex\todo[inline]{Example does not compile.}\label{ex:4.19b}
% \ex[*]{TM: \glll akira  jattoo/zjattoo.\\\jambox{[El: 111104]}
%      \textit{akira}  \textit{jar}/\textit{zjar=doo}\\
%      Akira  COP=ASS\\}
\z
\z

\noindent The example of \textit{kai} (DUB) was shown in (4-2 b).

\section{Basic morphological units}\label{bkm:Ref381195916}\hypertarget{RefHeadingToc395696998}{}
\subsection{Free form, clitic, and affix}\label{bkm:Ref364029002}\hypertarget{RefHeadingToc395696999}{}
As mentioned in §\ref{bkm:Ref347179371}, grammatical words comprise free forms and clitics. There are no prefixes or proclitics in Yuwan, although some personal names in Yuwan seem to have a prefix-like morpheme, e.g. \textit{u{}-mine} (PREFIX?-Mine) ‘Mine (personal name).’ The alleged formative \textit{u-}, however, is not productive in modern Yuwan, and only appears in the beginning of some personal names. Therefore, I treat it as a part of the root. The formative \textit{u-} seems to have originated from \textit{*o-}. This must have expressed politeness considering the cognate form \textit{o-} in standard Japanese, e.g. \textit{o-kasi} (POL-snack) and \textit{o-mise} (POL-shop). In fact, the speaker TM regards this /u/ as a part of the name, i.e., she thinks /mine/ is an official name and /umine/ is a private name. A similar argumentation can be made against the existence of the proclitic in Yuwan. For example, the formative \textit{naa} ‘more,’ as in \textit{naa+cˀjui} (more+one.NUM.HUM) ‘one more person,’ looks like a proclitc in the sense that it is a bound grammatical formative that attaches to a free form. However, \textit{naa} may also be analyzed as a free form, which can function as an adverb (see aslo \sectref{sec:key:4.3.6}). In this case, \textit{naa+cˀjui} should be analyzed as a compound. That is, \textit{naa} is not categorized as a clitic (i.e. particle) but instead as a word (i.e. adverb) (see also §\ref{bkm:Ref364286741}).

There are two main criteria for distinguishing among free forms, clitics, and affixes, shown in \tabref{tab:24}. The meaning of a “minimal utterance” here is a minimal unit that can be uttered only by itself. In fact, a compound does not conforms to this criterion, since each component of a compound can be uttered only by itself. Considering the cohesion of the compound, however, it is reasonable to regard it as a free form (cf. \citealt{DixonAikhenvald2002}). Similarly, the honorific auxiliary verb construction, which will be discussed in \sectref{sec:key:9.1.1}, expresses a strong cohesion. Considering the other auxiliary verb constructions, however, it is appropriate to think that the honorific auxiliary verb consruction is in the process of grammaticalization. Thus, I propose that it is composed of multiple free forms, i.e. verbs. A stronger feature that would distinguish free forms from clitics and affixes is prosody. It is likely true that free forms can have their own prosody but (most of) clitics and affixes cannot. However, the prosody of Yuwan is only partly clarified (see §\ref{bkm:Ref347175621}), and I use the criterion only partly in this grammar.

\begin{table}
\caption{Criteria for distinguishing among free forms, clitics, and affixes\label{tab:24}}
\begin{tabular}{llll}
\lsptoprule
Criterion & \multicolumn{2}{c}{Grammatical word} & \\\cmidrule(lr){2-3}
& {Free form} & {Clitic} & {Affix}\\\midrule
(a) Can constitute a minimal utterance &  + &  \textminus &  \textminus\\
(b) Can follow more than one word class &  + &  + &  \textminus\\
\lspbottomrule
\end{tabular}
\end{table}

Most of morphological units comform to the criteria in \tabref{tab:24}. However, there are some instances that cannot be classified clearly into free forms, clitics, or affixes. Those instances are discussed in the next section.
% 
\subsection{Problematic cases}\hypertarget{RefHeadingToc395697000}{}
\subsubsection{Clitic-like free forms}\hypertarget{RefHeadingToc395697001}{}
The previous section mentioned that there is no proclitic in Yuwan, but there are proclitic-like morphemes, namely adnominals (e.g. /a-n/ ‘that (one)’ or /wa-a/ ‘my’). However, I do not regard these units as proclitics, since adnominals have their own pitch patterns. In fact, the details are not very clear and should be investigated in future research.

Copula verbs cannot occur only by themselves (except for the case discussed in (8-40) in \sectref{sec:key:8.4.3.3}), and they do not seem to have their own pitch pattern. However, I do not regard them as (en)clitics, since copula verbs behave differently from clitics when they occur after infinitives in the sentence-final position. Infinitives before clitics in the sentence-final position become the lengthened forms, but infinitives before copula verbs in the sentence-final position become the simple forms (see (8-108) in \sectref{sec:key:8.4.4.1} for more details).

It should be mentioned that the stative verbs \textit{ar-} and \textit{nə-} cannot constitute a minimal utterance, and \textit{ar-} (STV) does not seem to have its own pitch pattern. (On the contrary, \textit{nə-} (STV) seems to have its own pitch pattern, i.e. the pitch pattern III.) In fact, \textit{ar-} (STV) is in the process of grammaticalization, which is apparent from the fact that it undergoes contraction with the preceding adjective in some environments (see \sectref{sec:key:9.2.2.2} for more details). I do not use the clitic-boundary marker “=” before \textit{ar-} (STV) to maintain the structural parallelism between \textit{ar-} (STV) and \textit{nə-} (STV), but it may be appropriate to regard the stative verb composed of \textit{ar-} as an enclitic in modern Yuwan.

\subsubsection{Affix-like clitics}\label{bkm:Ref347227284}\hypertarget{RefHeadingToc395697002}{}
Yuwan has two types of clitics that have similarity with affixes.

First, some clitics in Yuwan have similarity with affixes in terms of the formal boundedness of the host morpheme. In many cases, affixes can follow bound verbal stems, but clitics cannot. However, there are some clitics that can follow bound verbal stems, i.e. \textit{sɨ} (FN), \textit{doo} (ASS), \textit{ka} (DUB), \textit{kai} (DUB), \textit{kamo} (POS), \textit{ga} (CFM3), and \textit{gajaaroo} (DUB) (see also chapter 10). For example, the verbal affix \textit{{}-jur} (UMRK) cannot finish an utterance, and \textit{jum-jur} (read-UMRK) is a bound verbal stem. An inflectional affix, e.g. \textit{{}-i} (NPST), has to follow it to make it a free form, i.e. /jum-ju-i/ \textit{jum-jur-i} (read-UMRK-NPST) ‘(Someone) reads.’ According to the criteria shown in \tabref{tab:24}, the above seven clitics are not affixes, since they can follow more than one word class. However, those clitics are similar to the inflectional verbal affixes since they can follow bound verbal stems: /jum-ju=sɨ/\footnote{There is a morphophonological rule (see §\ref{bkm:Ref347177215}): jur + sɨ > jusɨ.} (read-UMRK=FN) ‘something to read,’ /jum-jut=too/\footnote{There is a morphophonological rule and a phonological rule (see §\ref{bkm:Ref347177096} and §\ref{bkm:Ref347178914}): jur + doo > juddoo > juttoo.} (read-UMRK=ASS) ‘(I) will read,’ and /jum-juk=kai/\footnote{There is a morphophonological rule (see §\ref{bkm:Ref347177096}): jur + kai > jukkai.} (read-UMRK=DUB) ‘Will you read?’, and so forth. Considering these facts, the above seven clitics are somewhere between clitics and affixes.

Second, a few clitics in Yuwan have similarity with affixes in terms of the constraint on the selection of the hosts’ stem classes. Briefly speaking, there are morphemes that do not conform to the second criterion in \tabref{tab:24}, but that will be treated as clitics, i.e. \textit{ban} (ADVRS) and \textit{mun} (ADVRS). They always follow a verb (concretely speaking, a participle). A participle usually fills the predicate slot of an adnominal clause, as in \REF{}. However, it can fill the predicate slot of an adverbial clause if it is followed by \textit{ban} (ADVRS) as in \REF{ex:4.20b}.

\ea
\ea\label{ex:4.20a}Participle in an adnominal clause\hfill\relax[Co: 110328\_00.txt]\\
 {TM:} \\ \glll {tarun}  {mukasinukutu}  {siccjun} {cˀjoo}  {wuranbajaa.}\\
   {\itshape ta-ru=n}  {\itshape mukasi=nu=kutu}  {\itshape sij-tur-n} {\itshape cˀju=ja}  {\itshape wur-an-ba=jaa}\\
   {who-NLZ=any}  {[past=GEN=event}  {know-PROG-PTCP]\textsubscript{Adnominal clause}} {person=TOP}  {exist-NEG-CSL=SOL}\\
   \glt ‘There is not anyone who knows the events of the past.’
\ex\label{ex:4.20b}Participle in an adverbial clause\hfill\relax[Co: 110328\_00.txt]\\
 {TM:} \\ \glll {wanna}  {honami-{\textbar}cjan{\textbar}}  {naaja}  {siccjunban,} {naakjaa}  {jumɨnu}  {naaja}  {sijandoojaa.}\\
   {\itshape wan=ja}  {\textit{honami-cjan}}  {\itshape naa=ja}  {\itshape sij-tur-n=ban} {\itshape naakjaa}  {\itshape jumɨ=nu}  {\itshape naa=ja}  {\itshape sij-an=doo=jaa}\\
   {[1SG=TOP}  {Honami-DIM}  {name=TOP}  {know-PROG-PTCP=ADVRS]\textsubscript{Adverbial clause}} {2PL.HON.ADNZ}  {daughter.in.law=GEN}  {name=TOP}  {know-NEG=ASS=SOL}\\
   \glt ‘I know Honami’s name, but don’t know the name of your daughter in law.’\\
\z
\z

\noindent Considering the second criterion in \tabref{tab:24}, \textit{ban} (ADVRS) has to be classified into affixes since it cannot follow more than one word class. However, I propose \textit{ban} (ADVRS) as an clitic (not an affix) because I do not assume there is an additional inflectional slot after the participial affix slot. In other words, there is no beneficial reason to interpret the participial affix \textit{{}-n} as an ambivalent affix that is able to both close and not close a word, similar to the past affix \textit{{}-tar} or the negative affix \textit{{}-an} (see §\ref{bkm:Ref303739828} for discussion about ambivalent affixes). The only possible candidates that can follow \textit{{}-n} (PTCP) are the two morphemes mentioned above, which is different from \textit{{}-tar} (PST) and \textit{{}-an} (NEG), which can precede a number of verbal inflectional affixes. Thus, I do not regard \textit{ban} (ADVRS) and \textit{mun} (ADVRS) as affixes. Rather, I propose that they are conjunctive particles (see §\ref{bkm:Ref367138946}).

\subsection{Stems and its morphological operations}\hypertarget{RefHeadingToc395697003}{}\label{bkm:Ref367387017}
The term stem is used to describe the combination of a root and a derivational affix (or affixes) (see §\ref{bkm:Ref303739828} for the distinction between derivational affixes and an inflectional affix).

\ea\label{ex:4.21} Stem: \{Root(-Derivational affix(es))\}\textsubscript{stem}\z

\noindent Thus, the minimal stem is made of a single root. The minimal word is made of a minimal stem, which is summarized as follows.

\ea\label{ex:4.22} Minimal word: [\{Root\}\textsubscript{stem}]\textsubscript{word}\z

In the following subsections, we will discuss two types of complex stems, i.e. compounding (see §\ref{bkm:Ref364023541} and §\ref{bkm:Ref364279917}) and reduplication (see §\ref{bkm:Ref364023547}). In §\ref{bkm:Ref347181344}, I will present the morphophonological rule for compounding, i.e. “rendaku” (sequential voicing).

\subsubsection{Compounding (ordinary type)}
\label{bkm:Ref364023541}\hypertarget{RefHeadingToc395697004}{}\label{bkm:Ref364286741}
A compound is a complex stem that usually constitutes a grammatical word as in \REF{ex:4.23a}. However, there is a case where the complex stem itself does not constitute a grammatical word, and such a stem needs an inflection to become a free form as in \REF{ex:4.23b}.

\ea
\ea \label{ex:4.23a}Compounded nominal stem\\
   \glll {\itshape sataa+jadui}\\
   {sugar+hut}\\
   {[\{Stem\textsubscript{1}+Stem\textsubscript{2}\}\textsubscript{compound}]\textsubscript{word}}\\
   \glt ‘hut (in order to make) sugar (from sugarcane)’
\ex \label{ex:4.23b}Compounded verbal stem\\
   \glll {\itshape izjas-i+kij-an}\\
   {let.out-INF+CAP-NEG}\\
   {[\{Stem\textsubscript{1}+Stem\textsubscript{2}\}\textsubscript{compound}{}-Affix]\textsubscript{word}}\\
   \glt ‘cannot let (them) go’
\z
\z

\noindent The first example shows a nominal compound made up of two stems, i.e. \textit{sataa} ‘sugar’ and \textit{jadui} ‘hut.’ The second example shows a verbal compound made up of two stems, where Stem\textsubscript{1} is composed of the infinitive \textit{izjas-i} (let.out-INF) and Stem\textsubscript{2} is composed of the verbal root \textit{kij-} (CAP). The compound becomes a verbal stem and it takes the verbal affix \textit{-an} (NEG). In many cases, the head of a compound is put at the final position in the compound as in \REF{ex:4.23b}, although there are a few exceptions.

The possible combinations of different classes of stems in the two-stem compounds are shown in \tabref{tab:25}.

\begin{table}
\caption{\label{tab:25}Combinations of stem classes in the compounds}
\begin{tabular}{llll}
\lsptoprule
& \multicolumn{3}{c}{Following stem class}\\\cmidrule(lr){2-4}
Preceding stem class & {N} & {V} & {A}\\\midrule
N(ominal) & N+N & N+V & N+A\\
V(erb) & V\textsubscript{inf}+N & V\textsubscript{inf}+V & V\textsubscript{inf}+A\\
A(djective) & A+N & A+V & A+A\\
Adv(erb) & Adv+N & \textminus & \textminus\\
D(emonstrative) & \textminus & \textminus & D+A\\
I(nterrogative) & I+N & \textminus & I+A\\
\lspbottomrule
\end{tabular}
\end{table}

In a compound, the verbal stem at non-stem-final position is in infinitive (V\textsubscript{inf} in the above table; see \sectref{sec:key:8.4.4}).
% 
Each combination in \tabref{tab:25} is illustrated below, with the exception of the combination V\textsubscript{inf}+A, which will be discussed in §\ref{bkm:Ref364279917}. The first examples are compounds that have nominal stems at thier final positions. The resulting compounds always become nominal stems.

\ea \label{ex:4.24}
\ea N+N\hfill\relax[PF: 090222\_00.txt]\\\label{ex:4.24a}
\relax [Context: Remembering the pear film]\\
 {TM:} \glll  {simahinzjaaja} {aranba.}\\
  {<\textit{sima+hinzjaa}>\textsubscript{Compound}\textit{=ja}} {\itshape ar-an-ba}\\
  {island+goat=TOP} {COP-NEG-CSL}\\
 \glt\hspaceThis{TM:} ‘Because (it) is not a goat of (our) island.’
\ex V+N\hfill\relax[El: 120926]\\\label{ex:4.24b}
 {TM:} \glll {hingimadoo} {nənta.}\\
  {<\textit{hingir-i+madu}>\textsubscript{Compound}\textit{=ja}} {\itshape nə-an=tar}\\
  {escape-INF+time=TOP} {exist-NEG=PST}\\
  \glt\hspaceThis{TM:} ‘There was no time to escape.’
{\smallex\ex A+N\hfill\relax[Co: 120415\_00.txt]\\\label{ex:4.24c}
 \relax[Context: Speaking about a referee of the sumo wrestling in a picture]\\
 {TM:} \glll {hakamankjagadɨ} {muccjutattu,} {sijukinnu.}\\
  {\itshape hakama=nkja=gadɨ} {\itshape mut-tur-tar-tu} {<\textit{siju+kin}>\textsubscript{Compound}\textit{=nu}}\\
  {hakama=APPR=LMT} {have-PROG-PST-CSL} {white+clothes=GEN}\\
  \glt\hspaceThis{TM:} ‘(He) had a hakama, (made) of white clothes.’}
\ex Adv+N\hfill\relax[Co: 120415\_00.txt]\\\label{ex:4.24d}
 \relax[Context: Seeing some acquaintances of TM in a picture]\\
 {TM:} \glll {naacˀjuinu} {cˀjoo} {koogi} {jappa.}\\
  {<\textit{naa+cˀjui}>\textsubscript{Compound}\textit{=nu}} {\itshape cˀju=ja} {\itshape koogi} {\itshape jar-ba}\\
  {other+one.CLF.person=GEN} {person=TOP} {Kogi} {COP-CSL}\\
  \glt\hspaceThis{TM:} ‘Since another person is Kogi.’
\ex I+N\hfill\relax[Co: 120415\_01.txt]\\\label{ex:4.24e}
 \relax[Context: Talking about an acquaintance of TM and MS]\\
 {TM:} \glll {an} {cˀju} {daacˀju} {jatakai?}\\
  {\itshape a-n} {\itshape cˀju} \Emph{\itshape daa+cˀju} {\itshape jar-tar=kai}\\
  {DIST-ADNZ} {person} {where+person} {COP-PST=DUB}\\
  \glt\hspaceThis{TM:} ‘Where did that person come from?\\
      \hspaceThis{TM:} [lit. That person was where’s person?]’
\z
\z

\noindent The verbal root preceding the nominal stem always takes the infinitival affix as in \REF{ex:4.24b} (see §\ref{bkm:Ref364074891} for infinitives). If the adverbial root \textit{naa} ‘other; already’ makes up a compound as in \REF{ex:4.24d}, the following nominal is always a numeral (see §\ref{bkm:Ref367139227} for discussion of numerals). I found only one example of the combination of I+N, i.e. \textit{daa}+\textit{cˀju} (where+person) as in \REF{ex:4.24e}.

The next examples are compounds that have verbal stems at thier final positions.

\ea
\ea N+V\hfill\relax[Co: 110328\_00.txt]\\\label{ex:4.25a}
\relax[Context: Talking about thatched houses with US]\\
 {TM:} \glll {naakjoo} {gajaurusinkjoo} {sɨrantaroo.}\\
  {\itshape naakja=ja} {<\textit{gaja+urus-i}>\textsubscript{Compound}\textit{=nkja=ja}} {\itshape sɨr-an-tar-oo}\\
  {2PL=TOP} {miscanthus+lower-INF=APPR=TOP} {do-NEG-PST-SUPP}\\
  \glt\hspaceThis{TM:} ‘I suppose that you have never brought miscanthus (for thatched roofs).’
\ex V+V\hfill\relax[Co: 111113\_01.txt]\\\label{ex:4.25b}
 \relax[Context: Talking about a man who came from mainland Japan to buy cycad leaves for business.]\\
 {TM:} \glll {kiihatɨppoo,} {sirɨtuppajaa.}\\
  {<\textit{kij-∅+hatɨr}>\textsubscript{Compound}\textit{}-boo} {\itshape sirɨr-tur-ba=jaa}\\
  {cut-INF+thoroughly-CND} {easy.to.understand-PROG-CSL=SOL}\\
  \glt\hspaceThis{TM:} ‘If (he) cut all the cycad leaves, you may know (what would happen then).’
\ex A+V\hfill\relax[Co: 111113\_02.txt]\\\label{ex:4.25c}
\relax[Context: Speaking about a person whose role was to hit a big bell in emergency]\\
 {TM:} \glll {{\textbar}hizjoo{\textbar}nu} {tukinga} {gan+gan} {gan+gan} {zjanaucii.}\\
  {\itshape hizjoo=nu} {\itshape tuki=n=ga} {\itshape gan+gan} {\itshape gan+gan} {<\textit{zjana+ut-i}>\textsubscript{Compound}}\\
  {emergency=GEN} {time=DAT1=FOC} {RED+clang} {RED+clang} {many+hit-INF}\\ 
  \glt\hspaceThis{TM:} ‘In case of emergency, (he) clanged (the bell) many times.’
\z
\z

\noindent If a stem that precedes a verbal stem is a nominal one as in \REF{ex:4.25a} or an adjectival one as in \REF{ex:4.25c}, the verbal stem always become an infinitive. However, if the initial stem is a verbal one, the final verbal stem can take any verbal inflection as in \REF{ex:4.25b} (see also §\ref{bkm:Ref359270049}).

Finally, the following examples are compounds that have adjectival stems at thier final positions. The examples of “V+A” will be discussed in the next section. The resulting compounds become adjectival stems as in (\ref{ex:4.26}a--b) or adverbial stems as in (\ref{ex:4.26}c--e).

\ea\label{ex:4.26}
\ea {N+A}\hfill\relax[Co: 120415\_00.txt]\\\label{ex:4.26a}
\relax[Context: Talking about a female singer of traditional songs; TM: ‘Actually, the recorded tape makes some noise, but ...’]\\
 {TM:} \glll {kuigjurasa} {utəəja} {sjuijaa.}\\
  {<\textit{kui+kjura>}\textsubscript{Compound}\textit{}-sa} {\itshape utaw-i=ja} {\itshape sɨr-jur-i=jaa}\\
  {voice+beautiful-ADJ} {sing-INF=TOP} {do-UMRK-NPST=SOL}\\
  \glt\hspaceThis{TM:} ‘(She) sings beautifully, you know.’
\ex {A+A}\hfill\relax[El: 130812]\\\label{ex:4.26b}
  {TM:} \glll {an} {wunaguja} {injagjurasajaa.}\\
  {\itshape a-n} {\itshape wunagu=ja} {\itshape inja+kjura-sa=jaa}\\
  {DIST-ADNZ} {woman=TOP} {small+beautiful-ADJ=SOL}\\
  \glt\hspaceThis{TM:} ‘That woman is small and beautiful.’
\ex {D+A}\hfill\relax[Co: 111113\_02.txt]\\\label{ex:4.26c}
\relax[Context: Talking about a big banyan tree, which was lost in World War II]\\
 {TM:} \glll {jɨdaja} {ganbəi} {sjasɨnkjanu,} {{\textbar}zuutto{\textbar},} {agatuubəigadɨ} {cˀjɨ,}\\
  {\itshape jɨda=ja} {\itshape ga-n=bəi} {\itshape sɨr-tar=sɨ=nkja=nu} {\itshape zuutto} {<\textit{aga+tuu}>\textsubscript{Compound}\textit{=bəi=gadɨ}} {\itshape k-tɨ}\\
  {brach=TOP} {MES-ADVZ=only} {do-PST=FN=APPR=NOM} {throughout} {DIST+distant=only=LMT} {come-SEQ}\\
  \glt\hspaceThis{TM:} ‘A branch, which was around this size, came to such a distance, and...’
\ex I+A\hfill\relax[Co: 111113\_02.txt]\\\label{ex:4.26d}
\relax[Context: TM wondered when winnows in the picture disappeared from their life.]\\
 {TM:} \glll {ikjanagəəbəi} {nakkai?}\\
  {<\textit{ikja+nagəə}>\textsubscript{Compound}\textit{=bəi}} {\itshape nar=kai}\\
  {how+long=only} {become=DUB}\\
  \glt\hspaceThis{TM:} ‘How long is (it)? [lit. How long does (it) become?]’
\ex I+A\hfill\relax[Co: 120415\_00.txt]\\\label{ex:4.26e}
\relax[Context: Talking about the pension for the wounded soldiers]\\
 {TM:} \glll {ikjanagən} {{\textbar}sjooigunzin{\textbar}nu ..} {{\textbar}tecuzuki{\textbar}ga} {sɨran=sjutɨ,}\\
  {<\textit{ikja+nagəə}>\textsubscript{Compound}\textit{=n}} {\itshape sjooi+gunzin=nu} {\itshape tecuzuki=ga} {\itshape sɨr-an=sjutɨ}\\
  {how+long=even} {wounded+soldier=GEN} {procedure=NOM} {do-NEGSEQ}\\
  \glt\hspaceThis{TM:} ‘For a while, (he) could not carry out the procedure for (the pension for) the wounded soldiers, and ...’
\z
\z

\noindent If the initial stem is a nominal one as in \REF{ex:4.26a} or an adjectival one as in \REF{ex:4.26b}, the final adjectival stem can take any adjectival inflection. However, if the initial stem is a demonstrative one as in \REF{ex:4.26c} or interrogative one as in (\ref{ex:4.26}d--e), the final adjectival stem does not take any adjectival inflection, and the resulting compounds always behave like adverbs. Especially, the compounds of D+A are frequently followed by \textit{gadɨ} (LMT). This type of combination is not very productive in Yuwan since there is a limited set of adjectival stems that can form compounds with demonstrative stems, namely \textit{taa-} ‘high,’ \textit{tuu-} ‘distant,’ and \textit{nagəə-} ‘long.’ Similarly, the combination of I+A is rare, and I have found only the combination of \textit{ikja-} ‘how’ and \textit{nagəə-} ‘long’ so far. This combination, i.e. \textit{ikja+nagəə} ‘how long,’ is always followed by one of the following limiter particles, i.e. \textit{gadɨ} (LMT), \textit{n} ‘even; ever; also,’ or \textit{bəi} ‘only; about.’

Among the above compounds, N+N and N+V are very productive. Compounds made of three roots, such as /kˀwa+dak-i+kˀjubii/ (child+hold-INF+cord) ‘a cord to hold a baby’ and /tuzi+kaməə-∅+juwəə/ (wife+ put.over.head-INF+celebration) ‘wedding,’ are likely to be N+V+N. I have not yet found a compound composed of more than three roots.

\subsubsection{Compounding (special type)}\label{bkm:Ref364279917}\hypertarget{RefHeadingToc395697005}{}
There are compounds whose final stems can appear only in compounding.

\ea\label{ex:4.27}
\ea\label{ex:4.27a} Nominal stems in the compounds “V+N”\\
\ea\label{ex:4.27a-1} \textit{zjaa} ‘place,’ \textit{bəə} ‘role’\\
\ex\label{ex:4.27a-2} \textit{mai} (OBL), \textit{madəə} ‘fail to,’ \textit{gjaa} (PURP)\\
\z
\ex\label{ex:4.27b} Adjectival stems in the compounds “V+A”\\
\textit{cja} ‘want,’ \textit{cjagɨ} ‘seem,’ \textit{jass} ‘easy,’ \textit{gussj} ‘difficult’
\z
\z

\noindent The compounds whose final stems are those in \REF{ex:4.27a} become nominal stems, and the compounds whose final stems are those in \REF{ex:4.27b} become adjectival stems. Semantically, the morpheme in \REF{ex:4.27a-1} have more concrete meaning than those in \REF{ex:4.27a-2}. In fact, the former can be an argument NP, but the latter cannot. Compounds composed of the morphemes in \REF{ex:4.27a-2} can fill the predicate slot, complement slot, or NP modifier slot.

I will present examples of \textit{zjaa} ‘place’ and \textit{bəə} ‘role’ in the following examples, in which the compounds are argument NPs as in (\ref{ex:4.28b}, e) and predicate NPs as in (\ref{ex:4.28a}, c, d). The compounds are \EmphName in the following examples.

\begin{exe}\ex\label{ex:4.28}
\textit{zjaa} ‘place’
\begin{xlist}
\ex\label{ex:4.28a}\relax [Co: 110328\_00.txt]\\ {TM:} \glll {umaga} {asɨbizjaa} {jatattujaa.}\\
  {\itshape u-ma=ga} \Emph{\itshape asɨb-i+zjaa} {\itshape jar-tar-tu=jaa}\\
  {MES-place=NOM} {play-INF+place} {COP-PST-CSL=SOL}\\
  \glt\hspaceThis{TM:} ‘That place was the place to play, you know.’
\ex\label{ex:4.28b} \relax[Co: 120415\_00.txt]\\{TM:} \glll {ukizjaa} {katəətattu.}\\
  \Emph{\itshape uk-i+zjaa} {\textit{kar-təər-tar-tu}}\\
  {put-INF+place} {borrow-RSL-PST-CSL}\\
  \glt\hspaceThis{TM:} ‘(They) had borrowed a place to put (something).’
\end{xlist}
\textit{bəə} ‘role’
\begin{xlist}\setcounter{xnumii}{2}
\ex\label{ex:4.28c}\relax[Co: 111113\_02.txt]\\ {TM:} \glll {un} {cˀjuga} {ucibəə.}\\
  {\itshape u-n} {\itshape cˀju=ga} \Emph{\itshape ut-i+bəə}\\
  {MES-ADNZ} {person=NOM} {hit-INF+role}\\
  \glt \hspaceThis{TM:} ‘That person (fills) the role of hitting (a big bell in emergency).’
\ex\label{ex:4.28d}\relax[Co: 120415\_00.txt]\\\relax[Context: Remembering a pond that was close to the community’s watering place]\\
  {TM:} \glll {waakja} {{\textbar}nenzjuu{\textbar}} {mɨzɨkˀumbəə} {jatattu.}\\
  {\itshape waakja} {\itshape nenzjuu} {\textit{mɨzɨ+\Emph{kˀum-∅+bəə}}} {\itshape jar-tar-tu}\\
  {1PL} {always} {water+scoop-INF+role} {COP-PST-CSL}\\
  \glt\hspaceThis{TM:} ‘I would always fill the role of scooping water.’
\ex\label{ex:4.28e}\relax[El: 140227]\\{TM:} \glll {ucibəənu} {wutattoo.}\\
  \Emph{\textit{ut-i+bəə}=nu} {\itshape wur-tar=doo}\\
  {hit-INF+role=NOM} {exist-PST=ASS}\\
  \glt\hspaceThis{TM:} ‘There was person (who filled) the role of hitting (a hand drum).’
\z
\z

\noindent These compounds are very similar in structure to the V+N compound in \REF{ex:4.24b} in §\ref{bkm:Ref364286741}, e.g. \textit{hing-i+madu} (escape-INF+time). However, \textit{zjaa} ‘place’ and \textit{bəə} ‘role’ are crucially different from \textit{madu} ‘time’ in that they cannot be analyzed as filling the head slot of an NP. As is shown in (\ref{ex:4.29}a--b), they cannot be modified by NP modifiers such as adnominal clasues.


\ea\label{ex:4.29}Cannot be modified by adnominal clauses
\ea\label{ex.4.29a} [El: 130816]\\
TM: \glll * kumoo {asɨbjun} {zjaadoo.}\\
  {} {\itshape  ku-ma=ja} \Emph{\itshape asɨb-jur-n} {\textit{zjaa=doo}}\\
  {} PROX-place=TOP {play-UMRK-PTCP} {place=ASS}\\
\glt\hspaceThis{TM: *} (Intended meaning) ‘Here is the place to play.’
\ex\label{ex:4.29b} [El: 130816]\\
 TM: \glll * arəə {ucjun} {bəədoo.}\\
  {} {\itshape  a-rɨ=ja} \Emph{\itshape ut-jur-n} {\textit{bəə=doo}}\\
  {} {DIST-NLZ=TOP} {hit-UMRK-PTCP} {role=ASS}\\
 \glt\hspaceThis{TM: *} (Intended meaning) ‘That person fills the role to hit (the bell).’
\z
\z
\noindent The above examples show that \textit{zjaa} ‘place’ and \textit{bəə} ‘role’ cannot head an NP. In this regard, they are distinct from formal nouns (see \sectref{sec:key:6.2.2}).

By contrast, the noun \textit{madu} ‘time’ can be modified by an adnominal clause just as in the case of ordinary nouns as in \REF{ex:4.30a}. Additionally, \textit{madu} ‘time’ can be used without any NP modifier as in \REF{ex:4.30b}. On the contrary, \textit{zjaa} ‘place’ and \textit{bəə} ‘role’ cannot be used in either case.

\ea\label{ex:4.30}
\ea\label{ex:4.30a}Can be modified by an adnominal clause\hfill\relax[El: 130816]\\
{TM:} \glll {asɨbjun} {madunkjoo} {nən.}\\
 {\itshape asɨb-jur-n} \textit{\Emph{madu}=nkja=ja} {\itshape nə-an}\\
 {\{[play-UMRK-PTCP]\textsubscript{Adnominal clause}} {time\}\textsubscript{NP}=APPR=TOP} {exist-NEG}\\
 \glt\hspaceThis{TM:} ‘There is no time to play.’
\ex\label{ex:4.30b}Can be used without any NP modifier\hfill\relax[El: 130816]\\
{TM:} \glll {uroo} {madoo} {nənna?}\\
 {\itshape ura=ja} \textit{\Emph{madu}=ja} {\itshape nə-an=na}\\
 {2.NHON.SG=TOP} {\{time\}\textsubscript{NP}=TOP} {exist-NEG=PLQ}\\
 \glt\hspaceThis{TM:} ‘Don’t you have the time?’
\z
\z

\noindent The comparison between \textit{zjaa} ‘place’ and \textit{bəə} ‘role’ on one hand, \textit{madu} ‘time’ on the other indicates that the former morphemes are bound nominal roots which cannot head an NP by itself. Hence, they are “special types” of the root which occurs only in compounding.

The second type of special componds involve \textit{mai} (OBL), \textit{madəə} ‘fail to,’ and \textit{gjaa} (PURP). These nominal stems are similar to \textit{zjaa} ‘place’ and \textit{bəə} ‘role’ in that they are always preceded by verbal infinitives and cannot head an NP. In \REF{ex:4.31}, \textit{mai} (OBL) serves as the nominal predicate. 

\ea\label{ex:4.31}
\textit{mai} (OBL) in the deontic modality\\
\ea \relax [Context: Remembering the bankruptcy of a shop in the past] [Co: 120415\_01.txt]\label{ex:4.31a}\\
 {TM:} \gllll {\textbar}sjeiri{\textbar} {siimai} {jatancjɨ} {aran?}\\
  {\itshape sjeiri} \Emph{\itshape sɨr-i+mai} {\itshape jar-tar-n=ccjɨ} {\itshape ar-an}\\
  {[disposal {do-INF+OBL}} {COP-PST-PTCP]=QT} {COP-NEG}\\
  {[Nominal predicate]} \\
  \glt\hspaceThis{TM:}\parbox{\linewidth - \widthof{TM:}}{‘(The people who had invested their money in the shop) had to dispose (the goods), hadn’t they?’}
\ex\relax [El: 111105]\\\label{ex:4.31b}
TM:\gllll kakimaija {aranta.}\\
  \textit{\Emph{kak-i+mai=}ja} {\itshape ar-an-tar}\\
  {[write-INF+OBL=TOP} {COP-NEG-PST]}\\
  {[Nominal predicate]}\\
  \glt\hspaceThis{TM:} ‘(It) is not necessary to write.’
\z
\z

\noindent As is illustrated in the above examples, \textit{mai} (OBL) designates “deontic modality” \citep[823]{Lyons1977}. When \textit{mai} (OBL) occurs in negative, the sentence means that there is no obligation to do the action indicated by the verbal stem as in \REF{ex:4.31b}. In addtion, \textit{mai} (OBL) designates “epistemic modality” \citep[793--809]{Lyons1977} as well, as in \REF{ex:4.32}.

\ea \label{ex:4.32}\textit{mai} (OBL) in the epistemic modality\hfill\relax[El: 120929]\\
TM: \gllll təəhunu {kjuncjuuba,} {amɨn} {huimaidoojaa.}\\
 {\itshape təəhu=nu} {\itshape k-jur-n=ccjɨ+jˀ-ba} {\itshape amɨ=n} \textit{\Emph{hur-i+mai}=doo=jaa}\\
 {typhoon=NOM} {come-UMRK-PTCP=QT+say-CSL} {[rain]=also} {[fall-INF+OBL]=ASS=SOL}\\
  {} {} {[Subject]}  {[Nominal predicate]}\\
\glt\hspaceThis{TM:} ‘Since (they said ) that the typhoon will come, it must rain [lit. the rain must fall].’
\z
\noindent This epistemic use of \textit{mai} (OBL) is only attested in elicitation.

In \REF{ex:4.33}, \textit{madəə} depicts that the action denoted by the stem failed to complete. Syntactically, the compound fills the predicate slot as in \REF{ex:4.33a} or fills the complement slot of the light verb construction (LVC) as in \REF{ex:4.33b}.

\ea \label{ex:4.33}
\ea \textit{madəə} ‘fail to’ in the predicate\hfill\relax[El: 111105]\\\label{ex:4.33a}
TM: \gllll {kakimadəə} {jata.}\\
 \Emph{\itshape kak-i+madəə} {\itshape jar-tar}\\
 {[write-INF+fail.to} {COP-PST]}\\
 {[Nominal predicate]}\\
\glt\hspaceThis{TM:} ‘(I wanted to write, but I) failed to write.’
\ex \textit{madəə} ‘fail to’ in the complement slot of LVC\hfill\relax[El: 111105]\\\label{ex:4.33b}
{TM:} \gllll {kakimadəə} {sja.}\\
 \Emph{\itshape kak-i+madəə} {\itshape sɨr-tar}\\
 {[write-INF+fail.to]} {do-PST}\\
 {[Complement]}\\
\glt\hspaceThis{TM:} ‘(I wanted to write, but I) failed to write.’
\z
\z


The final example is \textit{gjaa} (PURP), which means that the subject has a purpose to do the action indicated by the verbal stem. Syntactically, it fills the predicate as in (4-34 a) or is followed by the genitive case as in (4-34 b). Additionally, it can fill the complement slot of the deictic motion verbs \textit{ik-} ‘go’ and \textit{k-} ‘come’ as in (4-34 c-d).

\ea\label{ex:4.34}
\ea\label{ex:4.34a} \textit{gjaa} (PURP) in the predicate\hfill\relax[Co: 111113\_01.txt]\\
 \relax[Context: Explaining the difference between the Bon festival and the celebration of the New Year’s day]\\
 {TM:} \glll {{\textbar}sjoogacu{\textbar}ja,} {naa,} {jˀuuboo,} {namanu}\\ 
  {\itshape sjoogacu=ja} {\itshape naa} {\itshape jˀ-boo}  {\itshape nama=nu}\\ 
  {New.Year’s.day=TOP} {FIL} {say-CND} {now=GEN}\\ 
  \hspaceThis{TM:} \gllll {{\textbar}nentoo{\textbar}} {jˀiigjaa} {jappa.}\\
{\itshape nentoo} \Emph{\itshape jˀ-i+gjaa} {\itshape jar-ba}\\
{beginning.of.a.year} {[say-INF+PURP} {say-CSL]}\\
   {[Nominal predicate]}\\
  \glt\hspaceThis{TM:} \parbox{\linewidth-\widthof{TM:}}{‘About the New Year’s day, (the relatives gather just) in order to say (what), if we call (it in the terms) of these days, (we call) New year greetings.’}
  
\ex\label{ex:4.34b} \textit{gjaa} (PURP) followed by \textit{nu} (GEN)\hfill\relax[Co: 111113\_01.txt]\\
 {TM:} \gllll {jˀiigjaanu} {cɨmuisjɨ} {acɨmajunwakejo.}\\
  \Emph{\textit{jˀ-i+gjaa=nu}} {\itshape cɨmui=sjɨ} {\itshape acɨmar-jur-n=wake=joo}\\
  {[say-INF+PURP]=GEN} {intention=INST} {gather-UMRK-PTCP=CFP=CFM1}\\
  {[NP]=GEN}\\
  \glt\hspaceThis{TM:} \parbox{\linewidth-\widthof{TM:}}{‘(The relatives) gather (as if) they intended to say (only New year greetings.’}

\ex\label{ex:4.34c} \textit{gjaa} (PURP) in the complement slot of \textit{ik-} ‘go’\hfill\relax[Fo: 090307\_00.txt]\\
 {TM:} \gllll {usi} {tuigjaa} {izjattoo,}\\
  {\itshape usi} \Emph{\itshape tur-i+gjaa} {\itshape ik-tar-too}\\
  {cow} {[take-INF+PURP]} {[go-PST-CSL]}\\
   {} {[Complement]} {[Lexical verb]}\\
  \glt\hspaceThis{TM:} ‘(The man) went to take the cow, and then ...’
  
\ex\label{ex:4.34d}\textit{gjaa} (PURP) in the complement slot of \textit{k-} ‘come’\hfill\relax[Co: 101023\_01.txt]\\
 {TM:} \gllll {masakoga} {{\textbar}asaban{\textbar}} {kamgjaa} {kˀuuboo,}\\ 
  {\itshape masako=ga} {\itshape asa+ban} {\itshape kam-∅+gjaa} {\itshape kˀ-boo}\\ 
  {Masako=NOM} {morning+evening} {[eat-INF+PURP]} {[come-CND]}\\ 
  {} {} {[Complement]} {[Lexical verb]}\\
  \hspaceThis{TM:} \glll {jazin}   {{\textbar}medamajaki{\textbar}.}\\
{\itshape jazin} {medamajaki}\\
{necessarily} {sunny.side.up}\\
  \glt\hspaceThis{TM:} \parbox{\linewidth-\widthof{TM:}}{‘When Masako comes to eat the breakfast and the supper, (I) necessarily (bake the eggs) sunny side up.’}
\z
\z
% It should be mentioned that some preceding verbal stems in the compounds of V+N can retain their original argument structure (or “internal syntax” in \citealt{Haspelmath1996}: 52) as in (4-35 b-d).
% 
% \tablefirsthead{}
% 
% \tabletail{}
% \tablelasttail{}
% \begin{tabularx}{\textwidth}{XXXXXXXX}
% \lsptoprule
% { (4\nobreakdash-\stepcounter{Remark}{\theRemark})} {{ a. Original argument structure}}\\
% { TM:} { wanna} { urɨba} {{ kakjuttoo.}}\\
%  {\itshape wan=ja} {\itshape u-rɨ=ba} {{\itshape kak-jur=doo}}\\
%  { 1SG=TOP} { MES-NLZ=ACC} {{ write-UMRK=ASS}}\\
%   { Object} {}\\
%  {‘I will write it.’}\\
%  {\raggedleft [El: 130816]}\\
% {{ b. \textit{bəə} ‘role’}}\\
% { TM:} { wanna} { urɨba} {{ kakibəə}} {{ zjajaa.}}\\
%  {\itshape wan=ja} {\itshape u-rɨ=ba} {{\itshape kak-i+bəə}} {{\itshape zja=jaa}}\\
%  { 1SG=TOP} { MES-NLZ=ACC} {{ write-INF+role}} {{ COP=SOL}}\\
%   { Object} {}\\
%  {‘I fill the role to write it.’ [lit. ‘I am the role to write it.’]}\\
%  {\raggedleft [El: 130816]}\\
% {{ c. \textit{madu} ‘time’}}\\
% { TM:} { wanna} {{ urɨnkjoo}} {{ kakimadoo}} { nəndoo.}\\
%  {\itshape wan=ja} {{\itshape u-rɨ=nkja=ja}} {{\itshape kak-i+madu=ja}} {\itshape nə-an=doo}\\
%  { 1SG=TOP} {{ MES-NLZ=APPR=TOP}} {{ write-INF+time=TOP}} { exist-NEG=ASS}\\
%   {{ Object}} {} \\
%  {‘I have no time to write it.’ [lit. ‘For me, there is no time to write it.’]}\\
%  {\raggedleft [El: 130816]}\\
% {{ d. \textit{mai} (OBL)}}\\
% { TM:} { wanna} { urɨba} {{ kakimaidoo.}}\\
%  {\itshape wan=ja} {\itshape u-rɨ=ba} {{\itshape kak-i+mai=doo}}\\
%  { 1SG=TOP} { MES-NLZ=ACC} {{ write-INF+OBL=ASS}}\\
%   { Object} {}\\
%  {‘I have to write it.’}\\
%  {\raggedleft [El: 130816]}\\
% \lspbottomrule
% \end{tabularx}
% The example in (4-35 a) shows the original argument structure of \textit{kak-} ‘write,’ whose object \textit{u-rɨ} ‘that’ is marked by \textit{ba} (ACC). The examples in (4-35 b-d) show that the compounded \textit{kak-} ‘write’ still retains its object, although I could not elicitate the speaker to say an example where the object of \textit{kak-i+madu} (write-INF+time) was marked by \textit{ba} (ACC). Furtheremore, \textit{zjaa} ‘place’ cannot retain its original argument structure, e.g., */kumoo mɨzɨba numzjaadoo/ \textit{ku-ma=ja} \textit{mɨzɨ=ba} \textit{num-∅+zjaa=doo} (PROX-place=TOP water=ACC drink-INF+place=ASS) [Intended meaning] ‘Here is the place to drink water.’
% 
% Strictly speaking, the alleged nominal stems in the above examples, i.e. \textit{zjaa} ‘place,’ \textit{bəə} ‘role,’ \textit{mai} (OBL), \textit{madəə} ‘fail to,’ and \textit{gjaa} (PURP), cannot regarded as stems (or roots), since they cannot start an utterance by themselves (see \sectref{sec:key:4.2.3}). In fact, they are thought to be in the process of grammaticalization from roots to affixes (or nominalizers). However, I do not regard them as nominalizers in modern Yuwan, since their initial stems always become infinitives, which is the same as the ordinary type compounding (see \sectref{sec:key:4.2.3.1}). On the other hand, the genuine nominalizer in Yuwan, i.e. \textit{{}-jaa} ‘person,’ can directly attach to verbal roots, e.g., /hikjaa/ \textit{hik-jaa} (play-person) ‘player’ (see also §\ref{bkm:Ref355525814}). Therefore, I propose that the above forms are compounds (not nominalizer affixes). In order to distinguish these “nominal stems” from the ordinary nominal stems such as \textit{hinzjaa} ‘goat,’ it may be appropriate to call the former the “nominal stems only for compounding.”
% 
%   Finally, I will present examples of \textit{cja} ‘want,’ \textit{cjagɨ} ‘seem,’ \textit{jass} ‘easy,’ and \textit{gussj} ‘difficult.’ In principle, these adjectival stems always follow the verbal infinitives, and the resulting compound is always an adjectival stem. The example of \textit{cja} ‘want’ is shown below, and other examples are shown in §\ref{bkm:Ref347179613}.
% 
% \tablefirsthead{}
% 
% \tabletail{}
% \tablelasttail{}
% \begin{tabularx}{\textwidth}{XXXXX}
% \lsptoprule
% { (4\nobreakdash-\stepcounter{Remark}{\theRemark})} {{ \textit{cja} ‘want’}}\\
% {{ [Context: TM is introducing the present author to the hearer US saying that the present author has been looking for a good language teacher in the community.]}}\\
% { TM:} { simakutuba}  { narəəcjasaccjɨ} { jˀicjɨ,}\\
%  {\itshape sima+kutuba} { \textit{naraw-i+cja{}-sa=ccjɨ}} {\itshape jˀ-tɨ}\\
%  { community+language} { learn-INF+want-ADJ=QT} { say-SEQ}\\
%  {{ ‘(He) said, ‘(I) want to learn the language of the community,’ and ...’}}\\
%  {\raggedleft [Co: 110328\_00.txt]}\\
% \lspbottomrule
% \end{tabularx}
% Strictly speaking, the adjectival root \textit{cja-} ‘want’ in (4-36) cannot be analyzed as a stem (or a root) since it cannot start an utterance by itself (see \sectref{sec:key:4.2.3}). The same point can be made about \textit{cjagɨ-} ‘seem,’ \textit{jass-} ‘easy,’ and \textit{gussj-} ‘difficult.’ In fact, they are in the process of grammaticalization from roots to affixes as well as the “nominal stems only for compounding” discussed above. However, the phonotactic behavior of \textit{jass-} ‘easy’ discussed in (2-9) of \sectref{sec:key:2.3.2} slightly shows that it retains non-affixal property; in short, \textit{jass-} ‘easy’ does not induce palatalization of the preceding consonant on the contrary to the nominalizer \textit{{}-jaa} (NLZ), which induce palatalization. The above adjectival stems can also retain the original argument structures of the verbal stems. For example, \textit{sima+kutuba} ‘the language of the community’ is the argument of \textit{naraw-} ‘learn’ in (4-36). In order to distinguish these “adjectival stems” from the ordinary adjectival stems such as \textit{kjura-} ‘beautiful,’ it may be appropriate to call the former the “adjectival stems only for compounding.”
% 
% \subsubsection{Reduplication}
% \label{bkm:Ref364023547}\hypertarget{RefHeadingToc395697006}{}
% Reduplication in Yuwan concerns full reduplication, not partial reduplication. A reduplicated form consists of the base and the reduplicant. The reduplicant precedes the base, e.g. /sabii+sabi/ ‘smoothly,’ where /sabii/ is the reduplicant and /sabi/ is the base. Syntactically, reduplicated forms made of adjectival roots or onomatopoeic roots function as adverbs (see §\ref{bkm:Ref356042174} and §\ref{bkm:Ref347227985}). The reduplicated form made of the reflexive pronoun functions as a nominal (see §\ref{bkm:Ref367142588}). In some reduplicated forms, the base undergoes the sequential voicing (or “rendaku”), which is also founded in compounding (see \ref{bkm:Ref347181344} for more details). However, reduplication is different from compounding in other morphophonological characteristics. In paticular, reduplicated forms undergo vowel lengthening in some environments. Vowel lengthening occurs in reduplicants if neither the penultimate nor final syllable of the base is heavy as in (4-37 a-b) (see §\ref{bkm:Ref356042913} for more details). On the contrary, if the reduplicated form in the same condition is followed by a morpheme that is composed of only a syllable with a mora, e.g. \textit{{}-tu} (ADVZ) or \textit{nu} (GEN), the final vowel of the base (not the reduplicant) is lengthened as in (4-37 c-d) (see also §\ref{bkm:Ref347227985} and §\ref{bkm:Ref367142588}). 
% 
% \tablefirsthead{}
% 
% \tabletail{}
% \tablelasttail{}
% \begin{tabularx}{\textwidth}{XXXXXXX}
% \lsptoprule
% { (4\nobreakdash-\stepcounter{Remark}{\theRemark})} {{ Reduplication}}\\
% {{ A. Reduplicant is lengthened}}\\
% { a.} {\itshape siju-} { ‘white’} { >} { /sijuu+ziju/} { ‘whitely’}\\
% { b.} {\itshape sabi} { ‘smoothly’} { >} { /sabii+sabi/} { ‘smoothly’}\\
% {{ B. Base is lengthened}}\\
% { c.} {\itshape siju-} { ‘white’} { >} { /siju+zijuu{}-tu/} { ‘whitely’}\\
% { d.} {\itshape nusi} { (RFL)} { >} { /nusi+nusii=nu/} { ‘each of oneselves’’}\\
% \lspbottomrule
% \end{tabularx}
% The reduplicated forms that function as adverbs as in (4-37 a-c) express emphasis, but the reduplicated nominal as in (4-37 d) is roughly translated as ‘each’ in English (see §\ref{bkm:Ref367142588}).
% 
%   Additionally, the verbal infinitive in Yuwan may be reduplicated, although it does not go through the lengthening of the vowel discussed above.
% 
% \tablefirsthead{}
% 
% \tabletail{}
% \tablelasttail{}
% \begin{tabularx}{\textwidth}{XXXXXXXXXXXXXXXX}
% \lsptoprule
% { (4\nobreakdash-\stepcounter{Remark}{\theRemark})} { a.} { TM:} {{ umaga}} {{ naikwanu}} {{ dɨkɨppoo,}}\\
%   {{\itshape u-ma=ga}} {{\itshape naikwa=nu}} {{\itshape dɨkɨr-boo}}\\
%   {{ MES-place=FOC}} {{ department.of.internal.medicine=NOM}} {{ be.set.up-CND}}\\
%   {{ {\textbar}kamera{\textbar}}} {{ numgja}} {{ ikiiki.}}\\
%   {{\itshape kamera}} {{\itshape num-∅+gja}} {{\itshape ik-i+ik-i}}\\
%   {{ camera}} {{ swallow-INF+PURP}} {{ go-INF+go-INF}}\\
%   {{ ‘After the department of internal medicine was set up there, (I) often went (there) in order to swallow the (stomach) camera.’}}\\
%   {\raggedleft [Co: 120415\_01.txt]}\\
% { b.} { TM:} { abɨnəə} {{ gan}} {{ naroocjəə}} {{ siisii.}}\\
%   {\itshape abɨnəə} {{\itshape gan}} {{\itshape nar-oo=ccjɨ=ja}} {{\itshape sɨr-i+sɨr-i}}\\
%   { barely} {{ cancer}} {{ become-INT=QT=TOP}} {{ do-INF+do-INF}}\\
%   {{ ‘(I) was about to get cancer many times.’ [lit. ‘(I) did and did to become cancer’]}}\\
%   {\raggedleft [Co: 120415\_01.txt]}\\
% { c.} { TM:} {{ {\textbar}poketto{\textbar}nan}} {{ ɨrɨtɨ,}} {{ mucjɨ}} {{ cˀjəə,}} { ukkaci}\\
%   {{\itshape poketto=nan}} {{\itshape ɨrɨr-tɨ}} {{\itshape mut-tɨ}} {{\itshape k-tɨ=ja}} {\itshape u-rɨ=kaci}\\
%   {{ pocket=LOC1}} {{ put.in-SEQ}} {{ have-SEQ}} {{ come-SEQ=TOP}} { MES-NLZ=ALL}\\
%   {{ ɨrɨɨrɨ.}} {}\\
%   {{\itshape ɨrɨr-∅+ɨrɨr-∅}} {}\\
%   {{ put.in-INF+put.in-INF}} {}\\
%   {{ ‘(The old man) put (the oranges) in (his) pocket, brought (them), and put (them) into that [i.e. a large basket] again and again.’}}\\
%   {\raggedleft [PF: 090305\_01.txt]}\\
% \lspbottomrule
% \end{tabularx}
% The above examples show that the reduplication of the infinitive designates the iteration of the action.
% 
% \subsubsection{“Rendaku” (sequential voicing)}
% \label{bkm:Ref347181344}\hypertarget{RefHeadingToc395697007}{}
% The initial consonant of the non-initial stem of a certain kinds of compounds may be voiced if it is originally voiceless. In the following rule schemata, morphosyntactic information is supplied with its label (e.g., “Stem”) or with square brackets and labels at the lower right (e.g., “[    ]\textsubscript{stem}”).
% 
% \tablefirsthead{}
% 
% \tabletail{}
% \tablelasttail{}
% \begin{tabularx}{\textwidth}{XXXXXXXX}
% \lsptoprule
% { (4\nobreakdash-\stepcounter{Remark}{\theRemark})} {{ Rule shema}}\\
% { C} { >} { C} { /} { Stem} { +} { [ \_   ]\textsubscript{stem}}\\
% { [-v]}  { [+v]}    \\
% \lspbottomrule
% \end{tabularx}
% \tablefirsthead{}
% 
% \tabletail{}
% \tablelasttail{}
% \begin{tabularx}{\textwidth}{XXXXXXXXXX}
% \lsptoprule
% { (4\nobreakdash-\stepcounter{Remark}{\theRemark})} {{ Examples}}\\
% { a.} {{ t > d}} \\
%  { taa} { ‘high’} { +} { taatu} { (high.ADVZ)} { >} { taadaatu} { ‘highly’}\\
% { b.} {{ s > z}} \\
%  { kˀuru} { ‘black’} { +} { sataa} { ‘sugar’} { >} { kˀuruzataa} { ‘black sugar’}\\
% { c.} {{ k > g}} \\
%  { kui} { ‘voice’} { +} { kjurasa} { (beautiful.ADJ)} { >} { kuigjurasa} { ‘of beautiful voice’}\\
% { d.} {{ kˀ > g}}\\
%  { kˀuru} { ‘black’} { +} { kˀuru} { ‘black’} { >} { kˀuruuguru} { ‘blackly’}\\
% { e.} {{ c > z}} \\
%  { sɨnɨtooraa} { ‘sluggard’} { +} { cɨkɨ} { (pickle.INF)} { >} { sɨnɨtoorazɨkɨ} { ‘lightly-pickled radish’}\\
% { f.} {{ h > b}} \\
%  { sɨcɨzɨ} { ‘cycad’} { +} { haa} { ‘leaf’} { >} { sɨcɨzɨbaa} { ‘cycad leaf’}\\
% \lspbottomrule
% \end{tabularx}
% Regarding (4-40 a-d), the stem-initial phonemes alternate with their voiced counterparts in §\ref{bkm:Ref381399106}. On the other hand, the stem-initial voiced phonemes in (4-40 e-f) are different from the original phonemes both in the articulatory place and manner. The synchronic idiosyncracy in (4-40 e-f) is due to the histrical sound change. As for (4-40 e), internal reconstruction tells us that there was a voiced alveolar affricate */ʣ/, but the difference between the voiced alveolar affricate and fricative disappeared over time, and they have merged to /z/. Similarly, for (4-40 f), internal reconstruction tells us that the contemporary /h/ was */p/, which yields the perfect correspondence between */p/ and */b/ (cf. \citealt{Ueda1898}: 41-46).
% 
% Sequential voicing is very common, but not obligatory in every compound, as the following examples show.
% 
% \tablefirsthead{}
% 
% \tabletail{}
% \tablelasttail{}
% \begin{tabularx}{\textwidth}{XXXXXXXXXX}
% \lsptoprule
% { (4\nobreakdash-\stepcounter{Remark}{\theRemark})}  {{ hu > hu}}\\
%  { nui} { (ride.INF)} { +} { hunɨ} { ‘boat’} { >} { nuihunɨ} { ‘coffin’}\\
% { cf.} {{ hu > bu}}\\
%  { koo} { ‘river’} { +} { hunɨ} { ‘boat’} { >} { koobunɨ} { ‘riverboat’}\\
% \lspbottomrule
% \end{tabularx}
% We can, however, specify the environment, where sequential voicing does not occur. If the non-initial stem contains at least one phonologically-voiced phoneme (see §\ref{bkm:Ref381399106}), the compound cannot undergo sequential voicing. This process is known as “Lyman’s law” in Japanese linguistics \citep{Lyman1894}.
% 
% \tablefirsthead{}
% 
% \tabletail{}
% \tablelasttail{}
% \begin{tabularx}{\textwidth}{XXXXXXXXXX}
% \lsptoprule
% { (4\nobreakdash-\stepcounter{Remark}{\theRemark})} { a.} {{ /k/ > /k/: the following stem includes /b/}}\\
%  { sima} { ‘community’} { +} { kutuba} { ‘language’} { >} { simakutuba} { (*simagutuba)}\\
%        {{ ‘language of community’}}\\
% { b.} {{ /k/ > /k/: the following stem includes /z/}}\\
%  { nisi\footnotemark{}} { ‘north’} { +} { kazɨ} { ‘wind’} { >} { nisikazɨ} { (*nisigazɨ)}\\
%        {{ ‘north wind’}}\\
% { c.} {{ /k/ > /g/: the following stem includes /n/}}\\
%  { basja} { ‘banana plant’} { +} { kin} { ‘clothes’} { >} { basjagin} { (*basjakin)}\\
%        {{ ‘clothes made of fiber of banana plant’}}\\
% \lspbottomrule
% \end{tabularx}
% \footnotetext{ \textit{nisi} is a fossil morpheme, and it only appears in compounds such as \textit{mii+nisi} (new+north) ‘an autumn wind.’ If a speaker wants to indicate ‘north’ in a monomorphemic word, the word \textit{kita} ‘north’ is used.}
% 
% There should be distinction between phonological voicing and phonetical voincing in understanding this rule. For example, /b/ and /z/ in (4-42 a-b), which are voiced both in terms of phonological voincing and phonetic voicing, are subject to this constraint, whereas /n/ in (4-42 c), which is only phonetically voiced, escapes from this constraint.
% 
%   Before concluding this section, attention should be paid to a case in which sequential voicing helps us determine the phonological analysis of certain phonemes. For example, [(d͡)ʑi] is analyzed as /zi/ (not /di/), and [t͡ɕi] is analyzed as /ci/ (not /ti/). An example about [(d͡)ʑi] is shown below.
% 
% \tablefirsthead{}
% 
% \tabletail{}
% \tablelasttail{}
% \begin{tabularx}{\textwidth}{XXXXXXXXX}
% \lsptoprule
% { (4\nobreakdash-\stepcounter{Remark}{\theRemark})} {{ si > zi}} \\
% { siju} { ‘white’} { +} { siju} { ‘white’} { >} { [ɕijuː(d͡)ʑiju]} { ‘whitely’}\\
% \lspbottomrule
% \end{tabularx}
% In (4-43), the /si/ [ɕi]\footnote{For the reason for regarding [ɕi] as /si/, see the footnote Error: Reference source not found in §\ref{bkm:Ref347180720}.} of \textit{siju} ‘white’ becomes [(d͡)ʑi] in the non-initial position of compounds. Thus, we should interpret it as /zi/ not /di/. That is, if we interpret [(d͡)ʑi] as /di/, we would have to admit a certain discrepancy in the sequential voicing of //si// and //sa//. If we allow for this interpretation, //si// would become /di/ [(d͡)ʑi], e.g., /sijuudiju/ ‘whitely’ in (4-43), but //sa// would become /za/ [(d͡)ʑɑ̟], e.g., /kˀuruzataa/ ‘black sugar’ as in (4-40 b). This would mean that not only /z/ but also /d/ would be considered voiced phonemes formed from the sequential voicing of //s//, and we would have to assume that some voiced phonemes (in sequential voicing) would be chosen depending on the phonological environments, i.e. /d/ occurs before /i/, and /z/ occurs elsewhere. On the other hand, if we admit [(d͡)ʑi] as /zi/, this mismatch does not occur, and the result of sequential voicing is transparent, i.e. //s// > /z/ in all cases. Given that we have now recognized [(d͡)ʑi] as /zi/ (instead of /di/), we must also recognize [t͡ɕi] as /ci/ (instead of /ti/), since /ci/ [t͡ɕi] becomes [(d͡)ʑi] as in (4-44) and /cɨ/ [t͡sɨ] becomes [(d͡)zɨ] as in (4-40 e).
% 
% \tablefirsthead{}
% 
% \tabletail{}
% \tablelasttail{}
% \begin{tabularx}{\textwidth}{XXXXXXXXX}
% \lsptoprule
% { (4\nobreakdash-\stepcounter{Remark}{\theRemark})} {{ /ci/ [t͡ɕi] > /zi/ [(d͡)ʑi]}} \\
% { baka} { ‘fool’} { +} { /cikjara/}
% 
% { [t͡ɕikʲɑ̟ɾɑ̟]} { ‘power’} { >} { /bakazikjara/}
% 
% { [bɑ̟kɑ̟(d͡)ʑikʲɑ̟ɾɑ̟]} { ‘enormous strength’}\\
% \lspbottomrule
% \end{tabularx}
% \subsection{Compounding versus phrase}
% \label{bkm:Ref367146338}\hypertarget{RefHeadingToc395697008}{}
% There are two ways of modifying a noun: (a) compounding, which is morphological, and (b) phrasal modification, which is syntactic. In compounding, several adjectival roots in Yuwan (e.g. \textit{kjura-} ‘beautiful’ and \textit{inja-} ‘small’) are productive in forming compounds with transparent meanings, e.g. \textit{kjura+nɨsəə} (beautiful+young.man) ‘beautiful young man’ or \textit{kjura+jaa} (beautiful+house) ‘beautiful house.’ In phrasal modification, there are various ways of modifying a noun; modification by the genitive case particle, adnominals, and adnominal clauses.
% 
% \tablefirsthead{}
% 
% \tabletail{}
% \tablelasttail{}
% \begin{tabularx}{\textwidth}{XXXXXXXX}
% \lsptoprule
% { (4\nobreakdash-\stepcounter{Remark}{\theRemark})} { a.} {{ Compound}}\\
%  { TM:} {{ kjuranɨsəə}} {{ jatancjɨjo.}}\\
%   {{\itshape kjura+nɨsəə}} {{\itshape jar-tar-n=ccjɨ=joo}}\\
%   {{ <beautiful+young.man>\textsubscript{Compound}}} {{ COP-PST-PTCP=QT=CFM1}}\\
%   {{ ‘He was a beautiful young man.’}}\\
%   {\raggedleft [Co: 120415\_00.txt]}\\
% { b.} {{ Modifier and head in a nominal phrase}}\\
%  { TM:} { waa} {{ uinannja}} { micjai,}\\
%   { \{\textit{waa}\textsubscript{Modifier}} {{ \textit{ui}\textsubscript{Head}\}\textsubscript{Phrase}\textit{=nan=ja}}} {\itshape micjai}\\
%   { 1SG.ADNZ} {{ upper.side=LOC1=TOP}} { three.CLF.person}\\
%   {{ jutaidu}} {{ wuppa.}}\\
%   {{\itshape jutai=du}} {{\itshape wur-ba}}\\
%   {{ four.CLF.person=FOC}} {{ exist-CSL}}\\
%   {{ ‘There are three, four persons older than me [lit. on my upper side].’}}\\
%   {\raggedleft [Co: 111113\_02.txt]}\\
% \lspbottomrule
% \end{tabularx}
% As is illustrated in above examples, both types of modification (compounding and phrasal modification) exhibit a strong tendency for the head to be a common noun.
% 
% However, these two types of modification should be distinguished based on the following two characteristics: (a) occurrence of sequential voicing and (b) possibility of insertion of a clause.
% 
% With regard to (a), compounding may induce sequential voicing (i.e. “rendaku,” see §\ref{bkm:Ref347181344} for more details), but phrasal modification does not. That is, if sequential voicing applies, the whole composition must be a compound. For example, \textit{kumui} ‘hole’ has a voiceless consonant //k// in its initial position, but it becomes /g/ if it fills the second slot of a compound, as in /hansɨ+gumui/ \textit{hansɨ+kumui} (sweet.potato+hole) ‘a hole in the ground to store sweet potatoes.’ In fact, there is a case where the following stem does not go through sequential voicing, e.g., (4-45 a), and in this case, we could not distinguish it from the phrasal components such as (4-45 b).
% 
% With regard to (b), a compound cannot be interrupted by a clause because it is a word, whereas a phrase can.
% 
% \tablefirsthead{}
% 
% \tabletail{}
% \tablelasttail{}
% \begin{tabularx}{\textwidth}{XXXXXX}
% \lsptoprule
% { (4\nobreakdash-\stepcounter{Remark}{\theRemark})} { a.} {{ Compound}}\\
%  { TM:} {{ *kjurainjasannɨsəə}}\\
%   {{ <\textit{kjura+}[\textit{inja-sa+a-n}]\textsubscript{Clause}+nɨsəə>\textsubscript{Compound}}}\\
%   {{ beautiful+ small-ADJ+STV-PTCP+young.man}}\\
%   {{ (Intended meaning) ‘a beautiful small young man.’}}\\
%  {\raggedleft [El: 130812]}\\
% { b.} {{ Modifier and head in a phrase}}\\
%  {{ [Context: Talking about a man who used to dub tapes of songs voluntarily for}
% 
% { villagers; TM: ‘He said his recorder was not useful these days, and...’]}}\\
%  { TM:} { waa} { injasan} { {\textbar}kasetto{\textbar}kkwagadɨ}\\
%   { \{\textit{waa}} { [\textit{inja-sa+ar-n}]\textsubscript{Clause}} { \textit{kasetto-kkwa}\}\textsubscript{Phrase}\textit{=gadɨ}}\\
%   { 1SG.ADNZ} { small-ADJ+STV-PTCP} { cassette.recorder-DIM=LMT}\\
%   { muccjɨ} { izjɨ,} \\
%   {\itshape mut-tɨ} {\itshape ik-tɨ} \\
%   { have-SEQ} { go-SEQ} \\
%   {{ ‘(He) took even my small cassette recorder [lit. my cassette recorder that is small],}
% 
% { and...’}}\\
%  {\raggedleft [Co: 120415\_01.txt]}\\
% \lspbottomrule
% \end{tabularx}
% These examples show that the components of the NP in (4-46 b), i.e. /waa/ ‘my’ and /kasetto/ ‘cassette recorder,’ can be interrupted by the adnominal clause /injasan/ ‘(something) that is small.’ This example can be analyzed as follows. First, the modifier \textit{injasan} and the head \textit{kasetto} ‘cassette recorder’ consititute an NP, which recursively fills the head slot of a superordinate NP. This superordinate NP has the modifier \textit{waa} ‘my.’ By contrast, the components of the compound cannot be interrupted by the adnominal clause as in (4-46 a).
% 
%   The same argumentation can apply to the nominal juxtaposed in the modifier slot of an NP. Address nouns, e.g. \textit{anmaa} ‘mother,’ can fill the modifier slot of an NP only by themselves as in (4-47 a) (see also 6.1.1). The modifier \textit{anmaa} ‘mother’ and the head \textit{tɨɨ} ‘hand,’ which means ‘(my) mother’s hand,’ can be interrupted by the adnominal clause /hɨɨsan/ ‘(something) that is big’ as in (4-47 b), which means the combination \textit{anmaa} \textit{tɨɨ} ‘(my) mother’s hand’ is not a compound.
% 
% \tablefirsthead{}
% 
% \tabletail{}
% \tablelasttail{}
% \begin{tabularx}{\textwidth}{XXXXXXXX}
% \lsptoprule
% { (4\nobreakdash-\stepcounter{Remark}{\theRemark})} {{ Modifier and head in a phrase}}\\
% { a.} { TM:} { anmaa} {{ tɨɨ}} {}\\
%   { \{\textit{anmaa}} {{ \textit{tɨɨ}\}\textsubscript{Phrase}}} {}\\
%   { mother} {{ hand}} {}\\
%   {{ ‘(my) mother’s hand’}}\\
%  {\raggedleft [El: 140227]}\\
% { b.} { TM:} { anmaa} { hɨɨsan} {{ tɨɨnu}} { mjarɨttoo.}\\
%   { \{\textit{anmaa}} { [\textit{hɨɨ-sa+ar-n}]\textsubscript{Clause}} {{ \textit{tɨɨ}\}\textsubscript{Phrase}=nu}} {\itshape mj-arɨr=doo}\\
%   { mother} { big-ADJ+STV-PTCP} {{ hand=NOM}} { see-CAP=ASS}\\
%   {{ ‘(I) can see (my) mother’s big hand (in the picture).’}}\\
%  {\raggedleft [El: 140227]}\\
% \lspbottomrule
% \end{tabularx}
% \section{Word classes}
% \label{bkm:Ref306382551}\hypertarget{RefHeadingToc395697009}{}
% Yuwan has seven word classes: nominals, adnominals, verbs, adjectives, particles, adverbs, and interjections. The word classes are defined morphosyntactically. The criteria for the “word classes” are applied to “grammatical words” (see §\ref{bkm:Ref347179371}). Most of the word classes are free forms, but some nominals (i.e. formal nouns) and all particles are classified as clitics.
% 
% Out of approximately 1100 lexemes, the approximation of the number of each word class is as follows: nominals \REF{ex:key:700}, verbs \REF{ex:key:250}, adjectives \REF{ex:key:80}, adverbs \REF{ex:key:50}, particles \REF{ex:key:40}, interjections \REF{ex:key:10}, and adnominals \REF{ex:key:9}. Some notes on the word count. Word classes other than adnominals and particles have their own roots, e.g., nominal roots or verbal roots. Adnominals do not have “adnominal roots,” and the adnominal words are composed of the root of a cross-over category, e.g., the demonstratives root, and an adnominalizer affix (see Chapter 5). Here, the number of roots that can take adnominalizers are counted here as adnominals.
% 
% As is shown in \tabref{tab:key:26}, there are four criteria for the word class assignment.
% 
% \begin{table}
% \hypertarget{RefHeadingToc395697291}{}\textmd{Table \stepcounter{Table}{\theTable}}. Word class assignment
% \end{table}
% 
% \tablefirsthead{}
% 
% \tabletail{}
% \tablelasttail{}
% \begin{tabularx}{\textwidth}{XXXXXX} & Nominals & Adnominals & Verbs & Adjectives & The others\\
% \lsptoprule
% { Heads an NP} & + & {}- & {}- & {}- & {}-\\
% { Only appears in the modifier slot of an NP} & {}- & + & {}- & {}- & {}-\\
% { Takes a verbal inflectional affix} & {}- & {}- & + & {}- & {}-\\
% { Takes an adjectival inflectional affix} & {}- & {}- & {}- & + & {}-\\
% \lspbottomrule
% \end{tabularx}
% \subsection{Nominals}
% \hypertarget{RefHeadingToc395697010}{}
% The nominal is a word that heads an NP, e.g., \textit{hinzjaa} ‘goat’ (see Chapter 6 for more details about NPs). Nominals can be further divided into categories such as common nouns (e.g., \textit{hinzjaa} ‘goat’), address nouns (e.g., \textit{anmaa} ‘mother’), reflexives (e.g., \textit{nusi} ‘oneself’), numerals (e.g., \textit{tˀɨɨ} ‘one’), indefinites (e.g., \textit{taru-ka} ‘someone’) and formal nouns (e.g., \textit{sɨ} ‘thing; person; fact’). The first five subclasses are free forms (see Chapter 7), but the last one (i.e. formal nouns) is a clitic (see §\ref{bkm:Ref367137128} for more details). Personal pronouns such as \textit{wan} ‘I,’ demonstrative pronouns such as \textit{kurɨ} ‘this,’ and interrogative pronouns such as \textit{taru} ‘who’ are categorized as nominals. However, personal pronominals, demonstratives, and interrogatives are not always categorized into nominals since they can also become other word classes. I call them “cross-over categories,” which will be discussed in Chapter 5.
% 
% A nominal may be derived from a verbal stem (see §\ref{bkm:Ref347245173}). A few nominals that have temporal meanings, e.g., \textit{kjuu} ‘today,’ \textit{acja} ‘tomorrow,’ and \textit{kinju} ‘yesterday,’ can be used adverbially (put another way, they can convert to adverbs with no formal change) as in (4-48).
% 
% \tablefirsthead{}
% 
% \tabletail{}
% \tablelasttail{}
% \begin{tabularx}{\textwidth}{XXXX}
% \lsptoprule
% { (4\nobreakdash-\stepcounter{Remark}{\theRemark})} & \multicolumn{3}{X}{{ [Context: Speaking about the present author; TM: ‘Then, suddenly (he) came (here) yesterday.’]}}\\
% & { US:} & { kinjuu} & { umoocjɨ?}\\
% &  & {\itshape kinjuu} & {\itshape umoor-tɨ}\\
% &  & { yesterday} & { come.HON-SEQ}\\
% &  & \multicolumn{2}{X}{{ ‘Did (he) come (here) yesterday?’}}\\
% & \multicolumn{3}{X}{\raggedleft [Co: 110328\_00.txt]}\\
% \lspbottomrule
% \end{tabularx}
% \subsection{Adnominals}
% \hypertarget{RefHeadingToc395697011}{}
% There are three kinds of adnominals: personal pronominal adnominals like \textit{waa} ‘my,’ demonstrative adnominals like \textit{kun} ‘this,’ and interrogative adnominals like \textit{taa} ‘whose.’ The adnominal, e.g., \textit{kun} ‘this (one)’ and \textit{waa} ‘my,’ only occurs in the modifier slot of an NP. Even though an adnominal cannot stand alone, this feature comes from the fact that it always requires the head. That is, it is syntactically dependent. However, they exhibit much less selective restriction than clitics. 
% 
% Whereas nominals take genitive case in the modifier slot of an NP, adnominals do not. See the relevant descriptions in Chapter 5 for more details.
% 
% \subsection{Verbs}
% \hypertarget{RefHeadingToc395697012}{}
% The verb is identified by the occurence of a specific set of inflectional affixes (see §\ref{bkm:Ref381406686}), e.g., \textit{kam-ɨ} (eat-IMP) ‘Eat!’ The only exception is the copula verb \textit{zjar-}, which may lack an inflectional affix entirely (see \sectref{sec:key:8.3.3.2}). The verbal phrase is composed minimally of a verb, but it may also be composed of a lexical verb and an auxiliary verb (see §\ref{bkm:Ref361002248} for more details). Verbs involve complex morphophonological alternations (see §\ref{bkm:Ref356245430}). Verbal inflectional affixes can be grouped into four classes: finite-form affixes, participial affixes, converbal affixes, and an infinitival affix. These classes of affixes correspond to the following clause types: main clauses, adnominal clauses, adverbial clauses, and nominal clauses (see §\ref{bkm:Ref305683739} for more detai).
% 
% \subsection{Adjectives}
% \label{bkm:Ref347233244}\hypertarget{RefHeadingToc395697013}{}
% The adjective is identified by the occurence of the following set of inflectional affixes: \textit{{}-sa}/\textit{{}-soo},\textstyleFootnoteSymbol{} e.g., \textit{kjura-sa} or \textit{kjura-soo} (beautiful-ADJ) ‘beautiful.’ Adjectives and verbs are thus distinguished by the kind of inflectional affixes they carry.
% 
% Semantically, adjectival stems express various property concepts (the semantic categories conform to those of \citealt{Dixon2004}: 3-4): \textsc{dimension} (e.g., \textit{taa-} ‘high; tall,’ \textit{tuu-} ‘distant,’ \textit{inja-} ‘small’), \textsc{age} (e.g., \textit{waa-} ‘young,’ \textit{miisj-} ‘new’), \textsc{value} (e.g., \textit{jiccj-} ‘good,’ \textit{waru-} ‘bad’), \textsc{color} (e.g., \textit{aa-} ‘red,’ \textit{siju-} ‘white,’ \textit{kˀuru-} ‘black’), \textsc{physical} \textsc{property} (e.g., \textit{ubu-} ‘heavy’), \textsc{human} \textsc{propensity} (e.g., \textit{hoorasj-} ‘happy’), and \textsc{speed} (e.g., \textit{həə} ‘fast’).
% 
% Morphologically, the adjective is composed of an adjectival stem plus the adjectival inflectional affixes \textit{{}-sa}/\textit{{}-soo}. If they follow consonant-final stems, the initial morphophoneme //s// drops.
% 
% \tablefirsthead{}
% 
% \tabletail{}
% \tablelasttail{}
% \begin{tabularx}{\textwidth}{XXXXXXXXX}
% \lsptoprule
% { (4\nobreakdash-\stepcounter{Remark}{\theRemark})} & \multicolumn{8}{X}{{ Morphophonological alternation of \textit{{}-sa} (ADJ)}}\\
% & { a.} & \multicolumn{7}{X}{{ After vowel-final stem}}\\
% &  & {\itshape usɨ-} & { ‘ugly’} & { +} & {\itshape {}-sa} & { (ADJ)} & { >} & { usɨ-sa}\\
% &  & {\itshape siju-} & { ‘white’} &  &  &  & { >} & { siju-sa}\\
% &  & {\itshape hagoo-} & { ‘mortified’} &  &  &  & { >} & { hagoo-sa}\\
% &  & {\itshape judəə-} & { ‘slow’} &  &  &  & { >} & { judəə-sa}\\
% &  & {\itshape kjura-} & { ‘beautiful’} &  &  &  & { >} & { kjura-sa}\\
% & { b.} & \multicolumn{6}{X}{{ After consonant-final stem}} & \\
% &  & {\itshape cjuss-} & { ‘strong’} & { +} & {\itshape {}-sa} & { (ADJ)} & { >} & { cjuss-a}\\
% &  & {\itshape kjuugutt-} & { ‘tight’} &  &  &  & { >} & { kjuugutt-a}\\
% &  & {\itshape jiccj-} & { ‘good’} &  &  &  & { >} & { jiccj-a}\\
% &  & {\itshape hoorasj-} & { ‘happy’} &  &  &  & { >} & { hoorasj-a}\\
% \lspbottomrule
% \end{tabularx}
% The above examples show that \textit{{}-sa} (ADJ) has two allomorphs /-sa/ as in (4-49 a) and /-a/ as in (4-49 b). The same thing can apply to \textit{{}-soo} (ADJ), which has two allomorphs /-soo/ and /-oo/.
% 
% Syntactically, a single adjectival word can costitute the predicate as in (4-50 a-b). Additionally, an adjective can be followed by the stative verb \textit{ar-} (or \textit{nə-}) in some environments as in (4-50 c-d) (see §\ref{bkm:Ref357004964} for more details).
% 
% \tablefirsthead{}
% 
% \tabletail{}
% \tablelasttail{}
% \begin{tabularx}{\textwidth}{XXXXXXXXXX}
% \lsptoprule
% { (4\nobreakdash-\stepcounter{Remark}{\theRemark})} & { a.} & { TM:} & { agɨɨ,} & \multicolumn{6}{X}{{ nacɨkasja.}}\\
% &  &  & {\itshape agɨ} & \multicolumn{6}{X}{{\itshape nacɨkasj-sa}}\\
% &  &  & { oh} & \multicolumn{6}{X}{{ familiar-ADJ}}\\
% &  &  & \multicolumn{7}{X}{{ ‘(I) miss them (on the picture).’}}\\
% &  &  & \multicolumn{7}{X}{\raggedleft [Co: 120415\_00.txt]}\\
% & { b.} & { TM:} & \multicolumn{2}{X}{{ agɨɨ!}} & \multicolumn{5}{X}{{ wuganduusoo.}}\\
% &  &  & \multicolumn{2}{X}{{\itshape agɨ}} & \multicolumn{5}{X}{{\itshape wuganduu-soo}}\\
% &  &  & \multicolumn{2}{X}{{ oh}} & \multicolumn{5}{X}{{ not.see.for.a.long.time-ADJ}}\\
% &  &  & \multicolumn{7}{X}{{ ‘Oh! (I) haven’t seen (you) for a long time.’}}\\
% &  &  & \multicolumn{7}{X}{\raggedleft [El: 120912]}\\
% & { c.} & { TM:} & \multicolumn{4}{X}{{ nanga}} & { umoocjattu,} & { jiccja} & { ata.}\\
% &  &  & \multicolumn{4}{X}{{\itshape nan=ga}} & {\itshape umoor-tar-tu} & {\itshape jiccj-sa} & {\itshape ar-tar}\\
% &  &  & \multicolumn{4}{X}{{ 2.HON.SG=NOM}} & { come.HON-PST-CSL} & { good-ADJ} & { STV-PST}\\
% &  &  & \multicolumn{7}{X}{{ ‘Since you has come, (I’m) pleased.’ [lit. ‘Since you came, (it) was good.’]}}\\
% &  &  & \multicolumn{7}{X}{\raggedleft [Co: 110328\_00.tx]}\\
% & { d.} & { TM:} & \multicolumn{3}{X}{{ juwasoo}} & \multicolumn{4}{X}{{ nən?}}\\
% &  &  & \multicolumn{3}{X}{{\itshape juwa-soo}} & \multicolumn{4}{X}{{\itshape nə-an}}\\
% &  &  & \multicolumn{3}{X}{{ hungry-ADJ}} & \multicolumn{4}{X}{{ STV-NEG}}\\
% &  &  & \multicolumn{7}{X}{{ ‘Aren’t (you) hungry?’}}\\
% &  &  & \multicolumn{7}{X}{\raggedleft [El: 120926]}\\
% \lspbottomrule
% \end{tabularx}
% The text data indicates that an adjective takes the inflection \textit{{}-sa} (ADJ) when it is not followed by the stative verb. However, it can take \textit{{}-soo} (ADJ) in elicitation. On the other hand, when it is followed by the stative verb, the adjective takes either \textit{{}-sa} (ADJ) or \textit{{}-soo} (ADJ) in the text data. Generally, \textit{{}-sa} (ADJ) is used when the predicate is in affirmative, and -\textit{soo} (ADJ) in negative. However, \textit{{}-soo} (ADJ) can be used in affirmative when the adjective fills the complement slot of LVC or the lexical verb slot of AVC (see \sectref{sec:key:9.2.2.3} for more details). It is probable that \textit{{}-soo} (ADJ) is made of \textit{{}-sa} (ADJ) + \textit{ja} (TOP), considering the following two facts. First, there is a morphophonological rule of //a// + \textit{ja} (TOP) > /oo/ (see \sectref{sec:key:10.1.1.1}). Second, \textit{{}-soo} (ADJ) is used in negative of the adjectival predicate phrase as well as \textit{ja} (TOP) is used in negative in the nominal predicate phrase (see \sectref{sec:key:9.3.1}). However, I do not propose the underlying forms \textit{{}-sa=ja} (ADJ=TOP) for /-soo/, since there is no surface form realized as /-sa=ja/, and the form /-soo/ can finish a clause, which would not hold true if /-soo/ were composed of \textit{{}-sa} \textit{+} \textit{ja} (TOP).
% 
% Adjectives may also be used adverbially (put in another way, they can convert to adverbs with no formal change).
% 
% \tablefirsthead{}
% 
% \tabletail{}
% \tablelasttail{}
% \begin{tabularx}{\textwidth}{XXXXXXXXXXXXXXXXXXX}
% \lsptoprule
% { (4\nobreakdash-\stepcounter{Remark}{\theRemark})} & \multicolumn{18}{X}{{ Adverbial use of adjectives}}\\
% & { a.} & \multicolumn{17}{X}{{ [Context: Remembering an old scene in the neighborhood]}}\\
% &  & { TM:} & \multicolumn{5}{X}{{ an}} & \multicolumn{6}{X}{{ {\textbar}sutando{\textbar}nu}} & \multicolumn{2}{X}{{ umaga...}} & { aa...} & \multicolumn{2}{X}{{ kansjɨ...}}\\
% &  &  & \multicolumn{5}{X}{{\itshape a-n}} & \multicolumn{6}{X}{{\itshape sutando=nu}} & \multicolumn{2}{X}{{\itshape u-ma=ga}} &  & \multicolumn{2}{X}{{\itshape ka-nsjɨ}}\\
% &  &  & \multicolumn{5}{X}{{ DIST-ADNZ}} & \multicolumn{6}{X}{{ gas.station=GEN}} & \multicolumn{2}{X}{{ MES-place=FOC}} &  & \multicolumn{2}{X}{{ PROX-ADVZ}}\\
% &  &  & \multicolumn{3}{X}{taasa} & \multicolumn{5}{X}{{ isigaki}} & \multicolumn{8}{X}{{ natutattu.}}\\
% &  &  & \multicolumn{3}{X}{{\itshape taa-sa}} & \multicolumn{5}{X}{{\itshape isigaki}} & \multicolumn{8}{X}{{\itshape nar-tur-tar-tu}}\\
% &  &  & \multicolumn{3}{X}{{ high-ADJ}} & \multicolumn{5}{X}{{ stone.fence}} & \multicolumn{8}{X}{{ become-PROG-PST-CSL}}\\
% &  &  & \multicolumn{16}{X}{{ ‘That place, where a gas station is, was stone fence which (was) so high [lit. so highly].’}}\\
% &  &  & \multicolumn{16}{X}{\raggedleft [Co: 120415\_00.txt]}\\
% & { b.} & \multicolumn{17}{X}{{ [Context: Speaking of an acquaintance of TM and MS; MS: ‘(We) have not seen}
% 
% { (him) these days.’]}}\\
% &  & { TM:} & { {\textbar}un{\textbar},} & \multicolumn{4}{X}{{ naa}} & \multicolumn{4}{X}{nagəəsa} & \multicolumn{7}{X}{{ mjandoojaa.}}\\
% &  &  & {\itshape un} & \multicolumn{4}{X}{{\itshape naa}} & \multicolumn{4}{X}{{\itshape nagəə-sa}} & \multicolumn{7}{X}{{\itshape mj-an=doo=jaa}}\\
% &  &  & { yeah} & \multicolumn{4}{X}{{ yet}} & \multicolumn{4}{X}{{ long-ADJ}} & \multicolumn{7}{X}{{ see-NEG=ASS=SOL}}\\
% &  &  & \multicolumn{16}{X}{{ ‘Yeah, (we) have not seen (him) for a long time.’}}\\
% &  &  & \multicolumn{16}{X}{\raggedleft [Co: 120415\_01.txt]}\\
% & { c.} & { TM:} & \multicolumn{16}{X}{{ [Context: Speaking about an aquaintance]}}\\
% &  &  & \multicolumn{4}{X}{{ nasjeba}} & \multicolumn{3}{X}{{ izjɨ}} & \multicolumn{5}{X}{{ cˀjəəroo,}} & \multicolumn{3}{X}{{ akka}} & { taməə}\\
% &  &  & \multicolumn{4}{X}{{\itshape nasje=ba}} & \multicolumn{3}{X}{{\itshape ik-tɨ}} & \multicolumn{5}{X}{{\itshape k-təəra=ja}} & \multicolumn{3}{X}{{\itshape a-rɨ =ga}} & {\itshape taməə}\\
% &  &  & \multicolumn{4}{X}{{ Naze=ACC}} & \multicolumn{3}{X}{{ go-SEQ}} & \multicolumn{5}{X}{{ come-after=TOP}} & \multicolumn{3}{X}{{ DIST-NLZ=GEN}} & { sake}\\
% &  &  & \multicolumn{2}{X}{{ naa}} & \multicolumn{4}{X}{{ issai}} & \multicolumn{4}{X}{{ warusoo}} & \multicolumn{6}{X}{{ jˀantatto.}}\\
% &  &  & \multicolumn{2}{X}{{\itshape naa}} & \multicolumn{4}{X}{{\itshape issai}} & \multicolumn{4}{X}{{\itshape waru-soo}} & \multicolumn{6}{X}{{\itshape jˀ-an-tar-too}}\\
% &  &  & \multicolumn{2}{X}{{ already}} & \multicolumn{4}{X}{{ all}} & \multicolumn{4}{X}{{ bad-ADJ}} & \multicolumn{6}{X}{{ say-NEG-PST-CSL}}\\
% &  &  & \multicolumn{16}{X}{‘After going to and returning from Naze, (she) did not say anything bad [lit. badly] for him.’}\\
% &  &  & \multicolumn{16}{X}{\raggedleft [Co: 101023\_01.txt]}\\
% \lspbottomrule
% \end{tabularx}
% In (4-51 a), the predicate and its complement /isigaki natutattu/ ‘was stone fence’ are modified by \textit{taa-sa} (high-ADJ) ‘highly.’ In (4-51 b), the predicate /mjan/ ‘not see’ is modified by \textit{nagəə-sa} (long-ADJ) ‘for a long time.’ In (4-51 c), the predicate /jˀantatto/ ‘did not say’ is modified by /waru-soo/ (bad-ADJ) ‘badly.’
% 
%   There are very limited set of adjectives that take the adverbializer -\textit{sanma} or \textit{{}-ku}. And another limited set of adjectives undergo reduplication (sometimes with the affix \textit{{}-tu}), in order to make them adverbs (see §\ref{bkm:Ref347244590} and \sectref{sec:key:4.3.8.3}). Thus, we interpret them as derivational affixes and call them adverbializers.
% 
% \subsection{Particles}
% \hypertarget{RefHeadingToc395697014}{}
% All particles are clitics, but not vice versa (cf., formal nouns in §\ref{bkm:Ref367137128}). There are six subclasses of particles: case particles, limitter particles, conjunctive particles, clause-final particles, utterance-final particles A, and utterance-final particles B. See Chapter 10 for more details.
% 
% \subsection{Adverbs}
% \label{bkm:Ref367365251}\hypertarget{RefHeadingToc395697015}{}\label{bkm:Ref347244590}\label{bkm:Ref347246311}\label{bkm:Ref356042174}\label{bkm:Ref356042913}
% It is difficult to difine the formal categories with which adverbs establish the modificational relationships. They scope over entire propostion, predicate, or even a part of compound. Let us illustrate the adverbial modification with \textit{muru} ‘very,’ which is \EmphName below.
% 
% \tablefirsthead{}
% 
% \tabletail{}
% \tablelasttail{}
% \begin{tabularx}{\textwidth}{XXXXXXXXXX}
% \lsptoprule
% { (4\nobreakdash-\stepcounter{Remark}{\theRemark})} & \multicolumn{9}{X}{{ a. With verbal predicate}}\\
% & \multicolumn{9}{X}{{ [Context: Speaking about an acquaintance of TM and US]}}\\
% & { TM:} & \multicolumn{2}{X}{{ masahiko}} & \multicolumn{2}{X}{{ tuzija}} & { muru} & \multicolumn{3}{X}{{ sijan.}}\\
% &  & \multicolumn{2}{X}{{\itshape masahiko}} & \multicolumn{2}{X}{{\itshape tuzi=ja}} & {\itshape muru} & \multicolumn{3}{X}{{ [\textit{sij-an}]\textsubscript{Verbal predicate}}}\\
% &  & \multicolumn{2}{X}{{ Masahiko}} & \multicolumn{2}{X}{{ wife=TOP}} & { very} & \multicolumn{3}{X}{{ know-NEG}}\\
% &  & \multicolumn{8}{X}{{ ‘(I) don’t know Masahiko’s wife at all.’}}\\
% &  & \multicolumn{8}{X}{\raggedleft [Co: 110328\_00.txt]}\\
% & \multicolumn{9}{X}{{ b. With adjectival predicate}}\\
% & \multicolumn{9}{X}{{ [Context: Speaking about MS’s grandfather and his friends, who traded market stocks]}}\\
% & { TM:} & { muru} & \multicolumn{5}{X}{{ dujasanu,}} & { ikizɨmai} & { jatəəkkamojaa.}\\
% &  & {\itshape muru} & \multicolumn{5}{X}{{ [\textit{duja-sa}]\textsubscript{Adjectival predicate}\textit{=nu}}} & {\itshape ikizɨmai} & {\itshape jar-təər=kamo=jaa}\\
% &  & { very} & \multicolumn{5}{X}{{ rich-ADJ=SEQ}} & { extreme} & { COP-RSL=maybe=SOL}\\
% &  & \multicolumn{8}{X}{{ ‘(Maybe, they) were very rich, and (their life was) extremely (good).’}}\\
% &  & \multicolumn{8}{X}{\raggedleft [Co: 120415\_01.txt]}\\
% & \multicolumn{9}{X}{{ c. With nominal predicate}}\\
% & \multicolumn{9}{X}{{ [Context: Speaking about acquaintances of TM and MS; TM: ‘Muha is as old as those people, and...’]}}\\
% & { TM:} & { muru} & \multicolumn{2}{X}{{ dusi}} & \multicolumn{5}{X}{{ jata.}}\\
% &  & {\itshape muru} & \multicolumn{2}{X}{{ [\textit{dusi}}} & \multicolumn{5}{X}{{ \textit{jar-tar}]\textsubscript{Nominal predicate}}}\\
% &  & { very} & \multicolumn{2}{X}{{ friend}} & \multicolumn{5}{X}{{ COP-PST}}\\
% &  & \multicolumn{8}{X}{{ ‘(They) were very (good) friends.’}}\\
% &  & \multicolumn{8}{X}{\raggedleft [Co: 120415\_00.txt]}\\
% \lspbottomrule
% \end{tabularx}
% In the above examples, the adverb \textit{muru} ‘very’ occurs with the verbal predicate \textit{sij-an} (know-NEG) ‘don’t know’ in (4-52 a), the adjectival predicate \textit{duja-sa} (rich-ADJ) ‘(be) rich’ in (4-52 b), and the nominal predicate \textit{dusi} \textit{jar-tar} (friend COP-PST) ‘were friends’ in (4-52 c).
% 
%   Adverbs can be grouped into two groups: non-derived adverbs and derived ones. First, non-derived adverbs are all monomorphemic, e.g., \textit{atadan} ‘suddenly’ in (4-53).
% 
% \tablefirsthead{}
% 
% \tabletail{}
% \tablelasttail{}
% \begin{tabularx}{\textwidth}{XXXXXX}
% \lsptoprule
% { (4\nobreakdash-\stepcounter{Remark}{\theRemark})} & \multicolumn{5}{X}{{ [Context: Speaking about the present author; TM: ‘Then, I thought (he) already went back (home).’]}}\\
% & { TM:} & { sjatto,} & { kinjuu} & { atadan} & { umoocjɨ.}\\
% &  & {\itshape sɨr-tar-too} & {\itshape kinjuu} & {\itshape atadan} & {\itshape umoor-tɨ}\\
% &  & { do-PST-CND} & { yesterday} & { suddenly} & { come.HON-SEQ}\\
% &  & \multicolumn{4}{X}{{ ‘Then, suddenly (he) came (here) yesterday.’}}\\
% & \multicolumn{5}{X}{\raggedleft [Co: 110328\_00.txt]}\\
% \lspbottomrule
% \end{tabularx}
% Other non-derived adverbs are shown in the table below.
% 
% \begin{table}
% \hypertarget{RefHeadingToc395697292}{}\textmd{Table \stepcounter{Table}{\theTable}.} Non-derived adverbs
% \end{table}
% 
% \tablefirsthead{}
% 
% \tabletail{}
% \tablelasttail{}
% \begin{tabularx}{\textwidth}{XXXXX}
% \lsptoprule
% 
% { Form} & { Meaning} &  & { Form} & { Meaning}\\
% {\itshape abɨnəə} & { ‘barely’} &  & {\itshape jiikunma} & { ‘throughout’}\\
% {\itshape anmai} & { ‘not very much’} &  & {\itshape joikwa} & { ‘silently’}\\
% {\itshape atadan} & { ‘suddenly’} &  & {\itshape jukkadɨ} & { ‘continuously; always’}\\
% {\itshape cˀja} & { ‘without rest’} &  & {\itshape kattəə/kattənnən} & { ‘freely’}\\
% {\itshape cˀjakɨɨ} & { ‘soon’} &  & {\itshape kundoo} & { ‘next time’}\\
% {\itshape cˀjasuguu} & { ‘soon’} &  & {\itshape kunuguru} & { ‘recently’}\\
% {\itshape cjoo} & { ‘just’} &  & {\itshape mata} & { ‘again’}\\
% {\itshape dooka} & { ‘please’} &  & {\itshape minna} & { ‘everyone’}\\
% {\itshape doosje} & { ‘maybe’} &  & {\itshape muru/muruttu} & { ‘very’}\\
% {\itshape ganba} & { ‘therefore’} &  & {\itshape naa} & { ‘already; yet’}\\
% {\itshape ganboo} & { ‘if so’} &  & {\itshape naakɨssa} & { ‘so early’}\\
% {\itshape jappai} & { ‘after all’} &  & {\itshape nama} & { ‘now; still’}\\
% {\itshape jəito} & { ‘well; much’} &  & {\itshape saki} & { ‘first (of all)’}\\
% {\itshape jiccjan} & { ‘well’} &  & {\itshape sjəəroo} & { ‘then’}\\
% {\itshape jɨɨ} & { ‘often, well’} &  & {\itshape wadaatunma} & { ‘deliberately’}\\
% {\itshape jiicjan} & { ‘throughout’} &  & {\itshape zjenzjen} & { ‘(not) at all’}\\
% \lspbottomrule
% \end{tabularx}
% This table shows that \textit{ganba} ‘therefore’ and \textit{ganboo} ‘if so’ appear to be divided into demonstrative roots and affixes, i.e. \textit{ga-nba} and \textit{ga-nboo} (cf. §\ref{bkm:Ref347232488}); however, the demonstrative roots other than \textit{ga-} (MES) do not precede /nba/ or /nboo/, i.e. *\textit{ka-nba} or *\textit{aga-nba}, where \textit{ka-} (PROX) and \textit{aga-} (DIST) are demonstrative roots. Thus, we regard \textit{ganba} ‘therefore’ and \textit{ganboo} ‘if so’ as monomorphemic adverbs.
% 
% Second, some adverbs can be derived from reduplication such as \textit{buu+buu} ‘floating’ in (4-54 a) or /sabiisabi/ \textit{sabi+sabi} ‘smoothly’ in (4-54 b).
% 
% \tablefirsthead{}
% 
% \tabletail{}
% \tablelasttail{}
% \begin{tabularx}{\textwidth}{XXXXXXXXXX}
% \lsptoprule
% { (4\nobreakdash-\stepcounter{Remark}{\theRemark})} & { a.} & \multicolumn{8}{X}{{ [Context: Remembering the sight around the kitchen in the old days]}}\\
% &  & { TM:} & { haija} & \multicolumn{3}{X}{{ buubuu}} & \multicolumn{3}{X}{{ tubjakudɨ,}}\\
% &  &  & {\itshape hai=ja} & \multicolumn{3}{X}{{\itshape buu+buu}} & \multicolumn{3}{X}{{\itshape tubjakum-tɨ}}\\
% &  &  & { ash=TOP} & \multicolumn{3}{X}{{ RED+floating}} & \multicolumn{3}{X}{{ fly-SEQ}}\\
% &  &  & \multicolumn{7}{X}{{ ‘Ashes floated, and ...’}}\\
% &  & \multicolumn{8}{X}{\raggedleft [Co: 111113\_02.txt]}\\
% & { b.} & \multicolumn{8}{X}{{ [Context: At the lunch time]}}\\
% &  & { TM:} & \multicolumn{3}{X}{{ sabiisabi}} & \multicolumn{2}{X}{{ aikikippoo,}} & { cɨkɨmununkja} & \\
% &  &  & \multicolumn{3}{X}{{\itshape sabi+sabi}} & \multicolumn{2}{X}{{ \textit{aik-i+kij-boo}}} & {\itshape cɨkɨmun=nkja} & \\
% &  &  & \multicolumn{3}{X}{{ RED+smoothly}} & \multicolumn{2}{X}{{ walk-INF+CAP-CND}} & { pickle=APPR} & \\
% &  &  & \multicolumn{2}{X}{{ jaazjɨ}} & \multicolumn{5}{X}{{ tɨkkoorɨnmun.}}\\
% &  &  & \multicolumn{2}{X}{{\itshape jaa=zjɨ}} & \multicolumn{5}{X}{{\itshape tɨkk-arɨr-n=mun}}\\
% &  &  & \multicolumn{2}{X}{{ house=LOC3}} & \multicolumn{5}{X}{{ bring-CAP-PTCP=ADVRS}}\\
% &  &  & \multicolumn{7}{X}{{ ‘If (I) could walk smoothly, (I) could go home and bring some pickles, but (couldn’t).’}}\\
% &  & \multicolumn{8}{X}{\raggedleft [Co: 120415\_01.txt]}\\
% \lspbottomrule
% \end{tabularx}
% Other examples of reduplicated adverbs are shown in the table below.
% 
% \begin{table}
% \hypertarget{RefHeadingToc395697293}{}\textmd{Table \stepcounter{Table}{\theTable}}. Fully reduplicated adverbs (lengthened root being \EmphName)
% \end{table}
% 
% \tablefirsthead{}
% 
% \tabletail{}
% \tablelasttail{}
% \begin{tabularx}{\textwidth}{XXXXXX}
% \lsptoprule
% 
% { Original root} & { Penultimate syllable} & { Final syllable} &  & { Reduplicated adverb} & { Meaning}\\
% & \multicolumn{2}{X}{{ (H: heavy; L: light; -: no syllable)}} &  &  & \\
% { bocu} & { L} & { L} & { >} & { bocuu+bocu} & { ‘step by step’}\\
% { botto} & { H} & { L} & { >} & { botto+botto} & { ‘lazily’}\\
% { buu} & { {}-} & { H} & { >} & { buu+buu} & { ‘floating’}\\
% { gara} & { L} & { L} & { >} & { garaa+gara} & { ‘rattle’}\\
% { hui} & { {}-} & { H} & { >} & { hui+hui} & { ‘lightly’}\\
% { joi} & { {}-} & { H} & { >} & { joi+joi} & { ‘slowly; late’}\\
% { kjura} & { L} & { L} & { >} & { kjuraa+gjura} & { ‘beautifully’}\\
% { kˀumja} & { L} & { L} & { >} & { kˀumjaa+kˀumja} & { ‘with steps’}\\
% { muccjara} & { L} & { L} & { >} & { muccjaraa+muccjara} & { ‘chewing’}\\
% { potton} & { H} & { H} & { >} & { potton+potton} & { ‘dripping’}\\
% { sa} & { {}-} & { L} & { >} & { saa+sa} & { ‘without hesitation’}\\
% { sai} & { {}-} & { H} & { >} & { sai+sai} & { ‘fast’}\\
% { sabi} & { L} & { L} & { >} & { sabii+sabi} & { ‘smoothly’}\\
% { siju} & { L} & { L} & { >} & { sijuu+ziju} & { ‘whitely’}\\
% \lspbottomrule
% \end{tabularx}
% There are two points to make about the data shown in the above table: (a) syllable construction and (b) kinds of roots. First, some of the reduplicated adverbs lengthen their initial roots, e.g., //sabi// ‘smoothly’ > /sabii/. This lengthening occurs if neither penultimate nor final syllable of the original root is heavy. Second, reduplicated adverbs are made up of either onomatopoeic roots such as //gara// ‘rattle,’ which seems to represent the sound of metallic objects hitting each other, or adjectival roots such as //kjura// ‘beautiful’ and //siju// ‘white’ (which also go through sequential voicing, as discussed in §\ref{bkm:Ref347181344}). Logically, it would be difficult to characterize whether the initial root undergoes lengthening or omitting (of a vowel) seeing only cases of onomatopoeic roots. Although, the adjectival roots provide additional clues because their original forms are clearly not lengthened when compared to the other morphological processes of adjectival roots, e.g., /kjura-sa/ (white-ADJ) ‘white.’ Therefore, we can assume that all the initial roots of reduplicated adverbs originally did not undergo lengthening. In other words, the original root of /sabii+sabi/ ‘smoothly’ is //sabi// (not //sabii//).
% 
%   Furthermore, adjectival stems, demonstrative stems and interrogative stems can become adverbs by affixation, e.g., \textit{ubu-ku} (heavy-ADVZ) ‘heavily,’ \textit{ka-n} (PROX-ADVZ) ‘here’ and \textit{ikja-sjɨ} (how-ADVZ) ‘how’ (see \sectref{sec:key:4.3.8.3} and chapter 5).
% 
% Before concluding this section, I want to mention two affixes that can turn the interrogative stems into indefinite adverbs: \textit{{}-ninkuinin} and \textit{{}-sjɨnkaasjɨn}. The former, \textit{{}-ninkuinin,} follows only \textit{ta-ru} (who-NLZ) ‘who,’ and the latter, \textit{{}-sjɨnkaasjɨn,} follows only \textit{ikja-} ‘how’ (see §\ref{bkm:Ref356212678} for more details about interrogative words). The examples of these affixes are presented below.
% 
% \tablefirsthead{}
% 
% \tabletail{}
% \tablelasttail{}
% \begin{tabularx}{\textwidth}{XXXXXXXXXXXX}
% \lsptoprule
% { (4\nobreakdash-\stepcounter{Remark}{\theRemark})} & { a.} & \multicolumn{10}{X}{{\itshape {}-ninkuinin}}\\
% &  & \multicolumn{10}{X}{{ [Context: Remembering the work of thatching a roof]}}\\
% &  & { TM:} & \multicolumn{2}{X}{{ waakjoo...}} & { naa,} & \multicolumn{5}{X}{{ taruuninkuinin}} & { gajaurusi}\\
% &  &  & \multicolumn{2}{X}{{\itshape waa-kja=ja}} & {\itshape naa} & \multicolumn{5}{X}{{\itshape ta-ru-ninkuinin}} & {\itshape gaja+urus-i}\\
% &  &  & \multicolumn{2}{X}{{ 1-PL=TOP}} & { FIL} & \multicolumn{5}{X}{{ who-NLZ-INDFZ}} & { miscanthus+lower-INF}\\
% &  &  & \multicolumn{4}{X}{{ tanmarɨccjɨ}} & \multicolumn{3}{X}{{ jˀii}} & \multicolumn{2}{X}{{ natɨ, ...}}\\
% &  &  & \multicolumn{4}{X}{{\itshape tanm-ar-ɨ=ccjɨ}} & \multicolumn{3}{X}{{\itshape jˀ-i}} & \multicolumn{2}{X}{{\itshape nar-tɨ}}\\
% &  &  & \multicolumn{4}{X}{{ ask-PASS-IMP=QT}} & \multicolumn{3}{X}{{ say-INF}} & \multicolumn{2}{X}{{ COP-SEQ}}\\
% &  &  & \multicolumn{9}{X}{{ ‘Everyone said that, “Please undertake the carrying of [lit. Be asked to carry] the miscanthus (from the mountains)” Thus, I ...’}}\\
% &  &  & \multicolumn{9}{X}{\raggedleft [Co: 110328\_00.txt]}\\
% & { b.} & \multicolumn{10}{X}{{\itshape {}-sjɨnkaasjɨn}}\\
% &  & \multicolumn{10}{X}{{ [Context: Speaking about play in the old days; TM: ‘Didn’t you play hitting balls?’]}}\\
% &  & { US:} & { cjaa,} & \multicolumn{4}{X}{{ cjaa,}} & { naa,} & \multicolumn{3}{X}{{ ikjaasjɨnkaasjɨn.jo.}}\\
% &  &  & {\itshape cjaa} & \multicolumn{4}{X}{{\itshape cjaa}} & {\itshape naa} & \multicolumn{3}{X}{{ \textit{ikja-sjɨnkaasjɨn=joo}}}\\
% &  &  & { I.think.so} & \multicolumn{4}{X}{{ I.think.so}} & { FIL} & \multicolumn{3}{X}{{ how-INDFZ=CFM1}}\\
% &  &  & \multicolumn{9}{X}{{ ‘Yeah, yeah, (I played a game) no matter how (it is).’}}\\
% &  &  & \multicolumn{9}{X}{\raggedleft [Co: 110328\_00.txt]}\\
% \lspbottomrule
% \end{tabularx}
% These examples show that the second vowels of the interrogative stems should be lengthened before \textit{{}-ninkuinin} or \textit{{}-sjɨnkaasjɨn}: \textit{ta-ru} (who-NLZ) > /taruu/ and \textit{ikja-} ‘how’ > /ikjaa/. Perhaps, these affixes may be divided into several morphemes such as \textit{{}-ninkuinin} > \textit{=n=n} \textit{kui=n=n} (DAT1=even ECHO=DAT1=even) and \textit{{}-sjɨnkaasjɨn} > \textit{{}-sjɨ=n kaa-sjɨ=n} (ADVZ=even ECHO=ADVZ=even) (ECHO means an echo morpheme). I do not, however, take these analyses, because these morphemes are always closely united and no other morphemes intervene or replace them. Therefore, I interpret these alleged combinations as affixes, at least in modern Yuwan (see also §\ref{bkm:Ref367143341} for the indefinite pronoun).
% 
% \subsection{Interjections}
% \hypertarget{RefHeadingToc395697016}{}\begin{styleSketch}
% The interjection cannot directly modify a predicate.
% \end{styleSketch}
% 
% \tablefirsthead{}
% 
% \tabletail{}
% \tablelasttail{}
% \begin{tabularx}{\textwidth}{XXXXXX}
% \lsptoprule
% { (4\nobreakdash-\stepcounter{Remark}{\theRemark})} & \multicolumn{5}{X}{{ [Context: Both TM and the hearer MS were trying to remember a}
% 
% { person’s name, and MS said the name of a candidate to TM.]}}\\
% & { TM:} & { agɨ.} & { cjaa} & { zjaga.} & \\
% &  & {\itshape agɨ} & {\itshape cjaa} & {\itshape zjar=ga} & \\
% &  & { oh} & { that.is.right} & { COP=CFM3} & \\
% &  & \multicolumn{4}{X}{{ ‘Oh! That’s right.’}}\\
% & \multicolumn{5}{X}{\raggedleft [Co: 120415\_00.txt]}\\
% \lspbottomrule
% \end{tabularx}
% In the above example, the interjection \textit{agɨ} expresses the speaker’s surprise, and it does not directly modify the predicate. Other examples are shown below.
% 
% \begin{table}
% \hypertarget{RefHeadingToc395697294}{}\textmd{Table \stepcounter{Table}{\theTable}.} Interjections
% \end{table}
% 
% \tablefirsthead{}
% 
% \tabletail{}
% \tablelasttail{}
% \begin{tabularx}{\textwidth}{XXX}
% \lsptoprule
% 
% Form & Gloss & Context\\
% {\itshape agɨ} & oh & Being surprised\\
% {\itshape ai} & no & Giving a negative response\\
% {\itshape baa} & not.want & Expressing reluctance\\
% {\itshape cjaa} & that.is.right & Agreeing with the hearer\\
% {\itshape dɨɨ} & hey & Calling the hearer\\
% {\itshape hagɨɨ} & oh & Being impressed\\
% {\itshape ido} & oh & Drawing the hearer’s attention\\
% {\itshape in} & yes & Giving an affirmative response\\
% {\itshape ɨɨ} & yes & Giving an affirmative response\\
% {\itshape jaa} & SOL & Requiring empathy (or expressing the speaker’s empathy)\\
% {\itshape joo} & CFM1 & Drawing hearer’s attention\\
% {\itshape mattai} & wait.IMP.POL & Asking the hearer to wait\\
% {\itshape naa} & FIL & Filling the interval of utterance\\
% { \textit{ude}\footnotemark{}} & well & Trying to do something\\
% { \textit{un}\footnotemark{}} & BCH & Backchannel\\
% \lspbottomrule
% \end{tabularx}
% \addtocounter{footnote}{-2}
% \stepcounter{footnote}\footnotetext{ \textit{ude} ‘well’ is frequently pronounced as [uɾe].}
% \stepcounter{footnote}\footnotetext{ \textit{un} (BCH) is frequently pronounced as [ʔmː].}
% 
%   Almost all of the morphemes regarded as interjections by the criteria discussed in §\ref{bkm:Ref306382551} are used in the following conditions: they are used only by themselves, or they are embedded into a clause in the direct speech, which is always followed by the quotative marker \textit{ccjɨ} (see also §\ref{bkm:Ref366579906}).
% 
% \tablefirsthead{}
% 
% \tabletail{}
% \tablelasttail{}
% \begin{tabularx}{\textwidth}{XXXXXXXXXXX}
% \lsptoprule
% { (4\nobreakdash-\stepcounter{Remark}{\theRemark})} & \multicolumn{10}{X}{{ [Context: Distributing some of her lunch to the present author’s plate; TM: ‘Old peoples...’; MS: ‘Yeah.’]}}\\
% & { TM:} & { ude,} & { naa,} & \multicolumn{2}{X}{{ ganboo,}} & \multicolumn{2}{X}{{ urakjoo}} & { ude,} & { ude,} & { kamanboo,}\\
% &  & {\itshape ude} & {\itshape naa} & \multicolumn{2}{X}{{\itshape ganboo}} & \multicolumn{2}{X}{{\itshape urakja=ja}} & {\itshape ude} & {\itshape ude} & {\itshape kam-an-boo}\\
% &  & { well} & { FIL} & \multicolumn{2}{X}{{ if.so}} & \multicolumn{2}{X}{{ 2.NHON.SG=TOP}} & { well} & { well} & { eat-NEG-CND}\\
% &  & \multicolumn{3}{X}{{ udeccjɨdu}} & \multicolumn{2}{X}{{ xxx}} & \multicolumn{4}{X}{{ jˀutattujaa.}}\\
% &  & \multicolumn{3}{X}{{ \textit{ude=ccjɨ=du}}} & \multicolumn{2}{X}{} & \multicolumn{4}{X}{{\itshape jˀ-jur-tar-tu=jaa}}\\
% &  & \multicolumn{3}{X}{{ well=QT=FOC}} & \multicolumn{2}{X}{} & \multicolumn{4}{X}{{ say-UMRK-PST-CSL=SOL}}\\
% &  & \multicolumn{9}{X}{{ ‘(The old people) would say, ‘Well, now, then, you have to eat (more).’’}}\\
% & \multicolumn{10}{X}{\raggedleft [Co: 120415\_01.txt]}\\
% \lspbottomrule
% \end{tabularx}
% All of the occurrences of \textit{ude} ‘well’ in (4-57) are integrated in the main clause as direct speech, which is followed by \textit{ccjɨ} (QT).
% 
% There are, however, morphemes that can be integrated into a clause without \textit{ccjɨ} (QT) despite being classified into interjections according to the criteria presented in §\ref{bkm:Ref306382551}, e.g., \textit{cjaa} ‘I think so!’ and \textit{baa} ‘No!’
% 
% First, \textit{cjaa} ‘I think so!’ is a free form and can be uttered only by itself. However, it can also fill the predicate slot followed by the copula verb as in (4-56). \textit{cjaa} behaves similarly to the nominal in this case. However, it cannot take any case particle. Thus, we assume it as a special kind of interjection.  
% 
%   Second, I will show an example of \textit{baa} ‘No!’
% 
% \tablefirsthead{}
% 
% \tabletail{}
% \tablelasttail{}
% \begin{tabularx}{\textwidth}{XXXX}
% \lsptoprule
% { (4\nobreakdash-\stepcounter{Remark}{\theRemark})} & { TM:} & { kurɨsjəə} & { baadoo.}\\
% &  & {\itshape ku-rɨ=sjɨ=ja} & {\itshape baa=doo}\\
% &  & { PROX-NLZ=INST=TOP} & { not.want=ASS}\\
% &  & \multicolumn{2}{X}{{ ‘(If it is) so, (it) does not (work).’}}\\
% & \multicolumn{3}{X}{\raggedleft [El: 110827]}\\
% \lspbottomrule
% \end{tabularx}
% In this example, \textit{baa} fills the predicate slot followed by \textit{doo} (ASS); however, \textit{baa} cannot fall into nominals (since it cannot take any case or copula verb) or verbs (since it cannot take any verbal affix). Thus, we interpret \textit{baa} as a special kind of interjection.
% 
% \subsection{Class-changing derivation}
% \hypertarget{RefHeadingToc395697017}{}
% We attach the same label to a free form and a stem only if the stem can become the word class by itself or with a minimal inflection (cf., \citealt{Lehman2008}: 8). For example, the stem \textit{isi} ‘stone’ can be a nominal word by itself, and so we label \textit{isi} ‘stone’ as a “nominal stem.” The stem \textit{kam-} ‘eat’ can be a verbal word with a minimal inflection \textit{{}-ɨ} (IMP) as in \textit{kam-ɨ} ‘Eat!,’ and so we regard \textit{kam-} ‘eat’ as a “verbal stem.”
% 
%   In the following sections, we examine a few cases where a particular stem class becomes another stem class. For example, a verbal stem becomes a nominal stem (see §\ref{bkm:Ref347245173}), a verbal stem becomes an adjectival stem (see §\ref{bkm:Ref347179613}), and an adjectival stem becomes an adverbial stem (see §\ref{bkm:Ref347227985}).
% 
% \subsubsection{Verbal stem to nominal stem}
% \label{bkm:Ref347245173}\hypertarget{RefHeadingToc395697018}{}
% There are several morphemes that can change verbal stems to nominal stems: \textit{{}-jaa} ‘person,’ \textit{zjaa} ‘place,’ \textit{bəə} ‘role,’ \textit{mai} (OBL), \textit{madəə} ‘fail to,’ and \textit{gjaa} (PURP). The first one may be called nominalizer (see \sectref{sec:key:7.6}). The others are a kind of nominal roots that are compounded with verbal infinitives (see \sectref{sec:key:4.2.3.2} for more details). The affix-like clitic \textit{sɨ} (FN) can also form a nominal stem from a verbal stem (see §\ref{bkm:Ref367143723}).
% 
% \subsubsection{Verbal stem to adjectival stem}
% \label{bkm:Ref347179613}\hypertarget{RefHeadingToc395697019}{}
% There are four adjectival roots that can change verbal stems to adjectival stems: \textit{cja} ‘want,’ \textit{cjagɨ} ‘seem,’ \textit{jass} ‘easy,’ and \textit{gussj} ‘difficult.’ In principle, they are compouned with verbal infinitives.
% 
% \tablefirsthead{}
% 
% \tabletail{}
% \tablelasttail{}
% \begin{tabularx}{\textwidth}{XXXXXXXXXXXXX}
% \lsptoprule
% { (4\nobreakdash-\stepcounter{Remark}{\theRemark})} & { a.} & \multicolumn{11}{X}{{ \textit{cja} ‘want’ [= (4-36)]}}\\
% &  & \multicolumn{11}{X}{{ [Context: TM is introducing the present author to the hearer U saying that the present}
% 
% { author has been looking for a good language teacher in the community.]}}\\
% &  & { TM:} & \multicolumn{4}{X}{{ simakutuba} } & \multicolumn{5}{X}{{ narəəcjasaccjɨ}} & { jˀicjɨ,}\\
% &  &  & \multicolumn{4}{X}{{\itshape sima+kutuba}} & \multicolumn{5}{X}{{ \textit{naraw-i+cja-sa=ccjɨ}}} & {\itshape jˀ-tɨ}\\
% &  &  & \multicolumn{4}{X}{{ community+language}} & \multicolumn{5}{X}{{ learn-INF+want-ADJ=QT}} & { say-SEQ}\\
% &  &  & \multicolumn{10}{X}{{ ‘(He) said, ‘(I) want to learn the language of the community,’ and ...’}}\\
% &  &  & \multicolumn{10}{X}{\raggedleft [Co: 110328\_00.txt]}\\
% & \raggedleft b. & \multicolumn{11}{X}{{ \textit{cjagɨ} ‘seem’}}\\
% &  & \multicolumn{11}{X}{{ [Context: Speaking of a person who used to copy the music tapes for everyone]}}\\
% &  & { TM:} & { arɨ} & \multicolumn{6}{X}{{ siicjagɨsan}} & \multicolumn{3}{X}{{ cˀjunkjaga}}\\
% &  &  & {\itshape a-rɨ} & \multicolumn{6}{X}{{ \textit{sɨr-i+cjagɨ-sa+ar-n}}} & \multicolumn{3}{X}{{\itshape cˀju=nkja=ga}}\\
% &  &  & { DIST-NLZ} & \multicolumn{6}{X}{{ do-INF+seem-ADJ+STV-ADN}} & \multicolumn{3}{X}{{ person=APPR=FOC}}\\
% &  &  & \multicolumn{4}{X}{{ cˀjuin}} & \multicolumn{2}{X}{{ umooran}} & \multicolumn{4}{X}{{ natattujaa.}}\\
% &  &  & \multicolumn{4}{X}{{\itshape cˀjui=n}} & \multicolumn{2}{X}{{\itshape umoor-an}} & \multicolumn{4}{X}{{\itshape nar-tar-tu=jaa}}\\
% &  &  & \multicolumn{4}{X}{{ one.NUM.person=also}} & \multicolumn{2}{X}{{ exist.HON-NEG}} & \multicolumn{4}{X}{{ become-PST-CSL=SOL}}\\
% &  &  & \multicolumn{10}{X}{{ ‘(Now) there are no people who are likely to do that (i.e. recording), you know.’}}\\
% &  &  & \multicolumn{10}{X}{\raggedleft [Co: 120415\_01.txt]}\\
% & { c.} & \multicolumn{11}{X}{{ \textit{{}-jass} ‘easy’}}\\
% &  & \multicolumn{11}{X}{{ [Context: Speaking of pickles that are easy to make]}}\\
% &  & { TM:} & \multicolumn{2}{X}{{ urɨga}} & \multicolumn{3}{X}{{ {\textbar}iciban{\textbar}}} & \multicolumn{3}{X}{{ siijassa}} & \multicolumn{2}{X}{{ appa.}}\\
% &  &  & \multicolumn{2}{X}{{\itshape u-rɨ=ga}} & \multicolumn{3}{X}{{\itshape iciban}} & \multicolumn{3}{X}{{\itshape sɨr-i+jass-sa}} & \multicolumn{2}{X}{{\itshape ar-ba}}\\
% &  &  & \multicolumn{2}{X}{{ MES-NLZ=FOC}} & \multicolumn{3}{X}{{ mostly}} & \multicolumn{3}{X}{{ do-INF+easy-ADJ}} & \multicolumn{2}{X}{{ STV-CSL}}\\
% &  &  & \multicolumn{10}{X}{{ ‘Since it (i.e. the pickles) is mostly easy to do.’}}\\
% &  &  & \multicolumn{10}{X}{\raggedleft [Co: 101023\_01.txt]}\\
% & \raggedleft d. & \multicolumn{11}{X}{{ \textit{{}-gussj} ‘difficult’}}\\
% &  & { TM:} & \multicolumn{3}{X}{{ mɨsikjarusanu}} & \multicolumn{7}{X}{{ miigussja.}}\\
% &  &  & \multicolumn{3}{X}{{ \textit{mɨsikjaru-sa=nu}}} & \multicolumn{7}{X}{{\itshape mj-i+gussj-sa}}\\
% &  &  & \multicolumn{3}{X}{{ dazzling-ADJSEQ}} & \multicolumn{7}{X}{{ see-INF+difficult-ADJ}}\\
% &  &  & \multicolumn{10}{X}{{ ‘(It) is dazzling and (it) is difficult (for me) to see.’}}\\
% &  &  & \multicolumn{10}{X}{\raggedleft [El: 120921]}\\
% \lspbottomrule
% \end{tabularx}
% All of the above examples are followed by \textit{{}-sa} (ADJ) and become adjectives to fill the predicate slots. The above adjectival stems almost always follow the verbal infinitives. However, there is an example, where \textit{cjagɨ} ‘seem’ is compounded with the adjectival stem \textit{mˀa} ‘tasty’ as in \textit{mˀa+cjagɨ-sa} (tasty+seem-ADJ) ‘(It) seems tasty.’
% 
% \subsubsection{Adjectival stem to adverbial stem}
% \label{bkm:Ref347227985}\hypertarget{RefHeadingToc395697020}{}
% There are three ways to change adjectival stems to adverbial stems: (a) reduplication, (b) affixation, and (c) reduplication with affixation.
% 
%   First, reduplication of adjectival stems makes adverbs. As mentioned in §\ref{bkm:Ref347246311}, if the adjectival stem does not have a heavy syllable at the final or penultimate positions, the final mora of the preceding reduplicated stem is lengthened.
% 
% \tablefirsthead{}
% 
% \tabletail{}
% \tablelasttail{}
% \begin{tabularx}{\textwidth}{XXXX}
% \lsptoprule
% { (4\nobreakdash-\stepcounter{Remark}{\theRemark})} & { TM:} & { sijuuziju} & { natajaa.}\\
% &  & {\itshape siju+siju} & {\itshape nar-tar=jaa}\\
% &  & { RED+white} & { become-PST=SOL}\\
% &  & \multicolumn{2}{X}{{ ‘(It) became white.’}}\\
% &  & \multicolumn{2}{X}{\raggedleft [El: 111116]}\\
% \lspbottomrule
% \end{tabularx}
% Additionally, the following stem also goes through sequential voicing (cf. §\ref{bkm:Ref347181344}).
% 
% Second, there are two affixes that can change adjectival stems to adverbial stems: \textit{{}-ku} and \textit{{}-sanma}. We label these affixes as adverbializers. We categorize the adverbializers as derivational affixes and not types of converbal (inflectional) affixes since (a) they are not so productive and (b) there are no instances in texts where adverbs derived from adjectival stems take their own arguments. On the other hand, converbal affixes such as \textit{{}-tɨ} (SEQ) are very productive and can take their own arguments, i.e., they can make clauses.
% 
% \tablefirsthead{}
% 
% \tabletail{}
% \tablelasttail{}
% \begin{tabularx}{\textwidth}{XXXXXXXXXXXXX}
% \lsptoprule
% { (4\nobreakdash-\stepcounter{Remark}{\theRemark})} & { a.} & \multicolumn{11}{X}{{\itshape {}-ku}}\\
% &  & \multicolumn{11}{X}{{ [Context: Talking about the lifestyle in the old days, TM tells the hearer MS how to carry the baskets.]}}\\
% &  & { TM:} & \multicolumn{2}{X}{{ ubuku}} & \multicolumn{3}{X}{{ nappoo}} & \multicolumn{2}{X}{{ sɨgu}} & \multicolumn{2}{X}{{ cuburunan}} & { nusɨtɨ,}\\
% &  &  & \multicolumn{2}{X}{{\itshape ubu-ku}} & \multicolumn{3}{X}{{\itshape nar-boo}} & \multicolumn{2}{X}{{\itshape sɨgu}} & \multicolumn{2}{X}{{\itshape cuburu=nan}} & {\itshape nusɨr-tɨ}\\
% &  &  & \multicolumn{2}{X}{{ heavy-ADVZ}} & \multicolumn{3}{X}{{ become-CND}} & \multicolumn{2}{X}{{ immediately}} & \multicolumn{2}{X}{{ head=LOC1}} & { put.on-SEQ}\\
% &  &  & \multicolumn{10}{X}{{ ‘As soon as (it) becomes heavy, (the people) put (baskets) on (their) heads, and ...’}}\\
% &  &  & \multicolumn{10}{X}{\raggedleft [Co: 111113\_02.txt]}\\
% & \raggedleft b. & \multicolumn{11}{X}{{\itshape {}-sanma}}\\
% &  & \multicolumn{11}{X}{{ [Context: Talking about how to make pickles out of white radishes]}}\\
% &  & { TM:} & \multicolumn{3}{X}{{ dookunɨɨba}} & { koo} & \multicolumn{2}{X}{{ mucjɨ.}} & \multicolumn{2}{X}{{ kjuraasanma}} & \multicolumn{2}{X}{{ aratɨ,}}\\
% &  &  & \multicolumn{3}{X}{{\itshape dookunɨɨ=ba}} & {\itshape koo} & \multicolumn{2}{X}{{\itshape muk-tɨ}} & \multicolumn{2}{X}{{\itshape kjura-sanma}} & \multicolumn{2}{X}{{\itshape araw-tɨ}}\\
% &  &  & \multicolumn{3}{X}{{ white.radish=ACC}} & { skin} & \multicolumn{2}{X}{{ peel-SEQ}} & \multicolumn{2}{X}{{ beautiful-ADVZ}} & \multicolumn{2}{X}{{ wash-SEQ}}\\
% &  &  & { koo} & \multicolumn{9}{X}{{ mucjɨ.}}\\
% &  &  & {\itshape koo} & \multicolumn{9}{X}{{\itshape muk-tɨ}}\\
% &  &  & { skin} & \multicolumn{9}{X}{{ peel-SEQ}}\\
% &  &  & \multicolumn{10}{X}{{ ‘(I) peeled the white radish. (I) washed (it) beautiful, and peeled (it).’}}\\
% &  &  & \multicolumn{10}{X}{\raggedleft [Co: 101023\_01.txt]}\\
% \lspbottomrule
% \end{tabularx}
% The above example shows that \textit{{}-sanma} (ADVZ) requires that the preceding stem is lengthened, i.e. //kjura// > /kjuraa/, if the adjectival stem has a light syllable in the final position. Otherwise, lengthening does not occur: \textit{hɨɨ-} ‘large’ + \textit{{}-sanma} (ADVZ) > /hɨɨsanma/ ‘largely.’
% 
%   Finally, reduplication with affixation changes adjectival stems to adverbial stems. Morphophonologically, the following stem is lengthened with the adverbializer \textit{{}-tu}. Additionally, the following stem goes through sequential voicing (§\ref{bkm:Ref347181344}). Syntactically, these dirived adverbs can fill the complement slot of the light verb construction (see §\ref{bkm:Ref353886187} for more details).
% 
% \tablefirsthead{}
% 
% \tabletail{}
% \tablelasttail{}
% \begin{tabularx}{\textwidth}{XXXX}
% \lsptoprule
% { (4\nobreakdash-\stepcounter{Remark}{\theRemark})} & \multicolumn{3}{X}{{\itshape {}-tu}}\\
% & { TM:} & { sijuzijuutu} & { natɨjaa.}\\
% &  & {\itshape siju+siju-tu} & {\itshape nar-tɨ=jaa}\\
% &  & { RED+white-ADVZ} & { become-SEQ=SOL}\\
% &  & \multicolumn{2}{X}{{ ‘(It) became white.’}}\\
% &  & \multicolumn{2}{X}{\raggedleft [El: 111116]}\\
% \lspbottomrule
% \end{tabularx}
% We do not interpret \textit{{}-tu} (ADVZ) as \textit{tu} (COM) discussed in §\ref{bkm:Ref367143972} since the preceding form, e.g., /sijuzijuu/ in (4-62) cannot take other case particles or cannot be followed by the copula verb. These facts mean that the form cannot be a nominal. Furthermore, this type of adverbialization cannot apply to adjectival stems that express a kind of emotion, e.g., *\textit{utumara+utumara-tu} (RED+feel.strange-ADVZ).
